\documentclass[]{report}


% Title Page
\title{}
\author{}


\begin{document}
\maketitle


Spark est un framework pour le calul distribué. Il offre de bonnes performances que ce soit pour bach processing ou bien interactive processing.

Disponibilité des APIs de haut niveau : java, scala, python et R.

Spark propose les modules suivants:


\begin{itemize}
	\item Spark SQL : manipuler les données structurées. Utiliser les requêtes SQL avec Spark.
	\item Spark streaming 
	\item Mlib : algorithmes du machine learning.
	\item GraphX.
\end{itemize}

Les applications Spark peuvent être exécutées localement ou bien les distribuer dans un cluster. Afin de lancer les applications en mode distribué, il faut utiliser un cluster manager.

cluster managers:YARN and Spark Standalone

\section{Exemples } 

\paragraph{https://spark.apache.org/examples.html}

\paragraph{http://www.hpcsaudi.org/2018/wp-content/uploads/2018/04/20180305-HPCSaudi2018-Tutorial-Presentation_Sugi-2.pdf}

\section{Pretraitement des traceroutes en spark  scala}

https://www.toptal.com/spark/introduction-to-apache-spark














\begin{abstract}
\end{abstract}

\end{document}          
