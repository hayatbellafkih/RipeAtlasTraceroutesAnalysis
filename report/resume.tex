\chapter*{Résumé}


RIPE Atlas, un projet créé et géré par l'organisme RIPE NCC, qui a donné naissance à de
simples dispositifs, appelés sondes, à placer dans un routeur. Ces sondes consomment une quantité
légère d'électricité et de bande passante, mais elles sont capables de changer complètement des
implantations physiques dans certaines infrastructures.
La répartition abondante des sondes Atlas engendre quotidiennement une quantité importante
de données qui dépasse la capacité des outils traditionnels à stocker et à traiter ces données avec
efficacité. On parle des données massives. Certaines problématiques dans le domaine des réseaux
informatiques nécessitent une exploration plus profonde des données réseaux an d'aboutir à des
résultats importants.
Ce travail étudie les facilités et les contraintes engendrées par l'utilisation des services web
d'Amazon, relatifs au traitement du Big Data, pour analyser les données collectées par les sondes
Atlas, en particulier les données du type traceroute.
MotsClés:
Big Data, RIPE Atlas, Amazon Web Service, Traceroute, Analyse. de données