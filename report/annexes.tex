\appendix

\chapter{Amazon Athena } \label{athena-appendix}

\section{Création de la table traceroutes } \label{creer-table-traceroute}
La création d'une table reprend plusieurs parties :
\begin{itemize}
	\item les colonnes de la table avec le type correspondant (int, string, array pour définir une liste, struct pour définir un objet );
	\item LOCATION : c'est l'endroit où les données sont stockées dans Amazon S3, il faut préciser le chemin vers le compartiment de données;
	\item ROW FORMAT SERDE : elle définit la manière dont chaque ligne d'un fichier de données est sérialisée/désérialisée par Amazon Athena;
	\item PARTITIONED BY : elle définit la manière dont les données sont organisées dans le compartiment de données; 
	\item WITH serdeproperties : elle définit les options de la sérialisation/désérialisation.
\end{itemize}
\begin{lstlisting}[language=SQL, basicstyle=\footnotesize,]
CREATE EXTERNAL TABLE traceroutes(
	af int,
	bundle int,
	dst_addr string,
	dst_name string,
	fw int,
	endtime int,
	`from` string,
	group_id int,
	lts int,
	msm_id int,
	msm_name string,
	paris_id int,
	prb_id int,
	proto string,
	size int,
	src_addr string,
	`timestamp` int,
	ttr float,
	type string,
    result array< struct< hop:int,error:string, result:array<
        struct<x:string, err:string, `from`:string, ittl:int, edst:string, late:int, mtu:int, rtt:float, size:int, ttl:int , flags:string, dstoptsize:int, hbhoptsize:int, icmpext:
        	struct<version:int, rfc4884:int, obj:array< 
        		struct<class:int, type:int, mpls:array<struct< exp:int, label:int, s:int, ttl:int>>>>>>>>> 
)
PARTITIONED BY (
	af_ string,
	type_ string,
	msm string ,
	year string,
	month string,
	day string,
	hour string
) 
ROW FORMAT SERDE 'org.openx.data.jsonserde.JsonSerDe'
WITH serdeproperties ('paths'='af,bundle,dst_addr, dst_name,fw, endtime, from, lts, msm_id, paris_id, prb_id, proto, size, src_addr, timestamp, type,fw, msm_name' ) 
LOCATION 's3://ripeatlasdata/traceroute/source=api/'
\end{lstlisting}

\paragraph{Partitionnement de données dans un compartiment AWS S3} \label{subsubsection:partitionnement}~

Le partitionnement  de données présentes dans un compartiment Amazon S3 permet de limiter la quantité de données à analyser par une requête Amazon Athena. Le partitionnement améliore  les performances d'Amazon Athena. D'une part, on obtient une réponse rapidement, d'autre part, on réduit les coûts engendrés  suite à l'utilisation du service car on est facturé selon la quantité de données analysées.  

Les partitions créées joue un rôle similaire à celui d'un colonne durant l'interrogation d'une table dans Athena. 

Prenons un exemple, nous avons des traceroutes ayant comme adressage IP la version  4 et d'autres traceroutes ont l'adressage IPv6 :

\begin{lstlisting}
s3://ripeatlasdata/traceroute/
				type=4/
				type=6/
\end{lstlisting}

Sans l'utilisation du partitionnement et si on souhaite récupérer que les traceroutes ayant comme adressage IPv4, toutes les données (type = 4 et type = 6) sont analysées. Toutefois, en partitionnant les données suivant le type d'adressage, seuls les fichiers dans le dossier type = 4 qui sont analysés. Par conséquent, le partitionnement permet de réduire les coûts d'utilisation du service Amazon Athena, surtout si la quantité de données est très importante. 


