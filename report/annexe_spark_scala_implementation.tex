\chapter{Application sur traceroutes} \label{application:spark}
%\subsection{Complément d'information du processus de la détection avec le langage Scala}
Comme complément aux étapes décrites dans \ref{steps-rtt-analysis}, on présente les différentes classes permettant de modéliser les données tout au long du processus de l'analyse. La définition de ces classes est liée au langage \textit{Scala}. 

Soient les classes suivantes utilisées : 

\paragraph{La classe Signal} modélise un signal \footnote{Un signal dans le contexte d'un traceroute.}. Ainsi, \textit{from} est l'adresse IP du routeur émettant ce signal, \textit{rtt} est le Round Trip Time entre la sonde Atlas et ce routeur et enfin \textit{x} est un indicateur de l'échec du signal.
\begin{lstlisting}[language=scala, caption={La classe Signal en Scala }]
case class Signal(
	rtt:  Option[Double],
	x:    Option[String],
	from: Option[String])

\end{lstlisting}

\paragraph{La classe Hop} modélise un saut dans un traceroute. On caractérise un saut par son identifiant noté \textit{hop}. Celui-ci   prend comme valeur un entier commençant à $1$ et la liste des signaux relatifs à ce saut notée par \textit{result}. Généralement un saut est représenté par $3$ signaux.
\begin{lstlisting}[language=scala, caption={La classe Hop en Scala }]
case class Hop(
	var result: Seq[Signal],
	hop:        Int)
\end{lstlisting}
\paragraph{La classe Traceroutes} modélise le résultat d'une requête traceroute effectuée par une sonde Atlas. Cette modélisation se limite aux données qui nous intéressent dans la présente analyse. 

\textit{dst\_name} représente l'adresse IP de la destination de la requête traceroute, \textit{from} est l'adresse IP de la sonde, \textit{prb\_id} est l'identifiant de la sonde, \textit{msm\_id} est l'identifiant de mesure, \textit{timestamp} est le temps auquel la requête traceroute a été effectuée et enfin on trouve la liste des sauts qui représentent les routeurs traversés par le trafic entre la source et la destination. 

\begin{lstlisting}[language=scala, caption={La classe Traceroute en Scala }]
case class Traceroute(
	dst_name:  String,
	from:      String,
	prb_id:    BigInt,
	msm_id:    BigInt,
	timestamp: BigInt,
	result:    Seq[Hop])
\end{lstlisting}
\paragraph{La classe TraceroutesPerPeriod} permet de présenter les traceroutes après les avoir trié   suivant la période pendant laquelle ils ont été effectués.   \textit{timeWindow} est le temps unix marquant le début de la période \footnote{Pour précision, la fin de la période peut être inférée en prenant deux débuts de deux périodes car la durée d'une période est fixe tout au long de l'analyse.} et  \textit{traceroutes} est la liste des traceroutes effectués pendant cette période. 


A l'étape 2, l'objectif était d'agréger  les signaux par routeur source et ensuite calculer la médiane des RTTs par ce routeur. Par conséquent, un traceroute est présenté différemment, ce qui est  illustré par la classe \textit{MedianByHopTraceroute}.

\paragraph{La classe PreparedSignal }  est une agrégation de tous les signaux, d'un saut donné, par le routeur \textit{from},  la médiane des RTTs calculée est présentée par \textit{medianRtt}.
\begin{lstlisting}[language=scala, caption={La classe PreparedSignal en Scala }]
case class PreparedSignal(
	medianRtt: Double,
	from:      String)
\end{lstlisting}
\paragraph{La classe PreparedHop } modélise un saut après avoir agrégé ses signaux. 
\begin{lstlisting}[language=scala, caption={La classe PreparedHop en Scala }]
case class PreparedHop(
	var result: Seq[PreparedSignal],
	hop:        Int)
\end{lstlisting}


\paragraph{La classe MedianByHopTraceroute } modélise un traceroute après avoir agrégé ses sauts. Par rapport au traceroute d'avant l'agrégation, seule la liste des sauts  a subi un changement. 
\begin{lstlisting}[language=scala, caption={La classe MedianByHopTraceroute en Scala }]
case class MedianByHopTraceroute(
	dst_name:  String,
	from:      String,
	prb_id:    BigInt,
	msm_id:    BigInt,
	timestamp: BigInt,
	result:    Seq[PreparedHop])
\end{lstlisting}


\paragraph{La classe Link} modélise un lien topologique. Ce dernier est défini par deux adresses IP  \textit{ip1} et \textit{ip2} et par son RTT différentiel calculé \textit{rttDiff}.
\begin{lstlisting}[language=scala, caption={La classe Link en Scala }]
case class Link(
	ip1:     String,
	ip2:     String,
	rttDiff: Double)
\end{lstlisting}

\paragraph{La classe LinksTraceroute} permet de modéliser un traceroute après avoir inféré tous les liens de ce dernier. Ainsi, la liste des sauts est remplacée par la liste des liens (\textit{links}). 

\begin{lstlisting}[language=scala, caption={La classe LinksTraceroute en Scala }]
case class LinksTraceroute(
	dst_name:  String,
	from:      String,
	prb_id:    BigInt,
	msm_id:    BigInt,
	timestamp: BigInt,
	links:     Seq[Link])
\end{lstlisting}


A l'étape 5, l'objectif était de passer d'un traceroute à une liste de liens caractérisés par les informations générales sur la sonde Atlas, la mesure Atlas, etc. Chaque élément de cette liste est représenté par la classe \textit{DiffRtt}, où \textit{LinkIPs} représente les deux adresses IP d'un lien donné.
\paragraph{La classe LinkIPs} permet représenter un lien par seulement ses deux adresses IP \textit{ip1} et \textit{ip2}.
\begin{lstlisting}[language=scala, caption={La classe LinkIPs en Scala }]
case class LinkIPs(
	ip1: String,
	ip2: String)
\end{lstlisting}

\paragraph{La classe DiffRtt} est une représentation plus détaillée d'un lien, en plus de son RTT différentiel, on ajoute d'autres informations.  Les adresses IP d'un lien sont modélisées par la classe \textit{LinkIPs}.

\begin{lstlisting}[language=scala, caption={La classe DiffRtt en Scala }]
case class DiffRtt(
	rtt:      Double,
	var link: LinkIPs,
	probe:    BigInt)
\end{lstlisting}

A l'étape 6.3, on souhaite normaliser les dates de chaque lien; peu importe le moment pendant lequel le traceroute a été effectué durant une période $d_i$, on note seulement le début de cette période. Ainsi,  la classe  \textit{DiffRTTPeriod}  reprend un \textit{lien} donné, les différentes sondes Atlas ayant identifié ce lien (\textit{probes}), les RTTs différentiels de ce lien tout au long de la période et enfin les dates associées à chaque RTT différentiel.
\paragraph{La classe DiffRTTPeriod } ~
\begin{lstlisting}[language=scala, caption={La classe DiffRTTPeriod en Scala }]
case class DiffRTTPeriod(
	link:      LinkIPs,
	probes:    Seq[BigInt],
	rtts:      Seq[Double],
	var dates: Seq[Int])
\end{lstlisting}

A la fin des opérations de l'étape 6, on reprend pour chaque période, pour un lien donné, les RTTs différentiels ainsi que leurs dates. Ensuite, on construit les bornes de l'intervalle de confiance courants pour ce lien et les bornes de l'intervalle de confiance de référence, et ce afin de comparer ces deux intervalles en vue d'inférer les anomalies possibles du délais de ce lien.


\paragraph{La classe LinkState } permet de modéliser les intervalles de confiance d'un lien pendant une période $d_i$ donnée. \textit{valueLow} est la borne inférieur de l'intervalle de confiance, \textit{valueHi} est la borne supérieure de l'intervalle de confiance, \textit{valueMedian} est la médiane des RTTs différentiels et enfin \textit{valueMean} est la moyenne des RTTs différentiels. Pour précision, les données concernant l'état d'un lien sont sous forme d'une liste. L'idée est de garder l'historique de ces valeurs durant toute la période de l'analyse. Cette historique est exploitée pour tracer l'évolution du RTT différentiel du lien. Cependant, la comparaison utilise les valeurs du dernier état du lien.

  

%Pour toute période, on a une instance de \textit{LinkState} pour 
\begin{lstlisting}[language=scala, caption={La classe LinkState en Scala }]
case class LinkState(
	var valueMedian: Seq[Double],
	var valueHi:     Seq[Double],
	var valueLow:    Seq[Double],
	var valueMean:   Seq[Double])
\end{lstlisting}