\documentclass[12pt,a4paper]{memoire-umons}

\usepackage[utf8]{inputenc}
\usepackage[T1]{fontenc}
\usepackage[francais]{babel}
\usepackage{amssymb,amsmath,amsthm}
\usepackage[table]{xcolor}
%######################################################################################################
\usepackage[utf8]{inputenc}
\usepackage[T1]{fontenc}
\usepackage{listings}
\usepackage{makecell,tabularx}
\renewcommand\theadfont{\bfseries\scshape}
\usepackage{graphicx}
\setcounter{secnumdepth}{3}
\usepackage{float}
\usepackage{adjustbox}
\usepackage{amsmath}
\setlength{\parindent}{2em}
\usepackage{tcolorbox}
\usepackage{colortbl}
\usepackage[francais]{babel}
\frenchbsetup{StandardLists=true}
\usepackage{times}
\usepackage[hyphens]{url}
\usepackage{standalone}
\usepackage{comment}
\usepackage[subfigure]{tocloft} 
\usepackage{subfigure} 
%\usepackage{tabularx}
\usepackage{textcomp}
\usepackage{lscape}

\usepackage{color,soul}
\usepackage{csvsimple,booktabs}
\usepackage{tikz}
\usepackage{afterpage}
\usepackage{rotating}
\usepackage{csvsimple}
\usepackage{caption}

%\frenchbsetup{StandardLists=true}
\usepackage[hidelinks]{hyperref} 
\usepackage{array}
\colorlet{punct}{red!60!black}
\definecolor{background}{HTML}{EEEEEE}
\definecolor{delim}{RGB}{20,105,176}
\definecolor{mygray}{gray}{0.6}
\colorlet{numb}{magenta!60!black}
\newcolumntype{C}[1]{>{\centering\let\newline\\\arraybackslash\hspace{0pt}}m{#1}}
\usepackage[left=2.5cm,right=2.5cm,top=2.5cm,bottom=2.5cm]{geometry}
\usepackage[utf8]{inputenc}
\usepackage{amsmath}
\usepackage{amssymb}
\renewcommand{\bibname}{References}
\usepackage{enumerate}
\usepackage{tablefootnote}
\newcommand{\tabitem}{~~\llap{\textbullet}~~}
\usepackage{pstricks}
\usepackage{eqparbox}
\usepackage{hyperref}
\usepackage{tablefootnote}
\usepackage{float}
\usepackage{listings}
\usepackage{color}
\usepackage{graphicx}
\usepackage{verbatim}
\usepackage{pifont}
\usepackage{comment}
\usepackage{enumitem}
\usepackage{blindtext}
\usepackage{appendix}
\usepackage{tikz}
\usepackage{algorithm}
\usepackage{algpseudocode}
\usepackage[flushleft]{threeparttable}
\newcommand*\circled[1]{\tikz[baseline=(char.base)]{
		\node[shape=circle,draw,inner sep=2pt] (char) {#1};}}
\lstset{
	showstringspaces=false,
	string=[s]{"}{"},
	stringstyle=\color{blue},
	comment=[l]{:},
	commentstyle=\color{black},
	backgroundcolor=\color{lstcolor},
	breaklines=true,
}

\lstdefinelanguage{json}{
	basicstyle=\normalfont\ttfamily,
	numbers=left,
	numberstyle=\scriptsize,
	stepnumber=1,
	numbersep=8pt,
	showstringspaces=false,
	breaklines=true,
	frame=lines,
	backgroundcolor=\color{background},
	literate=
	*{0}{{{\color{numb}0}}}{1}
	{1}{{{\color{numb}1}}}{1}
	{2}{{{\color{numb}2}}}{1}
	{3}{{{\color{numb}3}}}{1}
	{4}{{{\color{numb}4}}}{1}
	{5}{{{\color{numb}5}}}{1}
	{6}{{{\color{numb}6}}}{1}
	{7}{{{\color{numb}7}}}{1}
	{8}{{{\color{numb}8}}}{1}
	{9}{{{\color{numb}9}}}{1}
	{:}{{{\color{punct}{:}}}}{1}
	{,}{{{\color{punct}{,}}}}{1}
	{\{}{{{\color{delim}{\{}}}}{1}
	{\}}{{{\color{delim}{\}}}}}{1}
	{[}{{{\color{delim}{[}}}}{1}
	{]}{{{\color{delim}{]}}}}{1},
}


\definecolor{lstcolor}{HTML}{F5F5F5}
\definecolor{orange}{HTML}{F8F9F9}
\definecolor{ligntrose}{HTML}{FFF8DC}
\tcbset{colback= orange,fonttitle=\bfseries, colframe=white }


%%%%%%%%%%%%%%%%%%%
\renewcommand\labelitemi{--}
\newcommand{\source}[1]{\caption*{Source: {#1}} }
\definecolor{lightgray}{gray}{0.9}
\renewcommand{\cleardoublepage}{}

%%%%%%%%% like and dislike icons
\usepackage{graphicx}
\newcommand*{\RightThumbsUpAux}[1]{%
	\begingroup
	\sbox0{Ag}%
	\raisebox{-\dp0}{%
		\includegraphics[{%
			height=\dimexpr\dp0+\ht0\relax,
			#1%
		}]{Symbol_thumbs_up-crop.pdf}%
	}%
	\endgroup
}
\newcommand*{\RightThumbsDownAux}[1]{%
	\begingroup
	\sbox0{Ag}%
	\raisebox{-\dp0}{%
		\includegraphics[{%
			height=\dimexpr\dp0+\ht0\relax,
			#1%
		}]{Symbol_thumbs_down-crop.pdf}%
	}%
	\endgroup
}
\newcommand*{\RightThumbsUp}{%
	\RightThumbsUpAux{}%
}
\newcommand*{\RightThumbsDown}{%
	\RightThumbsUpAux{origin=c,angle=180}%
}
\newcommand*{\LeftThumbsUp}{%
	\scalebox{-1}[1]{\RightThumbsUp}%
}
\newcommand*{\LeftThumbsDown}{%
	\scalebox{-1}[1]{\RightThumbsDown}%
}
%\renewcommand{\clearpage}{}
%#####################################################################################################
%\usepackage{hyperref}% hyperliens dans le PDF, pas pour impression

\title{Titre}
\author{Prénom \textsc{Nom}}
\date{2009--2010}
\directeur{Nom du directeur}
%\directeurs{Directeur1\\ Directeur 2}
%\codirecteurs{}
\service{Service dans lequel vous avez fait votre mémoire}
% \rapporteurs{
%   Rapporteur 1\\
%   Rapporteur 2}
%\discipline{math\'ematiques}

%%%%%%%%%%%%%%%%%%%%%%%%%%%%%%%%%%%%%%%%%%%%%%%%%%%%%%%%%%%%%%%%%%%%%%%%
%% Vos macros


%%%%%%%%%%%%%%%%%%%%%%%%%%%%%%%%%%%%%%%%%%%%%%%%%%%%%%%%%%%%%%%%%%%%%%%%

% Compile uniquement certains morceaux sans perdre les références
% automatiques et la table des matières des parties déjà compilées :
%\includeonly{introduction,chapitre1}

\begin{document}
% Éventuellement utiliser l'environnement « preface » pour avoir une
% numérotation des pages en chiffres romains.
\tableofcontents

%%introduction
\chapter*{Introduction générale}
\addcontentsline{toc}{chapter}{Introcuction générale}

%\section{Faits}
%Les sondes RIPE Atlas génère quotidiennement une quantité importante de données, environ $200$ Go par jour rien que pour des mesures de traceroute. Les systèmes de gestion des bases de données basés sur le modèle relationnel montrent certaines limitations.  En vue d'une bonne exploitation des données durant une période de quelques jours, les SGBD relationnels classiques sont incapables de gérer de grands volumes de données.  De plus, le modèle relationnel est peu adapté au stockage et à l'interrogation de certains types de données comme les données hiérarchiques. C'est le cas des données traceroutes générées par les sondes Atlas. Ce sont des données en format JSON. 


%accroche
Actuellement, plus de $ 10,300 $ \footnote{A la date de l'accès à la source \url{https://atlas.ripe.net/results/maps/network-coverage/}, le $14/08/2018$.} sondes Atlas sont déployées dans le monde pour effectuer des mesures réseaux comme le DNS, Ping, Traceroute, etc. Ces sondes sont maintenues par le RIPE NCC (Réseaux IP Européens - Network Coordination Centre).  Les données collectées par ces mesures sont  stockées et sont disponibles en accès libre \footnote{Les données des derniers $30$ jours, les données des autres périodes sont accessibles avec une API REST.}. Quotidiennement, plus de $ 18732 $ mesures sont planifiées au départ de ces sondes vers de multiples  autres destinations. En moyenne, $33$ Go de données \footnote{Au format compressé.} sont  collectées chaque jour pour tous les types de mesures.


Le besoin en stockage des données est en croissance continue avec la quantité de données générées par les transactions des clients, les réseaux sociaux, l'Internet des objets qui collectent constamment les données, etc.  Les solutions traditionnelles en terme de stockage, de calcul et de visualisation ne répondent pas aux besoins, surtout des grandes organisations. Ce qui les a encouragé  à  créer des solutions permettant de répondre aux nouveaux besoins, c'est le Big Data.


L'objectif du présent mémoire est de montrer la capacité des nouvelles technologies du Big Data à fournir des solutions efficaces capables d'assurer le stockage des données massives et d'effectuer des tâches de traitement sur ces quantités de données. Dans notre cas, ce sont des données 
collectées par les sondes Atlas. Ces  données apportant des informations utiles et pertinentes que nous recueillons sur l'état du réseau.



Les articles, les travaux publiés par RIPE Atlas et les présentations durant les rencontres organisées par RIPE NCC permettent d'avoir une idée générale sur les sujets à traiter en vue d'exploiter les données collectées par les sondes Atlas. Plus généralement,  les sujets abordés sont : la visualisation de certains indicateurs sur les performances d'un réseau, l'analyse des censures appliquées au niveau de certains pays, l'analyse des détours que subit un trafic local et l'étude des performances d'un réseau, par exemple :  le temps de la latence, la perte des paquets, l'asymétrie du trafic et la congestion des routeurs.




Les utilitaires ping et traceroute font partie des outils d'analyse de l'état du réseau et de résolution des problèmes dans les réseaux fortement utilisés. En particulier, l'utilitaire traceroute fournit des informations de l'aller et du retour entre une adresse IP source et une adresse IP destination sur un réseau. Il fournit les sauts impliqués tout au long du chemin entre la  source et la destination  ainsi que le temps requis pour les atteindre. Les détails fournis par traceroute permettent d'avoir des informations sur les réseaux traversés, la latence, etc.
%problematique


Les traceroutes effectués par toutes les sondes, durant une heure, génèrent des données dont la taille est d'environ $ 8 $ Go. La manipulation de cette quantité de données nécessite des ressources de hautes performances. On ne peut pas compter sur   les ressources traditionnelles comme les bases de données relationnelles pour le stockage, les processeurs des  machines ordinaires \footnote{Machines avec une mémoire RAM de $ 4 $ ou de $ 8 $ Go.} pour traiter les données  après la récupération de celles-ci.

L'analyse des données collectées par les sondes Atlas a prouvé l'utilité de ces données. Plusieurs cas d'utilisation sont régulièrement publiés. Nous nous intéresserons au sujet du délai d'un lien réseau (lien topologique), car traceroute fournit les sauts impliqués dans un chemin, entre une adresse IP source et une adresse IP destination, avec les informations de la latence. Nous allons étudier la capacité des technologies Big Data à gérer la quantité de données générées par les sondes Atlas. La gestion des données porte sur le stockage, le traitement et la visualisation.



Ce document est structuré comme suit : le premier chapitre reprend une présentation du projet RIPE Atlas où nous allons présenter les sondes Atlas, leur fonctionnement  et quelques cas d'utilisation spécifiques. Le deuxième chapitre introduit  le terme  Big Data, puis, il reprend les disciplines impliquées dans le Big Data, pour ensuite parcourir un ensemble d'outils Big Data. Pour conclure le deuxième chapitre, un choix sera énoncé concernant les outils à utiliser dans l'analyse des traceroutes. 
En ce qui concerne le troisième chapitre, il reprendra les étapes de l'analyse de données des traceroutes suivant un processus particulier. Depuis la collecte des données Atlas jusqu'à la génération des résultats de l'analyse. 
\chapter*{Introduction}

%L’accroche

L'analyse de données, en particulier des données à grande échelle, attire de plus en plus les entreprises à s'y investir.  Cette analyse peut affecter potentiellement la stratégie de ces entreprises. 
Les défis de l'analyse de données massives varient en fonction du processus suivi. Par exemple, les défis peuvent être liés à la définition  des objectifs d'une analyse, le choix de données, la collecte de données, etc.

%La présentation du sujet

L'idée de ce travail est  d'exploiter l'existence d'un dépôt de données en vue d'évaluer la mise en place ainsi que les performances de quelques technologies conçues pour la manipulation des données massives, appelées aussi Big Data.
% pour l'analyse de données massives, autrement dit Big Data. 
%Vos motivations personnelles liées au sujet ou au mémoire (facultatif)
Les données considérées sont des données  collectées par des dispositifs appelés sondes Atlas,  et  l'accès à ces données est publique. 
%De plus, il existe une variété de technologies destinées à la manipulation des données à grande échelle, généralement elles sont open source.  %De plus, certaines technologies  dont leur utilisation est facturée proposent des
%La présentation de votre cadre théorique 
%mettre les technologies Big Data en faveur de l'analyse des données disponibles.
 La manipulation de ces dernières nécessite l'utilisation des outils convenables dépassant les capacités des outils traditionnels; l'exemple des base de données relationnelles.  Ainsi, le choix d'utilisation d'une technologie Big Data donnée dépend de plusieurs critères. 
 % à prendre en considération afin de trouver la technologie Big Data la plus adaptée à l'analyse de données à grande échelle.


%L’objectif principal du mémoire
L'objectif du présent mémoire est de montrer la capacité des nouvelles technologies du Big Data à fournir des solutions efficaces capables d'assurer le stockage des données massives et d'effectuer des tâches de traitement sur de grandes quantités de données. Dans notre cas, ce sont des données collectées par les sondes Atlas. Ces  données apportent des informations utiles et pertinentes de l'état des réseaux informatiques.


%La problématique du mémoire
%La présentation de votre démarche ou méthodologie de recherche

%Notre démarche est comme suit. 
Dans un premier temps, nous avons étudié le projet RIPE  Atlas (Réseaux IP Européens Atlas) afin de maîtriser le contexte général de données  d'une part, et d'autre part de bien choisir les données de travail. Le choix a été fait sur les données générées par l'utilitaire Traceroute, en particulier, nous allons réutiliser un outil conçue dans le cadre d'un travail basé aussi sur les données collectées par les sondes Atlas. Cet outil permettant de détecter les anomalies dans les délais des liens dans les réseaux informatiques. Ensuite, nous avons passé en revue les technologies du Big Data disponibles afin de sélectionner celles qu'on puisse  évaluer leurs performances et convenances pour les données choisies. A l'issue de cette étape, nous avons abordé l'évaluation des technologies Big Data sélectionnées  pour réutiliser l'outil de la détection des anomalies.



%L’annonce du plan
Le présent document est organisé en quatre chapitres. Le premier chapitre présente le projet RIPE Atlas: la présentation des caractéristiques techniques et fonctionnelles des sondes Atlas ainsi qu'une liste non exhaustive des cas d'usage de ces sondes. Le deuxième chapitre reprend l'algorithme de la détection à évaluer par les technologies Big Data.
Le troisième chapitre énumère quelques concepts liés au Big Data ainsi qu'une liste non exhaustive des technologies Big Data. Pour finir, le quatrième chapitre aborde l'application de l'outil de détection en pratique pour les technologies sélectionnées.



%chapitre1

\chapter{RIPE Atlas}

\section{Introduction}
%Le présent chapitre  introduit   le projet  RIPE Atlas. La première section est consacrée à la présentation détaillée de ce projet. La deuxième section   reprend  une liste non exhaustive de quelques outils similaires aux sondes Atlas en matière d'objectifs. La troisième section expose quelques limites du système RIPE Atlas. La dernière section reprend brièvement quelques travaux basés sur le projet RIPE Atlas.



Le présent chapitre commence par une présentation détaillée du projet RIPE Atlas mené par l'organisme RIPE NCC.  Ce projet  a introduit l'utilisation des sondes pour  effectuer des mesures des réseaux dans le monde.
Ensuite, ce chapitre reprend une liste non exhaustive  de quelques outils similaires aux sondes Atlas en matière d'objectifs. Enfin, expose quelques limites du système RIPE Atlas. La dernière section reprend brièvement quelques travaux basés sur le projet RIPE Atlas.


\section{A propos RIPE NCC}
Le RIPE NCC  est un organisme qui alloue les blocs d'adresses IP et des numéros des Systèmes Autonomes dans l'Europe et une partie de l'Asie, notamment au Moyen-Orient. 

\begin{tcolorbox}
	Un\textit{ \textbf{Système Autonome}}, appelé AS, est un ensemble de réseaux et de routeurs sous la responsabilité d'une même autorité administrative. Chaque Système Autonome est identifié par un code sur 16 bits uniques. Les protocoles qui tournent au sein d'un Système Autonome peuvent être différents.
\end{tcolorbox}

RIPE NCC assure différents services relatifs à la gestion des réseaux informatiques. Il maintient multiples projets  pour un nombre de protocoles comme DNS (DNSMON), BGP (Routing Information Service ou RIS) et d'autres projets et services. En particulier, nous sommes intéressés  par projet RIPE Atlas géré aussi par RIPE NCC. L'objectif du projet RIPE Atlas est de déployer des dispositifs dans le monde, capables de collecter des données réseaux. Nous allons le détailler dans la section \ref{ripeatlassection}. 


\section{Présentation du projet RIPE Atlas } \label{ripeatlassection}

RIPE NCC a créé le projet RIPE Atlas en $2010$. Le nombre de sondes déployées est en augmentation constante, sachant qu'elles sont déployées par des volontaires.

\subsection{Les mesures  actives et passives de l'Internet}

Il existe plusieurs approches pour analyser l'état  d'un réseau. Les deux approches les plus répandues sont~:  active et  passive. L'approche passive fait référence au processus de mesure d'un réseau, sans créer ou modifier le trafic sur ce réseau.   L'approche active  repose sur l'injection des paquets  sur le réseau et surveiller le flux de ce trafic. Cette injection a pour objectif la collecte  des données relatives aux performances du réseau en question. Par exemple, la mesure du temps de réponse, le suivi du chemin des paquets, etc. 

Les données collectées permettent de surveiller les réseaux pour ensuite  proposer des améliorations de l'Internet. Le projet RIPE Atlas est un des outils s'inscrivant dans l'approche active. Ce sont  des dispositifs, appelés sondes, hébergés par des volontaires, ils sont distribués et maintenus par  RIPE NCC. Les données collectées par ces dispositifs sont disponibles au public \cite{ripe-atlas-data}.

Actuellement,  plus de $10,000$ sondes Atlas sont actives, ces dernières produisent environ $450$ millions de mesures par jour, ce qui correspond à  $5,000$ résultats par seconde \cite{WinNT}.


\subsection{Généralités sur les sondes  Atlas}
\begin{itemize}
	\item[--] Les sondes Atlas  mesurent les performances de la couche IP. Une sonde  envoie des paquets réels et observe la réponse en temps réel indépendamment des applications en dessus de la couche IP.
	
	\item[--] Les sondes Atlas ne sont pas des observatrices des données comme le trafic du routage BGP, ainsi, elles n'observent pas  le trafic de leurs hébergeurs.
	
	\item[--] Les sondes  Atlas se situent dans différents emplacements dans le monde, cette répartition permet de diversifier les mesures (voir les sections des mesures  \ref{par:whatmesureripeatlas} et \ref{par:udm}). 
	
	\item[--] Les sondes Atlas sont déployées volontairement dans une maison, un bureau,  un entrepôt de données, etc.
	
	\item[--] Les mesures peuvent être lancées à tout moment et pour n'importe quelle période\footnote{Si le nombre de crédits (voir la section des crédits \ref{credits-atlas} ) disponibles le permet et qu'il n'y a pas de dépassement du nombre de mesures autorisé.}.
	
	\item[--] La participation au projet RIPE Atlas est ouverte à toute personne qui s'y intéresse, cela inclut  les résultats de mesures, les outils d'analyse, l'hébergement des sondes elles-mêmes, les travaux, etc.
	
	\item[--] RIPE Atlas simule le comportement de la couche IP. Par exemple, avec RIPE Atlas, il est possible  de~:
	\begin{itemize}
		\item Suivre l'accessibilité d'une destination \footnote{Une destination représente une adresse IP dans le présent contexte.} depuis différents emplacements dans le monde et  depuis différents réseaux. Car les sondes Atlas sont réparties dans plusieurs pays et déployées dans différents réseaux.
		
		\item Étudier des problèmes du réseau remontés en effectuant des vérifications de connectivité ad-hoc via les mesures effectuées par les sondes  Atlas.
		
		\item Tester la connectivité IPv6.
		
		\item Vérifier l'infrastructure DNS.
	\end{itemize}
	
	La section  \ref{use-cases-atlas} reprend quelques cas d'utilisation du système RIPE Atlas et les sujets qu'on peut étudier. 
\end{itemize}

\subsection{Les générations des sondes  Atlas }

Depuis leur création   en $2010$, les sondes Atlas ont connu   trois générations du matériel. Le tableau   \ref{tab:differents-generations-probes} reprend quelques  caractérisations de ces trois générations des sondes  Atlas et  la figure 	\ref{fig:genarations} montre le matériel utilisé  dans chaque génération.


\begin{table}[H]
	\begin{tabularx}{\textwidth}{|X|X|X|X|}
		\hline
		&\textbf{ v$ 1 $}&\textbf{v$ 2 $}&\textbf{v$ 3 $} \\ \hline
		\textbf{ Matériel informatique}  & Lantronix XPort Pro \cite{LantronixXPortPro} &Lantronix XPort Pro \cite{LantronixXPortPro}&tp-link tl-mr$ 3020  $  \\ \hline
		\textbf{Début d'utilisation}  &$2010$&$2011$&$2013$ \\ \hline
		\textbf{Mémoire RAM} & $8$ Mo&$16$ Mo& $32$ Mo\\ \hline
		\textbf{Mémoire Flash} & $16$ Mo&$16$ Mo&$4$ Mo \\ \hline
		\textbf{CPU} &$ 32 $-bit& $ 32 $-bit & $ 32 $-bit\\ \hline
		\textbf{Support du Wi-Fi} &Non&Non&oui \\ \hline
		\textbf{Support du NAT} &oui&oui&oui \\ \hline
		\textbf{Vitesses supportées} &$ 10 $ Mbit/s et $ 100 $ Mbit/s&$ 10 $ Mbit/s et $ 100 $ Mbit/s&$ 10 $ Mbit/s et $ 100 $ Mbit/s \\ \hline
		
	\end{tabularx}
	\caption{Les caractéristiques des trois générations des sondes  Atlas }
	\label{tab:differents-generations-probes}
\end{table}


\begin{figure}[H]
	
	\parbox{.32\textwidth}{\includegraphics[width=.30\textwidth, height=.20\textwidth]{illustrations/v1} 	\captionsetup{justification=centering}\caption{Génération 1}}
	\hfill
	\parbox{.32\textwidth}{\includegraphics[width=.30\textwidth, height=.20\textwidth]{illustrations/v2}
		\captionsetup{justification=centering}
		\caption{Génération 2}}
	\hfill
	\parbox{.32\textwidth}{\includegraphics[width=.30\textwidth,height=.20\textwidth]{illustrations/v3} 	\captionsetup{justification=centering}\caption{Génération 3}}
	\caption{Les trois générations des sondes Atlas}
	\label{fig:genarations}
	\source{\url{https://atlas.ripe.net/docs/}, consultée le $05/08/2018$.}
\end{figure}

Pour précision, les générations $1$ et $2$ présentent une très faible consommation d'énergie, cependant, elles ont un temps de redémarrage et coûts de production élevés. 

%Pour la génération $3$, même si elle est conçue en utilisant tp-link tl-mr3020,  la sonde v3 ne sert pas comme un routeur Wi-Fi.
%, pourtant, elle a les mêmes fonctionnalités que les sondes v1 et v2.

En $2015$, plusieurs utilisateurs des sondes  Atlas ont montré un intérêt aux sondes virtuelles. Ces sondes virtuelles présentent des avantages et aussi des inconvénients. Parmi les avantages, la conception des sondes virtuelles permet d'explorer des emplacements qui sont difficilement accessibles. En effet, cela permet d'étendre le réseau des sondes  Atlas. D'autre part, les sondes virtuelles peuvent être installées sans contraintes physiques ou organisationnelles. Parmi les inconvénients, une complexité sera ajoutée au système RIPE Atlas, plus de ressources seront demandées. Ensuite, il y a le problème de la qualité des données; le manque de données peut faire référence à une perte de paquets ou bien la machine qui héberge la sonde n'est plus disponible pour continuer les mesures. 




\subsection{La connexion des sondes Atlas à Internet}

Les génération $1$ et $2$ des sondes  Atlas ont une interface Ethernet (RJ-45). La génération $3$ dispose techniquement des capacités Wi-Fi. Cependant, ces sondes ne sont pas suffisamment prêtes au niveau logiciel pour supporter le Wi-Fi. L'objectif était de garder l'indépendance des sondes Atlas du trafic de celui qui les héberge.

Une fois la sonde se connecte au port d'Ethernet, elle acquiert  une adresse IPv4, un résolveur DNS  en utilisant DHCP et la configuration IPv6 via \textit{Router Advertisement}. Ensuite, elle essaie de rejoindre l'infrastructure du RIPE Atlas. Pour ce faire, elle utilise le résolveur DNS et se connecte à l'infrastructure à travers SSH sur le port TCP de sortie $443$ comme il est illustré dans la figure \ref{fig:ssh-atlas-probe}. L'architecture du système RIPE Atlas est détaillée dans la section \ref{subsec:archi-probes}.

\begin{figure}[H]
	\centering
	\captionsetup{justification=centering}
	\includegraphics[width=0.5\linewidth]{illustrations/ssh-atlas-probe}
	\caption{La connexion des sondes Atlas à l'infrastructure RIPE Atlas \cite{how-we-manage-our-probe}}
	\label{fig:ssh-atlas-probe}
\end{figure}

\subsection{Architecture du système RIPE Atlas} \label{subsec:archi-probes}

Il existe deux catégories d'outils de surveillance du réseau: des outils matériels et d'autres logiciels. Les sondes  Atlas sont parmi les outils matériels. le choix d'utilisation d'un outil matériel au lieu d'un outil logiciel dépend de plusieurs facteurs, par exemple l'indépendance du système d'exploitation, la facilité de déploiement, la disponibilité des sondes tout le temps (au lieu d'être dépendante de la machine qui l'héberge) et d'autres facteurs liés à la sécurité.

Le système RIPE Atlas est conçu pour qu'il soit opérationnel de façon distribuée. La plupart des composantes ont assez de connaissances pour remplir leurs rôles, sans nécessairement avoir besoin de connaître les états des autres composantes du système. Cela assure que le système soit capable d'assurer la plupart des fonctionnalités en cas  d'un problème temporaire. Par exemple, si une sonde est déconnectée de l'infrastructure, elle continue les mesures planifiées et les données sont renvoyées dès sa reconnexion au système.


La figure  \ref{fig:archi-ripe-atlas} montre une vue d'ensemble de l'architecture du RIPE Atlas.

\begin{figure}[H]
	\centering
	\resizebox{\textwidth}{8 cm}{
		% Graphic for TeX using PGF
% Title: /home/bellafkih/Documents/2017-2018/Mémoire/report_memoire/illustrations/Diagramme1.dia
% Creator: Dia v0.97.3
% CreationDate: Thu Mar  1 16:26:10 2018
% For: bellafkih
% \usepackage{tikz}
% The following commands are not supported in PSTricks at present
% We define them conditionally, so when they are implemented,
% this pgf file will use them.
\ifx\du\undefined
  \newlength{\du}
\fi
\setlength{\du}{15\unitlength}
\begin{tikzpicture}
\pgftransformxscale{1.000000}
\pgftransformyscale{-1.000000}
\definecolor{dialinecolor}{rgb}{0.000000, 0.000000, 0.000000}
\pgfsetstrokecolor{dialinecolor}
\definecolor{dialinecolor}{rgb}{1.000000, 1.000000, 1.000000}
\pgfsetfillcolor{dialinecolor}
\pgfsetlinewidth{0.100000\du}
\pgfsetdash{}{0pt}
\pgfsetdash{}{0pt}
\pgfsetbuttcap
\pgfsetmiterjoin
\pgfsetlinewidth{0.100000\du}
\pgfsetbuttcap
\pgfsetmiterjoin
\pgfsetdash{}{0pt}
\definecolor{dialinecolor}{rgb}{0.698039, 0.690196, 0.690196}
\pgfsetfillcolor{dialinecolor}
\pgfpathmoveto{\pgfpoint{16.900000\du}{2.091667\du}}
\pgfpathcurveto{\pgfpoint{17.870000\du}{1.610417\du}}{\pgfpoint{18.355000\du}{1.450000\du}}{\pgfpoint{19.325000\du}{1.450000\du}}
\pgfpathcurveto{\pgfpoint{20.295000\du}{1.450000\du}}{\pgfpoint{20.780000\du}{1.610417\du}}{\pgfpoint{21.750000\du}{2.091667\du}}
\pgfpathlineto{\pgfpoint{21.750000\du}{4.658333\du}}
\pgfpathcurveto{\pgfpoint{20.780000\du}{5.139583\du}}{\pgfpoint{20.295000\du}{5.300000\du}}{\pgfpoint{19.325000\du}{5.300000\du}}
\pgfpathcurveto{\pgfpoint{18.355000\du}{5.300000\du}}{\pgfpoint{17.870000\du}{5.139583\du}}{\pgfpoint{16.900000\du}{4.658333\du}}
\pgfpathlineto{\pgfpoint{16.900000\du}{2.091667\du}}
\pgfusepath{fill}
\definecolor{dialinecolor}{rgb}{0.000000, 0.000000, 0.000000}
\pgfsetstrokecolor{dialinecolor}
\pgfpathmoveto{\pgfpoint{16.900000\du}{2.091667\du}}
\pgfpathcurveto{\pgfpoint{17.870000\du}{1.610417\du}}{\pgfpoint{18.355000\du}{1.450000\du}}{\pgfpoint{19.325000\du}{1.450000\du}}
\pgfpathcurveto{\pgfpoint{20.295000\du}{1.450000\du}}{\pgfpoint{20.780000\du}{1.610417\du}}{\pgfpoint{21.750000\du}{2.091667\du}}
\pgfpathlineto{\pgfpoint{21.750000\du}{4.658333\du}}
\pgfpathcurveto{\pgfpoint{20.780000\du}{5.139583\du}}{\pgfpoint{20.295000\du}{5.300000\du}}{\pgfpoint{19.325000\du}{5.300000\du}}
\pgfpathcurveto{\pgfpoint{18.355000\du}{5.300000\du}}{\pgfpoint{17.870000\du}{5.139583\du}}{\pgfpoint{16.900000\du}{4.658333\du}}
\pgfpathlineto{\pgfpoint{16.900000\du}{2.091667\du}}
\pgfusepath{stroke}
\pgfsetbuttcap
\pgfsetmiterjoin
\pgfsetdash{}{0pt}
\definecolor{dialinecolor}{rgb}{0.000000, 0.000000, 0.000000}
\pgfsetstrokecolor{dialinecolor}
\pgfpathmoveto{\pgfpoint{16.900000\du}{2.091667\du}}
\pgfpathcurveto{\pgfpoint{17.870000\du}{2.572917\du}}{\pgfpoint{18.355000\du}{2.733333\du}}{\pgfpoint{19.325000\du}{2.733333\du}}
\pgfpathcurveto{\pgfpoint{20.295000\du}{2.733333\du}}{\pgfpoint{20.780000\du}{2.572917\du}}{\pgfpoint{21.750000\du}{2.091667\du}}
\pgfusepath{stroke}
% setfont left to latex
\definecolor{dialinecolor}{rgb}{0.000000, 0.000000, 0.000000}
\pgfsetstrokecolor{dialinecolor}
\node at (19.325000\du,3.895833\du){DB};
\definecolor{dialinecolor}{rgb}{0.682353, 0.627451, 0.627451}
\pgfsetfillcolor{dialinecolor}
\fill (10.950000\du,4.100000\du)--(10.950000\du,6.400000\du)--(13.350000\du,6.400000\du)--(13.350000\du,4.100000\du)--cycle;
\pgfsetlinewidth{0.100000\du}
\pgfsetdash{}{0pt}
\pgfsetdash{}{0pt}
\pgfsetmiterjoin
\definecolor{dialinecolor}{rgb}{0.000000, 0.000000, 0.000000}
\pgfsetstrokecolor{dialinecolor}
\draw (10.950000\du,4.100000\du)--(10.950000\du,6.400000\du)--(13.350000\du,6.400000\du)--(13.350000\du,4.100000\du)--cycle;
% setfont left to latex
\definecolor{dialinecolor}{rgb}{0.000000, 0.000000, 0.000000}
\pgfsetstrokecolor{dialinecolor}
\node at (12.150000\du,5.445000\du){UI};
\definecolor{dialinecolor}{rgb}{0.898039, 0.898039, 0.898039}
\pgfsetfillcolor{dialinecolor}
\fill (17.586250\du,6.675000\du)--(17.586250\du,9.350000\du)--(21.750000\du,9.350000\du)--(21.750000\du,6.675000\du)--cycle;
\pgfsetlinewidth{0.050000\du}
\pgfsetdash{}{0pt}
\pgfsetdash{}{0pt}
\pgfsetmiterjoin
\definecolor{dialinecolor}{rgb}{0.000000, 0.000000, 0.000000}
\pgfsetstrokecolor{dialinecolor}
\draw (17.586250\du,6.675000\du)--(17.586250\du,9.350000\du)--(21.750000\du,9.350000\du)--(21.750000\du,6.675000\du)--cycle;
% setfont left to latex
\definecolor{dialinecolor}{rgb}{0.000000, 0.000000, 0.000000}
\pgfsetstrokecolor{dialinecolor}
\node at (19.668125\du,8.207500\du){};
\definecolor{dialinecolor}{rgb}{0.898039, 0.898039, 0.898039}
\pgfsetfillcolor{dialinecolor}
\fill (17.310000\du,6.405000\du)--(17.310000\du,9.080000\du)--(21.473750\du,9.080000\du)--(21.473750\du,6.405000\du)--cycle;
\pgfsetlinewidth{0.050000\du}
\pgfsetdash{}{0pt}
\pgfsetdash{}{0pt}
\pgfsetmiterjoin
\definecolor{dialinecolor}{rgb}{0.000000, 0.000000, 0.000000}
\pgfsetstrokecolor{dialinecolor}
\draw (17.310000\du,6.405000\du)--(17.310000\du,9.080000\du)--(21.473750\du,9.080000\du)--(21.473750\du,6.405000\du)--cycle;
% setfont left to latex
\definecolor{dialinecolor}{rgb}{0.000000, 0.000000, 0.000000}
\pgfsetstrokecolor{dialinecolor}
\node at (19.391875\du,7.937500\du){};
\definecolor{dialinecolor}{rgb}{0.498039, 0.498039, 0.498039}
\pgfsetfillcolor{dialinecolor}
\fill (25.385000\du,5.955000\du)--(25.385000\du,8.830000\du)--(29.885000\du,8.830000\du)--(29.885000\du,5.955000\du)--cycle;
\pgfsetlinewidth{0.050000\du}
\pgfsetdash{}{0pt}
\pgfsetdash{}{0pt}
\pgfsetmiterjoin
\definecolor{dialinecolor}{rgb}{0.000000, 0.000000, 0.000000}
\pgfsetstrokecolor{dialinecolor}
\draw (25.385000\du,5.955000\du)--(25.385000\du,8.830000\du)--(29.885000\du,8.830000\du)--(29.885000\du,5.955000\du)--cycle;
% setfont left to latex
\definecolor{dialinecolor}{rgb}{0.000000, 0.000000, 0.000000}
\pgfsetstrokecolor{dialinecolor}
\node at (27.635000\du,7.587500\du){};
\definecolor{dialinecolor}{rgb}{0.498039, 0.498039, 0.498039}
\pgfsetfillcolor{dialinecolor}
\fill (25.760000\du,6.255000\du)--(25.760000\du,9.130000\du)--(30.260000\du,9.130000\du)--(30.260000\du,6.255000\du)--cycle;
\pgfsetlinewidth{0.050000\du}
\pgfsetdash{}{0pt}
\pgfsetdash{}{0pt}
\pgfsetmiterjoin
\definecolor{dialinecolor}{rgb}{0.000000, 0.000000, 0.000000}
\pgfsetstrokecolor{dialinecolor}
\draw (25.760000\du,6.255000\du)--(25.760000\du,9.130000\du)--(30.260000\du,9.130000\du)--(30.260000\du,6.255000\du)--cycle;
% setfont left to latex
\definecolor{dialinecolor}{rgb}{0.000000, 0.000000, 0.000000}
\pgfsetstrokecolor{dialinecolor}
\node at (28.010000\du,7.887500\du){};
\definecolor{dialinecolor}{rgb}{0.847059, 0.898039, 0.898039}
\pgfsetfillcolor{dialinecolor}
\fill (18.300000\du,11.900000\du)--(18.300000\du,15.025000\du)--(21.850000\du,15.025000\du)--(21.850000\du,11.900000\du)--cycle;
\pgfsetlinewidth{0.050000\du}
\pgfsetdash{}{0pt}
\pgfsetdash{}{0pt}
\pgfsetmiterjoin
\definecolor{dialinecolor}{rgb}{0.000000, 0.000000, 0.000000}
\pgfsetstrokecolor{dialinecolor}
\draw (18.300000\du,11.900000\du)--(18.300000\du,15.025000\du)--(21.850000\du,15.025000\du)--(21.850000\du,11.900000\du)--cycle;
% setfont left to latex
\definecolor{dialinecolor}{rgb}{0.000000, 0.000000, 0.000000}
\pgfsetstrokecolor{dialinecolor}
\node at (20.075000\du,13.657500\du){};
\definecolor{dialinecolor}{rgb}{0.847059, 0.898039, 0.898039}
\pgfsetfillcolor{dialinecolor}
\fill (17.960000\du,11.455000\du)--(17.960000\du,14.580000\du)--(21.510000\du,14.580000\du)--(21.510000\du,11.455000\du)--cycle;
\pgfsetlinewidth{0.050000\du}
\pgfsetdash{}{0pt}
\pgfsetdash{}{0pt}
\pgfsetmiterjoin
\definecolor{dialinecolor}{rgb}{0.000000, 0.000000, 0.000000}
\pgfsetstrokecolor{dialinecolor}
\draw (17.960000\du,11.455000\du)--(17.960000\du,14.580000\du)--(21.510000\du,14.580000\du)--(21.510000\du,11.455000\du)--cycle;
% setfont left to latex
\definecolor{dialinecolor}{rgb}{0.000000, 0.000000, 0.000000}
\pgfsetstrokecolor{dialinecolor}
\node at (19.735000\du,13.212500\du){};
\pgfsetlinewidth{0.050000\du}
\pgfsetdash{}{0pt}
\pgfsetdash{}{0pt}
\pgfsetbuttcap
\pgfsetmiterjoin
\pgfsetlinewidth{0.050000\du}
\pgfsetbuttcap
\pgfsetmiterjoin
\pgfsetdash{}{0pt}
\definecolor{dialinecolor}{rgb}{0.854902, 0.788235, 0.854902}
\pgfsetfillcolor{dialinecolor}
\pgfpathmoveto{\pgfpoint{7.085270\du}{11.834673\du}}
\pgfpathcurveto{\pgfpoint{6.451876\du}{11.817409\du}}{\pgfpoint{5.223477\du}{12.179945\du}}{\pgfpoint{5.396221\du}{12.956808\du}}
\pgfpathcurveto{\pgfpoint{5.568963\du}{13.733671\du}}{\pgfpoint{6.394295\du}{13.906300\du}}{\pgfpoint{6.739782\du}{13.681880\du}}
\pgfpathcurveto{\pgfpoint{7.085270\du}{13.457453\du}}{\pgfpoint{6.202357\du}{14.769481\du}}{\pgfpoint{7.891408\du}{15.114754\du}}
\pgfpathcurveto{\pgfpoint{9.580443\du}{15.460026\du}}{\pgfpoint{10.444162\du}{14.907590\du}}{\pgfpoint{10.194643\du}{14.510527\du}}
\pgfpathcurveto{\pgfpoint{9.945124\du}{14.113464\du}}{\pgfpoint{11.672562\du}{15.442763\du}}{\pgfpoint{12.478700\du}{14.683163\du}}
\pgfpathcurveto{\pgfpoint{13.284837\du}{13.923564\du}}{\pgfpoint{11.653368\du}{13.198499\du}}{\pgfpoint{11.998856\du}{13.302081\du}}
\pgfpathcurveto{\pgfpoint{12.344343\du}{13.405662\du}}{\pgfpoint{13.400000\du}{13.267553\du}}{\pgfpoint{13.054512\du}{11.972782\du}}
\pgfpathcurveto{\pgfpoint{12.709025\du}{10.678010\du}}{\pgfpoint{9.599636\du}{11.679300\du}}{\pgfpoint{9.945124\du}{11.489401\du}}
\pgfpathcurveto{\pgfpoint{10.290612\du}{11.299501\du}}{\pgfpoint{9.426893\du}{10.350000\du}}{\pgfpoint{8.352058\du}{10.539900\du}}
\pgfpathcurveto{\pgfpoint{7.277207\du}{10.729801\du}}{\pgfpoint{7.200970\du}{11.074400\du}}{\pgfpoint{7.085807\du}{11.834000\du}}
\pgfpathlineto{\pgfpoint{7.085270\du}{11.834673\du}}
\pgfusepath{fill}
\definecolor{dialinecolor}{rgb}{0.000000, 0.000000, 0.000000}
\pgfsetstrokecolor{dialinecolor}
\pgfpathmoveto{\pgfpoint{7.085270\du}{11.834673\du}}
\pgfpathcurveto{\pgfpoint{6.451876\du}{11.817409\du}}{\pgfpoint{5.223477\du}{12.179945\du}}{\pgfpoint{5.396221\du}{12.956808\du}}
\pgfpathcurveto{\pgfpoint{5.568963\du}{13.733671\du}}{\pgfpoint{6.394295\du}{13.906300\du}}{\pgfpoint{6.739782\du}{13.681880\du}}
\pgfpathcurveto{\pgfpoint{7.085270\du}{13.457453\du}}{\pgfpoint{6.202357\du}{14.769481\du}}{\pgfpoint{7.891408\du}{15.114754\du}}
\pgfpathcurveto{\pgfpoint{9.580443\du}{15.460026\du}}{\pgfpoint{10.444162\du}{14.907590\du}}{\pgfpoint{10.194643\du}{14.510527\du}}
\pgfpathcurveto{\pgfpoint{9.945124\du}{14.113464\du}}{\pgfpoint{11.672562\du}{15.442763\du}}{\pgfpoint{12.478700\du}{14.683163\du}}
\pgfpathcurveto{\pgfpoint{13.284837\du}{13.923564\du}}{\pgfpoint{11.653368\du}{13.198499\du}}{\pgfpoint{11.998856\du}{13.302081\du}}
\pgfpathcurveto{\pgfpoint{12.344343\du}{13.405662\du}}{\pgfpoint{13.400000\du}{13.267553\du}}{\pgfpoint{13.054512\du}{11.972782\du}}
\pgfpathcurveto{\pgfpoint{12.709025\du}{10.678010\du}}{\pgfpoint{9.599636\du}{11.679300\du}}{\pgfpoint{9.945124\du}{11.489401\du}}
\pgfpathcurveto{\pgfpoint{10.290612\du}{11.299501\du}}{\pgfpoint{9.426893\du}{10.350000\du}}{\pgfpoint{8.352058\du}{10.539900\du}}
\pgfpathcurveto{\pgfpoint{7.277207\du}{10.729801\du}}{\pgfpoint{7.200970\du}{11.074400\du}}{\pgfpoint{7.085807\du}{11.834000\du}}
\pgfpathlineto{\pgfpoint{7.085270\du}{11.834673\du}}
\pgfusepath{stroke}
% setfont left to latex
\definecolor{dialinecolor}{rgb}{0.000000, 0.000000, 0.000000}
\pgfsetstrokecolor{dialinecolor}
\node at (9.530923\du,13.195092\du){Data Storage};
\definecolor{dialinecolor}{rgb}{0.380392, 0.549020, 0.603922}
\pgfsetfillcolor{dialinecolor}
\fill (22.111250\du,16.630000\du)--(22.111250\du,19.305000\du)--(26.700000\du,19.305000\du)--(26.700000\du,16.630000\du)--cycle;
\pgfsetlinewidth{0.050000\du}
\pgfsetdash{}{0pt}
\pgfsetdash{}{0pt}
\pgfsetmiterjoin
\definecolor{dialinecolor}{rgb}{0.000000, 0.000000, 0.000000}
\pgfsetstrokecolor{dialinecolor}
\draw (22.111250\du,16.630000\du)--(22.111250\du,19.305000\du)--(26.700000\du,19.305000\du)--(26.700000\du,16.630000\du)--cycle;
% setfont left to latex
\definecolor{dialinecolor}{rgb}{0.000000, 0.000000, 0.000000}
\pgfsetstrokecolor{dialinecolor}
\node at (24.405625\du,18.162500\du){Controller};
\definecolor{dialinecolor}{rgb}{0.380392, 0.549020, 0.603922}
\pgfsetfillcolor{dialinecolor}
\fill (13.760000\du,16.630000\du)--(13.760000\du,19.305000\du)--(18.348750\du,19.305000\du)--(18.348750\du,16.630000\du)--cycle;
\pgfsetlinewidth{0.050000\du}
\pgfsetdash{}{0pt}
\pgfsetdash{}{0pt}
\pgfsetmiterjoin
\definecolor{dialinecolor}{rgb}{0.000000, 0.000000, 0.000000}
\pgfsetstrokecolor{dialinecolor}
\draw (13.760000\du,16.630000\du)--(13.760000\du,19.305000\du)--(18.348750\du,19.305000\du)--(18.348750\du,16.630000\du)--cycle;
% setfont left to latex
\definecolor{dialinecolor}{rgb}{0.000000, 0.000000, 0.000000}
\pgfsetstrokecolor{dialinecolor}
\node at (16.054375\du,18.162500\du){Controller};
\definecolor{dialinecolor}{rgb}{0.996078, 0.960784, 1.000000}
\pgfsetfillcolor{dialinecolor}
\fill (12.471250\du,21.075000\du)--(12.471250\du,22.925000\du)--(15.528750\du,22.925000\du)--(15.528750\du,21.075000\du)--cycle;
\pgfsetlinewidth{0.050000\du}
\pgfsetdash{}{0pt}
\pgfsetdash{}{0pt}
\pgfsetmiterjoin
\definecolor{dialinecolor}{rgb}{0.000000, 0.000000, 0.000000}
\pgfsetstrokecolor{dialinecolor}
\draw (12.471250\du,21.075000\du)--(12.471250\du,22.925000\du)--(15.528750\du,22.925000\du)--(15.528750\du,21.075000\du)--cycle;
% setfont left to latex
\definecolor{dialinecolor}{rgb}{0.000000, 0.000000, 0.000000}
\pgfsetstrokecolor{dialinecolor}
\node at (14.000000\du,22.195000\du){Sonde};
\definecolor{dialinecolor}{rgb}{0.996078, 0.960784, 1.000000}
\pgfsetfillcolor{dialinecolor}
\fill (17.360000\du,21.080000\du)--(17.360000\du,22.930000\du)--(20.417500\du,22.930000\du)--(20.417500\du,21.080000\du)--cycle;
\pgfsetlinewidth{0.050000\du}
\pgfsetdash{}{0pt}
\pgfsetdash{}{0pt}
\pgfsetmiterjoin
\definecolor{dialinecolor}{rgb}{0.000000, 0.000000, 0.000000}
\pgfsetstrokecolor{dialinecolor}
\draw (17.360000\du,21.080000\du)--(17.360000\du,22.930000\du)--(20.417500\du,22.930000\du)--(20.417500\du,21.080000\du)--cycle;
% setfont left to latex
\definecolor{dialinecolor}{rgb}{0.000000, 0.000000, 0.000000}
\pgfsetstrokecolor{dialinecolor}
\node at (18.888750\du,22.200000\du){Sonde};
\definecolor{dialinecolor}{rgb}{0.996078, 0.960784, 1.000000}
\pgfsetfillcolor{dialinecolor}
\fill (20.845000\du,21.035000\du)--(20.845000\du,22.885000\du)--(23.902500\du,22.885000\du)--(23.902500\du,21.035000\du)--cycle;
\pgfsetlinewidth{0.050000\du}
\pgfsetdash{}{0pt}
\pgfsetdash{}{0pt}
\pgfsetmiterjoin
\definecolor{dialinecolor}{rgb}{0.000000, 0.000000, 0.000000}
\pgfsetstrokecolor{dialinecolor}
\draw (20.845000\du,21.035000\du)--(20.845000\du,22.885000\du)--(23.902500\du,22.885000\du)--(23.902500\du,21.035000\du)--cycle;
% setfont left to latex
\definecolor{dialinecolor}{rgb}{0.000000, 0.000000, 0.000000}
\pgfsetstrokecolor{dialinecolor}
\node at (22.373750\du,22.155000\du){Sonde};
\definecolor{dialinecolor}{rgb}{0.996078, 0.960784, 1.000000}
\pgfsetfillcolor{dialinecolor}
\fill (25.680000\du,20.990000\du)--(25.680000\du,22.840000\du)--(28.737500\du,22.840000\du)--(28.737500\du,20.990000\du)--cycle;
\pgfsetlinewidth{0.050000\du}
\pgfsetdash{}{0pt}
\pgfsetdash{}{0pt}
\pgfsetmiterjoin
\definecolor{dialinecolor}{rgb}{0.000000, 0.000000, 0.000000}
\pgfsetstrokecolor{dialinecolor}
\draw (25.680000\du,20.990000\du)--(25.680000\du,22.840000\du)--(28.737500\du,22.840000\du)--(28.737500\du,20.990000\du)--cycle;
% setfont left to latex
\definecolor{dialinecolor}{rgb}{0.000000, 0.000000, 0.000000}
\pgfsetstrokecolor{dialinecolor}
\node at (27.208750\du,22.110000\du){Sonde};
% setfont left to latex
\definecolor{dialinecolor}{rgb}{0.000000, 0.000000, 0.000000}
\pgfsetstrokecolor{dialinecolor}
\node[anchor=west] at (26.350000\du,7.750000\du){Reg.server};
\pgfsetlinewidth{0.050000\du}
\pgfsetdash{}{0pt}
\pgfsetdash{}{0pt}
\pgfsetbuttcap
{
\definecolor{dialinecolor}{rgb}{0.000000, 0.000000, 0.000000}
\pgfsetfillcolor{dialinecolor}
% was here!!!
\definecolor{dialinecolor}{rgb}{0.000000, 0.000000, 0.000000}
\pgfsetstrokecolor{dialinecolor}
\draw (21.750000\du,3.054167\du)--(25.385000\du,7.392500\du);
}
\pgfsetlinewidth{0.050000\du}
\pgfsetdash{}{0pt}
\pgfsetdash{}{0pt}
\pgfsetbuttcap
{
\definecolor{dialinecolor}{rgb}{0.000000, 0.000000, 0.000000}
\pgfsetfillcolor{dialinecolor}
% was here!!!
\definecolor{dialinecolor}{rgb}{0.000000, 0.000000, 0.000000}
\pgfsetstrokecolor{dialinecolor}
\draw (13.350000\du,5.250000\du)--(16.900000\du,3.054167\du);
}
% setfont left to latex
\definecolor{dialinecolor}{rgb}{0.000000, 0.000000, 0.000000}
\pgfsetstrokecolor{dialinecolor}
\node[anchor=west] at (18.600000\du,7.900000\du){Brain};
\pgfsetlinewidth{0.050000\du}
\pgfsetdash{}{0pt}
\pgfsetdash{}{0pt}
\pgfsetbuttcap
{
\definecolor{dialinecolor}{rgb}{0.000000, 0.000000, 0.000000}
\pgfsetfillcolor{dialinecolor}
% was here!!!
\definecolor{dialinecolor}{rgb}{0.000000, 0.000000, 0.000000}
\pgfsetstrokecolor{dialinecolor}
\draw (19.325000\du,5.300000\du)--(19.354568\du,6.379924\du);
}
% setfont left to latex
\definecolor{dialinecolor}{rgb}{0.000000, 0.000000, 0.000000}
\pgfsetstrokecolor{dialinecolor}
\node[anchor=west] at (19.150000\du,13.300000\du){MQ};
\pgfsetlinewidth{0.050000\du}
\pgfsetdash{}{0pt}
\pgfsetdash{}{0pt}
\pgfsetbuttcap
{
\definecolor{dialinecolor}{rgb}{0.000000, 0.000000, 0.000000}
\pgfsetfillcolor{dialinecolor}
% was here!!!
\definecolor{dialinecolor}{rgb}{0.000000, 0.000000, 0.000000}
\pgfsetstrokecolor{dialinecolor}
\draw (19.668125\du,9.350000\du)--(19.735000\du,11.455000\du);
}
\pgfsetlinewidth{0.050000\du}
\pgfsetdash{}{0pt}
\pgfsetdash{}{0pt}
\pgfsetbuttcap
{
\definecolor{dialinecolor}{rgb}{0.000000, 0.000000, 0.000000}
\pgfsetfillcolor{dialinecolor}
% was here!!!
\definecolor{dialinecolor}{rgb}{0.000000, 0.000000, 0.000000}
\pgfsetstrokecolor{dialinecolor}
\draw (10.026768\du,11.262769\du)--(12.150000\du,6.400000\du);
}
\pgfsetlinewidth{0.050000\du}
\pgfsetdash{}{0pt}
\pgfsetdash{}{0pt}
\pgfsetbuttcap
{
\definecolor{dialinecolor}{rgb}{0.000000, 0.000000, 0.000000}
\pgfsetfillcolor{dialinecolor}
% was here!!!
\definecolor{dialinecolor}{rgb}{0.000000, 0.000000, 0.000000}
\pgfsetstrokecolor{dialinecolor}
\draw (13.015462\du,12.942453\du)--(17.960000\du,13.017500\du);
}
\pgfsetlinewidth{0.050000\du}
\pgfsetdash{}{0pt}
\pgfsetdash{}{0pt}
\pgfsetbuttcap
{
\definecolor{dialinecolor}{rgb}{0.000000, 0.000000, 0.000000}
\pgfsetfillcolor{dialinecolor}
% was here!!!
\definecolor{dialinecolor}{rgb}{0.000000, 0.000000, 0.000000}
\pgfsetstrokecolor{dialinecolor}
\draw (20.075000\du,15.025000\du)--(24.405625\du,16.630000\du);
}
\pgfsetlinewidth{0.050000\du}
\pgfsetdash{}{0pt}
\pgfsetdash{}{0pt}
\pgfsetbuttcap
{
\definecolor{dialinecolor}{rgb}{0.000000, 0.000000, 0.000000}
\pgfsetfillcolor{dialinecolor}
% was here!!!
\definecolor{dialinecolor}{rgb}{0.000000, 0.000000, 0.000000}
\pgfsetstrokecolor{dialinecolor}
\draw (20.075000\du,15.025000\du)--(16.054375\du,16.630000\du);
}
\pgfsetlinewidth{0.050000\du}
\pgfsetdash{}{0pt}
\pgfsetdash{}{0pt}
\pgfsetbuttcap
{
\definecolor{dialinecolor}{rgb}{0.000000, 0.000000, 0.000000}
\pgfsetfillcolor{dialinecolor}
% was here!!!
\definecolor{dialinecolor}{rgb}{0.000000, 0.000000, 0.000000}
\pgfsetstrokecolor{dialinecolor}
\draw (14.907187\du,19.305000\du)--(14.000000\du,21.075000\du);
}
\pgfsetlinewidth{0.050000\du}
\pgfsetdash{}{0pt}
\pgfsetdash{}{0pt}
\pgfsetbuttcap
{
\definecolor{dialinecolor}{rgb}{0.000000, 0.000000, 0.000000}
\pgfsetfillcolor{dialinecolor}
% was here!!!
\definecolor{dialinecolor}{rgb}{0.000000, 0.000000, 0.000000}
\pgfsetstrokecolor{dialinecolor}
\draw (17.201562\du,19.305000\du)--(18.295186\du,21.055122\du);
}
\pgfsetlinewidth{0.050000\du}
\pgfsetdash{}{0pt}
\pgfsetdash{}{0pt}
\pgfsetbuttcap
{
\definecolor{dialinecolor}{rgb}{0.000000, 0.000000, 0.000000}
\pgfsetfillcolor{dialinecolor}
% was here!!!
\definecolor{dialinecolor}{rgb}{0.000000, 0.000000, 0.000000}
\pgfsetstrokecolor{dialinecolor}
\draw (23.503782\du,19.329003\du)--(22.373750\du,21.035000\du);
}
\pgfsetlinewidth{0.050000\du}
\pgfsetdash{}{0pt}
\pgfsetdash{}{0pt}
\pgfsetbuttcap
{
\definecolor{dialinecolor}{rgb}{0.000000, 0.000000, 0.000000}
\pgfsetfillcolor{dialinecolor}
% was here!!!
\definecolor{dialinecolor}{rgb}{0.000000, 0.000000, 0.000000}
\pgfsetstrokecolor{dialinecolor}
\draw (25.552813\du,19.305000\du)--(27.208750\du,20.990000\du);
}
\pgfsetlinewidth{0.100000\du}
\pgfsetdash{{\pgflinewidth}{0.200000\du}}{0cm}
\pgfsetdash{{\pgflinewidth}{0.200000\du}}{0cm}
\pgfsetbuttcap
{
\definecolor{dialinecolor}{rgb}{0.000000, 0.000000, 0.000000}
\pgfsetfillcolor{dialinecolor}
% was here!!!
\pgfsetarrowsend{stealth}
\definecolor{dialinecolor}{rgb}{0.000000, 0.000000, 0.000000}
\pgfsetstrokecolor{dialinecolor}
\pgfpathmoveto{\pgfpoint{28.736844\du}{21.915314\du}}
\pgfpatharc{65}{-52}{8.407488\du and 8.407488\du}
\pgfusepath{stroke}
}
\pgfsetlinewidth{0.050000\du}
\pgfsetdash{}{0pt}
\pgfsetdash{}{0pt}
\pgfsetbuttcap
{
\definecolor{dialinecolor}{rgb}{0.000000, 0.000000, 0.000000}
\pgfsetfillcolor{dialinecolor}
% was here!!!
\definecolor{dialinecolor}{rgb}{0.000000, 0.000000, 0.000000}
\pgfsetstrokecolor{dialinecolor}
\draw (25.385000\du,7.392500\du)--(19.735000\du,11.455000\du);
}
\pgfsetlinewidth{0.050000\du}
\pgfsetdash{}{0pt}
\pgfsetdash{}{0pt}
\pgfsetbuttcap
{
\definecolor{dialinecolor}{rgb}{0.000000, 0.000000, 0.000000}
\pgfsetfillcolor{dialinecolor}
% was here!!!
\definecolor{dialinecolor}{rgb}{0.000000, 0.000000, 0.000000}
\pgfsetstrokecolor{dialinecolor}
\draw (13.350000\du,6.400000\du)--(19.735000\du,11.455000\du);
}
\end{tikzpicture}
 
	}
	\caption{Architecture du système  RIPE Atlas \cite{WinNT}}
	\label{fig:archi-ripe-atlas}
	
	
\end{figure}
On distingue les composantes suivantes:
\begin{description}
	
	\item [Registration server] (Reg.server) : c'est le seul point d'entrée de confiance pour les sondes Atlas. Son rôle est de recevoir toutes les sondes désirant se connecter au système RIPE Atlas. Ensuite, il redirige chaque sonde vers le contrôleur adéquat, celui le plus proche de la sonde et que ce contrôleur  soit suffisamment non occupé.  Le serveur d'enregistrement  a un aperçu de haut niveau du système.
	
	\item [Controller] : les contrôleurs acceptent d'établir une connexion avec une sonde parmi les sondes dont ils ont reçu leurs clés du serveur d'enregistrement (Reg.server). Une fois la connexion est établie entre une sonde et un contrôleur, ce dernier garde cette connexion active pour recevoir les résultats et prévenir la sonde des mesures à effectuer.  Le rôle du contrôleur est de communiquer avec les sondes,  associer les mesures aux sondes en se basant sur la disponibilité de la sonde et autres critères, enfin, collecter les résultats intermédiaires des mesures.
	
	\item [Message Queue (MQ)] : Tout d'abord définissons MQ:
	
	\begin{tcolorbox}[title=Message Queue]
		\og    \textbf{\textit{Message Queue ou file d'attente de message}} \textit{:  est une technique de programmation utilisée pour la communication interprocessus ou la communication de serveur-à-serveur. Les files d'attente de message permettent le fonctionnement des liaisons asynchrones normalisées entre deux serveurs, c'est-à-dire de canaux de communications tels que l'expéditeur et le récepteur du message ne sont pas contraints de s'attendre l'un l'autre, mais poursuivent chacun l'exécution de leurs tâches} \footnote{Source : \url{https://fr.wikipedia.org/wiki/File\_d'attente\_de\_message}, consultée le $05/08/2018$.}. \fg{}
	\end{tcolorbox} 
	
	Un cluster de serveurs MQ  agit comme un système nerveux central au sein de l'architecture du RIPE Atlas. Il gère la connectivité entre les composantes de l'infrastructure et  assure l'échange de messages avec un délai minimal. C'est cette composante qui élimine le besoin que les autres composantes de l'infrastructure soient au courant des états des autres composantes de l'infrastructure. En plus, chaque composante peut être ajoutée ou retirée sans devoir synchroniser cette information à l'infrastructure entière. Si c'est le cas d'une déconnexion d'une composante, les messages seront sauvegardés sur différents niveaux jusqu'au moment de la reconnexion.
	
	
	
	\item [UI] (User Interface): elle s'occupe des interactions de l'utilisateur. Elle sert les pages pour l'interface graphique de mesures \cite{create-UDM}. Elle traite les appels en provenance de l'API \footnote{Source : \url{https://atlas.ripe.net/docs/api/v2/manual/}, consultée le $05/08/2018$.} et sert les demandes de téléchargement en provenance de l'API.
	
	\item [Brain] : il effectue des tâches de haut niveau dans le système, notamment la planification des mesures. Cette planification est basée sur les demandes reçues via l'interface graphique web de mesures (UI) ou bien via l'API. La planification passe par la  présélection des sondes Atlas et la négociation avec les contrôleurs pour voir la disponibilité des sondes Atlas. 
	
	
	\item [DB] : c'est une base de données SQL contenant toutes les informations du système RIPE Atlas : les informations sur les sondes et leurs propriétés, les meta-data des mesures, les utilisateurs, les crédits, etc. 
	
	\item [Data Storage] : c'est un cluster Hadoop/HBase pour le stockage à long terme de tous les résultats . Cette technologie permet aussi d'effectuer des calculs d'agrégation périodiques et  d'autres tâches. 
	
	
	\begin{tcolorbox}
		\textbf{\textit{Hadoop MapReduce}} est un modèle de programmation qui permet de traiter les données massives suivant une architecture distribuée dans un cluster.
		
		\textbf{\textit{HBase}} est une base de données non relationnelle et distribuée. Elle est adaptée au stockage de données massives.
	\end{tcolorbox} 
	
\end{description}


La figure \ref{fig:deroulement-connexion-ripe-atlas} présente les étapes d'établissement de la connexion entre une sonde Atlas  et  l'infrastructure RIPE Atlas.

\begin{figure}[H]
	\captionsetup{justification=centering}
	\centering
	\resizebox{\textwidth}{7 cm}{
		% Graphic for TeX using PGF
% Title: /home/bellafkih/Documents/2017-2018/Mémoire/report_memoire/illustrations/deroulement-connexion-ripe-atlas.dia
% Creator: Dia v0.97.3
% CreationDate: Thu Mar  1 20:05:29 2018
% For: bellafkih
% \usepackage{tikz}
% The following commands are not supported in PSTricks at present
% We define them conditionally, so when they are implemented,
% this pgf file will use them.
\ifx\du\undefined
  \newlength{\du}
\fi
\setlength{\du}{15\unitlength}
\begin{tikzpicture}
\pgftransformxscale{1.000000}
\pgftransformyscale{-1.000000}
\definecolor{dialinecolor}{rgb}{0.000000, 0.000000, 0.000000}
\pgfsetstrokecolor{dialinecolor}
\definecolor{dialinecolor}{rgb}{1.000000, 1.000000, 1.000000}
\pgfsetfillcolor{dialinecolor}
\pgfsetlinewidth{0.050000\du}
\pgfsetdash{}{0pt}
\pgfsetdash{}{0pt}
\pgfsetbuttcap
\pgfsetmiterjoin
\pgfsetlinewidth{0.050000\du}
\pgfsetbuttcap
\pgfsetmiterjoin
\pgfsetdash{}{0pt}
\definecolor{dialinecolor}{rgb}{1.000000, 1.000000, 1.000000}
\pgfsetfillcolor{dialinecolor}
\pgfpathmoveto{\pgfpoint{14.432641\du}{1.196287\du}}
\pgfpathcurveto{\pgfpoint{13.879227\du}{1.178888\du}}{\pgfpoint{12.805940\du}{1.544260\du}}{\pgfpoint{12.956871\du}{2.327200\du}}
\pgfpathcurveto{\pgfpoint{13.107801\du}{3.110140\du}}{\pgfpoint{13.828916\du}{3.284120\du}}{\pgfpoint{14.130778\du}{3.057944\du}}
\pgfpathcurveto{\pgfpoint{14.432641\du}{2.831762\du}}{\pgfpoint{13.661215\du}{4.154053\du}}{\pgfpoint{15.136986\du}{4.502027\du}}
\pgfpathcurveto{\pgfpoint{16.612744\du}{4.850000\du}}{\pgfpoint{17.367400\du}{4.293243\du}}{\pgfpoint{17.149388\du}{3.893073\du}}
\pgfpathcurveto{\pgfpoint{16.931377\du}{3.492904\du}}{\pgfpoint{18.440688\du}{4.832601\du}}{\pgfpoint{19.145034\du}{4.067060\du}}
\pgfpathcurveto{\pgfpoint{19.849379\du}{3.301519\du}}{\pgfpoint{18.423918\du}{2.570782\du}}{\pgfpoint{18.725781\du}{2.675174\du}}
\pgfpathcurveto{\pgfpoint{19.027643\du}{2.779566\du}}{\pgfpoint{19.950000\du}{2.640376\du}}{\pgfpoint{19.648138\du}{1.335476\du}}
\pgfpathcurveto{\pgfpoint{19.346275\du}{0.030576\du}}{\pgfpoint{16.629514\du}{1.039699\du}}{\pgfpoint{16.931377\du}{0.848314\du}}
\pgfpathcurveto{\pgfpoint{17.233239\du}{0.656928\du}}{\pgfpoint{16.478583\du}{-0.300000\du}}{\pgfpoint{15.539469\du}{-0.108615\du}}
\pgfpathcurveto{\pgfpoint{14.600342\du}{0.082772\du}}{\pgfpoint{14.533731\du}{0.430067\du}}{\pgfpoint{14.433110\du}{1.195608\du}}
\pgfpathlineto{\pgfpoint{14.432641\du}{1.196287\du}}
\pgfusepath{fill}
\definecolor{dialinecolor}{rgb}{0.890196, 0.835294, 0.835294}
\pgfsetstrokecolor{dialinecolor}
\pgfpathmoveto{\pgfpoint{14.432641\du}{1.196287\du}}
\pgfpathcurveto{\pgfpoint{13.879227\du}{1.178888\du}}{\pgfpoint{12.805940\du}{1.544260\du}}{\pgfpoint{12.956871\du}{2.327200\du}}
\pgfpathcurveto{\pgfpoint{13.107801\du}{3.110140\du}}{\pgfpoint{13.828916\du}{3.284120\du}}{\pgfpoint{14.130778\du}{3.057944\du}}
\pgfpathcurveto{\pgfpoint{14.432641\du}{2.831762\du}}{\pgfpoint{13.661215\du}{4.154053\du}}{\pgfpoint{15.136986\du}{4.502027\du}}
\pgfpathcurveto{\pgfpoint{16.612744\du}{4.850000\du}}{\pgfpoint{17.367400\du}{4.293243\du}}{\pgfpoint{17.149388\du}{3.893073\du}}
\pgfpathcurveto{\pgfpoint{16.931377\du}{3.492904\du}}{\pgfpoint{18.440688\du}{4.832601\du}}{\pgfpoint{19.145034\du}{4.067060\du}}
\pgfpathcurveto{\pgfpoint{19.849379\du}{3.301519\du}}{\pgfpoint{18.423918\du}{2.570782\du}}{\pgfpoint{18.725781\du}{2.675174\du}}
\pgfpathcurveto{\pgfpoint{19.027643\du}{2.779566\du}}{\pgfpoint{19.950000\du}{2.640376\du}}{\pgfpoint{19.648138\du}{1.335476\du}}
\pgfpathcurveto{\pgfpoint{19.346275\du}{0.030576\du}}{\pgfpoint{16.629514\du}{1.039699\du}}{\pgfpoint{16.931377\du}{0.848314\du}}
\pgfpathcurveto{\pgfpoint{17.233239\du}{0.656928\du}}{\pgfpoint{16.478583\du}{-0.300000\du}}{\pgfpoint{15.539469\du}{-0.108615\du}}
\pgfpathcurveto{\pgfpoint{14.600342\du}{0.082772\du}}{\pgfpoint{14.533731\du}{0.430067\du}}{\pgfpoint{14.433110\du}{1.195608\du}}
\pgfpathlineto{\pgfpoint{14.432641\du}{1.196287\du}}
\pgfusepath{stroke}
% setfont left to latex
\definecolor{dialinecolor}{rgb}{0.000000, 0.000000, 0.000000}
\pgfsetstrokecolor{dialinecolor}
\node at (16.569477\du,2.565784\du){Internet};
\definecolor{dialinecolor}{rgb}{1.000000, 1.000000, 1.000000}
\pgfsetfillcolor{dialinecolor}
\pgfpathellipse{\pgfpoint{8.350000\du}{11.400000\du}}{\pgfpoint{2.597682\du}{0\du}}{\pgfpoint{0\du}{1.298841\du}}
\pgfusepath{fill}
\pgfsetlinewidth{0.050000\du}
\pgfsetdash{}{0pt}
\pgfsetdash{}{0pt}
\pgfsetmiterjoin
\definecolor{dialinecolor}{rgb}{0.380392, 0.549020, 0.603922}
\pgfsetstrokecolor{dialinecolor}
\pgfpathellipse{\pgfpoint{8.350000\du}{11.400000\du}}{\pgfpoint{2.597682\du}{0\du}}{\pgfpoint{0\du}{1.298841\du}}
\pgfusepath{stroke}
% setfont left to latex
\definecolor{dialinecolor}{rgb}{0.000000, 0.000000, 0.000000}
\pgfsetstrokecolor{dialinecolor}
\node at (8.350000\du,11.595000\du){Reg.server};
\definecolor{dialinecolor}{rgb}{1.000000, 1.000000, 1.000000}
\pgfsetfillcolor{dialinecolor}
\pgfpathellipse{\pgfpoint{5.000000\du}{4.000000\du}}{\pgfpoint{2.248813\du}{0\du}}{\pgfpoint{0\du}{1.124406\du}}
\pgfusepath{fill}
\pgfsetlinewidth{0.050000\du}
\pgfsetdash{}{0pt}
\pgfsetdash{}{0pt}
\pgfsetmiterjoin
\definecolor{dialinecolor}{rgb}{0.780392, 0.156863, 0.156863}
\pgfsetstrokecolor{dialinecolor}
\pgfpathellipse{\pgfpoint{5.000000\du}{4.000000\du}}{\pgfpoint{2.248813\du}{0\du}}{\pgfpoint{0\du}{1.124406\du}}
\pgfusepath{stroke}
% setfont left to latex
\definecolor{dialinecolor}{rgb}{0.000000, 0.000000, 0.000000}
\pgfsetstrokecolor{dialinecolor}
\node at (5.000000\du,4.195000\du){Sonde s};
\pgfsetlinewidth{0.050000\du}
\pgfsetdash{}{0pt}
\pgfsetdash{}{0pt}
\pgfsetbuttcap
\pgfsetmiterjoin
\pgfsetlinewidth{0.050000\du}
\pgfsetbuttcap
\pgfsetmiterjoin
\pgfsetdash{}{0pt}
\definecolor{dialinecolor}{rgb}{1.000000, 1.000000, 1.000000}
\pgfsetfillcolor{dialinecolor}
\pgfpathmoveto{\pgfpoint{21.196250\du}{10.000000\du}}
\pgfpathlineto{\pgfpoint{25.203750\du}{10.000000\du}}
\pgfpathcurveto{\pgfpoint{25.757071\du}{10.000000\du}}{\pgfpoint{26.205625\du}{10.447715\du}}{\pgfpoint{26.205625\du}{11.000000\du}}
\pgfpathcurveto{\pgfpoint{26.205625\du}{11.552285\du}}{\pgfpoint{25.757071\du}{12.000000\du}}{\pgfpoint{25.203750\du}{12.000000\du}}
\pgfpathlineto{\pgfpoint{21.196250\du}{12.000000\du}}
\pgfpathcurveto{\pgfpoint{20.642929\du}{12.000000\du}}{\pgfpoint{20.194375\du}{11.552285\du}}{\pgfpoint{20.194375\du}{11.000000\du}}
\pgfpathcurveto{\pgfpoint{20.194375\du}{10.447715\du}}{\pgfpoint{20.642929\du}{10.000000\du}}{\pgfpoint{21.196250\du}{10.000000\du}}
\pgfusepath{fill}
\definecolor{dialinecolor}{rgb}{0.376471, 0.321569, 0.329412}
\pgfsetstrokecolor{dialinecolor}
\pgfpathmoveto{\pgfpoint{21.196250\du}{10.000000\du}}
\pgfpathlineto{\pgfpoint{25.203750\du}{10.000000\du}}
\pgfpathcurveto{\pgfpoint{25.757071\du}{10.000000\du}}{\pgfpoint{26.205625\du}{10.447715\du}}{\pgfpoint{26.205625\du}{11.000000\du}}
\pgfpathcurveto{\pgfpoint{26.205625\du}{11.552285\du}}{\pgfpoint{25.757071\du}{12.000000\du}}{\pgfpoint{25.203750\du}{12.000000\du}}
\pgfpathlineto{\pgfpoint{21.196250\du}{12.000000\du}}
\pgfpathcurveto{\pgfpoint{20.642929\du}{12.000000\du}}{\pgfpoint{20.194375\du}{11.552285\du}}{\pgfpoint{20.194375\du}{11.000000\du}}
\pgfpathcurveto{\pgfpoint{20.194375\du}{10.447715\du}}{\pgfpoint{20.642929\du}{10.000000\du}}{\pgfpoint{21.196250\du}{10.000000\du}}
\pgfusepath{stroke}
% setfont left to latex
\definecolor{dialinecolor}{rgb}{0.000000, 0.000000, 0.000000}
\pgfsetstrokecolor{dialinecolor}
\node at (23.200000\du,11.200000\du){Controller 1};
\pgfsetlinewidth{0.050000\du}
\pgfsetdash{}{0pt}
\pgfsetdash{}{0pt}
\pgfsetbuttcap
\pgfsetmiterjoin
\pgfsetlinewidth{0.050000\du}
\pgfsetbuttcap
\pgfsetmiterjoin
\pgfsetdash{}{0pt}
\definecolor{dialinecolor}{rgb}{1.000000, 1.000000, 1.000000}
\pgfsetfillcolor{dialinecolor}
\pgfpathmoveto{\pgfpoint{28.061875\du}{9.980000\du}}
\pgfpathlineto{\pgfpoint{32.069375\du}{9.980000\du}}
\pgfpathcurveto{\pgfpoint{32.622696\du}{9.980000\du}}{\pgfpoint{33.071250\du}{10.427715\du}}{\pgfpoint{33.071250\du}{10.980000\du}}
\pgfpathcurveto{\pgfpoint{33.071250\du}{11.532285\du}}{\pgfpoint{32.622696\du}{11.980000\du}}{\pgfpoint{32.069375\du}{11.980000\du}}
\pgfpathlineto{\pgfpoint{28.061875\du}{11.980000\du}}
\pgfpathcurveto{\pgfpoint{27.508554\du}{11.980000\du}}{\pgfpoint{27.060000\du}{11.532285\du}}{\pgfpoint{27.060000\du}{10.980000\du}}
\pgfpathcurveto{\pgfpoint{27.060000\du}{10.427715\du}}{\pgfpoint{27.508554\du}{9.980000\du}}{\pgfpoint{28.061875\du}{9.980000\du}}
\pgfusepath{fill}
\definecolor{dialinecolor}{rgb}{0.376471, 0.321569, 0.329412}
\pgfsetstrokecolor{dialinecolor}
\pgfpathmoveto{\pgfpoint{28.061875\du}{9.980000\du}}
\pgfpathlineto{\pgfpoint{32.069375\du}{9.980000\du}}
\pgfpathcurveto{\pgfpoint{32.622696\du}{9.980000\du}}{\pgfpoint{33.071250\du}{10.427715\du}}{\pgfpoint{33.071250\du}{10.980000\du}}
\pgfpathcurveto{\pgfpoint{33.071250\du}{11.532285\du}}{\pgfpoint{32.622696\du}{11.980000\du}}{\pgfpoint{32.069375\du}{11.980000\du}}
\pgfpathlineto{\pgfpoint{28.061875\du}{11.980000\du}}
\pgfpathcurveto{\pgfpoint{27.508554\du}{11.980000\du}}{\pgfpoint{27.060000\du}{11.532285\du}}{\pgfpoint{27.060000\du}{10.980000\du}}
\pgfpathcurveto{\pgfpoint{27.060000\du}{10.427715\du}}{\pgfpoint{27.508554\du}{9.980000\du}}{\pgfpoint{28.061875\du}{9.980000\du}}
\pgfusepath{stroke}
% setfont left to latex
\definecolor{dialinecolor}{rgb}{0.000000, 0.000000, 0.000000}
\pgfsetstrokecolor{dialinecolor}
\node at (30.065625\du,11.180000\du){Controller 2};
\pgfsetlinewidth{0.050000\du}
\pgfsetdash{}{0pt}
\pgfsetdash{}{0pt}
\pgfsetbuttcap
\pgfsetmiterjoin
\pgfsetlinewidth{0.050000\du}
\pgfsetbuttcap
\pgfsetmiterjoin
\pgfsetdash{}{0pt}
\definecolor{dialinecolor}{rgb}{1.000000, 1.000000, 1.000000}
\pgfsetfillcolor{dialinecolor}
\pgfpathmoveto{\pgfpoint{36.495625\du}{9.890000\du}}
\pgfpathlineto{\pgfpoint{40.575625\du}{9.890000\du}}
\pgfpathcurveto{\pgfpoint{41.138956\du}{9.890000\du}}{\pgfpoint{41.595625\du}{10.337715\du}}{\pgfpoint{41.595625\du}{10.890000\du}}
\pgfpathcurveto{\pgfpoint{41.595625\du}{11.442285\du}}{\pgfpoint{41.138956\du}{11.890000\du}}{\pgfpoint{40.575625\du}{11.890000\du}}
\pgfpathlineto{\pgfpoint{36.495625\du}{11.890000\du}}
\pgfpathcurveto{\pgfpoint{35.932294\du}{11.890000\du}}{\pgfpoint{35.475625\du}{11.442285\du}}{\pgfpoint{35.475625\du}{10.890000\du}}
\pgfpathcurveto{\pgfpoint{35.475625\du}{10.337715\du}}{\pgfpoint{35.932294\du}{9.890000\du}}{\pgfpoint{36.495625\du}{9.890000\du}}
\pgfusepath{fill}
\definecolor{dialinecolor}{rgb}{0.376471, 0.321569, 0.329412}
\pgfsetstrokecolor{dialinecolor}
\pgfpathmoveto{\pgfpoint{36.495625\du}{9.890000\du}}
\pgfpathlineto{\pgfpoint{40.575625\du}{9.890000\du}}
\pgfpathcurveto{\pgfpoint{41.138956\du}{9.890000\du}}{\pgfpoint{41.595625\du}{10.337715\du}}{\pgfpoint{41.595625\du}{10.890000\du}}
\pgfpathcurveto{\pgfpoint{41.595625\du}{11.442285\du}}{\pgfpoint{41.138956\du}{11.890000\du}}{\pgfpoint{40.575625\du}{11.890000\du}}
\pgfpathlineto{\pgfpoint{36.495625\du}{11.890000\du}}
\pgfpathcurveto{\pgfpoint{35.932294\du}{11.890000\du}}{\pgfpoint{35.475625\du}{11.442285\du}}{\pgfpoint{35.475625\du}{10.890000\du}}
\pgfpathcurveto{\pgfpoint{35.475625\du}{10.337715\du}}{\pgfpoint{35.932294\du}{9.890000\du}}{\pgfpoint{36.495625\du}{9.890000\du}}
\pgfusepath{stroke}
% setfont left to latex
\definecolor{dialinecolor}{rgb}{0.000000, 0.000000, 0.000000}
\pgfsetstrokecolor{dialinecolor}
\node at (38.535625\du,11.090000\du){Controller n};
\pgfsetlinewidth{0.050000\du}
\pgfsetdash{}{0pt}
\pgfsetdash{}{0pt}
\pgfsetbuttcap
{
\definecolor{dialinecolor}{rgb}{0.000000, 0.000000, 0.000000}
\pgfsetfillcolor{dialinecolor}
% was here!!!
\pgfsetarrowsend{stealth}
\definecolor{dialinecolor}{rgb}{0.000000, 0.000000, 0.000000}
\pgfsetstrokecolor{dialinecolor}
\pgfpathmoveto{\pgfpoint{14.509923\du}{0.625976\du}}
\pgfpatharc{301}{212}{6.611633\du and 6.611633\du}
\pgfusepath{stroke}
}
\pgfsetlinewidth{0.050000\du}
\pgfsetdash{}{0pt}
\pgfsetdash{}{0pt}
\pgfsetbuttcap
{
\definecolor{dialinecolor}{rgb}{0.000000, 0.000000, 0.000000}
\pgfsetfillcolor{dialinecolor}
% was here!!!
\pgfsetarrowsend{stealth}
\definecolor{dialinecolor}{rgb}{0.000000, 0.000000, 0.000000}
\pgfsetstrokecolor{dialinecolor}
\pgfpathmoveto{\pgfpoint{6.589978\du}{4.794797\du}}
\pgfpatharc{149}{30}{5.296909\du and 5.296909\du}
\pgfusepath{stroke}
}
\pgfsetlinewidth{0.050000\du}
\pgfsetdash{}{0pt}
\pgfsetdash{}{0pt}
\pgfsetbuttcap
{
\definecolor{dialinecolor}{rgb}{0.000000, 0.000000, 0.000000}
\pgfsetfillcolor{dialinecolor}
% was here!!!
\pgfsetarrowsend{stealth}
\definecolor{dialinecolor}{rgb}{0.000000, 0.000000, 0.000000}
\pgfsetstrokecolor{dialinecolor}
\pgfpathmoveto{\pgfpoint{3.409957\du}{4.795021\du}}
\pgfpatharc{244}{78}{3.532913\du and 3.532913\du}
\pgfusepath{stroke}
}
\pgfsetlinewidth{0.050000\du}
\pgfsetdash{}{0pt}
\pgfsetdash{}{0pt}
\pgfsetbuttcap
{
\definecolor{dialinecolor}{rgb}{0.000000, 0.000000, 0.000000}
\pgfsetfillcolor{dialinecolor}
% was here!!!
\pgfsetarrowsend{stealth}
\definecolor{dialinecolor}{rgb}{0.000000, 0.000000, 0.000000}
\pgfsetstrokecolor{dialinecolor}
\pgfpathmoveto{\pgfpoint{8.297850\du}{10.076361\du}}
\pgfpatharc{353}{301}{6.825697\du and 6.825697\du}
\pgfusepath{stroke}
}
\pgfsetlinewidth{0.050000\du}
\pgfsetdash{}{0pt}
\pgfsetdash{}{0pt}
\pgfsetbuttcap
{
\definecolor{dialinecolor}{rgb}{0.000000, 0.000000, 0.000000}
\pgfsetfillcolor{dialinecolor}
% was here!!!
\pgfsetarrowsend{stealth}
\definecolor{dialinecolor}{rgb}{0.000000, 0.000000, 0.000000}
\pgfsetstrokecolor{dialinecolor}
\pgfpathmoveto{\pgfpoint{10.028896\du}{12.411148\du}}
\pgfpatharc{119}{60}{18.865547\du and 18.865547\du}
\pgfusepath{stroke}
}
\pgfsetlinewidth{0.050000\du}
\pgfsetdash{}{0pt}
\pgfsetdash{}{0pt}
\pgfsetmiterjoin
\pgfsetbuttcap
{
\definecolor{dialinecolor}{rgb}{0.000000, 0.000000, 0.000000}
\pgfsetfillcolor{dialinecolor}
% was here!!!
\pgfsetarrowsend{stealth}
\definecolor{dialinecolor}{rgb}{0.000000, 0.000000, 0.000000}
\pgfsetstrokecolor{dialinecolor}
\pgfpathmoveto{\pgfpoint{10.186839\du}{10.481581\du}}
\pgfpathcurveto{\pgfpoint{12.686839\du}{6.381581\du}}{\pgfpoint{15.150000\du}{11.050000\du}}{\pgfpoint{11.200000\du}{11.450000\du}}
\pgfusepath{stroke}
}
\pgfsetlinewidth{0.050000\du}
\pgfsetdash{}{0pt}
\pgfsetdash{}{0pt}
\pgfsetbuttcap
{
\definecolor{dialinecolor}{rgb}{0.000000, 0.000000, 0.000000}
\pgfsetfillcolor{dialinecolor}
% was here!!!
\definecolor{dialinecolor}{rgb}{0.000000, 0.000000, 0.000000}
\pgfsetstrokecolor{dialinecolor}
\pgfpathmoveto{\pgfpoint{30.065371\du}{9.981310\du}}
\pgfpatharc{11}{-159}{13.073364\du and 13.073364\du}
\pgfusepath{stroke}
}
\definecolor{dialinecolor}{rgb}{0.698039, 0.690196, 0.690196}
\pgfsetfillcolor{dialinecolor}
\pgfpathellipse{\pgfpoint{9.432069\du}{-1.133941\du}}{\pgfpoint{0.995164\du}{0\du}}{\pgfpoint{0\du}{0.882443\du}}
\pgfusepath{fill}
\pgfsetlinewidth{0.050000\du}
\pgfsetdash{}{0pt}
\pgfsetdash{}{0pt}
\pgfsetmiterjoin
\definecolor{dialinecolor}{rgb}{0.000000, 0.000000, 0.000000}
\pgfsetstrokecolor{dialinecolor}
\pgfpathellipse{\pgfpoint{9.432069\du}{-1.133941\du}}{\pgfpoint{0.995164\du}{0\du}}{\pgfpoint{0\du}{0.882443\du}}
\pgfusepath{stroke}
% setfont left to latex
\definecolor{dialinecolor}{rgb}{0.000000, 0.000000, 0.000000}
\pgfsetstrokecolor{dialinecolor}
\node at (9.432069\du,-0.961719\du){1};
\definecolor{dialinecolor}{rgb}{0.698039, 0.690196, 0.690196}
\pgfsetfillcolor{dialinecolor}
\pgfpathellipse{\pgfpoint{12.005773\du}{6.041280\du}}{\pgfpoint{0.995164\du}{0\du}}{\pgfpoint{0\du}{0.882443\du}}
\pgfusepath{fill}
\pgfsetlinewidth{0.050000\du}
\pgfsetdash{}{0pt}
\pgfsetdash{}{0pt}
\pgfsetmiterjoin
\definecolor{dialinecolor}{rgb}{0.000000, 0.000000, 0.000000}
\pgfsetstrokecolor{dialinecolor}
\pgfpathellipse{\pgfpoint{12.005773\du}{6.041280\du}}{\pgfpoint{0.995164\du}{0\du}}{\pgfpoint{0\du}{0.882443\du}}
\pgfusepath{stroke}
% setfont left to latex
\definecolor{dialinecolor}{rgb}{0.000000, 0.000000, 0.000000}
\pgfsetstrokecolor{dialinecolor}
\node at (12.005773\du,6.213502\du){2};
\definecolor{dialinecolor}{rgb}{0.698039, 0.690196, 0.690196}
\pgfsetfillcolor{dialinecolor}
\pgfpathellipse{\pgfpoint{2.705773\du}{8.741280\du}}{\pgfpoint{0.995164\du}{0\du}}{\pgfpoint{0\du}{0.882443\du}}
\pgfusepath{fill}
\pgfsetlinewidth{0.050000\du}
\pgfsetdash{}{0pt}
\pgfsetdash{}{0pt}
\pgfsetmiterjoin
\definecolor{dialinecolor}{rgb}{0.000000, 0.000000, 0.000000}
\pgfsetstrokecolor{dialinecolor}
\pgfpathellipse{\pgfpoint{2.705773\du}{8.741280\du}}{\pgfpoint{0.995164\du}{0\du}}{\pgfpoint{0\du}{0.882443\du}}
\pgfusepath{stroke}
% setfont left to latex
\definecolor{dialinecolor}{rgb}{0.000000, 0.000000, 0.000000}
\pgfsetstrokecolor{dialinecolor}
\node at (2.705773\du,8.913502\du){3};
\definecolor{dialinecolor}{rgb}{0.698039, 0.690196, 0.690196}
\pgfsetfillcolor{dialinecolor}
\pgfpathellipse{\pgfpoint{13.955773\du}{8.641280\du}}{\pgfpoint{0.995164\du}{0\du}}{\pgfpoint{0\du}{0.882443\du}}
\pgfusepath{fill}
\pgfsetlinewidth{0.050000\du}
\pgfsetdash{}{0pt}
\pgfsetdash{}{0pt}
\pgfsetmiterjoin
\definecolor{dialinecolor}{rgb}{0.000000, 0.000000, 0.000000}
\pgfsetstrokecolor{dialinecolor}
\pgfpathellipse{\pgfpoint{13.955773\du}{8.641280\du}}{\pgfpoint{0.995164\du}{0\du}}{\pgfpoint{0\du}{0.882443\du}}
\pgfusepath{stroke}
% setfont left to latex
\definecolor{dialinecolor}{rgb}{0.000000, 0.000000, 0.000000}
\pgfsetstrokecolor{dialinecolor}
\node at (13.955773\du,8.813502\du){4};
\definecolor{dialinecolor}{rgb}{0.698039, 0.690196, 0.690196}
\pgfsetfillcolor{dialinecolor}
\pgfpathellipse{\pgfpoint{6.455773\du}{7.877012\du}}{\pgfpoint{0.995164\du}{0\du}}{\pgfpoint{0\du}{0.882443\du}}
\pgfusepath{fill}
\pgfsetlinewidth{0.050000\du}
\pgfsetdash{}{0pt}
\pgfsetdash{}{0pt}
\pgfsetmiterjoin
\definecolor{dialinecolor}{rgb}{0.000000, 0.000000, 0.000000}
\pgfsetstrokecolor{dialinecolor}
\pgfpathellipse{\pgfpoint{6.455773\du}{7.877012\du}}{\pgfpoint{0.995164\du}{0\du}}{\pgfpoint{0\du}{0.882443\du}}
\pgfusepath{stroke}
% setfont left to latex
\definecolor{dialinecolor}{rgb}{0.000000, 0.000000, 0.000000}
\pgfsetstrokecolor{dialinecolor}
\node at (6.455773\du,8.049235\du){5};
\definecolor{dialinecolor}{rgb}{0.698039, 0.690196, 0.690196}
\pgfsetfillcolor{dialinecolor}
\pgfpathellipse{\pgfpoint{18.655773\du}{13.627012\du}}{\pgfpoint{0.995164\du}{0\du}}{\pgfpoint{0\du}{0.882443\du}}
\pgfusepath{fill}
\pgfsetlinewidth{0.050000\du}
\pgfsetdash{}{0pt}
\pgfsetdash{}{0pt}
\pgfsetmiterjoin
\definecolor{dialinecolor}{rgb}{0.000000, 0.000000, 0.000000}
\pgfsetstrokecolor{dialinecolor}
\pgfpathellipse{\pgfpoint{18.655773\du}{13.627012\du}}{\pgfpoint{0.995164\du}{0\du}}{\pgfpoint{0\du}{0.882443\du}}
\pgfusepath{stroke}
% setfont left to latex
\definecolor{dialinecolor}{rgb}{0.000000, 0.000000, 0.000000}
\pgfsetstrokecolor{dialinecolor}
\node at (18.655773\du,13.799235\du){5};
\definecolor{dialinecolor}{rgb}{0.698039, 0.690196, 0.690196}
\pgfsetfillcolor{dialinecolor}
\pgfpathellipse{\pgfpoint{0.005773\du}{9.777012\du}}{\pgfpoint{0.995164\du}{0\du}}{\pgfpoint{0\du}{0.882443\du}}
\pgfusepath{fill}
\pgfsetlinewidth{0.050000\du}
\pgfsetdash{}{0pt}
\pgfsetdash{}{0pt}
\pgfsetmiterjoin
\definecolor{dialinecolor}{rgb}{0.000000, 0.000000, 0.000000}
\pgfsetstrokecolor{dialinecolor}
\pgfpathellipse{\pgfpoint{0.005773\du}{9.777012\du}}{\pgfpoint{0.995164\du}{0\du}}{\pgfpoint{0\du}{0.882443\du}}
\pgfusepath{stroke}
% setfont left to latex
\definecolor{dialinecolor}{rgb}{0.000000, 0.000000, 0.000000}
\pgfsetstrokecolor{dialinecolor}
\node at (0.005773\du,9.949235\du){6};
\definecolor{dialinecolor}{rgb}{0.698039, 0.690196, 0.690196}
\pgfsetfillcolor{dialinecolor}
\pgfpathellipse{\pgfpoint{22.855773\du}{-5.598747\du}}{\pgfpoint{0.995164\du}{0\du}}{\pgfpoint{0\du}{0.882443\du}}
\pgfusepath{fill}
\pgfsetlinewidth{0.050000\du}
\pgfsetdash{}{0pt}
\pgfsetdash{}{0pt}
\pgfsetmiterjoin
\definecolor{dialinecolor}{rgb}{0.000000, 0.000000, 0.000000}
\pgfsetstrokecolor{dialinecolor}
\pgfpathellipse{\pgfpoint{22.855773\du}{-5.598747\du}}{\pgfpoint{0.995164\du}{0\du}}{\pgfpoint{0\du}{0.882443\du}}
\pgfusepath{stroke}
% setfont left to latex
\definecolor{dialinecolor}{rgb}{0.000000, 0.000000, 0.000000}
\pgfsetstrokecolor{dialinecolor}
\node at (22.855773\du,-5.426525\du){7};
\pgfsetlinewidth{0.050000\du}
\pgfsetdash{}{0pt}
\pgfsetdash{}{0pt}
\pgfsetbuttcap
{
\definecolor{dialinecolor}{rgb}{0.000000, 0.000000, 0.000000}
\pgfsetfillcolor{dialinecolor}
% was here!!!
\pgfsetarrowsstart{stealth}
\definecolor{dialinecolor}{rgb}{0.000000, 0.000000, 0.000000}
\pgfsetstrokecolor{dialinecolor}
\pgfpathmoveto{\pgfpoint{2.751335\du}{3.999967\du}}
\pgfpatharc{258}{47}{5.068117\du and 5.068117\du}
\pgfusepath{stroke}
}
\end{tikzpicture}
 
	}
	\caption{Les étapes d'établissent d'une connexion entre la sonde Atlas et l'architecture  RIPE Atlas}
	\label{fig:deroulement-connexion-ripe-atlas}
\end{figure}


Les étapes suivantes illustrent le déroulement de la connexion d'une sonde Atlas $s$ à l'infrastructure RIPE Atlas. 

\begin{itemize}
	\item[--] La sonde Atlas se connecte à Internet via  le câble Ethernet $RJ45$ \circled{1}.
	\item[--] La sonde Atlas acquiert différentes informations : une adresse IPv4, une adresse IPv6 via Router Advertisement et les informations du résolveur DNS via DHCP \circled{2}. 
	
	\item[--] Les informations précédemment acquises permettent à la sonde Atlas de se connecter  au serveur d'enregistrement (Reg.server). C'est la première entrée vers l'infrastructure \circled{3}.
	
	\item[--] En se basant sur la géolocalisation de la sonde Atlas, la charge des différents contrôleurs et d'autres options,  le serveur d'enregistrement décide le contrôleur qui va  être associé à la sonde Atlas \circled{4}. 
	
	\item[--] Suite à la décision du serveur d'enregistrement, le contrôleur reçoit l'identifiant de la sonde Atlas à gérer et la sonde Atlas reçoit l'identifiant du contrôleur avec à qui elle sera associée \circled{5}.
	
	\item[--]  Une fois l'association entre la sonde Atlas et le contrôleur est faite,  la sonde Atlas se déconnecte du serveur d'enregistrement \circled{6}.
	
	\item[--] La connexion entre la sonde Atlas et le contrôleur est  maintenue le plus longtemps possible. Les contrôleurs gardent le contact avec les autres composantes via Message Queue. Dans le cas où  une des composantes se déconnecte de l'architecture, les événements sont conservés jusqu'au moment où la connexion est restaurée \circled{7}.
	
\end{itemize}

La connexion précédemment établie permet aux sondes Atlas d'envoyer leurs  rapports de mesures  aux serveurs de stockage. C'est la même connexion qui permet de passer les commandes aux sondes pour qu'elles puissent effectuer les mesures et les mises à jour de leur firmware.




\subsection{Les sondes  Atlas et la vie privée}
La sonde  Atlas n'a pas l'accès au trafic de son hébergeur. Elle maintient sa connexion avec l'infrastructure centrale et elle exécute les mesures planifiées vers les destinations publiques sur Internet. 

Les sondes  Atlas peuvent révéler l'adresse IP de leur hébergeur. Bien que, les informations personnelles telles que les adresses MAC et les adresses e-mail ne seront jamais affichées. Cependant, l'adresse IPv6 peut exposer l'adresse MAC. 

\subsection{La sécurité dans RIPE Atlas}

La connexion entre les composantes de l'infrastructure RIPE Atlas est maintenue le plus longtemps possible comme c'est décrit dans  la section \ref{subsec:archi-probes}. De ce fait, la sécurité des différentes connexions est primordiale. Afin de réduire la surface d'attaque contre ces sondes, les précautions suivantes sont prises:

\begin{itemize}
	\item[--] Les  hébergeurs des sondes  Atlas ne disposent d'aucun service qui leur permet de se connecter aux sondes (dans le sens de TCP/IP).
	\item[--] Les sondes   Atlas n'échangent aucune clé d'authentification entre elles. En effet, chaque sonde dispose de sa clé qu'elle l'utilise pour se connecter à l'infrastructure.
	\item[--] Comme les sondes  Atlas sont chez les hébergeurs, il est impossible qu'elles soient résilientes au démontage. Cependant, si c'était le cas, cela ne devrait pas affecter les autres sondes  Atlas.
	\item[--] Toutes les communications au sein de l'infrastructure RIPE Atlas se font d'une manière sécurisée. Les connexions entre les composantes sont maintenues grâce aux \textit{secure channels} avec \textit{mutual authentication}.
	\item[--] Le logiciel qui tourne dans les sondes  Atlas peut être facilement mis à niveau; la sonde  Atlas est capable de vérifier l'authenticité d'une nouvelle version du firmware et cela via les signatures cryptographiques. 
\end{itemize}

Le système RIPE Atlas est un système comme les autres, il n'est pas résilient à $100$ \% aux attaques. Cependant, l'équipe RIPE Atlas propose régulièrement des améliorations et des fixations de bugs surmontées par la communauté RIPE Atlas. 


\subsection{Les ancres VS sondes  Atlas} \label{subsec:ancre}

Les ancres  Atlas sont des dispositifs agissant comme cibles aux différentes mesures lancées par les sondes  Atlas. Il est possible de planifier des mesures entre les ancres RIPE Atlas, ces mesures permettent de vérifier l'état des réseaux qui hébergent ces ancres. Les ancres Atlas peuvent être considérées comme  cibles aux mesures suivantes:
\begin{itemize}
	\item[--] Ping.
	\item[--]Traceroute.
	\item[--]DNS : les ancres ont été configurées avec BIND pour qu'elles agissent en tant que serveur DNS faisant autorité.
	\item[--]HTTP et HTTPS : l'ancre fait tourner un serveur Web, ce dernier utilise un gestionnaire de réponses personnalisé aux requêtes HTTP(S) ayant comme seule option la taille du payload. 	 Cette taille peut prendre une valeur maximale de $4096$ et la réponse est fournie sous format JSON. L'exemple d'une requête HTTP avec une taille de $536$ depuis une sonde Atlas vers une ancre Atlas est: 
	\begin{center}
		\begin{tcolorbox}
			\textit{http://nl-ams-as3333.anchors.atlas.ripe.net/536}
		\end{tcolorbox}
	\end{center}
\end{itemize}
Les ancres sont configurées avec un certificat SSL auto-signé en utilisant une clé de $2048$ bit et un temps d'expiration de $100$ ans. Le tableau \ref{tab:comparaison-sonde-ancre} reprend une comparaison de certaines caractéristiques communes entre les sondes et les ancres  Atlas.

\begin{table}[H]
	\centering
	\resizebox{\textwidth}{!}{
		\begin{tabular}{l c c}
			& \textbf{Sonde Atlas } & \textbf{Ancre Atlas }  \\ \hline
			\textbf{Mesures originaires de}	    &oui& oui \\ \hline
			\textbf{Mesures à destination de}	    & --- \tablefootnote{--- : Non disponible.} & ping, traceroute, DNS, HTTP(S). \\ \hline
			\textbf{Nomination}                   & --- & structurée
			\tablefootnote{Exemple de \textit{de-mai-as2857.anchors.atlas.ripe.net} avec la structure suivante : \textit{pays-ville-ASN.anchors.atlas.ripe.net}.} \\ \hline
			\textbf{Crédit  gagnés  }              & $N$ & $10$ $*$ $N$ \\ \hline
			
			\textbf{Besoin en bande passante }                  & léger & important \\ \hline
			\textbf{Coût : gratuite }                   & oui &  non \tablefootnote{Le matériel est au frais de l'hébergeur.} \\ \hline
	\end{tabular}}
	\caption{Comparaison entre sondes et ancres RIPE Atlas}
	\label{tab:comparaison-sonde-ancre}
\end{table}




%Les ancres RIPE Atlas sont à la fois des sondes avec des fonctionnalisées étendues et avancées. Plus de capacités de mesure. Ces ancres fournissent des informations de valeurs sur la connectivité locale et régionale pour le réseau qui héberge ces sondes d'une part, et pour l'Internet d'autre part. Les problèmes régionaux de la connectivité peuvent être étudiés grâce à une ancre sans devoir passer du \textit{ping} ou du \textit{traceroute}. En plus des fonctionnalités avancées d'une ancre RIPE Atlas en comparaison avec une sonde RIPE Atlas, l'hébergement d'une ancre permet de gagner plus de crédits par rapport à une sonde (voire $10$ fois). Enfin, il est possible de recevoir une sonde RIPE Atlas pour l'héberger gratuitement, quant à une ancre, elle n'est pas gratuite.

%soekris net6501-70


\subsection{Les mesures intégrées : Built-in } \label{par:whatmesureripeatlas}

Une fois une sonde  Atlas connectée, elle lance automatiquement un ensemble de mesures prédéfinies, appelées \textit{Built-in Measurements}. Les mesures personnalisées sont détaillées dans la section \ref{par:udm}. Le choix du mode  IPv4,  IPv6 ou les deux, dépend de la capacité du réseau qui héberge la sonde  Atlas.  

Il existe deux types de mesures : les mesures \textit{One-Off}, ce sont les mesures qui s'exécutent une seule fois. Pour le deuxièmes type, ce sont les mesures qui s'exécutent   périodiquement, à chaque intervalle de temps. 

De base, les sondes Atlas assurent les mesures intégrées  suivantes: 

\begin{itemize}
	\item[--] Les informations sur la configuration du réseau dans lequel la sonde Atlas est déployée.
	\item[--] L'historique de la disponibilité de la sonde Atlas.
	\item[--] Les mesures du  RTT (Round Trip Time) par traceroute.
	\item[--] Les mesures ping vers un nombre de destinations prédéfinies.
	\item[--] Les mesures traceroute vers un nombre de destinations prédéfinies.
	\item[--] Les requêtes vers les instances des serveurs DNS (Domain Name System) racines.
	\item[--] Les requêtes SSL/TLS (Secure Socket Layer/Transport Layer Security) vers un nombre de destinations prédéfinies.
	\item[--] Les requêtes NTP (Network Time Protocol).
\end{itemize}

Chaque mesure a un identifiant ID unique. Cet identifiant indique le type de la mesure, s'il s'agit du ping, traceroute ou autres. Plus de détails sur la signification des identifiants des mesures sont disponibles dans la section  \ref{sec:builtin} dans l'annexe A.

En plus des mesures intégrées, les sondes Atlas peuvent effectuer des mesures personnalisées. Ces mesures peuvent être lancées via l'interface web \cite{create-UDM} ou bien via  HTTP REST API. Toutefois, la planification des  mesures personnalisées nécessite l'acquisition de ce qu'on appelle les "crédits" au sens RIPE Atlas.  

\subsection{Le système de crédits Atlas} \label{credits-atlas}

Le système de crédits RIPE Atlas est une sorte de reconnaissance de la contribution des participants à ce projet. Un hébergeur d'une sonde  Atlas reçoit un nombre de crédits en contrepartie de la durée pendant laquelle sa sonde reste connectée. D'autre part, il gagne d'autres crédits suivant les résultats de mesures générés par cette sonde. Les crédits gagnés peuvent être utilisés dans la création des mesures personnalisées, appelées  \textit{User Defined Measurements} (voir la section \ref{par:udm}). Les personnes ayant gagné des crédits peuvent les transférer vers une autre personne ayant besoin de ces crédits. Les crédits peuvent être obtenus via:
\begin{itemize}
	\item[--] L'hébergement d'une sonde Atlas; à chaque utilisation d'une sonde, son hébergeur reçoit un nombre de crédits.  La connexion d'une sonde  Atlas au système durant une minute apporte $15$ crédits.
	\item[--] L'hébergement d'une ancre  Atlas\footnote{Les ancres Atlas sont décrites dans  la section \ref{subsec:ancre}.}.
	\item[--] La recommandation à une personne d'héberger une sonde  Atlas.
	\item[--] En étant un sponsor du RIPE NCC. Le parrainage des sondes Atlas est disponible pour les organisations et les individus.  Le sponsor reçoit le même nombre de crédits que les hébergeurs de ces sondes.
	\item[--] En étant  un  registre Internet régional (Local Internet Registry).
	\item[--] La réception des crédits d'une autre personne via un transfert de crédits.
\end{itemize}

%Les crédits reçus sont utiles pour le lancement d'une des mesures citées dans la section \ref{par:whatmesureripeatlas}. 
Le lancement des mesures personnalisées exploite les ressources de l'infrastructure RIPE Atlas d'une part,  du réseau hébergeur de la sonde d'autre part. Par conséquent, les mesures sont organisées afin d'éviter toute surcharge du système. Le coût d'une  mesure dépend du type de la mesure et des options spécifiées. Le système calcule le nombre de crédits nécessaires pour effectuer une mesure donnée. Le nombre de crédits est déduit à chaque résultat reçu.  Ci-dessous le coût unitaire des différents types de mesures.

\begin{description}
	\item[Ping et ping6 ]  :
	\begin{tcolorbox}
		\begin{center}
			Coût unitaire = $N$ $\times$ ($\lfloor \frac{S}{1500} \rfloor $ $+$ $1)$
		\end{center}
	\end{tcolorbox}
	
	Où $N$ est le nombre de paquets dans le train (par défaut $3$) et $S$ est la taille du paquet (par défaut: $48$ octets).
	
	\item[DNS et DNS6 ] :
	
	\begin{tcolorbox}
		\begin{center}
			Coût unitaire pour UDP: $10$ crédits/résultat
			
			Coût unitaire pour TCP: $20$ crédits/résultat
		\end{center}
	\end{tcolorbox}
	
	
	\item[Traceroute et traceroute6 ] :
	
	\begin{tcolorbox}
		\begin{center}
			Coût unitaire = $10$ $\times$ $N$ $\times$ $(\lfloor \frac{S}{1500}\rfloor)$ $+$ $1)$
		\end{center}
	\end{tcolorbox}
	
	Où $N$ est le nombre de paquets dans le train (par défaut $3$) et $S$ est la taille du paquet (par défaut: $40$ octets).
	
	\item[SSLCert et SSLCert6 ] :
	
	\begin{tcolorbox}
		\begin{center}
			Coût unitaire = $10$ crédits/résultat.
		\end{center}
	\end{tcolorbox}	
	
\end{description}

\textbf{\textit{Exemple :}}

La planification d'une mesure ayant les caractéristiques suivantes nécessite $14,400$ crédits.

\begin{table}[H]
	\begin{tabular}{ l l }
		La fréquence &: deux fois par heure \\ 
		
		La durée &: deux jours ($48$ heures) \\
		
		Le nombre de sondes&: $5$\\
		
		Type de mesure &: \textit{traceroute}\\
	\end{tabular}
\end{table}

Tel que :

\begin{tcolorbox}
	\begin{center}
		$5$ $\times$ $2$ mesures/heure $\times$ $48$ = $480$ ligne résultat
		
		$30$ credits/result $\times$ $480$ results = $14,400$ crédits
	\end{center}
\end{tcolorbox}	

\subsection{Les mesures personnalisées : User Defined mesurement} \label{par:udm}

En plus des mesures intégrées, par défaut, dans une sonde Atlas, il est possible de planifier des mesures personnalisées. Ce sont les m\^{e}mes~  types de mesures: ping, traceroute,  HTTP Get, SSLCert , DNS, NTP et TLS. Cette planification coûte des crédits, en effet, il faut avoir assez de crédits pour lancer des mesures. L'interface web dédiée à la création d'une nouvelle mesure offre toute les possibilités comme la précision des éléments suivants :
\begin{itemize}
	\item[--] Le type de la mesure.
	\item[--] La sélection des sondes Atlas réalisant la mesure.
	\item[--] La fréquence de la mesure et sa durée.
\end{itemize}

Chaque mesure est suivie via son état. Plusieurs états à distinguer: \textit{specified}, \textit{scheduled}, \textit{ongoing}, \textit{stopped},  \textit{Forced to stop},  \textit{no suitale probes} et enfin  \textit{failed}.



\subsection{La sélection des sondes Atlas}

La sélection des sondes  Atlas pour effectuer une des mesures repose un des critères suivants~: 
\begin{itemize}
	\item[--] Numéro d'AS.
	\item[--] Zone géographique via  l'attitude et la longitude.
	\item[--] Pays (ou zone géographique comme  Europe).
	\item[--] Préfixe IP. % (en pratique, ne marche pas).
	\item[--] Manuellement, avec les identifiants des sondes Atlas.
	\item[--] Reprendre celles d'une  mesure précédente.
\end{itemize}

Il existe une autre manière de regrouper les sondes  avec des étiquettes.  Le système d'étiquettes sert comme indicateur des propriétés, des capacités, de la topologie du réseau ou d'autres classifications. On distingue les étiquettes système  et utilisateur. Chaque nom d'étiquette est lisible par un humain. 

Les étiquettes utilisateurs sont  associées à une sonde librement par son hébergeur. Les étiquettes système sont attribuées uniquement par l'équipe RIPE Atlas et sont mises à jour périodiquement, à priori chaque $4$ heures. Des exemples d'étiquettes système sont présentés dans la section \ref{sec:rags} de l'annexe A.


\subsection{Les sources de données Atlas} \label{subsec:sources-data}


Les sondes  Atlas génèrent trois types de données : leurs détails de connexions d'un jour donné, leurs résultats des mesures intégrées et personnalisées et les descriptions des mesures effectuées (meta-data). 

Premièrement on trouve les données sur les sondes  Atlas par jour. Les détails sur les sondes reprennent les informations des connexions, des réseaux et autres. A priori, les détails des connexions des sondes sont disponibles pour la période du  $13$ mars $2014$ jusqu'à ce jour\footnote{$15/08/2018$.}, un fichier JSON par jour (voir un exemple dans la section \ref{data-probes} de l'annexe A). Les données de certains jours sont manquantes.  La totalité des archives se trouve dans \cite{probes-data}. La taille d'une seule archive est entre $120 $ Ko et $921$ Ko. 

Deuxièmement, les résultats des mesures sont aussi archivés dans un serveur FTP. Seules les données des derniers $30$ jours  sont conservées en archives \footnote{Source : \url{https://data-store.ripe.net/datasets/atlas-daily-dumps/}, consultée le $05/04/2018$.}. Les fichiers ont été nommés jusqu'au  $ 15 $ mars $ 2018 $ de façon structurée comme suit: 

\begin{tcolorbox}
	\begin{center}
		\textit{\$TYPE-\$IPV-\$SUBTYPE-\$DATE.bz2}
	\end{center}
\end{tcolorbox}

\begin{itemize}
	\item[--] \$TYPE peut être {traceroute, ping, dns, ntp, http, sslcert}.
	\item[--] \$IPV version du protocole IP v4 ou v6.
	\item[--] \$DATE date au format  YEAR-MONTH-DAY. (etc. 2017-06-13)
	\item[--] \$SUBTYPE type de mesure builtin ou udm.
\end{itemize}

En considérant toutes les possibilités des types, la quantité de données générées quotidiennement est environ $25$ Go \footnote{Source :  \url{https://ftp.ripe.net/ripe/atlas/data/README}, consultée le $26/03/2018$.}    et la taille des  archives est entre $281$M et $3.2$G.

Depuis $15$ mars $2018$, les résultats des mesures ont été regroupés différemment. $24$ archives par jour,  une seule archive pour chaque heure et  type de mesure. L'archive ne distingue pas entre mesures IPv4 et IPv6,  entre mesures intégrées et  personnalisées. Il existe un attribut \textbf{"af"} qui distingue entre IPv4 et IPv6 et l'identifiant de la mesure pour distinguer les  mesures intégrées et celles personnalisées (identifiant > $1,000,000$). 

Streaming API propose un service de récupération des résultats de mesures en temps réel, depuis les sondes publiques. Ainsi, elle  fournit continuellement de nouveaux résultats en temps réel, obtenus par les sondes  Atlas publiques,  via une connexion de type HTTPS web-socket active tout le temps.

Troisièmement, on trouve des archives sauvegardées chaque semaine, elles décrivent les  méta-datas des mesures. Une ligne objet JSON pour chaque mesure publique. Au moment de la consultation, la taille de chaque archive était entre $124$ Mo et $1.5$ Go. L'accès à ce jeu de données se fait de deux façons, via le téléchargement direct depuis un serveur  FTP ou bien via streaming API. Les noms des archives sont bien structurés.

\subsection{Les versions du firmware des sondes Atlas} \label{subsec:firmwareversion}
En principe, toutes les sondes Atlas collectent la même information, indépendamment de leur version du firmware. On trouve les mêmes attributs\footnote{Attribut dans le sens du JSON : chaque résultat de mesure est enregistré comme étant un objet JSON.}  dans toutes les versions sauf de léger changements : ajout d'un ou de plusieurs attributs, la modification des noms des attributs, etc. Pour la simplification, nous donnons un identifiant entier pour chaque version, entre les parenthèses. Cet identifiant sera utilisé dans la suite de ce document. 

%Les résultats de mesures  sont sauvegardés dans des fichiers JSON. Chaque mesure est un tableau d'éléments. chaque élément est un tableau associative. La structure des tableau associatifs dépend de la version du firmware. \par


Il existe plusieurs versions du firmware:
\begin{itemize}
	\setlength\itemsep{0.1 cm}
	\item[--] La version $1$ est identifié par  $1$ (1). 
	\item[--] La version $4400$  est identifiée par une valeur entre  $4400$ et $4459$ (2).
	\item[--]  La version $4460$ est identifiée par une valeur entre $4460$ et $4539$ (3).
	\item[--]  La version $4540$  est identifiée par une valeur entre  $4540$ et $4569$ (4).
	\item[--]  La version $4570$  est identifiée par une valeur entre $4570$ et $4609$ (5).
	\item[--] La dernière version du firmeware \footnote{A la date de consultation $ 25/01/2018 $.} est $4610$ (6). 
\end{itemize}

\subsection{Les limitations du RIPE Atlas}


De nombreux travaux ayant exploité les données générées par les sondes Atlas. Néanmoins, ce système connaît des bugs et des limitations. Les membres de la communauté RIPE Atlas s'engagent à remonter les bogues liées aux sondes Atlas. Tous les bogues sont répertoriés sous une rubrique dédiée  \cite{bugs-ripe-atlas}.

RIPE Atlas connaît des limitations  liées à la visualisation. Actuellement, RIPE Atlas supporte la visualisation des mesures de type ping ayant utilisé au maximum $20$  sondes. Cette limitation concerne aussi le type traceroute, en effet,  il est possible de visualiser seulement les mesures IPv6 built-in.

Afin d'éviter la surcharge  des sondes et de l'infrastructure, RIPE Atlas a limité le nombre de mesures périodiques de $10$ à la fois et de $10$ mesures de type one-off vers n'importe quelle cible à un moment donné. De plus, il n'est pas possible d'utiliser  plus de $500$ sondes par mesure.

Pour les mesures one-off (non périodiques), une sonde peut effectuer au plus $10$ mesures en parallèle. RIPE Atlas limite aussi la fréquence des mesures personnalisées. Un hébergeur d'une sonde peut effectuer:

\begin{itemize}
	\item[--] Ping chaque $60$ secondes (par défaut  $240$ secondes).
	\item[--] Traceroute chaque $60$ secondes (par défaut  $900$ secondes).
	\item[--] SSL chaque $60$ secondes (par défaut  $900$ secondes).
	\item[--] DNS chaque $60$ secondes (par défaut $240$ secondes).
	
\end{itemize}

Dans le cas d'une déconnexion, la sonde continue à effectuer les mesures. Pour la version $1$ et $2$, la sonde est capable de sauvegarder les  $6$ dernières heures de données. Tandis qu'avec  la  version $3$, une sonde est capable de sauvegarder les résultats de plusieurs mois. Une fois la sonde est connectée, elle envoie les  données à l'infrastructure centrale.

Concernant la consommation des crédits, RIPE Atlas limite cette consommation à  $1,000,000$ crédits par jour.


\subsection{Confiance aux données Atlas}

De nombreux travaux ont exploité les données Atlas, cependant, peut-on faire confiance à la qualité des données? les données sont-elles complètes?

La question de la complétude des données est plus présente pour les mesures périodiques, celles qui se déroulent pendant une durée $d$ et à un intervalle $i$. W. Shao et al.  \cite{DBLP:journals/corr/ShaoRDV17} ont traité les mesures manquantes.  L'approche qu'ils ont adopté repose sur la corrélation entre  l'absence de certaines mesures et les périodes durant lesquelles les sondes Atlas sont déconnectées. Pour précision, RIPE Atlas maintient les détails des connexions/déconnexions des sondes Atlas. Ils ont étudié les mesures en provenance des  sondes v$3$, effectuées entre le $01/06/2016/$ et  le $01/07/2016/$ (UTC). Ils ont combiné les informations relatives à la connexion/déconnexion des sondes et leurs mesures planifiées, leur approche se base sur l'attribut $timestamp$ qui est présent dans chaque résultat de mesure et dans les états de connexions.


Nous avons discuté des limitations du RIPE Atlas en terme de mesures autorisées par jour. Cela n'empêche qu'il est possible qu'un nombre important de mesures  soit effectué. De plus, plus d'un utilisateur peut s'intéresser à la même sonde Atlas. C'est la question traitée dans le travail de T. Holterbach et al. dans \cite{Holterbach:2015:QIM:2815675.2815710}, si les mesures lancées par les autres utilisateurs affectent les résultats obtenus par un autre utilisateur, si c'est le cas, comment s'y entreprendre. Les expériences réalisées ont montré la présence de l'interférence entre les mesures à destination des sondes et cela de deux manières. Premièrement, les mesures depuis et à destination des sondes Atlas augmentent le temps reporté par la sonde et ils ont conclu que l'amélioration du CPU a permis de limiter les interférences sur le temps mesuré par les sondes Atlas. Deuxièmement,  ils ont conclu que les mesures perdent la synchronisation avec l'infrastructure d'Atlas, pendant plus d'une heure, à cause de la charge concurrentielle que subit le système d'Atlas. Dans ce cas, l'amélioration du matériel ne peut pas résoudre le problème.

%http://pages.cs.wisc.edu/~pb/hotplanet13a\_final.pdf
%http://pages.cs.wisc.edu/~pb/hotplanet13a_final.pdf


\section{Projets existants de mesures d'Internet}

Dans les sections précédentes, on a développé  le projet RIPE Atlas comme étant une plateforme pour la collecte des données des réseaux.  Toutefois, il existe d'autres projets similaires à RIPE Atlas. Les sections suivantes reprennent une liste non exhaustive des projets similaires  à RIPE Atlas.

\subsection{Test Traffic Measurement Service}

Avant l'arrivée du RIPE Atlas, Le RIPE NCC (Réseaux IP Européens Network Cordination Centre)   a assuré la mesure de la connectivité entre les réseaux via  d'autres plateformes, comme la plateforme Test Traffic Measurement Service (TTM). Il s'agit d'un projet qui permet de mesurer la connectivité entre un n\oe{}ud  source  et  un n\oe{}ud destination sur Internet.  C'était une des manières  pour  suivre la connectivité entre le réseau source et le réseau destination. 

L'idée était la mise en place d'un dispositif, test-box,  qui génère du trafic. Ce dernier n'affecte pas l'infrastructure réseau en matière de bande passante. De plus, il n'a pas l'accès aux données du réseau dans lequel il est mis en place.

Ce service a été assuré et géré, pendant une période de $6$ ans, par une équipe au sein du RIPE NCC. Les fonctionnalités assurées par ce service étaient de tester l'accessibilité à une destination via le \textit{ping}, ainsi, les mesures effectuées étaient indépendantes des applications, elles dépendaient du  réseau lui-même. RIPE NCC   a  arrêté la maintenance du  TTM depuis le $1$ juillet $2014$ \cite{TTM}.


\subsection{ProbeAPI}

\textit{ProbeAPI} \cite{PROBEAPI} est  une plateforme de mesure d'état  du réseau, cette plateforme couvre $170$ pays et des milliers d'ISPs. \textit{ProbeAPI} est utilisée par les développeurs, les administrateurs des réseaux et les chercheurs, ils peuvent lancer des mesures d'un réseau depuis différents réseaux.

Le logiciel \textit{ProbeAPI}  s'exécute dans plusieurs systèmes: dans des ordinateurs (Win$32$/$64$), Android via une installation dans les mobiles et les tablettes et   dans des routeurs au sein du DD-WRT.

\begin{tcolorbox}
	\textbf{DD-WRT} est un micrologiciel libre et gratuit, il est destiné aux routeurs sans fil et aux points d'accès. Il fonctionne  avec un système d'exploitation Linux. Le rôle du DD-WRT est de remplacer le micrologiciel intégré aux routeurs par leurs fabricants. Ainsi, il est possible  d'étendre des fonctionnalités du routeur en ajoutant d'autres fonctions supplémentaires.
\end{tcolorbox}


ProbeAPI s'agit d'un logiciel qui tourne dans la machine de l'hébergeur. En conséquence, le suivi des réseaux dépend de la disponibilité de la machine qui le fait  tourner. Cette dépendance  affecte la disponibilité de la sonde logicielle, sa configuration et aussi les résultats de mesures.

Une étude comparative \cite{COMPARE-ATLAS-PROBEAPI}  entre les sondes  Atlas et les sondes \textit{ProbeAPI} est résumée dans le tableau \ref{tab:compare-ripeatlas-probapi}. 
%Ce dernier reprend une comparaison entre les deux plateformes. 
En fin de cette étude, ils concluent qu'en comparant les résultats des mesures ICMP effectuées par les deux plateformes, des contrastes intéressantes ont été constatées.  Les sondes Atlas ont montré un comportement stable lors de la réalisation  des mesures, les résultats sont peu variables car les sondes sont indépendantes de l'utilisateur. Cependant, il était constaté qu'une forte variabilité au cours du temps pour les sondes logicielles (ProbeAPI), car elles dépendent fortement de l'hébergeur; sa configuration réseau, sa disponibilité, etc.

Enfin, la force des sondes logicielles comme  ProbeAPI réside dans sa capacité  à effectuer des mesures depuis la couche application, la plus proche de l'utilisateur. L'exemple de l'évaluation du Time To First Byte et le taux de transfert dans deux pays.

\begin{tcolorbox}
	\og \textit{	Le \textbf{Time to First Byte} (TTFB) est le temps de chargement du premier octet, c'est la mesure qui nous permet d'évaluer la vitesse d'accès à un serveur. Plus la mesure est basse et plus le serveur commencera à servir les ressources rapidement.} \fg{} \footnote{Source : \url{https://www.skyminds.net/calculer-le-time-to-first-byte-ttfb-dun-serveur/}, consultée le $10/08/2018$.}
\end{tcolorbox}

\begin{table}[H]
	\centering
	\begin{adjustbox}{max width=\textwidth}
		\begin{tabularx}{\textwidth}{|X|X|}
			\hline
			\thead{RIPE Atlas}& \thead{ProbeAPI} \\ \hline
			Matériel homogène a un comportement  prévisible&  Matériel hétérogène a un  comportement imprévisible \\ \hline
			Connexions stables vu l'indépendance du software utilisateur& Connexions instables vu la dépendance du software utilisateur\\ \hline
			Indépendance  de l'OS et ses limitations ou vulnérabilités&  Liaison à l'OS et ses limitations ou vulnérabilités, cependant utile pour les mesures au niveau application \\ \hline
			La distribution des sondes est coûteuse, difficile de couvrir certaines régions&  Mise en place du logiciel est rapide et moins chère, avec facilité de couvrir plusieurs régions\\ \hline
			Les mesures HTTP se limitent aux ancres	pour des raisons de sécurité& HttpGet, DNS et page-load sont disponibles via des librairies Mozilla et chromium, et ce pour toutes les destinations \\ \hline
			
		\end{tabularx}
	\end{adjustbox}
	\caption{Comparaison entre sondes  Atlas et ProbeAPI}
	\label{tab:compare-ripeatlas-probapi}
\end{table}

Malgré le niveau de couverture assuré par ProbeAPI, cependant ces sondes se connectent et se déconnectent fréquemment, ce qui montre une forte volatilité. Cette volatilité est liée à la dépendance des sondes ProbeAPI de leur hébergeur; tant qu'il est connecté, la sonde ProbeAPI est prête pour effectuer les mesures. Toutefois, si l'hébergeur est déconnecté, la sonde ProbeAPI ne peut pas effectuer des mesures, d'où le basculement fréquent entre les deux états : connectée et déconnectée. 


\subsection{Archipelago}

Archipelago (Ark) \cite{Archipelago} est l'infrastructure de mesures actives du CAIDA \cite{CAIDA}. Elle est au service des chercheurs en réseau depuis $2007$. L'objectif de ce projet est de couvrir un maximum de régions afin de collecter un maximum de mesures. Ensuite, produire des visualisations qui améliorent la vue globale de l'Internet. Pour précision, c'est un Raspberry Pi 2nd gen.


\subsection{DIMES}
DIMES \cite{Shavitt:2005:DLI:1096536.1096546} est un logiciel qui devrait être installé dans une machine. Une fois installé, il fonctionne de sorte que la consommation d'énergie soit minimale et qu'il n'existe aucun impact sur les performances de la machine ou sur la connexion. L'objectif de \textit{DIMES} est de collecter un maximum de données afin d'explorer la topologie d'Internet.


\subsection{SamKnows}
SamKnows \cite{SamKnows}  est une plateforme globale des performances d'Internet, elle regroupe les ISPs, ingénieurs, universitaires, codeurs et des organismes de régulation. Son objectif est d'évaluer les performances du haut débit des utilisateurs finaux et de trouver les problèmes avant que les clients ne commencent à se plaindre. 

\section{Quelques cas d'utilisation des données collectées par les sondes Atlas} \label{use-cases-atlas}


Plusieurs travaux ont exploités les données collectées par les sondes Atlas. Ces travaux peuvent être classés de plusieurs manières, par exemple  par thème,  par  type de mesures utilisé, etc. Nous distinguons les travaux ayant exploité les données collectées par les sondes Atlas à travers les mesures \textit{built-in} ou bien ceux ayant utilisé les données des mesures personnalisées.  Pour les premiers, ils permettent d'exploiter au mieux ces données sans surcharger le réseau des sondes Atlas, car ces données sont collectées quotidiennement. Cependant, les autres peuvent introduire une charge sur ces sondes. D'autre part, certains auteurs se sont intéressés  aux données traceroute, d'autres aux données ping ou HTTP, etc. Nous allons présenter brièvement quelques travaux par thème.

\subsection{Détection des coupures d'Internet}

Les données collectées par les sondes Atlas ont permis de valider certaines coupures d'Internet, par exemple la coupure concernant le point d'échange AMS-IX (Amsterdam Internet Exchange). En $2015$, Robert Kisteleki  et al. \cite{Robert-Kisteleki}  ont évalué l'état des pings en provenance des sondes Atlas à destination de trois ancres Atlas qui se trouvent dans AMS-IX. En effet, peu de pings ont réussi d'atteindre leurs destinations, cependant, certains pings n'ont pas réussi à le faire. Ils ont conclu qu'il existe un problème du réseau, et le problème concerne  les ancres plutôt que les sondes ayant lancé le ping. De même pour DNS, ils ont constaté l'absence des données DNS sensées être collectées par les ancres Atlas à destination du K-root.






\subsection{Aide à la prise de décision}
L'utilisation des sondes Atlas n'est pas limitée au domaine de recherche seulement, elle a permis aussi d'aider à la prise de décision pour certaines implantations et pour la mise en place des équipements comme les routeurs, les data-centers, les IXPs, etc. 

Les ingénieurs de \textit{Wikimedia Foundation} et du RIPE NCC ont collaboré dans un projet \cite{Wikipedia} pour étudier la  latence vers les sites  du Wikimedia. L'idée était d'exploiter la distribution des sondes  Atlas dans le monde en vue  de mesurer la latence vers les sites du Wikimedia. L'étude de la latence va permettre d'améliorer l'expérience des utilisateurs vers ces sites  en réduisant la latence. Comme Wikimedia avait l'intention d'étendre son réseau de datacenters, ils ont profité des résultats de cette étude pour choisir les futurs emplacements de leurs data-centers.


Un groupe de chercheurs africains a évalué le routage inter-domaine afin d'étudier les emplacements adéquats pour la mise en place d'un IXP \cite{FANOU-Roderick}. Après avoir analysé les données des mesures collectées par les sondes Atlas, ils ont constaté que le trafic de et à destination de l'Afrique quitte le continent vers les États-Unis ou bien  l'Europe pour revenir en Afrique, d'où l'intérêt d'investir dans la mise en place des IXPs dans ce continent.  \par


\subsection{Le suivi des censures}

En $2014$, des chercheurs ont examiné les incidents de type content-blocking en Turquie et en  Russie tout en prenant en considération le respect de l'aspect éthique des données. Ils ont aussi élaboré  un aperçu comparatif des différents outils permettant de mesurer les réseaux \cite{Collin-Anderson}. C. Anderson et al. ont repris deux cas d'études où une censure a été appliquée : la Turquie et la Russie. L'idée de C. Anderson et al. est de créer des méthodes pour analyser ces censures en se basant sur les données collectées par les sondes   Atlas. \par

Il existe plusieurs pratiques pour appliquer la censure. Ces pratiques dépendent des objectifs de cette censure; bloquer un site web, rediriger le trafic, filtrer l'accès à travers des mots clés, etc.

En mars $2014$, des utilisateurs turcs ont été interdits d'accéder au réseau social  \textit{Twitter}.  Ce filtrage a été fait en utilisant \textit{DNS Tampering} et \textit{IP Blocking}. Comme ces deux pratiques sont évaluables avec les sondes  Atlas, ils ont planifié des mesures vers plusieurs destinations et depuis un nombre de  sondes. Ces mesures sont  reprises en détail dans le tableau  \ref{ta:censorship-colin}.

\begin{table}[H]
	\centering
	\captionsetup{justification=centering}
	\begin{tabular}{ c c c c c}
		\textbf{Cible} &\textbf{Type} &	\textbf{Sondes} &\textbf{Fréquence (s)}	& \textbf{Crédits} \\ \hline
		Twitter &SSL &$ 10 $ &$ 3,600 $ &$ 2,400 $\\ \hline
		YouTube &SSL &$ 10 $ &$ 3,600 $ &$ 2,400 $ \\ \hline
		Tor & SSL &$ 10 $ &$ 3,600 $ &$ 2,400 $ \\ \hline
		Twitter & DNS (U) &$ 10 $ &$ 3,600 $ &$ 2,400 $ \\ \hline
		YouTube & DNS (U) &$ 10 $ &$ 3,600 $ &$ 2,400 $ \\ \hline
		Twitter &Tracert &$ 10 $ &$ 3,600 $ & $ 7,200 $ \\ \hline
	\end{tabular}
	\caption{Les détails des mesures effectuées dans le travail de C. Anderson \cite{Collin-Anderson} }
	\label{ta:censorship-colin}
\end{table}

L'analyse de données obtenues a permis de  détecter  six changements concernant les décisions du filtrage. Plus de détails se trouvent dans \cite{Collin-Anderson}.

Quant à la Russie, les autorités ont décidé de mettre le blog d'\textit{Alexei Navalny} sur  \textit{LiveJournal}  dans la liste noire. En même temps, certains médias indépendants ont été aussi filtrés, l'exemple du site grani.ru.  Pour le site  \textit{Grani}, les sondes Atlas ont reçu des réponses DNS aberrantes, d'où l'impossibilité de joindre   \textit{grani.ru}. Cependant, le filtrage du site \textit{navalny.livejournal.com}  a pris une autre forme, c'était une redirection d'adresse IP. La réponse d'une requête vers ce site donne $208.93.0.190$  au lieu de $ 208.93.0.150$. Ces deux adresses sont inclut dans le préfixe $208.93.0.0/22$ géré par \textit{LiveJournal Inc}. 
$208.93.0.190$  correspond au contenu  non-blacklisted, alors que  $ 208.93.0.150$ correspond au contenu correct.


\subsection{Le suivi des performances d'un réseau}


\paragraph{Les ancres  Atlas}~

Les ancres  Atlas ont des capacités avancées que les sondes  Atlas. Les ancres servent comme cibles aux mesures des sondes. De plus, elles sont capables de fournir des détails sur l'état du réseau dans lequel elles sont déployées.  S. Gasmi, un hébergeur d'une ancre  Atlas, a développé un outil  disponible au public \footnote{Source : \url{http://ripeanchor.sdv.fr/}, consultée le $08/08/2018$.}. 
A partir des données collectées par les ancres Atlas, cet outil permet d'analyser  la qualité de la connectivité d'un réseau (ou d'un AS) et permet de suivre les changements relatifs à la topologie des réseaux

Par exemple, il a constaté que la vérification du BGP Prepending et des communautés BGP peut être faite en  considérant les éléments suivants: adresse IP source, AS source, pays, le RTT du ping, les chemins du traceroute. En particulier, S. Gasmi a évalué deux corrélations. Dans un premier temps, il a visualisé la corrélation entre l'AS path et Round Trip Time (RTT). Il a regroupé des sondes par pays, ensuite, il a calculé, par ce pays, la moyenne du nombre de sauts et la moyenne du RTT des requêtes à destination de l'ancre depuis ces sondes Atlas.
La figure \ref{fig:1-AS-Path-Time-correlationv} reprend les résultats obtenus. Aucun renseignement sur la période des données. Pour les sondes en provenance de la France,  le nombre de sauts et le RTT entre les sondes déployées en France sont faibles car l'ancre (la cible) se trouve aussi en France.
\begin{figure}[H]
	\centering
	\captionsetup{justification=centering}
	\includegraphics[width=1\linewidth]{illustrations/1-AS-Path-Time-correlation}
	\caption{La corrélation entre la moyenne des AS paths et la moyenne des RTTs \cite{Salim-Gasmi}}
	\label{fig:1-AS-Path-Time-correlationv}
\end{figure}


Ensuite, S. Gasmi a mesuré le RTT entre des sondes Atlas dans le monde et son ancre, il a aussi visualisé le nombre de sauts parcourus entre des sondes Atlas à travers le monde et son ancre. Ces deux visualisations permettent d'avoir une idée sur la latence entre certains pays et le pays de l'ancre en question. Plus de détails sur l'approche sont disponibles dans \cite{Salim-Gasmi}.


\paragraph{La vérification de la cohérence du Traceroute}~

Les chemins parcourus par traceroute pour aller d'une source $s$ vers une destination $d$ changent au cours du temps pour plusieurs raisons. Par exemple, suite   à un changement BGP, à une  répartition des charges, à des pannes des routeurs,  à des pannes des liens physiques, etc.

\textit{Traceroute Consistency Check} peut reprendre les chemins obtenus via traceroute au cours du temps . L'objectif est de suivre les  n\oe{}uds apparaissant dans le chemin allant de  $s$ à $d$ aux instants $t$, $t+1$, $t+2$, etc, et cela afin de voir les n\oe{}uds traversés plus fréquemment au cours du temps. Le chemin est mis à jour via Atlas streaming API. 

L'outil proposé dessine les chemins traceroute comme étant un graphe dirigé, chaque n\oe{}ud est coloré suivant sa cohérence. Le code source du projet est disponible sur GitHub \cite{Traceroute-consistency-check}. La figure \ref{fig:Traceroute-consistency-check} présente un exemple de la visualisation proposée. Ce résultat concerne la mesure $1663314$ \footnote{Source : \url{https://atlas.ripe.net/measurements/1663314/}, consultée le $05/08/2018$.}. Ce sont des traceroutes à destination de l'adresse $213.171.160.1$ entre $02/05/2014$ $13:00$ et $03/05/2014$ $15:00$.

\begin{figure}[H]
	\centering
	\includegraphics[width=1\linewidth]{illustrations/traceroute-consitance.png}
	\caption{Visualisation des changements des chemins traceroute \cite{Traceroute-consistency-check}}
	\label{fig:Traceroute-consistency-check}
\end{figure}

\paragraph{BGP+traceroute} ~

C'est une combinaison des données BGP (RIPE RIS) et traceroute (RIPE Atlas). L'objectif de ce projet était de partir d'un AS path pour enfin géolocaliser les ASs. 
L'idée est de prendre un AS path des données RIPE RIS, puis, récupérer le préfixe (bloc d'adresses IP) annoncé via cet AS path, ensuite, lancer un traceroute vers une des adresses du bloc. Enfin, géolocaliser les ASs via les données du traceroute. 
Le code source et la présentation de ce projet sont disponibles GitHub \cite{bgp-traceroutes,pres-bgp-traceroute}. 


\paragraph{BGP Atlas Monitor "BAM!"} ~

Le projet \textit{BAM} vise la visualisation , en temps réel,  des informations utiles pour les opérateurs des réseaux. Par exemple, BAM montre  la visibilité des préfixes obtenus par RIPE RIS. De plus,  il est possible de voir  le délai du \textit{ping} obtenu via les sondes   Atlas. Le code source est disponible sur  GitHub \cite{bam}. En fournissant un ASN (identifiant d'un AS), \textit{BAM} récupère les préfixes IPv4 et IPv6 et leur visibilité et il montre aussi les sondes dans cet AS. L'outil offre  les  fonctionnalités suivantes:
\begin{itemize}
	\item[--] Les préfixes annoncés par un ASN.
	\item[--] La visibilité d'un ASN.
	\item[--] La visibilité d'un préfixe.
	\item[--] La liste des sondes par AS.
	\item[--] Les objets  route des préfixes.
\end{itemize}

\paragraph{Prédiction des routeurs provoquant la perte des paquets }~

Dans l'étude \cite{DBLP:journals/corr/FontugneAPB16}, Romain Fontugne et al. ont modélisé le comportement  des routeurs, ils ont développé un modèle qui permet d'estimer l'endroit de la  perte des paquets. A partir des traceroutes passant par un routeur \textit{r} à une destination \textit{d}, ils ont construit un modèle de forwarding pour ce routeur. Ce modèle reprend les prochains sauts (routeurs) et la fréquence de passage par ces derniers. Si le routeur \textit{r} change le prochain saut qui a eu "l'habitude" de traverser  pour atteindre \textit{d}, alors, il est possible d'estimer l'origine  de la perte de paquets.

\subsection{Le suivi des détours dans un trafic local}

Dans leur travail \cite{Emile-Aben-IXP-countries}, E. Aben et al. avaient l'objectif de  voir comment les mesures du RIPE Atlas peuvent fournir un aperçu sur le chemin du trafic local à un pays. Précisément si ce trafic traverse un autre pays en revenant au pays du départ.  Ce qui pourrait  aider à améliorer les performances et l'efficacité des IXPs.  L'objectif est  d'analyser les chemins identifiés dans le trafic d'Internet entre les sondes  Atlas dans un pays donné et essayer d'identifier si le trafic  traverse les IXPs.


France-IX est un point d'échange Internet (IXP) français créé en juin 2010. Afin d'apprendre la topologie de routage, un RIS route collector (RRC21) a été installé au sein du France-IX. Actuellement, la France compte $755$ sondes  Atlas et $9$ ancres. Une ancre sur les $9$ est installée au sein de France-IX.

Une des questions posées c'était si le trafic local de la France reste local, les sondes  Atlas ne permettent pas de mesurer le trafic entre deux points, cependant, elles permettent de calculer le chemin entre deux points, adresses IP, ce qui permet d'inférer les sauts par lesquels  le trafic passe le trafic. Le travail \cite{France-IX} s'intéresse au trafic depuis et vers une sonde en France en se basant sur l'étude dans \cite{Emile-Aben-IXP-countries}.


Les résultats obtenus de l'analyse des détours peuvent être intéressants pour les opérateurs des réseaux afin d'améliorer leurs services, ainsi intéressants pour les IXPs tels qu'ils peuvent  proposer des services de peering dans les endroits où il le faut.


\subsection{Visualisation : indicateurs et dashboard}

L'objectif de certains travaux était d'exploiter les données collectées par les sondes Atlas pour concevoir des tableaux des indicateurs. Par exemple, à partir des données de connexion/déconnexion des sondes Atlas, visualiser les sondes connectées, déconnectées, abandonnées. Un autre projet avait comme objectif la reconstruction d'un graphe reprenant les routeurs (n\oe{}uds) impliqués dans certains traceroutes, ainsi, identifier les n\oe{}uds les plus traversés. D'autres travaux ont repris les détails de la latence, essentiellement, sont les valeurs des RTT dans les pings et les traceroutes qui permettent de visualiser ce type d'information. 


La liste des travaux basés sur le projet RIPE Atlas est très longue. Nous avons essayé d'énumérer quelques projets, les classer par thèmes, toutefois, ce n'est pas un classement unique, tel qu'on peut retrouver un travail dans plus d'une catégorie, ou bien les classer par un autre classement.

\section{Conclusion}

Dans ce chapitre, nous avons présenté les sondes Atlas et leur fonctionnement, ainsi que quelques travaux qui ont impliqué les données collectées par ces sondes dans plusieurs domaines tels que la prise de décision, le suivi des censures, la conception des tableaux de visualisation, etc. Ces données sont cruciales pour mener à toute analyse. Cependant, ces données sont massives, elles sont  dans l'ordre d'une dizaine de Go pour une heure de mesures du type traceroute par exemple, y incluent toutes les destinations, d'où la nécessité d'impliquer des outils
du Big Data pour une meilleure extraction d'informations utiles. En effet, le chapitre 2 aborde le sujet du Big Data dans ses différentes dimensions. 

%Dans  ce chapitre, nous avons découvert le fonctionnement des sondes Atlas  ainsi que les travaux ayant impliqué les données collectées par ces sondes.  Nous avons appris aussi la nature des données que les sondes Atlas collectent, c'est  crucial pour mener à toute analyse. Vu l'ordre de grandeur de la  quantité de données collectées, les outils traditionnels ne peuvent pas traiter ces données de façon efficace.


%en particulier, à   l'analyse prévue dans le chapitre \ref{chap:implementation}. 


%
\chapter{La détection des anomalies dans les délais d'un lien}

\section{Introduction}

Dans le présent chapitre, nous allons présenter l'outil de détection des anomalies dans les délais d'un lien  conçu dans le cadre du travail de R. Fontugne et al \cite{DBLP:journals/corr/FontugneAPB16}. Nous avons choisi ce travail qui exploite des données massives en vue d'évaluer quelques technologies du Big Data.

\section{Présentation du travail de référence}

\paragraph{Une vue globale du travail de référence}~ 

Le travail de R. Fontugne \cite{DBLP:journals/corr/FontugneAPB16} et al exploite une des mesures effectuées par les sondes Atlas, c'est la requête traceroute. L'idée de ce travail est de collecter les résultats des requêtes traceroutes effectuées par les sondes Atlas, ensuite comparer la référence avec la valeur courante. Cette  référence pour le délai  d'un lien donné   sera mise à jour au fur et à mesure de l'analyse.

\section{Présentation de l'outil de la détection des anomalies }

Les entrées de l'algorithme de la détection est un ensemble de traceroutes. Un traceroute (Traceroute) est un ensemble de sauts (Hop) en plus les informations générales concernant la sonde ayant effectué la requête traceroute et la destination de la requête. Chaque saut est décrit par un ensemble de signaux (Signal), où un signal, dans le présent contexte, décrit le routeur ayant émet une réponse à la sonde Atlas parmi les routeurs traversés avant d'atteindre la destination finale.  Pour le saut $i$, on note trois signaux dont le routeur émettant le signal est $fromij$ avec un RTT égal à $rttij$, avec $j \in [1,3]$ comme illustré dans la figure \ref{fig:traceroute}.

\begin{figure}[H]
	\centering
	\resizebox{\textwidth}{!}{
	% Graphic for TeX using PGF
% Title: /home/bellafkih/Documents/2018-2019/memoire/rapport_memoire/dia/traceroute.dia
% Creator: Dia v0.97.3
% CreationDate: Sun Sep 23 22:12:05 2018
% For: bellafkih
% \usepackage{tikz}
% The following commands are not supported in PSTricks at present
% We define them conditionally, so when they are implemented,
% this pgf file will use them.
\ifx\du\undefined
  \newlength{\du}
\fi
\setlength{\du}{15\unitlength}
\begin{tikzpicture}
\pgftransformxscale{1.000000}
\pgftransformyscale{-1.000000}
\definecolor{dialinecolor}{rgb}{0.000000, 0.000000, 0.000000}
\pgfsetstrokecolor{dialinecolor}
\definecolor{dialinecolor}{rgb}{1.000000, 1.000000, 1.000000}
\pgfsetfillcolor{dialinecolor}
\definecolor{dialinecolor}{rgb}{0.847059, 0.898039, 0.898039}
\pgfsetfillcolor{dialinecolor}
\fill (6.953088\du,8.550000\du)--(8.629264\du,8.550000\du)--(7.974118\du,10.350000\du)--(6.297942\du,10.350000\du)--cycle;
\pgfsetlinewidth{0.000000\du}
\pgfsetdash{}{0pt}
\pgfsetdash{}{0pt}
\pgfsetmiterjoin
\definecolor{dialinecolor}{rgb}{0.000000, 0.000000, 0.000000}
\pgfsetstrokecolor{dialinecolor}
\draw (6.953088\du,8.550000\du)--(8.629264\du,8.550000\du)--(7.974118\du,10.350000\du)--(6.297942\du,10.350000\du)--cycle;
% setfont left to latex
\definecolor{dialinecolor}{rgb}{0.000000, 0.000000, 0.000000}
\pgfsetstrokecolor{dialinecolor}
\node at (7.463603\du,9.645000\du){};
\pgfsetlinewidth{0.000000\du}
\pgfsetdash{}{0pt}
\pgfsetdash{}{0pt}
\pgfsetbuttcap
\pgfsetmiterjoin
\pgfsetlinewidth{0.000000\du}
\pgfsetbuttcap
\pgfsetmiterjoin
\pgfsetdash{}{0pt}
\definecolor{dialinecolor}{rgb}{0.027451, 0.486275, 0.682353}
\pgfsetfillcolor{dialinecolor}
\pgfpathmoveto{\pgfpoint{18.279993\du}{10.312559\du}}
\pgfpathlineto{\pgfpoint{18.278532\du}{10.341767\du}}
\pgfpathlineto{\pgfpoint{18.271230\du}{10.371705\du}}
\pgfpathlineto{\pgfpoint{18.261008\du}{10.400183\du}}
\pgfpathlineto{\pgfpoint{18.246404\du}{10.428295\du}}
\pgfpathlineto{\pgfpoint{18.227054\du}{10.456407\du}}
\pgfpathlineto{\pgfpoint{18.205148\du}{10.483790\du}}
\pgfpathlineto{\pgfpoint{18.178496\du}{10.510807\du}}
\pgfpathlineto{\pgfpoint{18.147828\du}{10.537094\du}}
\pgfpathlineto{\pgfpoint{18.114969\du}{10.562286\du}}
\pgfpathlineto{\pgfpoint{18.077729\du}{10.587477\du}}
\pgfpathlineto{\pgfpoint{18.036838\du}{10.611574\du}}
\pgfpathlineto{\pgfpoint{17.993027\du}{10.634940\du}}
\pgfpathlineto{\pgfpoint{17.946659\du}{10.657576\du}}
\pgfpathlineto{\pgfpoint{17.896276\du}{10.679482\du}}
\pgfpathlineto{\pgfpoint{17.843337\du}{10.700292\du}}
\pgfpathlineto{\pgfpoint{17.787842\du}{10.720372\du}}
\pgfpathlineto{\pgfpoint{17.729792\du}{10.739723\du}}
\pgfpathlineto{\pgfpoint{17.669186\du}{10.757612\du}}
\pgfpathlineto{\pgfpoint{17.605294\du}{10.774772\du}}
\pgfpathlineto{\pgfpoint{17.540307\du}{10.791201\du}}
\pgfpathlineto{\pgfpoint{17.471668\du}{10.806170\du}}
\pgfpathlineto{\pgfpoint{17.400840\du}{10.819679\du}}
\pgfpathlineto{\pgfpoint{17.328916\du}{10.832457\du}}
\pgfpathlineto{\pgfpoint{17.254071\du}{10.844505\du}}
\pgfpathlineto{\pgfpoint{17.178131\du}{10.854363\du}}
\pgfpathlineto{\pgfpoint{17.099635\du}{10.863490\du}}
\pgfpathlineto{\pgfpoint{17.020044\du}{10.871157\du}}
\pgfpathlineto{\pgfpoint{16.938992\du}{10.877729\du}}
\pgfpathlineto{\pgfpoint{16.855750\du}{10.882840\du}}
\pgfpathlineto{\pgfpoint{16.772143\du}{10.886491\du}}
\pgfpathlineto{\pgfpoint{16.686710\du}{10.888682\du}}
\pgfpathlineto{\pgfpoint{16.600183\du}{10.889412\du}}
\pgfpathlineto{\pgfpoint{16.514020\du}{10.888682\du}}
\pgfpathlineto{\pgfpoint{16.428222\du}{10.886491\du}}
\pgfpathlineto{\pgfpoint{16.344615\du}{10.882840\du}}
\pgfpathlineto{\pgfpoint{16.261738\du}{10.877729\du}}
\pgfpathlineto{\pgfpoint{16.180321\du}{10.871157\du}}
\pgfpathlineto{\pgfpoint{16.100730\du}{10.863490\du}}
\pgfpathlineto{\pgfpoint{16.022965\du}{10.854363\du}}
\pgfpathlineto{\pgfpoint{15.946294\du}{10.844505\du}}
\pgfpathlineto{\pgfpoint{15.872180\du}{10.832457\du}}
\pgfpathlineto{\pgfpoint{15.799525\du}{10.819679\du}}
\pgfpathlineto{\pgfpoint{15.729062\du}{10.806170\du}}
\pgfpathlineto{\pgfpoint{15.660424\du}{10.791201\du}}
\pgfpathlineto{\pgfpoint{15.594706\du}{10.774772\du}}
\pgfpathlineto{\pgfpoint{15.531179\du}{10.757612\du}}
\pgfpathlineto{\pgfpoint{15.470208\du}{10.739723\du}}
\pgfpathlineto{\pgfpoint{15.411793\du}{10.720372\du}}
\pgfpathlineto{\pgfpoint{15.356663\du}{10.700292\du}}
\pgfpathlineto{\pgfpoint{15.303724\du}{10.679482\du}}
\pgfpathlineto{\pgfpoint{15.253706\du}{10.657576\du}}
\pgfpathlineto{\pgfpoint{15.206608\du}{10.634940\du}}
\pgfpathlineto{\pgfpoint{15.163162\du}{10.611574\du}}
\pgfpathlineto{\pgfpoint{15.122271\du}{10.587477\du}}
\pgfpathlineto{\pgfpoint{15.085031\du}{10.562286\du}}
\pgfpathlineto{\pgfpoint{15.051807\du}{10.537094\du}}
\pgfpathlineto{\pgfpoint{15.021504\du}{10.510807\du}}
\pgfpathlineto{\pgfpoint{14.994852\du}{10.483790\du}}
\pgfpathlineto{\pgfpoint{14.972946\du}{10.456407\du}}
\pgfpathlineto{\pgfpoint{14.953596\du}{10.428295\du}}
\pgfpathlineto{\pgfpoint{14.938992\du}{10.400183\du}}
\pgfpathlineto{\pgfpoint{14.928405\du}{10.371705\du}}
\pgfpathlineto{\pgfpoint{14.921468\du}{10.341767\du}}
\pgfpathlineto{\pgfpoint{14.919642\du}{10.312559\du}}
\pgfpathlineto{\pgfpoint{14.921468\du}{10.282621\du}}
\pgfpathlineto{\pgfpoint{14.928405\du}{10.253414\du}}
\pgfpathlineto{\pgfpoint{14.938992\du}{10.224206\du}}
\pgfpathlineto{\pgfpoint{14.953596\du}{10.196093\du}}
\pgfpathlineto{\pgfpoint{14.972946\du}{10.167981\du}}
\pgfpathlineto{\pgfpoint{14.994852\du}{10.140599\du}}
\pgfpathlineto{\pgfpoint{15.021504\du}{10.113947\du}}
\pgfpathlineto{\pgfpoint{15.051807\du}{10.087660\du}}
\pgfpathlineto{\pgfpoint{15.085031\du}{10.062103\du}}
\pgfpathlineto{\pgfpoint{15.122271\du}{10.037276\du}}
\pgfpathlineto{\pgfpoint{15.163162\du}{10.012815\du}}
\pgfpathlineto{\pgfpoint{15.206608\du}{9.989449\du}}
\pgfpathlineto{\pgfpoint{15.253706\du}{9.967178\du}}
\pgfpathlineto{\pgfpoint{15.303724\du}{9.944907\du}}
\pgfpathlineto{\pgfpoint{15.356663\du}{9.924096\du}}
\pgfpathlineto{\pgfpoint{15.411793\du}{9.904016\du}}
\pgfpathlineto{\pgfpoint{15.470208\du}{9.885396\du}}
\pgfpathlineto{\pgfpoint{15.531179\du}{9.866411\du}}
\pgfpathlineto{\pgfpoint{15.594706\du}{9.849617\du}}
\pgfpathlineto{\pgfpoint{15.660424\du}{9.833917\du}}
\pgfpathlineto{\pgfpoint{15.729062\du}{9.818583\du}}
\pgfpathlineto{\pgfpoint{15.799525\du}{9.804710\du}}
\pgfpathlineto{\pgfpoint{15.872180\du}{9.791566\du}}
\pgfpathlineto{\pgfpoint{15.946294\du}{9.780613\du}}
\pgfpathlineto{\pgfpoint{16.022965\du}{9.770026\du}}
\pgfpathlineto{\pgfpoint{16.100730\du}{9.760533\du}}
\pgfpathlineto{\pgfpoint{16.180321\du}{9.753231\du}}
\pgfpathlineto{\pgfpoint{16.261738\du}{9.746659\du}}
\pgfpathlineto{\pgfpoint{16.344615\du}{9.741548\du}}
\pgfpathlineto{\pgfpoint{16.428222\du}{9.737897\du}}
\pgfpathlineto{\pgfpoint{16.514020\du}{9.736072\du}}
\pgfpathlineto{\pgfpoint{16.600183\du}{9.734976\du}}
\pgfpathlineto{\pgfpoint{16.686710\du}{9.736072\du}}
\pgfpathlineto{\pgfpoint{16.772143\du}{9.737897\du}}
\pgfpathlineto{\pgfpoint{16.855750\du}{9.741548\du}}
\pgfpathlineto{\pgfpoint{16.938992\du}{9.746659\du}}
\pgfpathlineto{\pgfpoint{17.020044\du}{9.753231\du}}
\pgfpathlineto{\pgfpoint{17.099635\du}{9.760533\du}}
\pgfpathlineto{\pgfpoint{17.178131\du}{9.770026\du}}
\pgfpathlineto{\pgfpoint{17.254071\du}{9.780613\du}}
\pgfpathlineto{\pgfpoint{17.328916\du}{9.791566\du}}
\pgfpathlineto{\pgfpoint{17.400840\du}{9.804710\du}}
\pgfpathlineto{\pgfpoint{17.471668\du}{9.818583\du}}
\pgfpathlineto{\pgfpoint{17.540307\du}{9.833917\du}}
\pgfpathlineto{\pgfpoint{17.605294\du}{9.849617\du}}
\pgfpathlineto{\pgfpoint{17.669186\du}{9.866411\du}}
\pgfpathlineto{\pgfpoint{17.729792\du}{9.885396\du}}
\pgfpathlineto{\pgfpoint{17.787842\du}{9.904016\du}}
\pgfpathlineto{\pgfpoint{17.843337\du}{9.924096\du}}
\pgfpathlineto{\pgfpoint{17.896276\du}{9.944907\du}}
\pgfpathlineto{\pgfpoint{17.946659\du}{9.967178\du}}
\pgfpathlineto{\pgfpoint{17.993027\du}{9.989449\du}}
\pgfpathlineto{\pgfpoint{18.036838\du}{10.012815\du}}
\pgfpathlineto{\pgfpoint{18.077729\du}{10.037276\du}}
\pgfpathlineto{\pgfpoint{18.114969\du}{10.062103\du}}
\pgfpathlineto{\pgfpoint{18.147828\du}{10.087660\du}}
\pgfpathlineto{\pgfpoint{18.178496\du}{10.113947\du}}
\pgfpathlineto{\pgfpoint{18.205148\du}{10.140599\du}}
\pgfpathlineto{\pgfpoint{18.227054\du}{10.167981\du}}
\pgfpathlineto{\pgfpoint{18.246404\du}{10.196093\du}}
\pgfpathlineto{\pgfpoint{18.261008\du}{10.224206\du}}
\pgfpathlineto{\pgfpoint{18.271230\du}{10.253414\du}}
\pgfpathlineto{\pgfpoint{18.278532\du}{10.282621\du}}
\pgfpathlineto{\pgfpoint{18.279993\du}{10.312559\du}}
\pgfusepath{fill}
\pgfsetlinewidth{0.000000\du}
\pgfsetbuttcap
\pgfsetmiterjoin
\pgfsetdash{}{0pt}
\definecolor{dialinecolor}{rgb}{0.678431, 0.839216, 0.905882}
\pgfsetfillcolor{dialinecolor}
\pgfpathmoveto{\pgfpoint{16.600183\du}{10.900000\du}}
\pgfpathlineto{\pgfpoint{16.600183\du}{10.900000\du}}
\pgfpathlineto{\pgfpoint{16.643629\du}{10.900000\du}}
\pgfpathlineto{\pgfpoint{16.687076\du}{10.899270\du}}
\pgfpathlineto{\pgfpoint{16.730157\du}{10.898175\du}}
\pgfpathlineto{\pgfpoint{16.772143\du}{10.897079\du}}
\pgfpathlineto{\pgfpoint{16.814859\du}{10.895254\du}}
\pgfpathlineto{\pgfpoint{16.856480\du}{10.893063\du}}
\pgfpathlineto{\pgfpoint{16.898101\du}{10.890507\du}}
\pgfpathlineto{\pgfpoint{16.939723\du}{10.888317\du}}
\pgfpathlineto{\pgfpoint{16.980248\du}{10.885396\du}}
\pgfpathlineto{\pgfpoint{17.021139\du}{10.881745\du}}
\pgfpathlineto{\pgfpoint{17.060935\du}{10.877729\du}}
\pgfpathlineto{\pgfpoint{17.101095\du}{10.873713\du}}
\pgfpathlineto{\pgfpoint{17.139796\du}{10.869332\du}}
\pgfpathlineto{\pgfpoint{17.179226\du}{10.864951\du}}
\pgfpathlineto{\pgfpoint{17.217196\du}{10.859474\du}}
\pgfpathlineto{\pgfpoint{17.256261\du}{10.854363\du}}
\pgfpathlineto{\pgfpoint{17.293501\du}{10.848886\du}}
\pgfpathlineto{\pgfpoint{17.330376\du}{10.842680\du}}
\pgfpathlineto{\pgfpoint{17.366886\du}{10.836838\du}}
\pgfpathlineto{\pgfpoint{17.403395\du}{10.830267\du}}
\pgfpathlineto{\pgfpoint{17.438810\du}{10.823330\du}}
\pgfpathlineto{\pgfpoint{17.473494\du}{10.816393\du}}
\pgfpathlineto{\pgfpoint{17.508178\du}{10.808726\du}}
\pgfpathlineto{\pgfpoint{17.542132\du}{10.801059\du}}
\pgfpathlineto{\pgfpoint{17.575721\du}{10.792662\du}}
\pgfpathlineto{\pgfpoint{17.608580\du}{10.784629\du}}
\pgfpathlineto{\pgfpoint{17.640343\du}{10.776597\du}}
\pgfpathlineto{\pgfpoint{17.671742\du}{10.767835\du}}
\pgfpathlineto{\pgfpoint{17.687076\du}{10.763089\du}}
\pgfpathlineto{\pgfpoint{17.702410\du}{10.759073\du}}
\pgfpathlineto{\pgfpoint{17.718474\du}{10.754326\du}}
\pgfpathlineto{\pgfpoint{17.733078\du}{10.749580\du}}
\pgfpathlineto{\pgfpoint{17.747317\du}{10.744834\du}}
\pgfpathlineto{\pgfpoint{17.762286\du}{10.739723\du}}
\pgfpathlineto{\pgfpoint{17.777254\du}{10.734976\du}}
\pgfpathlineto{\pgfpoint{17.791128\du}{10.730230\du}}
\pgfpathlineto{\pgfpoint{17.805367\du}{10.725119\du}}
\pgfpathlineto{\pgfpoint{17.819241\du}{10.720372\du}}
\pgfpathlineto{\pgfpoint{17.833479\du}{10.714896\du}}
\pgfpathlineto{\pgfpoint{17.846623\du}{10.710515\du}}
\pgfpathlineto{\pgfpoint{17.860862\du}{10.705038\du}}
\pgfpathlineto{\pgfpoint{17.874005\du}{10.699927\du}}
\pgfpathlineto{\pgfpoint{17.887149\du}{10.694451\du}}
\pgfpathlineto{\pgfpoint{17.900657\du}{10.688609\du}}
\pgfpathlineto{\pgfpoint{17.913436\du}{10.683498\du}}
\pgfpathlineto{\pgfpoint{17.925484\du}{10.678021\du}}
\pgfpathlineto{\pgfpoint{17.938262\du}{10.672180\du}}
\pgfpathlineto{\pgfpoint{17.950675\du}{10.667068\du}}
\pgfpathlineto{\pgfpoint{17.963089\du}{10.661227\du}}
\pgfpathlineto{\pgfpoint{17.974407\du}{10.655750\du}}
\pgfpathlineto{\pgfpoint{17.986455\du}{10.649909\du}}
\pgfpathlineto{\pgfpoint{17.997408\du}{10.644067\du}}
\pgfpathlineto{\pgfpoint{18.009091\du}{10.638226\du}}
\pgfpathlineto{\pgfpoint{18.020409\du}{10.632384\du}}
\pgfpathlineto{\pgfpoint{18.031727\du}{10.626543\du}}
\pgfpathlineto{\pgfpoint{18.042315\du}{10.620336\du}}
\pgfpathlineto{\pgfpoint{18.052172\du}{10.614494\du}}
\pgfpathlineto{\pgfpoint{18.062760\du}{10.608653\du}}
\pgfpathlineto{\pgfpoint{18.072983\du}{10.602081\du}}
\pgfpathlineto{\pgfpoint{18.082840\du}{10.596240\du}}
\pgfpathlineto{\pgfpoint{18.092698\du}{10.589668\du}}
\pgfpathlineto{\pgfpoint{18.101825\du}{10.583461\du}}
\pgfpathlineto{\pgfpoint{18.110953\du}{10.576889\du}}
\pgfpathlineto{\pgfpoint{18.120445\du}{10.571048\du}}
\pgfpathlineto{\pgfpoint{18.129208\du}{10.564476\du}}
\pgfpathlineto{\pgfpoint{18.138335\du}{10.558269\du}}
\pgfpathlineto{\pgfpoint{18.146367\du}{10.551698\du}}
\pgfpathlineto{\pgfpoint{18.155130\du}{10.544761\du}}
\pgfpathlineto{\pgfpoint{18.162432\du}{10.538189\du}}
\pgfpathlineto{\pgfpoint{18.170464\du}{10.531982\du}}
\pgfpathlineto{\pgfpoint{18.178496\du}{10.524681\du}}
\pgfpathlineto{\pgfpoint{18.185068\du}{10.518474\du}}
\pgfpathlineto{\pgfpoint{18.192369\du}{10.511537\du}}
\pgfpathlineto{\pgfpoint{18.198941\du}{10.504235\du}}
\pgfpathlineto{\pgfpoint{18.205878\du}{10.498028\du}}
\pgfpathlineto{\pgfpoint{18.212450\du}{10.491092\du}}
\pgfpathlineto{\pgfpoint{18.218656\du}{10.483790\du}}
\pgfpathlineto{\pgfpoint{18.224498\du}{10.476853\du}}
\pgfpathlineto{\pgfpoint{18.229974\du}{10.469916\du}}
\pgfpathlineto{\pgfpoint{18.235816\du}{10.462979\du}}
\pgfpathlineto{\pgfpoint{18.240562\du}{10.455677\du}}
\pgfpathlineto{\pgfpoint{18.246039\du}{10.448375\du}}
\pgfpathlineto{\pgfpoint{18.250785\du}{10.441073\du}}
\pgfpathlineto{\pgfpoint{18.255166\du}{10.434137\du}}
\pgfpathlineto{\pgfpoint{18.259182\du}{10.426470\du}}
\pgfpathlineto{\pgfpoint{18.263198\du}{10.419533\du}}
\pgfpathlineto{\pgfpoint{18.266484\du}{10.411866\du}}
\pgfpathlineto{\pgfpoint{18.270500\du}{10.404199\du}}
\pgfpathlineto{\pgfpoint{18.273786\du}{10.396532\du}}
\pgfpathlineto{\pgfpoint{18.276342\du}{10.389230\du}}
\pgfpathlineto{\pgfpoint{18.279263\du}{10.381928\du}}
\pgfpathlineto{\pgfpoint{18.281088\du}{10.374626\du}}
\pgfpathlineto{\pgfpoint{18.284009\du}{10.366959\du}}
\pgfpathlineto{\pgfpoint{18.285104\du}{10.358562\du}}
\pgfpathlineto{\pgfpoint{18.287295\du}{10.350894\du}}
\pgfpathlineto{\pgfpoint{18.288390\du}{10.343593\du}}
\pgfpathlineto{\pgfpoint{18.289120\du}{10.335926\du}}
\pgfpathlineto{\pgfpoint{18.289850\du}{10.327528\du}}
\pgfpathlineto{\pgfpoint{18.290581\du}{10.320226\du}}
\pgfpathlineto{\pgfpoint{18.290581\du}{10.312559\du}}
\pgfpathlineto{\pgfpoint{18.270500\du}{10.312559\du}}
\pgfpathlineto{\pgfpoint{18.269770\du}{10.319496\du}}
\pgfpathlineto{\pgfpoint{18.269770\du}{10.326433\du}}
\pgfpathlineto{\pgfpoint{18.269405\du}{10.333370\du}}
\pgfpathlineto{\pgfpoint{18.267579\du}{10.340672\du}}
\pgfpathlineto{\pgfpoint{18.266484\du}{10.347609\du}}
\pgfpathlineto{\pgfpoint{18.265754\du}{10.354545\du}}
\pgfpathlineto{\pgfpoint{18.263563\du}{10.361482\du}}
\pgfpathlineto{\pgfpoint{18.262103\du}{10.368784\du}}
\pgfpathlineto{\pgfpoint{18.259912\du}{10.374991\du}}
\pgfpathlineto{\pgfpoint{18.257357\du}{10.381928\du}}
\pgfpathlineto{\pgfpoint{18.254436\du}{10.389230\du}}
\pgfpathlineto{\pgfpoint{18.251880\du}{10.396166\du}}
\pgfpathlineto{\pgfpoint{18.247864\du}{10.403103\du}}
\pgfpathlineto{\pgfpoint{18.244943\du}{10.409675\du}}
\pgfpathlineto{\pgfpoint{18.240927\du}{10.416612\du}}
\pgfpathlineto{\pgfpoint{18.238007\du}{10.423184\du}}
\pgfpathlineto{\pgfpoint{18.233625\du}{10.430120\du}}
\pgfpathlineto{\pgfpoint{18.229244\du}{10.437057\du}}
\pgfpathlineto{\pgfpoint{18.224498\du}{10.443629\du}}
\pgfpathlineto{\pgfpoint{18.219387\du}{10.449836\du}}
\pgfpathlineto{\pgfpoint{18.214640\du}{10.457138\du}}
\pgfpathlineto{\pgfpoint{18.208434\du}{10.464074\du}}
\pgfpathlineto{\pgfpoint{18.202957\du}{10.470281\du}}
\pgfpathlineto{\pgfpoint{18.197846\du}{10.476853\du}}
\pgfpathlineto{\pgfpoint{18.191274\du}{10.483425\du}}
\pgfpathlineto{\pgfpoint{18.184337\du}{10.490361\du}}
\pgfpathlineto{\pgfpoint{18.178496\du}{10.496933\du}}
\pgfpathlineto{\pgfpoint{18.171194\du}{10.503140\du}}
\pgfpathlineto{\pgfpoint{18.164622\du}{10.509712\du}}
\pgfpathlineto{\pgfpoint{18.156955\du}{10.515918\du}}
\pgfpathlineto{\pgfpoint{18.149653\du}{10.522490\du}}
\pgfpathlineto{\pgfpoint{18.141621\du}{10.529062\du}}
\pgfpathlineto{\pgfpoint{18.133954\du}{10.535268\du}}
\pgfpathlineto{\pgfpoint{18.125192\du}{10.541840\du}}
\pgfpathlineto{\pgfpoint{18.116794\du}{10.547682\du}}
\pgfpathlineto{\pgfpoint{18.108032\du}{10.554253\du}}
\pgfpathlineto{\pgfpoint{18.100000\du}{10.560460\du}}
\pgfpathlineto{\pgfpoint{18.091238\du}{10.566302\du}}
\pgfpathlineto{\pgfpoint{18.081745\du}{10.572873\du}}
\pgfpathlineto{\pgfpoint{18.071522\du}{10.578715\du}}
\pgfpathlineto{\pgfpoint{18.062395\du}{10.585287\du}}
\pgfpathlineto{\pgfpoint{18.052172\du}{10.591128\du}}
\pgfpathlineto{\pgfpoint{18.042315\du}{10.596970\du}}
\pgfpathlineto{\pgfpoint{18.032457\du}{10.602811\du}}
\pgfpathlineto{\pgfpoint{18.021504\du}{10.608653\du}}
\pgfpathlineto{\pgfpoint{18.010551\du}{10.614494\du}}
\pgfpathlineto{\pgfpoint{18.000329\du}{10.620336\du}}
\pgfpathlineto{\pgfpoint{17.988280\du}{10.626177\du}}
\pgfpathlineto{\pgfpoint{17.977693\du}{10.631289\du}}
\pgfpathlineto{\pgfpoint{17.965644\du}{10.637130\du}}
\pgfpathlineto{\pgfpoint{17.954326\du}{10.642972\du}}
\pgfpathlineto{\pgfpoint{17.941913\du}{10.648448\du}}
\pgfpathlineto{\pgfpoint{17.929865\du}{10.653560\du}}
\pgfpathlineto{\pgfpoint{17.917452\du}{10.659401\du}}
\pgfpathlineto{\pgfpoint{17.905403\du}{10.664878\du}}
\pgfpathlineto{\pgfpoint{17.892260\du}{10.669989\du}}
\pgfpathlineto{\pgfpoint{17.879482\du}{10.675100\du}}
\pgfpathlineto{\pgfpoint{17.866703\du}{10.680577\du}}
\pgfpathlineto{\pgfpoint{17.853560\du}{10.685688\du}}
\pgfpathlineto{\pgfpoint{17.840051\du}{10.691165\du}}
\pgfpathlineto{\pgfpoint{17.826908\du}{10.695546\du}}
\pgfpathlineto{\pgfpoint{17.812669\du}{10.701022\du}}
\pgfpathlineto{\pgfpoint{17.798795\du}{10.705769\du}}
\pgfpathlineto{\pgfpoint{17.784556\du}{10.710880\du}}
\pgfpathlineto{\pgfpoint{17.769953\du}{10.715626\du}}
\pgfpathlineto{\pgfpoint{17.756079\du}{10.720372\du}}
\pgfpathlineto{\pgfpoint{17.741475\du}{10.725119\du}}
\pgfpathlineto{\pgfpoint{17.727236\du}{10.729500\du}}
\pgfpathlineto{\pgfpoint{17.711537\du}{10.734246\du}}
\pgfpathlineto{\pgfpoint{17.697298\du}{10.738992\du}}
\pgfpathlineto{\pgfpoint{17.681599\du}{10.743739\du}}
\pgfpathlineto{\pgfpoint{17.666630\du}{10.747755\du}}
\pgfpathlineto{\pgfpoint{17.635232\du}{10.756517\du}}
\pgfpathlineto{\pgfpoint{17.603103\du}{10.764914\du}}
\pgfpathlineto{\pgfpoint{17.570975\du}{10.772946\du}}
\pgfpathlineto{\pgfpoint{17.537751\du}{10.780978\du}}
\pgfpathlineto{\pgfpoint{17.503432\du}{10.788645\du}}
\pgfpathlineto{\pgfpoint{17.469478\du}{10.795947\du}}
\pgfpathlineto{\pgfpoint{17.434794\du}{10.802884\du}}
\pgfpathlineto{\pgfpoint{17.399014\du}{10.809821\du}}
\pgfpathlineto{\pgfpoint{17.363600\du}{10.816393\du}}
\pgfpathlineto{\pgfpoint{17.326725\du}{10.822599\du}}
\pgfpathlineto{\pgfpoint{17.290215\du}{10.828441\du}}
\pgfpathlineto{\pgfpoint{17.252976\du}{10.834283\du}}
\pgfpathlineto{\pgfpoint{17.215371\du}{10.839759\du}}
\pgfpathlineto{\pgfpoint{17.176670\du}{10.844505\du}}
\pgfpathlineto{\pgfpoint{17.137970\du}{10.848886\du}}
\pgfpathlineto{\pgfpoint{17.098540\du}{10.853633\du}}
\pgfpathlineto{\pgfpoint{17.059474\du}{10.857284\du}}
\pgfpathlineto{\pgfpoint{17.019314\du}{10.861300\du}}
\pgfpathlineto{\pgfpoint{16.979153\du}{10.864221\du}}
\pgfpathlineto{\pgfpoint{16.937897\du}{10.867871\du}}
\pgfpathlineto{\pgfpoint{16.897006\du}{10.870792\du}}
\pgfpathlineto{\pgfpoint{16.855750\du}{10.872983\du}}
\pgfpathlineto{\pgfpoint{16.814129\du}{10.874808\du}}
\pgfpathlineto{\pgfpoint{16.771413\du}{10.876634\du}}
\pgfpathlineto{\pgfpoint{16.728697\du}{10.877729\du}}
\pgfpathlineto{\pgfpoint{16.686710\du}{10.878824\du}}
\pgfpathlineto{\pgfpoint{16.643264\du}{10.878824\du}}
\pgfpathlineto{\pgfpoint{16.600183\du}{10.879555\du}}
\pgfpathlineto{\pgfpoint{16.600183\du}{10.879555\du}}
\pgfpathlineto{\pgfpoint{16.600183\du}{10.879555\du}}
\pgfpathlineto{\pgfpoint{16.599452\du}{10.879555\du}}
\pgfpathlineto{\pgfpoint{16.597627\du}{10.879555\du}}
\pgfpathlineto{\pgfpoint{16.596532\du}{10.879920\du}}
\pgfpathlineto{\pgfpoint{16.595801\du}{10.879920\du}}
\pgfpathlineto{\pgfpoint{16.595436\du}{10.880650\du}}
\pgfpathlineto{\pgfpoint{16.593976\du}{10.881015\du}}
\pgfpathlineto{\pgfpoint{16.593246\du}{10.881745\du}}
\pgfpathlineto{\pgfpoint{16.592516\du}{10.882475\du}}
\pgfpathlineto{\pgfpoint{16.591420\du}{10.884301\du}}
\pgfpathlineto{\pgfpoint{16.590690\du}{10.885761\du}}
\pgfpathlineto{\pgfpoint{16.590690\du}{10.887587\du}}
\pgfpathlineto{\pgfpoint{16.589960\du}{10.889412\du}}
\pgfpathlineto{\pgfpoint{16.590690\du}{10.891603\du}}
\pgfpathlineto{\pgfpoint{16.590690\du}{10.893428\du}}
\pgfpathlineto{\pgfpoint{16.591420\du}{10.895254\du}}
\pgfpathlineto{\pgfpoint{16.592516\du}{10.897079\du}}
\pgfpathlineto{\pgfpoint{16.593246\du}{10.897444\du}}
\pgfpathlineto{\pgfpoint{16.593976\du}{10.898175\du}}
\pgfpathlineto{\pgfpoint{16.595436\du}{10.898905\du}}
\pgfpathlineto{\pgfpoint{16.595801\du}{10.899270\du}}
\pgfpathlineto{\pgfpoint{16.596532\du}{10.899270\du}}
\pgfpathlineto{\pgfpoint{16.597627\du}{10.900000\du}}
\pgfpathlineto{\pgfpoint{16.599452\du}{10.900000\du}}
\pgfpathlineto{\pgfpoint{16.600183\du}{10.900000\du}}
\pgfusepath{fill}
\pgfsetbuttcap
\pgfsetmiterjoin
\pgfsetdash{}{0pt}
\definecolor{dialinecolor}{rgb}{0.678431, 0.839216, 0.905882}
\pgfsetfillcolor{dialinecolor}
\pgfpathmoveto{\pgfpoint{14.909419\du}{10.312559\du}}
\pgfpathlineto{\pgfpoint{14.909419\du}{10.312559\du}}
\pgfpathlineto{\pgfpoint{14.909419\du}{10.320226\du}}
\pgfpathlineto{\pgfpoint{14.909785\du}{10.327528\du}}
\pgfpathlineto{\pgfpoint{14.910515\du}{10.335926\du}}
\pgfpathlineto{\pgfpoint{14.911610\du}{10.343593\du}}
\pgfpathlineto{\pgfpoint{14.912705\du}{10.350894\du}}
\pgfpathlineto{\pgfpoint{14.914531\du}{10.358562\du}}
\pgfpathlineto{\pgfpoint{14.916356\du}{10.366959\du}}
\pgfpathlineto{\pgfpoint{14.918547\du}{10.374626\du}}
\pgfpathlineto{\pgfpoint{14.920737\du}{10.381928\du}}
\pgfpathlineto{\pgfpoint{14.923293\du}{10.389230\du}}
\pgfpathlineto{\pgfpoint{14.926214\du}{10.396532\du}}
\pgfpathlineto{\pgfpoint{14.929865\du}{10.404199\du}}
\pgfpathlineto{\pgfpoint{14.933151\du}{10.411866\du}}
\pgfpathlineto{\pgfpoint{14.936802\du}{10.419533\du}}
\pgfpathlineto{\pgfpoint{14.941183\du}{10.426470\du}}
\pgfpathlineto{\pgfpoint{14.944834\du}{10.434137\du}}
\pgfpathlineto{\pgfpoint{14.949945\du}{10.441073\du}}
\pgfpathlineto{\pgfpoint{14.953961\du}{10.448375\du}}
\pgfpathlineto{\pgfpoint{14.959438\du}{10.455677\du}}
\pgfpathlineto{\pgfpoint{14.964184\du}{10.462979\du}}
\pgfpathlineto{\pgfpoint{14.969660\du}{10.469916\du}}
\pgfpathlineto{\pgfpoint{14.975502\du}{10.476853\du}}
\pgfpathlineto{\pgfpoint{14.981344\du}{10.483790\du}}
\pgfpathlineto{\pgfpoint{14.987185\du}{10.491092\du}}
\pgfpathlineto{\pgfpoint{14.994122\du}{10.498028\du}}
\pgfpathlineto{\pgfpoint{15.000694\du}{10.504235\du}}
\pgfpathlineto{\pgfpoint{15.007631\du}{10.511537\du}}
\pgfpathlineto{\pgfpoint{15.014567\du}{10.518474\du}}
\pgfpathlineto{\pgfpoint{15.021504\du}{10.524681\du}}
\pgfpathlineto{\pgfpoint{15.030267\du}{10.531982\du}}
\pgfpathlineto{\pgfpoint{15.037203\du}{10.538189\du}}
\pgfpathlineto{\pgfpoint{15.044870\du}{10.544761\du}}
\pgfpathlineto{\pgfpoint{15.053633\du}{10.551698\du}}
\pgfpathlineto{\pgfpoint{15.062030\du}{10.558269\du}}
\pgfpathlineto{\pgfpoint{15.070792\du}{10.564476\du}}
\pgfpathlineto{\pgfpoint{15.079189\du}{10.571048\du}}
\pgfpathlineto{\pgfpoint{15.089047\du}{10.576889\du}}
\pgfpathlineto{\pgfpoint{15.098175\du}{10.583461\du}}
\pgfpathlineto{\pgfpoint{15.107667\du}{10.589668\du}}
\pgfpathlineto{\pgfpoint{15.117160\du}{10.596240\du}}
\pgfpathlineto{\pgfpoint{15.127017\du}{10.602081\du}}
\pgfpathlineto{\pgfpoint{15.137240\du}{10.608653\du}}
\pgfpathlineto{\pgfpoint{15.147463\du}{10.614494\du}}
\pgfpathlineto{\pgfpoint{15.158050\du}{10.620336\du}}
\pgfpathlineto{\pgfpoint{15.168273\du}{10.626543\du}}
\pgfpathlineto{\pgfpoint{15.179226\du}{10.632384\du}}
\pgfpathlineto{\pgfpoint{15.190909\du}{10.638226\du}}
\pgfpathlineto{\pgfpoint{15.202227\du}{10.644067\du}}
\pgfpathlineto{\pgfpoint{15.213910\du}{10.649909\du}}
\pgfpathlineto{\pgfpoint{15.225228\du}{10.655750\du}}
\pgfpathlineto{\pgfpoint{15.236911\du}{10.661227\du}}
\pgfpathlineto{\pgfpoint{15.249325\du}{10.667068\du}}
\pgfpathlineto{\pgfpoint{15.261373\du}{10.672180\du}}
\pgfpathlineto{\pgfpoint{15.274516\du}{10.678021\du}}
\pgfpathlineto{\pgfpoint{15.286564\du}{10.683498\du}}
\pgfpathlineto{\pgfpoint{15.299343\du}{10.688609\du}}
\pgfpathlineto{\pgfpoint{15.313217\du}{10.694451\du}}
\pgfpathlineto{\pgfpoint{15.325630\du}{10.699927\du}}
\pgfpathlineto{\pgfpoint{15.338773\du}{10.705038\du}}
\pgfpathlineto{\pgfpoint{15.353012\du}{10.710515\du}}
\pgfpathlineto{\pgfpoint{15.366156\du}{10.714896\du}}
\pgfpathlineto{\pgfpoint{15.380394\du}{10.720372\du}}
\pgfpathlineto{\pgfpoint{15.394268\du}{10.725119\du}}
\pgfpathlineto{\pgfpoint{15.408872\du}{10.730230\du}}
\pgfpathlineto{\pgfpoint{15.422746\du}{10.734976\du}}
\pgfpathlineto{\pgfpoint{15.438445\du}{10.739723\du}}
\pgfpathlineto{\pgfpoint{15.452318\du}{10.744834\du}}
\pgfpathlineto{\pgfpoint{15.466922\du}{10.749580\du}}
\pgfpathlineto{\pgfpoint{15.482256\du}{10.754326\du}}
\pgfpathlineto{\pgfpoint{15.497955\du}{10.759073\du}}
\pgfpathlineto{\pgfpoint{15.512924\du}{10.763089\du}}
\pgfpathlineto{\pgfpoint{15.528624\du}{10.767835\du}}
\pgfpathlineto{\pgfpoint{15.560387\du}{10.776597\du}}
\pgfpathlineto{\pgfpoint{15.592150\du}{10.784629\du}}
\pgfpathlineto{\pgfpoint{15.625374\du}{10.792662\du}}
\pgfpathlineto{\pgfpoint{15.657868\du}{10.801059\du}}
\pgfpathlineto{\pgfpoint{15.692552\du}{10.808726\du}}
\pgfpathlineto{\pgfpoint{15.726871\du}{10.816393\du}}
\pgfpathlineto{\pgfpoint{15.761555\du}{10.823330\du}}
\pgfpathlineto{\pgfpoint{15.797335\du}{10.830267\du}}
\pgfpathlineto{\pgfpoint{15.833479\du}{10.836838\du}}
\pgfpathlineto{\pgfpoint{15.869989\du}{10.842680\du}}
\pgfpathlineto{\pgfpoint{15.907229\du}{10.848886\du}}
\pgfpathlineto{\pgfpoint{15.944834\du}{10.854363\du}}
\pgfpathlineto{\pgfpoint{15.982804\du}{10.859474\du}}
\pgfpathlineto{\pgfpoint{16.021504\du}{10.864951\du}}
\pgfpathlineto{\pgfpoint{16.060204\du}{10.869332\du}}
\pgfpathlineto{\pgfpoint{16.099270\du}{10.873713\du}}
\pgfpathlineto{\pgfpoint{16.139430\du}{10.877729\du}}
\pgfpathlineto{\pgfpoint{16.179226\du}{10.881745\du}}
\pgfpathlineto{\pgfpoint{16.220117\du}{10.885396\du}}
\pgfpathlineto{\pgfpoint{16.261008\du}{10.888317\du}}
\pgfpathlineto{\pgfpoint{16.302264\du}{10.890507\du}}
\pgfpathlineto{\pgfpoint{16.344250\du}{10.893063\du}}
\pgfpathlineto{\pgfpoint{16.385871\du}{10.895254\du}}
\pgfpathlineto{\pgfpoint{16.428222\du}{10.897079\du}}
\pgfpathlineto{\pgfpoint{16.470208\du}{10.898175\du}}
\pgfpathlineto{\pgfpoint{16.513655\du}{10.899270\du}}
\pgfpathlineto{\pgfpoint{16.556371\du}{10.900000\du}}
\pgfpathlineto{\pgfpoint{16.600183\du}{10.900000\du}}
\pgfpathlineto{\pgfpoint{16.600183\du}{10.879555\du}}
\pgfpathlineto{\pgfpoint{16.557466\du}{10.878824\du}}
\pgfpathlineto{\pgfpoint{16.514020\du}{10.878824\du}}
\pgfpathlineto{\pgfpoint{16.471668\du}{10.877729\du}}
\pgfpathlineto{\pgfpoint{16.428952\du}{10.876634\du}}
\pgfpathlineto{\pgfpoint{16.386601\du}{10.874808\du}}
\pgfpathlineto{\pgfpoint{16.344615\du}{10.872983\du}}
\pgfpathlineto{\pgfpoint{16.303724\du}{10.870792\du}}
\pgfpathlineto{\pgfpoint{16.262468\du}{10.867871\du}}
\pgfpathlineto{\pgfpoint{16.221577\du}{10.864221\du}}
\pgfpathlineto{\pgfpoint{16.181417\du}{10.861300\du}}
\pgfpathlineto{\pgfpoint{16.141621\du}{10.857284\du}}
\pgfpathlineto{\pgfpoint{16.102191\du}{10.853633\du}}
\pgfpathlineto{\pgfpoint{16.062760\du}{10.848886\du}}
\pgfpathlineto{\pgfpoint{16.023695\du}{10.844505\du}}
\pgfpathlineto{\pgfpoint{15.985360\du}{10.839759\du}}
\pgfpathlineto{\pgfpoint{15.947755\du}{10.834283\du}}
\pgfpathlineto{\pgfpoint{15.910515\du}{10.828441\du}}
\pgfpathlineto{\pgfpoint{15.874005\du}{10.822599\du}}
\pgfpathlineto{\pgfpoint{15.836765\du}{10.816393\du}}
\pgfpathlineto{\pgfpoint{15.801716\du}{10.809821\du}}
\pgfpathlineto{\pgfpoint{15.765571\du}{10.802884\du}}
\pgfpathlineto{\pgfpoint{15.730887\du}{10.795947\du}}
\pgfpathlineto{\pgfpoint{15.696568\du}{10.788645\du}}
\pgfpathlineto{\pgfpoint{15.662614\du}{10.780978\du}}
\pgfpathlineto{\pgfpoint{15.629390\du}{10.772946\du}}
\pgfpathlineto{\pgfpoint{15.597992\du}{10.764914\du}}
\pgfpathlineto{\pgfpoint{15.565498\du}{10.756517\du}}
\pgfpathlineto{\pgfpoint{15.534465\du}{10.747755\du}}
\pgfpathlineto{\pgfpoint{15.518766\du}{10.743739\du}}
\pgfpathlineto{\pgfpoint{15.503067\du}{10.738992\du}}
\pgfpathlineto{\pgfpoint{15.488463\du}{10.734246\du}}
\pgfpathlineto{\pgfpoint{15.473494\du}{10.729500\du}}
\pgfpathlineto{\pgfpoint{15.458160\du}{10.725119\du}}
\pgfpathlineto{\pgfpoint{15.443921\du}{10.720372\du}}
\pgfpathlineto{\pgfpoint{15.429682\du}{10.715626\du}}
\pgfpathlineto{\pgfpoint{15.415444\du}{10.710880\du}}
\pgfpathlineto{\pgfpoint{15.401205\du}{10.705769\du}}
\pgfpathlineto{\pgfpoint{15.387331\du}{10.701022\du}}
\pgfpathlineto{\pgfpoint{15.373457\du}{10.695546\du}}
\pgfpathlineto{\pgfpoint{15.359949\du}{10.691165\du}}
\pgfpathlineto{\pgfpoint{15.346440\du}{10.685688\du}}
\pgfpathlineto{\pgfpoint{15.333662\du}{10.680577\du}}
\pgfpathlineto{\pgfpoint{15.320153\du}{10.675100\du}}
\pgfpathlineto{\pgfpoint{15.307375\du}{10.669989\du}}
\pgfpathlineto{\pgfpoint{15.294962\du}{10.664878\du}}
\pgfpathlineto{\pgfpoint{15.281818\du}{10.659401\du}}
\pgfpathlineto{\pgfpoint{15.269770\du}{10.653560\du}}
\pgfpathlineto{\pgfpoint{15.258452\du}{10.648448\du}}
\pgfpathlineto{\pgfpoint{15.245674\du}{10.642972\du}}
\pgfpathlineto{\pgfpoint{15.233991\du}{10.637130\du}}
\pgfpathlineto{\pgfpoint{15.222307\du}{10.631289\du}}
\pgfpathlineto{\pgfpoint{15.211355\du}{10.626177\du}}
\pgfpathlineto{\pgfpoint{15.199671\du}{10.620336\du}}
\pgfpathlineto{\pgfpoint{15.189814\du}{10.614494\du}}
\pgfpathlineto{\pgfpoint{15.178496\du}{10.608653\du}}
\pgfpathlineto{\pgfpoint{15.167908\du}{10.602811\du}}
\pgfpathlineto{\pgfpoint{15.158050\du}{10.596970\du}}
\pgfpathlineto{\pgfpoint{15.147463\du}{10.591128\du}}
\pgfpathlineto{\pgfpoint{15.137605\du}{10.585287\du}}
\pgfpathlineto{\pgfpoint{15.128112\du}{10.578715\du}}
\pgfpathlineto{\pgfpoint{15.118255\du}{10.572873\du}}
\pgfpathlineto{\pgfpoint{15.109127\du}{10.566302\du}}
\pgfpathlineto{\pgfpoint{15.100730\du}{10.560460\du}}
\pgfpathlineto{\pgfpoint{15.091968\du}{10.554253\du}}
\pgfpathlineto{\pgfpoint{15.082840\du}{10.547682\du}}
\pgfpathlineto{\pgfpoint{15.074078\du}{10.541840\du}}
\pgfpathlineto{\pgfpoint{15.065681\du}{10.535268\du}}
\pgfpathlineto{\pgfpoint{15.058379\du}{10.529062\du}}
\pgfpathlineto{\pgfpoint{15.050347\du}{10.522490\du}}
\pgfpathlineto{\pgfpoint{15.042680\du}{10.515918\du}}
\pgfpathlineto{\pgfpoint{15.035743\du}{10.509712\du}}
\pgfpathlineto{\pgfpoint{15.028441\du}{10.503140\du}}
\pgfpathlineto{\pgfpoint{15.021504\du}{10.496933\du}}
\pgfpathlineto{\pgfpoint{15.015298\du}{10.490361\du}}
\pgfpathlineto{\pgfpoint{15.008726\du}{10.483425\du}}
\pgfpathlineto{\pgfpoint{15.002884\du}{10.476853\du}}
\pgfpathlineto{\pgfpoint{14.996678\du}{10.470281\du}}
\pgfpathlineto{\pgfpoint{14.991566\du}{10.464074\du}}
\pgfpathlineto{\pgfpoint{14.985360\du}{10.457138\du}}
\pgfpathlineto{\pgfpoint{14.980978\du}{10.450566\du}}
\pgfpathlineto{\pgfpoint{14.975502\du}{10.443629\du}}
\pgfpathlineto{\pgfpoint{14.971121\du}{10.437057\du}}
\pgfpathlineto{\pgfpoint{14.966740\du}{10.430120\du}}
\pgfpathlineto{\pgfpoint{14.962359\du}{10.423184\du}}
\pgfpathlineto{\pgfpoint{14.959073\du}{10.416612\du}}
\pgfpathlineto{\pgfpoint{14.955057\du}{10.409675\du}}
\pgfpathlineto{\pgfpoint{14.951041\du}{10.403103\du}}
\pgfpathlineto{\pgfpoint{14.948120\du}{10.396166\du}}
\pgfpathlineto{\pgfpoint{14.945564\du}{10.389230\du}}
\pgfpathlineto{\pgfpoint{14.942278\du}{10.381928\du}}
\pgfpathlineto{\pgfpoint{14.940088\du}{10.374991\du}}
\pgfpathlineto{\pgfpoint{14.937897\du}{10.368784\du}}
\pgfpathlineto{\pgfpoint{14.936437\du}{10.361482\du}}
\pgfpathlineto{\pgfpoint{14.934246\du}{10.354545\du}}
\pgfpathlineto{\pgfpoint{14.933151\du}{10.347609\du}}
\pgfpathlineto{\pgfpoint{14.932055\du}{10.340672\du}}
\pgfpathlineto{\pgfpoint{14.930595\du}{10.333370\du}}
\pgfpathlineto{\pgfpoint{14.930230\du}{10.326433\du}}
\pgfpathlineto{\pgfpoint{14.930230\du}{10.319496\du}}
\pgfpathlineto{\pgfpoint{14.929865\du}{10.312559\du}}
\pgfpathlineto{\pgfpoint{14.929865\du}{10.312559\du}}
\pgfpathlineto{\pgfpoint{14.929865\du}{10.312559\du}}
\pgfpathlineto{\pgfpoint{14.929865\du}{10.310734\du}}
\pgfpathlineto{\pgfpoint{14.929865\du}{10.310004\du}}
\pgfpathlineto{\pgfpoint{14.929500\du}{10.308908\du}}
\pgfpathlineto{\pgfpoint{14.929500\du}{10.307813\du}}
\pgfpathlineto{\pgfpoint{14.928405\du}{10.307083\du}}
\pgfpathlineto{\pgfpoint{14.928039\du}{10.305988\du}}
\pgfpathlineto{\pgfpoint{14.927674\du}{10.305257\du}}
\pgfpathlineto{\pgfpoint{14.926579\du}{10.304892\du}}
\pgfpathlineto{\pgfpoint{14.925119\du}{10.303797\du}}
\pgfpathlineto{\pgfpoint{14.923293\du}{10.302337\du}}
\pgfpathlineto{\pgfpoint{14.921468\du}{10.301972\du}}
\pgfpathlineto{\pgfpoint{14.919642\du}{10.301972\du}}
\pgfpathlineto{\pgfpoint{14.917452\du}{10.301972\du}}
\pgfpathlineto{\pgfpoint{14.915991\du}{10.302337\du}}
\pgfpathlineto{\pgfpoint{14.913801\du}{10.303797\du}}
\pgfpathlineto{\pgfpoint{14.911975\du}{10.304892\du}}
\pgfpathlineto{\pgfpoint{14.911610\du}{10.305257\du}}
\pgfpathlineto{\pgfpoint{14.911245\du}{10.305988\du}}
\pgfpathlineto{\pgfpoint{14.910515\du}{10.307083\du}}
\pgfpathlineto{\pgfpoint{14.909785\du}{10.307813\du}}
\pgfpathlineto{\pgfpoint{14.909785\du}{10.308908\du}}
\pgfpathlineto{\pgfpoint{14.909419\du}{10.310004\du}}
\pgfpathlineto{\pgfpoint{14.909419\du}{10.310734\du}}
\pgfpathlineto{\pgfpoint{14.909419\du}{10.312559\du}}
\pgfusepath{fill}
\pgfsetbuttcap
\pgfsetmiterjoin
\pgfsetdash{}{0pt}
\definecolor{dialinecolor}{rgb}{0.678431, 0.839216, 0.905882}
\pgfsetfillcolor{dialinecolor}
\pgfpathmoveto{\pgfpoint{16.600183\du}{9.725119\du}}
\pgfpathlineto{\pgfpoint{16.600183\du}{9.725119\du}}
\pgfpathlineto{\pgfpoint{16.556371\du}{9.725119\du}}
\pgfpathlineto{\pgfpoint{16.513655\du}{9.725484\du}}
\pgfpathlineto{\pgfpoint{16.470208\du}{9.726579\du}}
\pgfpathlineto{\pgfpoint{16.428222\du}{9.728039\du}}
\pgfpathlineto{\pgfpoint{16.385871\du}{9.729500\du}}
\pgfpathlineto{\pgfpoint{16.344250\du}{9.731325\du}}
\pgfpathlineto{\pgfpoint{16.302264\du}{9.733881\du}}
\pgfpathlineto{\pgfpoint{16.261008\du}{9.736802\du}}
\pgfpathlineto{\pgfpoint{16.220117\du}{9.739723\du}}
\pgfpathlineto{\pgfpoint{16.179226\du}{9.742643\du}}
\pgfpathlineto{\pgfpoint{16.139430\du}{9.746659\du}}
\pgfpathlineto{\pgfpoint{16.099270\du}{9.750675\du}}
\pgfpathlineto{\pgfpoint{16.060204\du}{9.755422\du}}
\pgfpathlineto{\pgfpoint{16.021504\du}{9.760168\du}}
\pgfpathlineto{\pgfpoint{15.982804\du}{9.764914\du}}
\pgfpathlineto{\pgfpoint{15.944834\du}{9.770026\du}}
\pgfpathlineto{\pgfpoint{15.907229\du}{9.775867\du}}
\pgfpathlineto{\pgfpoint{15.869989\du}{9.781709\du}}
\pgfpathlineto{\pgfpoint{15.833479\du}{9.788280\du}}
\pgfpathlineto{\pgfpoint{15.797335\du}{9.794487\du}}
\pgfpathlineto{\pgfpoint{15.761555\du}{9.801789\du}}
\pgfpathlineto{\pgfpoint{15.726871\du}{9.808726\du}}
\pgfpathlineto{\pgfpoint{15.692552\du}{9.815663\du}}
\pgfpathlineto{\pgfpoint{15.657868\du}{9.823695\du}}
\pgfpathlineto{\pgfpoint{15.625374\du}{9.831362\du}}
\pgfpathlineto{\pgfpoint{15.592150\du}{9.839759\du}}
\pgfpathlineto{\pgfpoint{15.560387\du}{9.848521\du}}
\pgfpathlineto{\pgfpoint{15.528624\du}{9.857284\du}}
\pgfpathlineto{\pgfpoint{15.497955\du}{9.866046\du}}
\pgfpathlineto{\pgfpoint{15.466922\du}{9.875539\du}}
\pgfpathlineto{\pgfpoint{15.452318\du}{9.879920\du}}
\pgfpathlineto{\pgfpoint{15.438445\du}{9.884666\du}}
\pgfpathlineto{\pgfpoint{15.422746\du}{9.889412\du}}
\pgfpathlineto{\pgfpoint{15.408872\du}{9.894524\du}}
\pgfpathlineto{\pgfpoint{15.394268\du}{9.899270\du}}
\pgfpathlineto{\pgfpoint{15.380394\du}{9.904746\du}}
\pgfpathlineto{\pgfpoint{15.366156\du}{9.909127\du}}
\pgfpathlineto{\pgfpoint{15.353012\du}{9.914604\du}}
\pgfpathlineto{\pgfpoint{15.338773\du}{9.919715\du}}
\pgfpathlineto{\pgfpoint{15.325630\du}{9.925192\du}}
\pgfpathlineto{\pgfpoint{15.313217\du}{9.930303\du}}
\pgfpathlineto{\pgfpoint{15.299343\du}{9.935779\du}}
\pgfpathlineto{\pgfpoint{15.286564\du}{9.940891\du}}
\pgfpathlineto{\pgfpoint{15.274516\du}{9.946002\du}}
\pgfpathlineto{\pgfpoint{15.261373\du}{9.951844\du}}
\pgfpathlineto{\pgfpoint{15.249325\du}{9.958050\du}}
\pgfpathlineto{\pgfpoint{15.236911\du}{9.963162\du}}
\pgfpathlineto{\pgfpoint{15.225228\du}{9.969003\du}}
\pgfpathlineto{\pgfpoint{15.213910\du}{9.974845\du}}
\pgfpathlineto{\pgfpoint{15.202227\du}{9.980686\du}}
\pgfpathlineto{\pgfpoint{15.190909\du}{9.986528\du}}
\pgfpathlineto{\pgfpoint{15.179226\du}{9.991639\du}}
\pgfpathlineto{\pgfpoint{15.168273\du}{9.998211\du}}
\pgfpathlineto{\pgfpoint{15.158050\du}{10.004053\du}}
\pgfpathlineto{\pgfpoint{15.147463\du}{10.009894\du}}
\pgfpathlineto{\pgfpoint{15.137240\du}{10.015736\du}}
\pgfpathlineto{\pgfpoint{15.127017\du}{10.022307\du}}
\pgfpathlineto{\pgfpoint{15.117160\du}{10.028514\du}}
\pgfpathlineto{\pgfpoint{15.107667\du}{10.034356\du}}
\pgfpathlineto{\pgfpoint{15.098175\du}{10.040927\du}}
\pgfpathlineto{\pgfpoint{15.089047\du}{10.047499\du}}
\pgfpathlineto{\pgfpoint{15.079189\du}{10.053706\du}}
\pgfpathlineto{\pgfpoint{15.070792\du}{10.060277\du}}
\pgfpathlineto{\pgfpoint{15.062030\du}{10.066849\du}}
\pgfpathlineto{\pgfpoint{15.053633\du}{10.073056\du}}
\pgfpathlineto{\pgfpoint{15.044870\du}{10.079628\du}}
\pgfpathlineto{\pgfpoint{15.037203\du}{10.086564\du}}
\pgfpathlineto{\pgfpoint{15.030267\du}{10.093136\du}}
\pgfpathlineto{\pgfpoint{15.021504\du}{10.099343\du}}
\pgfpathlineto{\pgfpoint{15.014567\du}{10.106645\du}}
\pgfpathlineto{\pgfpoint{15.007631\du}{10.112851\du}}
\pgfpathlineto{\pgfpoint{15.000694\du}{10.119788\du}}
\pgfpathlineto{\pgfpoint{14.994122\du}{10.127090\du}}
\pgfpathlineto{\pgfpoint{14.987185\du}{10.133297\du}}
\pgfpathlineto{\pgfpoint{14.981344\du}{10.140599\du}}
\pgfpathlineto{\pgfpoint{14.975502\du}{10.147536\du}}
\pgfpathlineto{\pgfpoint{14.969660\du}{10.155203\du}}
\pgfpathlineto{\pgfpoint{14.964184\du}{10.162139\du}}
\pgfpathlineto{\pgfpoint{14.959438\du}{10.169076\du}}
\pgfpathlineto{\pgfpoint{14.953961\du}{10.176013\du}}
\pgfpathlineto{\pgfpoint{14.949945\du}{10.183315\du}}
\pgfpathlineto{\pgfpoint{14.944834\du}{10.190617\du}}
\pgfpathlineto{\pgfpoint{14.941183\du}{10.197919\du}}
\pgfpathlineto{\pgfpoint{14.936802\du}{10.205221\du}}
\pgfpathlineto{\pgfpoint{14.933151\du}{10.212888\du}}
\pgfpathlineto{\pgfpoint{14.929865\du}{10.220555\du}}
\pgfpathlineto{\pgfpoint{14.926214\du}{10.227492\du}}
\pgfpathlineto{\pgfpoint{14.923293\du}{10.235159\du}}
\pgfpathlineto{\pgfpoint{14.920737\du}{10.242826\du}}
\pgfpathlineto{\pgfpoint{14.918547\du}{10.250493\du}}
\pgfpathlineto{\pgfpoint{14.916356\du}{10.258160\du}}
\pgfpathlineto{\pgfpoint{14.914531\du}{10.265462\du}}
\pgfpathlineto{\pgfpoint{14.912705\du}{10.273129\du}}
\pgfpathlineto{\pgfpoint{14.911610\du}{10.280796\du}}
\pgfpathlineto{\pgfpoint{14.910515\du}{10.289193\du}}
\pgfpathlineto{\pgfpoint{14.909785\du}{10.296495\du}}
\pgfpathlineto{\pgfpoint{14.909419\du}{10.304162\du}}
\pgfpathlineto{\pgfpoint{14.909419\du}{10.312559\du}}
\pgfpathlineto{\pgfpoint{14.929865\du}{10.312559\du}}
\pgfpathlineto{\pgfpoint{14.930230\du}{10.305257\du}}
\pgfpathlineto{\pgfpoint{14.930230\du}{10.298321\du}}
\pgfpathlineto{\pgfpoint{14.930595\du}{10.291384\du}}
\pgfpathlineto{\pgfpoint{14.932055\du}{10.283717\du}}
\pgfpathlineto{\pgfpoint{14.933151\du}{10.277510\du}}
\pgfpathlineto{\pgfpoint{14.934246\du}{10.270208\du}}
\pgfpathlineto{\pgfpoint{14.936437\du}{10.263271\du}}
\pgfpathlineto{\pgfpoint{14.937897\du}{10.256334\du}}
\pgfpathlineto{\pgfpoint{14.940088\du}{10.249398\du}}
\pgfpathlineto{\pgfpoint{14.942278\du}{10.242096\du}}
\pgfpathlineto{\pgfpoint{14.945564\du}{10.235889\du}}
\pgfpathlineto{\pgfpoint{14.948120\du}{10.228952\du}}
\pgfpathlineto{\pgfpoint{14.951041\du}{10.221650\du}}
\pgfpathlineto{\pgfpoint{14.955057\du}{10.214713\du}}
\pgfpathlineto{\pgfpoint{14.959073\du}{10.207777\du}}
\pgfpathlineto{\pgfpoint{14.962359\du}{10.201205\du}}
\pgfpathlineto{\pgfpoint{14.966375\du}{10.194268\du}}
\pgfpathlineto{\pgfpoint{14.971121\du}{10.187696\du}}
\pgfpathlineto{\pgfpoint{14.975502\du}{10.180759\du}}
\pgfpathlineto{\pgfpoint{14.980978\du}{10.174188\du}}
\pgfpathlineto{\pgfpoint{14.985360\du}{10.167981\du}}
\pgfpathlineto{\pgfpoint{14.991566\du}{10.161044\du}}
\pgfpathlineto{\pgfpoint{14.996678\du}{10.154472\du}}
\pgfpathlineto{\pgfpoint{15.002884\du}{10.147536\du}}
\pgfpathlineto{\pgfpoint{15.008726\du}{10.140964\du}}
\pgfpathlineto{\pgfpoint{15.015298\du}{10.134757\du}}
\pgfpathlineto{\pgfpoint{15.021504\du}{10.128185\du}}
\pgfpathlineto{\pgfpoint{15.028441\du}{10.121249\du}}
\pgfpathlineto{\pgfpoint{15.035743\du}{10.115407\du}}
\pgfpathlineto{\pgfpoint{15.042680\du}{10.108105\du}}
\pgfpathlineto{\pgfpoint{15.050347\du}{10.101899\du}}
\pgfpathlineto{\pgfpoint{15.058379\du}{10.096057\du}}
\pgfpathlineto{\pgfpoint{15.065681\du}{10.089485\du}}
\pgfpathlineto{\pgfpoint{15.074078\du}{10.082913\du}}
\pgfpathlineto{\pgfpoint{15.082840\du}{10.076707\du}}
\pgfpathlineto{\pgfpoint{15.091968\du}{10.070135\du}}
\pgfpathlineto{\pgfpoint{15.100730\du}{10.064294\du}}
\pgfpathlineto{\pgfpoint{15.109127\du}{10.058087\du}}
\pgfpathlineto{\pgfpoint{15.118255\du}{10.052245\du}}
\pgfpathlineto{\pgfpoint{15.128112\du}{10.045674\du}}
\pgfpathlineto{\pgfpoint{15.137605\du}{10.039832\du}}
\pgfpathlineto{\pgfpoint{15.147463\du}{10.033991\du}}
\pgfpathlineto{\pgfpoint{15.158050\du}{10.028149\du}}
\pgfpathlineto{\pgfpoint{15.167908\du}{10.022307\du}}
\pgfpathlineto{\pgfpoint{15.178496\du}{10.015736\du}}
\pgfpathlineto{\pgfpoint{15.189814\du}{10.010624\du}}
\pgfpathlineto{\pgfpoint{15.199671\du}{10.004783\du}}
\pgfpathlineto{\pgfpoint{15.211355\du}{9.998941\du}}
\pgfpathlineto{\pgfpoint{15.222307\du}{9.993100\du}}
\pgfpathlineto{\pgfpoint{15.233991\du}{9.987258\du}}
\pgfpathlineto{\pgfpoint{15.245674\du}{9.981782\du}}
\pgfpathlineto{\pgfpoint{15.258452\du}{9.976670\du}}
\pgfpathlineto{\pgfpoint{15.269770\du}{9.970829\du}}
\pgfpathlineto{\pgfpoint{15.281818\du}{9.965352\du}}
\pgfpathlineto{\pgfpoint{15.294962\du}{9.960241\du}}
\pgfpathlineto{\pgfpoint{15.307375\du}{9.954765\du}}
\pgfpathlineto{\pgfpoint{15.320153\du}{9.949653\du}}
\pgfpathlineto{\pgfpoint{15.333662\du}{9.943812\du}}
\pgfpathlineto{\pgfpoint{15.346440\du}{9.939065\du}}
\pgfpathlineto{\pgfpoint{15.359949\du}{9.933954\du}}
\pgfpathlineto{\pgfpoint{15.373457\du}{9.928478\du}}
\pgfpathlineto{\pgfpoint{15.387331\du}{9.924096\du}}
\pgfpathlineto{\pgfpoint{15.401205\du}{9.918620\du}}
\pgfpathlineto{\pgfpoint{15.415444\du}{9.913874\du}}
\pgfpathlineto{\pgfpoint{15.429682\du}{9.909127\du}}
\pgfpathlineto{\pgfpoint{15.443921\du}{9.904016\du}}
\pgfpathlineto{\pgfpoint{15.458160\du}{9.899270\du}}
\pgfpathlineto{\pgfpoint{15.473494\du}{9.894524\du}}
\pgfpathlineto{\pgfpoint{15.503067\du}{9.885761\du}}
\pgfpathlineto{\pgfpoint{15.534465\du}{9.876634\du}}
\pgfpathlineto{\pgfpoint{15.565498\du}{9.868237\du}}
\pgfpathlineto{\pgfpoint{15.597992\du}{9.859474\du}}
\pgfpathlineto{\pgfpoint{15.629390\du}{9.851442\du}}
\pgfpathlineto{\pgfpoint{15.662614\du}{9.843775\du}}
\pgfpathlineto{\pgfpoint{15.696568\du}{9.836108\du}}
\pgfpathlineto{\pgfpoint{15.730887\du}{9.828441\du}}
\pgfpathlineto{\pgfpoint{15.765571\du}{9.821504\du}}
\pgfpathlineto{\pgfpoint{15.801716\du}{9.814567\du}}
\pgfpathlineto{\pgfpoint{15.836765\du}{9.808726\du}}
\pgfpathlineto{\pgfpoint{15.874005\du}{9.802154\du}}
\pgfpathlineto{\pgfpoint{15.910515\du}{9.796313\du}}
\pgfpathlineto{\pgfpoint{15.947755\du}{9.790471\du}}
\pgfpathlineto{\pgfpoint{15.985360\du}{9.785360\du}}
\pgfpathlineto{\pgfpoint{16.023695\du}{9.779883\du}}
\pgfpathlineto{\pgfpoint{16.062760\du}{9.775137\du}}
\pgfpathlineto{\pgfpoint{16.102191\du}{9.771121\du}}
\pgfpathlineto{\pgfpoint{16.141621\du}{9.767105\du}}
\pgfpathlineto{\pgfpoint{16.181417\du}{9.763454\du}}
\pgfpathlineto{\pgfpoint{16.221577\du}{9.760168\du}}
\pgfpathlineto{\pgfpoint{16.262468\du}{9.757247\du}}
\pgfpathlineto{\pgfpoint{16.303724\du}{9.754326\du}}
\pgfpathlineto{\pgfpoint{16.344615\du}{9.751771\du}}
\pgfpathlineto{\pgfpoint{16.386601\du}{9.749580\du}}
\pgfpathlineto{\pgfpoint{16.428952\du}{9.748485\du}}
\pgfpathlineto{\pgfpoint{16.471668\du}{9.746659\du}}
\pgfpathlineto{\pgfpoint{16.514020\du}{9.745929\du}}
\pgfpathlineto{\pgfpoint{16.557466\du}{9.745564\du}}
\pgfpathlineto{\pgfpoint{16.600183\du}{9.745564\du}}
\pgfpathlineto{\pgfpoint{16.600183\du}{9.745564\du}}
\pgfpathlineto{\pgfpoint{16.600183\du}{9.745564\du}}
\pgfpathlineto{\pgfpoint{16.601278\du}{9.744834\du}}
\pgfpathlineto{\pgfpoint{16.602373\du}{9.744834\du}}
\pgfpathlineto{\pgfpoint{16.603834\du}{9.744834\du}}
\pgfpathlineto{\pgfpoint{16.604929\du}{9.744469\du}}
\pgfpathlineto{\pgfpoint{16.605294\du}{9.743739\du}}
\pgfpathlineto{\pgfpoint{16.606389\du}{9.743739\du}}
\pgfpathlineto{\pgfpoint{16.607119\du}{9.742643\du}}
\pgfpathlineto{\pgfpoint{16.608215\du}{9.741913\du}}
\pgfpathlineto{\pgfpoint{16.609310\du}{9.740818\du}}
\pgfpathlineto{\pgfpoint{16.610040\du}{9.738992\du}}
\pgfpathlineto{\pgfpoint{16.610040\du}{9.736802\du}}
\pgfpathlineto{\pgfpoint{16.610770\du}{9.734976\du}}
\pgfpathlineto{\pgfpoint{16.610040\du}{9.733151\du}}
\pgfpathlineto{\pgfpoint{16.610040\du}{9.731325\du}}
\pgfpathlineto{\pgfpoint{16.609310\du}{9.729500\du}}
\pgfpathlineto{\pgfpoint{16.608215\du}{9.728039\du}}
\pgfpathlineto{\pgfpoint{16.607119\du}{9.727309\du}}
\pgfpathlineto{\pgfpoint{16.606389\du}{9.726579\du}}
\pgfpathlineto{\pgfpoint{16.605294\du}{9.726214\du}}
\pgfpathlineto{\pgfpoint{16.604929\du}{9.725484\du}}
\pgfpathlineto{\pgfpoint{16.603834\du}{9.725119\du}}
\pgfpathlineto{\pgfpoint{16.602373\du}{9.725119\du}}
\pgfpathlineto{\pgfpoint{16.601278\du}{9.725119\du}}
\pgfpathlineto{\pgfpoint{16.600183\du}{9.725119\du}}
\pgfusepath{fill}
\pgfsetbuttcap
\pgfsetmiterjoin
\pgfsetdash{}{0pt}
\definecolor{dialinecolor}{rgb}{0.678431, 0.839216, 0.905882}
\pgfsetfillcolor{dialinecolor}
\pgfpathmoveto{\pgfpoint{18.290581\du}{10.312559\du}}
\pgfpathlineto{\pgfpoint{18.290581\du}{10.304162\du}}
\pgfpathlineto{\pgfpoint{18.289850\du}{10.296495\du}}
\pgfpathlineto{\pgfpoint{18.289120\du}{10.289193\du}}
\pgfpathlineto{\pgfpoint{18.288390\du}{10.280796\du}}
\pgfpathlineto{\pgfpoint{18.287295\du}{10.273129\du}}
\pgfpathlineto{\pgfpoint{18.285104\du}{10.265462\du}}
\pgfpathlineto{\pgfpoint{18.284009\du}{10.258160\du}}
\pgfpathlineto{\pgfpoint{18.281088\du}{10.250493\du}}
\pgfpathlineto{\pgfpoint{18.279263\du}{10.242826\du}}
\pgfpathlineto{\pgfpoint{18.276342\du}{10.235159\du}}
\pgfpathlineto{\pgfpoint{18.273786\du}{10.227492\du}}
\pgfpathlineto{\pgfpoint{18.270500\du}{10.220555\du}}
\pgfpathlineto{\pgfpoint{18.266484\du}{10.212888\du}}
\pgfpathlineto{\pgfpoint{18.263198\du}{10.205221\du}}
\pgfpathlineto{\pgfpoint{18.259182\du}{10.197919\du}}
\pgfpathlineto{\pgfpoint{18.255166\du}{10.190617\du}}
\pgfpathlineto{\pgfpoint{18.250785\du}{10.183315\du}}
\pgfpathlineto{\pgfpoint{18.246039\du}{10.176013\du}}
\pgfpathlineto{\pgfpoint{18.240562\du}{10.169076\du}}
\pgfpathlineto{\pgfpoint{18.235816\du}{10.162139\du}}
\pgfpathlineto{\pgfpoint{18.229974\du}{10.154472\du}}
\pgfpathlineto{\pgfpoint{18.224498\du}{10.147536\du}}
\pgfpathlineto{\pgfpoint{18.218656\du}{10.140599\du}}
\pgfpathlineto{\pgfpoint{18.212450\du}{10.133297\du}}
\pgfpathlineto{\pgfpoint{18.205878\du}{10.127090\du}}
\pgfpathlineto{\pgfpoint{18.198941\du}{10.119788\du}}
\pgfpathlineto{\pgfpoint{18.192369\du}{10.112851\du}}
\pgfpathlineto{\pgfpoint{18.185068\du}{10.106645\du}}
\pgfpathlineto{\pgfpoint{18.178496\du}{10.099343\du}}
\pgfpathlineto{\pgfpoint{18.170464\du}{10.093136\du}}
\pgfpathlineto{\pgfpoint{18.162432\du}{10.086564\du}}
\pgfpathlineto{\pgfpoint{18.155130\du}{10.079628\du}}
\pgfpathlineto{\pgfpoint{18.146367\du}{10.073056\du}}
\pgfpathlineto{\pgfpoint{18.138335\du}{10.066849\du}}
\pgfpathlineto{\pgfpoint{18.129208\du}{10.060277\du}}
\pgfpathlineto{\pgfpoint{18.120445\du}{10.053706\du}}
\pgfpathlineto{\pgfpoint{18.110953\du}{10.047499\du}}
\pgfpathlineto{\pgfpoint{18.101825\du}{10.040927\du}}
\pgfpathlineto{\pgfpoint{18.092698\du}{10.034356\du}}
\pgfpathlineto{\pgfpoint{18.082840\du}{10.028514\du}}
\pgfpathlineto{\pgfpoint{18.072983\du}{10.022307\du}}
\pgfpathlineto{\pgfpoint{18.062760\du}{10.015736\du}}
\pgfpathlineto{\pgfpoint{18.052172\du}{10.009894\du}}
\pgfpathlineto{\pgfpoint{18.042315\du}{10.004053\du}}
\pgfpathlineto{\pgfpoint{18.031727\du}{9.998211\du}}
\pgfpathlineto{\pgfpoint{18.020409\du}{9.991639\du}}
\pgfpathlineto{\pgfpoint{18.009091\du}{9.986528\du}}
\pgfpathlineto{\pgfpoint{17.997408\du}{9.980686\du}}
\pgfpathlineto{\pgfpoint{17.986455\du}{9.974845\du}}
\pgfpathlineto{\pgfpoint{17.974407\du}{9.969003\du}}
\pgfpathlineto{\pgfpoint{17.963089\du}{9.963162\du}}
\pgfpathlineto{\pgfpoint{17.950675\du}{9.958050\du}}
\pgfpathlineto{\pgfpoint{17.938262\du}{9.951844\du}}
\pgfpathlineto{\pgfpoint{17.925484\du}{9.946002\du}}
\pgfpathlineto{\pgfpoint{17.913436\du}{9.940891\du}}
\pgfpathlineto{\pgfpoint{17.900657\du}{9.935779\du}}
\pgfpathlineto{\pgfpoint{17.887149\du}{9.930303\du}}
\pgfpathlineto{\pgfpoint{17.874005\du}{9.925192\du}}
\pgfpathlineto{\pgfpoint{17.860862\du}{9.919715\du}}
\pgfpathlineto{\pgfpoint{17.846623\du}{9.914604\du}}
\pgfpathlineto{\pgfpoint{17.833479\du}{9.909127\du}}
\pgfpathlineto{\pgfpoint{17.819241\du}{9.904746\du}}
\pgfpathlineto{\pgfpoint{17.805367\du}{9.899270\du}}
\pgfpathlineto{\pgfpoint{17.791128\du}{9.894524\du}}
\pgfpathlineto{\pgfpoint{17.777254\du}{9.889412\du}}
\pgfpathlineto{\pgfpoint{17.762286\du}{9.884666\du}}
\pgfpathlineto{\pgfpoint{17.747317\du}{9.879920\du}}
\pgfpathlineto{\pgfpoint{17.733078\du}{9.875539\du}}
\pgfpathlineto{\pgfpoint{17.702410\du}{9.866046\du}}
\pgfpathlineto{\pgfpoint{17.671742\du}{9.857284\du}}
\pgfpathlineto{\pgfpoint{17.640343\du}{9.848521\du}}
\pgfpathlineto{\pgfpoint{17.608580\du}{9.839759\du}}
\pgfpathlineto{\pgfpoint{17.575721\du}{9.831362\du}}
\pgfpathlineto{\pgfpoint{17.542132\du}{9.823695\du}}
\pgfpathlineto{\pgfpoint{17.508178\du}{9.815663\du}}
\pgfpathlineto{\pgfpoint{17.473494\du}{9.808726\du}}
\pgfpathlineto{\pgfpoint{17.438810\du}{9.801789\du}}
\pgfpathlineto{\pgfpoint{17.403395\du}{9.794487\du}}
\pgfpathlineto{\pgfpoint{17.366886\du}{9.788280\du}}
\pgfpathlineto{\pgfpoint{17.330376\du}{9.781709\du}}
\pgfpathlineto{\pgfpoint{17.293501\du}{9.775867\du}}
\pgfpathlineto{\pgfpoint{17.256261\du}{9.770026\du}}
\pgfpathlineto{\pgfpoint{17.217196\du}{9.764914\du}}
\pgfpathlineto{\pgfpoint{17.179226\du}{9.760168\du}}
\pgfpathlineto{\pgfpoint{17.139796\du}{9.755422\du}}
\pgfpathlineto{\pgfpoint{17.101095\du}{9.750675\du}}
\pgfpathlineto{\pgfpoint{17.060935\du}{9.746659\du}}
\pgfpathlineto{\pgfpoint{17.021139\du}{9.742643\du}}
\pgfpathlineto{\pgfpoint{16.980248\du}{9.739723\du}}
\pgfpathlineto{\pgfpoint{16.939723\du}{9.736802\du}}
\pgfpathlineto{\pgfpoint{16.898101\du}{9.733881\du}}
\pgfpathlineto{\pgfpoint{16.856480\du}{9.731325\du}}
\pgfpathlineto{\pgfpoint{16.814859\du}{9.729500\du}}
\pgfpathlineto{\pgfpoint{16.772143\du}{9.728039\du}}
\pgfpathlineto{\pgfpoint{16.730157\du}{9.726579\du}}
\pgfpathlineto{\pgfpoint{16.687076\du}{9.725484\du}}
\pgfpathlineto{\pgfpoint{16.643629\du}{9.725119\du}}
\pgfpathlineto{\pgfpoint{16.600183\du}{9.725119\du}}
\pgfpathlineto{\pgfpoint{16.600183\du}{9.745564\du}}
\pgfpathlineto{\pgfpoint{16.643264\du}{9.745564\du}}
\pgfpathlineto{\pgfpoint{16.686710\du}{9.745929\du}}
\pgfpathlineto{\pgfpoint{16.728697\du}{9.746659\du}}
\pgfpathlineto{\pgfpoint{16.771413\du}{9.748485\du}}
\pgfpathlineto{\pgfpoint{16.814129\du}{9.749580\du}}
\pgfpathlineto{\pgfpoint{16.855750\du}{9.751771\du}}
\pgfpathlineto{\pgfpoint{16.897006\du}{9.754326\du}}
\pgfpathlineto{\pgfpoint{16.937897\du}{9.757247\du}}
\pgfpathlineto{\pgfpoint{16.979153\du}{9.760168\du}}
\pgfpathlineto{\pgfpoint{17.019314\du}{9.763454\du}}
\pgfpathlineto{\pgfpoint{17.059474\du}{9.767105\du}}
\pgfpathlineto{\pgfpoint{17.098540\du}{9.771121\du}}
\pgfpathlineto{\pgfpoint{17.137970\du}{9.775137\du}}
\pgfpathlineto{\pgfpoint{17.176670\du}{9.779883\du}}
\pgfpathlineto{\pgfpoint{17.215371\du}{9.785360\du}}
\pgfpathlineto{\pgfpoint{17.252976\du}{9.790471\du}}
\pgfpathlineto{\pgfpoint{17.290215\du}{9.796313\du}}
\pgfpathlineto{\pgfpoint{17.326725\du}{9.802154\du}}
\pgfpathlineto{\pgfpoint{17.363600\du}{9.808726\du}}
\pgfpathlineto{\pgfpoint{17.399014\du}{9.814567\du}}
\pgfpathlineto{\pgfpoint{17.434794\du}{9.821504\du}}
\pgfpathlineto{\pgfpoint{17.469478\du}{9.828441\du}}
\pgfpathlineto{\pgfpoint{17.503432\du}{9.836108\du}}
\pgfpathlineto{\pgfpoint{17.537751\du}{9.843775\du}}
\pgfpathlineto{\pgfpoint{17.570975\du}{9.851442\du}}
\pgfpathlineto{\pgfpoint{17.603103\du}{9.859474\du}}
\pgfpathlineto{\pgfpoint{17.635232\du}{9.868237\du}}
\pgfpathlineto{\pgfpoint{17.666630\du}{9.876634\du}}
\pgfpathlineto{\pgfpoint{17.697298\du}{9.885761\du}}
\pgfpathlineto{\pgfpoint{17.727236\du}{9.894524\du}}
\pgfpathlineto{\pgfpoint{17.741475\du}{9.899270\du}}
\pgfpathlineto{\pgfpoint{17.756079\du}{9.904016\du}}
\pgfpathlineto{\pgfpoint{17.769953\du}{9.909127\du}}
\pgfpathlineto{\pgfpoint{17.784556\du}{9.913874\du}}
\pgfpathlineto{\pgfpoint{17.798795\du}{9.918620\du}}
\pgfpathlineto{\pgfpoint{17.812669\du}{9.924096\du}}
\pgfpathlineto{\pgfpoint{17.826908\du}{9.928478\du}}
\pgfpathlineto{\pgfpoint{17.840051\du}{9.933954\du}}
\pgfpathlineto{\pgfpoint{17.853560\du}{9.939065\du}}
\pgfpathlineto{\pgfpoint{17.866703\du}{9.943812\du}}
\pgfpathlineto{\pgfpoint{17.879482\du}{9.949653\du}}
\pgfpathlineto{\pgfpoint{17.892260\du}{9.954765\du}}
\pgfpathlineto{\pgfpoint{17.905403\du}{9.960241\du}}
\pgfpathlineto{\pgfpoint{17.917452\du}{9.965352\du}}
\pgfpathlineto{\pgfpoint{17.929865\du}{9.970829\du}}
\pgfpathlineto{\pgfpoint{17.941913\du}{9.976670\du}}
\pgfpathlineto{\pgfpoint{17.954326\du}{9.981782\du}}
\pgfpathlineto{\pgfpoint{17.965644\du}{9.987258\du}}
\pgfpathlineto{\pgfpoint{17.977693\du}{9.993100\du}}
\pgfpathlineto{\pgfpoint{17.988280\du}{9.998941\du}}
\pgfpathlineto{\pgfpoint{18.000329\du}{10.004783\du}}
\pgfpathlineto{\pgfpoint{18.010551\du}{10.010624\du}}
\pgfpathlineto{\pgfpoint{18.021504\du}{10.015736\du}}
\pgfpathlineto{\pgfpoint{18.032457\du}{10.022307\du}}
\pgfpathlineto{\pgfpoint{18.042315\du}{10.028149\du}}
\pgfpathlineto{\pgfpoint{18.052172\du}{10.033991\du}}
\pgfpathlineto{\pgfpoint{18.062395\du}{10.039832\du}}
\pgfpathlineto{\pgfpoint{18.071522\du}{10.045674\du}}
\pgfpathlineto{\pgfpoint{18.081745\du}{10.052245\du}}
\pgfpathlineto{\pgfpoint{18.091238\du}{10.058087\du}}
\pgfpathlineto{\pgfpoint{18.100000\du}{10.064294\du}}
\pgfpathlineto{\pgfpoint{18.108032\du}{10.070135\du}}
\pgfpathlineto{\pgfpoint{18.116794\du}{10.076707\du}}
\pgfpathlineto{\pgfpoint{18.125192\du}{10.082913\du}}
\pgfpathlineto{\pgfpoint{18.133954\du}{10.089485\du}}
\pgfpathlineto{\pgfpoint{18.141621\du}{10.096057\du}}
\pgfpathlineto{\pgfpoint{18.149653\du}{10.101899\du}}
\pgfpathlineto{\pgfpoint{18.156955\du}{10.108105\du}}
\pgfpathlineto{\pgfpoint{18.164622\du}{10.115407\du}}
\pgfpathlineto{\pgfpoint{18.171194\du}{10.121249\du}}
\pgfpathlineto{\pgfpoint{18.178496\du}{10.128185\du}}
\pgfpathlineto{\pgfpoint{18.184337\du}{10.134757\du}}
\pgfpathlineto{\pgfpoint{18.191274\du}{10.140964\du}}
\pgfpathlineto{\pgfpoint{18.197846\du}{10.147536\du}}
\pgfpathlineto{\pgfpoint{18.202957\du}{10.154472\du}}
\pgfpathlineto{\pgfpoint{18.208434\du}{10.161044\du}}
\pgfpathlineto{\pgfpoint{18.214640\du}{10.167981\du}}
\pgfpathlineto{\pgfpoint{18.219387\du}{10.174188\du}}
\pgfpathlineto{\pgfpoint{18.224498\du}{10.180759\du}}
\pgfpathlineto{\pgfpoint{18.229244\du}{10.187696\du}}
\pgfpathlineto{\pgfpoint{18.233625\du}{10.194268\du}}
\pgfpathlineto{\pgfpoint{18.238007\du}{10.201205\du}}
\pgfpathlineto{\pgfpoint{18.240927\du}{10.207777\du}}
\pgfpathlineto{\pgfpoint{18.244943\du}{10.214713\du}}
\pgfpathlineto{\pgfpoint{18.247864\du}{10.221650\du}}
\pgfpathlineto{\pgfpoint{18.251880\du}{10.228952\du}}
\pgfpathlineto{\pgfpoint{18.254436\du}{10.235159\du}}
\pgfpathlineto{\pgfpoint{18.257357\du}{10.242096\du}}
\pgfpathlineto{\pgfpoint{18.259912\du}{10.249398\du}}
\pgfpathlineto{\pgfpoint{18.262103\du}{10.256334\du}}
\pgfpathlineto{\pgfpoint{18.263563\du}{10.263271\du}}
\pgfpathlineto{\pgfpoint{18.265754\du}{10.270208\du}}
\pgfpathlineto{\pgfpoint{18.266484\du}{10.277510\du}}
\pgfpathlineto{\pgfpoint{18.267579\du}{10.283717\du}}
\pgfpathlineto{\pgfpoint{18.269405\du}{10.291384\du}}
\pgfpathlineto{\pgfpoint{18.269770\du}{10.298321\du}}
\pgfpathlineto{\pgfpoint{18.269770\du}{10.305257\du}}
\pgfpathlineto{\pgfpoint{18.270500\du}{10.312559\du}}
\pgfpathlineto{\pgfpoint{18.290581\du}{10.312559\du}}
\pgfusepath{fill}
\pgfsetbuttcap
\pgfsetmiterjoin
\pgfsetdash{}{0pt}
\definecolor{dialinecolor}{rgb}{0.027451, 0.486275, 0.682353}
\pgfsetfillcolor{dialinecolor}
\pgfpathmoveto{\pgfpoint{14.914531\du}{9.503140\du}}
\pgfpathlineto{\pgfpoint{14.914531\du}{10.327528\du}}
\pgfpathlineto{\pgfpoint{18.279993\du}{10.327528\du}}
\pgfpathlineto{\pgfpoint{18.280723\du}{9.503870\du}}
\pgfpathlineto{\pgfpoint{14.914531\du}{9.503140\du}}
\pgfusepath{fill}
\pgfsetbuttcap
\pgfsetmiterjoin
\pgfsetdash{}{0pt}
\definecolor{dialinecolor}{rgb}{0.235294, 0.686275, 0.894118}
\pgfsetfillcolor{dialinecolor}
\pgfpathmoveto{\pgfpoint{18.279993\du}{9.487441\du}}
\pgfpathlineto{\pgfpoint{18.278532\du}{9.517379\du}}
\pgfpathlineto{\pgfpoint{18.271230\du}{9.546586\du}}
\pgfpathlineto{\pgfpoint{18.261008\du}{9.575794\du}}
\pgfpathlineto{\pgfpoint{18.246404\du}{9.603907\du}}
\pgfpathlineto{\pgfpoint{18.227054\du}{9.632019\du}}
\pgfpathlineto{\pgfpoint{18.205148\du}{9.659401\du}}
\pgfpathlineto{\pgfpoint{18.178496\du}{9.685688\du}}
\pgfpathlineto{\pgfpoint{18.147828\du}{9.711975\du}}
\pgfpathlineto{\pgfpoint{18.114969\du}{9.737897\du}}
\pgfpathlineto{\pgfpoint{18.077729\du}{9.763089\du}}
\pgfpathlineto{\pgfpoint{18.036838\du}{9.787185\du}}
\pgfpathlineto{\pgfpoint{17.993027\du}{9.810551\du}}
\pgfpathlineto{\pgfpoint{17.946659\du}{9.832822\du}}
\pgfpathlineto{\pgfpoint{17.896276\du}{9.854728\du}}
\pgfpathlineto{\pgfpoint{17.843337\du}{9.875904\du}}
\pgfpathlineto{\pgfpoint{17.787842\du}{9.895984\du}}
\pgfpathlineto{\pgfpoint{17.729792\du}{9.914604\du}}
\pgfpathlineto{\pgfpoint{17.669186\du}{9.933224\du}}
\pgfpathlineto{\pgfpoint{17.605294\du}{9.950383\du}}
\pgfpathlineto{\pgfpoint{17.540307\du}{9.966083\du}}
\pgfpathlineto{\pgfpoint{17.471668\du}{9.981417\du}}
\pgfpathlineto{\pgfpoint{17.400840\du}{9.995290\du}}
\pgfpathlineto{\pgfpoint{17.328916\du}{10.008069\du}}
\pgfpathlineto{\pgfpoint{17.254071\du}{10.019387\du}}
\pgfpathlineto{\pgfpoint{17.178131\du}{10.029974\du}}
\pgfpathlineto{\pgfpoint{17.099635\du}{10.039102\du}}
\pgfpathlineto{\pgfpoint{17.020044\du}{10.046769\du}}
\pgfpathlineto{\pgfpoint{16.938992\du}{10.053341\du}}
\pgfpathlineto{\pgfpoint{16.855750\du}{10.058452\du}}
\pgfpathlineto{\pgfpoint{16.772143\du}{10.062103\du}}
\pgfpathlineto{\pgfpoint{16.686710\du}{10.063928\du}}
\pgfpathlineto{\pgfpoint{16.600183\du}{10.065024\du}}
\pgfpathlineto{\pgfpoint{16.514020\du}{10.063928\du}}
\pgfpathlineto{\pgfpoint{16.428222\du}{10.062103\du}}
\pgfpathlineto{\pgfpoint{16.344615\du}{10.058452\du}}
\pgfpathlineto{\pgfpoint{16.261738\du}{10.053341\du}}
\pgfpathlineto{\pgfpoint{16.180321\du}{10.046769\du}}
\pgfpathlineto{\pgfpoint{16.100730\du}{10.039102\du}}
\pgfpathlineto{\pgfpoint{16.022965\du}{10.029974\du}}
\pgfpathlineto{\pgfpoint{15.946294\du}{10.019387\du}}
\pgfpathlineto{\pgfpoint{15.872180\du}{10.008069\du}}
\pgfpathlineto{\pgfpoint{15.799525\du}{9.995290\du}}
\pgfpathlineto{\pgfpoint{15.729062\du}{9.981417\du}}
\pgfpathlineto{\pgfpoint{15.660424\du}{9.966083\du}}
\pgfpathlineto{\pgfpoint{15.594706\du}{9.950383\du}}
\pgfpathlineto{\pgfpoint{15.531179\du}{9.933224\du}}
\pgfpathlineto{\pgfpoint{15.470208\du}{9.914604\du}}
\pgfpathlineto{\pgfpoint{15.411793\du}{9.895984\du}}
\pgfpathlineto{\pgfpoint{15.356663\du}{9.875904\du}}
\pgfpathlineto{\pgfpoint{15.303724\du}{9.854728\du}}
\pgfpathlineto{\pgfpoint{15.253706\du}{9.832822\du}}
\pgfpathlineto{\pgfpoint{15.206608\du}{9.810551\du}}
\pgfpathlineto{\pgfpoint{15.163162\du}{9.787185\du}}
\pgfpathlineto{\pgfpoint{15.122271\du}{9.763089\du}}
\pgfpathlineto{\pgfpoint{15.085031\du}{9.737897\du}}
\pgfpathlineto{\pgfpoint{15.051807\du}{9.711975\du}}
\pgfpathlineto{\pgfpoint{15.021504\du}{9.685688\du}}
\pgfpathlineto{\pgfpoint{14.994852\du}{9.659401\du}}
\pgfpathlineto{\pgfpoint{14.972946\du}{9.632019\du}}
\pgfpathlineto{\pgfpoint{14.953596\du}{9.603907\du}}
\pgfpathlineto{\pgfpoint{14.938992\du}{9.575794\du}}
\pgfpathlineto{\pgfpoint{14.928405\du}{9.546586\du}}
\pgfpathlineto{\pgfpoint{14.921468\du}{9.517379\du}}
\pgfpathlineto{\pgfpoint{14.919642\du}{9.487441\du}}
\pgfpathlineto{\pgfpoint{14.921468\du}{9.458233\du}}
\pgfpathlineto{\pgfpoint{14.928405\du}{9.428295\du}}
\pgfpathlineto{\pgfpoint{14.938992\du}{9.399817\du}}
\pgfpathlineto{\pgfpoint{14.953596\du}{9.371705\du}}
\pgfpathlineto{\pgfpoint{14.972946\du}{9.343593\du}}
\pgfpathlineto{\pgfpoint{14.994852\du}{9.315845\du}}
\pgfpathlineto{\pgfpoint{15.021504\du}{9.289193\du}}
\pgfpathlineto{\pgfpoint{15.051807\du}{9.262906\du}}
\pgfpathlineto{\pgfpoint{15.085031\du}{9.237714\du}}
\pgfpathlineto{\pgfpoint{15.122271\du}{9.212523\du}}
\pgfpathlineto{\pgfpoint{15.163162\du}{9.188426\du}}
\pgfpathlineto{\pgfpoint{15.206608\du}{9.165060\du}}
\pgfpathlineto{\pgfpoint{15.253706\du}{9.142059\du}}
\pgfpathlineto{\pgfpoint{15.303724\du}{9.120518\du}}
\pgfpathlineto{\pgfpoint{15.356663\du}{9.099343\du}}
\pgfpathlineto{\pgfpoint{15.411793\du}{9.079628\du}}
\pgfpathlineto{\pgfpoint{15.470208\du}{9.060277\du}}
\pgfpathlineto{\pgfpoint{15.531179\du}{9.042023\du}}
\pgfpathlineto{\pgfpoint{15.594706\du}{9.025228\du}}
\pgfpathlineto{\pgfpoint{15.660424\du}{9.008799\du}}
\pgfpathlineto{\pgfpoint{15.729062\du}{8.993465\du}}
\pgfpathlineto{\pgfpoint{15.799525\du}{8.979956\du}}
\pgfpathlineto{\pgfpoint{15.872180\du}{8.967178\du}}
\pgfpathlineto{\pgfpoint{15.946294\du}{8.955495\du}}
\pgfpathlineto{\pgfpoint{16.022965\du}{8.945637\du}}
\pgfpathlineto{\pgfpoint{16.100730\du}{8.936145\du}}
\pgfpathlineto{\pgfpoint{16.180321\du}{8.928478\du}}
\pgfpathlineto{\pgfpoint{16.261738\du}{8.921906\du}}
\pgfpathlineto{\pgfpoint{16.344615\du}{8.916794\du}}
\pgfpathlineto{\pgfpoint{16.428222\du}{8.913509\du}}
\pgfpathlineto{\pgfpoint{16.514020\du}{8.910953\du}}
\pgfpathlineto{\pgfpoint{16.600183\du}{8.910588\du}}
\pgfpathlineto{\pgfpoint{16.686710\du}{8.910953\du}}
\pgfpathlineto{\pgfpoint{16.772143\du}{8.913509\du}}
\pgfpathlineto{\pgfpoint{16.855750\du}{8.916794\du}}
\pgfpathlineto{\pgfpoint{16.938992\du}{8.921906\du}}
\pgfpathlineto{\pgfpoint{17.020044\du}{8.928478\du}}
\pgfpathlineto{\pgfpoint{17.099635\du}{8.936145\du}}
\pgfpathlineto{\pgfpoint{17.178131\du}{8.945637\du}}
\pgfpathlineto{\pgfpoint{17.254071\du}{8.955495\du}}
\pgfpathlineto{\pgfpoint{17.328916\du}{8.967178\du}}
\pgfpathlineto{\pgfpoint{17.400840\du}{8.979956\du}}
\pgfpathlineto{\pgfpoint{17.471668\du}{8.993465\du}}
\pgfpathlineto{\pgfpoint{17.540307\du}{9.008799\du}}
\pgfpathlineto{\pgfpoint{17.605294\du}{9.025228\du}}
\pgfpathlineto{\pgfpoint{17.669186\du}{9.042023\du}}
\pgfpathlineto{\pgfpoint{17.729792\du}{9.060277\du}}
\pgfpathlineto{\pgfpoint{17.787842\du}{9.079628\du}}
\pgfpathlineto{\pgfpoint{17.843337\du}{9.099343\du}}
\pgfpathlineto{\pgfpoint{17.896276\du}{9.120518\du}}
\pgfpathlineto{\pgfpoint{17.946659\du}{9.142059\du}}
\pgfpathlineto{\pgfpoint{17.993027\du}{9.165060\du}}
\pgfpathlineto{\pgfpoint{18.036838\du}{9.188426\du}}
\pgfpathlineto{\pgfpoint{18.077729\du}{9.212523\du}}
\pgfpathlineto{\pgfpoint{18.114969\du}{9.237714\du}}
\pgfpathlineto{\pgfpoint{18.147828\du}{9.262906\du}}
\pgfpathlineto{\pgfpoint{18.178496\du}{9.289193\du}}
\pgfpathlineto{\pgfpoint{18.205148\du}{9.315845\du}}
\pgfpathlineto{\pgfpoint{18.227054\du}{9.343593\du}}
\pgfpathlineto{\pgfpoint{18.246404\du}{9.371705\du}}
\pgfpathlineto{\pgfpoint{18.261008\du}{9.399817\du}}
\pgfpathlineto{\pgfpoint{18.271230\du}{9.428295\du}}
\pgfpathlineto{\pgfpoint{18.278532\du}{9.458233\du}}
\pgfpathlineto{\pgfpoint{18.279993\du}{9.487441\du}}
\pgfusepath{fill}
\pgfsetbuttcap
\pgfsetmiterjoin
\pgfsetdash{}{0pt}
\definecolor{dialinecolor}{rgb}{0.678431, 0.839216, 0.905882}
\pgfsetfillcolor{dialinecolor}
\pgfpathmoveto{\pgfpoint{16.600183\du}{10.074881\du}}
\pgfpathlineto{\pgfpoint{16.600183\du}{10.074881\du}}
\pgfpathlineto{\pgfpoint{16.643629\du}{10.074881\du}}
\pgfpathlineto{\pgfpoint{16.687076\du}{10.074151\du}}
\pgfpathlineto{\pgfpoint{16.730157\du}{10.073056\du}}
\pgfpathlineto{\pgfpoint{16.772143\du}{10.071961\du}}
\pgfpathlineto{\pgfpoint{16.814859\du}{10.070135\du}}
\pgfpathlineto{\pgfpoint{16.856480\du}{10.068310\du}}
\pgfpathlineto{\pgfpoint{16.898101\du}{10.066119\du}}
\pgfpathlineto{\pgfpoint{16.939723\du}{10.063198\du}}
\pgfpathlineto{\pgfpoint{16.980248\du}{10.060277\du}}
\pgfpathlineto{\pgfpoint{17.021139\du}{10.057357\du}}
\pgfpathlineto{\pgfpoint{17.060935\du}{10.053341\du}}
\pgfpathlineto{\pgfpoint{17.101095\du}{10.049325\du}}
\pgfpathlineto{\pgfpoint{17.139796\du}{10.044578\du}}
\pgfpathlineto{\pgfpoint{17.179226\du}{10.039832\du}}
\pgfpathlineto{\pgfpoint{17.217196\du}{10.035086\du}}
\pgfpathlineto{\pgfpoint{17.256261\du}{10.029974\du}}
\pgfpathlineto{\pgfpoint{17.293501\du}{10.024133\du}}
\pgfpathlineto{\pgfpoint{17.330376\du}{10.018291\du}}
\pgfpathlineto{\pgfpoint{17.366886\du}{10.011720\du}}
\pgfpathlineto{\pgfpoint{17.403395\du}{10.005148\du}}
\pgfpathlineto{\pgfpoint{17.438810\du}{9.998211\du}}
\pgfpathlineto{\pgfpoint{17.473494\du}{9.991274\du}}
\pgfpathlineto{\pgfpoint{17.508178\du}{9.984337\du}}
\pgfpathlineto{\pgfpoint{17.542132\du}{9.976670\du}}
\pgfpathlineto{\pgfpoint{17.575721\du}{9.968273\du}}
\pgfpathlineto{\pgfpoint{17.608580\du}{9.960241\du}}
\pgfpathlineto{\pgfpoint{17.640343\du}{9.951844\du}}
\pgfpathlineto{\pgfpoint{17.671742\du}{9.942716\du}}
\pgfpathlineto{\pgfpoint{17.687076\du}{9.938700\du}}
\pgfpathlineto{\pgfpoint{17.702410\du}{9.933954\du}}
\pgfpathlineto{\pgfpoint{17.718474\du}{9.929208\du}}
\pgfpathlineto{\pgfpoint{17.733078\du}{9.924461\du}}
\pgfpathlineto{\pgfpoint{17.747317\du}{9.919715\du}}
\pgfpathlineto{\pgfpoint{17.762286\du}{9.915334\du}}
\pgfpathlineto{\pgfpoint{17.777254\du}{9.910588\du}}
\pgfpathlineto{\pgfpoint{17.791128\du}{9.905111\du}}
\pgfpathlineto{\pgfpoint{17.805367\du}{9.900365\du}}
\pgfpathlineto{\pgfpoint{17.819241\du}{9.895254\du}}
\pgfpathlineto{\pgfpoint{17.833479\du}{9.890507\du}}
\pgfpathlineto{\pgfpoint{17.846623\du}{9.885396\du}}
\pgfpathlineto{\pgfpoint{17.860862\du}{9.879920\du}}
\pgfpathlineto{\pgfpoint{17.874005\du}{9.874808\du}}
\pgfpathlineto{\pgfpoint{17.887149\du}{9.869697\du}}
\pgfpathlineto{\pgfpoint{17.900657\du}{9.864221\du}}
\pgfpathlineto{\pgfpoint{17.913436\du}{9.859109\du}}
\pgfpathlineto{\pgfpoint{17.925484\du}{9.853633\du}}
\pgfpathlineto{\pgfpoint{17.938262\du}{9.847791\du}}
\pgfpathlineto{\pgfpoint{17.950675\du}{9.841950\du}}
\pgfpathlineto{\pgfpoint{17.963089\du}{9.836838\du}}
\pgfpathlineto{\pgfpoint{17.974407\du}{9.830997\du}}
\pgfpathlineto{\pgfpoint{17.986455\du}{9.825155\du}}
\pgfpathlineto{\pgfpoint{17.997408\du}{9.819314\du}}
\pgfpathlineto{\pgfpoint{18.009091\du}{9.813837\du}}
\pgfpathlineto{\pgfpoint{18.020409\du}{9.807996\du}}
\pgfpathlineto{\pgfpoint{18.031727\du}{9.801789\du}}
\pgfpathlineto{\pgfpoint{18.042315\du}{9.795947\du}}
\pgfpathlineto{\pgfpoint{18.052172\du}{9.790106\du}}
\pgfpathlineto{\pgfpoint{18.062760\du}{9.784264\du}}
\pgfpathlineto{\pgfpoint{18.072983\du}{9.777693\du}}
\pgfpathlineto{\pgfpoint{18.082840\du}{9.771121\du}}
\pgfpathlineto{\pgfpoint{18.092698\du}{9.765279\du}}
\pgfpathlineto{\pgfpoint{18.101825\du}{9.759073\du}}
\pgfpathlineto{\pgfpoint{18.110953\du}{9.752501\du}}
\pgfpathlineto{\pgfpoint{18.120445\du}{9.745929\du}}
\pgfpathlineto{\pgfpoint{18.129208\du}{9.739723\du}}
\pgfpathlineto{\pgfpoint{18.138335\du}{9.733151\du}}
\pgfpathlineto{\pgfpoint{18.146367\du}{9.726579\du}}
\pgfpathlineto{\pgfpoint{18.155130\du}{9.720372\du}}
\pgfpathlineto{\pgfpoint{18.162432\du}{9.713801\du}}
\pgfpathlineto{\pgfpoint{18.170464\du}{9.706864\du}}
\pgfpathlineto{\pgfpoint{18.178496\du}{9.700292\du}}
\pgfpathlineto{\pgfpoint{18.185068\du}{9.693355\du}}
\pgfpathlineto{\pgfpoint{18.192369\du}{9.686783\du}}
\pgfpathlineto{\pgfpoint{18.198941\du}{9.679847\du}}
\pgfpathlineto{\pgfpoint{18.205878\du}{9.672910\du}}
\pgfpathlineto{\pgfpoint{18.212450\du}{9.666338\du}}
\pgfpathlineto{\pgfpoint{18.218656\du}{9.659401\du}}
\pgfpathlineto{\pgfpoint{18.224498\du}{9.652464\du}}
\pgfpathlineto{\pgfpoint{18.229974\du}{9.645528\du}}
\pgfpathlineto{\pgfpoint{18.235816\du}{9.637861\du}}
\pgfpathlineto{\pgfpoint{18.240562\du}{9.630924\du}}
\pgfpathlineto{\pgfpoint{18.246039\du}{9.623622\du}}
\pgfpathlineto{\pgfpoint{18.250785\du}{9.616685\du}}
\pgfpathlineto{\pgfpoint{18.255166\du}{9.609018\du}}
\pgfpathlineto{\pgfpoint{18.259182\du}{9.602081\du}}
\pgfpathlineto{\pgfpoint{18.263198\du}{9.594414\du}}
\pgfpathlineto{\pgfpoint{18.266484\du}{9.586747\du}}
\pgfpathlineto{\pgfpoint{18.270500\du}{9.579445\du}}
\pgfpathlineto{\pgfpoint{18.273786\du}{9.572143\du}}
\pgfpathlineto{\pgfpoint{18.276342\du}{9.564841\du}}
\pgfpathlineto{\pgfpoint{18.279263\du}{9.557174\du}}
\pgfpathlineto{\pgfpoint{18.281088\du}{9.549507\du}}
\pgfpathlineto{\pgfpoint{18.284009\du}{9.541840\du}}
\pgfpathlineto{\pgfpoint{18.285104\du}{9.534173\du}}
\pgfpathlineto{\pgfpoint{18.287295\du}{9.526506\du}}
\pgfpathlineto{\pgfpoint{18.288390\du}{9.519204\du}}
\pgfpathlineto{\pgfpoint{18.289120\du}{9.510807\du}}
\pgfpathlineto{\pgfpoint{18.289850\du}{9.503140\du}}
\pgfpathlineto{\pgfpoint{18.290581\du}{9.495473\du}}
\pgfpathlineto{\pgfpoint{18.290581\du}{9.487441\du}}
\pgfpathlineto{\pgfpoint{18.270500\du}{9.487441\du}}
\pgfpathlineto{\pgfpoint{18.269770\du}{9.494378\du}}
\pgfpathlineto{\pgfpoint{18.269770\du}{9.502045\du}}
\pgfpathlineto{\pgfpoint{18.269405\du}{9.508981\du}}
\pgfpathlineto{\pgfpoint{18.267579\du}{9.515918\du}}
\pgfpathlineto{\pgfpoint{18.266484\du}{9.523220\du}}
\pgfpathlineto{\pgfpoint{18.265754\du}{9.529427\du}}
\pgfpathlineto{\pgfpoint{18.263563\du}{9.536729\du}}
\pgfpathlineto{\pgfpoint{18.262103\du}{9.543666\du}}
\pgfpathlineto{\pgfpoint{18.259912\du}{9.550602\du}}
\pgfpathlineto{\pgfpoint{18.257357\du}{9.557539\du}}
\pgfpathlineto{\pgfpoint{18.254436\du}{9.564841\du}}
\pgfpathlineto{\pgfpoint{18.251880\du}{9.571048\du}}
\pgfpathlineto{\pgfpoint{18.247864\du}{9.577985\du}}
\pgfpathlineto{\pgfpoint{18.244943\du}{9.585287\du}}
\pgfpathlineto{\pgfpoint{18.240927\du}{9.592223\du}}
\pgfpathlineto{\pgfpoint{18.238007\du}{9.598430\du}}
\pgfpathlineto{\pgfpoint{18.233625\du}{9.605732\du}}
\pgfpathlineto{\pgfpoint{18.229244\du}{9.611939\du}}
\pgfpathlineto{\pgfpoint{18.224498\du}{9.619241\du}}
\pgfpathlineto{\pgfpoint{18.219387\du}{9.625447\du}}
\pgfpathlineto{\pgfpoint{18.214640\du}{9.632384\du}}
\pgfpathlineto{\pgfpoint{18.208434\du}{9.638956\du}}
\pgfpathlineto{\pgfpoint{18.202957\du}{9.645528\du}}
\pgfpathlineto{\pgfpoint{18.197846\du}{9.652464\du}}
\pgfpathlineto{\pgfpoint{18.191274\du}{9.659036\du}}
\pgfpathlineto{\pgfpoint{18.184337\du}{9.665243\du}}
\pgfpathlineto{\pgfpoint{18.178496\du}{9.671815\du}}
\pgfpathlineto{\pgfpoint{18.171194\du}{9.678751\du}}
\pgfpathlineto{\pgfpoint{18.164622\du}{9.685323\du}}
\pgfpathlineto{\pgfpoint{18.156955\du}{9.691530\du}}
\pgfpathlineto{\pgfpoint{18.149653\du}{9.698101\du}}
\pgfpathlineto{\pgfpoint{18.141621\du}{9.703943\du}}
\pgfpathlineto{\pgfpoint{18.133954\du}{9.710515\du}}
\pgfpathlineto{\pgfpoint{18.125192\du}{9.716721\du}}
\pgfpathlineto{\pgfpoint{18.116794\du}{9.723293\du}}
\pgfpathlineto{\pgfpoint{18.108032\du}{9.729500\du}}
\pgfpathlineto{\pgfpoint{18.100000\du}{9.735341\du}}
\pgfpathlineto{\pgfpoint{18.091238\du}{9.741913\du}}
\pgfpathlineto{\pgfpoint{18.081745\du}{9.747755\du}}
\pgfpathlineto{\pgfpoint{18.071522\du}{9.754326\du}}
\pgfpathlineto{\pgfpoint{18.062395\du}{9.760168\du}}
\pgfpathlineto{\pgfpoint{18.052172\du}{9.766009\du}}
\pgfpathlineto{\pgfpoint{18.042315\du}{9.771851\du}}
\pgfpathlineto{\pgfpoint{18.032457\du}{9.777693\du}}
\pgfpathlineto{\pgfpoint{18.021504\du}{9.784264\du}}
\pgfpathlineto{\pgfpoint{18.010551\du}{9.789376\du}}
\pgfpathlineto{\pgfpoint{18.000329\du}{9.795217\du}}
\pgfpathlineto{\pgfpoint{17.988280\du}{9.801059\du}}
\pgfpathlineto{\pgfpoint{17.977693\du}{9.806900\du}}
\pgfpathlineto{\pgfpoint{17.965644\du}{9.812742\du}}
\pgfpathlineto{\pgfpoint{17.954326\du}{9.817853\du}}
\pgfpathlineto{\pgfpoint{17.941913\du}{9.823330\du}}
\pgfpathlineto{\pgfpoint{17.929865\du}{9.829171\du}}
\pgfpathlineto{\pgfpoint{17.917452\du}{9.834283\du}}
\pgfpathlineto{\pgfpoint{17.905403\du}{9.839759\du}}
\pgfpathlineto{\pgfpoint{17.892260\du}{9.844870\du}}
\pgfpathlineto{\pgfpoint{17.879482\du}{9.850347\du}}
\pgfpathlineto{\pgfpoint{17.866703\du}{9.856188\du}}
\pgfpathlineto{\pgfpoint{17.853560\du}{9.860570\du}}
\pgfpathlineto{\pgfpoint{17.840051\du}{9.866046\du}}
\pgfpathlineto{\pgfpoint{17.826908\du}{9.871157\du}}
\pgfpathlineto{\pgfpoint{17.812669\du}{9.875904\du}}
\pgfpathlineto{\pgfpoint{17.798795\du}{9.881380\du}}
\pgfpathlineto{\pgfpoint{17.784556\du}{9.885761\du}}
\pgfpathlineto{\pgfpoint{17.769953\du}{9.890507\du}}
\pgfpathlineto{\pgfpoint{17.756079\du}{9.895984\du}}
\pgfpathlineto{\pgfpoint{17.741475\du}{9.900365\du}}
\pgfpathlineto{\pgfpoint{17.727236\du}{9.905111\du}}
\pgfpathlineto{\pgfpoint{17.711537\du}{9.909858\du}}
\pgfpathlineto{\pgfpoint{17.697298\du}{9.913874\du}}
\pgfpathlineto{\pgfpoint{17.681599\du}{9.918620\du}}
\pgfpathlineto{\pgfpoint{17.666630\du}{9.923366\du}}
\pgfpathlineto{\pgfpoint{17.635232\du}{9.931398\du}}
\pgfpathlineto{\pgfpoint{17.603103\du}{9.940161\du}}
\pgfpathlineto{\pgfpoint{17.570975\du}{9.948558\du}}
\pgfpathlineto{\pgfpoint{17.537751\du}{9.956225\du}}
\pgfpathlineto{\pgfpoint{17.503432\du}{9.963892\du}}
\pgfpathlineto{\pgfpoint{17.469478\du}{9.971194\du}}
\pgfpathlineto{\pgfpoint{17.434794\du}{9.978496\du}}
\pgfpathlineto{\pgfpoint{17.399014\du}{9.985433\du}}
\pgfpathlineto{\pgfpoint{17.363600\du}{9.991639\du}}
\pgfpathlineto{\pgfpoint{17.326725\du}{9.997481\du}}
\pgfpathlineto{\pgfpoint{17.290215\du}{10.003687\du}}
\pgfpathlineto{\pgfpoint{17.252976\du}{10.009529\du}}
\pgfpathlineto{\pgfpoint{17.215371\du}{10.014640\du}}
\pgfpathlineto{\pgfpoint{17.176670\du}{10.019752\du}}
\pgfpathlineto{\pgfpoint{17.137970\du}{10.024498\du}}
\pgfpathlineto{\pgfpoint{17.098540\du}{10.028514\du}}
\pgfpathlineto{\pgfpoint{17.059474\du}{10.032895\du}}
\pgfpathlineto{\pgfpoint{17.019314\du}{10.036181\du}}
\pgfpathlineto{\pgfpoint{16.979153\du}{10.039832\du}}
\pgfpathlineto{\pgfpoint{16.937897\du}{10.042753\du}}
\pgfpathlineto{\pgfpoint{16.897006\du}{10.045674\du}}
\pgfpathlineto{\pgfpoint{16.855750\du}{10.047864\du}}
\pgfpathlineto{\pgfpoint{16.814129\du}{10.050420\du}}
\pgfpathlineto{\pgfpoint{16.771413\du}{10.051515\du}}
\pgfpathlineto{\pgfpoint{16.728697\du}{10.053341\du}}
\pgfpathlineto{\pgfpoint{16.686710\du}{10.053706\du}}
\pgfpathlineto{\pgfpoint{16.643264\du}{10.054436\du}}
\pgfpathlineto{\pgfpoint{16.600183\du}{10.054436\du}}
\pgfpathlineto{\pgfpoint{16.600183\du}{10.054436\du}}
\pgfpathlineto{\pgfpoint{16.600183\du}{10.054436\du}}
\pgfpathlineto{\pgfpoint{16.599452\du}{10.055166\du}}
\pgfpathlineto{\pgfpoint{16.597627\du}{10.055166\du}}
\pgfpathlineto{\pgfpoint{16.596532\du}{10.055166\du}}
\pgfpathlineto{\pgfpoint{16.595801\du}{10.055531\du}}
\pgfpathlineto{\pgfpoint{16.595436\du}{10.056261\du}}
\pgfpathlineto{\pgfpoint{16.593976\du}{10.056261\du}}
\pgfpathlineto{\pgfpoint{16.593246\du}{10.057357\du}}
\pgfpathlineto{\pgfpoint{16.592516\du}{10.058087\du}}
\pgfpathlineto{\pgfpoint{16.591420\du}{10.059182\du}}
\pgfpathlineto{\pgfpoint{16.590690\du}{10.061008\du}}
\pgfpathlineto{\pgfpoint{16.590690\du}{10.063198\du}}
\pgfpathlineto{\pgfpoint{16.589960\du}{10.065024\du}}
\pgfpathlineto{\pgfpoint{16.590690\du}{10.066849\du}}
\pgfpathlineto{\pgfpoint{16.590690\du}{10.068310\du}}
\pgfpathlineto{\pgfpoint{16.591420\du}{10.070135\du}}
\pgfpathlineto{\pgfpoint{16.592516\du}{10.071961\du}}
\pgfpathlineto{\pgfpoint{16.593246\du}{10.072691\du}}
\pgfpathlineto{\pgfpoint{16.593976\du}{10.073056\du}}
\pgfpathlineto{\pgfpoint{16.595436\du}{10.073786\du}}
\pgfpathlineto{\pgfpoint{16.595801\du}{10.074151\du}}
\pgfpathlineto{\pgfpoint{16.596532\du}{10.074881\du}}
\pgfpathlineto{\pgfpoint{16.597627\du}{10.074881\du}}
\pgfpathlineto{\pgfpoint{16.599452\du}{10.074881\du}}
\pgfpathlineto{\pgfpoint{16.600183\du}{10.074881\du}}
\pgfusepath{fill}
\pgfsetbuttcap
\pgfsetmiterjoin
\pgfsetdash{}{0pt}
\definecolor{dialinecolor}{rgb}{0.678431, 0.839216, 0.905882}
\pgfsetfillcolor{dialinecolor}
\pgfpathmoveto{\pgfpoint{14.909419\du}{9.487441\du}}
\pgfpathlineto{\pgfpoint{14.909419\du}{9.487441\du}}
\pgfpathlineto{\pgfpoint{14.909419\du}{9.495473\du}}
\pgfpathlineto{\pgfpoint{14.909785\du}{9.503140\du}}
\pgfpathlineto{\pgfpoint{14.910515\du}{9.510807\du}}
\pgfpathlineto{\pgfpoint{14.911610\du}{9.519204\du}}
\pgfpathlineto{\pgfpoint{14.912705\du}{9.526506\du}}
\pgfpathlineto{\pgfpoint{14.914531\du}{9.534173\du}}
\pgfpathlineto{\pgfpoint{14.916356\du}{9.541840\du}}
\pgfpathlineto{\pgfpoint{14.918547\du}{9.549507\du}}
\pgfpathlineto{\pgfpoint{14.920737\du}{9.557174\du}}
\pgfpathlineto{\pgfpoint{14.923293\du}{9.564841\du}}
\pgfpathlineto{\pgfpoint{14.926214\du}{9.572143\du}}
\pgfpathlineto{\pgfpoint{14.929865\du}{9.579445\du}}
\pgfpathlineto{\pgfpoint{14.933151\du}{9.586747\du}}
\pgfpathlineto{\pgfpoint{14.936802\du}{9.594414\du}}
\pgfpathlineto{\pgfpoint{14.941183\du}{9.602081\du}}
\pgfpathlineto{\pgfpoint{14.944834\du}{9.609018\du}}
\pgfpathlineto{\pgfpoint{14.949945\du}{9.616685\du}}
\pgfpathlineto{\pgfpoint{14.953961\du}{9.623622\du}}
\pgfpathlineto{\pgfpoint{14.959438\du}{9.630924\du}}
\pgfpathlineto{\pgfpoint{14.964184\du}{9.637861\du}}
\pgfpathlineto{\pgfpoint{14.969660\du}{9.645528\du}}
\pgfpathlineto{\pgfpoint{14.975502\du}{9.652464\du}}
\pgfpathlineto{\pgfpoint{14.981344\du}{9.659401\du}}
\pgfpathlineto{\pgfpoint{14.987185\du}{9.666338\du}}
\pgfpathlineto{\pgfpoint{14.994122\du}{9.672910\du}}
\pgfpathlineto{\pgfpoint{15.000694\du}{9.679847\du}}
\pgfpathlineto{\pgfpoint{15.007631\du}{9.686783\du}}
\pgfpathlineto{\pgfpoint{15.014567\du}{9.693355\du}}
\pgfpathlineto{\pgfpoint{15.021504\du}{9.700292\du}}
\pgfpathlineto{\pgfpoint{15.030267\du}{9.706864\du}}
\pgfpathlineto{\pgfpoint{15.037203\du}{9.713801\du}}
\pgfpathlineto{\pgfpoint{15.044870\du}{9.720372\du}}
\pgfpathlineto{\pgfpoint{15.053633\du}{9.726579\du}}
\pgfpathlineto{\pgfpoint{15.062030\du}{9.733151\du}}
\pgfpathlineto{\pgfpoint{15.070792\du}{9.739723\du}}
\pgfpathlineto{\pgfpoint{15.079189\du}{9.745929\du}}
\pgfpathlineto{\pgfpoint{15.089047\du}{9.752501\du}}
\pgfpathlineto{\pgfpoint{15.098175\du}{9.759073\du}}
\pgfpathlineto{\pgfpoint{15.107667\du}{9.765279\du}}
\pgfpathlineto{\pgfpoint{15.117160\du}{9.771121\du}}
\pgfpathlineto{\pgfpoint{15.127017\du}{9.777693\du}}
\pgfpathlineto{\pgfpoint{15.137240\du}{9.784264\du}}
\pgfpathlineto{\pgfpoint{15.147463\du}{9.790106\du}}
\pgfpathlineto{\pgfpoint{15.158050\du}{9.795947\du}}
\pgfpathlineto{\pgfpoint{15.168273\du}{9.801789\du}}
\pgfpathlineto{\pgfpoint{15.179226\du}{9.807996\du}}
\pgfpathlineto{\pgfpoint{15.190909\du}{9.813837\du}}
\pgfpathlineto{\pgfpoint{15.202227\du}{9.819314\du}}
\pgfpathlineto{\pgfpoint{15.213910\du}{9.825155\du}}
\pgfpathlineto{\pgfpoint{15.225228\du}{9.830997\du}}
\pgfpathlineto{\pgfpoint{15.236911\du}{9.836838\du}}
\pgfpathlineto{\pgfpoint{15.249325\du}{9.841950\du}}
\pgfpathlineto{\pgfpoint{15.261373\du}{9.847791\du}}
\pgfpathlineto{\pgfpoint{15.274516\du}{9.853633\du}}
\pgfpathlineto{\pgfpoint{15.286564\du}{9.859109\du}}
\pgfpathlineto{\pgfpoint{15.299343\du}{9.864221\du}}
\pgfpathlineto{\pgfpoint{15.313217\du}{9.869697\du}}
\pgfpathlineto{\pgfpoint{15.325630\du}{9.874808\du}}
\pgfpathlineto{\pgfpoint{15.338773\du}{9.879920\du}}
\pgfpathlineto{\pgfpoint{15.353012\du}{9.885396\du}}
\pgfpathlineto{\pgfpoint{15.366156\du}{9.890507\du}}
\pgfpathlineto{\pgfpoint{15.380394\du}{9.895254\du}}
\pgfpathlineto{\pgfpoint{15.394268\du}{9.900365\du}}
\pgfpathlineto{\pgfpoint{15.408872\du}{9.905111\du}}
\pgfpathlineto{\pgfpoint{15.422746\du}{9.910588\du}}
\pgfpathlineto{\pgfpoint{15.438445\du}{9.915334\du}}
\pgfpathlineto{\pgfpoint{15.452318\du}{9.919715\du}}
\pgfpathlineto{\pgfpoint{15.466922\du}{9.924461\du}}
\pgfpathlineto{\pgfpoint{15.482256\du}{9.929208\du}}
\pgfpathlineto{\pgfpoint{15.497955\du}{9.933954\du}}
\pgfpathlineto{\pgfpoint{15.512924\du}{9.938700\du}}
\pgfpathlineto{\pgfpoint{15.528624\du}{9.942716\du}}
\pgfpathlineto{\pgfpoint{15.560387\du}{9.951844\du}}
\pgfpathlineto{\pgfpoint{15.592150\du}{9.960241\du}}
\pgfpathlineto{\pgfpoint{15.625374\du}{9.968273\du}}
\pgfpathlineto{\pgfpoint{15.657868\du}{9.976670\du}}
\pgfpathlineto{\pgfpoint{15.692552\du}{9.984337\du}}
\pgfpathlineto{\pgfpoint{15.726871\du}{9.991274\du}}
\pgfpathlineto{\pgfpoint{15.761555\du}{9.998211\du}}
\pgfpathlineto{\pgfpoint{15.797335\du}{10.005148\du}}
\pgfpathlineto{\pgfpoint{15.833479\du}{10.011720\du}}
\pgfpathlineto{\pgfpoint{15.869989\du}{10.018291\du}}
\pgfpathlineto{\pgfpoint{15.907229\du}{10.024133\du}}
\pgfpathlineto{\pgfpoint{15.944834\du}{10.029974\du}}
\pgfpathlineto{\pgfpoint{15.982804\du}{10.035086\du}}
\pgfpathlineto{\pgfpoint{16.021504\du}{10.039832\du}}
\pgfpathlineto{\pgfpoint{16.060204\du}{10.044578\du}}
\pgfpathlineto{\pgfpoint{16.099270\du}{10.049325\du}}
\pgfpathlineto{\pgfpoint{16.139430\du}{10.053341\du}}
\pgfpathlineto{\pgfpoint{16.179226\du}{10.057357\du}}
\pgfpathlineto{\pgfpoint{16.220117\du}{10.060277\du}}
\pgfpathlineto{\pgfpoint{16.261008\du}{10.063198\du}}
\pgfpathlineto{\pgfpoint{16.302264\du}{10.066119\du}}
\pgfpathlineto{\pgfpoint{16.344250\du}{10.068310\du}}
\pgfpathlineto{\pgfpoint{16.385871\du}{10.070135\du}}
\pgfpathlineto{\pgfpoint{16.428222\du}{10.071961\du}}
\pgfpathlineto{\pgfpoint{16.470208\du}{10.073056\du}}
\pgfpathlineto{\pgfpoint{16.513655\du}{10.074151\du}}
\pgfpathlineto{\pgfpoint{16.556371\du}{10.074881\du}}
\pgfpathlineto{\pgfpoint{16.600183\du}{10.074881\du}}
\pgfpathlineto{\pgfpoint{16.600183\du}{10.054436\du}}
\pgfpathlineto{\pgfpoint{16.557466\du}{10.054436\du}}
\pgfpathlineto{\pgfpoint{16.514020\du}{10.053706\du}}
\pgfpathlineto{\pgfpoint{16.471668\du}{10.053341\du}}
\pgfpathlineto{\pgfpoint{16.428952\du}{10.051515\du}}
\pgfpathlineto{\pgfpoint{16.386601\du}{10.050420\du}}
\pgfpathlineto{\pgfpoint{16.344615\du}{10.047864\du}}
\pgfpathlineto{\pgfpoint{16.303724\du}{10.045674\du}}
\pgfpathlineto{\pgfpoint{16.262468\du}{10.042753\du}}
\pgfpathlineto{\pgfpoint{16.221577\du}{10.039832\du}}
\pgfpathlineto{\pgfpoint{16.181417\du}{10.036181\du}}
\pgfpathlineto{\pgfpoint{16.141621\du}{10.032895\du}}
\pgfpathlineto{\pgfpoint{16.102191\du}{10.028514\du}}
\pgfpathlineto{\pgfpoint{16.062760\du}{10.024498\du}}
\pgfpathlineto{\pgfpoint{16.023695\du}{10.019752\du}}
\pgfpathlineto{\pgfpoint{15.985360\du}{10.014640\du}}
\pgfpathlineto{\pgfpoint{15.947755\du}{10.009529\du}}
\pgfpathlineto{\pgfpoint{15.910515\du}{10.003687\du}}
\pgfpathlineto{\pgfpoint{15.874005\du}{9.997481\du}}
\pgfpathlineto{\pgfpoint{15.836765\du}{9.991639\du}}
\pgfpathlineto{\pgfpoint{15.801716\du}{9.985433\du}}
\pgfpathlineto{\pgfpoint{15.765571\du}{9.978496\du}}
\pgfpathlineto{\pgfpoint{15.730887\du}{9.971194\du}}
\pgfpathlineto{\pgfpoint{15.696568\du}{9.963892\du}}
\pgfpathlineto{\pgfpoint{15.662614\du}{9.956225\du}}
\pgfpathlineto{\pgfpoint{15.629390\du}{9.948558\du}}
\pgfpathlineto{\pgfpoint{15.597992\du}{9.940161\du}}
\pgfpathlineto{\pgfpoint{15.565498\du}{9.931398\du}}
\pgfpathlineto{\pgfpoint{15.534465\du}{9.923366\du}}
\pgfpathlineto{\pgfpoint{15.518766\du}{9.918620\du}}
\pgfpathlineto{\pgfpoint{15.503067\du}{9.913874\du}}
\pgfpathlineto{\pgfpoint{15.488463\du}{9.909858\du}}
\pgfpathlineto{\pgfpoint{15.473494\du}{9.905111\du}}
\pgfpathlineto{\pgfpoint{15.458160\du}{9.900365\du}}
\pgfpathlineto{\pgfpoint{15.443921\du}{9.895984\du}}
\pgfpathlineto{\pgfpoint{15.429682\du}{9.890507\du}}
\pgfpathlineto{\pgfpoint{15.415444\du}{9.885761\du}}
\pgfpathlineto{\pgfpoint{15.401205\du}{9.881380\du}}
\pgfpathlineto{\pgfpoint{15.387331\du}{9.875904\du}}
\pgfpathlineto{\pgfpoint{15.373457\du}{9.871157\du}}
\pgfpathlineto{\pgfpoint{15.359949\du}{9.866046\du}}
\pgfpathlineto{\pgfpoint{15.346440\du}{9.860570\du}}
\pgfpathlineto{\pgfpoint{15.333662\du}{9.856188\du}}
\pgfpathlineto{\pgfpoint{15.320153\du}{9.850347\du}}
\pgfpathlineto{\pgfpoint{15.307375\du}{9.844870\du}}
\pgfpathlineto{\pgfpoint{15.294962\du}{9.839759\du}}
\pgfpathlineto{\pgfpoint{15.281818\du}{9.834283\du}}
\pgfpathlineto{\pgfpoint{15.269770\du}{9.829171\du}}
\pgfpathlineto{\pgfpoint{15.258452\du}{9.823330\du}}
\pgfpathlineto{\pgfpoint{15.245674\du}{9.817853\du}}
\pgfpathlineto{\pgfpoint{15.233991\du}{9.812742\du}}
\pgfpathlineto{\pgfpoint{15.222307\du}{9.806900\du}}
\pgfpathlineto{\pgfpoint{15.211355\du}{9.801059\du}}
\pgfpathlineto{\pgfpoint{15.199671\du}{9.795217\du}}
\pgfpathlineto{\pgfpoint{15.189814\du}{9.789376\du}}
\pgfpathlineto{\pgfpoint{15.178496\du}{9.784264\du}}
\pgfpathlineto{\pgfpoint{15.167908\du}{9.777693\du}}
\pgfpathlineto{\pgfpoint{15.158050\du}{9.771851\du}}
\pgfpathlineto{\pgfpoint{15.147463\du}{9.766009\du}}
\pgfpathlineto{\pgfpoint{15.137605\du}{9.760168\du}}
\pgfpathlineto{\pgfpoint{15.128112\du}{9.754326\du}}
\pgfpathlineto{\pgfpoint{15.118255\du}{9.747755\du}}
\pgfpathlineto{\pgfpoint{15.109127\du}{9.741913\du}}
\pgfpathlineto{\pgfpoint{15.100730\du}{9.735341\du}}
\pgfpathlineto{\pgfpoint{15.091968\du}{9.729500\du}}
\pgfpathlineto{\pgfpoint{15.082840\du}{9.723293\du}}
\pgfpathlineto{\pgfpoint{15.074078\du}{9.716721\du}}
\pgfpathlineto{\pgfpoint{15.065681\du}{9.710515\du}}
\pgfpathlineto{\pgfpoint{15.058379\du}{9.703943\du}}
\pgfpathlineto{\pgfpoint{15.050347\du}{9.698101\du}}
\pgfpathlineto{\pgfpoint{15.042680\du}{9.691530\du}}
\pgfpathlineto{\pgfpoint{15.035743\du}{9.685323\du}}
\pgfpathlineto{\pgfpoint{15.028441\du}{9.678751\du}}
\pgfpathlineto{\pgfpoint{15.021504\du}{9.671815\du}}
\pgfpathlineto{\pgfpoint{15.015298\du}{9.665243\du}}
\pgfpathlineto{\pgfpoint{15.008726\du}{9.659036\du}}
\pgfpathlineto{\pgfpoint{15.002884\du}{9.652464\du}}
\pgfpathlineto{\pgfpoint{14.996678\du}{9.645528\du}}
\pgfpathlineto{\pgfpoint{14.991566\du}{9.638956\du}}
\pgfpathlineto{\pgfpoint{14.985360\du}{9.632384\du}}
\pgfpathlineto{\pgfpoint{14.980978\du}{9.625447\du}}
\pgfpathlineto{\pgfpoint{14.975502\du}{9.619241\du}}
\pgfpathlineto{\pgfpoint{14.971121\du}{9.611939\du}}
\pgfpathlineto{\pgfpoint{14.966740\du}{9.605732\du}}
\pgfpathlineto{\pgfpoint{14.962359\du}{9.598430\du}}
\pgfpathlineto{\pgfpoint{14.959073\du}{9.592223\du}}
\pgfpathlineto{\pgfpoint{14.955057\du}{9.585287\du}}
\pgfpathlineto{\pgfpoint{14.951041\du}{9.577985\du}}
\pgfpathlineto{\pgfpoint{14.948120\du}{9.571048\du}}
\pgfpathlineto{\pgfpoint{14.945564\du}{9.564841\du}}
\pgfpathlineto{\pgfpoint{14.942278\du}{9.557539\du}}
\pgfpathlineto{\pgfpoint{14.940088\du}{9.550602\du}}
\pgfpathlineto{\pgfpoint{14.937897\du}{9.543666\du}}
\pgfpathlineto{\pgfpoint{14.936437\du}{9.536729\du}}
\pgfpathlineto{\pgfpoint{14.934246\du}{9.529427\du}}
\pgfpathlineto{\pgfpoint{14.933151\du}{9.523220\du}}
\pgfpathlineto{\pgfpoint{14.932055\du}{9.515918\du}}
\pgfpathlineto{\pgfpoint{14.930595\du}{9.508981\du}}
\pgfpathlineto{\pgfpoint{14.930230\du}{9.502045\du}}
\pgfpathlineto{\pgfpoint{14.930230\du}{9.494378\du}}
\pgfpathlineto{\pgfpoint{14.929865\du}{9.487441\du}}
\pgfpathlineto{\pgfpoint{14.929865\du}{9.487441\du}}
\pgfpathlineto{\pgfpoint{14.929865\du}{9.487441\du}}
\pgfpathlineto{\pgfpoint{14.929865\du}{9.486345\du}}
\pgfpathlineto{\pgfpoint{14.929865\du}{9.485250\du}}
\pgfpathlineto{\pgfpoint{14.929500\du}{9.483790\du}}
\pgfpathlineto{\pgfpoint{14.929500\du}{9.483425\du}}
\pgfpathlineto{\pgfpoint{14.928405\du}{9.482329\du}}
\pgfpathlineto{\pgfpoint{14.928039\du}{9.480869\du}}
\pgfpathlineto{\pgfpoint{14.927674\du}{9.480504\du}}
\pgfpathlineto{\pgfpoint{14.926579\du}{9.479774\du}}
\pgfpathlineto{\pgfpoint{14.925119\du}{9.478678\du}}
\pgfpathlineto{\pgfpoint{14.923293\du}{9.477948\du}}
\pgfpathlineto{\pgfpoint{14.921468\du}{9.477583\du}}
\pgfpathlineto{\pgfpoint{14.919642\du}{9.477583\du}}
\pgfpathlineto{\pgfpoint{14.917452\du}{9.477583\du}}
\pgfpathlineto{\pgfpoint{14.915991\du}{9.477948\du}}
\pgfpathlineto{\pgfpoint{14.913801\du}{9.478678\du}}
\pgfpathlineto{\pgfpoint{14.911975\du}{9.479774\du}}
\pgfpathlineto{\pgfpoint{14.911610\du}{9.480504\du}}
\pgfpathlineto{\pgfpoint{14.911245\du}{9.480869\du}}
\pgfpathlineto{\pgfpoint{14.910515\du}{9.482329\du}}
\pgfpathlineto{\pgfpoint{14.909785\du}{9.483425\du}}
\pgfpathlineto{\pgfpoint{14.909785\du}{9.483790\du}}
\pgfpathlineto{\pgfpoint{14.909419\du}{9.485250\du}}
\pgfpathlineto{\pgfpoint{14.909419\du}{9.486345\du}}
\pgfpathlineto{\pgfpoint{14.909419\du}{9.487441\du}}
\pgfusepath{fill}
\pgfsetbuttcap
\pgfsetmiterjoin
\pgfsetdash{}{0pt}
\definecolor{dialinecolor}{rgb}{0.678431, 0.839216, 0.905882}
\pgfsetfillcolor{dialinecolor}
\pgfpathmoveto{\pgfpoint{16.600183\du}{8.900000\du}}
\pgfpathlineto{\pgfpoint{16.600183\du}{8.900000\du}}
\pgfpathlineto{\pgfpoint{16.556371\du}{8.900000\du}}
\pgfpathlineto{\pgfpoint{16.513655\du}{8.900730\du}}
\pgfpathlineto{\pgfpoint{16.470208\du}{8.901825\du}}
\pgfpathlineto{\pgfpoint{16.428222\du}{8.902921\du}}
\pgfpathlineto{\pgfpoint{16.385871\du}{8.904746\du}}
\pgfpathlineto{\pgfpoint{16.344250\du}{8.906937\du}}
\pgfpathlineto{\pgfpoint{16.302264\du}{8.909493\du}}
\pgfpathlineto{\pgfpoint{16.261008\du}{8.911683\du}}
\pgfpathlineto{\pgfpoint{16.220117\du}{8.914604\du}}
\pgfpathlineto{\pgfpoint{16.179226\du}{8.918255\du}}
\pgfpathlineto{\pgfpoint{16.139430\du}{8.921906\du}}
\pgfpathlineto{\pgfpoint{16.099270\du}{8.925922\du}}
\pgfpathlineto{\pgfpoint{16.060204\du}{8.930303\du}}
\pgfpathlineto{\pgfpoint{16.021504\du}{8.935049\du}}
\pgfpathlineto{\pgfpoint{15.982804\du}{8.940161\du}}
\pgfpathlineto{\pgfpoint{15.944834\du}{8.945637\du}}
\pgfpathlineto{\pgfpoint{15.907229\du}{8.950748\du}}
\pgfpathlineto{\pgfpoint{15.869989\du}{8.957320\du}}
\pgfpathlineto{\pgfpoint{15.833479\du}{8.963162\du}}
\pgfpathlineto{\pgfpoint{15.797335\du}{8.969733\du}}
\pgfpathlineto{\pgfpoint{15.761555\du}{8.976670\du}}
\pgfpathlineto{\pgfpoint{15.726871\du}{8.983607\du}}
\pgfpathlineto{\pgfpoint{15.692552\du}{8.991274\du}}
\pgfpathlineto{\pgfpoint{15.657868\du}{8.998941\du}}
\pgfpathlineto{\pgfpoint{15.625374\du}{9.006973\du}}
\pgfpathlineto{\pgfpoint{15.592150\du}{9.015371\du}}
\pgfpathlineto{\pgfpoint{15.560387\du}{9.023403\du}}
\pgfpathlineto{\pgfpoint{15.528624\du}{9.032165\du}}
\pgfpathlineto{\pgfpoint{15.497955\du}{9.041658\du}}
\pgfpathlineto{\pgfpoint{15.466922\du}{9.050420\du}}
\pgfpathlineto{\pgfpoint{15.452318\du}{9.055166\du}}
\pgfpathlineto{\pgfpoint{15.438445\du}{9.060277\du}}
\pgfpathlineto{\pgfpoint{15.422746\du}{9.065024\du}}
\pgfpathlineto{\pgfpoint{15.408872\du}{9.069770\du}}
\pgfpathlineto{\pgfpoint{15.394268\du}{9.074881\du}}
\pgfpathlineto{\pgfpoint{15.380394\du}{9.079628\du}}
\pgfpathlineto{\pgfpoint{15.366156\du}{9.084739\du}}
\pgfpathlineto{\pgfpoint{15.353012\du}{9.089485\du}}
\pgfpathlineto{\pgfpoint{15.338773\du}{9.094962\du}}
\pgfpathlineto{\pgfpoint{15.325630\du}{9.100073\du}}
\pgfpathlineto{\pgfpoint{15.313217\du}{9.105184\du}}
\pgfpathlineto{\pgfpoint{15.299343\du}{9.111026\du}}
\pgfpathlineto{\pgfpoint{15.286564\du}{9.116502\du}}
\pgfpathlineto{\pgfpoint{15.274516\du}{9.121614\du}}
\pgfpathlineto{\pgfpoint{15.261373\du}{9.127455\du}}
\pgfpathlineto{\pgfpoint{15.249325\du}{9.132932\du}}
\pgfpathlineto{\pgfpoint{15.236911\du}{9.138773\du}}
\pgfpathlineto{\pgfpoint{15.225228\du}{9.143885\du}}
\pgfpathlineto{\pgfpoint{15.213910\du}{9.149726\du}}
\pgfpathlineto{\pgfpoint{15.202227\du}{9.155568\du}}
\pgfpathlineto{\pgfpoint{15.190909\du}{9.161409\du}}
\pgfpathlineto{\pgfpoint{15.179226\du}{9.167251\du}}
\pgfpathlineto{\pgfpoint{15.168273\du}{9.173092\du}}
\pgfpathlineto{\pgfpoint{15.158050\du}{9.179664\du}}
\pgfpathlineto{\pgfpoint{15.147463\du}{9.185506\du}}
\pgfpathlineto{\pgfpoint{15.137240\du}{9.191347\du}}
\pgfpathlineto{\pgfpoint{15.127017\du}{9.197919\du}}
\pgfpathlineto{\pgfpoint{15.117160\du}{9.203760\du}}
\pgfpathlineto{\pgfpoint{15.107667\du}{9.209967\du}}
\pgfpathlineto{\pgfpoint{15.098175\du}{9.216539\du}}
\pgfpathlineto{\pgfpoint{15.089047\du}{9.223111\du}}
\pgfpathlineto{\pgfpoint{15.079189\du}{9.228952\du}}
\pgfpathlineto{\pgfpoint{15.070792\du}{9.235159\du}}
\pgfpathlineto{\pgfpoint{15.062030\du}{9.241731\du}}
\pgfpathlineto{\pgfpoint{15.053633\du}{9.247937\du}}
\pgfpathlineto{\pgfpoint{15.044870\du}{9.255239\du}}
\pgfpathlineto{\pgfpoint{15.037203\du}{9.261446\du}}
\pgfpathlineto{\pgfpoint{15.030267\du}{9.268018\du}}
\pgfpathlineto{\pgfpoint{15.021504\du}{9.274954\du}}
\pgfpathlineto{\pgfpoint{15.014567\du}{9.281526\du}}
\pgfpathlineto{\pgfpoint{15.007631\du}{9.288463\du}}
\pgfpathlineto{\pgfpoint{15.000694\du}{9.295400\du}}
\pgfpathlineto{\pgfpoint{14.994122\du}{9.301972\du}}
\pgfpathlineto{\pgfpoint{14.987185\du}{9.308908\du}}
\pgfpathlineto{\pgfpoint{14.981344\du}{9.315845\du}}
\pgfpathlineto{\pgfpoint{14.975502\du}{9.323147\du}}
\pgfpathlineto{\pgfpoint{14.969660\du}{9.330084\du}}
\pgfpathlineto{\pgfpoint{14.964184\du}{9.337021\du}}
\pgfpathlineto{\pgfpoint{14.959438\du}{9.343958\du}}
\pgfpathlineto{\pgfpoint{14.953961\du}{9.351625\du}}
\pgfpathlineto{\pgfpoint{14.949945\du}{9.358562\du}}
\pgfpathlineto{\pgfpoint{14.944834\du}{9.365863\du}}
\pgfpathlineto{\pgfpoint{14.941183\du}{9.373165\du}}
\pgfpathlineto{\pgfpoint{14.936802\du}{9.380467\du}}
\pgfpathlineto{\pgfpoint{14.933151\du}{9.388134\du}}
\pgfpathlineto{\pgfpoint{14.929865\du}{9.395436\du}}
\pgfpathlineto{\pgfpoint{14.926214\du}{9.403103\du}}
\pgfpathlineto{\pgfpoint{14.923293\du}{9.410770\du}}
\pgfpathlineto{\pgfpoint{14.920737\du}{9.417707\du}}
\pgfpathlineto{\pgfpoint{14.918547\du}{9.425374\du}}
\pgfpathlineto{\pgfpoint{14.916356\du}{9.433041\du}}
\pgfpathlineto{\pgfpoint{14.914531\du}{9.441073\du}}
\pgfpathlineto{\pgfpoint{14.912705\du}{9.448740\du}}
\pgfpathlineto{\pgfpoint{14.911610\du}{9.456407\du}}
\pgfpathlineto{\pgfpoint{14.910515\du}{9.464074\du}}
\pgfpathlineto{\pgfpoint{14.909785\du}{9.472107\du}}
\pgfpathlineto{\pgfpoint{14.909419\du}{9.479774\du}}
\pgfpathlineto{\pgfpoint{14.909419\du}{9.487441\du}}
\pgfpathlineto{\pgfpoint{14.929865\du}{9.487441\du}}
\pgfpathlineto{\pgfpoint{14.930230\du}{9.480504\du}}
\pgfpathlineto{\pgfpoint{14.930230\du}{9.473567\du}}
\pgfpathlineto{\pgfpoint{14.930595\du}{9.466265\du}}
\pgfpathlineto{\pgfpoint{14.932055\du}{9.459328\du}}
\pgfpathlineto{\pgfpoint{14.933151\du}{9.452391\du}}
\pgfpathlineto{\pgfpoint{14.934246\du}{9.445455\du}}
\pgfpathlineto{\pgfpoint{14.936437\du}{9.438153\du}}
\pgfpathlineto{\pgfpoint{14.937897\du}{9.431946\du}}
\pgfpathlineto{\pgfpoint{14.940088\du}{9.424644\du}}
\pgfpathlineto{\pgfpoint{14.942278\du}{9.417707\du}}
\pgfpathlineto{\pgfpoint{14.945564\du}{9.410770\du}}
\pgfpathlineto{\pgfpoint{14.948120\du}{9.403834\du}}
\pgfpathlineto{\pgfpoint{14.951041\du}{9.396897\du}}
\pgfpathlineto{\pgfpoint{14.955057\du}{9.390325\du}}
\pgfpathlineto{\pgfpoint{14.959073\du}{9.383388\du}}
\pgfpathlineto{\pgfpoint{14.962359\du}{9.376816\du}}
\pgfpathlineto{\pgfpoint{14.966375\du}{9.369880\du}}
\pgfpathlineto{\pgfpoint{14.971121\du}{9.362943\du}}
\pgfpathlineto{\pgfpoint{14.975502\du}{9.356371\du}}
\pgfpathlineto{\pgfpoint{14.980978\du}{9.349799\du}}
\pgfpathlineto{\pgfpoint{14.985360\du}{9.342862\du}}
\pgfpathlineto{\pgfpoint{14.991566\du}{9.336291\du}}
\pgfpathlineto{\pgfpoint{14.996678\du}{9.329354\du}}
\pgfpathlineto{\pgfpoint{15.002884\du}{9.323147\du}}
\pgfpathlineto{\pgfpoint{15.008726\du}{9.316575\du}}
\pgfpathlineto{\pgfpoint{15.015298\du}{9.309639\du}}
\pgfpathlineto{\pgfpoint{15.021504\du}{9.303067\du}}
\pgfpathlineto{\pgfpoint{15.028441\du}{9.296860\du}}
\pgfpathlineto{\pgfpoint{15.035743\du}{9.290288\du}}
\pgfpathlineto{\pgfpoint{15.042680\du}{9.283717\du}}
\pgfpathlineto{\pgfpoint{15.050347\du}{9.277510\du}}
\pgfpathlineto{\pgfpoint{15.058379\du}{9.270938\du}}
\pgfpathlineto{\pgfpoint{15.065681\du}{9.264367\du}}
\pgfpathlineto{\pgfpoint{15.074078\du}{9.258160\du}}
\pgfpathlineto{\pgfpoint{15.082840\du}{9.252318\du}}
\pgfpathlineto{\pgfpoint{15.091968\du}{9.245382\du}}
\pgfpathlineto{\pgfpoint{15.100730\du}{9.239905\du}}
\pgfpathlineto{\pgfpoint{15.109127\du}{9.233333\du}}
\pgfpathlineto{\pgfpoint{15.118255\du}{9.227127\du}}
\pgfpathlineto{\pgfpoint{15.128112\du}{9.221285\du}}
\pgfpathlineto{\pgfpoint{15.137605\du}{9.215444\du}}
\pgfpathlineto{\pgfpoint{15.147463\du}{9.208872\du}}
\pgfpathlineto{\pgfpoint{15.158050\du}{9.203030\du}}
\pgfpathlineto{\pgfpoint{15.167908\du}{9.197189\du}}
\pgfpathlineto{\pgfpoint{15.178496\du}{9.191347\du}}
\pgfpathlineto{\pgfpoint{15.189814\du}{9.185506\du}}
\pgfpathlineto{\pgfpoint{15.199671\du}{9.179664\du}}
\pgfpathlineto{\pgfpoint{15.211355\du}{9.173823\du}}
\pgfpathlineto{\pgfpoint{15.222307\du}{9.168711\du}}
\pgfpathlineto{\pgfpoint{15.233991\du}{9.162505\du}}
\pgfpathlineto{\pgfpoint{15.245674\du}{9.156663\du}}
\pgfpathlineto{\pgfpoint{15.258452\du}{9.151552\du}}
\pgfpathlineto{\pgfpoint{15.269770\du}{9.146440\du}}
\pgfpathlineto{\pgfpoint{15.281818\du}{9.140599\du}}
\pgfpathlineto{\pgfpoint{15.294962\du}{9.135122\du}}
\pgfpathlineto{\pgfpoint{15.307375\du}{9.130011\du}}
\pgfpathlineto{\pgfpoint{15.320153\du}{9.124535\du}}
\pgfpathlineto{\pgfpoint{15.333662\du}{9.119423\du}}
\pgfpathlineto{\pgfpoint{15.346440\du}{9.113947\du}}
\pgfpathlineto{\pgfpoint{15.359949\du}{9.108470\du}}
\pgfpathlineto{\pgfpoint{15.373457\du}{9.104089\du}}
\pgfpathlineto{\pgfpoint{15.387331\du}{9.098978\du}}
\pgfpathlineto{\pgfpoint{15.401205\du}{9.094231\du}}
\pgfpathlineto{\pgfpoint{15.415444\du}{9.089120\du}}
\pgfpathlineto{\pgfpoint{15.429682\du}{9.084374\du}}
\pgfpathlineto{\pgfpoint{15.443921\du}{9.079628\du}}
\pgfpathlineto{\pgfpoint{15.458160\du}{9.074881\du}}
\pgfpathlineto{\pgfpoint{15.473494\du}{9.070135\du}}
\pgfpathlineto{\pgfpoint{15.503067\du}{9.061008\du}}
\pgfpathlineto{\pgfpoint{15.534465\du}{9.052245\du}}
\pgfpathlineto{\pgfpoint{15.565498\du}{9.043483\du}}
\pgfpathlineto{\pgfpoint{15.597992\du}{9.035086\du}}
\pgfpathlineto{\pgfpoint{15.629390\du}{9.027054\du}}
\pgfpathlineto{\pgfpoint{15.662614\du}{9.018656\du}}
\pgfpathlineto{\pgfpoint{15.696568\du}{9.010989\du}}
\pgfpathlineto{\pgfpoint{15.730887\du}{9.004053\du}}
\pgfpathlineto{\pgfpoint{15.765571\du}{8.997116\du}}
\pgfpathlineto{\pgfpoint{15.801716\du}{8.989814\du}}
\pgfpathlineto{\pgfpoint{15.836765\du}{8.983607\du}}
\pgfpathlineto{\pgfpoint{15.874005\du}{8.976670\du}}
\pgfpathlineto{\pgfpoint{15.910515\du}{8.971194\du}}
\pgfpathlineto{\pgfpoint{15.947755\du}{8.965352\du}}
\pgfpathlineto{\pgfpoint{15.985360\du}{8.960241\du}}
\pgfpathlineto{\pgfpoint{16.023695\du}{8.955495\du}}
\pgfpathlineto{\pgfpoint{16.062760\du}{8.950748\du}}
\pgfpathlineto{\pgfpoint{16.102191\du}{8.946367\du}}
\pgfpathlineto{\pgfpoint{16.141621\du}{8.942716\du}}
\pgfpathlineto{\pgfpoint{16.181417\du}{8.938700\du}}
\pgfpathlineto{\pgfpoint{16.221577\du}{8.935779\du}}
\pgfpathlineto{\pgfpoint{16.262468\du}{8.932129\du}}
\pgfpathlineto{\pgfpoint{16.303724\du}{8.929208\du}}
\pgfpathlineto{\pgfpoint{16.344615\du}{8.927017\du}}
\pgfpathlineto{\pgfpoint{16.386601\du}{8.925192\du}}
\pgfpathlineto{\pgfpoint{16.428952\du}{8.923366\du}}
\pgfpathlineto{\pgfpoint{16.471668\du}{8.921906\du}}
\pgfpathlineto{\pgfpoint{16.514020\du}{8.921541\du}}
\pgfpathlineto{\pgfpoint{16.557466\du}{8.921176\du}}
\pgfpathlineto{\pgfpoint{16.600183\du}{8.920445\du}}
\pgfpathlineto{\pgfpoint{16.600183\du}{8.920445\du}}
\pgfpathlineto{\pgfpoint{16.600183\du}{8.920445\du}}
\pgfpathlineto{\pgfpoint{16.601278\du}{8.920445\du}}
\pgfpathlineto{\pgfpoint{16.602373\du}{8.920445\du}}
\pgfpathlineto{\pgfpoint{16.603834\du}{8.919715\du}}
\pgfpathlineto{\pgfpoint{16.604929\du}{8.919715\du}}
\pgfpathlineto{\pgfpoint{16.605294\du}{8.919350\du}}
\pgfpathlineto{\pgfpoint{16.606389\du}{8.918620\du}}
\pgfpathlineto{\pgfpoint{16.607119\du}{8.918255\du}}
\pgfpathlineto{\pgfpoint{16.608215\du}{8.917525\du}}
\pgfpathlineto{\pgfpoint{16.609310\du}{8.915699\du}}
\pgfpathlineto{\pgfpoint{16.610040\du}{8.913874\du}}
\pgfpathlineto{\pgfpoint{16.610040\du}{8.912413\du}}
\pgfpathlineto{\pgfpoint{16.610770\du}{8.910588\du}}
\pgfpathlineto{\pgfpoint{16.610040\du}{8.908762\du}}
\pgfpathlineto{\pgfpoint{16.610040\du}{8.906572\du}}
\pgfpathlineto{\pgfpoint{16.609310\du}{8.904746\du}}
\pgfpathlineto{\pgfpoint{16.608215\du}{8.902921\du}}
\pgfpathlineto{\pgfpoint{16.607119\du}{8.902191\du}}
\pgfpathlineto{\pgfpoint{16.606389\du}{8.901825\du}}
\pgfpathlineto{\pgfpoint{16.605294\du}{8.901095\du}}
\pgfpathlineto{\pgfpoint{16.604929\du}{8.900730\du}}
\pgfpathlineto{\pgfpoint{16.603834\du}{8.900730\du}}
\pgfpathlineto{\pgfpoint{16.602373\du}{8.900000\du}}
\pgfpathlineto{\pgfpoint{16.601278\du}{8.900000\du}}
\pgfpathlineto{\pgfpoint{16.600183\du}{8.900000\du}}
\pgfusepath{fill}
\pgfsetbuttcap
\pgfsetmiterjoin
\pgfsetdash{}{0pt}
\definecolor{dialinecolor}{rgb}{0.678431, 0.839216, 0.905882}
\pgfsetfillcolor{dialinecolor}
\pgfpathmoveto{\pgfpoint{18.290581\du}{9.487441\du}}
\pgfpathlineto{\pgfpoint{18.290581\du}{9.479774\du}}
\pgfpathlineto{\pgfpoint{18.289850\du}{9.472107\du}}
\pgfpathlineto{\pgfpoint{18.289120\du}{9.464074\du}}
\pgfpathlineto{\pgfpoint{18.288390\du}{9.456407\du}}
\pgfpathlineto{\pgfpoint{18.287295\du}{9.448740\du}}
\pgfpathlineto{\pgfpoint{18.285104\du}{9.441073\du}}
\pgfpathlineto{\pgfpoint{18.284009\du}{9.433041\du}}
\pgfpathlineto{\pgfpoint{18.281088\du}{9.425374\du}}
\pgfpathlineto{\pgfpoint{18.279263\du}{9.417707\du}}
\pgfpathlineto{\pgfpoint{18.276342\du}{9.410770\du}}
\pgfpathlineto{\pgfpoint{18.273786\du}{9.403103\du}}
\pgfpathlineto{\pgfpoint{18.270500\du}{9.395436\du}}
\pgfpathlineto{\pgfpoint{18.266484\du}{9.388134\du}}
\pgfpathlineto{\pgfpoint{18.263198\du}{9.380467\du}}
\pgfpathlineto{\pgfpoint{18.259182\du}{9.373165\du}}
\pgfpathlineto{\pgfpoint{18.255166\du}{9.365863\du}}
\pgfpathlineto{\pgfpoint{18.250785\du}{9.358562\du}}
\pgfpathlineto{\pgfpoint{18.246039\du}{9.351625\du}}
\pgfpathlineto{\pgfpoint{18.240562\du}{9.343958\du}}
\pgfpathlineto{\pgfpoint{18.235816\du}{9.337021\du}}
\pgfpathlineto{\pgfpoint{18.229974\du}{9.330084\du}}
\pgfpathlineto{\pgfpoint{18.224498\du}{9.323147\du}}
\pgfpathlineto{\pgfpoint{18.218656\du}{9.315845\du}}
\pgfpathlineto{\pgfpoint{18.212450\du}{9.308908\du}}
\pgfpathlineto{\pgfpoint{18.205878\du}{9.301972\du}}
\pgfpathlineto{\pgfpoint{18.198941\du}{9.295400\du}}
\pgfpathlineto{\pgfpoint{18.192369\du}{9.288463\du}}
\pgfpathlineto{\pgfpoint{18.185068\du}{9.281526\du}}
\pgfpathlineto{\pgfpoint{18.178496\du}{9.274954\du}}
\pgfpathlineto{\pgfpoint{18.170464\du}{9.268018\du}}
\pgfpathlineto{\pgfpoint{18.162432\du}{9.261446\du}}
\pgfpathlineto{\pgfpoint{18.155130\du}{9.255239\du}}
\pgfpathlineto{\pgfpoint{18.146367\du}{9.247937\du}}
\pgfpathlineto{\pgfpoint{18.138335\du}{9.241731\du}}
\pgfpathlineto{\pgfpoint{18.129208\du}{9.235159\du}}
\pgfpathlineto{\pgfpoint{18.120445\du}{9.228952\du}}
\pgfpathlineto{\pgfpoint{18.110953\du}{9.223111\du}}
\pgfpathlineto{\pgfpoint{18.101825\du}{9.216539\du}}
\pgfpathlineto{\pgfpoint{18.092698\du}{9.209967\du}}
\pgfpathlineto{\pgfpoint{18.082840\du}{9.203760\du}}
\pgfpathlineto{\pgfpoint{18.072983\du}{9.197919\du}}
\pgfpathlineto{\pgfpoint{18.062760\du}{9.191347\du}}
\pgfpathlineto{\pgfpoint{18.052172\du}{9.185506\du}}
\pgfpathlineto{\pgfpoint{18.042315\du}{9.179664\du}}
\pgfpathlineto{\pgfpoint{18.031727\du}{9.173092\du}}
\pgfpathlineto{\pgfpoint{18.020409\du}{9.167251\du}}
\pgfpathlineto{\pgfpoint{18.009091\du}{9.161409\du}}
\pgfpathlineto{\pgfpoint{17.997408\du}{9.155568\du}}
\pgfpathlineto{\pgfpoint{17.986455\du}{9.149726\du}}
\pgfpathlineto{\pgfpoint{17.974407\du}{9.143885\du}}
\pgfpathlineto{\pgfpoint{17.963089\du}{9.138773\du}}
\pgfpathlineto{\pgfpoint{17.950675\du}{9.132932\du}}
\pgfpathlineto{\pgfpoint{17.938262\du}{9.127455\du}}
\pgfpathlineto{\pgfpoint{17.925484\du}{9.121614\du}}
\pgfpathlineto{\pgfpoint{17.913436\du}{9.116502\du}}
\pgfpathlineto{\pgfpoint{17.900657\du}{9.111026\du}}
\pgfpathlineto{\pgfpoint{17.887149\du}{9.105184\du}}
\pgfpathlineto{\pgfpoint{17.874005\du}{9.100073\du}}
\pgfpathlineto{\pgfpoint{17.860862\du}{9.094962\du}}
\pgfpathlineto{\pgfpoint{17.846623\du}{9.089485\du}}
\pgfpathlineto{\pgfpoint{17.833479\du}{9.084739\du}}
\pgfpathlineto{\pgfpoint{17.819241\du}{9.079628\du}}
\pgfpathlineto{\pgfpoint{17.805367\du}{9.074881\du}}
\pgfpathlineto{\pgfpoint{17.791128\du}{9.069770\du}}
\pgfpathlineto{\pgfpoint{17.777254\du}{9.065024\du}}
\pgfpathlineto{\pgfpoint{17.762286\du}{9.060277\du}}
\pgfpathlineto{\pgfpoint{17.747317\du}{9.055166\du}}
\pgfpathlineto{\pgfpoint{17.733078\du}{9.050420\du}}
\pgfpathlineto{\pgfpoint{17.702410\du}{9.041658\du}}
\pgfpathlineto{\pgfpoint{17.671742\du}{9.032165\du}}
\pgfpathlineto{\pgfpoint{17.640343\du}{9.023403\du}}
\pgfpathlineto{\pgfpoint{17.608580\du}{9.015371\du}}
\pgfpathlineto{\pgfpoint{17.575721\du}{9.006973\du}}
\pgfpathlineto{\pgfpoint{17.542132\du}{8.998941\du}}
\pgfpathlineto{\pgfpoint{17.508178\du}{8.991274\du}}
\pgfpathlineto{\pgfpoint{17.473494\du}{8.983607\du}}
\pgfpathlineto{\pgfpoint{17.438810\du}{8.976670\du}}
\pgfpathlineto{\pgfpoint{17.403395\du}{8.969733\du}}
\pgfpathlineto{\pgfpoint{17.366886\du}{8.963162\du}}
\pgfpathlineto{\pgfpoint{17.330376\du}{8.957320\du}}
\pgfpathlineto{\pgfpoint{17.293501\du}{8.950748\du}}
\pgfpathlineto{\pgfpoint{17.256261\du}{8.945637\du}}
\pgfpathlineto{\pgfpoint{17.217196\du}{8.940161\du}}
\pgfpathlineto{\pgfpoint{17.179226\du}{8.935049\du}}
\pgfpathlineto{\pgfpoint{17.139796\du}{8.930303\du}}
\pgfpathlineto{\pgfpoint{17.101095\du}{8.925922\du}}
\pgfpathlineto{\pgfpoint{17.060935\du}{8.921906\du}}
\pgfpathlineto{\pgfpoint{17.021139\du}{8.918255\du}}
\pgfpathlineto{\pgfpoint{16.980248\du}{8.914604\du}}
\pgfpathlineto{\pgfpoint{16.939723\du}{8.911683\du}}
\pgfpathlineto{\pgfpoint{16.898101\du}{8.909493\du}}
\pgfpathlineto{\pgfpoint{16.856480\du}{8.906937\du}}
\pgfpathlineto{\pgfpoint{16.814859\du}{8.904746\du}}
\pgfpathlineto{\pgfpoint{16.772143\du}{8.902921\du}}
\pgfpathlineto{\pgfpoint{16.730157\du}{8.901825\du}}
\pgfpathlineto{\pgfpoint{16.687076\du}{8.900730\du}}
\pgfpathlineto{\pgfpoint{16.643629\du}{8.900000\du}}
\pgfpathlineto{\pgfpoint{16.600183\du}{8.900000\du}}
\pgfpathlineto{\pgfpoint{16.600183\du}{8.920445\du}}
\pgfpathlineto{\pgfpoint{16.643264\du}{8.921176\du}}
\pgfpathlineto{\pgfpoint{16.686710\du}{8.921541\du}}
\pgfpathlineto{\pgfpoint{16.728697\du}{8.921906\du}}
\pgfpathlineto{\pgfpoint{16.771413\du}{8.923366\du}}
\pgfpathlineto{\pgfpoint{16.814129\du}{8.925192\du}}
\pgfpathlineto{\pgfpoint{16.855750\du}{8.927017\du}}
\pgfpathlineto{\pgfpoint{16.897006\du}{8.929208\du}}
\pgfpathlineto{\pgfpoint{16.937897\du}{8.932129\du}}
\pgfpathlineto{\pgfpoint{16.979153\du}{8.935779\du}}
\pgfpathlineto{\pgfpoint{17.019314\du}{8.938700\du}}
\pgfpathlineto{\pgfpoint{17.059474\du}{8.942716\du}}
\pgfpathlineto{\pgfpoint{17.098540\du}{8.946367\du}}
\pgfpathlineto{\pgfpoint{17.137970\du}{8.950748\du}}
\pgfpathlineto{\pgfpoint{17.176670\du}{8.955495\du}}
\pgfpathlineto{\pgfpoint{17.215371\du}{8.960241\du}}
\pgfpathlineto{\pgfpoint{17.252976\du}{8.965352\du}}
\pgfpathlineto{\pgfpoint{17.290215\du}{8.971194\du}}
\pgfpathlineto{\pgfpoint{17.326725\du}{8.976670\du}}
\pgfpathlineto{\pgfpoint{17.363600\du}{8.983607\du}}
\pgfpathlineto{\pgfpoint{17.399014\du}{8.989814\du}}
\pgfpathlineto{\pgfpoint{17.434794\du}{8.997116\du}}
\pgfpathlineto{\pgfpoint{17.469478\du}{9.004053\du}}
\pgfpathlineto{\pgfpoint{17.503432\du}{9.010989\du}}
\pgfpathlineto{\pgfpoint{17.537751\du}{9.018656\du}}
\pgfpathlineto{\pgfpoint{17.570975\du}{9.027054\du}}
\pgfpathlineto{\pgfpoint{17.603103\du}{9.035086\du}}
\pgfpathlineto{\pgfpoint{17.635232\du}{9.043483\du}}
\pgfpathlineto{\pgfpoint{17.666630\du}{9.052245\du}}
\pgfpathlineto{\pgfpoint{17.697298\du}{9.061008\du}}
\pgfpathlineto{\pgfpoint{17.727236\du}{9.070135\du}}
\pgfpathlineto{\pgfpoint{17.741475\du}{9.074881\du}}
\pgfpathlineto{\pgfpoint{17.756079\du}{9.079628\du}}
\pgfpathlineto{\pgfpoint{17.769953\du}{9.084374\du}}
\pgfpathlineto{\pgfpoint{17.784556\du}{9.089120\du}}
\pgfpathlineto{\pgfpoint{17.798795\du}{9.094231\du}}
\pgfpathlineto{\pgfpoint{17.812669\du}{9.098978\du}}
\pgfpathlineto{\pgfpoint{17.826908\du}{9.104089\du}}
\pgfpathlineto{\pgfpoint{17.840051\du}{9.108470\du}}
\pgfpathlineto{\pgfpoint{17.853560\du}{9.113947\du}}
\pgfpathlineto{\pgfpoint{17.866703\du}{9.119423\du}}
\pgfpathlineto{\pgfpoint{17.879482\du}{9.124535\du}}
\pgfpathlineto{\pgfpoint{17.892260\du}{9.130011\du}}
\pgfpathlineto{\pgfpoint{17.905403\du}{9.135122\du}}
\pgfpathlineto{\pgfpoint{17.917452\du}{9.140599\du}}
\pgfpathlineto{\pgfpoint{17.929865\du}{9.146440\du}}
\pgfpathlineto{\pgfpoint{17.941913\du}{9.151552\du}}
\pgfpathlineto{\pgfpoint{17.954326\du}{9.156663\du}}
\pgfpathlineto{\pgfpoint{17.965644\du}{9.162505\du}}
\pgfpathlineto{\pgfpoint{17.977693\du}{9.168711\du}}
\pgfpathlineto{\pgfpoint{17.988280\du}{9.173823\du}}
\pgfpathlineto{\pgfpoint{18.000329\du}{9.179664\du}}
\pgfpathlineto{\pgfpoint{18.010551\du}{9.185506\du}}
\pgfpathlineto{\pgfpoint{18.021504\du}{9.191347\du}}
\pgfpathlineto{\pgfpoint{18.032457\du}{9.197189\du}}
\pgfpathlineto{\pgfpoint{18.042315\du}{9.203030\du}}
\pgfpathlineto{\pgfpoint{18.052172\du}{9.208872\du}}
\pgfpathlineto{\pgfpoint{18.062395\du}{9.215444\du}}
\pgfpathlineto{\pgfpoint{18.071522\du}{9.221285\du}}
\pgfpathlineto{\pgfpoint{18.081745\du}{9.227127\du}}
\pgfpathlineto{\pgfpoint{18.091238\du}{9.233333\du}}
\pgfpathlineto{\pgfpoint{18.100000\du}{9.239905\du}}
\pgfpathlineto{\pgfpoint{18.108032\du}{9.245382\du}}
\pgfpathlineto{\pgfpoint{18.116794\du}{9.252318\du}}
\pgfpathlineto{\pgfpoint{18.125192\du}{9.258160\du}}
\pgfpathlineto{\pgfpoint{18.133954\du}{9.264367\du}}
\pgfpathlineto{\pgfpoint{18.141621\du}{9.270938\du}}
\pgfpathlineto{\pgfpoint{18.149653\du}{9.277510\du}}
\pgfpathlineto{\pgfpoint{18.156955\du}{9.283717\du}}
\pgfpathlineto{\pgfpoint{18.164622\du}{9.290288\du}}
\pgfpathlineto{\pgfpoint{18.171194\du}{9.296860\du}}
\pgfpathlineto{\pgfpoint{18.178496\du}{9.303067\du}}
\pgfpathlineto{\pgfpoint{18.184337\du}{9.309639\du}}
\pgfpathlineto{\pgfpoint{18.191274\du}{9.316575\du}}
\pgfpathlineto{\pgfpoint{18.197846\du}{9.323147\du}}
\pgfpathlineto{\pgfpoint{18.202957\du}{9.329354\du}}
\pgfpathlineto{\pgfpoint{18.208434\du}{9.336291\du}}
\pgfpathlineto{\pgfpoint{18.214640\du}{9.342862\du}}
\pgfpathlineto{\pgfpoint{18.219387\du}{9.349799\du}}
\pgfpathlineto{\pgfpoint{18.224498\du}{9.356371\du}}
\pgfpathlineto{\pgfpoint{18.229244\du}{9.362943\du}}
\pgfpathlineto{\pgfpoint{18.233625\du}{9.369880\du}}
\pgfpathlineto{\pgfpoint{18.238007\du}{9.376816\du}}
\pgfpathlineto{\pgfpoint{18.240927\du}{9.383388\du}}
\pgfpathlineto{\pgfpoint{18.244943\du}{9.390325\du}}
\pgfpathlineto{\pgfpoint{18.247864\du}{9.396897\du}}
\pgfpathlineto{\pgfpoint{18.251880\du}{9.403834\du}}
\pgfpathlineto{\pgfpoint{18.254436\du}{9.410770\du}}
\pgfpathlineto{\pgfpoint{18.257357\du}{9.417707\du}}
\pgfpathlineto{\pgfpoint{18.259912\du}{9.424644\du}}
\pgfpathlineto{\pgfpoint{18.262103\du}{9.431946\du}}
\pgfpathlineto{\pgfpoint{18.263563\du}{9.438153\du}}
\pgfpathlineto{\pgfpoint{18.265754\du}{9.445455\du}}
\pgfpathlineto{\pgfpoint{18.266484\du}{9.452391\du}}
\pgfpathlineto{\pgfpoint{18.267579\du}{9.459328\du}}
\pgfpathlineto{\pgfpoint{18.269405\du}{9.466265\du}}
\pgfpathlineto{\pgfpoint{18.269770\du}{9.473567\du}}
\pgfpathlineto{\pgfpoint{18.269770\du}{9.480504\du}}
\pgfpathlineto{\pgfpoint{18.270500\du}{9.487441\du}}
\pgfpathlineto{\pgfpoint{18.290581\du}{9.487441\du}}
\pgfusepath{fill}
\pgfsetbuttcap
\pgfsetmiterjoin
\pgfsetdash{}{0pt}
\definecolor{dialinecolor}{rgb}{0.074510, 0.082353, 0.086275}
\pgfsetfillcolor{dialinecolor}
\pgfpathmoveto{\pgfpoint{16.643264\du}{9.360022\du}}
\pgfpathlineto{\pgfpoint{16.891165\du}{9.442534\du}}
\pgfpathlineto{\pgfpoint{17.476415\du}{9.208142\du}}
\pgfpathlineto{\pgfpoint{17.749142\du}{9.275685\du}}
\pgfpathlineto{\pgfpoint{17.605294\du}{9.067214\du}}
\pgfpathlineto{\pgfpoint{16.901022\du}{9.067214\du}}
\pgfpathlineto{\pgfpoint{17.195290\du}{9.139869\du}}
\pgfpathlineto{\pgfpoint{16.643264\du}{9.360022\du}}
\pgfusepath{fill}
\pgfsetbuttcap
\pgfsetmiterjoin
\pgfsetdash{}{0pt}
\definecolor{dialinecolor}{rgb}{0.074510, 0.082353, 0.086275}
\pgfsetfillcolor{dialinecolor}
\pgfpathmoveto{\pgfpoint{16.541402\du}{9.598065\du}}
\pgfpathlineto{\pgfpoint{16.293501\du}{9.515918\du}}
\pgfpathlineto{\pgfpoint{15.708251\du}{9.749580\du}}
\pgfpathlineto{\pgfpoint{15.435159\du}{9.682767\du}}
\pgfpathlineto{\pgfpoint{15.579007\du}{9.890507\du}}
\pgfpathlineto{\pgfpoint{16.284374\du}{9.890507\du}}
\pgfpathlineto{\pgfpoint{15.989376\du}{9.818583\du}}
\pgfpathlineto{\pgfpoint{16.541402\du}{9.598065\du}}
\pgfusepath{fill}
\pgfsetbuttcap
\pgfsetmiterjoin
\pgfsetdash{}{0pt}
\definecolor{dialinecolor}{rgb}{0.074510, 0.082353, 0.086275}
\pgfsetfillcolor{dialinecolor}
\pgfpathmoveto{\pgfpoint{15.495400\du}{9.139138\du}}
\pgfpathlineto{\pgfpoint{15.742935\du}{9.057357\du}}
\pgfpathlineto{\pgfpoint{16.328185\du}{9.290654\du}}
\pgfpathlineto{\pgfpoint{16.601278\du}{9.224206\du}}
\pgfpathlineto{\pgfpoint{16.457430\du}{9.431946\du}}
\pgfpathlineto{\pgfpoint{15.752428\du}{9.431946\du}}
\pgfpathlineto{\pgfpoint{16.047426\du}{9.360022\du}}
\pgfpathlineto{\pgfpoint{15.495400\du}{9.139138\du}}
\pgfusepath{fill}
\pgfsetbuttcap
\pgfsetmiterjoin
\pgfsetdash{}{0pt}
\definecolor{dialinecolor}{rgb}{0.074510, 0.082353, 0.086275}
\pgfsetfillcolor{dialinecolor}
\pgfpathmoveto{\pgfpoint{17.713728\du}{9.834283\du}}
\pgfpathlineto{\pgfpoint{17.466192\du}{9.916429\du}}
\pgfpathlineto{\pgfpoint{16.880942\du}{9.682767\du}}
\pgfpathlineto{\pgfpoint{16.607119\du}{9.749580\du}}
\pgfpathlineto{\pgfpoint{16.751698\du}{9.541840\du}}
\pgfpathlineto{\pgfpoint{17.457065\du}{9.541840\du}}
\pgfpathlineto{\pgfpoint{17.161701\du}{9.613764\du}}
\pgfpathlineto{\pgfpoint{17.713728\du}{9.834283\du}}
\pgfusepath{fill}
\pgfsetbuttcap
\pgfsetmiterjoin
\pgfsetdash{}{0pt}
\definecolor{dialinecolor}{rgb}{1.000000, 1.000000, 1.000000}
\pgfsetfillcolor{dialinecolor}
\pgfpathmoveto{\pgfpoint{16.664074\du}{9.380467\du}}
\pgfpathlineto{\pgfpoint{16.911610\du}{9.462979\du}}
\pgfpathlineto{\pgfpoint{17.496860\du}{9.228952\du}}
\pgfpathlineto{\pgfpoint{17.769222\du}{9.296130\du}}
\pgfpathlineto{\pgfpoint{17.626470\du}{9.087660\du}}
\pgfpathlineto{\pgfpoint{16.921103\du}{9.087660\du}}
\pgfpathlineto{\pgfpoint{17.216101\du}{9.160314\du}}
\pgfpathlineto{\pgfpoint{16.664074\du}{9.380467\du}}
\pgfusepath{fill}
\pgfsetbuttcap
\pgfsetmiterjoin
\pgfsetdash{}{0pt}
\definecolor{dialinecolor}{rgb}{1.000000, 1.000000, 1.000000}
\pgfsetfillcolor{dialinecolor}
\pgfpathmoveto{\pgfpoint{16.562212\du}{9.619241\du}}
\pgfpathlineto{\pgfpoint{16.313582\du}{9.536729\du}}
\pgfpathlineto{\pgfpoint{15.728697\du}{9.770756\du}}
\pgfpathlineto{\pgfpoint{15.455239\du}{9.703213\du}}
\pgfpathlineto{\pgfpoint{15.600183\du}{9.911683\du}}
\pgfpathlineto{\pgfpoint{16.304819\du}{9.911683\du}}
\pgfpathlineto{\pgfpoint{16.010186\du}{9.839029\du}}
\pgfpathlineto{\pgfpoint{16.562212\du}{9.619241\du}}
\pgfusepath{fill}
\pgfsetbuttcap
\pgfsetmiterjoin
\pgfsetdash{}{0pt}
\definecolor{dialinecolor}{rgb}{1.000000, 1.000000, 1.000000}
\pgfsetfillcolor{dialinecolor}
\pgfpathmoveto{\pgfpoint{15.515845\du}{9.159584\du}}
\pgfpathlineto{\pgfpoint{15.763381\du}{9.077802\du}}
\pgfpathlineto{\pgfpoint{16.348996\du}{9.311829\du}}
\pgfpathlineto{\pgfpoint{16.622088\du}{9.245016\du}}
\pgfpathlineto{\pgfpoint{16.477145\du}{9.452391\du}}
\pgfpathlineto{\pgfpoint{15.772873\du}{9.452391\du}}
\pgfpathlineto{\pgfpoint{16.067506\du}{9.380467\du}}
\pgfpathlineto{\pgfpoint{15.515845\du}{9.159584\du}}
\pgfusepath{fill}
\pgfsetbuttcap
\pgfsetmiterjoin
\pgfsetdash{}{0pt}
\definecolor{dialinecolor}{rgb}{1.000000, 1.000000, 1.000000}
\pgfsetfillcolor{dialinecolor}
\pgfpathmoveto{\pgfpoint{17.733808\du}{9.854728\du}}
\pgfpathlineto{\pgfpoint{17.486272\du}{9.936875\du}}
\pgfpathlineto{\pgfpoint{16.901387\du}{9.703213\du}}
\pgfpathlineto{\pgfpoint{16.627930\du}{9.770026\du}}
\pgfpathlineto{\pgfpoint{16.772143\du}{9.562286\du}}
\pgfpathlineto{\pgfpoint{17.477145\du}{9.562286\du}}
\pgfpathlineto{\pgfpoint{17.182877\du}{9.634210\du}}
\pgfpathlineto{\pgfpoint{17.733808\du}{9.854728\du}}
\pgfusepath{fill}
\pgfsetbuttcap
\pgfsetmiterjoin
\pgfsetdash{}{0pt}
\definecolor{dialinecolor}{rgb}{0.678431, 0.839216, 0.905882}
\pgfsetfillcolor{dialinecolor}
\pgfpathmoveto{\pgfpoint{14.929865\du}{9.498028\du}}
\pgfpathlineto{\pgfpoint{14.929865\du}{9.487441\du}}
\pgfpathlineto{\pgfpoint{14.909419\du}{9.487441\du}}
\pgfpathlineto{\pgfpoint{14.909419\du}{9.498028\du}}
\pgfpathlineto{\pgfpoint{14.929865\du}{9.498028\du}}
\pgfusepath{fill}
\pgfsetbuttcap
\pgfsetmiterjoin
\pgfsetdash{}{0pt}
\definecolor{dialinecolor}{rgb}{0.678431, 0.839216, 0.905882}
\pgfsetfillcolor{dialinecolor}
\pgfpathmoveto{\pgfpoint{14.929865\du}{10.327528\du}}
\pgfpathlineto{\pgfpoint{14.929865\du}{9.498028\du}}
\pgfpathlineto{\pgfpoint{14.909419\du}{9.498028\du}}
\pgfpathlineto{\pgfpoint{14.909419\du}{10.327528\du}}
\pgfpathlineto{\pgfpoint{14.929865\du}{10.327528\du}}
\pgfusepath{fill}
\pgfsetbuttcap
\pgfsetmiterjoin
\pgfsetdash{}{0pt}
\definecolor{dialinecolor}{rgb}{0.678431, 0.839216, 0.905882}
\pgfsetfillcolor{dialinecolor}
\pgfpathmoveto{\pgfpoint{14.909419\du}{10.327528\du}}
\pgfpathlineto{\pgfpoint{14.909419\du}{10.338116\du}}
\pgfpathlineto{\pgfpoint{14.929865\du}{10.338116\du}}
\pgfpathlineto{\pgfpoint{14.929865\du}{10.327528\du}}
\pgfpathlineto{\pgfpoint{14.909419\du}{10.327528\du}}
\pgfusepath{fill}
\pgfsetbuttcap
\pgfsetmiterjoin
\pgfsetdash{}{0pt}
\definecolor{dialinecolor}{rgb}{0.678431, 0.839216, 0.905882}
\pgfsetfillcolor{dialinecolor}
\pgfpathmoveto{\pgfpoint{18.290581\du}{9.498028\du}}
\pgfpathlineto{\pgfpoint{18.290581\du}{9.487441\du}}
\pgfpathlineto{\pgfpoint{18.270500\du}{9.487441\du}}
\pgfpathlineto{\pgfpoint{18.270500\du}{9.498028\du}}
\pgfpathlineto{\pgfpoint{18.290581\du}{9.498028\du}}
\pgfusepath{fill}
\pgfsetbuttcap
\pgfsetmiterjoin
\pgfsetdash{}{0pt}
\definecolor{dialinecolor}{rgb}{0.678431, 0.839216, 0.905882}
\pgfsetfillcolor{dialinecolor}
\pgfpathmoveto{\pgfpoint{18.290581\du}{10.327528\du}}
\pgfpathlineto{\pgfpoint{18.290581\du}{9.498028\du}}
\pgfpathlineto{\pgfpoint{18.270500\du}{9.498028\du}}
\pgfpathlineto{\pgfpoint{18.270500\du}{10.327528\du}}
\pgfpathlineto{\pgfpoint{18.290581\du}{10.327528\du}}
\pgfusepath{fill}
\pgfsetbuttcap
\pgfsetmiterjoin
\pgfsetdash{}{0pt}
\definecolor{dialinecolor}{rgb}{0.678431, 0.839216, 0.905882}
\pgfsetfillcolor{dialinecolor}
\pgfpathmoveto{\pgfpoint{18.270500\du}{10.327528\du}}
\pgfpathlineto{\pgfpoint{18.270500\du}{10.338116\du}}
\pgfpathlineto{\pgfpoint{18.290581\du}{10.338116\du}}
\pgfpathlineto{\pgfpoint{18.290581\du}{10.327528\du}}
\pgfpathlineto{\pgfpoint{18.270500\du}{10.327528\du}}
\pgfusepath{fill}
\pgfsetbuttcap
\pgfsetmiterjoin
\pgfsetdash{}{0pt}
\definecolor{dialinecolor}{rgb}{0.027451, 0.372549, 0.529412}
\pgfsetfillcolor{dialinecolor}
\pgfpathmoveto{\pgfpoint{17.216831\du}{10.476853\du}}
\pgfpathlineto{\pgfpoint{17.216466\du}{10.494378\du}}
\pgfpathlineto{\pgfpoint{17.213910\du}{10.511537\du}}
\pgfpathlineto{\pgfpoint{17.210259\du}{10.527601\du}}
\pgfpathlineto{\pgfpoint{17.204418\du}{10.544761\du}}
\pgfpathlineto{\pgfpoint{17.197846\du}{10.560460\du}}
\pgfpathlineto{\pgfpoint{17.189449\du}{10.576524\du}}
\pgfpathlineto{\pgfpoint{17.179956\du}{10.592223\du}}
\pgfpathlineto{\pgfpoint{17.168273\du}{10.607192\du}}
\pgfpathlineto{\pgfpoint{17.156590\du}{10.622161\du}}
\pgfpathlineto{\pgfpoint{17.143081\du}{10.636765\du}}
\pgfpathlineto{\pgfpoint{17.128112\du}{10.650639\du}}
\pgfpathlineto{\pgfpoint{17.112048\du}{10.664878\du}}
\pgfpathlineto{\pgfpoint{17.094524\du}{10.677656\du}}
\pgfpathlineto{\pgfpoint{17.076269\du}{10.690434\du}}
\pgfpathlineto{\pgfpoint{17.057284\du}{10.702848\du}}
\pgfpathlineto{\pgfpoint{17.037203\du}{10.714531\du}}
\pgfpathlineto{\pgfpoint{17.015663\du}{10.725119\du}}
\pgfpathlineto{\pgfpoint{16.993027\du}{10.736072\du}}
\pgfpathlineto{\pgfpoint{16.970026\du}{10.745929\du}}
\pgfpathlineto{\pgfpoint{16.946294\du}{10.755422\du}}
\pgfpathlineto{\pgfpoint{16.921103\du}{10.763454\du}}
\pgfpathlineto{\pgfpoint{16.895546\du}{10.771851\du}}
\pgfpathlineto{\pgfpoint{16.868894\du}{10.779518\du}}
\pgfpathlineto{\pgfpoint{16.842242\du}{10.786455\du}}
\pgfpathlineto{\pgfpoint{16.814129\du}{10.792296\du}}
\pgfpathlineto{\pgfpoint{16.785652\du}{10.797408\du}}
\pgfpathlineto{\pgfpoint{16.756079\du}{10.801789\du}}
\pgfpathlineto{\pgfpoint{16.725776\du}{10.805805\du}}
\pgfpathlineto{\pgfpoint{16.695838\du}{10.808726\du}}
\pgfpathlineto{\pgfpoint{16.665170\du}{10.810916\du}}
\pgfpathlineto{\pgfpoint{16.633771\du}{10.812012\du}}
\pgfpathlineto{\pgfpoint{16.602373\du}{10.812742\du}}
\pgfpathlineto{\pgfpoint{16.571340\du}{10.812012\du}}
\pgfpathlineto{\pgfpoint{16.539942\du}{10.810916\du}}
\pgfpathlineto{\pgfpoint{16.508908\du}{10.808726\du}}
\pgfpathlineto{\pgfpoint{16.478605\du}{10.805805\du}}
\pgfpathlineto{\pgfpoint{16.449032\du}{10.801789\du}}
\pgfpathlineto{\pgfpoint{16.419460\du}{10.797408\du}}
\pgfpathlineto{\pgfpoint{16.390982\du}{10.792296\du}}
\pgfpathlineto{\pgfpoint{16.363235\du}{10.786455\du}}
\pgfpathlineto{\pgfpoint{16.335853\du}{10.779518\du}}
\pgfpathlineto{\pgfpoint{16.309566\du}{10.771851\du}}
\pgfpathlineto{\pgfpoint{16.284374\du}{10.763454\du}}
\pgfpathlineto{\pgfpoint{16.258817\du}{10.755422\du}}
\pgfpathlineto{\pgfpoint{16.234721\du}{10.745929\du}}
\pgfpathlineto{\pgfpoint{16.212085\du}{10.736072\du}}
\pgfpathlineto{\pgfpoint{16.189449\du}{10.725119\du}}
\pgfpathlineto{\pgfpoint{16.168638\du}{10.714531\du}}
\pgfpathlineto{\pgfpoint{16.148193\du}{10.702848\du}}
\pgfpathlineto{\pgfpoint{16.127747\du}{10.690434\du}}
\pgfpathlineto{\pgfpoint{16.109858\du}{10.677656\du}}
\pgfpathlineto{\pgfpoint{16.093428\du}{10.664878\du}}
\pgfpathlineto{\pgfpoint{16.076634\du}{10.650639\du}}
\pgfpathlineto{\pgfpoint{16.062030\du}{10.636765\du}}
\pgfpathlineto{\pgfpoint{16.048886\du}{10.622161\du}}
\pgfpathlineto{\pgfpoint{16.036473\du}{10.607192\du}}
\pgfpathlineto{\pgfpoint{16.025520\du}{10.592223\du}}
\pgfpathlineto{\pgfpoint{16.016028\du}{10.576524\du}}
\pgfpathlineto{\pgfpoint{16.007265\du}{10.560460\du}}
\pgfpathlineto{\pgfpoint{16.000329\du}{10.544761\du}}
\pgfpathlineto{\pgfpoint{15.995582\du}{10.527601\du}}
\pgfpathlineto{\pgfpoint{15.991201\du}{10.511537\du}}
\pgfpathlineto{\pgfpoint{15.988645\du}{10.494378\du}}
\pgfpathlineto{\pgfpoint{15.987550\du}{10.476853\du}}
\pgfpathlineto{\pgfpoint{15.988645\du}{10.459328\du}}
\pgfpathlineto{\pgfpoint{15.991201\du}{10.442169\du}}
\pgfpathlineto{\pgfpoint{15.995582\du}{10.426104\du}}
\pgfpathlineto{\pgfpoint{16.000329\du}{10.408945\du}}
\pgfpathlineto{\pgfpoint{16.007265\du}{10.393246\du}}
\pgfpathlineto{\pgfpoint{16.016028\du}{10.377547\du}}
\pgfpathlineto{\pgfpoint{16.025520\du}{10.361482\du}}
\pgfpathlineto{\pgfpoint{16.036473\du}{10.346513\du}}
\pgfpathlineto{\pgfpoint{16.048886\du}{10.331909\du}}
\pgfpathlineto{\pgfpoint{16.062030\du}{10.317306\du}}
\pgfpathlineto{\pgfpoint{16.076634\du}{10.303067\du}}
\pgfpathlineto{\pgfpoint{16.093428\du}{10.289193\du}}
\pgfpathlineto{\pgfpoint{16.109858\du}{10.276050\du}}
\pgfpathlineto{\pgfpoint{16.127747\du}{10.264001\du}}
\pgfpathlineto{\pgfpoint{16.148193\du}{10.251588\du}}
\pgfpathlineto{\pgfpoint{16.168638\du}{10.239905\du}}
\pgfpathlineto{\pgfpoint{16.189449\du}{10.228952\du}}
\pgfpathlineto{\pgfpoint{16.212085\du}{10.218364\du}}
\pgfpathlineto{\pgfpoint{16.234721\du}{10.207777\du}}
\pgfpathlineto{\pgfpoint{16.258817\du}{10.199014\du}}
\pgfpathlineto{\pgfpoint{16.284374\du}{10.190252\du}}
\pgfpathlineto{\pgfpoint{16.309566\du}{10.181855\du}}
\pgfpathlineto{\pgfpoint{16.335853\du}{10.174188\du}}
\pgfpathlineto{\pgfpoint{16.363235\du}{10.167981\du}}
\pgfpathlineto{\pgfpoint{16.390982\du}{10.161409\du}}
\pgfpathlineto{\pgfpoint{16.419460\du}{10.156298\du}}
\pgfpathlineto{\pgfpoint{16.449032\du}{10.152282\du}}
\pgfpathlineto{\pgfpoint{16.478605\du}{10.147901\du}}
\pgfpathlineto{\pgfpoint{16.508908\du}{10.144980\du}}
\pgfpathlineto{\pgfpoint{16.539942\du}{10.143520\du}}
\pgfpathlineto{\pgfpoint{16.571340\du}{10.141694\du}}
\pgfpathlineto{\pgfpoint{16.602373\du}{10.141694\du}}
\pgfpathlineto{\pgfpoint{16.633771\du}{10.141694\du}}
\pgfpathlineto{\pgfpoint{16.665170\du}{10.143520\du}}
\pgfpathlineto{\pgfpoint{16.695838\du}{10.144980\du}}
\pgfpathlineto{\pgfpoint{16.725776\du}{10.147901\du}}
\pgfpathlineto{\pgfpoint{16.756079\du}{10.152282\du}}
\pgfpathlineto{\pgfpoint{16.785652\du}{10.156298\du}}
\pgfpathlineto{\pgfpoint{16.814129\du}{10.161409\du}}
\pgfpathlineto{\pgfpoint{16.842242\du}{10.167981\du}}
\pgfpathlineto{\pgfpoint{16.868894\du}{10.174188\du}}
\pgfpathlineto{\pgfpoint{16.895546\du}{10.181855\du}}
\pgfpathlineto{\pgfpoint{16.921103\du}{10.190252\du}}
\pgfpathlineto{\pgfpoint{16.946294\du}{10.199014\du}}
\pgfpathlineto{\pgfpoint{16.970026\du}{10.207777\du}}
\pgfpathlineto{\pgfpoint{16.993027\du}{10.218364\du}}
\pgfpathlineto{\pgfpoint{17.015663\du}{10.228952\du}}
\pgfpathlineto{\pgfpoint{17.037203\du}{10.239905\du}}
\pgfpathlineto{\pgfpoint{17.057284\du}{10.251588\du}}
\pgfpathlineto{\pgfpoint{17.076269\du}{10.264001\du}}
\pgfpathlineto{\pgfpoint{17.094524\du}{10.276050\du}}
\pgfpathlineto{\pgfpoint{17.112048\du}{10.289193\du}}
\pgfpathlineto{\pgfpoint{17.128112\du}{10.303067\du}}
\pgfpathlineto{\pgfpoint{17.143081\du}{10.317306\du}}
\pgfpathlineto{\pgfpoint{17.156590\du}{10.331909\du}}
\pgfpathlineto{\pgfpoint{17.168273\du}{10.346513\du}}
\pgfpathlineto{\pgfpoint{17.179956\du}{10.361482\du}}
\pgfpathlineto{\pgfpoint{17.189449\du}{10.377547\du}}
\pgfpathlineto{\pgfpoint{17.197846\du}{10.393246\du}}
\pgfpathlineto{\pgfpoint{17.204418\du}{10.408945\du}}
\pgfpathlineto{\pgfpoint{17.210259\du}{10.426104\du}}
\pgfpathlineto{\pgfpoint{17.213910\du}{10.442169\du}}
\pgfpathlineto{\pgfpoint{17.216466\du}{10.459328\du}}
\pgfpathlineto{\pgfpoint{17.216831\du}{10.476853\du}}
\pgfusepath{fill}
\pgfsetbuttcap
\pgfsetmiterjoin
\pgfsetdash{}{0pt}
\definecolor{dialinecolor}{rgb}{0.678431, 0.839216, 0.905882}
\pgfsetfillcolor{dialinecolor}
\pgfpathmoveto{\pgfpoint{16.602373\du}{10.822599\du}}
\pgfpathlineto{\pgfpoint{16.602373\du}{10.822599\du}}
\pgfpathlineto{\pgfpoint{16.618437\du}{10.822599\du}}
\pgfpathlineto{\pgfpoint{16.634502\du}{10.822234\du}}
\pgfpathlineto{\pgfpoint{16.650566\du}{10.821504\du}}
\pgfpathlineto{\pgfpoint{16.665900\du}{10.820774\du}}
\pgfpathlineto{\pgfpoint{16.681599\du}{10.819679\du}}
\pgfpathlineto{\pgfpoint{16.696933\du}{10.818583\du}}
\pgfpathlineto{\pgfpoint{16.711902\du}{10.817488\du}}
\pgfpathlineto{\pgfpoint{16.727966\du}{10.815663\du}}
\pgfpathlineto{\pgfpoint{16.742205\du}{10.813837\du}}
\pgfpathlineto{\pgfpoint{16.757174\du}{10.812012\du}}
\pgfpathlineto{\pgfpoint{16.772143\du}{10.809821\du}}
\pgfpathlineto{\pgfpoint{16.787477\du}{10.807631\du}}
\pgfpathlineto{\pgfpoint{16.801716\du}{10.804710\du}}
\pgfpathlineto{\pgfpoint{16.815590\du}{10.802154\du}}
\pgfpathlineto{\pgfpoint{16.830194\du}{10.799233\du}}
\pgfpathlineto{\pgfpoint{16.844067\du}{10.796313\du}}
\pgfpathlineto{\pgfpoint{16.857576\du}{10.792662\du}}
\pgfpathlineto{\pgfpoint{16.871449\du}{10.789376\du}}
\pgfpathlineto{\pgfpoint{16.884958\du}{10.785725\du}}
\pgfpathlineto{\pgfpoint{16.898101\du}{10.781709\du}}
\pgfpathlineto{\pgfpoint{16.911245\du}{10.777693\du}}
\pgfpathlineto{\pgfpoint{16.924388\du}{10.773677\du}}
\pgfpathlineto{\pgfpoint{16.937167\du}{10.768930\du}}
\pgfpathlineto{\pgfpoint{16.949945\du}{10.764914\du}}
\pgfpathlineto{\pgfpoint{16.961628\du}{10.760168\du}}
\pgfpathlineto{\pgfpoint{16.974407\du}{10.755422\du}}
\pgfpathlineto{\pgfpoint{16.985725\du}{10.749945\du}}
\pgfpathlineto{\pgfpoint{16.997408\du}{10.744834\du}}
\pgfpathlineto{\pgfpoint{17.008726\du}{10.739723\du}}
\pgfpathlineto{\pgfpoint{17.020044\du}{10.734246\du}}
\pgfpathlineto{\pgfpoint{17.030997\du}{10.729135\du}}
\pgfpathlineto{\pgfpoint{17.042315\du}{10.723293\du}}
\pgfpathlineto{\pgfpoint{17.052172\du}{10.717452\du}}
\pgfpathlineto{\pgfpoint{17.062395\du}{10.710880\du}}
\pgfpathlineto{\pgfpoint{17.072253\du}{10.705038\du}}
\pgfpathlineto{\pgfpoint{17.082475\du}{10.698467\du}}
\pgfpathlineto{\pgfpoint{17.091968\du}{10.692260\du}}
\pgfpathlineto{\pgfpoint{17.101095\du}{10.685688\du}}
\pgfpathlineto{\pgfpoint{17.109858\du}{10.679482\du}}
\pgfpathlineto{\pgfpoint{17.118255\du}{10.672180\du}}
\pgfpathlineto{\pgfpoint{17.126652\du}{10.665243\du}}
\pgfpathlineto{\pgfpoint{17.134684\du}{10.658306\du}}
\pgfpathlineto{\pgfpoint{17.142716\du}{10.651369\du}}
\pgfpathlineto{\pgfpoint{17.149653\du}{10.644067\du}}
\pgfpathlineto{\pgfpoint{17.157320\du}{10.636765\du}}
\pgfpathlineto{\pgfpoint{17.163892\du}{10.629098\du}}
\pgfpathlineto{\pgfpoint{17.170464\du}{10.621431\du}}
\pgfpathlineto{\pgfpoint{17.176670\du}{10.613764\du}}
\pgfpathlineto{\pgfpoint{17.182877\du}{10.605732\du}}
\pgfpathlineto{\pgfpoint{17.188719\du}{10.598065\du}}
\pgfpathlineto{\pgfpoint{17.193465\du}{10.589668\du}}
\pgfpathlineto{\pgfpoint{17.198211\du}{10.581636\du}}
\pgfpathlineto{\pgfpoint{17.202957\du}{10.573238\du}}
\pgfpathlineto{\pgfpoint{17.207338\du}{10.565206\du}}
\pgfpathlineto{\pgfpoint{17.210624\du}{10.556444\du}}
\pgfpathlineto{\pgfpoint{17.214640\du}{10.548412\du}}
\pgfpathlineto{\pgfpoint{17.217196\du}{10.539650\du}}
\pgfpathlineto{\pgfpoint{17.220117\du}{10.530887\du}}
\pgfpathlineto{\pgfpoint{17.221942\du}{10.521760\du}}
\pgfpathlineto{\pgfpoint{17.224133\du}{10.512997\du}}
\pgfpathlineto{\pgfpoint{17.225593\du}{10.504235\du}}
\pgfpathlineto{\pgfpoint{17.226689\du}{10.495108\du}}
\pgfpathlineto{\pgfpoint{17.227419\du}{10.486345\du}}
\pgfpathlineto{\pgfpoint{17.227419\du}{10.476853\du}}
\pgfpathlineto{\pgfpoint{17.207338\du}{10.476853\du}}
\pgfpathlineto{\pgfpoint{17.206608\du}{10.484885\du}}
\pgfpathlineto{\pgfpoint{17.206243\du}{10.493282\du}}
\pgfpathlineto{\pgfpoint{17.205513\du}{10.501314\du}}
\pgfpathlineto{\pgfpoint{17.204053\du}{10.508981\du}}
\pgfpathlineto{\pgfpoint{17.202592\du}{10.517379\du}}
\pgfpathlineto{\pgfpoint{17.200037\du}{10.525411\du}}
\pgfpathlineto{\pgfpoint{17.197846\du}{10.533078\du}}
\pgfpathlineto{\pgfpoint{17.194560\du}{10.540745\du}}
\pgfpathlineto{\pgfpoint{17.192004\du}{10.548412\du}}
\pgfpathlineto{\pgfpoint{17.188719\du}{10.556444\du}}
\pgfpathlineto{\pgfpoint{17.184702\du}{10.564111\du}}
\pgfpathlineto{\pgfpoint{17.179956\du}{10.571778\du}}
\pgfpathlineto{\pgfpoint{17.175940\du}{10.579445\du}}
\pgfpathlineto{\pgfpoint{17.170829\du}{10.586382\du}}
\pgfpathlineto{\pgfpoint{17.166083\du}{10.594049\du}}
\pgfpathlineto{\pgfpoint{17.160606\du}{10.600986\du}}
\pgfpathlineto{\pgfpoint{17.155495\du}{10.608653\du}}
\pgfpathlineto{\pgfpoint{17.148558\du}{10.615590\du}}
\pgfpathlineto{\pgfpoint{17.142716\du}{10.622526\du}}
\pgfpathlineto{\pgfpoint{17.135414\du}{10.629463\du}}
\pgfpathlineto{\pgfpoint{17.128478\du}{10.636765\du}}
\pgfpathlineto{\pgfpoint{17.121176\du}{10.642972\du}}
\pgfpathlineto{\pgfpoint{17.114239\du}{10.649909\du}}
\pgfpathlineto{\pgfpoint{17.105842\du}{10.656480\du}}
\pgfpathlineto{\pgfpoint{17.097809\du}{10.663052\du}}
\pgfpathlineto{\pgfpoint{17.088682\du}{10.669259\du}}
\pgfpathlineto{\pgfpoint{17.079920\du}{10.675100\du}}
\pgfpathlineto{\pgfpoint{17.070427\du}{10.681672\du}}
\pgfpathlineto{\pgfpoint{17.061300\du}{10.688244\du}}
\pgfpathlineto{\pgfpoint{17.052172\du}{10.693355\du}}
\pgfpathlineto{\pgfpoint{17.042315\du}{10.699197\du}}
\pgfpathlineto{\pgfpoint{17.032457\du}{10.705038\du}}
\pgfpathlineto{\pgfpoint{17.021869\du}{10.710515\du}}
\pgfpathlineto{\pgfpoint{17.011281\du}{10.716356\du}}
\pgfpathlineto{\pgfpoint{17.000694\du}{10.720737\du}}
\pgfpathlineto{\pgfpoint{16.989011\du}{10.726579\du}}
\pgfpathlineto{\pgfpoint{16.977693\du}{10.731325\du}}
\pgfpathlineto{\pgfpoint{16.966375\du}{10.736072\du}}
\pgfpathlineto{\pgfpoint{16.954691\du}{10.740818\du}}
\pgfpathlineto{\pgfpoint{16.942643\du}{10.745564\du}}
\pgfpathlineto{\pgfpoint{16.930230\du}{10.749580\du}}
\pgfpathlineto{\pgfpoint{16.918547\du}{10.754326\du}}
\pgfpathlineto{\pgfpoint{16.905769\du}{10.758342\du}}
\pgfpathlineto{\pgfpoint{16.892260\du}{10.761993\du}}
\pgfpathlineto{\pgfpoint{16.879482\du}{10.766009\du}}
\pgfpathlineto{\pgfpoint{16.866338\du}{10.769295\du}}
\pgfpathlineto{\pgfpoint{16.852829\du}{10.772946\du}}
\pgfpathlineto{\pgfpoint{16.839321\du}{10.775867\du}}
\pgfpathlineto{\pgfpoint{16.825447\du}{10.779518\du}}
\pgfpathlineto{\pgfpoint{16.811574\du}{10.781709\du}}
\pgfpathlineto{\pgfpoint{16.797700\du}{10.784629\du}}
\pgfpathlineto{\pgfpoint{16.783461\du}{10.786820\du}}
\pgfpathlineto{\pgfpoint{16.769222\du}{10.789376\du}}
\pgfpathlineto{\pgfpoint{16.754984\du}{10.791566\du}}
\pgfpathlineto{\pgfpoint{16.740380\du}{10.793392\du}}
\pgfpathlineto{\pgfpoint{16.725411\du}{10.795217\du}}
\pgfpathlineto{\pgfpoint{16.710077\du}{10.797043\du}}
\pgfpathlineto{\pgfpoint{16.695473\du}{10.798138\du}}
\pgfpathlineto{\pgfpoint{16.679774\du}{10.799233\du}}
\pgfpathlineto{\pgfpoint{16.664805\du}{10.800329\du}}
\pgfpathlineto{\pgfpoint{16.649471\du}{10.801059\du}}
\pgfpathlineto{\pgfpoint{16.633771\du}{10.801789\du}}
\pgfpathlineto{\pgfpoint{16.618437\du}{10.802154\du}}
\pgfpathlineto{\pgfpoint{16.602373\du}{10.802154\du}}
\pgfpathlineto{\pgfpoint{16.602373\du}{10.802154\du}}
\pgfpathlineto{\pgfpoint{16.602373\du}{10.802154\du}}
\pgfpathlineto{\pgfpoint{16.601278\du}{10.802154\du}}
\pgfpathlineto{\pgfpoint{16.600183\du}{10.802154\du}}
\pgfpathlineto{\pgfpoint{16.599452\du}{10.802884\du}}
\pgfpathlineto{\pgfpoint{16.597627\du}{10.802884\du}}
\pgfpathlineto{\pgfpoint{16.597262\du}{10.803249\du}}
\pgfpathlineto{\pgfpoint{16.596166\du}{10.803980\du}}
\pgfpathlineto{\pgfpoint{16.595801\du}{10.804710\du}}
\pgfpathlineto{\pgfpoint{16.595436\du}{10.805075\du}}
\pgfpathlineto{\pgfpoint{16.593976\du}{10.806900\du}}
\pgfpathlineto{\pgfpoint{16.592516\du}{10.808726\du}}
\pgfpathlineto{\pgfpoint{16.592516\du}{10.810551\du}}
\pgfpathlineto{\pgfpoint{16.592150\du}{10.812742\du}}
\pgfpathlineto{\pgfpoint{16.592516\du}{10.814567\du}}
\pgfpathlineto{\pgfpoint{16.592516\du}{10.816393\du}}
\pgfpathlineto{\pgfpoint{16.593976\du}{10.817853\du}}
\pgfpathlineto{\pgfpoint{16.595436\du}{10.819314\du}}
\pgfpathlineto{\pgfpoint{16.595801\du}{10.820409\du}}
\pgfpathlineto{\pgfpoint{16.596166\du}{10.820774\du}}
\pgfpathlineto{\pgfpoint{16.597262\du}{10.821504\du}}
\pgfpathlineto{\pgfpoint{16.597627\du}{10.821504\du}}
\pgfpathlineto{\pgfpoint{16.599452\du}{10.822234\du}}
\pgfpathlineto{\pgfpoint{16.600183\du}{10.822599\du}}
\pgfpathlineto{\pgfpoint{16.601278\du}{10.822599\du}}
\pgfpathlineto{\pgfpoint{16.602373\du}{10.822599\du}}
\pgfusepath{fill}
\pgfsetbuttcap
\pgfsetmiterjoin
\pgfsetdash{}{0pt}
\definecolor{dialinecolor}{rgb}{0.678431, 0.839216, 0.905882}
\pgfsetfillcolor{dialinecolor}
\pgfpathmoveto{\pgfpoint{15.977693\du}{10.476853\du}}
\pgfpathlineto{\pgfpoint{15.977693\du}{10.476853\du}}
\pgfpathlineto{\pgfpoint{15.977693\du}{10.485615\du}}
\pgfpathlineto{\pgfpoint{15.978423\du}{10.495108\du}}
\pgfpathlineto{\pgfpoint{15.979518\du}{10.504235\du}}
\pgfpathlineto{\pgfpoint{15.981709\du}{10.512997\du}}
\pgfpathlineto{\pgfpoint{15.982804\du}{10.521760\du}}
\pgfpathlineto{\pgfpoint{15.985360\du}{10.530887\du}}
\pgfpathlineto{\pgfpoint{15.987550\du}{10.539650\du}}
\pgfpathlineto{\pgfpoint{15.990836\du}{10.548412\du}}
\pgfpathlineto{\pgfpoint{15.994122\du}{10.556444\du}}
\pgfpathlineto{\pgfpoint{15.998138\du}{10.565206\du}}
\pgfpathlineto{\pgfpoint{16.002519\du}{10.573238\du}}
\pgfpathlineto{\pgfpoint{16.006900\du}{10.581636\du}}
\pgfpathlineto{\pgfpoint{16.011647\du}{10.589668\du}}
\pgfpathlineto{\pgfpoint{16.016758\du}{10.598065\du}}
\pgfpathlineto{\pgfpoint{16.022965\du}{10.605732\du}}
\pgfpathlineto{\pgfpoint{16.027711\du}{10.613764\du}}
\pgfpathlineto{\pgfpoint{16.034283\du}{10.621431\du}}
\pgfpathlineto{\pgfpoint{16.041219\du}{10.629098\du}}
\pgfpathlineto{\pgfpoint{16.047791\du}{10.636765\du}}
\pgfpathlineto{\pgfpoint{16.054728\du}{10.644067\du}}
\pgfpathlineto{\pgfpoint{16.062030\du}{10.651369\du}}
\pgfpathlineto{\pgfpoint{16.070792\du}{10.658306\du}}
\pgfpathlineto{\pgfpoint{16.077729\du}{10.665243\du}}
\pgfpathlineto{\pgfpoint{16.086857\du}{10.672180\du}}
\pgfpathlineto{\pgfpoint{16.094889\du}{10.679482\du}}
\pgfpathlineto{\pgfpoint{16.104016\du}{10.685688\du}}
\pgfpathlineto{\pgfpoint{16.113874\du}{10.692260\du}}
\pgfpathlineto{\pgfpoint{16.123001\du}{10.698467\du}}
\pgfpathlineto{\pgfpoint{16.132494\du}{10.705038\du}}
\pgfpathlineto{\pgfpoint{16.142351\du}{10.710880\du}}
\pgfpathlineto{\pgfpoint{16.152574\du}{10.717452\du}}
\pgfpathlineto{\pgfpoint{16.162797\du}{10.723293\du}}
\pgfpathlineto{\pgfpoint{16.173750\du}{10.729135\du}}
\pgfpathlineto{\pgfpoint{16.185068\du}{10.734246\du}}
\pgfpathlineto{\pgfpoint{16.196386\du}{10.739723\du}}
\pgfpathlineto{\pgfpoint{16.207704\du}{10.744834\du}}
\pgfpathlineto{\pgfpoint{16.219752\du}{10.749945\du}}
\pgfpathlineto{\pgfpoint{16.231070\du}{10.755422\du}}
\pgfpathlineto{\pgfpoint{16.243483\du}{10.760168\du}}
\pgfpathlineto{\pgfpoint{16.255531\du}{10.764914\du}}
\pgfpathlineto{\pgfpoint{16.267579\du}{10.768930\du}}
\pgfpathlineto{\pgfpoint{16.280723\du}{10.773677\du}}
\pgfpathlineto{\pgfpoint{16.293501\du}{10.777693\du}}
\pgfpathlineto{\pgfpoint{16.307010\du}{10.781709\du}}
\pgfpathlineto{\pgfpoint{16.320518\du}{10.785725\du}}
\pgfpathlineto{\pgfpoint{16.333297\du}{10.789376\du}}
\pgfpathlineto{\pgfpoint{16.346805\du}{10.792662\du}}
\pgfpathlineto{\pgfpoint{16.360679\du}{10.796313\du}}
\pgfpathlineto{\pgfpoint{16.375283\du}{10.799233\du}}
\pgfpathlineto{\pgfpoint{16.389887\du}{10.802154\du}}
\pgfpathlineto{\pgfpoint{16.403760\du}{10.804710\du}}
\pgfpathlineto{\pgfpoint{16.417999\du}{10.807631\du}}
\pgfpathlineto{\pgfpoint{16.432968\du}{10.809821\du}}
\pgfpathlineto{\pgfpoint{16.447937\du}{10.812012\du}}
\pgfpathlineto{\pgfpoint{16.462541\du}{10.813837\du}}
\pgfpathlineto{\pgfpoint{16.477145\du}{10.815663\du}}
\pgfpathlineto{\pgfpoint{16.492844\du}{10.817488\du}}
\pgfpathlineto{\pgfpoint{16.508543\du}{10.818583\du}}
\pgfpathlineto{\pgfpoint{16.523147\du}{10.819679\du}}
\pgfpathlineto{\pgfpoint{16.539211\du}{10.820774\du}}
\pgfpathlineto{\pgfpoint{16.554545\du}{10.821504\du}}
\pgfpathlineto{\pgfpoint{16.570610\du}{10.822234\du}}
\pgfpathlineto{\pgfpoint{16.586674\du}{10.822599\du}}
\pgfpathlineto{\pgfpoint{16.602373\du}{10.822599\du}}
\pgfpathlineto{\pgfpoint{16.602373\du}{10.802154\du}}
\pgfpathlineto{\pgfpoint{16.586674\du}{10.802154\du}}
\pgfpathlineto{\pgfpoint{16.571340\du}{10.801789\du}}
\pgfpathlineto{\pgfpoint{16.555276\du}{10.801059\du}}
\pgfpathlineto{\pgfpoint{16.540672\du}{10.800329\du}}
\pgfpathlineto{\pgfpoint{16.524973\du}{10.799233\du}}
\pgfpathlineto{\pgfpoint{16.510004\du}{10.798138\du}}
\pgfpathlineto{\pgfpoint{16.495035\du}{10.797043\du}}
\pgfpathlineto{\pgfpoint{16.480066\du}{10.795217\du}}
\pgfpathlineto{\pgfpoint{16.465097\du}{10.793392\du}}
\pgfpathlineto{\pgfpoint{16.450493\du}{10.791566\du}}
\pgfpathlineto{\pgfpoint{16.435889\du}{10.789376\du}}
\pgfpathlineto{\pgfpoint{16.421650\du}{10.786820\du}}
\pgfpathlineto{\pgfpoint{16.407777\du}{10.784629\du}}
\pgfpathlineto{\pgfpoint{16.393903\du}{10.781709\du}}
\pgfpathlineto{\pgfpoint{16.379664\du}{10.779518\du}}
\pgfpathlineto{\pgfpoint{16.365790\du}{10.775867\du}}
\pgfpathlineto{\pgfpoint{16.352282\du}{10.772946\du}}
\pgfpathlineto{\pgfpoint{16.339138\du}{10.769295\du}}
\pgfpathlineto{\pgfpoint{16.325630\du}{10.766009\du}}
\pgfpathlineto{\pgfpoint{16.312851\du}{10.761993\du}}
\pgfpathlineto{\pgfpoint{16.299708\du}{10.758342\du}}
\pgfpathlineto{\pgfpoint{16.286930\du}{10.754326\du}}
\pgfpathlineto{\pgfpoint{16.274881\du}{10.749580\du}}
\pgfpathlineto{\pgfpoint{16.262468\du}{10.745564\du}}
\pgfpathlineto{\pgfpoint{16.250055\du}{10.740818\du}}
\pgfpathlineto{\pgfpoint{16.238737\du}{10.736072\du}}
\pgfpathlineto{\pgfpoint{16.227054\du}{10.731325\du}}
\pgfpathlineto{\pgfpoint{16.216101\du}{10.726579\du}}
\pgfpathlineto{\pgfpoint{16.204783\du}{10.720737\du}}
\pgfpathlineto{\pgfpoint{16.194195\du}{10.716356\du}}
\pgfpathlineto{\pgfpoint{16.183242\du}{10.710515\du}}
\pgfpathlineto{\pgfpoint{16.173019\du}{10.705038\du}}
\pgfpathlineto{\pgfpoint{16.162797\du}{10.699197\du}}
\pgfpathlineto{\pgfpoint{16.152939\du}{10.693355\du}}
\pgfpathlineto{\pgfpoint{16.143447\du}{10.688244\du}}
\pgfpathlineto{\pgfpoint{16.133589\du}{10.681672\du}}
\pgfpathlineto{\pgfpoint{16.125192\du}{10.675100\du}}
\pgfpathlineto{\pgfpoint{16.116064\du}{10.669259\du}}
\pgfpathlineto{\pgfpoint{16.107667\du}{10.663052\du}}
\pgfpathlineto{\pgfpoint{16.099270\du}{10.656480\du}}
\pgfpathlineto{\pgfpoint{16.091238\du}{10.649909\du}}
\pgfpathlineto{\pgfpoint{16.083936\du}{10.642972\du}}
\pgfpathlineto{\pgfpoint{16.076269\du}{10.636765\du}}
\pgfpathlineto{\pgfpoint{16.069697\du}{10.629463\du}}
\pgfpathlineto{\pgfpoint{16.062760\du}{10.622526\du}}
\pgfpathlineto{\pgfpoint{16.056553\du}{10.615590\du}}
\pgfpathlineto{\pgfpoint{16.050347\du}{10.608653\du}}
\pgfpathlineto{\pgfpoint{16.044140\du}{10.600986\du}}
\pgfpathlineto{\pgfpoint{16.039029\du}{10.594049\du}}
\pgfpathlineto{\pgfpoint{16.033917\du}{10.586382\du}}
\pgfpathlineto{\pgfpoint{16.029171\du}{10.579445\du}}
\pgfpathlineto{\pgfpoint{16.024790\du}{10.571778\du}}
\pgfpathlineto{\pgfpoint{16.020409\du}{10.564111\du}}
\pgfpathlineto{\pgfpoint{16.016758\du}{10.556444\du}}
\pgfpathlineto{\pgfpoint{16.013472\du}{10.548412\du}}
\pgfpathlineto{\pgfpoint{16.010186\du}{10.540745\du}}
\pgfpathlineto{\pgfpoint{16.007265\du}{10.533078\du}}
\pgfpathlineto{\pgfpoint{16.004710\du}{10.525411\du}}
\pgfpathlineto{\pgfpoint{16.002884\du}{10.517379\du}}
\pgfpathlineto{\pgfpoint{16.001059\du}{10.508981\du}}
\pgfpathlineto{\pgfpoint{15.999963\du}{10.501314\du}}
\pgfpathlineto{\pgfpoint{15.998503\du}{10.493282\du}}
\pgfpathlineto{\pgfpoint{15.998138\du}{10.484885\du}}
\pgfpathlineto{\pgfpoint{15.998138\du}{10.476853\du}}
\pgfpathlineto{\pgfpoint{15.998138\du}{10.476853\du}}
\pgfpathlineto{\pgfpoint{15.998138\du}{10.476853\du}}
\pgfpathlineto{\pgfpoint{15.998138\du}{10.475758\du}}
\pgfpathlineto{\pgfpoint{15.998138\du}{10.474662\du}}
\pgfpathlineto{\pgfpoint{15.997773\du}{10.473202\du}}
\pgfpathlineto{\pgfpoint{15.997773\du}{10.472107\du}}
\pgfpathlineto{\pgfpoint{15.997408\du}{10.471742\du}}
\pgfpathlineto{\pgfpoint{15.996313\du}{10.470281\du}}
\pgfpathlineto{\pgfpoint{15.995582\du}{10.469916\du}}
\pgfpathlineto{\pgfpoint{15.995582\du}{10.469186\du}}
\pgfpathlineto{\pgfpoint{15.993392\du}{10.468091\du}}
\pgfpathlineto{\pgfpoint{15.991566\du}{10.467360\du}}
\pgfpathlineto{\pgfpoint{15.989741\du}{10.466995\du}}
\pgfpathlineto{\pgfpoint{15.987550\du}{10.466265\du}}
\pgfpathlineto{\pgfpoint{15.986090\du}{10.466995\du}}
\pgfpathlineto{\pgfpoint{15.984264\du}{10.467360\du}}
\pgfpathlineto{\pgfpoint{15.982439\du}{10.468091\du}}
\pgfpathlineto{\pgfpoint{15.980613\du}{10.469186\du}}
\pgfpathlineto{\pgfpoint{15.979883\du}{10.469916\du}}
\pgfpathlineto{\pgfpoint{15.979518\du}{10.470281\du}}
\pgfpathlineto{\pgfpoint{15.979153\du}{10.471742\du}}
\pgfpathlineto{\pgfpoint{15.978423\du}{10.472107\du}}
\pgfpathlineto{\pgfpoint{15.978423\du}{10.473202\du}}
\pgfpathlineto{\pgfpoint{15.977693\du}{10.474662\du}}
\pgfpathlineto{\pgfpoint{15.977693\du}{10.475758\du}}
\pgfpathlineto{\pgfpoint{15.977693\du}{10.476853\du}}
\pgfusepath{fill}
\pgfsetbuttcap
\pgfsetmiterjoin
\pgfsetdash{}{0pt}
\definecolor{dialinecolor}{rgb}{0.678431, 0.839216, 0.905882}
\pgfsetfillcolor{dialinecolor}
\pgfpathmoveto{\pgfpoint{16.602373\du}{10.131106\du}}
\pgfpathlineto{\pgfpoint{16.602373\du}{10.131106\du}}
\pgfpathlineto{\pgfpoint{16.586674\du}{10.131106\du}}
\pgfpathlineto{\pgfpoint{16.570610\du}{10.131471\du}}
\pgfpathlineto{\pgfpoint{16.554545\du}{10.132202\du}}
\pgfpathlineto{\pgfpoint{16.539211\du}{10.132932\du}}
\pgfpathlineto{\pgfpoint{16.523147\du}{10.134027\du}}
\pgfpathlineto{\pgfpoint{16.508543\du}{10.135122\du}}
\pgfpathlineto{\pgfpoint{16.492844\du}{10.136218\du}}
\pgfpathlineto{\pgfpoint{16.477145\du}{10.138043\du}}
\pgfpathlineto{\pgfpoint{16.462541\du}{10.139869\du}}
\pgfpathlineto{\pgfpoint{16.447937\du}{10.141694\du}}
\pgfpathlineto{\pgfpoint{16.432968\du}{10.143885\du}}
\pgfpathlineto{\pgfpoint{16.417999\du}{10.146440\du}}
\pgfpathlineto{\pgfpoint{16.403760\du}{10.149361\du}}
\pgfpathlineto{\pgfpoint{16.389887\du}{10.151552\du}}
\pgfpathlineto{\pgfpoint{16.375283\du}{10.154472\du}}
\pgfpathlineto{\pgfpoint{16.360679\du}{10.157393\du}}
\pgfpathlineto{\pgfpoint{16.346805\du}{10.161044\du}}
\pgfpathlineto{\pgfpoint{16.333297\du}{10.164330\du}}
\pgfpathlineto{\pgfpoint{16.320518\du}{10.168346\du}}
\pgfpathlineto{\pgfpoint{16.307010\du}{10.171997\du}}
\pgfpathlineto{\pgfpoint{16.293501\du}{10.176013\du}}
\pgfpathlineto{\pgfpoint{16.280723\du}{10.180394\du}}
\pgfpathlineto{\pgfpoint{16.267579\du}{10.184775\du}}
\pgfpathlineto{\pgfpoint{16.255531\du}{10.189157\du}}
\pgfpathlineto{\pgfpoint{16.243483\du}{10.193538\du}}
\pgfpathlineto{\pgfpoint{16.231070\du}{10.199014\du}}
\pgfpathlineto{\pgfpoint{16.219752\du}{10.203760\du}}
\pgfpathlineto{\pgfpoint{16.207704\du}{10.208872\du}}
\pgfpathlineto{\pgfpoint{16.196386\du}{10.213983\du}}
\pgfpathlineto{\pgfpoint{16.185068\du}{10.219460\du}}
\pgfpathlineto{\pgfpoint{16.173750\du}{10.225301\du}}
\pgfpathlineto{\pgfpoint{16.162797\du}{10.230413\du}}
\pgfpathlineto{\pgfpoint{16.152574\du}{10.236254\du}}
\pgfpathlineto{\pgfpoint{16.142351\du}{10.242826\du}}
\pgfpathlineto{\pgfpoint{16.132494\du}{10.248667\du}}
\pgfpathlineto{\pgfpoint{16.123001\du}{10.255239\du}}
\pgfpathlineto{\pgfpoint{16.113874\du}{10.261446\du}}
\pgfpathlineto{\pgfpoint{16.104016\du}{10.268018\du}}
\pgfpathlineto{\pgfpoint{16.094889\du}{10.274589\du}}
\pgfpathlineto{\pgfpoint{16.086857\du}{10.281526\du}}
\pgfpathlineto{\pgfpoint{16.077729\du}{10.288463\du}}
\pgfpathlineto{\pgfpoint{16.070792\du}{10.295400\du}}
\pgfpathlineto{\pgfpoint{16.062030\du}{10.302337\du}}
\pgfpathlineto{\pgfpoint{16.054728\du}{10.309639\du}}
\pgfpathlineto{\pgfpoint{16.047791\du}{10.317306\du}}
\pgfpathlineto{\pgfpoint{16.041219\du}{10.324608\du}}
\pgfpathlineto{\pgfpoint{16.034283\du}{10.332275\du}}
\pgfpathlineto{\pgfpoint{16.027711\du}{10.339942\du}}
\pgfpathlineto{\pgfpoint{16.022965\du}{10.347974\du}}
\pgfpathlineto{\pgfpoint{16.016758\du}{10.355641\du}}
\pgfpathlineto{\pgfpoint{16.011647\du}{10.364038\du}}
\pgfpathlineto{\pgfpoint{16.006900\du}{10.372070\du}}
\pgfpathlineto{\pgfpoint{16.002519\du}{10.380467\du}}
\pgfpathlineto{\pgfpoint{15.998138\du}{10.388499\du}}
\pgfpathlineto{\pgfpoint{15.994122\du}{10.397262\du}}
\pgfpathlineto{\pgfpoint{15.990836\du}{10.405659\du}}
\pgfpathlineto{\pgfpoint{15.987550\du}{10.414421\du}}
\pgfpathlineto{\pgfpoint{15.985360\du}{10.423184\du}}
\pgfpathlineto{\pgfpoint{15.982804\du}{10.431946\du}}
\pgfpathlineto{\pgfpoint{15.981709\du}{10.440708\du}}
\pgfpathlineto{\pgfpoint{15.979518\du}{10.449471\du}}
\pgfpathlineto{\pgfpoint{15.978423\du}{10.458598\du}}
\pgfpathlineto{\pgfpoint{15.977693\du}{10.468091\du}}
\pgfpathlineto{\pgfpoint{15.977693\du}{10.476853\du}}
\pgfpathlineto{\pgfpoint{15.998138\du}{10.476853\du}}
\pgfpathlineto{\pgfpoint{15.998138\du}{10.468821\du}}
\pgfpathlineto{\pgfpoint{15.998503\du}{10.460424\du}}
\pgfpathlineto{\pgfpoint{15.999963\du}{10.452391\du}}
\pgfpathlineto{\pgfpoint{16.001059\du}{10.444724\du}}
\pgfpathlineto{\pgfpoint{16.002884\du}{10.436327\du}}
\pgfpathlineto{\pgfpoint{16.004710\du}{10.428295\du}}
\pgfpathlineto{\pgfpoint{16.007265\du}{10.420628\du}}
\pgfpathlineto{\pgfpoint{16.010186\du}{10.412961\du}}
\pgfpathlineto{\pgfpoint{16.013472\du}{10.405659\du}}
\pgfpathlineto{\pgfpoint{16.016758\du}{10.397262\du}}
\pgfpathlineto{\pgfpoint{16.020409\du}{10.389595\du}}
\pgfpathlineto{\pgfpoint{16.024790\du}{10.381928\du}}
\pgfpathlineto{\pgfpoint{16.029171\du}{10.374991\du}}
\pgfpathlineto{\pgfpoint{16.033917\du}{10.367324\du}}
\pgfpathlineto{\pgfpoint{16.039029\du}{10.360022\du}}
\pgfpathlineto{\pgfpoint{16.044140\du}{10.352720\du}}
\pgfpathlineto{\pgfpoint{16.050347\du}{10.345053\du}}
\pgfpathlineto{\pgfpoint{16.056553\du}{10.338116\du}}
\pgfpathlineto{\pgfpoint{16.062760\du}{10.331179\du}}
\pgfpathlineto{\pgfpoint{16.069697\du}{10.324242\du}}
\pgfpathlineto{\pgfpoint{16.076269\du}{10.317671\du}}
\pgfpathlineto{\pgfpoint{16.083936\du}{10.310734\du}}
\pgfpathlineto{\pgfpoint{16.091238\du}{10.303797\du}}
\pgfpathlineto{\pgfpoint{16.099270\du}{10.297225\du}}
\pgfpathlineto{\pgfpoint{16.107667\du}{10.290654\du}}
\pgfpathlineto{\pgfpoint{16.116064\du}{10.284447\du}}
\pgfpathlineto{\pgfpoint{16.125192\du}{10.278605\du}}
\pgfpathlineto{\pgfpoint{16.133589\du}{10.272034\du}}
\pgfpathlineto{\pgfpoint{16.143447\du}{10.266192\du}}
\pgfpathlineto{\pgfpoint{16.152939\du}{10.260350\du}}
\pgfpathlineto{\pgfpoint{16.162797\du}{10.254509\du}}
\pgfpathlineto{\pgfpoint{16.173019\du}{10.248667\du}}
\pgfpathlineto{\pgfpoint{16.183242\du}{10.243556\du}}
\pgfpathlineto{\pgfpoint{16.194195\du}{10.237714\du}}
\pgfpathlineto{\pgfpoint{16.204783\du}{10.232968\du}}
\pgfpathlineto{\pgfpoint{16.216101\du}{10.227492\du}}
\pgfpathlineto{\pgfpoint{16.227054\du}{10.222380\du}}
\pgfpathlineto{\pgfpoint{16.238737\du}{10.217634\du}}
\pgfpathlineto{\pgfpoint{16.250055\du}{10.212888\du}}
\pgfpathlineto{\pgfpoint{16.262468\du}{10.208142\du}}
\pgfpathlineto{\pgfpoint{16.274881\du}{10.204126\du}}
\pgfpathlineto{\pgfpoint{16.286930\du}{10.199379\du}}
\pgfpathlineto{\pgfpoint{16.299708\du}{10.195363\du}}
\pgfpathlineto{\pgfpoint{16.312851\du}{10.192077\du}}
\pgfpathlineto{\pgfpoint{16.325630\du}{10.187696\du}}
\pgfpathlineto{\pgfpoint{16.339138\du}{10.184410\du}}
\pgfpathlineto{\pgfpoint{16.352282\du}{10.180759\du}}
\pgfpathlineto{\pgfpoint{16.365790\du}{10.177839\du}}
\pgfpathlineto{\pgfpoint{16.379664\du}{10.174918\du}}
\pgfpathlineto{\pgfpoint{16.393903\du}{10.171997\du}}
\pgfpathlineto{\pgfpoint{16.407777\du}{10.169076\du}}
\pgfpathlineto{\pgfpoint{16.421650\du}{10.166886\du}}
\pgfpathlineto{\pgfpoint{16.435889\du}{10.164330\du}}
\pgfpathlineto{\pgfpoint{16.450493\du}{10.162139\du}}
\pgfpathlineto{\pgfpoint{16.465097\du}{10.160314\du}}
\pgfpathlineto{\pgfpoint{16.480066\du}{10.158488\du}}
\pgfpathlineto{\pgfpoint{16.495035\du}{10.156663\du}}
\pgfpathlineto{\pgfpoint{16.510004\du}{10.155568\du}}
\pgfpathlineto{\pgfpoint{16.524973\du}{10.154472\du}}
\pgfpathlineto{\pgfpoint{16.540672\du}{10.153377\du}}
\pgfpathlineto{\pgfpoint{16.555276\du}{10.152647\du}}
\pgfpathlineto{\pgfpoint{16.571340\du}{10.152282\du}}
\pgfpathlineto{\pgfpoint{16.586674\du}{10.151552\du}}
\pgfpathlineto{\pgfpoint{16.602373\du}{10.151552\du}}
\pgfpathlineto{\pgfpoint{16.602373\du}{10.151552\du}}
\pgfpathlineto{\pgfpoint{16.602373\du}{10.151552\du}}
\pgfpathlineto{\pgfpoint{16.603834\du}{10.151552\du}}
\pgfpathlineto{\pgfpoint{16.604929\du}{10.151552\du}}
\pgfpathlineto{\pgfpoint{16.606024\du}{10.151552\du}}
\pgfpathlineto{\pgfpoint{16.607119\du}{10.150821\du}}
\pgfpathlineto{\pgfpoint{16.608215\du}{10.150456\du}}
\pgfpathlineto{\pgfpoint{16.609310\du}{10.149726\du}}
\pgfpathlineto{\pgfpoint{16.609310\du}{10.149361\du}}
\pgfpathlineto{\pgfpoint{16.610040\du}{10.148631\du}}
\pgfpathlineto{\pgfpoint{16.611135\du}{10.146805\du}}
\pgfpathlineto{\pgfpoint{16.612231\du}{10.144980\du}}
\pgfpathlineto{\pgfpoint{16.612596\du}{10.143520\du}}
\pgfpathlineto{\pgfpoint{16.612596\du}{10.141694\du}}
\pgfpathlineto{\pgfpoint{16.612596\du}{10.139138\du}}
\pgfpathlineto{\pgfpoint{16.612231\du}{10.137678\du}}
\pgfpathlineto{\pgfpoint{16.611135\du}{10.135853\du}}
\pgfpathlineto{\pgfpoint{16.610040\du}{10.134757\du}}
\pgfpathlineto{\pgfpoint{16.609310\du}{10.133297\du}}
\pgfpathlineto{\pgfpoint{16.609310\du}{10.132932\du}}
\pgfpathlineto{\pgfpoint{16.608215\du}{10.132202\du}}
\pgfpathlineto{\pgfpoint{16.607119\du}{10.132202\du}}
\pgfpathlineto{\pgfpoint{16.606024\du}{10.131471\du}}
\pgfpathlineto{\pgfpoint{16.604929\du}{10.131471\du}}
\pgfpathlineto{\pgfpoint{16.603834\du}{10.131106\du}}
\pgfpathlineto{\pgfpoint{16.602373\du}{10.131106\du}}
\pgfusepath{fill}
\pgfsetbuttcap
\pgfsetmiterjoin
\pgfsetdash{}{0pt}
\definecolor{dialinecolor}{rgb}{0.678431, 0.839216, 0.905882}
\pgfsetfillcolor{dialinecolor}
\pgfpathmoveto{\pgfpoint{17.227419\du}{10.476853\du}}
\pgfpathlineto{\pgfpoint{17.227419\du}{10.467360\du}}
\pgfpathlineto{\pgfpoint{17.226689\du}{10.458598\du}}
\pgfpathlineto{\pgfpoint{17.225593\du}{10.449471\du}}
\pgfpathlineto{\pgfpoint{17.224133\du}{10.440708\du}}
\pgfpathlineto{\pgfpoint{17.221942\du}{10.431946\du}}
\pgfpathlineto{\pgfpoint{17.220117\du}{10.423184\du}}
\pgfpathlineto{\pgfpoint{17.217196\du}{10.414421\du}}
\pgfpathlineto{\pgfpoint{17.214640\du}{10.405659\du}}
\pgfpathlineto{\pgfpoint{17.210624\du}{10.397262\du}}
\pgfpathlineto{\pgfpoint{17.207338\du}{10.388499\du}}
\pgfpathlineto{\pgfpoint{17.202957\du}{10.380467\du}}
\pgfpathlineto{\pgfpoint{17.198211\du}{10.372070\du}}
\pgfpathlineto{\pgfpoint{17.193465\du}{10.364038\du}}
\pgfpathlineto{\pgfpoint{17.188719\du}{10.355641\du}}
\pgfpathlineto{\pgfpoint{17.182877\du}{10.347974\du}}
\pgfpathlineto{\pgfpoint{17.176670\du}{10.339942\du}}
\pgfpathlineto{\pgfpoint{17.170464\du}{10.332275\du}}
\pgfpathlineto{\pgfpoint{17.163892\du}{10.324608\du}}
\pgfpathlineto{\pgfpoint{17.157320\du}{10.317306\du}}
\pgfpathlineto{\pgfpoint{17.149653\du}{10.309639\du}}
\pgfpathlineto{\pgfpoint{17.142716\du}{10.302337\du}}
\pgfpathlineto{\pgfpoint{17.134684\du}{10.295400\du}}
\pgfpathlineto{\pgfpoint{17.126652\du}{10.288463\du}}
\pgfpathlineto{\pgfpoint{17.118255\du}{10.281526\du}}
\pgfpathlineto{\pgfpoint{17.109858\du}{10.274589\du}}
\pgfpathlineto{\pgfpoint{17.101095\du}{10.268018\du}}
\pgfpathlineto{\pgfpoint{17.091968\du}{10.261446\du}}
\pgfpathlineto{\pgfpoint{17.082475\du}{10.255239\du}}
\pgfpathlineto{\pgfpoint{17.072253\du}{10.248667\du}}
\pgfpathlineto{\pgfpoint{17.062395\du}{10.242826\du}}
\pgfpathlineto{\pgfpoint{17.052172\du}{10.236254\du}}
\pgfpathlineto{\pgfpoint{17.042315\du}{10.230413\du}}
\pgfpathlineto{\pgfpoint{17.030997\du}{10.225301\du}}
\pgfpathlineto{\pgfpoint{17.020044\du}{10.219460\du}}
\pgfpathlineto{\pgfpoint{17.008726\du}{10.213983\du}}
\pgfpathlineto{\pgfpoint{16.997408\du}{10.208872\du}}
\pgfpathlineto{\pgfpoint{16.985725\du}{10.203760\du}}
\pgfpathlineto{\pgfpoint{16.974407\du}{10.199014\du}}
\pgfpathlineto{\pgfpoint{16.961628\du}{10.193538\du}}
\pgfpathlineto{\pgfpoint{16.949945\du}{10.189157\du}}
\pgfpathlineto{\pgfpoint{16.937167\du}{10.184775\du}}
\pgfpathlineto{\pgfpoint{16.924388\du}{10.180394\du}}
\pgfpathlineto{\pgfpoint{16.911245\du}{10.176013\du}}
\pgfpathlineto{\pgfpoint{16.898101\du}{10.171997\du}}
\pgfpathlineto{\pgfpoint{16.884958\du}{10.168346\du}}
\pgfpathlineto{\pgfpoint{16.871449\du}{10.164330\du}}
\pgfpathlineto{\pgfpoint{16.857576\du}{10.161044\du}}
\pgfpathlineto{\pgfpoint{16.844067\du}{10.157393\du}}
\pgfpathlineto{\pgfpoint{16.830194\du}{10.154472\du}}
\pgfpathlineto{\pgfpoint{16.815590\du}{10.151552\du}}
\pgfpathlineto{\pgfpoint{16.801716\du}{10.149361\du}}
\pgfpathlineto{\pgfpoint{16.787477\du}{10.146440\du}}
\pgfpathlineto{\pgfpoint{16.772143\du}{10.143885\du}}
\pgfpathlineto{\pgfpoint{16.757174\du}{10.141694\du}}
\pgfpathlineto{\pgfpoint{16.742205\du}{10.139869\du}}
\pgfpathlineto{\pgfpoint{16.727966\du}{10.138043\du}}
\pgfpathlineto{\pgfpoint{16.711902\du}{10.136218\du}}
\pgfpathlineto{\pgfpoint{16.696933\du}{10.135122\du}}
\pgfpathlineto{\pgfpoint{16.681599\du}{10.134027\du}}
\pgfpathlineto{\pgfpoint{16.665900\du}{10.132932\du}}
\pgfpathlineto{\pgfpoint{16.650566\du}{10.132202\du}}
\pgfpathlineto{\pgfpoint{16.634502\du}{10.131471\du}}
\pgfpathlineto{\pgfpoint{16.618437\du}{10.131106\du}}
\pgfpathlineto{\pgfpoint{16.602373\du}{10.131106\du}}
\pgfpathlineto{\pgfpoint{16.602373\du}{10.151552\du}}
\pgfpathlineto{\pgfpoint{16.618437\du}{10.151552\du}}
\pgfpathlineto{\pgfpoint{16.633771\du}{10.152282\du}}
\pgfpathlineto{\pgfpoint{16.649471\du}{10.152647\du}}
\pgfpathlineto{\pgfpoint{16.664805\du}{10.153377\du}}
\pgfpathlineto{\pgfpoint{16.679774\du}{10.154472\du}}
\pgfpathlineto{\pgfpoint{16.695473\du}{10.155568\du}}
\pgfpathlineto{\pgfpoint{16.710077\du}{10.156663\du}}
\pgfpathlineto{\pgfpoint{16.725411\du}{10.158488\du}}
\pgfpathlineto{\pgfpoint{16.740380\du}{10.160314\du}}
\pgfpathlineto{\pgfpoint{16.754984\du}{10.162139\du}}
\pgfpathlineto{\pgfpoint{16.769222\du}{10.164330\du}}
\pgfpathlineto{\pgfpoint{16.783461\du}{10.166886\du}}
\pgfpathlineto{\pgfpoint{16.797700\du}{10.169076\du}}
\pgfpathlineto{\pgfpoint{16.811574\du}{10.171997\du}}
\pgfpathlineto{\pgfpoint{16.825447\du}{10.174918\du}}
\pgfpathlineto{\pgfpoint{16.839321\du}{10.177839\du}}
\pgfpathlineto{\pgfpoint{16.852829\du}{10.180759\du}}
\pgfpathlineto{\pgfpoint{16.866338\du}{10.184410\du}}
\pgfpathlineto{\pgfpoint{16.879482\du}{10.187696\du}}
\pgfpathlineto{\pgfpoint{16.892260\du}{10.192077\du}}
\pgfpathlineto{\pgfpoint{16.905769\du}{10.195363\du}}
\pgfpathlineto{\pgfpoint{16.918547\du}{10.199379\du}}
\pgfpathlineto{\pgfpoint{16.930230\du}{10.204126\du}}
\pgfpathlineto{\pgfpoint{16.942643\du}{10.208142\du}}
\pgfpathlineto{\pgfpoint{16.954691\du}{10.212888\du}}
\pgfpathlineto{\pgfpoint{16.966375\du}{10.217634\du}}
\pgfpathlineto{\pgfpoint{16.977693\du}{10.222380\du}}
\pgfpathlineto{\pgfpoint{16.989011\du}{10.227492\du}}
\pgfpathlineto{\pgfpoint{17.000694\du}{10.232968\du}}
\pgfpathlineto{\pgfpoint{17.011281\du}{10.237714\du}}
\pgfpathlineto{\pgfpoint{17.021869\du}{10.243556\du}}
\pgfpathlineto{\pgfpoint{17.032457\du}{10.248667\du}}
\pgfpathlineto{\pgfpoint{17.042315\du}{10.254509\du}}
\pgfpathlineto{\pgfpoint{17.052172\du}{10.260350\du}}
\pgfpathlineto{\pgfpoint{17.061300\du}{10.266192\du}}
\pgfpathlineto{\pgfpoint{17.070427\du}{10.272034\du}}
\pgfpathlineto{\pgfpoint{17.079920\du}{10.278605\du}}
\pgfpathlineto{\pgfpoint{17.088682\du}{10.284447\du}}
\pgfpathlineto{\pgfpoint{17.097809\du}{10.290654\du}}
\pgfpathlineto{\pgfpoint{17.105842\du}{10.297225\du}}
\pgfpathlineto{\pgfpoint{17.114239\du}{10.303797\du}}
\pgfpathlineto{\pgfpoint{17.121176\du}{10.310734\du}}
\pgfpathlineto{\pgfpoint{17.128478\du}{10.317671\du}}
\pgfpathlineto{\pgfpoint{17.135414\du}{10.324242\du}}
\pgfpathlineto{\pgfpoint{17.142716\du}{10.331179\du}}
\pgfpathlineto{\pgfpoint{17.148558\du}{10.338116\du}}
\pgfpathlineto{\pgfpoint{17.155495\du}{10.345053\du}}
\pgfpathlineto{\pgfpoint{17.160606\du}{10.352720\du}}
\pgfpathlineto{\pgfpoint{17.166083\du}{10.360022\du}}
\pgfpathlineto{\pgfpoint{17.170829\du}{10.367324\du}}
\pgfpathlineto{\pgfpoint{17.175940\du}{10.374991\du}}
\pgfpathlineto{\pgfpoint{17.179956\du}{10.381928\du}}
\pgfpathlineto{\pgfpoint{17.184702\du}{10.389595\du}}
\pgfpathlineto{\pgfpoint{17.188719\du}{10.397262\du}}
\pgfpathlineto{\pgfpoint{17.192004\du}{10.405659\du}}
\pgfpathlineto{\pgfpoint{17.194560\du}{10.412961\du}}
\pgfpathlineto{\pgfpoint{17.197846\du}{10.420628\du}}
\pgfpathlineto{\pgfpoint{17.200037\du}{10.428295\du}}
\pgfpathlineto{\pgfpoint{17.202592\du}{10.436327\du}}
\pgfpathlineto{\pgfpoint{17.204053\du}{10.444724\du}}
\pgfpathlineto{\pgfpoint{17.205513\du}{10.452391\du}}
\pgfpathlineto{\pgfpoint{17.206243\du}{10.460424\du}}
\pgfpathlineto{\pgfpoint{17.206608\du}{10.468821\du}}
\pgfpathlineto{\pgfpoint{17.207338\du}{10.476853\du}}
\pgfpathlineto{\pgfpoint{17.227419\du}{10.476853\du}}
\pgfusepath{fill}
\pgfsetbuttcap
\pgfsetmiterjoin
\pgfsetdash{}{0pt}
\definecolor{dialinecolor}{rgb}{0.074510, 0.082353, 0.086275}
\pgfsetfillcolor{dialinecolor}
\pgfpathmoveto{\pgfpoint{16.281453\du}{10.571048\du}}
\pgfpathlineto{\pgfpoint{16.508908\du}{10.342862\du}}
\pgfpathlineto{\pgfpoint{16.449032\du}{10.281891\du}}
\pgfpathlineto{\pgfpoint{16.629390\du}{10.281891\du}}
\pgfpathlineto{\pgfpoint{16.629390\du}{10.470281\du}}
\pgfpathlineto{\pgfpoint{16.569149\du}{10.410040\du}}
\pgfpathlineto{\pgfpoint{16.348996\du}{10.631289\du}}
\pgfpathlineto{\pgfpoint{16.281453\du}{10.571048\du}}
\pgfusepath{fill}
\pgfsetbuttcap
\pgfsetmiterjoin
\pgfsetdash{}{0pt}
\definecolor{dialinecolor}{rgb}{0.074510, 0.082353, 0.086275}
\pgfsetfillcolor{dialinecolor}
\pgfpathmoveto{\pgfpoint{16.549434\du}{10.691530\du}}
\pgfpathlineto{\pgfpoint{16.776524\du}{10.463344\du}}
\pgfpathlineto{\pgfpoint{16.715918\du}{10.403103\du}}
\pgfpathlineto{\pgfpoint{16.897006\du}{10.403103\du}}
\pgfpathlineto{\pgfpoint{16.897006\du}{10.591128\du}}
\pgfpathlineto{\pgfpoint{16.836400\du}{10.530887\du}}
\pgfpathlineto{\pgfpoint{16.615882\du}{10.751771\du}}
\pgfpathlineto{\pgfpoint{16.549434\du}{10.691530\du}}
\pgfusepath{fill}
\pgfsetbuttcap
\pgfsetmiterjoin
\pgfsetdash{}{0pt}
\definecolor{dialinecolor}{rgb}{1.000000, 1.000000, 1.000000}
\pgfsetfillcolor{dialinecolor}
\pgfpathmoveto{\pgfpoint{16.268310\du}{10.557539\du}}
\pgfpathlineto{\pgfpoint{16.495400\du}{10.329354\du}}
\pgfpathlineto{\pgfpoint{16.435889\du}{10.269113\du}}
\pgfpathlineto{\pgfpoint{16.615882\du}{10.269113\du}}
\pgfpathlineto{\pgfpoint{16.615882\du}{10.457138\du}}
\pgfpathlineto{\pgfpoint{16.556006\du}{10.396166\du}}
\pgfpathlineto{\pgfpoint{16.335487\du}{10.617780\du}}
\pgfpathlineto{\pgfpoint{16.268310\du}{10.557539\du}}
\pgfusepath{fill}
\pgfsetbuttcap
\pgfsetmiterjoin
\pgfsetdash{}{0pt}
\definecolor{dialinecolor}{rgb}{1.000000, 1.000000, 1.000000}
\pgfsetfillcolor{dialinecolor}
\pgfpathmoveto{\pgfpoint{16.535926\du}{10.678021\du}}
\pgfpathlineto{\pgfpoint{16.763016\du}{10.449836\du}}
\pgfpathlineto{\pgfpoint{16.702775\du}{10.389595\du}}
\pgfpathlineto{\pgfpoint{16.883133\du}{10.389595\du}}
\pgfpathlineto{\pgfpoint{16.883133\du}{10.577620\du}}
\pgfpathlineto{\pgfpoint{16.823622\du}{10.517379\du}}
\pgfpathlineto{\pgfpoint{16.602373\du}{10.738262\du}}
\pgfpathlineto{\pgfpoint{16.535926\du}{10.678021\du}}
\pgfusepath{fill}
\pgfsetlinewidth{0.000000\du}
\pgfsetdash{}{0pt}
\pgfsetdash{}{0pt}
\pgfsetbuttcap
\pgfsetmiterjoin
\pgfsetlinewidth{0.000000\du}
\pgfsetbuttcap
\pgfsetmiterjoin
\pgfsetdash{}{0pt}
\definecolor{dialinecolor}{rgb}{0.027451, 0.486275, 0.682353}
\pgfsetfillcolor{dialinecolor}
\pgfpathmoveto{\pgfpoint{27.879993\du}{10.562559\du}}
\pgfpathlineto{\pgfpoint{27.878532\du}{10.591767\du}}
\pgfpathlineto{\pgfpoint{27.871230\du}{10.621705\du}}
\pgfpathlineto{\pgfpoint{27.861008\du}{10.650183\du}}
\pgfpathlineto{\pgfpoint{27.846404\du}{10.678295\du}}
\pgfpathlineto{\pgfpoint{27.827054\du}{10.706407\du}}
\pgfpathlineto{\pgfpoint{27.805148\du}{10.733790\du}}
\pgfpathlineto{\pgfpoint{27.778496\du}{10.760807\du}}
\pgfpathlineto{\pgfpoint{27.747828\du}{10.787094\du}}
\pgfpathlineto{\pgfpoint{27.714969\du}{10.812286\du}}
\pgfpathlineto{\pgfpoint{27.677729\du}{10.837477\du}}
\pgfpathlineto{\pgfpoint{27.636838\du}{10.861574\du}}
\pgfpathlineto{\pgfpoint{27.593027\du}{10.884940\du}}
\pgfpathlineto{\pgfpoint{27.546659\du}{10.907576\du}}
\pgfpathlineto{\pgfpoint{27.496276\du}{10.929482\du}}
\pgfpathlineto{\pgfpoint{27.443337\du}{10.950292\du}}
\pgfpathlineto{\pgfpoint{27.387842\du}{10.970372\du}}
\pgfpathlineto{\pgfpoint{27.329792\du}{10.989723\du}}
\pgfpathlineto{\pgfpoint{27.269186\du}{11.007612\du}}
\pgfpathlineto{\pgfpoint{27.205294\du}{11.024772\du}}
\pgfpathlineto{\pgfpoint{27.140307\du}{11.041201\du}}
\pgfpathlineto{\pgfpoint{27.071668\du}{11.056170\du}}
\pgfpathlineto{\pgfpoint{27.000840\du}{11.069679\du}}
\pgfpathlineto{\pgfpoint{26.928916\du}{11.082457\du}}
\pgfpathlineto{\pgfpoint{26.854071\du}{11.094505\du}}
\pgfpathlineto{\pgfpoint{26.778131\du}{11.104363\du}}
\pgfpathlineto{\pgfpoint{26.699635\du}{11.113490\du}}
\pgfpathlineto{\pgfpoint{26.620044\du}{11.121157\du}}
\pgfpathlineto{\pgfpoint{26.538992\du}{11.127729\du}}
\pgfpathlineto{\pgfpoint{26.455750\du}{11.132840\du}}
\pgfpathlineto{\pgfpoint{26.372143\du}{11.136491\du}}
\pgfpathlineto{\pgfpoint{26.286710\du}{11.138682\du}}
\pgfpathlineto{\pgfpoint{26.200183\du}{11.139412\du}}
\pgfpathlineto{\pgfpoint{26.114020\du}{11.138682\du}}
\pgfpathlineto{\pgfpoint{26.028222\du}{11.136491\du}}
\pgfpathlineto{\pgfpoint{25.944615\du}{11.132840\du}}
\pgfpathlineto{\pgfpoint{25.861738\du}{11.127729\du}}
\pgfpathlineto{\pgfpoint{25.780321\du}{11.121157\du}}
\pgfpathlineto{\pgfpoint{25.700730\du}{11.113490\du}}
\pgfpathlineto{\pgfpoint{25.622965\du}{11.104363\du}}
\pgfpathlineto{\pgfpoint{25.546294\du}{11.094505\du}}
\pgfpathlineto{\pgfpoint{25.472180\du}{11.082457\du}}
\pgfpathlineto{\pgfpoint{25.399525\du}{11.069679\du}}
\pgfpathlineto{\pgfpoint{25.329062\du}{11.056170\du}}
\pgfpathlineto{\pgfpoint{25.260424\du}{11.041201\du}}
\pgfpathlineto{\pgfpoint{25.194706\du}{11.024772\du}}
\pgfpathlineto{\pgfpoint{25.131179\du}{11.007612\du}}
\pgfpathlineto{\pgfpoint{25.070208\du}{10.989723\du}}
\pgfpathlineto{\pgfpoint{25.011793\du}{10.970372\du}}
\pgfpathlineto{\pgfpoint{24.956663\du}{10.950292\du}}
\pgfpathlineto{\pgfpoint{24.903724\du}{10.929482\du}}
\pgfpathlineto{\pgfpoint{24.853706\du}{10.907576\du}}
\pgfpathlineto{\pgfpoint{24.806608\du}{10.884940\du}}
\pgfpathlineto{\pgfpoint{24.763162\du}{10.861574\du}}
\pgfpathlineto{\pgfpoint{24.722271\du}{10.837477\du}}
\pgfpathlineto{\pgfpoint{24.685031\du}{10.812286\du}}
\pgfpathlineto{\pgfpoint{24.651807\du}{10.787094\du}}
\pgfpathlineto{\pgfpoint{24.621504\du}{10.760807\du}}
\pgfpathlineto{\pgfpoint{24.594852\du}{10.733790\du}}
\pgfpathlineto{\pgfpoint{24.572946\du}{10.706407\du}}
\pgfpathlineto{\pgfpoint{24.553596\du}{10.678295\du}}
\pgfpathlineto{\pgfpoint{24.538992\du}{10.650183\du}}
\pgfpathlineto{\pgfpoint{24.528405\du}{10.621705\du}}
\pgfpathlineto{\pgfpoint{24.521468\du}{10.591767\du}}
\pgfpathlineto{\pgfpoint{24.519642\du}{10.562559\du}}
\pgfpathlineto{\pgfpoint{24.521468\du}{10.532621\du}}
\pgfpathlineto{\pgfpoint{24.528405\du}{10.503414\du}}
\pgfpathlineto{\pgfpoint{24.538992\du}{10.474206\du}}
\pgfpathlineto{\pgfpoint{24.553596\du}{10.446093\du}}
\pgfpathlineto{\pgfpoint{24.572946\du}{10.417981\du}}
\pgfpathlineto{\pgfpoint{24.594852\du}{10.390599\du}}
\pgfpathlineto{\pgfpoint{24.621504\du}{10.363947\du}}
\pgfpathlineto{\pgfpoint{24.651807\du}{10.337660\du}}
\pgfpathlineto{\pgfpoint{24.685031\du}{10.312103\du}}
\pgfpathlineto{\pgfpoint{24.722271\du}{10.287276\du}}
\pgfpathlineto{\pgfpoint{24.763162\du}{10.262815\du}}
\pgfpathlineto{\pgfpoint{24.806608\du}{10.239449\du}}
\pgfpathlineto{\pgfpoint{24.853706\du}{10.217178\du}}
\pgfpathlineto{\pgfpoint{24.903724\du}{10.194907\du}}
\pgfpathlineto{\pgfpoint{24.956663\du}{10.174096\du}}
\pgfpathlineto{\pgfpoint{25.011793\du}{10.154016\du}}
\pgfpathlineto{\pgfpoint{25.070208\du}{10.135396\du}}
\pgfpathlineto{\pgfpoint{25.131179\du}{10.116411\du}}
\pgfpathlineto{\pgfpoint{25.194706\du}{10.099617\du}}
\pgfpathlineto{\pgfpoint{25.260424\du}{10.083917\du}}
\pgfpathlineto{\pgfpoint{25.329062\du}{10.068583\du}}
\pgfpathlineto{\pgfpoint{25.399525\du}{10.054710\du}}
\pgfpathlineto{\pgfpoint{25.472180\du}{10.041566\du}}
\pgfpathlineto{\pgfpoint{25.546294\du}{10.030613\du}}
\pgfpathlineto{\pgfpoint{25.622965\du}{10.020026\du}}
\pgfpathlineto{\pgfpoint{25.700730\du}{10.010533\du}}
\pgfpathlineto{\pgfpoint{25.780321\du}{10.003231\du}}
\pgfpathlineto{\pgfpoint{25.861738\du}{9.996659\du}}
\pgfpathlineto{\pgfpoint{25.944615\du}{9.991548\du}}
\pgfpathlineto{\pgfpoint{26.028222\du}{9.987897\du}}
\pgfpathlineto{\pgfpoint{26.114020\du}{9.986072\du}}
\pgfpathlineto{\pgfpoint{26.200183\du}{9.984976\du}}
\pgfpathlineto{\pgfpoint{26.286710\du}{9.986072\du}}
\pgfpathlineto{\pgfpoint{26.372143\du}{9.987897\du}}
\pgfpathlineto{\pgfpoint{26.455750\du}{9.991548\du}}
\pgfpathlineto{\pgfpoint{26.538992\du}{9.996659\du}}
\pgfpathlineto{\pgfpoint{26.620044\du}{10.003231\du}}
\pgfpathlineto{\pgfpoint{26.699635\du}{10.010533\du}}
\pgfpathlineto{\pgfpoint{26.778131\du}{10.020026\du}}
\pgfpathlineto{\pgfpoint{26.854071\du}{10.030613\du}}
\pgfpathlineto{\pgfpoint{26.928916\du}{10.041566\du}}
\pgfpathlineto{\pgfpoint{27.000840\du}{10.054710\du}}
\pgfpathlineto{\pgfpoint{27.071668\du}{10.068583\du}}
\pgfpathlineto{\pgfpoint{27.140307\du}{10.083917\du}}
\pgfpathlineto{\pgfpoint{27.205294\du}{10.099617\du}}
\pgfpathlineto{\pgfpoint{27.269186\du}{10.116411\du}}
\pgfpathlineto{\pgfpoint{27.329792\du}{10.135396\du}}
\pgfpathlineto{\pgfpoint{27.387842\du}{10.154016\du}}
\pgfpathlineto{\pgfpoint{27.443337\du}{10.174096\du}}
\pgfpathlineto{\pgfpoint{27.496276\du}{10.194907\du}}
\pgfpathlineto{\pgfpoint{27.546659\du}{10.217178\du}}
\pgfpathlineto{\pgfpoint{27.593027\du}{10.239449\du}}
\pgfpathlineto{\pgfpoint{27.636838\du}{10.262815\du}}
\pgfpathlineto{\pgfpoint{27.677729\du}{10.287276\du}}
\pgfpathlineto{\pgfpoint{27.714969\du}{10.312103\du}}
\pgfpathlineto{\pgfpoint{27.747828\du}{10.337660\du}}
\pgfpathlineto{\pgfpoint{27.778496\du}{10.363947\du}}
\pgfpathlineto{\pgfpoint{27.805148\du}{10.390599\du}}
\pgfpathlineto{\pgfpoint{27.827054\du}{10.417981\du}}
\pgfpathlineto{\pgfpoint{27.846404\du}{10.446093\du}}
\pgfpathlineto{\pgfpoint{27.861008\du}{10.474206\du}}
\pgfpathlineto{\pgfpoint{27.871230\du}{10.503414\du}}
\pgfpathlineto{\pgfpoint{27.878532\du}{10.532621\du}}
\pgfpathlineto{\pgfpoint{27.879993\du}{10.562559\du}}
\pgfusepath{fill}
\pgfsetlinewidth{0.000000\du}
\pgfsetbuttcap
\pgfsetmiterjoin
\pgfsetdash{}{0pt}
\definecolor{dialinecolor}{rgb}{0.678431, 0.839216, 0.905882}
\pgfsetfillcolor{dialinecolor}
\pgfpathmoveto{\pgfpoint{26.200183\du}{11.150000\du}}
\pgfpathlineto{\pgfpoint{26.200183\du}{11.150000\du}}
\pgfpathlineto{\pgfpoint{26.243629\du}{11.150000\du}}
\pgfpathlineto{\pgfpoint{26.287076\du}{11.149270\du}}
\pgfpathlineto{\pgfpoint{26.330157\du}{11.148175\du}}
\pgfpathlineto{\pgfpoint{26.372143\du}{11.147079\du}}
\pgfpathlineto{\pgfpoint{26.414859\du}{11.145254\du}}
\pgfpathlineto{\pgfpoint{26.456480\du}{11.143063\du}}
\pgfpathlineto{\pgfpoint{26.498101\du}{11.140507\du}}
\pgfpathlineto{\pgfpoint{26.539723\du}{11.138317\du}}
\pgfpathlineto{\pgfpoint{26.580248\du}{11.135396\du}}
\pgfpathlineto{\pgfpoint{26.621139\du}{11.131745\du}}
\pgfpathlineto{\pgfpoint{26.660935\du}{11.127729\du}}
\pgfpathlineto{\pgfpoint{26.701095\du}{11.123713\du}}
\pgfpathlineto{\pgfpoint{26.739796\du}{11.119332\du}}
\pgfpathlineto{\pgfpoint{26.779226\du}{11.114951\du}}
\pgfpathlineto{\pgfpoint{26.817196\du}{11.109474\du}}
\pgfpathlineto{\pgfpoint{26.856261\du}{11.104363\du}}
\pgfpathlineto{\pgfpoint{26.893501\du}{11.098886\du}}
\pgfpathlineto{\pgfpoint{26.930376\du}{11.092680\du}}
\pgfpathlineto{\pgfpoint{26.966886\du}{11.086838\du}}
\pgfpathlineto{\pgfpoint{27.003395\du}{11.080267\du}}
\pgfpathlineto{\pgfpoint{27.038810\du}{11.073330\du}}
\pgfpathlineto{\pgfpoint{27.073494\du}{11.066393\du}}
\pgfpathlineto{\pgfpoint{27.108178\du}{11.058726\du}}
\pgfpathlineto{\pgfpoint{27.142132\du}{11.051059\du}}
\pgfpathlineto{\pgfpoint{27.175721\du}{11.042662\du}}
\pgfpathlineto{\pgfpoint{27.208580\du}{11.034629\du}}
\pgfpathlineto{\pgfpoint{27.240343\du}{11.026597\du}}
\pgfpathlineto{\pgfpoint{27.271742\du}{11.017835\du}}
\pgfpathlineto{\pgfpoint{27.287076\du}{11.013089\du}}
\pgfpathlineto{\pgfpoint{27.302410\du}{11.009073\du}}
\pgfpathlineto{\pgfpoint{27.318474\du}{11.004326\du}}
\pgfpathlineto{\pgfpoint{27.333078\du}{10.999580\du}}
\pgfpathlineto{\pgfpoint{27.347317\du}{10.994834\du}}
\pgfpathlineto{\pgfpoint{27.362286\du}{10.989723\du}}
\pgfpathlineto{\pgfpoint{27.377254\du}{10.984976\du}}
\pgfpathlineto{\pgfpoint{27.391128\du}{10.980230\du}}
\pgfpathlineto{\pgfpoint{27.405367\du}{10.975119\du}}
\pgfpathlineto{\pgfpoint{27.419241\du}{10.970372\du}}
\pgfpathlineto{\pgfpoint{27.433479\du}{10.964896\du}}
\pgfpathlineto{\pgfpoint{27.446623\du}{10.960515\du}}
\pgfpathlineto{\pgfpoint{27.460862\du}{10.955038\du}}
\pgfpathlineto{\pgfpoint{27.474005\du}{10.949927\du}}
\pgfpathlineto{\pgfpoint{27.487149\du}{10.944451\du}}
\pgfpathlineto{\pgfpoint{27.500657\du}{10.938609\du}}
\pgfpathlineto{\pgfpoint{27.513436\du}{10.933498\du}}
\pgfpathlineto{\pgfpoint{27.525484\du}{10.928021\du}}
\pgfpathlineto{\pgfpoint{27.538262\du}{10.922180\du}}
\pgfpathlineto{\pgfpoint{27.550675\du}{10.917068\du}}
\pgfpathlineto{\pgfpoint{27.563089\du}{10.911227\du}}
\pgfpathlineto{\pgfpoint{27.574407\du}{10.905750\du}}
\pgfpathlineto{\pgfpoint{27.586455\du}{10.899909\du}}
\pgfpathlineto{\pgfpoint{27.597408\du}{10.894067\du}}
\pgfpathlineto{\pgfpoint{27.609091\du}{10.888226\du}}
\pgfpathlineto{\pgfpoint{27.620409\du}{10.882384\du}}
\pgfpathlineto{\pgfpoint{27.631727\du}{10.876543\du}}
\pgfpathlineto{\pgfpoint{27.642315\du}{10.870336\du}}
\pgfpathlineto{\pgfpoint{27.652172\du}{10.864494\du}}
\pgfpathlineto{\pgfpoint{27.662760\du}{10.858653\du}}
\pgfpathlineto{\pgfpoint{27.672983\du}{10.852081\du}}
\pgfpathlineto{\pgfpoint{27.682840\du}{10.846240\du}}
\pgfpathlineto{\pgfpoint{27.692698\du}{10.839668\du}}
\pgfpathlineto{\pgfpoint{27.701825\du}{10.833461\du}}
\pgfpathlineto{\pgfpoint{27.710953\du}{10.826889\du}}
\pgfpathlineto{\pgfpoint{27.720445\du}{10.821048\du}}
\pgfpathlineto{\pgfpoint{27.729208\du}{10.814476\du}}
\pgfpathlineto{\pgfpoint{27.738335\du}{10.808269\du}}
\pgfpathlineto{\pgfpoint{27.746367\du}{10.801698\du}}
\pgfpathlineto{\pgfpoint{27.755130\du}{10.794761\du}}
\pgfpathlineto{\pgfpoint{27.762432\du}{10.788189\du}}
\pgfpathlineto{\pgfpoint{27.770464\du}{10.781982\du}}
\pgfpathlineto{\pgfpoint{27.778496\du}{10.774681\du}}
\pgfpathlineto{\pgfpoint{27.785068\du}{10.768474\du}}
\pgfpathlineto{\pgfpoint{27.792369\du}{10.761537\du}}
\pgfpathlineto{\pgfpoint{27.798941\du}{10.754235\du}}
\pgfpathlineto{\pgfpoint{27.805878\du}{10.748028\du}}
\pgfpathlineto{\pgfpoint{27.812450\du}{10.741092\du}}
\pgfpathlineto{\pgfpoint{27.818656\du}{10.733790\du}}
\pgfpathlineto{\pgfpoint{27.824498\du}{10.726853\du}}
\pgfpathlineto{\pgfpoint{27.829974\du}{10.719916\du}}
\pgfpathlineto{\pgfpoint{27.835816\du}{10.712979\du}}
\pgfpathlineto{\pgfpoint{27.840562\du}{10.705677\du}}
\pgfpathlineto{\pgfpoint{27.846039\du}{10.698375\du}}
\pgfpathlineto{\pgfpoint{27.850785\du}{10.691073\du}}
\pgfpathlineto{\pgfpoint{27.855166\du}{10.684137\du}}
\pgfpathlineto{\pgfpoint{27.859182\du}{10.676470\du}}
\pgfpathlineto{\pgfpoint{27.863198\du}{10.669533\du}}
\pgfpathlineto{\pgfpoint{27.866484\du}{10.661866\du}}
\pgfpathlineto{\pgfpoint{27.870500\du}{10.654199\du}}
\pgfpathlineto{\pgfpoint{27.873786\du}{10.646532\du}}
\pgfpathlineto{\pgfpoint{27.876342\du}{10.639230\du}}
\pgfpathlineto{\pgfpoint{27.879263\du}{10.631928\du}}
\pgfpathlineto{\pgfpoint{27.881088\du}{10.624626\du}}
\pgfpathlineto{\pgfpoint{27.884009\du}{10.616959\du}}
\pgfpathlineto{\pgfpoint{27.885104\du}{10.608562\du}}
\pgfpathlineto{\pgfpoint{27.887295\du}{10.600894\du}}
\pgfpathlineto{\pgfpoint{27.888390\du}{10.593593\du}}
\pgfpathlineto{\pgfpoint{27.889120\du}{10.585926\du}}
\pgfpathlineto{\pgfpoint{27.889850\du}{10.577528\du}}
\pgfpathlineto{\pgfpoint{27.890581\du}{10.570226\du}}
\pgfpathlineto{\pgfpoint{27.890581\du}{10.562559\du}}
\pgfpathlineto{\pgfpoint{27.870500\du}{10.562559\du}}
\pgfpathlineto{\pgfpoint{27.869770\du}{10.569496\du}}
\pgfpathlineto{\pgfpoint{27.869770\du}{10.576433\du}}
\pgfpathlineto{\pgfpoint{27.869405\du}{10.583370\du}}
\pgfpathlineto{\pgfpoint{27.867579\du}{10.590672\du}}
\pgfpathlineto{\pgfpoint{27.866484\du}{10.597609\du}}
\pgfpathlineto{\pgfpoint{27.865754\du}{10.604545\du}}
\pgfpathlineto{\pgfpoint{27.863563\du}{10.611482\du}}
\pgfpathlineto{\pgfpoint{27.862103\du}{10.618784\du}}
\pgfpathlineto{\pgfpoint{27.859912\du}{10.624991\du}}
\pgfpathlineto{\pgfpoint{27.857357\du}{10.631928\du}}
\pgfpathlineto{\pgfpoint{27.854436\du}{10.639230\du}}
\pgfpathlineto{\pgfpoint{27.851880\du}{10.646166\du}}
\pgfpathlineto{\pgfpoint{27.847864\du}{10.653103\du}}
\pgfpathlineto{\pgfpoint{27.844943\du}{10.659675\du}}
\pgfpathlineto{\pgfpoint{27.840927\du}{10.666612\du}}
\pgfpathlineto{\pgfpoint{27.838007\du}{10.673184\du}}
\pgfpathlineto{\pgfpoint{27.833625\du}{10.680120\du}}
\pgfpathlineto{\pgfpoint{27.829244\du}{10.687057\du}}
\pgfpathlineto{\pgfpoint{27.824498\du}{10.693629\du}}
\pgfpathlineto{\pgfpoint{27.819387\du}{10.699836\du}}
\pgfpathlineto{\pgfpoint{27.814640\du}{10.707138\du}}
\pgfpathlineto{\pgfpoint{27.808434\du}{10.714074\du}}
\pgfpathlineto{\pgfpoint{27.802957\du}{10.720281\du}}
\pgfpathlineto{\pgfpoint{27.797846\du}{10.726853\du}}
\pgfpathlineto{\pgfpoint{27.791274\du}{10.733425\du}}
\pgfpathlineto{\pgfpoint{27.784337\du}{10.740361\du}}
\pgfpathlineto{\pgfpoint{27.778496\du}{10.746933\du}}
\pgfpathlineto{\pgfpoint{27.771194\du}{10.753140\du}}
\pgfpathlineto{\pgfpoint{27.764622\du}{10.759712\du}}
\pgfpathlineto{\pgfpoint{27.756955\du}{10.765918\du}}
\pgfpathlineto{\pgfpoint{27.749653\du}{10.772490\du}}
\pgfpathlineto{\pgfpoint{27.741621\du}{10.779062\du}}
\pgfpathlineto{\pgfpoint{27.733954\du}{10.785268\du}}
\pgfpathlineto{\pgfpoint{27.725192\du}{10.791840\du}}
\pgfpathlineto{\pgfpoint{27.716794\du}{10.797682\du}}
\pgfpathlineto{\pgfpoint{27.708032\du}{10.804253\du}}
\pgfpathlineto{\pgfpoint{27.700000\du}{10.810460\du}}
\pgfpathlineto{\pgfpoint{27.691238\du}{10.816302\du}}
\pgfpathlineto{\pgfpoint{27.681745\du}{10.822873\du}}
\pgfpathlineto{\pgfpoint{27.671522\du}{10.828715\du}}
\pgfpathlineto{\pgfpoint{27.662395\du}{10.835287\du}}
\pgfpathlineto{\pgfpoint{27.652172\du}{10.841128\du}}
\pgfpathlineto{\pgfpoint{27.642315\du}{10.846970\du}}
\pgfpathlineto{\pgfpoint{27.632457\du}{10.852811\du}}
\pgfpathlineto{\pgfpoint{27.621504\du}{10.858653\du}}
\pgfpathlineto{\pgfpoint{27.610551\du}{10.864494\du}}
\pgfpathlineto{\pgfpoint{27.600329\du}{10.870336\du}}
\pgfpathlineto{\pgfpoint{27.588280\du}{10.876177\du}}
\pgfpathlineto{\pgfpoint{27.577693\du}{10.881289\du}}
\pgfpathlineto{\pgfpoint{27.565644\du}{10.887130\du}}
\pgfpathlineto{\pgfpoint{27.554326\du}{10.892972\du}}
\pgfpathlineto{\pgfpoint{27.541913\du}{10.898448\du}}
\pgfpathlineto{\pgfpoint{27.529865\du}{10.903560\du}}
\pgfpathlineto{\pgfpoint{27.517452\du}{10.909401\du}}
\pgfpathlineto{\pgfpoint{27.505403\du}{10.914878\du}}
\pgfpathlineto{\pgfpoint{27.492260\du}{10.919989\du}}
\pgfpathlineto{\pgfpoint{27.479482\du}{10.925100\du}}
\pgfpathlineto{\pgfpoint{27.466703\du}{10.930577\du}}
\pgfpathlineto{\pgfpoint{27.453560\du}{10.935688\du}}
\pgfpathlineto{\pgfpoint{27.440051\du}{10.941165\du}}
\pgfpathlineto{\pgfpoint{27.426908\du}{10.945546\du}}
\pgfpathlineto{\pgfpoint{27.412669\du}{10.951022\du}}
\pgfpathlineto{\pgfpoint{27.398795\du}{10.955769\du}}
\pgfpathlineto{\pgfpoint{27.384556\du}{10.960880\du}}
\pgfpathlineto{\pgfpoint{27.369953\du}{10.965626\du}}
\pgfpathlineto{\pgfpoint{27.356079\du}{10.970372\du}}
\pgfpathlineto{\pgfpoint{27.341475\du}{10.975119\du}}
\pgfpathlineto{\pgfpoint{27.327236\du}{10.979500\du}}
\pgfpathlineto{\pgfpoint{27.311537\du}{10.984246\du}}
\pgfpathlineto{\pgfpoint{27.297298\du}{10.988992\du}}
\pgfpathlineto{\pgfpoint{27.281599\du}{10.993739\du}}
\pgfpathlineto{\pgfpoint{27.266630\du}{10.997755\du}}
\pgfpathlineto{\pgfpoint{27.235232\du}{11.006517\du}}
\pgfpathlineto{\pgfpoint{27.203103\du}{11.014914\du}}
\pgfpathlineto{\pgfpoint{27.170975\du}{11.022946\du}}
\pgfpathlineto{\pgfpoint{27.137751\du}{11.030978\du}}
\pgfpathlineto{\pgfpoint{27.103432\du}{11.038645\du}}
\pgfpathlineto{\pgfpoint{27.069478\du}{11.045947\du}}
\pgfpathlineto{\pgfpoint{27.034794\du}{11.052884\du}}
\pgfpathlineto{\pgfpoint{26.999014\du}{11.059821\du}}
\pgfpathlineto{\pgfpoint{26.963600\du}{11.066393\du}}
\pgfpathlineto{\pgfpoint{26.926725\du}{11.072599\du}}
\pgfpathlineto{\pgfpoint{26.890215\du}{11.078441\du}}
\pgfpathlineto{\pgfpoint{26.852976\du}{11.084283\du}}
\pgfpathlineto{\pgfpoint{26.815371\du}{11.089759\du}}
\pgfpathlineto{\pgfpoint{26.776670\du}{11.094505\du}}
\pgfpathlineto{\pgfpoint{26.737970\du}{11.098886\du}}
\pgfpathlineto{\pgfpoint{26.698540\du}{11.103633\du}}
\pgfpathlineto{\pgfpoint{26.659474\du}{11.107284\du}}
\pgfpathlineto{\pgfpoint{26.619314\du}{11.111300\du}}
\pgfpathlineto{\pgfpoint{26.579153\du}{11.114221\du}}
\pgfpathlineto{\pgfpoint{26.537897\du}{11.117871\du}}
\pgfpathlineto{\pgfpoint{26.497006\du}{11.120792\du}}
\pgfpathlineto{\pgfpoint{26.455750\du}{11.122983\du}}
\pgfpathlineto{\pgfpoint{26.414129\du}{11.124808\du}}
\pgfpathlineto{\pgfpoint{26.371413\du}{11.126634\du}}
\pgfpathlineto{\pgfpoint{26.328697\du}{11.127729\du}}
\pgfpathlineto{\pgfpoint{26.286710\du}{11.128824\du}}
\pgfpathlineto{\pgfpoint{26.243264\du}{11.128824\du}}
\pgfpathlineto{\pgfpoint{26.200183\du}{11.129555\du}}
\pgfpathlineto{\pgfpoint{26.200183\du}{11.129555\du}}
\pgfpathlineto{\pgfpoint{26.200183\du}{11.129555\du}}
\pgfpathlineto{\pgfpoint{26.199452\du}{11.129555\du}}
\pgfpathlineto{\pgfpoint{26.197627\du}{11.129555\du}}
\pgfpathlineto{\pgfpoint{26.196532\du}{11.129920\du}}
\pgfpathlineto{\pgfpoint{26.195801\du}{11.129920\du}}
\pgfpathlineto{\pgfpoint{26.195436\du}{11.130650\du}}
\pgfpathlineto{\pgfpoint{26.193976\du}{11.131015\du}}
\pgfpathlineto{\pgfpoint{26.193246\du}{11.131745\du}}
\pgfpathlineto{\pgfpoint{26.192516\du}{11.132475\du}}
\pgfpathlineto{\pgfpoint{26.191420\du}{11.134301\du}}
\pgfpathlineto{\pgfpoint{26.190690\du}{11.135761\du}}
\pgfpathlineto{\pgfpoint{26.190690\du}{11.137587\du}}
\pgfpathlineto{\pgfpoint{26.189960\du}{11.139412\du}}
\pgfpathlineto{\pgfpoint{26.190690\du}{11.141603\du}}
\pgfpathlineto{\pgfpoint{26.190690\du}{11.143428\du}}
\pgfpathlineto{\pgfpoint{26.191420\du}{11.145254\du}}
\pgfpathlineto{\pgfpoint{26.192516\du}{11.147079\du}}
\pgfpathlineto{\pgfpoint{26.193246\du}{11.147444\du}}
\pgfpathlineto{\pgfpoint{26.193976\du}{11.148175\du}}
\pgfpathlineto{\pgfpoint{26.195436\du}{11.148905\du}}
\pgfpathlineto{\pgfpoint{26.195801\du}{11.149270\du}}
\pgfpathlineto{\pgfpoint{26.196532\du}{11.149270\du}}
\pgfpathlineto{\pgfpoint{26.197627\du}{11.150000\du}}
\pgfpathlineto{\pgfpoint{26.199452\du}{11.150000\du}}
\pgfpathlineto{\pgfpoint{26.200183\du}{11.150000\du}}
\pgfusepath{fill}
\pgfsetbuttcap
\pgfsetmiterjoin
\pgfsetdash{}{0pt}
\definecolor{dialinecolor}{rgb}{0.678431, 0.839216, 0.905882}
\pgfsetfillcolor{dialinecolor}
\pgfpathmoveto{\pgfpoint{24.509419\du}{10.562559\du}}
\pgfpathlineto{\pgfpoint{24.509419\du}{10.562559\du}}
\pgfpathlineto{\pgfpoint{24.509419\du}{10.570226\du}}
\pgfpathlineto{\pgfpoint{24.509785\du}{10.577528\du}}
\pgfpathlineto{\pgfpoint{24.510515\du}{10.585926\du}}
\pgfpathlineto{\pgfpoint{24.511610\du}{10.593593\du}}
\pgfpathlineto{\pgfpoint{24.512705\du}{10.600894\du}}
\pgfpathlineto{\pgfpoint{24.514531\du}{10.608562\du}}
\pgfpathlineto{\pgfpoint{24.516356\du}{10.616959\du}}
\pgfpathlineto{\pgfpoint{24.518547\du}{10.624626\du}}
\pgfpathlineto{\pgfpoint{24.520737\du}{10.631928\du}}
\pgfpathlineto{\pgfpoint{24.523293\du}{10.639230\du}}
\pgfpathlineto{\pgfpoint{24.526214\du}{10.646532\du}}
\pgfpathlineto{\pgfpoint{24.529865\du}{10.654199\du}}
\pgfpathlineto{\pgfpoint{24.533151\du}{10.661866\du}}
\pgfpathlineto{\pgfpoint{24.536802\du}{10.669533\du}}
\pgfpathlineto{\pgfpoint{24.541183\du}{10.676470\du}}
\pgfpathlineto{\pgfpoint{24.544834\du}{10.684137\du}}
\pgfpathlineto{\pgfpoint{24.549945\du}{10.691073\du}}
\pgfpathlineto{\pgfpoint{24.553961\du}{10.698375\du}}
\pgfpathlineto{\pgfpoint{24.559438\du}{10.705677\du}}
\pgfpathlineto{\pgfpoint{24.564184\du}{10.712979\du}}
\pgfpathlineto{\pgfpoint{24.569660\du}{10.719916\du}}
\pgfpathlineto{\pgfpoint{24.575502\du}{10.726853\du}}
\pgfpathlineto{\pgfpoint{24.581344\du}{10.733790\du}}
\pgfpathlineto{\pgfpoint{24.587185\du}{10.741092\du}}
\pgfpathlineto{\pgfpoint{24.594122\du}{10.748028\du}}
\pgfpathlineto{\pgfpoint{24.600694\du}{10.754235\du}}
\pgfpathlineto{\pgfpoint{24.607631\du}{10.761537\du}}
\pgfpathlineto{\pgfpoint{24.614567\du}{10.768474\du}}
\pgfpathlineto{\pgfpoint{24.621504\du}{10.774681\du}}
\pgfpathlineto{\pgfpoint{24.630267\du}{10.781982\du}}
\pgfpathlineto{\pgfpoint{24.637203\du}{10.788189\du}}
\pgfpathlineto{\pgfpoint{24.644870\du}{10.794761\du}}
\pgfpathlineto{\pgfpoint{24.653633\du}{10.801698\du}}
\pgfpathlineto{\pgfpoint{24.662030\du}{10.808269\du}}
\pgfpathlineto{\pgfpoint{24.670792\du}{10.814476\du}}
\pgfpathlineto{\pgfpoint{24.679189\du}{10.821048\du}}
\pgfpathlineto{\pgfpoint{24.689047\du}{10.826889\du}}
\pgfpathlineto{\pgfpoint{24.698175\du}{10.833461\du}}
\pgfpathlineto{\pgfpoint{24.707667\du}{10.839668\du}}
\pgfpathlineto{\pgfpoint{24.717160\du}{10.846240\du}}
\pgfpathlineto{\pgfpoint{24.727017\du}{10.852081\du}}
\pgfpathlineto{\pgfpoint{24.737240\du}{10.858653\du}}
\pgfpathlineto{\pgfpoint{24.747463\du}{10.864494\du}}
\pgfpathlineto{\pgfpoint{24.758050\du}{10.870336\du}}
\pgfpathlineto{\pgfpoint{24.768273\du}{10.876543\du}}
\pgfpathlineto{\pgfpoint{24.779226\du}{10.882384\du}}
\pgfpathlineto{\pgfpoint{24.790909\du}{10.888226\du}}
\pgfpathlineto{\pgfpoint{24.802227\du}{10.894067\du}}
\pgfpathlineto{\pgfpoint{24.813910\du}{10.899909\du}}
\pgfpathlineto{\pgfpoint{24.825228\du}{10.905750\du}}
\pgfpathlineto{\pgfpoint{24.836911\du}{10.911227\du}}
\pgfpathlineto{\pgfpoint{24.849325\du}{10.917068\du}}
\pgfpathlineto{\pgfpoint{24.861373\du}{10.922180\du}}
\pgfpathlineto{\pgfpoint{24.874516\du}{10.928021\du}}
\pgfpathlineto{\pgfpoint{24.886564\du}{10.933498\du}}
\pgfpathlineto{\pgfpoint{24.899343\du}{10.938609\du}}
\pgfpathlineto{\pgfpoint{24.913217\du}{10.944451\du}}
\pgfpathlineto{\pgfpoint{24.925630\du}{10.949927\du}}
\pgfpathlineto{\pgfpoint{24.938773\du}{10.955038\du}}
\pgfpathlineto{\pgfpoint{24.953012\du}{10.960515\du}}
\pgfpathlineto{\pgfpoint{24.966156\du}{10.964896\du}}
\pgfpathlineto{\pgfpoint{24.980394\du}{10.970372\du}}
\pgfpathlineto{\pgfpoint{24.994268\du}{10.975119\du}}
\pgfpathlineto{\pgfpoint{25.008872\du}{10.980230\du}}
\pgfpathlineto{\pgfpoint{25.022746\du}{10.984976\du}}
\pgfpathlineto{\pgfpoint{25.038445\du}{10.989723\du}}
\pgfpathlineto{\pgfpoint{25.052318\du}{10.994834\du}}
\pgfpathlineto{\pgfpoint{25.066922\du}{10.999580\du}}
\pgfpathlineto{\pgfpoint{25.082256\du}{11.004326\du}}
\pgfpathlineto{\pgfpoint{25.097955\du}{11.009073\du}}
\pgfpathlineto{\pgfpoint{25.112924\du}{11.013089\du}}
\pgfpathlineto{\pgfpoint{25.128624\du}{11.017835\du}}
\pgfpathlineto{\pgfpoint{25.160387\du}{11.026597\du}}
\pgfpathlineto{\pgfpoint{25.192150\du}{11.034629\du}}
\pgfpathlineto{\pgfpoint{25.225374\du}{11.042662\du}}
\pgfpathlineto{\pgfpoint{25.257868\du}{11.051059\du}}
\pgfpathlineto{\pgfpoint{25.292552\du}{11.058726\du}}
\pgfpathlineto{\pgfpoint{25.326871\du}{11.066393\du}}
\pgfpathlineto{\pgfpoint{25.361555\du}{11.073330\du}}
\pgfpathlineto{\pgfpoint{25.397335\du}{11.080267\du}}
\pgfpathlineto{\pgfpoint{25.433479\du}{11.086838\du}}
\pgfpathlineto{\pgfpoint{25.469989\du}{11.092680\du}}
\pgfpathlineto{\pgfpoint{25.507229\du}{11.098886\du}}
\pgfpathlineto{\pgfpoint{25.544834\du}{11.104363\du}}
\pgfpathlineto{\pgfpoint{25.582804\du}{11.109474\du}}
\pgfpathlineto{\pgfpoint{25.621504\du}{11.114951\du}}
\pgfpathlineto{\pgfpoint{25.660204\du}{11.119332\du}}
\pgfpathlineto{\pgfpoint{25.699270\du}{11.123713\du}}
\pgfpathlineto{\pgfpoint{25.739430\du}{11.127729\du}}
\pgfpathlineto{\pgfpoint{25.779226\du}{11.131745\du}}
\pgfpathlineto{\pgfpoint{25.820117\du}{11.135396\du}}
\pgfpathlineto{\pgfpoint{25.861008\du}{11.138317\du}}
\pgfpathlineto{\pgfpoint{25.902264\du}{11.140507\du}}
\pgfpathlineto{\pgfpoint{25.944250\du}{11.143063\du}}
\pgfpathlineto{\pgfpoint{25.985871\du}{11.145254\du}}
\pgfpathlineto{\pgfpoint{26.028222\du}{11.147079\du}}
\pgfpathlineto{\pgfpoint{26.070208\du}{11.148175\du}}
\pgfpathlineto{\pgfpoint{26.113655\du}{11.149270\du}}
\pgfpathlineto{\pgfpoint{26.156371\du}{11.150000\du}}
\pgfpathlineto{\pgfpoint{26.200183\du}{11.150000\du}}
\pgfpathlineto{\pgfpoint{26.200183\du}{11.129555\du}}
\pgfpathlineto{\pgfpoint{26.157466\du}{11.128824\du}}
\pgfpathlineto{\pgfpoint{26.114020\du}{11.128824\du}}
\pgfpathlineto{\pgfpoint{26.071668\du}{11.127729\du}}
\pgfpathlineto{\pgfpoint{26.028952\du}{11.126634\du}}
\pgfpathlineto{\pgfpoint{25.986601\du}{11.124808\du}}
\pgfpathlineto{\pgfpoint{25.944615\du}{11.122983\du}}
\pgfpathlineto{\pgfpoint{25.903724\du}{11.120792\du}}
\pgfpathlineto{\pgfpoint{25.862468\du}{11.117871\du}}
\pgfpathlineto{\pgfpoint{25.821577\du}{11.114221\du}}
\pgfpathlineto{\pgfpoint{25.781417\du}{11.111300\du}}
\pgfpathlineto{\pgfpoint{25.741621\du}{11.107284\du}}
\pgfpathlineto{\pgfpoint{25.702191\du}{11.103633\du}}
\pgfpathlineto{\pgfpoint{25.662760\du}{11.098886\du}}
\pgfpathlineto{\pgfpoint{25.623695\du}{11.094505\du}}
\pgfpathlineto{\pgfpoint{25.585360\du}{11.089759\du}}
\pgfpathlineto{\pgfpoint{25.547755\du}{11.084283\du}}
\pgfpathlineto{\pgfpoint{25.510515\du}{11.078441\du}}
\pgfpathlineto{\pgfpoint{25.474005\du}{11.072599\du}}
\pgfpathlineto{\pgfpoint{25.436765\du}{11.066393\du}}
\pgfpathlineto{\pgfpoint{25.401716\du}{11.059821\du}}
\pgfpathlineto{\pgfpoint{25.365571\du}{11.052884\du}}
\pgfpathlineto{\pgfpoint{25.330887\du}{11.045947\du}}
\pgfpathlineto{\pgfpoint{25.296568\du}{11.038645\du}}
\pgfpathlineto{\pgfpoint{25.262614\du}{11.030978\du}}
\pgfpathlineto{\pgfpoint{25.229390\du}{11.022946\du}}
\pgfpathlineto{\pgfpoint{25.197992\du}{11.014914\du}}
\pgfpathlineto{\pgfpoint{25.165498\du}{11.006517\du}}
\pgfpathlineto{\pgfpoint{25.134465\du}{10.997755\du}}
\pgfpathlineto{\pgfpoint{25.118766\du}{10.993739\du}}
\pgfpathlineto{\pgfpoint{25.103067\du}{10.988992\du}}
\pgfpathlineto{\pgfpoint{25.088463\du}{10.984246\du}}
\pgfpathlineto{\pgfpoint{25.073494\du}{10.979500\du}}
\pgfpathlineto{\pgfpoint{25.058160\du}{10.975119\du}}
\pgfpathlineto{\pgfpoint{25.043921\du}{10.970372\du}}
\pgfpathlineto{\pgfpoint{25.029682\du}{10.965626\du}}
\pgfpathlineto{\pgfpoint{25.015444\du}{10.960880\du}}
\pgfpathlineto{\pgfpoint{25.001205\du}{10.955769\du}}
\pgfpathlineto{\pgfpoint{24.987331\du}{10.951022\du}}
\pgfpathlineto{\pgfpoint{24.973457\du}{10.945546\du}}
\pgfpathlineto{\pgfpoint{24.959949\du}{10.941165\du}}
\pgfpathlineto{\pgfpoint{24.946440\du}{10.935688\du}}
\pgfpathlineto{\pgfpoint{24.933662\du}{10.930577\du}}
\pgfpathlineto{\pgfpoint{24.920153\du}{10.925100\du}}
\pgfpathlineto{\pgfpoint{24.907375\du}{10.919989\du}}
\pgfpathlineto{\pgfpoint{24.894962\du}{10.914878\du}}
\pgfpathlineto{\pgfpoint{24.881818\du}{10.909401\du}}
\pgfpathlineto{\pgfpoint{24.869770\du}{10.903560\du}}
\pgfpathlineto{\pgfpoint{24.858452\du}{10.898448\du}}
\pgfpathlineto{\pgfpoint{24.845674\du}{10.892972\du}}
\pgfpathlineto{\pgfpoint{24.833991\du}{10.887130\du}}
\pgfpathlineto{\pgfpoint{24.822307\du}{10.881289\du}}
\pgfpathlineto{\pgfpoint{24.811355\du}{10.876177\du}}
\pgfpathlineto{\pgfpoint{24.799671\du}{10.870336\du}}
\pgfpathlineto{\pgfpoint{24.789814\du}{10.864494\du}}
\pgfpathlineto{\pgfpoint{24.778496\du}{10.858653\du}}
\pgfpathlineto{\pgfpoint{24.767908\du}{10.852811\du}}
\pgfpathlineto{\pgfpoint{24.758050\du}{10.846970\du}}
\pgfpathlineto{\pgfpoint{24.747463\du}{10.841128\du}}
\pgfpathlineto{\pgfpoint{24.737605\du}{10.835287\du}}
\pgfpathlineto{\pgfpoint{24.728112\du}{10.828715\du}}
\pgfpathlineto{\pgfpoint{24.718255\du}{10.822873\du}}
\pgfpathlineto{\pgfpoint{24.709127\du}{10.816302\du}}
\pgfpathlineto{\pgfpoint{24.700730\du}{10.810460\du}}
\pgfpathlineto{\pgfpoint{24.691968\du}{10.804253\du}}
\pgfpathlineto{\pgfpoint{24.682840\du}{10.797682\du}}
\pgfpathlineto{\pgfpoint{24.674078\du}{10.791840\du}}
\pgfpathlineto{\pgfpoint{24.665681\du}{10.785268\du}}
\pgfpathlineto{\pgfpoint{24.658379\du}{10.779062\du}}
\pgfpathlineto{\pgfpoint{24.650347\du}{10.772490\du}}
\pgfpathlineto{\pgfpoint{24.642680\du}{10.765918\du}}
\pgfpathlineto{\pgfpoint{24.635743\du}{10.759712\du}}
\pgfpathlineto{\pgfpoint{24.628441\du}{10.753140\du}}
\pgfpathlineto{\pgfpoint{24.621504\du}{10.746933\du}}
\pgfpathlineto{\pgfpoint{24.615298\du}{10.740361\du}}
\pgfpathlineto{\pgfpoint{24.608726\du}{10.733425\du}}
\pgfpathlineto{\pgfpoint{24.602884\du}{10.726853\du}}
\pgfpathlineto{\pgfpoint{24.596678\du}{10.720281\du}}
\pgfpathlineto{\pgfpoint{24.591566\du}{10.714074\du}}
\pgfpathlineto{\pgfpoint{24.585360\du}{10.707138\du}}
\pgfpathlineto{\pgfpoint{24.580978\du}{10.700566\du}}
\pgfpathlineto{\pgfpoint{24.575502\du}{10.693629\du}}
\pgfpathlineto{\pgfpoint{24.571121\du}{10.687057\du}}
\pgfpathlineto{\pgfpoint{24.566740\du}{10.680120\du}}
\pgfpathlineto{\pgfpoint{24.562359\du}{10.673184\du}}
\pgfpathlineto{\pgfpoint{24.559073\du}{10.666612\du}}
\pgfpathlineto{\pgfpoint{24.555057\du}{10.659675\du}}
\pgfpathlineto{\pgfpoint{24.551041\du}{10.653103\du}}
\pgfpathlineto{\pgfpoint{24.548120\du}{10.646166\du}}
\pgfpathlineto{\pgfpoint{24.545564\du}{10.639230\du}}
\pgfpathlineto{\pgfpoint{24.542278\du}{10.631928\du}}
\pgfpathlineto{\pgfpoint{24.540088\du}{10.624991\du}}
\pgfpathlineto{\pgfpoint{24.537897\du}{10.618784\du}}
\pgfpathlineto{\pgfpoint{24.536437\du}{10.611482\du}}
\pgfpathlineto{\pgfpoint{24.534246\du}{10.604545\du}}
\pgfpathlineto{\pgfpoint{24.533151\du}{10.597609\du}}
\pgfpathlineto{\pgfpoint{24.532055\du}{10.590672\du}}
\pgfpathlineto{\pgfpoint{24.530595\du}{10.583370\du}}
\pgfpathlineto{\pgfpoint{24.530230\du}{10.576433\du}}
\pgfpathlineto{\pgfpoint{24.530230\du}{10.569496\du}}
\pgfpathlineto{\pgfpoint{24.529865\du}{10.562559\du}}
\pgfpathlineto{\pgfpoint{24.529865\du}{10.562559\du}}
\pgfpathlineto{\pgfpoint{24.529865\du}{10.562559\du}}
\pgfpathlineto{\pgfpoint{24.529865\du}{10.560734\du}}
\pgfpathlineto{\pgfpoint{24.529865\du}{10.560004\du}}
\pgfpathlineto{\pgfpoint{24.529500\du}{10.558908\du}}
\pgfpathlineto{\pgfpoint{24.529500\du}{10.557813\du}}
\pgfpathlineto{\pgfpoint{24.528405\du}{10.557083\du}}
\pgfpathlineto{\pgfpoint{24.528039\du}{10.555988\du}}
\pgfpathlineto{\pgfpoint{24.527674\du}{10.555257\du}}
\pgfpathlineto{\pgfpoint{24.526579\du}{10.554892\du}}
\pgfpathlineto{\pgfpoint{24.525119\du}{10.553797\du}}
\pgfpathlineto{\pgfpoint{24.523293\du}{10.552337\du}}
\pgfpathlineto{\pgfpoint{24.521468\du}{10.551972\du}}
\pgfpathlineto{\pgfpoint{24.519642\du}{10.551972\du}}
\pgfpathlineto{\pgfpoint{24.517452\du}{10.551972\du}}
\pgfpathlineto{\pgfpoint{24.515991\du}{10.552337\du}}
\pgfpathlineto{\pgfpoint{24.513801\du}{10.553797\du}}
\pgfpathlineto{\pgfpoint{24.511975\du}{10.554892\du}}
\pgfpathlineto{\pgfpoint{24.511610\du}{10.555257\du}}
\pgfpathlineto{\pgfpoint{24.511245\du}{10.555988\du}}
\pgfpathlineto{\pgfpoint{24.510515\du}{10.557083\du}}
\pgfpathlineto{\pgfpoint{24.509785\du}{10.557813\du}}
\pgfpathlineto{\pgfpoint{24.509785\du}{10.558908\du}}
\pgfpathlineto{\pgfpoint{24.509419\du}{10.560004\du}}
\pgfpathlineto{\pgfpoint{24.509419\du}{10.560734\du}}
\pgfpathlineto{\pgfpoint{24.509419\du}{10.562559\du}}
\pgfusepath{fill}
\pgfsetbuttcap
\pgfsetmiterjoin
\pgfsetdash{}{0pt}
\definecolor{dialinecolor}{rgb}{0.678431, 0.839216, 0.905882}
\pgfsetfillcolor{dialinecolor}
\pgfpathmoveto{\pgfpoint{26.200183\du}{9.975119\du}}
\pgfpathlineto{\pgfpoint{26.200183\du}{9.975119\du}}
\pgfpathlineto{\pgfpoint{26.156371\du}{9.975119\du}}
\pgfpathlineto{\pgfpoint{26.113655\du}{9.975484\du}}
\pgfpathlineto{\pgfpoint{26.070208\du}{9.976579\du}}
\pgfpathlineto{\pgfpoint{26.028222\du}{9.978039\du}}
\pgfpathlineto{\pgfpoint{25.985871\du}{9.979500\du}}
\pgfpathlineto{\pgfpoint{25.944250\du}{9.981325\du}}
\pgfpathlineto{\pgfpoint{25.902264\du}{9.983881\du}}
\pgfpathlineto{\pgfpoint{25.861008\du}{9.986802\du}}
\pgfpathlineto{\pgfpoint{25.820117\du}{9.989723\du}}
\pgfpathlineto{\pgfpoint{25.779226\du}{9.992643\du}}
\pgfpathlineto{\pgfpoint{25.739430\du}{9.996659\du}}
\pgfpathlineto{\pgfpoint{25.699270\du}{10.000675\du}}
\pgfpathlineto{\pgfpoint{25.660204\du}{10.005422\du}}
\pgfpathlineto{\pgfpoint{25.621504\du}{10.010168\du}}
\pgfpathlineto{\pgfpoint{25.582804\du}{10.014914\du}}
\pgfpathlineto{\pgfpoint{25.544834\du}{10.020026\du}}
\pgfpathlineto{\pgfpoint{25.507229\du}{10.025867\du}}
\pgfpathlineto{\pgfpoint{25.469989\du}{10.031709\du}}
\pgfpathlineto{\pgfpoint{25.433479\du}{10.038280\du}}
\pgfpathlineto{\pgfpoint{25.397335\du}{10.044487\du}}
\pgfpathlineto{\pgfpoint{25.361555\du}{10.051789\du}}
\pgfpathlineto{\pgfpoint{25.326871\du}{10.058726\du}}
\pgfpathlineto{\pgfpoint{25.292552\du}{10.065663\du}}
\pgfpathlineto{\pgfpoint{25.257868\du}{10.073695\du}}
\pgfpathlineto{\pgfpoint{25.225374\du}{10.081362\du}}
\pgfpathlineto{\pgfpoint{25.192150\du}{10.089759\du}}
\pgfpathlineto{\pgfpoint{25.160387\du}{10.098521\du}}
\pgfpathlineto{\pgfpoint{25.128624\du}{10.107284\du}}
\pgfpathlineto{\pgfpoint{25.097955\du}{10.116046\du}}
\pgfpathlineto{\pgfpoint{25.066922\du}{10.125539\du}}
\pgfpathlineto{\pgfpoint{25.052318\du}{10.129920\du}}
\pgfpathlineto{\pgfpoint{25.038445\du}{10.134666\du}}
\pgfpathlineto{\pgfpoint{25.022746\du}{10.139412\du}}
\pgfpathlineto{\pgfpoint{25.008872\du}{10.144524\du}}
\pgfpathlineto{\pgfpoint{24.994268\du}{10.149270\du}}
\pgfpathlineto{\pgfpoint{24.980394\du}{10.154746\du}}
\pgfpathlineto{\pgfpoint{24.966156\du}{10.159127\du}}
\pgfpathlineto{\pgfpoint{24.953012\du}{10.164604\du}}
\pgfpathlineto{\pgfpoint{24.938773\du}{10.169715\du}}
\pgfpathlineto{\pgfpoint{24.925630\du}{10.175192\du}}
\pgfpathlineto{\pgfpoint{24.913217\du}{10.180303\du}}
\pgfpathlineto{\pgfpoint{24.899343\du}{10.185779\du}}
\pgfpathlineto{\pgfpoint{24.886564\du}{10.190891\du}}
\pgfpathlineto{\pgfpoint{24.874516\du}{10.196002\du}}
\pgfpathlineto{\pgfpoint{24.861373\du}{10.201844\du}}
\pgfpathlineto{\pgfpoint{24.849325\du}{10.208050\du}}
\pgfpathlineto{\pgfpoint{24.836911\du}{10.213162\du}}
\pgfpathlineto{\pgfpoint{24.825228\du}{10.219003\du}}
\pgfpathlineto{\pgfpoint{24.813910\du}{10.224845\du}}
\pgfpathlineto{\pgfpoint{24.802227\du}{10.230686\du}}
\pgfpathlineto{\pgfpoint{24.790909\du}{10.236528\du}}
\pgfpathlineto{\pgfpoint{24.779226\du}{10.241639\du}}
\pgfpathlineto{\pgfpoint{24.768273\du}{10.248211\du}}
\pgfpathlineto{\pgfpoint{24.758050\du}{10.254053\du}}
\pgfpathlineto{\pgfpoint{24.747463\du}{10.259894\du}}
\pgfpathlineto{\pgfpoint{24.737240\du}{10.265736\du}}
\pgfpathlineto{\pgfpoint{24.727017\du}{10.272307\du}}
\pgfpathlineto{\pgfpoint{24.717160\du}{10.278514\du}}
\pgfpathlineto{\pgfpoint{24.707667\du}{10.284356\du}}
\pgfpathlineto{\pgfpoint{24.698175\du}{10.290927\du}}
\pgfpathlineto{\pgfpoint{24.689047\du}{10.297499\du}}
\pgfpathlineto{\pgfpoint{24.679189\du}{10.303706\du}}
\pgfpathlineto{\pgfpoint{24.670792\du}{10.310277\du}}
\pgfpathlineto{\pgfpoint{24.662030\du}{10.316849\du}}
\pgfpathlineto{\pgfpoint{24.653633\du}{10.323056\du}}
\pgfpathlineto{\pgfpoint{24.644870\du}{10.329628\du}}
\pgfpathlineto{\pgfpoint{24.637203\du}{10.336564\du}}
\pgfpathlineto{\pgfpoint{24.630267\du}{10.343136\du}}
\pgfpathlineto{\pgfpoint{24.621504\du}{10.349343\du}}
\pgfpathlineto{\pgfpoint{24.614567\du}{10.356645\du}}
\pgfpathlineto{\pgfpoint{24.607631\du}{10.362851\du}}
\pgfpathlineto{\pgfpoint{24.600694\du}{10.369788\du}}
\pgfpathlineto{\pgfpoint{24.594122\du}{10.377090\du}}
\pgfpathlineto{\pgfpoint{24.587185\du}{10.383297\du}}
\pgfpathlineto{\pgfpoint{24.581344\du}{10.390599\du}}
\pgfpathlineto{\pgfpoint{24.575502\du}{10.397536\du}}
\pgfpathlineto{\pgfpoint{24.569660\du}{10.405203\du}}
\pgfpathlineto{\pgfpoint{24.564184\du}{10.412139\du}}
\pgfpathlineto{\pgfpoint{24.559438\du}{10.419076\du}}
\pgfpathlineto{\pgfpoint{24.553961\du}{10.426013\du}}
\pgfpathlineto{\pgfpoint{24.549945\du}{10.433315\du}}
\pgfpathlineto{\pgfpoint{24.544834\du}{10.440617\du}}
\pgfpathlineto{\pgfpoint{24.541183\du}{10.447919\du}}
\pgfpathlineto{\pgfpoint{24.536802\du}{10.455221\du}}
\pgfpathlineto{\pgfpoint{24.533151\du}{10.462888\du}}
\pgfpathlineto{\pgfpoint{24.529865\du}{10.470555\du}}
\pgfpathlineto{\pgfpoint{24.526214\du}{10.477492\du}}
\pgfpathlineto{\pgfpoint{24.523293\du}{10.485159\du}}
\pgfpathlineto{\pgfpoint{24.520737\du}{10.492826\du}}
\pgfpathlineto{\pgfpoint{24.518547\du}{10.500493\du}}
\pgfpathlineto{\pgfpoint{24.516356\du}{10.508160\du}}
\pgfpathlineto{\pgfpoint{24.514531\du}{10.515462\du}}
\pgfpathlineto{\pgfpoint{24.512705\du}{10.523129\du}}
\pgfpathlineto{\pgfpoint{24.511610\du}{10.530796\du}}
\pgfpathlineto{\pgfpoint{24.510515\du}{10.539193\du}}
\pgfpathlineto{\pgfpoint{24.509785\du}{10.546495\du}}
\pgfpathlineto{\pgfpoint{24.509419\du}{10.554162\du}}
\pgfpathlineto{\pgfpoint{24.509419\du}{10.562559\du}}
\pgfpathlineto{\pgfpoint{24.529865\du}{10.562559\du}}
\pgfpathlineto{\pgfpoint{24.530230\du}{10.555257\du}}
\pgfpathlineto{\pgfpoint{24.530230\du}{10.548321\du}}
\pgfpathlineto{\pgfpoint{24.530595\du}{10.541384\du}}
\pgfpathlineto{\pgfpoint{24.532055\du}{10.533717\du}}
\pgfpathlineto{\pgfpoint{24.533151\du}{10.527510\du}}
\pgfpathlineto{\pgfpoint{24.534246\du}{10.520208\du}}
\pgfpathlineto{\pgfpoint{24.536437\du}{10.513271\du}}
\pgfpathlineto{\pgfpoint{24.537897\du}{10.506334\du}}
\pgfpathlineto{\pgfpoint{24.540088\du}{10.499398\du}}
\pgfpathlineto{\pgfpoint{24.542278\du}{10.492096\du}}
\pgfpathlineto{\pgfpoint{24.545564\du}{10.485889\du}}
\pgfpathlineto{\pgfpoint{24.548120\du}{10.478952\du}}
\pgfpathlineto{\pgfpoint{24.551041\du}{10.471650\du}}
\pgfpathlineto{\pgfpoint{24.555057\du}{10.464713\du}}
\pgfpathlineto{\pgfpoint{24.559073\du}{10.457777\du}}
\pgfpathlineto{\pgfpoint{24.562359\du}{10.451205\du}}
\pgfpathlineto{\pgfpoint{24.566375\du}{10.444268\du}}
\pgfpathlineto{\pgfpoint{24.571121\du}{10.437696\du}}
\pgfpathlineto{\pgfpoint{24.575502\du}{10.430759\du}}
\pgfpathlineto{\pgfpoint{24.580978\du}{10.424188\du}}
\pgfpathlineto{\pgfpoint{24.585360\du}{10.417981\du}}
\pgfpathlineto{\pgfpoint{24.591566\du}{10.411044\du}}
\pgfpathlineto{\pgfpoint{24.596678\du}{10.404472\du}}
\pgfpathlineto{\pgfpoint{24.602884\du}{10.397536\du}}
\pgfpathlineto{\pgfpoint{24.608726\du}{10.390964\du}}
\pgfpathlineto{\pgfpoint{24.615298\du}{10.384757\du}}
\pgfpathlineto{\pgfpoint{24.621504\du}{10.378185\du}}
\pgfpathlineto{\pgfpoint{24.628441\du}{10.371249\du}}
\pgfpathlineto{\pgfpoint{24.635743\du}{10.365407\du}}
\pgfpathlineto{\pgfpoint{24.642680\du}{10.358105\du}}
\pgfpathlineto{\pgfpoint{24.650347\du}{10.351899\du}}
\pgfpathlineto{\pgfpoint{24.658379\du}{10.346057\du}}
\pgfpathlineto{\pgfpoint{24.665681\du}{10.339485\du}}
\pgfpathlineto{\pgfpoint{24.674078\du}{10.332913\du}}
\pgfpathlineto{\pgfpoint{24.682840\du}{10.326707\du}}
\pgfpathlineto{\pgfpoint{24.691968\du}{10.320135\du}}
\pgfpathlineto{\pgfpoint{24.700730\du}{10.314294\du}}
\pgfpathlineto{\pgfpoint{24.709127\du}{10.308087\du}}
\pgfpathlineto{\pgfpoint{24.718255\du}{10.302245\du}}
\pgfpathlineto{\pgfpoint{24.728112\du}{10.295674\du}}
\pgfpathlineto{\pgfpoint{24.737605\du}{10.289832\du}}
\pgfpathlineto{\pgfpoint{24.747463\du}{10.283991\du}}
\pgfpathlineto{\pgfpoint{24.758050\du}{10.278149\du}}
\pgfpathlineto{\pgfpoint{24.767908\du}{10.272307\du}}
\pgfpathlineto{\pgfpoint{24.778496\du}{10.265736\du}}
\pgfpathlineto{\pgfpoint{24.789814\du}{10.260624\du}}
\pgfpathlineto{\pgfpoint{24.799671\du}{10.254783\du}}
\pgfpathlineto{\pgfpoint{24.811355\du}{10.248941\du}}
\pgfpathlineto{\pgfpoint{24.822307\du}{10.243100\du}}
\pgfpathlineto{\pgfpoint{24.833991\du}{10.237258\du}}
\pgfpathlineto{\pgfpoint{24.845674\du}{10.231782\du}}
\pgfpathlineto{\pgfpoint{24.858452\du}{10.226670\du}}
\pgfpathlineto{\pgfpoint{24.869770\du}{10.220829\du}}
\pgfpathlineto{\pgfpoint{24.881818\du}{10.215352\du}}
\pgfpathlineto{\pgfpoint{24.894962\du}{10.210241\du}}
\pgfpathlineto{\pgfpoint{24.907375\du}{10.204765\du}}
\pgfpathlineto{\pgfpoint{24.920153\du}{10.199653\du}}
\pgfpathlineto{\pgfpoint{24.933662\du}{10.193812\du}}
\pgfpathlineto{\pgfpoint{24.946440\du}{10.189065\du}}
\pgfpathlineto{\pgfpoint{24.959949\du}{10.183954\du}}
\pgfpathlineto{\pgfpoint{24.973457\du}{10.178478\du}}
\pgfpathlineto{\pgfpoint{24.987331\du}{10.174096\du}}
\pgfpathlineto{\pgfpoint{25.001205\du}{10.168620\du}}
\pgfpathlineto{\pgfpoint{25.015444\du}{10.163874\du}}
\pgfpathlineto{\pgfpoint{25.029682\du}{10.159127\du}}
\pgfpathlineto{\pgfpoint{25.043921\du}{10.154016\du}}
\pgfpathlineto{\pgfpoint{25.058160\du}{10.149270\du}}
\pgfpathlineto{\pgfpoint{25.073494\du}{10.144524\du}}
\pgfpathlineto{\pgfpoint{25.103067\du}{10.135761\du}}
\pgfpathlineto{\pgfpoint{25.134465\du}{10.126634\du}}
\pgfpathlineto{\pgfpoint{25.165498\du}{10.118237\du}}
\pgfpathlineto{\pgfpoint{25.197992\du}{10.109474\du}}
\pgfpathlineto{\pgfpoint{25.229390\du}{10.101442\du}}
\pgfpathlineto{\pgfpoint{25.262614\du}{10.093775\du}}
\pgfpathlineto{\pgfpoint{25.296568\du}{10.086108\du}}
\pgfpathlineto{\pgfpoint{25.330887\du}{10.078441\du}}
\pgfpathlineto{\pgfpoint{25.365571\du}{10.071504\du}}
\pgfpathlineto{\pgfpoint{25.401716\du}{10.064567\du}}
\pgfpathlineto{\pgfpoint{25.436765\du}{10.058726\du}}
\pgfpathlineto{\pgfpoint{25.474005\du}{10.052154\du}}
\pgfpathlineto{\pgfpoint{25.510515\du}{10.046313\du}}
\pgfpathlineto{\pgfpoint{25.547755\du}{10.040471\du}}
\pgfpathlineto{\pgfpoint{25.585360\du}{10.035360\du}}
\pgfpathlineto{\pgfpoint{25.623695\du}{10.029883\du}}
\pgfpathlineto{\pgfpoint{25.662760\du}{10.025137\du}}
\pgfpathlineto{\pgfpoint{25.702191\du}{10.021121\du}}
\pgfpathlineto{\pgfpoint{25.741621\du}{10.017105\du}}
\pgfpathlineto{\pgfpoint{25.781417\du}{10.013454\du}}
\pgfpathlineto{\pgfpoint{25.821577\du}{10.010168\du}}
\pgfpathlineto{\pgfpoint{25.862468\du}{10.007247\du}}
\pgfpathlineto{\pgfpoint{25.903724\du}{10.004326\du}}
\pgfpathlineto{\pgfpoint{25.944615\du}{10.001771\du}}
\pgfpathlineto{\pgfpoint{25.986601\du}{9.999580\du}}
\pgfpathlineto{\pgfpoint{26.028952\du}{9.998485\du}}
\pgfpathlineto{\pgfpoint{26.071668\du}{9.996659\du}}
\pgfpathlineto{\pgfpoint{26.114020\du}{9.995929\du}}
\pgfpathlineto{\pgfpoint{26.157466\du}{9.995564\du}}
\pgfpathlineto{\pgfpoint{26.200183\du}{9.995564\du}}
\pgfpathlineto{\pgfpoint{26.200183\du}{9.995564\du}}
\pgfpathlineto{\pgfpoint{26.200183\du}{9.995564\du}}
\pgfpathlineto{\pgfpoint{26.201278\du}{9.994834\du}}
\pgfpathlineto{\pgfpoint{26.202373\du}{9.994834\du}}
\pgfpathlineto{\pgfpoint{26.203834\du}{9.994834\du}}
\pgfpathlineto{\pgfpoint{26.204929\du}{9.994469\du}}
\pgfpathlineto{\pgfpoint{26.205294\du}{9.993739\du}}
\pgfpathlineto{\pgfpoint{26.206389\du}{9.993739\du}}
\pgfpathlineto{\pgfpoint{26.207119\du}{9.992643\du}}
\pgfpathlineto{\pgfpoint{26.208215\du}{9.991913\du}}
\pgfpathlineto{\pgfpoint{26.209310\du}{9.990818\du}}
\pgfpathlineto{\pgfpoint{26.210040\du}{9.988992\du}}
\pgfpathlineto{\pgfpoint{26.210040\du}{9.986802\du}}
\pgfpathlineto{\pgfpoint{26.210770\du}{9.984976\du}}
\pgfpathlineto{\pgfpoint{26.210040\du}{9.983151\du}}
\pgfpathlineto{\pgfpoint{26.210040\du}{9.981325\du}}
\pgfpathlineto{\pgfpoint{26.209310\du}{9.979500\du}}
\pgfpathlineto{\pgfpoint{26.208215\du}{9.978039\du}}
\pgfpathlineto{\pgfpoint{26.207119\du}{9.977309\du}}
\pgfpathlineto{\pgfpoint{26.206389\du}{9.976579\du}}
\pgfpathlineto{\pgfpoint{26.205294\du}{9.976214\du}}
\pgfpathlineto{\pgfpoint{26.204929\du}{9.975484\du}}
\pgfpathlineto{\pgfpoint{26.203834\du}{9.975119\du}}
\pgfpathlineto{\pgfpoint{26.202373\du}{9.975119\du}}
\pgfpathlineto{\pgfpoint{26.201278\du}{9.975119\du}}
\pgfpathlineto{\pgfpoint{26.200183\du}{9.975119\du}}
\pgfusepath{fill}
\pgfsetbuttcap
\pgfsetmiterjoin
\pgfsetdash{}{0pt}
\definecolor{dialinecolor}{rgb}{0.678431, 0.839216, 0.905882}
\pgfsetfillcolor{dialinecolor}
\pgfpathmoveto{\pgfpoint{27.890581\du}{10.562559\du}}
\pgfpathlineto{\pgfpoint{27.890581\du}{10.554162\du}}
\pgfpathlineto{\pgfpoint{27.889850\du}{10.546495\du}}
\pgfpathlineto{\pgfpoint{27.889120\du}{10.539193\du}}
\pgfpathlineto{\pgfpoint{27.888390\du}{10.530796\du}}
\pgfpathlineto{\pgfpoint{27.887295\du}{10.523129\du}}
\pgfpathlineto{\pgfpoint{27.885104\du}{10.515462\du}}
\pgfpathlineto{\pgfpoint{27.884009\du}{10.508160\du}}
\pgfpathlineto{\pgfpoint{27.881088\du}{10.500493\du}}
\pgfpathlineto{\pgfpoint{27.879263\du}{10.492826\du}}
\pgfpathlineto{\pgfpoint{27.876342\du}{10.485159\du}}
\pgfpathlineto{\pgfpoint{27.873786\du}{10.477492\du}}
\pgfpathlineto{\pgfpoint{27.870500\du}{10.470555\du}}
\pgfpathlineto{\pgfpoint{27.866484\du}{10.462888\du}}
\pgfpathlineto{\pgfpoint{27.863198\du}{10.455221\du}}
\pgfpathlineto{\pgfpoint{27.859182\du}{10.447919\du}}
\pgfpathlineto{\pgfpoint{27.855166\du}{10.440617\du}}
\pgfpathlineto{\pgfpoint{27.850785\du}{10.433315\du}}
\pgfpathlineto{\pgfpoint{27.846039\du}{10.426013\du}}
\pgfpathlineto{\pgfpoint{27.840562\du}{10.419076\du}}
\pgfpathlineto{\pgfpoint{27.835816\du}{10.412139\du}}
\pgfpathlineto{\pgfpoint{27.829974\du}{10.404472\du}}
\pgfpathlineto{\pgfpoint{27.824498\du}{10.397536\du}}
\pgfpathlineto{\pgfpoint{27.818656\du}{10.390599\du}}
\pgfpathlineto{\pgfpoint{27.812450\du}{10.383297\du}}
\pgfpathlineto{\pgfpoint{27.805878\du}{10.377090\du}}
\pgfpathlineto{\pgfpoint{27.798941\du}{10.369788\du}}
\pgfpathlineto{\pgfpoint{27.792369\du}{10.362851\du}}
\pgfpathlineto{\pgfpoint{27.785068\du}{10.356645\du}}
\pgfpathlineto{\pgfpoint{27.778496\du}{10.349343\du}}
\pgfpathlineto{\pgfpoint{27.770464\du}{10.343136\du}}
\pgfpathlineto{\pgfpoint{27.762432\du}{10.336564\du}}
\pgfpathlineto{\pgfpoint{27.755130\du}{10.329628\du}}
\pgfpathlineto{\pgfpoint{27.746367\du}{10.323056\du}}
\pgfpathlineto{\pgfpoint{27.738335\du}{10.316849\du}}
\pgfpathlineto{\pgfpoint{27.729208\du}{10.310277\du}}
\pgfpathlineto{\pgfpoint{27.720445\du}{10.303706\du}}
\pgfpathlineto{\pgfpoint{27.710953\du}{10.297499\du}}
\pgfpathlineto{\pgfpoint{27.701825\du}{10.290927\du}}
\pgfpathlineto{\pgfpoint{27.692698\du}{10.284356\du}}
\pgfpathlineto{\pgfpoint{27.682840\du}{10.278514\du}}
\pgfpathlineto{\pgfpoint{27.672983\du}{10.272307\du}}
\pgfpathlineto{\pgfpoint{27.662760\du}{10.265736\du}}
\pgfpathlineto{\pgfpoint{27.652172\du}{10.259894\du}}
\pgfpathlineto{\pgfpoint{27.642315\du}{10.254053\du}}
\pgfpathlineto{\pgfpoint{27.631727\du}{10.248211\du}}
\pgfpathlineto{\pgfpoint{27.620409\du}{10.241639\du}}
\pgfpathlineto{\pgfpoint{27.609091\du}{10.236528\du}}
\pgfpathlineto{\pgfpoint{27.597408\du}{10.230686\du}}
\pgfpathlineto{\pgfpoint{27.586455\du}{10.224845\du}}
\pgfpathlineto{\pgfpoint{27.574407\du}{10.219003\du}}
\pgfpathlineto{\pgfpoint{27.563089\du}{10.213162\du}}
\pgfpathlineto{\pgfpoint{27.550675\du}{10.208050\du}}
\pgfpathlineto{\pgfpoint{27.538262\du}{10.201844\du}}
\pgfpathlineto{\pgfpoint{27.525484\du}{10.196002\du}}
\pgfpathlineto{\pgfpoint{27.513436\du}{10.190891\du}}
\pgfpathlineto{\pgfpoint{27.500657\du}{10.185779\du}}
\pgfpathlineto{\pgfpoint{27.487149\du}{10.180303\du}}
\pgfpathlineto{\pgfpoint{27.474005\du}{10.175192\du}}
\pgfpathlineto{\pgfpoint{27.460862\du}{10.169715\du}}
\pgfpathlineto{\pgfpoint{27.446623\du}{10.164604\du}}
\pgfpathlineto{\pgfpoint{27.433479\du}{10.159127\du}}
\pgfpathlineto{\pgfpoint{27.419241\du}{10.154746\du}}
\pgfpathlineto{\pgfpoint{27.405367\du}{10.149270\du}}
\pgfpathlineto{\pgfpoint{27.391128\du}{10.144524\du}}
\pgfpathlineto{\pgfpoint{27.377254\du}{10.139412\du}}
\pgfpathlineto{\pgfpoint{27.362286\du}{10.134666\du}}
\pgfpathlineto{\pgfpoint{27.347317\du}{10.129920\du}}
\pgfpathlineto{\pgfpoint{27.333078\du}{10.125539\du}}
\pgfpathlineto{\pgfpoint{27.302410\du}{10.116046\du}}
\pgfpathlineto{\pgfpoint{27.271742\du}{10.107284\du}}
\pgfpathlineto{\pgfpoint{27.240343\du}{10.098521\du}}
\pgfpathlineto{\pgfpoint{27.208580\du}{10.089759\du}}
\pgfpathlineto{\pgfpoint{27.175721\du}{10.081362\du}}
\pgfpathlineto{\pgfpoint{27.142132\du}{10.073695\du}}
\pgfpathlineto{\pgfpoint{27.108178\du}{10.065663\du}}
\pgfpathlineto{\pgfpoint{27.073494\du}{10.058726\du}}
\pgfpathlineto{\pgfpoint{27.038810\du}{10.051789\du}}
\pgfpathlineto{\pgfpoint{27.003395\du}{10.044487\du}}
\pgfpathlineto{\pgfpoint{26.966886\du}{10.038280\du}}
\pgfpathlineto{\pgfpoint{26.930376\du}{10.031709\du}}
\pgfpathlineto{\pgfpoint{26.893501\du}{10.025867\du}}
\pgfpathlineto{\pgfpoint{26.856261\du}{10.020026\du}}
\pgfpathlineto{\pgfpoint{26.817196\du}{10.014914\du}}
\pgfpathlineto{\pgfpoint{26.779226\du}{10.010168\du}}
\pgfpathlineto{\pgfpoint{26.739796\du}{10.005422\du}}
\pgfpathlineto{\pgfpoint{26.701095\du}{10.000675\du}}
\pgfpathlineto{\pgfpoint{26.660935\du}{9.996659\du}}
\pgfpathlineto{\pgfpoint{26.621139\du}{9.992643\du}}
\pgfpathlineto{\pgfpoint{26.580248\du}{9.989723\du}}
\pgfpathlineto{\pgfpoint{26.539723\du}{9.986802\du}}
\pgfpathlineto{\pgfpoint{26.498101\du}{9.983881\du}}
\pgfpathlineto{\pgfpoint{26.456480\du}{9.981325\du}}
\pgfpathlineto{\pgfpoint{26.414859\du}{9.979500\du}}
\pgfpathlineto{\pgfpoint{26.372143\du}{9.978039\du}}
\pgfpathlineto{\pgfpoint{26.330157\du}{9.976579\du}}
\pgfpathlineto{\pgfpoint{26.287076\du}{9.975484\du}}
\pgfpathlineto{\pgfpoint{26.243629\du}{9.975119\du}}
\pgfpathlineto{\pgfpoint{26.200183\du}{9.975119\du}}
\pgfpathlineto{\pgfpoint{26.200183\du}{9.995564\du}}
\pgfpathlineto{\pgfpoint{26.243264\du}{9.995564\du}}
\pgfpathlineto{\pgfpoint{26.286710\du}{9.995929\du}}
\pgfpathlineto{\pgfpoint{26.328697\du}{9.996659\du}}
\pgfpathlineto{\pgfpoint{26.371413\du}{9.998485\du}}
\pgfpathlineto{\pgfpoint{26.414129\du}{9.999580\du}}
\pgfpathlineto{\pgfpoint{26.455750\du}{10.001771\du}}
\pgfpathlineto{\pgfpoint{26.497006\du}{10.004326\du}}
\pgfpathlineto{\pgfpoint{26.537897\du}{10.007247\du}}
\pgfpathlineto{\pgfpoint{26.579153\du}{10.010168\du}}
\pgfpathlineto{\pgfpoint{26.619314\du}{10.013454\du}}
\pgfpathlineto{\pgfpoint{26.659474\du}{10.017105\du}}
\pgfpathlineto{\pgfpoint{26.698540\du}{10.021121\du}}
\pgfpathlineto{\pgfpoint{26.737970\du}{10.025137\du}}
\pgfpathlineto{\pgfpoint{26.776670\du}{10.029883\du}}
\pgfpathlineto{\pgfpoint{26.815371\du}{10.035360\du}}
\pgfpathlineto{\pgfpoint{26.852976\du}{10.040471\du}}
\pgfpathlineto{\pgfpoint{26.890215\du}{10.046313\du}}
\pgfpathlineto{\pgfpoint{26.926725\du}{10.052154\du}}
\pgfpathlineto{\pgfpoint{26.963600\du}{10.058726\du}}
\pgfpathlineto{\pgfpoint{26.999014\du}{10.064567\du}}
\pgfpathlineto{\pgfpoint{27.034794\du}{10.071504\du}}
\pgfpathlineto{\pgfpoint{27.069478\du}{10.078441\du}}
\pgfpathlineto{\pgfpoint{27.103432\du}{10.086108\du}}
\pgfpathlineto{\pgfpoint{27.137751\du}{10.093775\du}}
\pgfpathlineto{\pgfpoint{27.170975\du}{10.101442\du}}
\pgfpathlineto{\pgfpoint{27.203103\du}{10.109474\du}}
\pgfpathlineto{\pgfpoint{27.235232\du}{10.118237\du}}
\pgfpathlineto{\pgfpoint{27.266630\du}{10.126634\du}}
\pgfpathlineto{\pgfpoint{27.297298\du}{10.135761\du}}
\pgfpathlineto{\pgfpoint{27.327236\du}{10.144524\du}}
\pgfpathlineto{\pgfpoint{27.341475\du}{10.149270\du}}
\pgfpathlineto{\pgfpoint{27.356079\du}{10.154016\du}}
\pgfpathlineto{\pgfpoint{27.369953\du}{10.159127\du}}
\pgfpathlineto{\pgfpoint{27.384556\du}{10.163874\du}}
\pgfpathlineto{\pgfpoint{27.398795\du}{10.168620\du}}
\pgfpathlineto{\pgfpoint{27.412669\du}{10.174096\du}}
\pgfpathlineto{\pgfpoint{27.426908\du}{10.178478\du}}
\pgfpathlineto{\pgfpoint{27.440051\du}{10.183954\du}}
\pgfpathlineto{\pgfpoint{27.453560\du}{10.189065\du}}
\pgfpathlineto{\pgfpoint{27.466703\du}{10.193812\du}}
\pgfpathlineto{\pgfpoint{27.479482\du}{10.199653\du}}
\pgfpathlineto{\pgfpoint{27.492260\du}{10.204765\du}}
\pgfpathlineto{\pgfpoint{27.505403\du}{10.210241\du}}
\pgfpathlineto{\pgfpoint{27.517452\du}{10.215352\du}}
\pgfpathlineto{\pgfpoint{27.529865\du}{10.220829\du}}
\pgfpathlineto{\pgfpoint{27.541913\du}{10.226670\du}}
\pgfpathlineto{\pgfpoint{27.554326\du}{10.231782\du}}
\pgfpathlineto{\pgfpoint{27.565644\du}{10.237258\du}}
\pgfpathlineto{\pgfpoint{27.577693\du}{10.243100\du}}
\pgfpathlineto{\pgfpoint{27.588280\du}{10.248941\du}}
\pgfpathlineto{\pgfpoint{27.600329\du}{10.254783\du}}
\pgfpathlineto{\pgfpoint{27.610551\du}{10.260624\du}}
\pgfpathlineto{\pgfpoint{27.621504\du}{10.265736\du}}
\pgfpathlineto{\pgfpoint{27.632457\du}{10.272307\du}}
\pgfpathlineto{\pgfpoint{27.642315\du}{10.278149\du}}
\pgfpathlineto{\pgfpoint{27.652172\du}{10.283991\du}}
\pgfpathlineto{\pgfpoint{27.662395\du}{10.289832\du}}
\pgfpathlineto{\pgfpoint{27.671522\du}{10.295674\du}}
\pgfpathlineto{\pgfpoint{27.681745\du}{10.302245\du}}
\pgfpathlineto{\pgfpoint{27.691238\du}{10.308087\du}}
\pgfpathlineto{\pgfpoint{27.700000\du}{10.314294\du}}
\pgfpathlineto{\pgfpoint{27.708032\du}{10.320135\du}}
\pgfpathlineto{\pgfpoint{27.716794\du}{10.326707\du}}
\pgfpathlineto{\pgfpoint{27.725192\du}{10.332913\du}}
\pgfpathlineto{\pgfpoint{27.733954\du}{10.339485\du}}
\pgfpathlineto{\pgfpoint{27.741621\du}{10.346057\du}}
\pgfpathlineto{\pgfpoint{27.749653\du}{10.351899\du}}
\pgfpathlineto{\pgfpoint{27.756955\du}{10.358105\du}}
\pgfpathlineto{\pgfpoint{27.764622\du}{10.365407\du}}
\pgfpathlineto{\pgfpoint{27.771194\du}{10.371249\du}}
\pgfpathlineto{\pgfpoint{27.778496\du}{10.378185\du}}
\pgfpathlineto{\pgfpoint{27.784337\du}{10.384757\du}}
\pgfpathlineto{\pgfpoint{27.791274\du}{10.390964\du}}
\pgfpathlineto{\pgfpoint{27.797846\du}{10.397536\du}}
\pgfpathlineto{\pgfpoint{27.802957\du}{10.404472\du}}
\pgfpathlineto{\pgfpoint{27.808434\du}{10.411044\du}}
\pgfpathlineto{\pgfpoint{27.814640\du}{10.417981\du}}
\pgfpathlineto{\pgfpoint{27.819387\du}{10.424188\du}}
\pgfpathlineto{\pgfpoint{27.824498\du}{10.430759\du}}
\pgfpathlineto{\pgfpoint{27.829244\du}{10.437696\du}}
\pgfpathlineto{\pgfpoint{27.833625\du}{10.444268\du}}
\pgfpathlineto{\pgfpoint{27.838007\du}{10.451205\du}}
\pgfpathlineto{\pgfpoint{27.840927\du}{10.457777\du}}
\pgfpathlineto{\pgfpoint{27.844943\du}{10.464713\du}}
\pgfpathlineto{\pgfpoint{27.847864\du}{10.471650\du}}
\pgfpathlineto{\pgfpoint{27.851880\du}{10.478952\du}}
\pgfpathlineto{\pgfpoint{27.854436\du}{10.485159\du}}
\pgfpathlineto{\pgfpoint{27.857357\du}{10.492096\du}}
\pgfpathlineto{\pgfpoint{27.859912\du}{10.499398\du}}
\pgfpathlineto{\pgfpoint{27.862103\du}{10.506334\du}}
\pgfpathlineto{\pgfpoint{27.863563\du}{10.513271\du}}
\pgfpathlineto{\pgfpoint{27.865754\du}{10.520208\du}}
\pgfpathlineto{\pgfpoint{27.866484\du}{10.527510\du}}
\pgfpathlineto{\pgfpoint{27.867579\du}{10.533717\du}}
\pgfpathlineto{\pgfpoint{27.869405\du}{10.541384\du}}
\pgfpathlineto{\pgfpoint{27.869770\du}{10.548321\du}}
\pgfpathlineto{\pgfpoint{27.869770\du}{10.555257\du}}
\pgfpathlineto{\pgfpoint{27.870500\du}{10.562559\du}}
\pgfpathlineto{\pgfpoint{27.890581\du}{10.562559\du}}
\pgfusepath{fill}
\pgfsetbuttcap
\pgfsetmiterjoin
\pgfsetdash{}{0pt}
\definecolor{dialinecolor}{rgb}{0.027451, 0.486275, 0.682353}
\pgfsetfillcolor{dialinecolor}
\pgfpathmoveto{\pgfpoint{24.514531\du}{9.753140\du}}
\pgfpathlineto{\pgfpoint{24.514531\du}{10.577528\du}}
\pgfpathlineto{\pgfpoint{27.879993\du}{10.577528\du}}
\pgfpathlineto{\pgfpoint{27.880723\du}{9.753870\du}}
\pgfpathlineto{\pgfpoint{24.514531\du}{9.753140\du}}
\pgfusepath{fill}
\pgfsetbuttcap
\pgfsetmiterjoin
\pgfsetdash{}{0pt}
\definecolor{dialinecolor}{rgb}{0.235294, 0.686275, 0.894118}
\pgfsetfillcolor{dialinecolor}
\pgfpathmoveto{\pgfpoint{27.879993\du}{9.737441\du}}
\pgfpathlineto{\pgfpoint{27.878532\du}{9.767379\du}}
\pgfpathlineto{\pgfpoint{27.871230\du}{9.796586\du}}
\pgfpathlineto{\pgfpoint{27.861008\du}{9.825794\du}}
\pgfpathlineto{\pgfpoint{27.846404\du}{9.853907\du}}
\pgfpathlineto{\pgfpoint{27.827054\du}{9.882019\du}}
\pgfpathlineto{\pgfpoint{27.805148\du}{9.909401\du}}
\pgfpathlineto{\pgfpoint{27.778496\du}{9.935688\du}}
\pgfpathlineto{\pgfpoint{27.747828\du}{9.961975\du}}
\pgfpathlineto{\pgfpoint{27.714969\du}{9.987897\du}}
\pgfpathlineto{\pgfpoint{27.677729\du}{10.013089\du}}
\pgfpathlineto{\pgfpoint{27.636838\du}{10.037185\du}}
\pgfpathlineto{\pgfpoint{27.593027\du}{10.060551\du}}
\pgfpathlineto{\pgfpoint{27.546659\du}{10.082822\du}}
\pgfpathlineto{\pgfpoint{27.496276\du}{10.104728\du}}
\pgfpathlineto{\pgfpoint{27.443337\du}{10.125904\du}}
\pgfpathlineto{\pgfpoint{27.387842\du}{10.145984\du}}
\pgfpathlineto{\pgfpoint{27.329792\du}{10.164604\du}}
\pgfpathlineto{\pgfpoint{27.269186\du}{10.183224\du}}
\pgfpathlineto{\pgfpoint{27.205294\du}{10.200383\du}}
\pgfpathlineto{\pgfpoint{27.140307\du}{10.216083\du}}
\pgfpathlineto{\pgfpoint{27.071668\du}{10.231417\du}}
\pgfpathlineto{\pgfpoint{27.000840\du}{10.245290\du}}
\pgfpathlineto{\pgfpoint{26.928916\du}{10.258069\du}}
\pgfpathlineto{\pgfpoint{26.854071\du}{10.269387\du}}
\pgfpathlineto{\pgfpoint{26.778131\du}{10.279974\du}}
\pgfpathlineto{\pgfpoint{26.699635\du}{10.289102\du}}
\pgfpathlineto{\pgfpoint{26.620044\du}{10.296769\du}}
\pgfpathlineto{\pgfpoint{26.538992\du}{10.303341\du}}
\pgfpathlineto{\pgfpoint{26.455750\du}{10.308452\du}}
\pgfpathlineto{\pgfpoint{26.372143\du}{10.312103\du}}
\pgfpathlineto{\pgfpoint{26.286710\du}{10.313928\du}}
\pgfpathlineto{\pgfpoint{26.200183\du}{10.315024\du}}
\pgfpathlineto{\pgfpoint{26.114020\du}{10.313928\du}}
\pgfpathlineto{\pgfpoint{26.028222\du}{10.312103\du}}
\pgfpathlineto{\pgfpoint{25.944615\du}{10.308452\du}}
\pgfpathlineto{\pgfpoint{25.861738\du}{10.303341\du}}
\pgfpathlineto{\pgfpoint{25.780321\du}{10.296769\du}}
\pgfpathlineto{\pgfpoint{25.700730\du}{10.289102\du}}
\pgfpathlineto{\pgfpoint{25.622965\du}{10.279974\du}}
\pgfpathlineto{\pgfpoint{25.546294\du}{10.269387\du}}
\pgfpathlineto{\pgfpoint{25.472180\du}{10.258069\du}}
\pgfpathlineto{\pgfpoint{25.399525\du}{10.245290\du}}
\pgfpathlineto{\pgfpoint{25.329062\du}{10.231417\du}}
\pgfpathlineto{\pgfpoint{25.260424\du}{10.216083\du}}
\pgfpathlineto{\pgfpoint{25.194706\du}{10.200383\du}}
\pgfpathlineto{\pgfpoint{25.131179\du}{10.183224\du}}
\pgfpathlineto{\pgfpoint{25.070208\du}{10.164604\du}}
\pgfpathlineto{\pgfpoint{25.011793\du}{10.145984\du}}
\pgfpathlineto{\pgfpoint{24.956663\du}{10.125904\du}}
\pgfpathlineto{\pgfpoint{24.903724\du}{10.104728\du}}
\pgfpathlineto{\pgfpoint{24.853706\du}{10.082822\du}}
\pgfpathlineto{\pgfpoint{24.806608\du}{10.060551\du}}
\pgfpathlineto{\pgfpoint{24.763162\du}{10.037185\du}}
\pgfpathlineto{\pgfpoint{24.722271\du}{10.013089\du}}
\pgfpathlineto{\pgfpoint{24.685031\du}{9.987897\du}}
\pgfpathlineto{\pgfpoint{24.651807\du}{9.961975\du}}
\pgfpathlineto{\pgfpoint{24.621504\du}{9.935688\du}}
\pgfpathlineto{\pgfpoint{24.594852\du}{9.909401\du}}
\pgfpathlineto{\pgfpoint{24.572946\du}{9.882019\du}}
\pgfpathlineto{\pgfpoint{24.553596\du}{9.853907\du}}
\pgfpathlineto{\pgfpoint{24.538992\du}{9.825794\du}}
\pgfpathlineto{\pgfpoint{24.528405\du}{9.796586\du}}
\pgfpathlineto{\pgfpoint{24.521468\du}{9.767379\du}}
\pgfpathlineto{\pgfpoint{24.519642\du}{9.737441\du}}
\pgfpathlineto{\pgfpoint{24.521468\du}{9.708233\du}}
\pgfpathlineto{\pgfpoint{24.528405\du}{9.678295\du}}
\pgfpathlineto{\pgfpoint{24.538992\du}{9.649817\du}}
\pgfpathlineto{\pgfpoint{24.553596\du}{9.621705\du}}
\pgfpathlineto{\pgfpoint{24.572946\du}{9.593593\du}}
\pgfpathlineto{\pgfpoint{24.594852\du}{9.565845\du}}
\pgfpathlineto{\pgfpoint{24.621504\du}{9.539193\du}}
\pgfpathlineto{\pgfpoint{24.651807\du}{9.512906\du}}
\pgfpathlineto{\pgfpoint{24.685031\du}{9.487714\du}}
\pgfpathlineto{\pgfpoint{24.722271\du}{9.462523\du}}
\pgfpathlineto{\pgfpoint{24.763162\du}{9.438426\du}}
\pgfpathlineto{\pgfpoint{24.806608\du}{9.415060\du}}
\pgfpathlineto{\pgfpoint{24.853706\du}{9.392059\du}}
\pgfpathlineto{\pgfpoint{24.903724\du}{9.370518\du}}
\pgfpathlineto{\pgfpoint{24.956663\du}{9.349343\du}}
\pgfpathlineto{\pgfpoint{25.011793\du}{9.329628\du}}
\pgfpathlineto{\pgfpoint{25.070208\du}{9.310277\du}}
\pgfpathlineto{\pgfpoint{25.131179\du}{9.292023\du}}
\pgfpathlineto{\pgfpoint{25.194706\du}{9.275228\du}}
\pgfpathlineto{\pgfpoint{25.260424\du}{9.258799\du}}
\pgfpathlineto{\pgfpoint{25.329062\du}{9.243465\du}}
\pgfpathlineto{\pgfpoint{25.399525\du}{9.229956\du}}
\pgfpathlineto{\pgfpoint{25.472180\du}{9.217178\du}}
\pgfpathlineto{\pgfpoint{25.546294\du}{9.205495\du}}
\pgfpathlineto{\pgfpoint{25.622965\du}{9.195637\du}}
\pgfpathlineto{\pgfpoint{25.700730\du}{9.186145\du}}
\pgfpathlineto{\pgfpoint{25.780321\du}{9.178478\du}}
\pgfpathlineto{\pgfpoint{25.861738\du}{9.171906\du}}
\pgfpathlineto{\pgfpoint{25.944615\du}{9.166794\du}}
\pgfpathlineto{\pgfpoint{26.028222\du}{9.163509\du}}
\pgfpathlineto{\pgfpoint{26.114020\du}{9.160953\du}}
\pgfpathlineto{\pgfpoint{26.200183\du}{9.160588\du}}
\pgfpathlineto{\pgfpoint{26.286710\du}{9.160953\du}}
\pgfpathlineto{\pgfpoint{26.372143\du}{9.163509\du}}
\pgfpathlineto{\pgfpoint{26.455750\du}{9.166794\du}}
\pgfpathlineto{\pgfpoint{26.538992\du}{9.171906\du}}
\pgfpathlineto{\pgfpoint{26.620044\du}{9.178478\du}}
\pgfpathlineto{\pgfpoint{26.699635\du}{9.186145\du}}
\pgfpathlineto{\pgfpoint{26.778131\du}{9.195637\du}}
\pgfpathlineto{\pgfpoint{26.854071\du}{9.205495\du}}
\pgfpathlineto{\pgfpoint{26.928916\du}{9.217178\du}}
\pgfpathlineto{\pgfpoint{27.000840\du}{9.229956\du}}
\pgfpathlineto{\pgfpoint{27.071668\du}{9.243465\du}}
\pgfpathlineto{\pgfpoint{27.140307\du}{9.258799\du}}
\pgfpathlineto{\pgfpoint{27.205294\du}{9.275228\du}}
\pgfpathlineto{\pgfpoint{27.269186\du}{9.292023\du}}
\pgfpathlineto{\pgfpoint{27.329792\du}{9.310277\du}}
\pgfpathlineto{\pgfpoint{27.387842\du}{9.329628\du}}
\pgfpathlineto{\pgfpoint{27.443337\du}{9.349343\du}}
\pgfpathlineto{\pgfpoint{27.496276\du}{9.370518\du}}
\pgfpathlineto{\pgfpoint{27.546659\du}{9.392059\du}}
\pgfpathlineto{\pgfpoint{27.593027\du}{9.415060\du}}
\pgfpathlineto{\pgfpoint{27.636838\du}{9.438426\du}}
\pgfpathlineto{\pgfpoint{27.677729\du}{9.462523\du}}
\pgfpathlineto{\pgfpoint{27.714969\du}{9.487714\du}}
\pgfpathlineto{\pgfpoint{27.747828\du}{9.512906\du}}
\pgfpathlineto{\pgfpoint{27.778496\du}{9.539193\du}}
\pgfpathlineto{\pgfpoint{27.805148\du}{9.565845\du}}
\pgfpathlineto{\pgfpoint{27.827054\du}{9.593593\du}}
\pgfpathlineto{\pgfpoint{27.846404\du}{9.621705\du}}
\pgfpathlineto{\pgfpoint{27.861008\du}{9.649817\du}}
\pgfpathlineto{\pgfpoint{27.871230\du}{9.678295\du}}
\pgfpathlineto{\pgfpoint{27.878532\du}{9.708233\du}}
\pgfpathlineto{\pgfpoint{27.879993\du}{9.737441\du}}
\pgfusepath{fill}
\pgfsetbuttcap
\pgfsetmiterjoin
\pgfsetdash{}{0pt}
\definecolor{dialinecolor}{rgb}{0.678431, 0.839216, 0.905882}
\pgfsetfillcolor{dialinecolor}
\pgfpathmoveto{\pgfpoint{26.200183\du}{10.324881\du}}
\pgfpathlineto{\pgfpoint{26.200183\du}{10.324881\du}}
\pgfpathlineto{\pgfpoint{26.243629\du}{10.324881\du}}
\pgfpathlineto{\pgfpoint{26.287076\du}{10.324151\du}}
\pgfpathlineto{\pgfpoint{26.330157\du}{10.323056\du}}
\pgfpathlineto{\pgfpoint{26.372143\du}{10.321961\du}}
\pgfpathlineto{\pgfpoint{26.414859\du}{10.320135\du}}
\pgfpathlineto{\pgfpoint{26.456480\du}{10.318310\du}}
\pgfpathlineto{\pgfpoint{26.498101\du}{10.316119\du}}
\pgfpathlineto{\pgfpoint{26.539723\du}{10.313198\du}}
\pgfpathlineto{\pgfpoint{26.580248\du}{10.310277\du}}
\pgfpathlineto{\pgfpoint{26.621139\du}{10.307357\du}}
\pgfpathlineto{\pgfpoint{26.660935\du}{10.303341\du}}
\pgfpathlineto{\pgfpoint{26.701095\du}{10.299325\du}}
\pgfpathlineto{\pgfpoint{26.739796\du}{10.294578\du}}
\pgfpathlineto{\pgfpoint{26.779226\du}{10.289832\du}}
\pgfpathlineto{\pgfpoint{26.817196\du}{10.285086\du}}
\pgfpathlineto{\pgfpoint{26.856261\du}{10.279974\du}}
\pgfpathlineto{\pgfpoint{26.893501\du}{10.274133\du}}
\pgfpathlineto{\pgfpoint{26.930376\du}{10.268291\du}}
\pgfpathlineto{\pgfpoint{26.966886\du}{10.261720\du}}
\pgfpathlineto{\pgfpoint{27.003395\du}{10.255148\du}}
\pgfpathlineto{\pgfpoint{27.038810\du}{10.248211\du}}
\pgfpathlineto{\pgfpoint{27.073494\du}{10.241274\du}}
\pgfpathlineto{\pgfpoint{27.108178\du}{10.234337\du}}
\pgfpathlineto{\pgfpoint{27.142132\du}{10.226670\du}}
\pgfpathlineto{\pgfpoint{27.175721\du}{10.218273\du}}
\pgfpathlineto{\pgfpoint{27.208580\du}{10.210241\du}}
\pgfpathlineto{\pgfpoint{27.240343\du}{10.201844\du}}
\pgfpathlineto{\pgfpoint{27.271742\du}{10.192716\du}}
\pgfpathlineto{\pgfpoint{27.287076\du}{10.188700\du}}
\pgfpathlineto{\pgfpoint{27.302410\du}{10.183954\du}}
\pgfpathlineto{\pgfpoint{27.318474\du}{10.179208\du}}
\pgfpathlineto{\pgfpoint{27.333078\du}{10.174461\du}}
\pgfpathlineto{\pgfpoint{27.347317\du}{10.169715\du}}
\pgfpathlineto{\pgfpoint{27.362286\du}{10.165334\du}}
\pgfpathlineto{\pgfpoint{27.377254\du}{10.160588\du}}
\pgfpathlineto{\pgfpoint{27.391128\du}{10.155111\du}}
\pgfpathlineto{\pgfpoint{27.405367\du}{10.150365\du}}
\pgfpathlineto{\pgfpoint{27.419241\du}{10.145254\du}}
\pgfpathlineto{\pgfpoint{27.433479\du}{10.140507\du}}
\pgfpathlineto{\pgfpoint{27.446623\du}{10.135396\du}}
\pgfpathlineto{\pgfpoint{27.460862\du}{10.129920\du}}
\pgfpathlineto{\pgfpoint{27.474005\du}{10.124808\du}}
\pgfpathlineto{\pgfpoint{27.487149\du}{10.119697\du}}
\pgfpathlineto{\pgfpoint{27.500657\du}{10.114221\du}}
\pgfpathlineto{\pgfpoint{27.513436\du}{10.109109\du}}
\pgfpathlineto{\pgfpoint{27.525484\du}{10.103633\du}}
\pgfpathlineto{\pgfpoint{27.538262\du}{10.097791\du}}
\pgfpathlineto{\pgfpoint{27.550675\du}{10.091950\du}}
\pgfpathlineto{\pgfpoint{27.563089\du}{10.086838\du}}
\pgfpathlineto{\pgfpoint{27.574407\du}{10.080997\du}}
\pgfpathlineto{\pgfpoint{27.586455\du}{10.075155\du}}
\pgfpathlineto{\pgfpoint{27.597408\du}{10.069314\du}}
\pgfpathlineto{\pgfpoint{27.609091\du}{10.063837\du}}
\pgfpathlineto{\pgfpoint{27.620409\du}{10.057996\du}}
\pgfpathlineto{\pgfpoint{27.631727\du}{10.051789\du}}
\pgfpathlineto{\pgfpoint{27.642315\du}{10.045947\du}}
\pgfpathlineto{\pgfpoint{27.652172\du}{10.040106\du}}
\pgfpathlineto{\pgfpoint{27.662760\du}{10.034264\du}}
\pgfpathlineto{\pgfpoint{27.672983\du}{10.027693\du}}
\pgfpathlineto{\pgfpoint{27.682840\du}{10.021121\du}}
\pgfpathlineto{\pgfpoint{27.692698\du}{10.015279\du}}
\pgfpathlineto{\pgfpoint{27.701825\du}{10.009073\du}}
\pgfpathlineto{\pgfpoint{27.710953\du}{10.002501\du}}
\pgfpathlineto{\pgfpoint{27.720445\du}{9.995929\du}}
\pgfpathlineto{\pgfpoint{27.729208\du}{9.989723\du}}
\pgfpathlineto{\pgfpoint{27.738335\du}{9.983151\du}}
\pgfpathlineto{\pgfpoint{27.746367\du}{9.976579\du}}
\pgfpathlineto{\pgfpoint{27.755130\du}{9.970372\du}}
\pgfpathlineto{\pgfpoint{27.762432\du}{9.963801\du}}
\pgfpathlineto{\pgfpoint{27.770464\du}{9.956864\du}}
\pgfpathlineto{\pgfpoint{27.778496\du}{9.950292\du}}
\pgfpathlineto{\pgfpoint{27.785068\du}{9.943355\du}}
\pgfpathlineto{\pgfpoint{27.792369\du}{9.936783\du}}
\pgfpathlineto{\pgfpoint{27.798941\du}{9.929847\du}}
\pgfpathlineto{\pgfpoint{27.805878\du}{9.922910\du}}
\pgfpathlineto{\pgfpoint{27.812450\du}{9.916338\du}}
\pgfpathlineto{\pgfpoint{27.818656\du}{9.909401\du}}
\pgfpathlineto{\pgfpoint{27.824498\du}{9.902464\du}}
\pgfpathlineto{\pgfpoint{27.829974\du}{9.895528\du}}
\pgfpathlineto{\pgfpoint{27.835816\du}{9.887861\du}}
\pgfpathlineto{\pgfpoint{27.840562\du}{9.880924\du}}
\pgfpathlineto{\pgfpoint{27.846039\du}{9.873622\du}}
\pgfpathlineto{\pgfpoint{27.850785\du}{9.866685\du}}
\pgfpathlineto{\pgfpoint{27.855166\du}{9.859018\du}}
\pgfpathlineto{\pgfpoint{27.859182\du}{9.852081\du}}
\pgfpathlineto{\pgfpoint{27.863198\du}{9.844414\du}}
\pgfpathlineto{\pgfpoint{27.866484\du}{9.836747\du}}
\pgfpathlineto{\pgfpoint{27.870500\du}{9.829445\du}}
\pgfpathlineto{\pgfpoint{27.873786\du}{9.822143\du}}
\pgfpathlineto{\pgfpoint{27.876342\du}{9.814841\du}}
\pgfpathlineto{\pgfpoint{27.879263\du}{9.807174\du}}
\pgfpathlineto{\pgfpoint{27.881088\du}{9.799507\du}}
\pgfpathlineto{\pgfpoint{27.884009\du}{9.791840\du}}
\pgfpathlineto{\pgfpoint{27.885104\du}{9.784173\du}}
\pgfpathlineto{\pgfpoint{27.887295\du}{9.776506\du}}
\pgfpathlineto{\pgfpoint{27.888390\du}{9.769204\du}}
\pgfpathlineto{\pgfpoint{27.889120\du}{9.760807\du}}
\pgfpathlineto{\pgfpoint{27.889850\du}{9.753140\du}}
\pgfpathlineto{\pgfpoint{27.890581\du}{9.745473\du}}
\pgfpathlineto{\pgfpoint{27.890581\du}{9.737441\du}}
\pgfpathlineto{\pgfpoint{27.870500\du}{9.737441\du}}
\pgfpathlineto{\pgfpoint{27.869770\du}{9.744378\du}}
\pgfpathlineto{\pgfpoint{27.869770\du}{9.752045\du}}
\pgfpathlineto{\pgfpoint{27.869405\du}{9.758981\du}}
\pgfpathlineto{\pgfpoint{27.867579\du}{9.765918\du}}
\pgfpathlineto{\pgfpoint{27.866484\du}{9.773220\du}}
\pgfpathlineto{\pgfpoint{27.865754\du}{9.779427\du}}
\pgfpathlineto{\pgfpoint{27.863563\du}{9.786729\du}}
\pgfpathlineto{\pgfpoint{27.862103\du}{9.793666\du}}
\pgfpathlineto{\pgfpoint{27.859912\du}{9.800602\du}}
\pgfpathlineto{\pgfpoint{27.857357\du}{9.807539\du}}
\pgfpathlineto{\pgfpoint{27.854436\du}{9.814841\du}}
\pgfpathlineto{\pgfpoint{27.851880\du}{9.821048\du}}
\pgfpathlineto{\pgfpoint{27.847864\du}{9.827985\du}}
\pgfpathlineto{\pgfpoint{27.844943\du}{9.835287\du}}
\pgfpathlineto{\pgfpoint{27.840927\du}{9.842223\du}}
\pgfpathlineto{\pgfpoint{27.838007\du}{9.848430\du}}
\pgfpathlineto{\pgfpoint{27.833625\du}{9.855732\du}}
\pgfpathlineto{\pgfpoint{27.829244\du}{9.861939\du}}
\pgfpathlineto{\pgfpoint{27.824498\du}{9.869241\du}}
\pgfpathlineto{\pgfpoint{27.819387\du}{9.875447\du}}
\pgfpathlineto{\pgfpoint{27.814640\du}{9.882384\du}}
\pgfpathlineto{\pgfpoint{27.808434\du}{9.888956\du}}
\pgfpathlineto{\pgfpoint{27.802957\du}{9.895528\du}}
\pgfpathlineto{\pgfpoint{27.797846\du}{9.902464\du}}
\pgfpathlineto{\pgfpoint{27.791274\du}{9.909036\du}}
\pgfpathlineto{\pgfpoint{27.784337\du}{9.915243\du}}
\pgfpathlineto{\pgfpoint{27.778496\du}{9.921815\du}}
\pgfpathlineto{\pgfpoint{27.771194\du}{9.928751\du}}
\pgfpathlineto{\pgfpoint{27.764622\du}{9.935323\du}}
\pgfpathlineto{\pgfpoint{27.756955\du}{9.941530\du}}
\pgfpathlineto{\pgfpoint{27.749653\du}{9.948101\du}}
\pgfpathlineto{\pgfpoint{27.741621\du}{9.953943\du}}
\pgfpathlineto{\pgfpoint{27.733954\du}{9.960515\du}}
\pgfpathlineto{\pgfpoint{27.725192\du}{9.966721\du}}
\pgfpathlineto{\pgfpoint{27.716794\du}{9.973293\du}}
\pgfpathlineto{\pgfpoint{27.708032\du}{9.979500\du}}
\pgfpathlineto{\pgfpoint{27.700000\du}{9.985341\du}}
\pgfpathlineto{\pgfpoint{27.691238\du}{9.991913\du}}
\pgfpathlineto{\pgfpoint{27.681745\du}{9.997755\du}}
\pgfpathlineto{\pgfpoint{27.671522\du}{10.004326\du}}
\pgfpathlineto{\pgfpoint{27.662395\du}{10.010168\du}}
\pgfpathlineto{\pgfpoint{27.652172\du}{10.016009\du}}
\pgfpathlineto{\pgfpoint{27.642315\du}{10.021851\du}}
\pgfpathlineto{\pgfpoint{27.632457\du}{10.027693\du}}
\pgfpathlineto{\pgfpoint{27.621504\du}{10.034264\du}}
\pgfpathlineto{\pgfpoint{27.610551\du}{10.039376\du}}
\pgfpathlineto{\pgfpoint{27.600329\du}{10.045217\du}}
\pgfpathlineto{\pgfpoint{27.588280\du}{10.051059\du}}
\pgfpathlineto{\pgfpoint{27.577693\du}{10.056900\du}}
\pgfpathlineto{\pgfpoint{27.565644\du}{10.062742\du}}
\pgfpathlineto{\pgfpoint{27.554326\du}{10.067853\du}}
\pgfpathlineto{\pgfpoint{27.541913\du}{10.073330\du}}
\pgfpathlineto{\pgfpoint{27.529865\du}{10.079171\du}}
\pgfpathlineto{\pgfpoint{27.517452\du}{10.084283\du}}
\pgfpathlineto{\pgfpoint{27.505403\du}{10.089759\du}}
\pgfpathlineto{\pgfpoint{27.492260\du}{10.094870\du}}
\pgfpathlineto{\pgfpoint{27.479482\du}{10.100347\du}}
\pgfpathlineto{\pgfpoint{27.466703\du}{10.106188\du}}
\pgfpathlineto{\pgfpoint{27.453560\du}{10.110570\du}}
\pgfpathlineto{\pgfpoint{27.440051\du}{10.116046\du}}
\pgfpathlineto{\pgfpoint{27.426908\du}{10.121157\du}}
\pgfpathlineto{\pgfpoint{27.412669\du}{10.125904\du}}
\pgfpathlineto{\pgfpoint{27.398795\du}{10.131380\du}}
\pgfpathlineto{\pgfpoint{27.384556\du}{10.135761\du}}
\pgfpathlineto{\pgfpoint{27.369953\du}{10.140507\du}}
\pgfpathlineto{\pgfpoint{27.356079\du}{10.145984\du}}
\pgfpathlineto{\pgfpoint{27.341475\du}{10.150365\du}}
\pgfpathlineto{\pgfpoint{27.327236\du}{10.155111\du}}
\pgfpathlineto{\pgfpoint{27.311537\du}{10.159858\du}}
\pgfpathlineto{\pgfpoint{27.297298\du}{10.163874\du}}
\pgfpathlineto{\pgfpoint{27.281599\du}{10.168620\du}}
\pgfpathlineto{\pgfpoint{27.266630\du}{10.173366\du}}
\pgfpathlineto{\pgfpoint{27.235232\du}{10.181398\du}}
\pgfpathlineto{\pgfpoint{27.203103\du}{10.190161\du}}
\pgfpathlineto{\pgfpoint{27.170975\du}{10.198558\du}}
\pgfpathlineto{\pgfpoint{27.137751\du}{10.206225\du}}
\pgfpathlineto{\pgfpoint{27.103432\du}{10.213892\du}}
\pgfpathlineto{\pgfpoint{27.069478\du}{10.221194\du}}
\pgfpathlineto{\pgfpoint{27.034794\du}{10.228496\du}}
\pgfpathlineto{\pgfpoint{26.999014\du}{10.235433\du}}
\pgfpathlineto{\pgfpoint{26.963600\du}{10.241639\du}}
\pgfpathlineto{\pgfpoint{26.926725\du}{10.247481\du}}
\pgfpathlineto{\pgfpoint{26.890215\du}{10.253687\du}}
\pgfpathlineto{\pgfpoint{26.852976\du}{10.259529\du}}
\pgfpathlineto{\pgfpoint{26.815371\du}{10.264640\du}}
\pgfpathlineto{\pgfpoint{26.776670\du}{10.269752\du}}
\pgfpathlineto{\pgfpoint{26.737970\du}{10.274498\du}}
\pgfpathlineto{\pgfpoint{26.698540\du}{10.278514\du}}
\pgfpathlineto{\pgfpoint{26.659474\du}{10.282895\du}}
\pgfpathlineto{\pgfpoint{26.619314\du}{10.286181\du}}
\pgfpathlineto{\pgfpoint{26.579153\du}{10.289832\du}}
\pgfpathlineto{\pgfpoint{26.537897\du}{10.292753\du}}
\pgfpathlineto{\pgfpoint{26.497006\du}{10.295674\du}}
\pgfpathlineto{\pgfpoint{26.455750\du}{10.297864\du}}
\pgfpathlineto{\pgfpoint{26.414129\du}{10.300420\du}}
\pgfpathlineto{\pgfpoint{26.371413\du}{10.301515\du}}
\pgfpathlineto{\pgfpoint{26.328697\du}{10.303341\du}}
\pgfpathlineto{\pgfpoint{26.286710\du}{10.303706\du}}
\pgfpathlineto{\pgfpoint{26.243264\du}{10.304436\du}}
\pgfpathlineto{\pgfpoint{26.200183\du}{10.304436\du}}
\pgfpathlineto{\pgfpoint{26.200183\du}{10.304436\du}}
\pgfpathlineto{\pgfpoint{26.200183\du}{10.304436\du}}
\pgfpathlineto{\pgfpoint{26.199452\du}{10.305166\du}}
\pgfpathlineto{\pgfpoint{26.197627\du}{10.305166\du}}
\pgfpathlineto{\pgfpoint{26.196532\du}{10.305166\du}}
\pgfpathlineto{\pgfpoint{26.195801\du}{10.305531\du}}
\pgfpathlineto{\pgfpoint{26.195436\du}{10.306261\du}}
\pgfpathlineto{\pgfpoint{26.193976\du}{10.306261\du}}
\pgfpathlineto{\pgfpoint{26.193246\du}{10.307357\du}}
\pgfpathlineto{\pgfpoint{26.192516\du}{10.308087\du}}
\pgfpathlineto{\pgfpoint{26.191420\du}{10.309182\du}}
\pgfpathlineto{\pgfpoint{26.190690\du}{10.311008\du}}
\pgfpathlineto{\pgfpoint{26.190690\du}{10.313198\du}}
\pgfpathlineto{\pgfpoint{26.189960\du}{10.315024\du}}
\pgfpathlineto{\pgfpoint{26.190690\du}{10.316849\du}}
\pgfpathlineto{\pgfpoint{26.190690\du}{10.318310\du}}
\pgfpathlineto{\pgfpoint{26.191420\du}{10.320135\du}}
\pgfpathlineto{\pgfpoint{26.192516\du}{10.321961\du}}
\pgfpathlineto{\pgfpoint{26.193246\du}{10.322691\du}}
\pgfpathlineto{\pgfpoint{26.193976\du}{10.323056\du}}
\pgfpathlineto{\pgfpoint{26.195436\du}{10.323786\du}}
\pgfpathlineto{\pgfpoint{26.195801\du}{10.324151\du}}
\pgfpathlineto{\pgfpoint{26.196532\du}{10.324881\du}}
\pgfpathlineto{\pgfpoint{26.197627\du}{10.324881\du}}
\pgfpathlineto{\pgfpoint{26.199452\du}{10.324881\du}}
\pgfpathlineto{\pgfpoint{26.200183\du}{10.324881\du}}
\pgfusepath{fill}
\pgfsetbuttcap
\pgfsetmiterjoin
\pgfsetdash{}{0pt}
\definecolor{dialinecolor}{rgb}{0.678431, 0.839216, 0.905882}
\pgfsetfillcolor{dialinecolor}
\pgfpathmoveto{\pgfpoint{24.509419\du}{9.737441\du}}
\pgfpathlineto{\pgfpoint{24.509419\du}{9.737441\du}}
\pgfpathlineto{\pgfpoint{24.509419\du}{9.745473\du}}
\pgfpathlineto{\pgfpoint{24.509785\du}{9.753140\du}}
\pgfpathlineto{\pgfpoint{24.510515\du}{9.760807\du}}
\pgfpathlineto{\pgfpoint{24.511610\du}{9.769204\du}}
\pgfpathlineto{\pgfpoint{24.512705\du}{9.776506\du}}
\pgfpathlineto{\pgfpoint{24.514531\du}{9.784173\du}}
\pgfpathlineto{\pgfpoint{24.516356\du}{9.791840\du}}
\pgfpathlineto{\pgfpoint{24.518547\du}{9.799507\du}}
\pgfpathlineto{\pgfpoint{24.520737\du}{9.807174\du}}
\pgfpathlineto{\pgfpoint{24.523293\du}{9.814841\du}}
\pgfpathlineto{\pgfpoint{24.526214\du}{9.822143\du}}
\pgfpathlineto{\pgfpoint{24.529865\du}{9.829445\du}}
\pgfpathlineto{\pgfpoint{24.533151\du}{9.836747\du}}
\pgfpathlineto{\pgfpoint{24.536802\du}{9.844414\du}}
\pgfpathlineto{\pgfpoint{24.541183\du}{9.852081\du}}
\pgfpathlineto{\pgfpoint{24.544834\du}{9.859018\du}}
\pgfpathlineto{\pgfpoint{24.549945\du}{9.866685\du}}
\pgfpathlineto{\pgfpoint{24.553961\du}{9.873622\du}}
\pgfpathlineto{\pgfpoint{24.559438\du}{9.880924\du}}
\pgfpathlineto{\pgfpoint{24.564184\du}{9.887861\du}}
\pgfpathlineto{\pgfpoint{24.569660\du}{9.895528\du}}
\pgfpathlineto{\pgfpoint{24.575502\du}{9.902464\du}}
\pgfpathlineto{\pgfpoint{24.581344\du}{9.909401\du}}
\pgfpathlineto{\pgfpoint{24.587185\du}{9.916338\du}}
\pgfpathlineto{\pgfpoint{24.594122\du}{9.922910\du}}
\pgfpathlineto{\pgfpoint{24.600694\du}{9.929847\du}}
\pgfpathlineto{\pgfpoint{24.607631\du}{9.936783\du}}
\pgfpathlineto{\pgfpoint{24.614567\du}{9.943355\du}}
\pgfpathlineto{\pgfpoint{24.621504\du}{9.950292\du}}
\pgfpathlineto{\pgfpoint{24.630267\du}{9.956864\du}}
\pgfpathlineto{\pgfpoint{24.637203\du}{9.963801\du}}
\pgfpathlineto{\pgfpoint{24.644870\du}{9.970372\du}}
\pgfpathlineto{\pgfpoint{24.653633\du}{9.976579\du}}
\pgfpathlineto{\pgfpoint{24.662030\du}{9.983151\du}}
\pgfpathlineto{\pgfpoint{24.670792\du}{9.989723\du}}
\pgfpathlineto{\pgfpoint{24.679189\du}{9.995929\du}}
\pgfpathlineto{\pgfpoint{24.689047\du}{10.002501\du}}
\pgfpathlineto{\pgfpoint{24.698175\du}{10.009073\du}}
\pgfpathlineto{\pgfpoint{24.707667\du}{10.015279\du}}
\pgfpathlineto{\pgfpoint{24.717160\du}{10.021121\du}}
\pgfpathlineto{\pgfpoint{24.727017\du}{10.027693\du}}
\pgfpathlineto{\pgfpoint{24.737240\du}{10.034264\du}}
\pgfpathlineto{\pgfpoint{24.747463\du}{10.040106\du}}
\pgfpathlineto{\pgfpoint{24.758050\du}{10.045947\du}}
\pgfpathlineto{\pgfpoint{24.768273\du}{10.051789\du}}
\pgfpathlineto{\pgfpoint{24.779226\du}{10.057996\du}}
\pgfpathlineto{\pgfpoint{24.790909\du}{10.063837\du}}
\pgfpathlineto{\pgfpoint{24.802227\du}{10.069314\du}}
\pgfpathlineto{\pgfpoint{24.813910\du}{10.075155\du}}
\pgfpathlineto{\pgfpoint{24.825228\du}{10.080997\du}}
\pgfpathlineto{\pgfpoint{24.836911\du}{10.086838\du}}
\pgfpathlineto{\pgfpoint{24.849325\du}{10.091950\du}}
\pgfpathlineto{\pgfpoint{24.861373\du}{10.097791\du}}
\pgfpathlineto{\pgfpoint{24.874516\du}{10.103633\du}}
\pgfpathlineto{\pgfpoint{24.886564\du}{10.109109\du}}
\pgfpathlineto{\pgfpoint{24.899343\du}{10.114221\du}}
\pgfpathlineto{\pgfpoint{24.913217\du}{10.119697\du}}
\pgfpathlineto{\pgfpoint{24.925630\du}{10.124808\du}}
\pgfpathlineto{\pgfpoint{24.938773\du}{10.129920\du}}
\pgfpathlineto{\pgfpoint{24.953012\du}{10.135396\du}}
\pgfpathlineto{\pgfpoint{24.966156\du}{10.140507\du}}
\pgfpathlineto{\pgfpoint{24.980394\du}{10.145254\du}}
\pgfpathlineto{\pgfpoint{24.994268\du}{10.150365\du}}
\pgfpathlineto{\pgfpoint{25.008872\du}{10.155111\du}}
\pgfpathlineto{\pgfpoint{25.022746\du}{10.160588\du}}
\pgfpathlineto{\pgfpoint{25.038445\du}{10.165334\du}}
\pgfpathlineto{\pgfpoint{25.052318\du}{10.169715\du}}
\pgfpathlineto{\pgfpoint{25.066922\du}{10.174461\du}}
\pgfpathlineto{\pgfpoint{25.082256\du}{10.179208\du}}
\pgfpathlineto{\pgfpoint{25.097955\du}{10.183954\du}}
\pgfpathlineto{\pgfpoint{25.112924\du}{10.188700\du}}
\pgfpathlineto{\pgfpoint{25.128624\du}{10.192716\du}}
\pgfpathlineto{\pgfpoint{25.160387\du}{10.201844\du}}
\pgfpathlineto{\pgfpoint{25.192150\du}{10.210241\du}}
\pgfpathlineto{\pgfpoint{25.225374\du}{10.218273\du}}
\pgfpathlineto{\pgfpoint{25.257868\du}{10.226670\du}}
\pgfpathlineto{\pgfpoint{25.292552\du}{10.234337\du}}
\pgfpathlineto{\pgfpoint{25.326871\du}{10.241274\du}}
\pgfpathlineto{\pgfpoint{25.361555\du}{10.248211\du}}
\pgfpathlineto{\pgfpoint{25.397335\du}{10.255148\du}}
\pgfpathlineto{\pgfpoint{25.433479\du}{10.261720\du}}
\pgfpathlineto{\pgfpoint{25.469989\du}{10.268291\du}}
\pgfpathlineto{\pgfpoint{25.507229\du}{10.274133\du}}
\pgfpathlineto{\pgfpoint{25.544834\du}{10.279974\du}}
\pgfpathlineto{\pgfpoint{25.582804\du}{10.285086\du}}
\pgfpathlineto{\pgfpoint{25.621504\du}{10.289832\du}}
\pgfpathlineto{\pgfpoint{25.660204\du}{10.294578\du}}
\pgfpathlineto{\pgfpoint{25.699270\du}{10.299325\du}}
\pgfpathlineto{\pgfpoint{25.739430\du}{10.303341\du}}
\pgfpathlineto{\pgfpoint{25.779226\du}{10.307357\du}}
\pgfpathlineto{\pgfpoint{25.820117\du}{10.310277\du}}
\pgfpathlineto{\pgfpoint{25.861008\du}{10.313198\du}}
\pgfpathlineto{\pgfpoint{25.902264\du}{10.316119\du}}
\pgfpathlineto{\pgfpoint{25.944250\du}{10.318310\du}}
\pgfpathlineto{\pgfpoint{25.985871\du}{10.320135\du}}
\pgfpathlineto{\pgfpoint{26.028222\du}{10.321961\du}}
\pgfpathlineto{\pgfpoint{26.070208\du}{10.323056\du}}
\pgfpathlineto{\pgfpoint{26.113655\du}{10.324151\du}}
\pgfpathlineto{\pgfpoint{26.156371\du}{10.324881\du}}
\pgfpathlineto{\pgfpoint{26.200183\du}{10.324881\du}}
\pgfpathlineto{\pgfpoint{26.200183\du}{10.304436\du}}
\pgfpathlineto{\pgfpoint{26.157466\du}{10.304436\du}}
\pgfpathlineto{\pgfpoint{26.114020\du}{10.303706\du}}
\pgfpathlineto{\pgfpoint{26.071668\du}{10.303341\du}}
\pgfpathlineto{\pgfpoint{26.028952\du}{10.301515\du}}
\pgfpathlineto{\pgfpoint{25.986601\du}{10.300420\du}}
\pgfpathlineto{\pgfpoint{25.944615\du}{10.297864\du}}
\pgfpathlineto{\pgfpoint{25.903724\du}{10.295674\du}}
\pgfpathlineto{\pgfpoint{25.862468\du}{10.292753\du}}
\pgfpathlineto{\pgfpoint{25.821577\du}{10.289832\du}}
\pgfpathlineto{\pgfpoint{25.781417\du}{10.286181\du}}
\pgfpathlineto{\pgfpoint{25.741621\du}{10.282895\du}}
\pgfpathlineto{\pgfpoint{25.702191\du}{10.278514\du}}
\pgfpathlineto{\pgfpoint{25.662760\du}{10.274498\du}}
\pgfpathlineto{\pgfpoint{25.623695\du}{10.269752\du}}
\pgfpathlineto{\pgfpoint{25.585360\du}{10.264640\du}}
\pgfpathlineto{\pgfpoint{25.547755\du}{10.259529\du}}
\pgfpathlineto{\pgfpoint{25.510515\du}{10.253687\du}}
\pgfpathlineto{\pgfpoint{25.474005\du}{10.247481\du}}
\pgfpathlineto{\pgfpoint{25.436765\du}{10.241639\du}}
\pgfpathlineto{\pgfpoint{25.401716\du}{10.235433\du}}
\pgfpathlineto{\pgfpoint{25.365571\du}{10.228496\du}}
\pgfpathlineto{\pgfpoint{25.330887\du}{10.221194\du}}
\pgfpathlineto{\pgfpoint{25.296568\du}{10.213892\du}}
\pgfpathlineto{\pgfpoint{25.262614\du}{10.206225\du}}
\pgfpathlineto{\pgfpoint{25.229390\du}{10.198558\du}}
\pgfpathlineto{\pgfpoint{25.197992\du}{10.190161\du}}
\pgfpathlineto{\pgfpoint{25.165498\du}{10.181398\du}}
\pgfpathlineto{\pgfpoint{25.134465\du}{10.173366\du}}
\pgfpathlineto{\pgfpoint{25.118766\du}{10.168620\du}}
\pgfpathlineto{\pgfpoint{25.103067\du}{10.163874\du}}
\pgfpathlineto{\pgfpoint{25.088463\du}{10.159858\du}}
\pgfpathlineto{\pgfpoint{25.073494\du}{10.155111\du}}
\pgfpathlineto{\pgfpoint{25.058160\du}{10.150365\du}}
\pgfpathlineto{\pgfpoint{25.043921\du}{10.145984\du}}
\pgfpathlineto{\pgfpoint{25.029682\du}{10.140507\du}}
\pgfpathlineto{\pgfpoint{25.015444\du}{10.135761\du}}
\pgfpathlineto{\pgfpoint{25.001205\du}{10.131380\du}}
\pgfpathlineto{\pgfpoint{24.987331\du}{10.125904\du}}
\pgfpathlineto{\pgfpoint{24.973457\du}{10.121157\du}}
\pgfpathlineto{\pgfpoint{24.959949\du}{10.116046\du}}
\pgfpathlineto{\pgfpoint{24.946440\du}{10.110570\du}}
\pgfpathlineto{\pgfpoint{24.933662\du}{10.106188\du}}
\pgfpathlineto{\pgfpoint{24.920153\du}{10.100347\du}}
\pgfpathlineto{\pgfpoint{24.907375\du}{10.094870\du}}
\pgfpathlineto{\pgfpoint{24.894962\du}{10.089759\du}}
\pgfpathlineto{\pgfpoint{24.881818\du}{10.084283\du}}
\pgfpathlineto{\pgfpoint{24.869770\du}{10.079171\du}}
\pgfpathlineto{\pgfpoint{24.858452\du}{10.073330\du}}
\pgfpathlineto{\pgfpoint{24.845674\du}{10.067853\du}}
\pgfpathlineto{\pgfpoint{24.833991\du}{10.062742\du}}
\pgfpathlineto{\pgfpoint{24.822307\du}{10.056900\du}}
\pgfpathlineto{\pgfpoint{24.811355\du}{10.051059\du}}
\pgfpathlineto{\pgfpoint{24.799671\du}{10.045217\du}}
\pgfpathlineto{\pgfpoint{24.789814\du}{10.039376\du}}
\pgfpathlineto{\pgfpoint{24.778496\du}{10.034264\du}}
\pgfpathlineto{\pgfpoint{24.767908\du}{10.027693\du}}
\pgfpathlineto{\pgfpoint{24.758050\du}{10.021851\du}}
\pgfpathlineto{\pgfpoint{24.747463\du}{10.016009\du}}
\pgfpathlineto{\pgfpoint{24.737605\du}{10.010168\du}}
\pgfpathlineto{\pgfpoint{24.728112\du}{10.004326\du}}
\pgfpathlineto{\pgfpoint{24.718255\du}{9.997755\du}}
\pgfpathlineto{\pgfpoint{24.709127\du}{9.991913\du}}
\pgfpathlineto{\pgfpoint{24.700730\du}{9.985341\du}}
\pgfpathlineto{\pgfpoint{24.691968\du}{9.979500\du}}
\pgfpathlineto{\pgfpoint{24.682840\du}{9.973293\du}}
\pgfpathlineto{\pgfpoint{24.674078\du}{9.966721\du}}
\pgfpathlineto{\pgfpoint{24.665681\du}{9.960515\du}}
\pgfpathlineto{\pgfpoint{24.658379\du}{9.953943\du}}
\pgfpathlineto{\pgfpoint{24.650347\du}{9.948101\du}}
\pgfpathlineto{\pgfpoint{24.642680\du}{9.941530\du}}
\pgfpathlineto{\pgfpoint{24.635743\du}{9.935323\du}}
\pgfpathlineto{\pgfpoint{24.628441\du}{9.928751\du}}
\pgfpathlineto{\pgfpoint{24.621504\du}{9.921815\du}}
\pgfpathlineto{\pgfpoint{24.615298\du}{9.915243\du}}
\pgfpathlineto{\pgfpoint{24.608726\du}{9.909036\du}}
\pgfpathlineto{\pgfpoint{24.602884\du}{9.902464\du}}
\pgfpathlineto{\pgfpoint{24.596678\du}{9.895528\du}}
\pgfpathlineto{\pgfpoint{24.591566\du}{9.888956\du}}
\pgfpathlineto{\pgfpoint{24.585360\du}{9.882384\du}}
\pgfpathlineto{\pgfpoint{24.580978\du}{9.875447\du}}
\pgfpathlineto{\pgfpoint{24.575502\du}{9.869241\du}}
\pgfpathlineto{\pgfpoint{24.571121\du}{9.861939\du}}
\pgfpathlineto{\pgfpoint{24.566740\du}{9.855732\du}}
\pgfpathlineto{\pgfpoint{24.562359\du}{9.848430\du}}
\pgfpathlineto{\pgfpoint{24.559073\du}{9.842223\du}}
\pgfpathlineto{\pgfpoint{24.555057\du}{9.835287\du}}
\pgfpathlineto{\pgfpoint{24.551041\du}{9.827985\du}}
\pgfpathlineto{\pgfpoint{24.548120\du}{9.821048\du}}
\pgfpathlineto{\pgfpoint{24.545564\du}{9.814841\du}}
\pgfpathlineto{\pgfpoint{24.542278\du}{9.807539\du}}
\pgfpathlineto{\pgfpoint{24.540088\du}{9.800602\du}}
\pgfpathlineto{\pgfpoint{24.537897\du}{9.793666\du}}
\pgfpathlineto{\pgfpoint{24.536437\du}{9.786729\du}}
\pgfpathlineto{\pgfpoint{24.534246\du}{9.779427\du}}
\pgfpathlineto{\pgfpoint{24.533151\du}{9.773220\du}}
\pgfpathlineto{\pgfpoint{24.532055\du}{9.765918\du}}
\pgfpathlineto{\pgfpoint{24.530595\du}{9.758981\du}}
\pgfpathlineto{\pgfpoint{24.530230\du}{9.752045\du}}
\pgfpathlineto{\pgfpoint{24.530230\du}{9.744378\du}}
\pgfpathlineto{\pgfpoint{24.529865\du}{9.737441\du}}
\pgfpathlineto{\pgfpoint{24.529865\du}{9.737441\du}}
\pgfpathlineto{\pgfpoint{24.529865\du}{9.737441\du}}
\pgfpathlineto{\pgfpoint{24.529865\du}{9.736345\du}}
\pgfpathlineto{\pgfpoint{24.529865\du}{9.735250\du}}
\pgfpathlineto{\pgfpoint{24.529500\du}{9.733790\du}}
\pgfpathlineto{\pgfpoint{24.529500\du}{9.733425\du}}
\pgfpathlineto{\pgfpoint{24.528405\du}{9.732329\du}}
\pgfpathlineto{\pgfpoint{24.528039\du}{9.730869\du}}
\pgfpathlineto{\pgfpoint{24.527674\du}{9.730504\du}}
\pgfpathlineto{\pgfpoint{24.526579\du}{9.729774\du}}
\pgfpathlineto{\pgfpoint{24.525119\du}{9.728678\du}}
\pgfpathlineto{\pgfpoint{24.523293\du}{9.727948\du}}
\pgfpathlineto{\pgfpoint{24.521468\du}{9.727583\du}}
\pgfpathlineto{\pgfpoint{24.519642\du}{9.727583\du}}
\pgfpathlineto{\pgfpoint{24.517452\du}{9.727583\du}}
\pgfpathlineto{\pgfpoint{24.515991\du}{9.727948\du}}
\pgfpathlineto{\pgfpoint{24.513801\du}{9.728678\du}}
\pgfpathlineto{\pgfpoint{24.511975\du}{9.729774\du}}
\pgfpathlineto{\pgfpoint{24.511610\du}{9.730504\du}}
\pgfpathlineto{\pgfpoint{24.511245\du}{9.730869\du}}
\pgfpathlineto{\pgfpoint{24.510515\du}{9.732329\du}}
\pgfpathlineto{\pgfpoint{24.509785\du}{9.733425\du}}
\pgfpathlineto{\pgfpoint{24.509785\du}{9.733790\du}}
\pgfpathlineto{\pgfpoint{24.509419\du}{9.735250\du}}
\pgfpathlineto{\pgfpoint{24.509419\du}{9.736345\du}}
\pgfpathlineto{\pgfpoint{24.509419\du}{9.737441\du}}
\pgfusepath{fill}
\pgfsetbuttcap
\pgfsetmiterjoin
\pgfsetdash{}{0pt}
\definecolor{dialinecolor}{rgb}{0.678431, 0.839216, 0.905882}
\pgfsetfillcolor{dialinecolor}
\pgfpathmoveto{\pgfpoint{26.200183\du}{9.150000\du}}
\pgfpathlineto{\pgfpoint{26.200183\du}{9.150000\du}}
\pgfpathlineto{\pgfpoint{26.156371\du}{9.150000\du}}
\pgfpathlineto{\pgfpoint{26.113655\du}{9.150730\du}}
\pgfpathlineto{\pgfpoint{26.070208\du}{9.151825\du}}
\pgfpathlineto{\pgfpoint{26.028222\du}{9.152921\du}}
\pgfpathlineto{\pgfpoint{25.985871\du}{9.154746\du}}
\pgfpathlineto{\pgfpoint{25.944250\du}{9.156937\du}}
\pgfpathlineto{\pgfpoint{25.902264\du}{9.159493\du}}
\pgfpathlineto{\pgfpoint{25.861008\du}{9.161683\du}}
\pgfpathlineto{\pgfpoint{25.820117\du}{9.164604\du}}
\pgfpathlineto{\pgfpoint{25.779226\du}{9.168255\du}}
\pgfpathlineto{\pgfpoint{25.739430\du}{9.171906\du}}
\pgfpathlineto{\pgfpoint{25.699270\du}{9.175922\du}}
\pgfpathlineto{\pgfpoint{25.660204\du}{9.180303\du}}
\pgfpathlineto{\pgfpoint{25.621504\du}{9.185049\du}}
\pgfpathlineto{\pgfpoint{25.582804\du}{9.190161\du}}
\pgfpathlineto{\pgfpoint{25.544834\du}{9.195637\du}}
\pgfpathlineto{\pgfpoint{25.507229\du}{9.200748\du}}
\pgfpathlineto{\pgfpoint{25.469989\du}{9.207320\du}}
\pgfpathlineto{\pgfpoint{25.433479\du}{9.213162\du}}
\pgfpathlineto{\pgfpoint{25.397335\du}{9.219733\du}}
\pgfpathlineto{\pgfpoint{25.361555\du}{9.226670\du}}
\pgfpathlineto{\pgfpoint{25.326871\du}{9.233607\du}}
\pgfpathlineto{\pgfpoint{25.292552\du}{9.241274\du}}
\pgfpathlineto{\pgfpoint{25.257868\du}{9.248941\du}}
\pgfpathlineto{\pgfpoint{25.225374\du}{9.256973\du}}
\pgfpathlineto{\pgfpoint{25.192150\du}{9.265371\du}}
\pgfpathlineto{\pgfpoint{25.160387\du}{9.273403\du}}
\pgfpathlineto{\pgfpoint{25.128624\du}{9.282165\du}}
\pgfpathlineto{\pgfpoint{25.097955\du}{9.291658\du}}
\pgfpathlineto{\pgfpoint{25.066922\du}{9.300420\du}}
\pgfpathlineto{\pgfpoint{25.052318\du}{9.305166\du}}
\pgfpathlineto{\pgfpoint{25.038445\du}{9.310277\du}}
\pgfpathlineto{\pgfpoint{25.022746\du}{9.315024\du}}
\pgfpathlineto{\pgfpoint{25.008872\du}{9.319770\du}}
\pgfpathlineto{\pgfpoint{24.994268\du}{9.324881\du}}
\pgfpathlineto{\pgfpoint{24.980394\du}{9.329628\du}}
\pgfpathlineto{\pgfpoint{24.966156\du}{9.334739\du}}
\pgfpathlineto{\pgfpoint{24.953012\du}{9.339485\du}}
\pgfpathlineto{\pgfpoint{24.938773\du}{9.344962\du}}
\pgfpathlineto{\pgfpoint{24.925630\du}{9.350073\du}}
\pgfpathlineto{\pgfpoint{24.913217\du}{9.355184\du}}
\pgfpathlineto{\pgfpoint{24.899343\du}{9.361026\du}}
\pgfpathlineto{\pgfpoint{24.886564\du}{9.366502\du}}
\pgfpathlineto{\pgfpoint{24.874516\du}{9.371614\du}}
\pgfpathlineto{\pgfpoint{24.861373\du}{9.377455\du}}
\pgfpathlineto{\pgfpoint{24.849325\du}{9.382932\du}}
\pgfpathlineto{\pgfpoint{24.836911\du}{9.388773\du}}
\pgfpathlineto{\pgfpoint{24.825228\du}{9.393885\du}}
\pgfpathlineto{\pgfpoint{24.813910\du}{9.399726\du}}
\pgfpathlineto{\pgfpoint{24.802227\du}{9.405568\du}}
\pgfpathlineto{\pgfpoint{24.790909\du}{9.411409\du}}
\pgfpathlineto{\pgfpoint{24.779226\du}{9.417251\du}}
\pgfpathlineto{\pgfpoint{24.768273\du}{9.423092\du}}
\pgfpathlineto{\pgfpoint{24.758050\du}{9.429664\du}}
\pgfpathlineto{\pgfpoint{24.747463\du}{9.435506\du}}
\pgfpathlineto{\pgfpoint{24.737240\du}{9.441347\du}}
\pgfpathlineto{\pgfpoint{24.727017\du}{9.447919\du}}
\pgfpathlineto{\pgfpoint{24.717160\du}{9.453760\du}}
\pgfpathlineto{\pgfpoint{24.707667\du}{9.459967\du}}
\pgfpathlineto{\pgfpoint{24.698175\du}{9.466539\du}}
\pgfpathlineto{\pgfpoint{24.689047\du}{9.473111\du}}
\pgfpathlineto{\pgfpoint{24.679189\du}{9.478952\du}}
\pgfpathlineto{\pgfpoint{24.670792\du}{9.485159\du}}
\pgfpathlineto{\pgfpoint{24.662030\du}{9.491731\du}}
\pgfpathlineto{\pgfpoint{24.653633\du}{9.497937\du}}
\pgfpathlineto{\pgfpoint{24.644870\du}{9.505239\du}}
\pgfpathlineto{\pgfpoint{24.637203\du}{9.511446\du}}
\pgfpathlineto{\pgfpoint{24.630267\du}{9.518018\du}}
\pgfpathlineto{\pgfpoint{24.621504\du}{9.524954\du}}
\pgfpathlineto{\pgfpoint{24.614567\du}{9.531526\du}}
\pgfpathlineto{\pgfpoint{24.607631\du}{9.538463\du}}
\pgfpathlineto{\pgfpoint{24.600694\du}{9.545400\du}}
\pgfpathlineto{\pgfpoint{24.594122\du}{9.551972\du}}
\pgfpathlineto{\pgfpoint{24.587185\du}{9.558908\du}}
\pgfpathlineto{\pgfpoint{24.581344\du}{9.565845\du}}
\pgfpathlineto{\pgfpoint{24.575502\du}{9.573147\du}}
\pgfpathlineto{\pgfpoint{24.569660\du}{9.580084\du}}
\pgfpathlineto{\pgfpoint{24.564184\du}{9.587021\du}}
\pgfpathlineto{\pgfpoint{24.559438\du}{9.593958\du}}
\pgfpathlineto{\pgfpoint{24.553961\du}{9.601625\du}}
\pgfpathlineto{\pgfpoint{24.549945\du}{9.608562\du}}
\pgfpathlineto{\pgfpoint{24.544834\du}{9.615863\du}}
\pgfpathlineto{\pgfpoint{24.541183\du}{9.623165\du}}
\pgfpathlineto{\pgfpoint{24.536802\du}{9.630467\du}}
\pgfpathlineto{\pgfpoint{24.533151\du}{9.638134\du}}
\pgfpathlineto{\pgfpoint{24.529865\du}{9.645436\du}}
\pgfpathlineto{\pgfpoint{24.526214\du}{9.653103\du}}
\pgfpathlineto{\pgfpoint{24.523293\du}{9.660770\du}}
\pgfpathlineto{\pgfpoint{24.520737\du}{9.667707\du}}
\pgfpathlineto{\pgfpoint{24.518547\du}{9.675374\du}}
\pgfpathlineto{\pgfpoint{24.516356\du}{9.683041\du}}
\pgfpathlineto{\pgfpoint{24.514531\du}{9.691073\du}}
\pgfpathlineto{\pgfpoint{24.512705\du}{9.698740\du}}
\pgfpathlineto{\pgfpoint{24.511610\du}{9.706407\du}}
\pgfpathlineto{\pgfpoint{24.510515\du}{9.714074\du}}
\pgfpathlineto{\pgfpoint{24.509785\du}{9.722107\du}}
\pgfpathlineto{\pgfpoint{24.509419\du}{9.729774\du}}
\pgfpathlineto{\pgfpoint{24.509419\du}{9.737441\du}}
\pgfpathlineto{\pgfpoint{24.529865\du}{9.737441\du}}
\pgfpathlineto{\pgfpoint{24.530230\du}{9.730504\du}}
\pgfpathlineto{\pgfpoint{24.530230\du}{9.723567\du}}
\pgfpathlineto{\pgfpoint{24.530595\du}{9.716265\du}}
\pgfpathlineto{\pgfpoint{24.532055\du}{9.709328\du}}
\pgfpathlineto{\pgfpoint{24.533151\du}{9.702391\du}}
\pgfpathlineto{\pgfpoint{24.534246\du}{9.695455\du}}
\pgfpathlineto{\pgfpoint{24.536437\du}{9.688153\du}}
\pgfpathlineto{\pgfpoint{24.537897\du}{9.681946\du}}
\pgfpathlineto{\pgfpoint{24.540088\du}{9.674644\du}}
\pgfpathlineto{\pgfpoint{24.542278\du}{9.667707\du}}
\pgfpathlineto{\pgfpoint{24.545564\du}{9.660770\du}}
\pgfpathlineto{\pgfpoint{24.548120\du}{9.653834\du}}
\pgfpathlineto{\pgfpoint{24.551041\du}{9.646897\du}}
\pgfpathlineto{\pgfpoint{24.555057\du}{9.640325\du}}
\pgfpathlineto{\pgfpoint{24.559073\du}{9.633388\du}}
\pgfpathlineto{\pgfpoint{24.562359\du}{9.626816\du}}
\pgfpathlineto{\pgfpoint{24.566375\du}{9.619880\du}}
\pgfpathlineto{\pgfpoint{24.571121\du}{9.612943\du}}
\pgfpathlineto{\pgfpoint{24.575502\du}{9.606371\du}}
\pgfpathlineto{\pgfpoint{24.580978\du}{9.599799\du}}
\pgfpathlineto{\pgfpoint{24.585360\du}{9.592862\du}}
\pgfpathlineto{\pgfpoint{24.591566\du}{9.586291\du}}
\pgfpathlineto{\pgfpoint{24.596678\du}{9.579354\du}}
\pgfpathlineto{\pgfpoint{24.602884\du}{9.573147\du}}
\pgfpathlineto{\pgfpoint{24.608726\du}{9.566575\du}}
\pgfpathlineto{\pgfpoint{24.615298\du}{9.559639\du}}
\pgfpathlineto{\pgfpoint{24.621504\du}{9.553067\du}}
\pgfpathlineto{\pgfpoint{24.628441\du}{9.546860\du}}
\pgfpathlineto{\pgfpoint{24.635743\du}{9.540288\du}}
\pgfpathlineto{\pgfpoint{24.642680\du}{9.533717\du}}
\pgfpathlineto{\pgfpoint{24.650347\du}{9.527510\du}}
\pgfpathlineto{\pgfpoint{24.658379\du}{9.520938\du}}
\pgfpathlineto{\pgfpoint{24.665681\du}{9.514367\du}}
\pgfpathlineto{\pgfpoint{24.674078\du}{9.508160\du}}
\pgfpathlineto{\pgfpoint{24.682840\du}{9.502318\du}}
\pgfpathlineto{\pgfpoint{24.691968\du}{9.495382\du}}
\pgfpathlineto{\pgfpoint{24.700730\du}{9.489905\du}}
\pgfpathlineto{\pgfpoint{24.709127\du}{9.483333\du}}
\pgfpathlineto{\pgfpoint{24.718255\du}{9.477127\du}}
\pgfpathlineto{\pgfpoint{24.728112\du}{9.471285\du}}
\pgfpathlineto{\pgfpoint{24.737605\du}{9.465444\du}}
\pgfpathlineto{\pgfpoint{24.747463\du}{9.458872\du}}
\pgfpathlineto{\pgfpoint{24.758050\du}{9.453030\du}}
\pgfpathlineto{\pgfpoint{24.767908\du}{9.447189\du}}
\pgfpathlineto{\pgfpoint{24.778496\du}{9.441347\du}}
\pgfpathlineto{\pgfpoint{24.789814\du}{9.435506\du}}
\pgfpathlineto{\pgfpoint{24.799671\du}{9.429664\du}}
\pgfpathlineto{\pgfpoint{24.811355\du}{9.423823\du}}
\pgfpathlineto{\pgfpoint{24.822307\du}{9.418711\du}}
\pgfpathlineto{\pgfpoint{24.833991\du}{9.412505\du}}
\pgfpathlineto{\pgfpoint{24.845674\du}{9.406663\du}}
\pgfpathlineto{\pgfpoint{24.858452\du}{9.401552\du}}
\pgfpathlineto{\pgfpoint{24.869770\du}{9.396440\du}}
\pgfpathlineto{\pgfpoint{24.881818\du}{9.390599\du}}
\pgfpathlineto{\pgfpoint{24.894962\du}{9.385122\du}}
\pgfpathlineto{\pgfpoint{24.907375\du}{9.380011\du}}
\pgfpathlineto{\pgfpoint{24.920153\du}{9.374535\du}}
\pgfpathlineto{\pgfpoint{24.933662\du}{9.369423\du}}
\pgfpathlineto{\pgfpoint{24.946440\du}{9.363947\du}}
\pgfpathlineto{\pgfpoint{24.959949\du}{9.358470\du}}
\pgfpathlineto{\pgfpoint{24.973457\du}{9.354089\du}}
\pgfpathlineto{\pgfpoint{24.987331\du}{9.348978\du}}
\pgfpathlineto{\pgfpoint{25.001205\du}{9.344231\du}}
\pgfpathlineto{\pgfpoint{25.015444\du}{9.339120\du}}
\pgfpathlineto{\pgfpoint{25.029682\du}{9.334374\du}}
\pgfpathlineto{\pgfpoint{25.043921\du}{9.329628\du}}
\pgfpathlineto{\pgfpoint{25.058160\du}{9.324881\du}}
\pgfpathlineto{\pgfpoint{25.073494\du}{9.320135\du}}
\pgfpathlineto{\pgfpoint{25.103067\du}{9.311008\du}}
\pgfpathlineto{\pgfpoint{25.134465\du}{9.302245\du}}
\pgfpathlineto{\pgfpoint{25.165498\du}{9.293483\du}}
\pgfpathlineto{\pgfpoint{25.197992\du}{9.285086\du}}
\pgfpathlineto{\pgfpoint{25.229390\du}{9.277054\du}}
\pgfpathlineto{\pgfpoint{25.262614\du}{9.268656\du}}
\pgfpathlineto{\pgfpoint{25.296568\du}{9.260989\du}}
\pgfpathlineto{\pgfpoint{25.330887\du}{9.254053\du}}
\pgfpathlineto{\pgfpoint{25.365571\du}{9.247116\du}}
\pgfpathlineto{\pgfpoint{25.401716\du}{9.239814\du}}
\pgfpathlineto{\pgfpoint{25.436765\du}{9.233607\du}}
\pgfpathlineto{\pgfpoint{25.474005\du}{9.226670\du}}
\pgfpathlineto{\pgfpoint{25.510515\du}{9.221194\du}}
\pgfpathlineto{\pgfpoint{25.547755\du}{9.215352\du}}
\pgfpathlineto{\pgfpoint{25.585360\du}{9.210241\du}}
\pgfpathlineto{\pgfpoint{25.623695\du}{9.205495\du}}
\pgfpathlineto{\pgfpoint{25.662760\du}{9.200748\du}}
\pgfpathlineto{\pgfpoint{25.702191\du}{9.196367\du}}
\pgfpathlineto{\pgfpoint{25.741621\du}{9.192716\du}}
\pgfpathlineto{\pgfpoint{25.781417\du}{9.188700\du}}
\pgfpathlineto{\pgfpoint{25.821577\du}{9.185779\du}}
\pgfpathlineto{\pgfpoint{25.862468\du}{9.182129\du}}
\pgfpathlineto{\pgfpoint{25.903724\du}{9.179208\du}}
\pgfpathlineto{\pgfpoint{25.944615\du}{9.177017\du}}
\pgfpathlineto{\pgfpoint{25.986601\du}{9.175192\du}}
\pgfpathlineto{\pgfpoint{26.028952\du}{9.173366\du}}
\pgfpathlineto{\pgfpoint{26.071668\du}{9.171906\du}}
\pgfpathlineto{\pgfpoint{26.114020\du}{9.171541\du}}
\pgfpathlineto{\pgfpoint{26.157466\du}{9.171176\du}}
\pgfpathlineto{\pgfpoint{26.200183\du}{9.170445\du}}
\pgfpathlineto{\pgfpoint{26.200183\du}{9.170445\du}}
\pgfpathlineto{\pgfpoint{26.200183\du}{9.170445\du}}
\pgfpathlineto{\pgfpoint{26.201278\du}{9.170445\du}}
\pgfpathlineto{\pgfpoint{26.202373\du}{9.170445\du}}
\pgfpathlineto{\pgfpoint{26.203834\du}{9.169715\du}}
\pgfpathlineto{\pgfpoint{26.204929\du}{9.169715\du}}
\pgfpathlineto{\pgfpoint{26.205294\du}{9.169350\du}}
\pgfpathlineto{\pgfpoint{26.206389\du}{9.168620\du}}
\pgfpathlineto{\pgfpoint{26.207119\du}{9.168255\du}}
\pgfpathlineto{\pgfpoint{26.208215\du}{9.167525\du}}
\pgfpathlineto{\pgfpoint{26.209310\du}{9.165699\du}}
\pgfpathlineto{\pgfpoint{26.210040\du}{9.163874\du}}
\pgfpathlineto{\pgfpoint{26.210040\du}{9.162413\du}}
\pgfpathlineto{\pgfpoint{26.210770\du}{9.160588\du}}
\pgfpathlineto{\pgfpoint{26.210040\du}{9.158762\du}}
\pgfpathlineto{\pgfpoint{26.210040\du}{9.156572\du}}
\pgfpathlineto{\pgfpoint{26.209310\du}{9.154746\du}}
\pgfpathlineto{\pgfpoint{26.208215\du}{9.152921\du}}
\pgfpathlineto{\pgfpoint{26.207119\du}{9.152191\du}}
\pgfpathlineto{\pgfpoint{26.206389\du}{9.151825\du}}
\pgfpathlineto{\pgfpoint{26.205294\du}{9.151095\du}}
\pgfpathlineto{\pgfpoint{26.204929\du}{9.150730\du}}
\pgfpathlineto{\pgfpoint{26.203834\du}{9.150730\du}}
\pgfpathlineto{\pgfpoint{26.202373\du}{9.150000\du}}
\pgfpathlineto{\pgfpoint{26.201278\du}{9.150000\du}}
\pgfpathlineto{\pgfpoint{26.200183\du}{9.150000\du}}
\pgfusepath{fill}
\pgfsetbuttcap
\pgfsetmiterjoin
\pgfsetdash{}{0pt}
\definecolor{dialinecolor}{rgb}{0.678431, 0.839216, 0.905882}
\pgfsetfillcolor{dialinecolor}
\pgfpathmoveto{\pgfpoint{27.890581\du}{9.737441\du}}
\pgfpathlineto{\pgfpoint{27.890581\du}{9.729774\du}}
\pgfpathlineto{\pgfpoint{27.889850\du}{9.722107\du}}
\pgfpathlineto{\pgfpoint{27.889120\du}{9.714074\du}}
\pgfpathlineto{\pgfpoint{27.888390\du}{9.706407\du}}
\pgfpathlineto{\pgfpoint{27.887295\du}{9.698740\du}}
\pgfpathlineto{\pgfpoint{27.885104\du}{9.691073\du}}
\pgfpathlineto{\pgfpoint{27.884009\du}{9.683041\du}}
\pgfpathlineto{\pgfpoint{27.881088\du}{9.675374\du}}
\pgfpathlineto{\pgfpoint{27.879263\du}{9.667707\du}}
\pgfpathlineto{\pgfpoint{27.876342\du}{9.660770\du}}
\pgfpathlineto{\pgfpoint{27.873786\du}{9.653103\du}}
\pgfpathlineto{\pgfpoint{27.870500\du}{9.645436\du}}
\pgfpathlineto{\pgfpoint{27.866484\du}{9.638134\du}}
\pgfpathlineto{\pgfpoint{27.863198\du}{9.630467\du}}
\pgfpathlineto{\pgfpoint{27.859182\du}{9.623165\du}}
\pgfpathlineto{\pgfpoint{27.855166\du}{9.615863\du}}
\pgfpathlineto{\pgfpoint{27.850785\du}{9.608562\du}}
\pgfpathlineto{\pgfpoint{27.846039\du}{9.601625\du}}
\pgfpathlineto{\pgfpoint{27.840562\du}{9.593958\du}}
\pgfpathlineto{\pgfpoint{27.835816\du}{9.587021\du}}
\pgfpathlineto{\pgfpoint{27.829974\du}{9.580084\du}}
\pgfpathlineto{\pgfpoint{27.824498\du}{9.573147\du}}
\pgfpathlineto{\pgfpoint{27.818656\du}{9.565845\du}}
\pgfpathlineto{\pgfpoint{27.812450\du}{9.558908\du}}
\pgfpathlineto{\pgfpoint{27.805878\du}{9.551972\du}}
\pgfpathlineto{\pgfpoint{27.798941\du}{9.545400\du}}
\pgfpathlineto{\pgfpoint{27.792369\du}{9.538463\du}}
\pgfpathlineto{\pgfpoint{27.785068\du}{9.531526\du}}
\pgfpathlineto{\pgfpoint{27.778496\du}{9.524954\du}}
\pgfpathlineto{\pgfpoint{27.770464\du}{9.518018\du}}
\pgfpathlineto{\pgfpoint{27.762432\du}{9.511446\du}}
\pgfpathlineto{\pgfpoint{27.755130\du}{9.505239\du}}
\pgfpathlineto{\pgfpoint{27.746367\du}{9.497937\du}}
\pgfpathlineto{\pgfpoint{27.738335\du}{9.491731\du}}
\pgfpathlineto{\pgfpoint{27.729208\du}{9.485159\du}}
\pgfpathlineto{\pgfpoint{27.720445\du}{9.478952\du}}
\pgfpathlineto{\pgfpoint{27.710953\du}{9.473111\du}}
\pgfpathlineto{\pgfpoint{27.701825\du}{9.466539\du}}
\pgfpathlineto{\pgfpoint{27.692698\du}{9.459967\du}}
\pgfpathlineto{\pgfpoint{27.682840\du}{9.453760\du}}
\pgfpathlineto{\pgfpoint{27.672983\du}{9.447919\du}}
\pgfpathlineto{\pgfpoint{27.662760\du}{9.441347\du}}
\pgfpathlineto{\pgfpoint{27.652172\du}{9.435506\du}}
\pgfpathlineto{\pgfpoint{27.642315\du}{9.429664\du}}
\pgfpathlineto{\pgfpoint{27.631727\du}{9.423092\du}}
\pgfpathlineto{\pgfpoint{27.620409\du}{9.417251\du}}
\pgfpathlineto{\pgfpoint{27.609091\du}{9.411409\du}}
\pgfpathlineto{\pgfpoint{27.597408\du}{9.405568\du}}
\pgfpathlineto{\pgfpoint{27.586455\du}{9.399726\du}}
\pgfpathlineto{\pgfpoint{27.574407\du}{9.393885\du}}
\pgfpathlineto{\pgfpoint{27.563089\du}{9.388773\du}}
\pgfpathlineto{\pgfpoint{27.550675\du}{9.382932\du}}
\pgfpathlineto{\pgfpoint{27.538262\du}{9.377455\du}}
\pgfpathlineto{\pgfpoint{27.525484\du}{9.371614\du}}
\pgfpathlineto{\pgfpoint{27.513436\du}{9.366502\du}}
\pgfpathlineto{\pgfpoint{27.500657\du}{9.361026\du}}
\pgfpathlineto{\pgfpoint{27.487149\du}{9.355184\du}}
\pgfpathlineto{\pgfpoint{27.474005\du}{9.350073\du}}
\pgfpathlineto{\pgfpoint{27.460862\du}{9.344962\du}}
\pgfpathlineto{\pgfpoint{27.446623\du}{9.339485\du}}
\pgfpathlineto{\pgfpoint{27.433479\du}{9.334739\du}}
\pgfpathlineto{\pgfpoint{27.419241\du}{9.329628\du}}
\pgfpathlineto{\pgfpoint{27.405367\du}{9.324881\du}}
\pgfpathlineto{\pgfpoint{27.391128\du}{9.319770\du}}
\pgfpathlineto{\pgfpoint{27.377254\du}{9.315024\du}}
\pgfpathlineto{\pgfpoint{27.362286\du}{9.310277\du}}
\pgfpathlineto{\pgfpoint{27.347317\du}{9.305166\du}}
\pgfpathlineto{\pgfpoint{27.333078\du}{9.300420\du}}
\pgfpathlineto{\pgfpoint{27.302410\du}{9.291658\du}}
\pgfpathlineto{\pgfpoint{27.271742\du}{9.282165\du}}
\pgfpathlineto{\pgfpoint{27.240343\du}{9.273403\du}}
\pgfpathlineto{\pgfpoint{27.208580\du}{9.265371\du}}
\pgfpathlineto{\pgfpoint{27.175721\du}{9.256973\du}}
\pgfpathlineto{\pgfpoint{27.142132\du}{9.248941\du}}
\pgfpathlineto{\pgfpoint{27.108178\du}{9.241274\du}}
\pgfpathlineto{\pgfpoint{27.073494\du}{9.233607\du}}
\pgfpathlineto{\pgfpoint{27.038810\du}{9.226670\du}}
\pgfpathlineto{\pgfpoint{27.003395\du}{9.219733\du}}
\pgfpathlineto{\pgfpoint{26.966886\du}{9.213162\du}}
\pgfpathlineto{\pgfpoint{26.930376\du}{9.207320\du}}
\pgfpathlineto{\pgfpoint{26.893501\du}{9.200748\du}}
\pgfpathlineto{\pgfpoint{26.856261\du}{9.195637\du}}
\pgfpathlineto{\pgfpoint{26.817196\du}{9.190161\du}}
\pgfpathlineto{\pgfpoint{26.779226\du}{9.185049\du}}
\pgfpathlineto{\pgfpoint{26.739796\du}{9.180303\du}}
\pgfpathlineto{\pgfpoint{26.701095\du}{9.175922\du}}
\pgfpathlineto{\pgfpoint{26.660935\du}{9.171906\du}}
\pgfpathlineto{\pgfpoint{26.621139\du}{9.168255\du}}
\pgfpathlineto{\pgfpoint{26.580248\du}{9.164604\du}}
\pgfpathlineto{\pgfpoint{26.539723\du}{9.161683\du}}
\pgfpathlineto{\pgfpoint{26.498101\du}{9.159493\du}}
\pgfpathlineto{\pgfpoint{26.456480\du}{9.156937\du}}
\pgfpathlineto{\pgfpoint{26.414859\du}{9.154746\du}}
\pgfpathlineto{\pgfpoint{26.372143\du}{9.152921\du}}
\pgfpathlineto{\pgfpoint{26.330157\du}{9.151825\du}}
\pgfpathlineto{\pgfpoint{26.287076\du}{9.150730\du}}
\pgfpathlineto{\pgfpoint{26.243629\du}{9.150000\du}}
\pgfpathlineto{\pgfpoint{26.200183\du}{9.150000\du}}
\pgfpathlineto{\pgfpoint{26.200183\du}{9.170445\du}}
\pgfpathlineto{\pgfpoint{26.243264\du}{9.171176\du}}
\pgfpathlineto{\pgfpoint{26.286710\du}{9.171541\du}}
\pgfpathlineto{\pgfpoint{26.328697\du}{9.171906\du}}
\pgfpathlineto{\pgfpoint{26.371413\du}{9.173366\du}}
\pgfpathlineto{\pgfpoint{26.414129\du}{9.175192\du}}
\pgfpathlineto{\pgfpoint{26.455750\du}{9.177017\du}}
\pgfpathlineto{\pgfpoint{26.497006\du}{9.179208\du}}
\pgfpathlineto{\pgfpoint{26.537897\du}{9.182129\du}}
\pgfpathlineto{\pgfpoint{26.579153\du}{9.185779\du}}
\pgfpathlineto{\pgfpoint{26.619314\du}{9.188700\du}}
\pgfpathlineto{\pgfpoint{26.659474\du}{9.192716\du}}
\pgfpathlineto{\pgfpoint{26.698540\du}{9.196367\du}}
\pgfpathlineto{\pgfpoint{26.737970\du}{9.200748\du}}
\pgfpathlineto{\pgfpoint{26.776670\du}{9.205495\du}}
\pgfpathlineto{\pgfpoint{26.815371\du}{9.210241\du}}
\pgfpathlineto{\pgfpoint{26.852976\du}{9.215352\du}}
\pgfpathlineto{\pgfpoint{26.890215\du}{9.221194\du}}
\pgfpathlineto{\pgfpoint{26.926725\du}{9.226670\du}}
\pgfpathlineto{\pgfpoint{26.963600\du}{9.233607\du}}
\pgfpathlineto{\pgfpoint{26.999014\du}{9.239814\du}}
\pgfpathlineto{\pgfpoint{27.034794\du}{9.247116\du}}
\pgfpathlineto{\pgfpoint{27.069478\du}{9.254053\du}}
\pgfpathlineto{\pgfpoint{27.103432\du}{9.260989\du}}
\pgfpathlineto{\pgfpoint{27.137751\du}{9.268656\du}}
\pgfpathlineto{\pgfpoint{27.170975\du}{9.277054\du}}
\pgfpathlineto{\pgfpoint{27.203103\du}{9.285086\du}}
\pgfpathlineto{\pgfpoint{27.235232\du}{9.293483\du}}
\pgfpathlineto{\pgfpoint{27.266630\du}{9.302245\du}}
\pgfpathlineto{\pgfpoint{27.297298\du}{9.311008\du}}
\pgfpathlineto{\pgfpoint{27.327236\du}{9.320135\du}}
\pgfpathlineto{\pgfpoint{27.341475\du}{9.324881\du}}
\pgfpathlineto{\pgfpoint{27.356079\du}{9.329628\du}}
\pgfpathlineto{\pgfpoint{27.369953\du}{9.334374\du}}
\pgfpathlineto{\pgfpoint{27.384556\du}{9.339120\du}}
\pgfpathlineto{\pgfpoint{27.398795\du}{9.344231\du}}
\pgfpathlineto{\pgfpoint{27.412669\du}{9.348978\du}}
\pgfpathlineto{\pgfpoint{27.426908\du}{9.354089\du}}
\pgfpathlineto{\pgfpoint{27.440051\du}{9.358470\du}}
\pgfpathlineto{\pgfpoint{27.453560\du}{9.363947\du}}
\pgfpathlineto{\pgfpoint{27.466703\du}{9.369423\du}}
\pgfpathlineto{\pgfpoint{27.479482\du}{9.374535\du}}
\pgfpathlineto{\pgfpoint{27.492260\du}{9.380011\du}}
\pgfpathlineto{\pgfpoint{27.505403\du}{9.385122\du}}
\pgfpathlineto{\pgfpoint{27.517452\du}{9.390599\du}}
\pgfpathlineto{\pgfpoint{27.529865\du}{9.396440\du}}
\pgfpathlineto{\pgfpoint{27.541913\du}{9.401552\du}}
\pgfpathlineto{\pgfpoint{27.554326\du}{9.406663\du}}
\pgfpathlineto{\pgfpoint{27.565644\du}{9.412505\du}}
\pgfpathlineto{\pgfpoint{27.577693\du}{9.418711\du}}
\pgfpathlineto{\pgfpoint{27.588280\du}{9.423823\du}}
\pgfpathlineto{\pgfpoint{27.600329\du}{9.429664\du}}
\pgfpathlineto{\pgfpoint{27.610551\du}{9.435506\du}}
\pgfpathlineto{\pgfpoint{27.621504\du}{9.441347\du}}
\pgfpathlineto{\pgfpoint{27.632457\du}{9.447189\du}}
\pgfpathlineto{\pgfpoint{27.642315\du}{9.453030\du}}
\pgfpathlineto{\pgfpoint{27.652172\du}{9.458872\du}}
\pgfpathlineto{\pgfpoint{27.662395\du}{9.465444\du}}
\pgfpathlineto{\pgfpoint{27.671522\du}{9.471285\du}}
\pgfpathlineto{\pgfpoint{27.681745\du}{9.477127\du}}
\pgfpathlineto{\pgfpoint{27.691238\du}{9.483333\du}}
\pgfpathlineto{\pgfpoint{27.700000\du}{9.489905\du}}
\pgfpathlineto{\pgfpoint{27.708032\du}{9.495382\du}}
\pgfpathlineto{\pgfpoint{27.716794\du}{9.502318\du}}
\pgfpathlineto{\pgfpoint{27.725192\du}{9.508160\du}}
\pgfpathlineto{\pgfpoint{27.733954\du}{9.514367\du}}
\pgfpathlineto{\pgfpoint{27.741621\du}{9.520938\du}}
\pgfpathlineto{\pgfpoint{27.749653\du}{9.527510\du}}
\pgfpathlineto{\pgfpoint{27.756955\du}{9.533717\du}}
\pgfpathlineto{\pgfpoint{27.764622\du}{9.540288\du}}
\pgfpathlineto{\pgfpoint{27.771194\du}{9.546860\du}}
\pgfpathlineto{\pgfpoint{27.778496\du}{9.553067\du}}
\pgfpathlineto{\pgfpoint{27.784337\du}{9.559639\du}}
\pgfpathlineto{\pgfpoint{27.791274\du}{9.566575\du}}
\pgfpathlineto{\pgfpoint{27.797846\du}{9.573147\du}}
\pgfpathlineto{\pgfpoint{27.802957\du}{9.579354\du}}
\pgfpathlineto{\pgfpoint{27.808434\du}{9.586291\du}}
\pgfpathlineto{\pgfpoint{27.814640\du}{9.592862\du}}
\pgfpathlineto{\pgfpoint{27.819387\du}{9.599799\du}}
\pgfpathlineto{\pgfpoint{27.824498\du}{9.606371\du}}
\pgfpathlineto{\pgfpoint{27.829244\du}{9.612943\du}}
\pgfpathlineto{\pgfpoint{27.833625\du}{9.619880\du}}
\pgfpathlineto{\pgfpoint{27.838007\du}{9.626816\du}}
\pgfpathlineto{\pgfpoint{27.840927\du}{9.633388\du}}
\pgfpathlineto{\pgfpoint{27.844943\du}{9.640325\du}}
\pgfpathlineto{\pgfpoint{27.847864\du}{9.646897\du}}
\pgfpathlineto{\pgfpoint{27.851880\du}{9.653834\du}}
\pgfpathlineto{\pgfpoint{27.854436\du}{9.660770\du}}
\pgfpathlineto{\pgfpoint{27.857357\du}{9.667707\du}}
\pgfpathlineto{\pgfpoint{27.859912\du}{9.674644\du}}
\pgfpathlineto{\pgfpoint{27.862103\du}{9.681946\du}}
\pgfpathlineto{\pgfpoint{27.863563\du}{9.688153\du}}
\pgfpathlineto{\pgfpoint{27.865754\du}{9.695455\du}}
\pgfpathlineto{\pgfpoint{27.866484\du}{9.702391\du}}
\pgfpathlineto{\pgfpoint{27.867579\du}{9.709328\du}}
\pgfpathlineto{\pgfpoint{27.869405\du}{9.716265\du}}
\pgfpathlineto{\pgfpoint{27.869770\du}{9.723567\du}}
\pgfpathlineto{\pgfpoint{27.869770\du}{9.730504\du}}
\pgfpathlineto{\pgfpoint{27.870500\du}{9.737441\du}}
\pgfpathlineto{\pgfpoint{27.890581\du}{9.737441\du}}
\pgfusepath{fill}
\pgfsetbuttcap
\pgfsetmiterjoin
\pgfsetdash{}{0pt}
\definecolor{dialinecolor}{rgb}{0.074510, 0.082353, 0.086275}
\pgfsetfillcolor{dialinecolor}
\pgfpathmoveto{\pgfpoint{26.243264\du}{9.610022\du}}
\pgfpathlineto{\pgfpoint{26.491165\du}{9.692534\du}}
\pgfpathlineto{\pgfpoint{27.076415\du}{9.458142\du}}
\pgfpathlineto{\pgfpoint{27.349142\du}{9.525685\du}}
\pgfpathlineto{\pgfpoint{27.205294\du}{9.317214\du}}
\pgfpathlineto{\pgfpoint{26.501022\du}{9.317214\du}}
\pgfpathlineto{\pgfpoint{26.795290\du}{9.389869\du}}
\pgfpathlineto{\pgfpoint{26.243264\du}{9.610022\du}}
\pgfusepath{fill}
\pgfsetbuttcap
\pgfsetmiterjoin
\pgfsetdash{}{0pt}
\definecolor{dialinecolor}{rgb}{0.074510, 0.082353, 0.086275}
\pgfsetfillcolor{dialinecolor}
\pgfpathmoveto{\pgfpoint{26.141402\du}{9.848065\du}}
\pgfpathlineto{\pgfpoint{25.893501\du}{9.765918\du}}
\pgfpathlineto{\pgfpoint{25.308251\du}{9.999580\du}}
\pgfpathlineto{\pgfpoint{25.035159\du}{9.932767\du}}
\pgfpathlineto{\pgfpoint{25.179007\du}{10.140507\du}}
\pgfpathlineto{\pgfpoint{25.884374\du}{10.140507\du}}
\pgfpathlineto{\pgfpoint{25.589376\du}{10.068583\du}}
\pgfpathlineto{\pgfpoint{26.141402\du}{9.848065\du}}
\pgfusepath{fill}
\pgfsetbuttcap
\pgfsetmiterjoin
\pgfsetdash{}{0pt}
\definecolor{dialinecolor}{rgb}{0.074510, 0.082353, 0.086275}
\pgfsetfillcolor{dialinecolor}
\pgfpathmoveto{\pgfpoint{25.095400\du}{9.389138\du}}
\pgfpathlineto{\pgfpoint{25.342935\du}{9.307357\du}}
\pgfpathlineto{\pgfpoint{25.928185\du}{9.540654\du}}
\pgfpathlineto{\pgfpoint{26.201278\du}{9.474206\du}}
\pgfpathlineto{\pgfpoint{26.057430\du}{9.681946\du}}
\pgfpathlineto{\pgfpoint{25.352428\du}{9.681946\du}}
\pgfpathlineto{\pgfpoint{25.647426\du}{9.610022\du}}
\pgfpathlineto{\pgfpoint{25.095400\du}{9.389138\du}}
\pgfusepath{fill}
\pgfsetbuttcap
\pgfsetmiterjoin
\pgfsetdash{}{0pt}
\definecolor{dialinecolor}{rgb}{0.074510, 0.082353, 0.086275}
\pgfsetfillcolor{dialinecolor}
\pgfpathmoveto{\pgfpoint{27.313728\du}{10.084283\du}}
\pgfpathlineto{\pgfpoint{27.066192\du}{10.166429\du}}
\pgfpathlineto{\pgfpoint{26.480942\du}{9.932767\du}}
\pgfpathlineto{\pgfpoint{26.207119\du}{9.999580\du}}
\pgfpathlineto{\pgfpoint{26.351698\du}{9.791840\du}}
\pgfpathlineto{\pgfpoint{27.057065\du}{9.791840\du}}
\pgfpathlineto{\pgfpoint{26.761701\du}{9.863764\du}}
\pgfpathlineto{\pgfpoint{27.313728\du}{10.084283\du}}
\pgfusepath{fill}
\pgfsetbuttcap
\pgfsetmiterjoin
\pgfsetdash{}{0pt}
\definecolor{dialinecolor}{rgb}{1.000000, 1.000000, 1.000000}
\pgfsetfillcolor{dialinecolor}
\pgfpathmoveto{\pgfpoint{26.264074\du}{9.630467\du}}
\pgfpathlineto{\pgfpoint{26.511610\du}{9.712979\du}}
\pgfpathlineto{\pgfpoint{27.096860\du}{9.478952\du}}
\pgfpathlineto{\pgfpoint{27.369222\du}{9.546130\du}}
\pgfpathlineto{\pgfpoint{27.226470\du}{9.337660\du}}
\pgfpathlineto{\pgfpoint{26.521103\du}{9.337660\du}}
\pgfpathlineto{\pgfpoint{26.816101\du}{9.410314\du}}
\pgfpathlineto{\pgfpoint{26.264074\du}{9.630467\du}}
\pgfusepath{fill}
\pgfsetbuttcap
\pgfsetmiterjoin
\pgfsetdash{}{0pt}
\definecolor{dialinecolor}{rgb}{1.000000, 1.000000, 1.000000}
\pgfsetfillcolor{dialinecolor}
\pgfpathmoveto{\pgfpoint{26.162212\du}{9.869241\du}}
\pgfpathlineto{\pgfpoint{25.913582\du}{9.786729\du}}
\pgfpathlineto{\pgfpoint{25.328697\du}{10.020756\du}}
\pgfpathlineto{\pgfpoint{25.055239\du}{9.953213\du}}
\pgfpathlineto{\pgfpoint{25.200183\du}{10.161683\du}}
\pgfpathlineto{\pgfpoint{25.904819\du}{10.161683\du}}
\pgfpathlineto{\pgfpoint{25.610186\du}{10.089029\du}}
\pgfpathlineto{\pgfpoint{26.162212\du}{9.869241\du}}
\pgfusepath{fill}
\pgfsetbuttcap
\pgfsetmiterjoin
\pgfsetdash{}{0pt}
\definecolor{dialinecolor}{rgb}{1.000000, 1.000000, 1.000000}
\pgfsetfillcolor{dialinecolor}
\pgfpathmoveto{\pgfpoint{25.115845\du}{9.409584\du}}
\pgfpathlineto{\pgfpoint{25.363381\du}{9.327802\du}}
\pgfpathlineto{\pgfpoint{25.948996\du}{9.561829\du}}
\pgfpathlineto{\pgfpoint{26.222088\du}{9.495016\du}}
\pgfpathlineto{\pgfpoint{26.077145\du}{9.702391\du}}
\pgfpathlineto{\pgfpoint{25.372873\du}{9.702391\du}}
\pgfpathlineto{\pgfpoint{25.667506\du}{9.630467\du}}
\pgfpathlineto{\pgfpoint{25.115845\du}{9.409584\du}}
\pgfusepath{fill}
\pgfsetbuttcap
\pgfsetmiterjoin
\pgfsetdash{}{0pt}
\definecolor{dialinecolor}{rgb}{1.000000, 1.000000, 1.000000}
\pgfsetfillcolor{dialinecolor}
\pgfpathmoveto{\pgfpoint{27.333808\du}{10.104728\du}}
\pgfpathlineto{\pgfpoint{27.086272\du}{10.186875\du}}
\pgfpathlineto{\pgfpoint{26.501387\du}{9.953213\du}}
\pgfpathlineto{\pgfpoint{26.227930\du}{10.020026\du}}
\pgfpathlineto{\pgfpoint{26.372143\du}{9.812286\du}}
\pgfpathlineto{\pgfpoint{27.077145\du}{9.812286\du}}
\pgfpathlineto{\pgfpoint{26.782877\du}{9.884210\du}}
\pgfpathlineto{\pgfpoint{27.333808\du}{10.104728\du}}
\pgfusepath{fill}
\pgfsetbuttcap
\pgfsetmiterjoin
\pgfsetdash{}{0pt}
\definecolor{dialinecolor}{rgb}{0.678431, 0.839216, 0.905882}
\pgfsetfillcolor{dialinecolor}
\pgfpathmoveto{\pgfpoint{24.529865\du}{9.748028\du}}
\pgfpathlineto{\pgfpoint{24.529865\du}{9.737441\du}}
\pgfpathlineto{\pgfpoint{24.509419\du}{9.737441\du}}
\pgfpathlineto{\pgfpoint{24.509419\du}{9.748028\du}}
\pgfpathlineto{\pgfpoint{24.529865\du}{9.748028\du}}
\pgfusepath{fill}
\pgfsetbuttcap
\pgfsetmiterjoin
\pgfsetdash{}{0pt}
\definecolor{dialinecolor}{rgb}{0.678431, 0.839216, 0.905882}
\pgfsetfillcolor{dialinecolor}
\pgfpathmoveto{\pgfpoint{24.529865\du}{10.577528\du}}
\pgfpathlineto{\pgfpoint{24.529865\du}{9.748028\du}}
\pgfpathlineto{\pgfpoint{24.509419\du}{9.748028\du}}
\pgfpathlineto{\pgfpoint{24.509419\du}{10.577528\du}}
\pgfpathlineto{\pgfpoint{24.529865\du}{10.577528\du}}
\pgfusepath{fill}
\pgfsetbuttcap
\pgfsetmiterjoin
\pgfsetdash{}{0pt}
\definecolor{dialinecolor}{rgb}{0.678431, 0.839216, 0.905882}
\pgfsetfillcolor{dialinecolor}
\pgfpathmoveto{\pgfpoint{24.509419\du}{10.577528\du}}
\pgfpathlineto{\pgfpoint{24.509419\du}{10.588116\du}}
\pgfpathlineto{\pgfpoint{24.529865\du}{10.588116\du}}
\pgfpathlineto{\pgfpoint{24.529865\du}{10.577528\du}}
\pgfpathlineto{\pgfpoint{24.509419\du}{10.577528\du}}
\pgfusepath{fill}
\pgfsetbuttcap
\pgfsetmiterjoin
\pgfsetdash{}{0pt}
\definecolor{dialinecolor}{rgb}{0.678431, 0.839216, 0.905882}
\pgfsetfillcolor{dialinecolor}
\pgfpathmoveto{\pgfpoint{27.890581\du}{9.748028\du}}
\pgfpathlineto{\pgfpoint{27.890581\du}{9.737441\du}}
\pgfpathlineto{\pgfpoint{27.870500\du}{9.737441\du}}
\pgfpathlineto{\pgfpoint{27.870500\du}{9.748028\du}}
\pgfpathlineto{\pgfpoint{27.890581\du}{9.748028\du}}
\pgfusepath{fill}
\pgfsetbuttcap
\pgfsetmiterjoin
\pgfsetdash{}{0pt}
\definecolor{dialinecolor}{rgb}{0.678431, 0.839216, 0.905882}
\pgfsetfillcolor{dialinecolor}
\pgfpathmoveto{\pgfpoint{27.890581\du}{10.577528\du}}
\pgfpathlineto{\pgfpoint{27.890581\du}{9.748028\du}}
\pgfpathlineto{\pgfpoint{27.870500\du}{9.748028\du}}
\pgfpathlineto{\pgfpoint{27.870500\du}{10.577528\du}}
\pgfpathlineto{\pgfpoint{27.890581\du}{10.577528\du}}
\pgfusepath{fill}
\pgfsetbuttcap
\pgfsetmiterjoin
\pgfsetdash{}{0pt}
\definecolor{dialinecolor}{rgb}{0.678431, 0.839216, 0.905882}
\pgfsetfillcolor{dialinecolor}
\pgfpathmoveto{\pgfpoint{27.870500\du}{10.577528\du}}
\pgfpathlineto{\pgfpoint{27.870500\du}{10.588116\du}}
\pgfpathlineto{\pgfpoint{27.890581\du}{10.588116\du}}
\pgfpathlineto{\pgfpoint{27.890581\du}{10.577528\du}}
\pgfpathlineto{\pgfpoint{27.870500\du}{10.577528\du}}
\pgfusepath{fill}
\pgfsetbuttcap
\pgfsetmiterjoin
\pgfsetdash{}{0pt}
\definecolor{dialinecolor}{rgb}{0.027451, 0.372549, 0.529412}
\pgfsetfillcolor{dialinecolor}
\pgfpathmoveto{\pgfpoint{26.816831\du}{10.726853\du}}
\pgfpathlineto{\pgfpoint{26.816466\du}{10.744378\du}}
\pgfpathlineto{\pgfpoint{26.813910\du}{10.761537\du}}
\pgfpathlineto{\pgfpoint{26.810259\du}{10.777601\du}}
\pgfpathlineto{\pgfpoint{26.804418\du}{10.794761\du}}
\pgfpathlineto{\pgfpoint{26.797846\du}{10.810460\du}}
\pgfpathlineto{\pgfpoint{26.789449\du}{10.826524\du}}
\pgfpathlineto{\pgfpoint{26.779956\du}{10.842223\du}}
\pgfpathlineto{\pgfpoint{26.768273\du}{10.857192\du}}
\pgfpathlineto{\pgfpoint{26.756590\du}{10.872161\du}}
\pgfpathlineto{\pgfpoint{26.743081\du}{10.886765\du}}
\pgfpathlineto{\pgfpoint{26.728112\du}{10.900639\du}}
\pgfpathlineto{\pgfpoint{26.712048\du}{10.914878\du}}
\pgfpathlineto{\pgfpoint{26.694524\du}{10.927656\du}}
\pgfpathlineto{\pgfpoint{26.676269\du}{10.940434\du}}
\pgfpathlineto{\pgfpoint{26.657284\du}{10.952848\du}}
\pgfpathlineto{\pgfpoint{26.637203\du}{10.964531\du}}
\pgfpathlineto{\pgfpoint{26.615663\du}{10.975119\du}}
\pgfpathlineto{\pgfpoint{26.593027\du}{10.986072\du}}
\pgfpathlineto{\pgfpoint{26.570026\du}{10.995929\du}}
\pgfpathlineto{\pgfpoint{26.546294\du}{11.005422\du}}
\pgfpathlineto{\pgfpoint{26.521103\du}{11.013454\du}}
\pgfpathlineto{\pgfpoint{26.495546\du}{11.021851\du}}
\pgfpathlineto{\pgfpoint{26.468894\du}{11.029518\du}}
\pgfpathlineto{\pgfpoint{26.442242\du}{11.036455\du}}
\pgfpathlineto{\pgfpoint{26.414129\du}{11.042296\du}}
\pgfpathlineto{\pgfpoint{26.385652\du}{11.047408\du}}
\pgfpathlineto{\pgfpoint{26.356079\du}{11.051789\du}}
\pgfpathlineto{\pgfpoint{26.325776\du}{11.055805\du}}
\pgfpathlineto{\pgfpoint{26.295838\du}{11.058726\du}}
\pgfpathlineto{\pgfpoint{26.265170\du}{11.060916\du}}
\pgfpathlineto{\pgfpoint{26.233771\du}{11.062012\du}}
\pgfpathlineto{\pgfpoint{26.202373\du}{11.062742\du}}
\pgfpathlineto{\pgfpoint{26.171340\du}{11.062012\du}}
\pgfpathlineto{\pgfpoint{26.139942\du}{11.060916\du}}
\pgfpathlineto{\pgfpoint{26.108908\du}{11.058726\du}}
\pgfpathlineto{\pgfpoint{26.078605\du}{11.055805\du}}
\pgfpathlineto{\pgfpoint{26.049032\du}{11.051789\du}}
\pgfpathlineto{\pgfpoint{26.019460\du}{11.047408\du}}
\pgfpathlineto{\pgfpoint{25.990982\du}{11.042296\du}}
\pgfpathlineto{\pgfpoint{25.963235\du}{11.036455\du}}
\pgfpathlineto{\pgfpoint{25.935853\du}{11.029518\du}}
\pgfpathlineto{\pgfpoint{25.909566\du}{11.021851\du}}
\pgfpathlineto{\pgfpoint{25.884374\du}{11.013454\du}}
\pgfpathlineto{\pgfpoint{25.858817\du}{11.005422\du}}
\pgfpathlineto{\pgfpoint{25.834721\du}{10.995929\du}}
\pgfpathlineto{\pgfpoint{25.812085\du}{10.986072\du}}
\pgfpathlineto{\pgfpoint{25.789449\du}{10.975119\du}}
\pgfpathlineto{\pgfpoint{25.768638\du}{10.964531\du}}
\pgfpathlineto{\pgfpoint{25.748193\du}{10.952848\du}}
\pgfpathlineto{\pgfpoint{25.727747\du}{10.940434\du}}
\pgfpathlineto{\pgfpoint{25.709858\du}{10.927656\du}}
\pgfpathlineto{\pgfpoint{25.693428\du}{10.914878\du}}
\pgfpathlineto{\pgfpoint{25.676634\du}{10.900639\du}}
\pgfpathlineto{\pgfpoint{25.662030\du}{10.886765\du}}
\pgfpathlineto{\pgfpoint{25.648886\du}{10.872161\du}}
\pgfpathlineto{\pgfpoint{25.636473\du}{10.857192\du}}
\pgfpathlineto{\pgfpoint{25.625520\du}{10.842223\du}}
\pgfpathlineto{\pgfpoint{25.616028\du}{10.826524\du}}
\pgfpathlineto{\pgfpoint{25.607265\du}{10.810460\du}}
\pgfpathlineto{\pgfpoint{25.600329\du}{10.794761\du}}
\pgfpathlineto{\pgfpoint{25.595582\du}{10.777601\du}}
\pgfpathlineto{\pgfpoint{25.591201\du}{10.761537\du}}
\pgfpathlineto{\pgfpoint{25.588645\du}{10.744378\du}}
\pgfpathlineto{\pgfpoint{25.587550\du}{10.726853\du}}
\pgfpathlineto{\pgfpoint{25.588645\du}{10.709328\du}}
\pgfpathlineto{\pgfpoint{25.591201\du}{10.692169\du}}
\pgfpathlineto{\pgfpoint{25.595582\du}{10.676104\du}}
\pgfpathlineto{\pgfpoint{25.600329\du}{10.658945\du}}
\pgfpathlineto{\pgfpoint{25.607265\du}{10.643246\du}}
\pgfpathlineto{\pgfpoint{25.616028\du}{10.627547\du}}
\pgfpathlineto{\pgfpoint{25.625520\du}{10.611482\du}}
\pgfpathlineto{\pgfpoint{25.636473\du}{10.596513\du}}
\pgfpathlineto{\pgfpoint{25.648886\du}{10.581909\du}}
\pgfpathlineto{\pgfpoint{25.662030\du}{10.567306\du}}
\pgfpathlineto{\pgfpoint{25.676634\du}{10.553067\du}}
\pgfpathlineto{\pgfpoint{25.693428\du}{10.539193\du}}
\pgfpathlineto{\pgfpoint{25.709858\du}{10.526050\du}}
\pgfpathlineto{\pgfpoint{25.727747\du}{10.514001\du}}
\pgfpathlineto{\pgfpoint{25.748193\du}{10.501588\du}}
\pgfpathlineto{\pgfpoint{25.768638\du}{10.489905\du}}
\pgfpathlineto{\pgfpoint{25.789449\du}{10.478952\du}}
\pgfpathlineto{\pgfpoint{25.812085\du}{10.468364\du}}
\pgfpathlineto{\pgfpoint{25.834721\du}{10.457777\du}}
\pgfpathlineto{\pgfpoint{25.858817\du}{10.449014\du}}
\pgfpathlineto{\pgfpoint{25.884374\du}{10.440252\du}}
\pgfpathlineto{\pgfpoint{25.909566\du}{10.431855\du}}
\pgfpathlineto{\pgfpoint{25.935853\du}{10.424188\du}}
\pgfpathlineto{\pgfpoint{25.963235\du}{10.417981\du}}
\pgfpathlineto{\pgfpoint{25.990982\du}{10.411409\du}}
\pgfpathlineto{\pgfpoint{26.019460\du}{10.406298\du}}
\pgfpathlineto{\pgfpoint{26.049032\du}{10.402282\du}}
\pgfpathlineto{\pgfpoint{26.078605\du}{10.397901\du}}
\pgfpathlineto{\pgfpoint{26.108908\du}{10.394980\du}}
\pgfpathlineto{\pgfpoint{26.139942\du}{10.393520\du}}
\pgfpathlineto{\pgfpoint{26.171340\du}{10.391694\du}}
\pgfpathlineto{\pgfpoint{26.202373\du}{10.391694\du}}
\pgfpathlineto{\pgfpoint{26.233771\du}{10.391694\du}}
\pgfpathlineto{\pgfpoint{26.265170\du}{10.393520\du}}
\pgfpathlineto{\pgfpoint{26.295838\du}{10.394980\du}}
\pgfpathlineto{\pgfpoint{26.325776\du}{10.397901\du}}
\pgfpathlineto{\pgfpoint{26.356079\du}{10.402282\du}}
\pgfpathlineto{\pgfpoint{26.385652\du}{10.406298\du}}
\pgfpathlineto{\pgfpoint{26.414129\du}{10.411409\du}}
\pgfpathlineto{\pgfpoint{26.442242\du}{10.417981\du}}
\pgfpathlineto{\pgfpoint{26.468894\du}{10.424188\du}}
\pgfpathlineto{\pgfpoint{26.495546\du}{10.431855\du}}
\pgfpathlineto{\pgfpoint{26.521103\du}{10.440252\du}}
\pgfpathlineto{\pgfpoint{26.546294\du}{10.449014\du}}
\pgfpathlineto{\pgfpoint{26.570026\du}{10.457777\du}}
\pgfpathlineto{\pgfpoint{26.593027\du}{10.468364\du}}
\pgfpathlineto{\pgfpoint{26.615663\du}{10.478952\du}}
\pgfpathlineto{\pgfpoint{26.637203\du}{10.489905\du}}
\pgfpathlineto{\pgfpoint{26.657284\du}{10.501588\du}}
\pgfpathlineto{\pgfpoint{26.676269\du}{10.514001\du}}
\pgfpathlineto{\pgfpoint{26.694524\du}{10.526050\du}}
\pgfpathlineto{\pgfpoint{26.712048\du}{10.539193\du}}
\pgfpathlineto{\pgfpoint{26.728112\du}{10.553067\du}}
\pgfpathlineto{\pgfpoint{26.743081\du}{10.567306\du}}
\pgfpathlineto{\pgfpoint{26.756590\du}{10.581909\du}}
\pgfpathlineto{\pgfpoint{26.768273\du}{10.596513\du}}
\pgfpathlineto{\pgfpoint{26.779956\du}{10.611482\du}}
\pgfpathlineto{\pgfpoint{26.789449\du}{10.627547\du}}
\pgfpathlineto{\pgfpoint{26.797846\du}{10.643246\du}}
\pgfpathlineto{\pgfpoint{26.804418\du}{10.658945\du}}
\pgfpathlineto{\pgfpoint{26.810259\du}{10.676104\du}}
\pgfpathlineto{\pgfpoint{26.813910\du}{10.692169\du}}
\pgfpathlineto{\pgfpoint{26.816466\du}{10.709328\du}}
\pgfpathlineto{\pgfpoint{26.816831\du}{10.726853\du}}
\pgfusepath{fill}
\pgfsetbuttcap
\pgfsetmiterjoin
\pgfsetdash{}{0pt}
\definecolor{dialinecolor}{rgb}{0.678431, 0.839216, 0.905882}
\pgfsetfillcolor{dialinecolor}
\pgfpathmoveto{\pgfpoint{26.202373\du}{11.072599\du}}
\pgfpathlineto{\pgfpoint{26.202373\du}{11.072599\du}}
\pgfpathlineto{\pgfpoint{26.218437\du}{11.072599\du}}
\pgfpathlineto{\pgfpoint{26.234502\du}{11.072234\du}}
\pgfpathlineto{\pgfpoint{26.250566\du}{11.071504\du}}
\pgfpathlineto{\pgfpoint{26.265900\du}{11.070774\du}}
\pgfpathlineto{\pgfpoint{26.281599\du}{11.069679\du}}
\pgfpathlineto{\pgfpoint{26.296933\du}{11.068583\du}}
\pgfpathlineto{\pgfpoint{26.311902\du}{11.067488\du}}
\pgfpathlineto{\pgfpoint{26.327966\du}{11.065663\du}}
\pgfpathlineto{\pgfpoint{26.342205\du}{11.063837\du}}
\pgfpathlineto{\pgfpoint{26.357174\du}{11.062012\du}}
\pgfpathlineto{\pgfpoint{26.372143\du}{11.059821\du}}
\pgfpathlineto{\pgfpoint{26.387477\du}{11.057631\du}}
\pgfpathlineto{\pgfpoint{26.401716\du}{11.054710\du}}
\pgfpathlineto{\pgfpoint{26.415590\du}{11.052154\du}}
\pgfpathlineto{\pgfpoint{26.430194\du}{11.049233\du}}
\pgfpathlineto{\pgfpoint{26.444067\du}{11.046313\du}}
\pgfpathlineto{\pgfpoint{26.457576\du}{11.042662\du}}
\pgfpathlineto{\pgfpoint{26.471449\du}{11.039376\du}}
\pgfpathlineto{\pgfpoint{26.484958\du}{11.035725\du}}
\pgfpathlineto{\pgfpoint{26.498101\du}{11.031709\du}}
\pgfpathlineto{\pgfpoint{26.511245\du}{11.027693\du}}
\pgfpathlineto{\pgfpoint{26.524388\du}{11.023677\du}}
\pgfpathlineto{\pgfpoint{26.537167\du}{11.018930\du}}
\pgfpathlineto{\pgfpoint{26.549945\du}{11.014914\du}}
\pgfpathlineto{\pgfpoint{26.561628\du}{11.010168\du}}
\pgfpathlineto{\pgfpoint{26.574407\du}{11.005422\du}}
\pgfpathlineto{\pgfpoint{26.585725\du}{10.999945\du}}
\pgfpathlineto{\pgfpoint{26.597408\du}{10.994834\du}}
\pgfpathlineto{\pgfpoint{26.608726\du}{10.989723\du}}
\pgfpathlineto{\pgfpoint{26.620044\du}{10.984246\du}}
\pgfpathlineto{\pgfpoint{26.630997\du}{10.979135\du}}
\pgfpathlineto{\pgfpoint{26.642315\du}{10.973293\du}}
\pgfpathlineto{\pgfpoint{26.652172\du}{10.967452\du}}
\pgfpathlineto{\pgfpoint{26.662395\du}{10.960880\du}}
\pgfpathlineto{\pgfpoint{26.672253\du}{10.955038\du}}
\pgfpathlineto{\pgfpoint{26.682475\du}{10.948467\du}}
\pgfpathlineto{\pgfpoint{26.691968\du}{10.942260\du}}
\pgfpathlineto{\pgfpoint{26.701095\du}{10.935688\du}}
\pgfpathlineto{\pgfpoint{26.709858\du}{10.929482\du}}
\pgfpathlineto{\pgfpoint{26.718255\du}{10.922180\du}}
\pgfpathlineto{\pgfpoint{26.726652\du}{10.915243\du}}
\pgfpathlineto{\pgfpoint{26.734684\du}{10.908306\du}}
\pgfpathlineto{\pgfpoint{26.742716\du}{10.901369\du}}
\pgfpathlineto{\pgfpoint{26.749653\du}{10.894067\du}}
\pgfpathlineto{\pgfpoint{26.757320\du}{10.886765\du}}
\pgfpathlineto{\pgfpoint{26.763892\du}{10.879098\du}}
\pgfpathlineto{\pgfpoint{26.770464\du}{10.871431\du}}
\pgfpathlineto{\pgfpoint{26.776670\du}{10.863764\du}}
\pgfpathlineto{\pgfpoint{26.782877\du}{10.855732\du}}
\pgfpathlineto{\pgfpoint{26.788719\du}{10.848065\du}}
\pgfpathlineto{\pgfpoint{26.793465\du}{10.839668\du}}
\pgfpathlineto{\pgfpoint{26.798211\du}{10.831636\du}}
\pgfpathlineto{\pgfpoint{26.802957\du}{10.823238\du}}
\pgfpathlineto{\pgfpoint{26.807338\du}{10.815206\du}}
\pgfpathlineto{\pgfpoint{26.810624\du}{10.806444\du}}
\pgfpathlineto{\pgfpoint{26.814640\du}{10.798412\du}}
\pgfpathlineto{\pgfpoint{26.817196\du}{10.789650\du}}
\pgfpathlineto{\pgfpoint{26.820117\du}{10.780887\du}}
\pgfpathlineto{\pgfpoint{26.821942\du}{10.771760\du}}
\pgfpathlineto{\pgfpoint{26.824133\du}{10.762997\du}}
\pgfpathlineto{\pgfpoint{26.825593\du}{10.754235\du}}
\pgfpathlineto{\pgfpoint{26.826689\du}{10.745108\du}}
\pgfpathlineto{\pgfpoint{26.827419\du}{10.736345\du}}
\pgfpathlineto{\pgfpoint{26.827419\du}{10.726853\du}}
\pgfpathlineto{\pgfpoint{26.807338\du}{10.726853\du}}
\pgfpathlineto{\pgfpoint{26.806608\du}{10.734885\du}}
\pgfpathlineto{\pgfpoint{26.806243\du}{10.743282\du}}
\pgfpathlineto{\pgfpoint{26.805513\du}{10.751314\du}}
\pgfpathlineto{\pgfpoint{26.804053\du}{10.758981\du}}
\pgfpathlineto{\pgfpoint{26.802592\du}{10.767379\du}}
\pgfpathlineto{\pgfpoint{26.800037\du}{10.775411\du}}
\pgfpathlineto{\pgfpoint{26.797846\du}{10.783078\du}}
\pgfpathlineto{\pgfpoint{26.794560\du}{10.790745\du}}
\pgfpathlineto{\pgfpoint{26.792004\du}{10.798412\du}}
\pgfpathlineto{\pgfpoint{26.788719\du}{10.806444\du}}
\pgfpathlineto{\pgfpoint{26.784702\du}{10.814111\du}}
\pgfpathlineto{\pgfpoint{26.779956\du}{10.821778\du}}
\pgfpathlineto{\pgfpoint{26.775940\du}{10.829445\du}}
\pgfpathlineto{\pgfpoint{26.770829\du}{10.836382\du}}
\pgfpathlineto{\pgfpoint{26.766083\du}{10.844049\du}}
\pgfpathlineto{\pgfpoint{26.760606\du}{10.850986\du}}
\pgfpathlineto{\pgfpoint{26.755495\du}{10.858653\du}}
\pgfpathlineto{\pgfpoint{26.748558\du}{10.865590\du}}
\pgfpathlineto{\pgfpoint{26.742716\du}{10.872526\du}}
\pgfpathlineto{\pgfpoint{26.735414\du}{10.879463\du}}
\pgfpathlineto{\pgfpoint{26.728478\du}{10.886765\du}}
\pgfpathlineto{\pgfpoint{26.721176\du}{10.892972\du}}
\pgfpathlineto{\pgfpoint{26.714239\du}{10.899909\du}}
\pgfpathlineto{\pgfpoint{26.705842\du}{10.906480\du}}
\pgfpathlineto{\pgfpoint{26.697809\du}{10.913052\du}}
\pgfpathlineto{\pgfpoint{26.688682\du}{10.919259\du}}
\pgfpathlineto{\pgfpoint{26.679920\du}{10.925100\du}}
\pgfpathlineto{\pgfpoint{26.670427\du}{10.931672\du}}
\pgfpathlineto{\pgfpoint{26.661300\du}{10.938244\du}}
\pgfpathlineto{\pgfpoint{26.652172\du}{10.943355\du}}
\pgfpathlineto{\pgfpoint{26.642315\du}{10.949197\du}}
\pgfpathlineto{\pgfpoint{26.632457\du}{10.955038\du}}
\pgfpathlineto{\pgfpoint{26.621869\du}{10.960515\du}}
\pgfpathlineto{\pgfpoint{26.611281\du}{10.966356\du}}
\pgfpathlineto{\pgfpoint{26.600694\du}{10.970737\du}}
\pgfpathlineto{\pgfpoint{26.589011\du}{10.976579\du}}
\pgfpathlineto{\pgfpoint{26.577693\du}{10.981325\du}}
\pgfpathlineto{\pgfpoint{26.566375\du}{10.986072\du}}
\pgfpathlineto{\pgfpoint{26.554691\du}{10.990818\du}}
\pgfpathlineto{\pgfpoint{26.542643\du}{10.995564\du}}
\pgfpathlineto{\pgfpoint{26.530230\du}{10.999580\du}}
\pgfpathlineto{\pgfpoint{26.518547\du}{11.004326\du}}
\pgfpathlineto{\pgfpoint{26.505769\du}{11.008342\du}}
\pgfpathlineto{\pgfpoint{26.492260\du}{11.011993\du}}
\pgfpathlineto{\pgfpoint{26.479482\du}{11.016009\du}}
\pgfpathlineto{\pgfpoint{26.466338\du}{11.019295\du}}
\pgfpathlineto{\pgfpoint{26.452829\du}{11.022946\du}}
\pgfpathlineto{\pgfpoint{26.439321\du}{11.025867\du}}
\pgfpathlineto{\pgfpoint{26.425447\du}{11.029518\du}}
\pgfpathlineto{\pgfpoint{26.411574\du}{11.031709\du}}
\pgfpathlineto{\pgfpoint{26.397700\du}{11.034629\du}}
\pgfpathlineto{\pgfpoint{26.383461\du}{11.036820\du}}
\pgfpathlineto{\pgfpoint{26.369222\du}{11.039376\du}}
\pgfpathlineto{\pgfpoint{26.354984\du}{11.041566\du}}
\pgfpathlineto{\pgfpoint{26.340380\du}{11.043392\du}}
\pgfpathlineto{\pgfpoint{26.325411\du}{11.045217\du}}
\pgfpathlineto{\pgfpoint{26.310077\du}{11.047043\du}}
\pgfpathlineto{\pgfpoint{26.295473\du}{11.048138\du}}
\pgfpathlineto{\pgfpoint{26.279774\du}{11.049233\du}}
\pgfpathlineto{\pgfpoint{26.264805\du}{11.050329\du}}
\pgfpathlineto{\pgfpoint{26.249471\du}{11.051059\du}}
\pgfpathlineto{\pgfpoint{26.233771\du}{11.051789\du}}
\pgfpathlineto{\pgfpoint{26.218437\du}{11.052154\du}}
\pgfpathlineto{\pgfpoint{26.202373\du}{11.052154\du}}
\pgfpathlineto{\pgfpoint{26.202373\du}{11.052154\du}}
\pgfpathlineto{\pgfpoint{26.202373\du}{11.052154\du}}
\pgfpathlineto{\pgfpoint{26.201278\du}{11.052154\du}}
\pgfpathlineto{\pgfpoint{26.200183\du}{11.052154\du}}
\pgfpathlineto{\pgfpoint{26.199452\du}{11.052884\du}}
\pgfpathlineto{\pgfpoint{26.197627\du}{11.052884\du}}
\pgfpathlineto{\pgfpoint{26.197262\du}{11.053249\du}}
\pgfpathlineto{\pgfpoint{26.196166\du}{11.053980\du}}
\pgfpathlineto{\pgfpoint{26.195801\du}{11.054710\du}}
\pgfpathlineto{\pgfpoint{26.195436\du}{11.055075\du}}
\pgfpathlineto{\pgfpoint{26.193976\du}{11.056900\du}}
\pgfpathlineto{\pgfpoint{26.192516\du}{11.058726\du}}
\pgfpathlineto{\pgfpoint{26.192516\du}{11.060551\du}}
\pgfpathlineto{\pgfpoint{26.192150\du}{11.062742\du}}
\pgfpathlineto{\pgfpoint{26.192516\du}{11.064567\du}}
\pgfpathlineto{\pgfpoint{26.192516\du}{11.066393\du}}
\pgfpathlineto{\pgfpoint{26.193976\du}{11.067853\du}}
\pgfpathlineto{\pgfpoint{26.195436\du}{11.069314\du}}
\pgfpathlineto{\pgfpoint{26.195801\du}{11.070409\du}}
\pgfpathlineto{\pgfpoint{26.196166\du}{11.070774\du}}
\pgfpathlineto{\pgfpoint{26.197262\du}{11.071504\du}}
\pgfpathlineto{\pgfpoint{26.197627\du}{11.071504\du}}
\pgfpathlineto{\pgfpoint{26.199452\du}{11.072234\du}}
\pgfpathlineto{\pgfpoint{26.200183\du}{11.072599\du}}
\pgfpathlineto{\pgfpoint{26.201278\du}{11.072599\du}}
\pgfpathlineto{\pgfpoint{26.202373\du}{11.072599\du}}
\pgfusepath{fill}
\pgfsetbuttcap
\pgfsetmiterjoin
\pgfsetdash{}{0pt}
\definecolor{dialinecolor}{rgb}{0.678431, 0.839216, 0.905882}
\pgfsetfillcolor{dialinecolor}
\pgfpathmoveto{\pgfpoint{25.577693\du}{10.726853\du}}
\pgfpathlineto{\pgfpoint{25.577693\du}{10.726853\du}}
\pgfpathlineto{\pgfpoint{25.577693\du}{10.735615\du}}
\pgfpathlineto{\pgfpoint{25.578423\du}{10.745108\du}}
\pgfpathlineto{\pgfpoint{25.579518\du}{10.754235\du}}
\pgfpathlineto{\pgfpoint{25.581709\du}{10.762997\du}}
\pgfpathlineto{\pgfpoint{25.582804\du}{10.771760\du}}
\pgfpathlineto{\pgfpoint{25.585360\du}{10.780887\du}}
\pgfpathlineto{\pgfpoint{25.587550\du}{10.789650\du}}
\pgfpathlineto{\pgfpoint{25.590836\du}{10.798412\du}}
\pgfpathlineto{\pgfpoint{25.594122\du}{10.806444\du}}
\pgfpathlineto{\pgfpoint{25.598138\du}{10.815206\du}}
\pgfpathlineto{\pgfpoint{25.602519\du}{10.823238\du}}
\pgfpathlineto{\pgfpoint{25.606900\du}{10.831636\du}}
\pgfpathlineto{\pgfpoint{25.611647\du}{10.839668\du}}
\pgfpathlineto{\pgfpoint{25.616758\du}{10.848065\du}}
\pgfpathlineto{\pgfpoint{25.622965\du}{10.855732\du}}
\pgfpathlineto{\pgfpoint{25.627711\du}{10.863764\du}}
\pgfpathlineto{\pgfpoint{25.634283\du}{10.871431\du}}
\pgfpathlineto{\pgfpoint{25.641219\du}{10.879098\du}}
\pgfpathlineto{\pgfpoint{25.647791\du}{10.886765\du}}
\pgfpathlineto{\pgfpoint{25.654728\du}{10.894067\du}}
\pgfpathlineto{\pgfpoint{25.662030\du}{10.901369\du}}
\pgfpathlineto{\pgfpoint{25.670792\du}{10.908306\du}}
\pgfpathlineto{\pgfpoint{25.677729\du}{10.915243\du}}
\pgfpathlineto{\pgfpoint{25.686857\du}{10.922180\du}}
\pgfpathlineto{\pgfpoint{25.694889\du}{10.929482\du}}
\pgfpathlineto{\pgfpoint{25.704016\du}{10.935688\du}}
\pgfpathlineto{\pgfpoint{25.713874\du}{10.942260\du}}
\pgfpathlineto{\pgfpoint{25.723001\du}{10.948467\du}}
\pgfpathlineto{\pgfpoint{25.732494\du}{10.955038\du}}
\pgfpathlineto{\pgfpoint{25.742351\du}{10.960880\du}}
\pgfpathlineto{\pgfpoint{25.752574\du}{10.967452\du}}
\pgfpathlineto{\pgfpoint{25.762797\du}{10.973293\du}}
\pgfpathlineto{\pgfpoint{25.773750\du}{10.979135\du}}
\pgfpathlineto{\pgfpoint{25.785068\du}{10.984246\du}}
\pgfpathlineto{\pgfpoint{25.796386\du}{10.989723\du}}
\pgfpathlineto{\pgfpoint{25.807704\du}{10.994834\du}}
\pgfpathlineto{\pgfpoint{25.819752\du}{10.999945\du}}
\pgfpathlineto{\pgfpoint{25.831070\du}{11.005422\du}}
\pgfpathlineto{\pgfpoint{25.843483\du}{11.010168\du}}
\pgfpathlineto{\pgfpoint{25.855531\du}{11.014914\du}}
\pgfpathlineto{\pgfpoint{25.867579\du}{11.018930\du}}
\pgfpathlineto{\pgfpoint{25.880723\du}{11.023677\du}}
\pgfpathlineto{\pgfpoint{25.893501\du}{11.027693\du}}
\pgfpathlineto{\pgfpoint{25.907010\du}{11.031709\du}}
\pgfpathlineto{\pgfpoint{25.920518\du}{11.035725\du}}
\pgfpathlineto{\pgfpoint{25.933297\du}{11.039376\du}}
\pgfpathlineto{\pgfpoint{25.946805\du}{11.042662\du}}
\pgfpathlineto{\pgfpoint{25.960679\du}{11.046313\du}}
\pgfpathlineto{\pgfpoint{25.975283\du}{11.049233\du}}
\pgfpathlineto{\pgfpoint{25.989887\du}{11.052154\du}}
\pgfpathlineto{\pgfpoint{26.003760\du}{11.054710\du}}
\pgfpathlineto{\pgfpoint{26.017999\du}{11.057631\du}}
\pgfpathlineto{\pgfpoint{26.032968\du}{11.059821\du}}
\pgfpathlineto{\pgfpoint{26.047937\du}{11.062012\du}}
\pgfpathlineto{\pgfpoint{26.062541\du}{11.063837\du}}
\pgfpathlineto{\pgfpoint{26.077145\du}{11.065663\du}}
\pgfpathlineto{\pgfpoint{26.092844\du}{11.067488\du}}
\pgfpathlineto{\pgfpoint{26.108543\du}{11.068583\du}}
\pgfpathlineto{\pgfpoint{26.123147\du}{11.069679\du}}
\pgfpathlineto{\pgfpoint{26.139211\du}{11.070774\du}}
\pgfpathlineto{\pgfpoint{26.154545\du}{11.071504\du}}
\pgfpathlineto{\pgfpoint{26.170610\du}{11.072234\du}}
\pgfpathlineto{\pgfpoint{26.186674\du}{11.072599\du}}
\pgfpathlineto{\pgfpoint{26.202373\du}{11.072599\du}}
\pgfpathlineto{\pgfpoint{26.202373\du}{11.052154\du}}
\pgfpathlineto{\pgfpoint{26.186674\du}{11.052154\du}}
\pgfpathlineto{\pgfpoint{26.171340\du}{11.051789\du}}
\pgfpathlineto{\pgfpoint{26.155276\du}{11.051059\du}}
\pgfpathlineto{\pgfpoint{26.140672\du}{11.050329\du}}
\pgfpathlineto{\pgfpoint{26.124973\du}{11.049233\du}}
\pgfpathlineto{\pgfpoint{26.110004\du}{11.048138\du}}
\pgfpathlineto{\pgfpoint{26.095035\du}{11.047043\du}}
\pgfpathlineto{\pgfpoint{26.080066\du}{11.045217\du}}
\pgfpathlineto{\pgfpoint{26.065097\du}{11.043392\du}}
\pgfpathlineto{\pgfpoint{26.050493\du}{11.041566\du}}
\pgfpathlineto{\pgfpoint{26.035889\du}{11.039376\du}}
\pgfpathlineto{\pgfpoint{26.021650\du}{11.036820\du}}
\pgfpathlineto{\pgfpoint{26.007777\du}{11.034629\du}}
\pgfpathlineto{\pgfpoint{25.993903\du}{11.031709\du}}
\pgfpathlineto{\pgfpoint{25.979664\du}{11.029518\du}}
\pgfpathlineto{\pgfpoint{25.965790\du}{11.025867\du}}
\pgfpathlineto{\pgfpoint{25.952282\du}{11.022946\du}}
\pgfpathlineto{\pgfpoint{25.939138\du}{11.019295\du}}
\pgfpathlineto{\pgfpoint{25.925630\du}{11.016009\du}}
\pgfpathlineto{\pgfpoint{25.912851\du}{11.011993\du}}
\pgfpathlineto{\pgfpoint{25.899708\du}{11.008342\du}}
\pgfpathlineto{\pgfpoint{25.886930\du}{11.004326\du}}
\pgfpathlineto{\pgfpoint{25.874881\du}{10.999580\du}}
\pgfpathlineto{\pgfpoint{25.862468\du}{10.995564\du}}
\pgfpathlineto{\pgfpoint{25.850055\du}{10.990818\du}}
\pgfpathlineto{\pgfpoint{25.838737\du}{10.986072\du}}
\pgfpathlineto{\pgfpoint{25.827054\du}{10.981325\du}}
\pgfpathlineto{\pgfpoint{25.816101\du}{10.976579\du}}
\pgfpathlineto{\pgfpoint{25.804783\du}{10.970737\du}}
\pgfpathlineto{\pgfpoint{25.794195\du}{10.966356\du}}
\pgfpathlineto{\pgfpoint{25.783242\du}{10.960515\du}}
\pgfpathlineto{\pgfpoint{25.773019\du}{10.955038\du}}
\pgfpathlineto{\pgfpoint{25.762797\du}{10.949197\du}}
\pgfpathlineto{\pgfpoint{25.752939\du}{10.943355\du}}
\pgfpathlineto{\pgfpoint{25.743447\du}{10.938244\du}}
\pgfpathlineto{\pgfpoint{25.733589\du}{10.931672\du}}
\pgfpathlineto{\pgfpoint{25.725192\du}{10.925100\du}}
\pgfpathlineto{\pgfpoint{25.716064\du}{10.919259\du}}
\pgfpathlineto{\pgfpoint{25.707667\du}{10.913052\du}}
\pgfpathlineto{\pgfpoint{25.699270\du}{10.906480\du}}
\pgfpathlineto{\pgfpoint{25.691238\du}{10.899909\du}}
\pgfpathlineto{\pgfpoint{25.683936\du}{10.892972\du}}
\pgfpathlineto{\pgfpoint{25.676269\du}{10.886765\du}}
\pgfpathlineto{\pgfpoint{25.669697\du}{10.879463\du}}
\pgfpathlineto{\pgfpoint{25.662760\du}{10.872526\du}}
\pgfpathlineto{\pgfpoint{25.656553\du}{10.865590\du}}
\pgfpathlineto{\pgfpoint{25.650347\du}{10.858653\du}}
\pgfpathlineto{\pgfpoint{25.644140\du}{10.850986\du}}
\pgfpathlineto{\pgfpoint{25.639029\du}{10.844049\du}}
\pgfpathlineto{\pgfpoint{25.633917\du}{10.836382\du}}
\pgfpathlineto{\pgfpoint{25.629171\du}{10.829445\du}}
\pgfpathlineto{\pgfpoint{25.624790\du}{10.821778\du}}
\pgfpathlineto{\pgfpoint{25.620409\du}{10.814111\du}}
\pgfpathlineto{\pgfpoint{25.616758\du}{10.806444\du}}
\pgfpathlineto{\pgfpoint{25.613472\du}{10.798412\du}}
\pgfpathlineto{\pgfpoint{25.610186\du}{10.790745\du}}
\pgfpathlineto{\pgfpoint{25.607265\du}{10.783078\du}}
\pgfpathlineto{\pgfpoint{25.604710\du}{10.775411\du}}
\pgfpathlineto{\pgfpoint{25.602884\du}{10.767379\du}}
\pgfpathlineto{\pgfpoint{25.601059\du}{10.758981\du}}
\pgfpathlineto{\pgfpoint{25.599963\du}{10.751314\du}}
\pgfpathlineto{\pgfpoint{25.598503\du}{10.743282\du}}
\pgfpathlineto{\pgfpoint{25.598138\du}{10.734885\du}}
\pgfpathlineto{\pgfpoint{25.598138\du}{10.726853\du}}
\pgfpathlineto{\pgfpoint{25.598138\du}{10.726853\du}}
\pgfpathlineto{\pgfpoint{25.598138\du}{10.726853\du}}
\pgfpathlineto{\pgfpoint{25.598138\du}{10.725758\du}}
\pgfpathlineto{\pgfpoint{25.598138\du}{10.724662\du}}
\pgfpathlineto{\pgfpoint{25.597773\du}{10.723202\du}}
\pgfpathlineto{\pgfpoint{25.597773\du}{10.722107\du}}
\pgfpathlineto{\pgfpoint{25.597408\du}{10.721742\du}}
\pgfpathlineto{\pgfpoint{25.596313\du}{10.720281\du}}
\pgfpathlineto{\pgfpoint{25.595582\du}{10.719916\du}}
\pgfpathlineto{\pgfpoint{25.595582\du}{10.719186\du}}
\pgfpathlineto{\pgfpoint{25.593392\du}{10.718091\du}}
\pgfpathlineto{\pgfpoint{25.591566\du}{10.717360\du}}
\pgfpathlineto{\pgfpoint{25.589741\du}{10.716995\du}}
\pgfpathlineto{\pgfpoint{25.587550\du}{10.716265\du}}
\pgfpathlineto{\pgfpoint{25.586090\du}{10.716995\du}}
\pgfpathlineto{\pgfpoint{25.584264\du}{10.717360\du}}
\pgfpathlineto{\pgfpoint{25.582439\du}{10.718091\du}}
\pgfpathlineto{\pgfpoint{25.580613\du}{10.719186\du}}
\pgfpathlineto{\pgfpoint{25.579883\du}{10.719916\du}}
\pgfpathlineto{\pgfpoint{25.579518\du}{10.720281\du}}
\pgfpathlineto{\pgfpoint{25.579153\du}{10.721742\du}}
\pgfpathlineto{\pgfpoint{25.578423\du}{10.722107\du}}
\pgfpathlineto{\pgfpoint{25.578423\du}{10.723202\du}}
\pgfpathlineto{\pgfpoint{25.577693\du}{10.724662\du}}
\pgfpathlineto{\pgfpoint{25.577693\du}{10.725758\du}}
\pgfpathlineto{\pgfpoint{25.577693\du}{10.726853\du}}
\pgfusepath{fill}
\pgfsetbuttcap
\pgfsetmiterjoin
\pgfsetdash{}{0pt}
\definecolor{dialinecolor}{rgb}{0.678431, 0.839216, 0.905882}
\pgfsetfillcolor{dialinecolor}
\pgfpathmoveto{\pgfpoint{26.202373\du}{10.381106\du}}
\pgfpathlineto{\pgfpoint{26.202373\du}{10.381106\du}}
\pgfpathlineto{\pgfpoint{26.186674\du}{10.381106\du}}
\pgfpathlineto{\pgfpoint{26.170610\du}{10.381471\du}}
\pgfpathlineto{\pgfpoint{26.154545\du}{10.382202\du}}
\pgfpathlineto{\pgfpoint{26.139211\du}{10.382932\du}}
\pgfpathlineto{\pgfpoint{26.123147\du}{10.384027\du}}
\pgfpathlineto{\pgfpoint{26.108543\du}{10.385122\du}}
\pgfpathlineto{\pgfpoint{26.092844\du}{10.386218\du}}
\pgfpathlineto{\pgfpoint{26.077145\du}{10.388043\du}}
\pgfpathlineto{\pgfpoint{26.062541\du}{10.389869\du}}
\pgfpathlineto{\pgfpoint{26.047937\du}{10.391694\du}}
\pgfpathlineto{\pgfpoint{26.032968\du}{10.393885\du}}
\pgfpathlineto{\pgfpoint{26.017999\du}{10.396440\du}}
\pgfpathlineto{\pgfpoint{26.003760\du}{10.399361\du}}
\pgfpathlineto{\pgfpoint{25.989887\du}{10.401552\du}}
\pgfpathlineto{\pgfpoint{25.975283\du}{10.404472\du}}
\pgfpathlineto{\pgfpoint{25.960679\du}{10.407393\du}}
\pgfpathlineto{\pgfpoint{25.946805\du}{10.411044\du}}
\pgfpathlineto{\pgfpoint{25.933297\du}{10.414330\du}}
\pgfpathlineto{\pgfpoint{25.920518\du}{10.418346\du}}
\pgfpathlineto{\pgfpoint{25.907010\du}{10.421997\du}}
\pgfpathlineto{\pgfpoint{25.893501\du}{10.426013\du}}
\pgfpathlineto{\pgfpoint{25.880723\du}{10.430394\du}}
\pgfpathlineto{\pgfpoint{25.867579\du}{10.434775\du}}
\pgfpathlineto{\pgfpoint{25.855531\du}{10.439157\du}}
\pgfpathlineto{\pgfpoint{25.843483\du}{10.443538\du}}
\pgfpathlineto{\pgfpoint{25.831070\du}{10.449014\du}}
\pgfpathlineto{\pgfpoint{25.819752\du}{10.453760\du}}
\pgfpathlineto{\pgfpoint{25.807704\du}{10.458872\du}}
\pgfpathlineto{\pgfpoint{25.796386\du}{10.463983\du}}
\pgfpathlineto{\pgfpoint{25.785068\du}{10.469460\du}}
\pgfpathlineto{\pgfpoint{25.773750\du}{10.475301\du}}
\pgfpathlineto{\pgfpoint{25.762797\du}{10.480413\du}}
\pgfpathlineto{\pgfpoint{25.752574\du}{10.486254\du}}
\pgfpathlineto{\pgfpoint{25.742351\du}{10.492826\du}}
\pgfpathlineto{\pgfpoint{25.732494\du}{10.498667\du}}
\pgfpathlineto{\pgfpoint{25.723001\du}{10.505239\du}}
\pgfpathlineto{\pgfpoint{25.713874\du}{10.511446\du}}
\pgfpathlineto{\pgfpoint{25.704016\du}{10.518018\du}}
\pgfpathlineto{\pgfpoint{25.694889\du}{10.524589\du}}
\pgfpathlineto{\pgfpoint{25.686857\du}{10.531526\du}}
\pgfpathlineto{\pgfpoint{25.677729\du}{10.538463\du}}
\pgfpathlineto{\pgfpoint{25.670792\du}{10.545400\du}}
\pgfpathlineto{\pgfpoint{25.662030\du}{10.552337\du}}
\pgfpathlineto{\pgfpoint{25.654728\du}{10.559639\du}}
\pgfpathlineto{\pgfpoint{25.647791\du}{10.567306\du}}
\pgfpathlineto{\pgfpoint{25.641219\du}{10.574608\du}}
\pgfpathlineto{\pgfpoint{25.634283\du}{10.582275\du}}
\pgfpathlineto{\pgfpoint{25.627711\du}{10.589942\du}}
\pgfpathlineto{\pgfpoint{25.622965\du}{10.597974\du}}
\pgfpathlineto{\pgfpoint{25.616758\du}{10.605641\du}}
\pgfpathlineto{\pgfpoint{25.611647\du}{10.614038\du}}
\pgfpathlineto{\pgfpoint{25.606900\du}{10.622070\du}}
\pgfpathlineto{\pgfpoint{25.602519\du}{10.630467\du}}
\pgfpathlineto{\pgfpoint{25.598138\du}{10.638499\du}}
\pgfpathlineto{\pgfpoint{25.594122\du}{10.647262\du}}
\pgfpathlineto{\pgfpoint{25.590836\du}{10.655659\du}}
\pgfpathlineto{\pgfpoint{25.587550\du}{10.664421\du}}
\pgfpathlineto{\pgfpoint{25.585360\du}{10.673184\du}}
\pgfpathlineto{\pgfpoint{25.582804\du}{10.681946\du}}
\pgfpathlineto{\pgfpoint{25.581709\du}{10.690708\du}}
\pgfpathlineto{\pgfpoint{25.579518\du}{10.699471\du}}
\pgfpathlineto{\pgfpoint{25.578423\du}{10.708598\du}}
\pgfpathlineto{\pgfpoint{25.577693\du}{10.718091\du}}
\pgfpathlineto{\pgfpoint{25.577693\du}{10.726853\du}}
\pgfpathlineto{\pgfpoint{25.598138\du}{10.726853\du}}
\pgfpathlineto{\pgfpoint{25.598138\du}{10.718821\du}}
\pgfpathlineto{\pgfpoint{25.598503\du}{10.710424\du}}
\pgfpathlineto{\pgfpoint{25.599963\du}{10.702391\du}}
\pgfpathlineto{\pgfpoint{25.601059\du}{10.694724\du}}
\pgfpathlineto{\pgfpoint{25.602884\du}{10.686327\du}}
\pgfpathlineto{\pgfpoint{25.604710\du}{10.678295\du}}
\pgfpathlineto{\pgfpoint{25.607265\du}{10.670628\du}}
\pgfpathlineto{\pgfpoint{25.610186\du}{10.662961\du}}
\pgfpathlineto{\pgfpoint{25.613472\du}{10.655659\du}}
\pgfpathlineto{\pgfpoint{25.616758\du}{10.647262\du}}
\pgfpathlineto{\pgfpoint{25.620409\du}{10.639595\du}}
\pgfpathlineto{\pgfpoint{25.624790\du}{10.631928\du}}
\pgfpathlineto{\pgfpoint{25.629171\du}{10.624991\du}}
\pgfpathlineto{\pgfpoint{25.633917\du}{10.617324\du}}
\pgfpathlineto{\pgfpoint{25.639029\du}{10.610022\du}}
\pgfpathlineto{\pgfpoint{25.644140\du}{10.602720\du}}
\pgfpathlineto{\pgfpoint{25.650347\du}{10.595053\du}}
\pgfpathlineto{\pgfpoint{25.656553\du}{10.588116\du}}
\pgfpathlineto{\pgfpoint{25.662760\du}{10.581179\du}}
\pgfpathlineto{\pgfpoint{25.669697\du}{10.574242\du}}
\pgfpathlineto{\pgfpoint{25.676269\du}{10.567671\du}}
\pgfpathlineto{\pgfpoint{25.683936\du}{10.560734\du}}
\pgfpathlineto{\pgfpoint{25.691238\du}{10.553797\du}}
\pgfpathlineto{\pgfpoint{25.699270\du}{10.547225\du}}
\pgfpathlineto{\pgfpoint{25.707667\du}{10.540654\du}}
\pgfpathlineto{\pgfpoint{25.716064\du}{10.534447\du}}
\pgfpathlineto{\pgfpoint{25.725192\du}{10.528605\du}}
\pgfpathlineto{\pgfpoint{25.733589\du}{10.522034\du}}
\pgfpathlineto{\pgfpoint{25.743447\du}{10.516192\du}}
\pgfpathlineto{\pgfpoint{25.752939\du}{10.510350\du}}
\pgfpathlineto{\pgfpoint{25.762797\du}{10.504509\du}}
\pgfpathlineto{\pgfpoint{25.773019\du}{10.498667\du}}
\pgfpathlineto{\pgfpoint{25.783242\du}{10.493556\du}}
\pgfpathlineto{\pgfpoint{25.794195\du}{10.487714\du}}
\pgfpathlineto{\pgfpoint{25.804783\du}{10.482968\du}}
\pgfpathlineto{\pgfpoint{25.816101\du}{10.477492\du}}
\pgfpathlineto{\pgfpoint{25.827054\du}{10.472380\du}}
\pgfpathlineto{\pgfpoint{25.838737\du}{10.467634\du}}
\pgfpathlineto{\pgfpoint{25.850055\du}{10.462888\du}}
\pgfpathlineto{\pgfpoint{25.862468\du}{10.458142\du}}
\pgfpathlineto{\pgfpoint{25.874881\du}{10.454126\du}}
\pgfpathlineto{\pgfpoint{25.886930\du}{10.449379\du}}
\pgfpathlineto{\pgfpoint{25.899708\du}{10.445363\du}}
\pgfpathlineto{\pgfpoint{25.912851\du}{10.442077\du}}
\pgfpathlineto{\pgfpoint{25.925630\du}{10.437696\du}}
\pgfpathlineto{\pgfpoint{25.939138\du}{10.434410\du}}
\pgfpathlineto{\pgfpoint{25.952282\du}{10.430759\du}}
\pgfpathlineto{\pgfpoint{25.965790\du}{10.427839\du}}
\pgfpathlineto{\pgfpoint{25.979664\du}{10.424918\du}}
\pgfpathlineto{\pgfpoint{25.993903\du}{10.421997\du}}
\pgfpathlineto{\pgfpoint{26.007777\du}{10.419076\du}}
\pgfpathlineto{\pgfpoint{26.021650\du}{10.416886\du}}
\pgfpathlineto{\pgfpoint{26.035889\du}{10.414330\du}}
\pgfpathlineto{\pgfpoint{26.050493\du}{10.412139\du}}
\pgfpathlineto{\pgfpoint{26.065097\du}{10.410314\du}}
\pgfpathlineto{\pgfpoint{26.080066\du}{10.408488\du}}
\pgfpathlineto{\pgfpoint{26.095035\du}{10.406663\du}}
\pgfpathlineto{\pgfpoint{26.110004\du}{10.405568\du}}
\pgfpathlineto{\pgfpoint{26.124973\du}{10.404472\du}}
\pgfpathlineto{\pgfpoint{26.140672\du}{10.403377\du}}
\pgfpathlineto{\pgfpoint{26.155276\du}{10.402647\du}}
\pgfpathlineto{\pgfpoint{26.171340\du}{10.402282\du}}
\pgfpathlineto{\pgfpoint{26.186674\du}{10.401552\du}}
\pgfpathlineto{\pgfpoint{26.202373\du}{10.401552\du}}
\pgfpathlineto{\pgfpoint{26.202373\du}{10.401552\du}}
\pgfpathlineto{\pgfpoint{26.202373\du}{10.401552\du}}
\pgfpathlineto{\pgfpoint{26.203834\du}{10.401552\du}}
\pgfpathlineto{\pgfpoint{26.204929\du}{10.401552\du}}
\pgfpathlineto{\pgfpoint{26.206024\du}{10.401552\du}}
\pgfpathlineto{\pgfpoint{26.207119\du}{10.400821\du}}
\pgfpathlineto{\pgfpoint{26.208215\du}{10.400456\du}}
\pgfpathlineto{\pgfpoint{26.209310\du}{10.399726\du}}
\pgfpathlineto{\pgfpoint{26.209310\du}{10.399361\du}}
\pgfpathlineto{\pgfpoint{26.210040\du}{10.398631\du}}
\pgfpathlineto{\pgfpoint{26.211135\du}{10.396805\du}}
\pgfpathlineto{\pgfpoint{26.212231\du}{10.394980\du}}
\pgfpathlineto{\pgfpoint{26.212596\du}{10.393520\du}}
\pgfpathlineto{\pgfpoint{26.212596\du}{10.391694\du}}
\pgfpathlineto{\pgfpoint{26.212596\du}{10.389138\du}}
\pgfpathlineto{\pgfpoint{26.212231\du}{10.387678\du}}
\pgfpathlineto{\pgfpoint{26.211135\du}{10.385853\du}}
\pgfpathlineto{\pgfpoint{26.210040\du}{10.384757\du}}
\pgfpathlineto{\pgfpoint{26.209310\du}{10.383297\du}}
\pgfpathlineto{\pgfpoint{26.209310\du}{10.382932\du}}
\pgfpathlineto{\pgfpoint{26.208215\du}{10.382202\du}}
\pgfpathlineto{\pgfpoint{26.207119\du}{10.382202\du}}
\pgfpathlineto{\pgfpoint{26.206024\du}{10.381471\du}}
\pgfpathlineto{\pgfpoint{26.204929\du}{10.381471\du}}
\pgfpathlineto{\pgfpoint{26.203834\du}{10.381106\du}}
\pgfpathlineto{\pgfpoint{26.202373\du}{10.381106\du}}
\pgfusepath{fill}
\pgfsetbuttcap
\pgfsetmiterjoin
\pgfsetdash{}{0pt}
\definecolor{dialinecolor}{rgb}{0.678431, 0.839216, 0.905882}
\pgfsetfillcolor{dialinecolor}
\pgfpathmoveto{\pgfpoint{26.827419\du}{10.726853\du}}
\pgfpathlineto{\pgfpoint{26.827419\du}{10.717360\du}}
\pgfpathlineto{\pgfpoint{26.826689\du}{10.708598\du}}
\pgfpathlineto{\pgfpoint{26.825593\du}{10.699471\du}}
\pgfpathlineto{\pgfpoint{26.824133\du}{10.690708\du}}
\pgfpathlineto{\pgfpoint{26.821942\du}{10.681946\du}}
\pgfpathlineto{\pgfpoint{26.820117\du}{10.673184\du}}
\pgfpathlineto{\pgfpoint{26.817196\du}{10.664421\du}}
\pgfpathlineto{\pgfpoint{26.814640\du}{10.655659\du}}
\pgfpathlineto{\pgfpoint{26.810624\du}{10.647262\du}}
\pgfpathlineto{\pgfpoint{26.807338\du}{10.638499\du}}
\pgfpathlineto{\pgfpoint{26.802957\du}{10.630467\du}}
\pgfpathlineto{\pgfpoint{26.798211\du}{10.622070\du}}
\pgfpathlineto{\pgfpoint{26.793465\du}{10.614038\du}}
\pgfpathlineto{\pgfpoint{26.788719\du}{10.605641\du}}
\pgfpathlineto{\pgfpoint{26.782877\du}{10.597974\du}}
\pgfpathlineto{\pgfpoint{26.776670\du}{10.589942\du}}
\pgfpathlineto{\pgfpoint{26.770464\du}{10.582275\du}}
\pgfpathlineto{\pgfpoint{26.763892\du}{10.574608\du}}
\pgfpathlineto{\pgfpoint{26.757320\du}{10.567306\du}}
\pgfpathlineto{\pgfpoint{26.749653\du}{10.559639\du}}
\pgfpathlineto{\pgfpoint{26.742716\du}{10.552337\du}}
\pgfpathlineto{\pgfpoint{26.734684\du}{10.545400\du}}
\pgfpathlineto{\pgfpoint{26.726652\du}{10.538463\du}}
\pgfpathlineto{\pgfpoint{26.718255\du}{10.531526\du}}
\pgfpathlineto{\pgfpoint{26.709858\du}{10.524589\du}}
\pgfpathlineto{\pgfpoint{26.701095\du}{10.518018\du}}
\pgfpathlineto{\pgfpoint{26.691968\du}{10.511446\du}}
\pgfpathlineto{\pgfpoint{26.682475\du}{10.505239\du}}
\pgfpathlineto{\pgfpoint{26.672253\du}{10.498667\du}}
\pgfpathlineto{\pgfpoint{26.662395\du}{10.492826\du}}
\pgfpathlineto{\pgfpoint{26.652172\du}{10.486254\du}}
\pgfpathlineto{\pgfpoint{26.642315\du}{10.480413\du}}
\pgfpathlineto{\pgfpoint{26.630997\du}{10.475301\du}}
\pgfpathlineto{\pgfpoint{26.620044\du}{10.469460\du}}
\pgfpathlineto{\pgfpoint{26.608726\du}{10.463983\du}}
\pgfpathlineto{\pgfpoint{26.597408\du}{10.458872\du}}
\pgfpathlineto{\pgfpoint{26.585725\du}{10.453760\du}}
\pgfpathlineto{\pgfpoint{26.574407\du}{10.449014\du}}
\pgfpathlineto{\pgfpoint{26.561628\du}{10.443538\du}}
\pgfpathlineto{\pgfpoint{26.549945\du}{10.439157\du}}
\pgfpathlineto{\pgfpoint{26.537167\du}{10.434775\du}}
\pgfpathlineto{\pgfpoint{26.524388\du}{10.430394\du}}
\pgfpathlineto{\pgfpoint{26.511245\du}{10.426013\du}}
\pgfpathlineto{\pgfpoint{26.498101\du}{10.421997\du}}
\pgfpathlineto{\pgfpoint{26.484958\du}{10.418346\du}}
\pgfpathlineto{\pgfpoint{26.471449\du}{10.414330\du}}
\pgfpathlineto{\pgfpoint{26.457576\du}{10.411044\du}}
\pgfpathlineto{\pgfpoint{26.444067\du}{10.407393\du}}
\pgfpathlineto{\pgfpoint{26.430194\du}{10.404472\du}}
\pgfpathlineto{\pgfpoint{26.415590\du}{10.401552\du}}
\pgfpathlineto{\pgfpoint{26.401716\du}{10.399361\du}}
\pgfpathlineto{\pgfpoint{26.387477\du}{10.396440\du}}
\pgfpathlineto{\pgfpoint{26.372143\du}{10.393885\du}}
\pgfpathlineto{\pgfpoint{26.357174\du}{10.391694\du}}
\pgfpathlineto{\pgfpoint{26.342205\du}{10.389869\du}}
\pgfpathlineto{\pgfpoint{26.327966\du}{10.388043\du}}
\pgfpathlineto{\pgfpoint{26.311902\du}{10.386218\du}}
\pgfpathlineto{\pgfpoint{26.296933\du}{10.385122\du}}
\pgfpathlineto{\pgfpoint{26.281599\du}{10.384027\du}}
\pgfpathlineto{\pgfpoint{26.265900\du}{10.382932\du}}
\pgfpathlineto{\pgfpoint{26.250566\du}{10.382202\du}}
\pgfpathlineto{\pgfpoint{26.234502\du}{10.381471\du}}
\pgfpathlineto{\pgfpoint{26.218437\du}{10.381106\du}}
\pgfpathlineto{\pgfpoint{26.202373\du}{10.381106\du}}
\pgfpathlineto{\pgfpoint{26.202373\du}{10.401552\du}}
\pgfpathlineto{\pgfpoint{26.218437\du}{10.401552\du}}
\pgfpathlineto{\pgfpoint{26.233771\du}{10.402282\du}}
\pgfpathlineto{\pgfpoint{26.249471\du}{10.402647\du}}
\pgfpathlineto{\pgfpoint{26.264805\du}{10.403377\du}}
\pgfpathlineto{\pgfpoint{26.279774\du}{10.404472\du}}
\pgfpathlineto{\pgfpoint{26.295473\du}{10.405568\du}}
\pgfpathlineto{\pgfpoint{26.310077\du}{10.406663\du}}
\pgfpathlineto{\pgfpoint{26.325411\du}{10.408488\du}}
\pgfpathlineto{\pgfpoint{26.340380\du}{10.410314\du}}
\pgfpathlineto{\pgfpoint{26.354984\du}{10.412139\du}}
\pgfpathlineto{\pgfpoint{26.369222\du}{10.414330\du}}
\pgfpathlineto{\pgfpoint{26.383461\du}{10.416886\du}}
\pgfpathlineto{\pgfpoint{26.397700\du}{10.419076\du}}
\pgfpathlineto{\pgfpoint{26.411574\du}{10.421997\du}}
\pgfpathlineto{\pgfpoint{26.425447\du}{10.424918\du}}
\pgfpathlineto{\pgfpoint{26.439321\du}{10.427839\du}}
\pgfpathlineto{\pgfpoint{26.452829\du}{10.430759\du}}
\pgfpathlineto{\pgfpoint{26.466338\du}{10.434410\du}}
\pgfpathlineto{\pgfpoint{26.479482\du}{10.437696\du}}
\pgfpathlineto{\pgfpoint{26.492260\du}{10.442077\du}}
\pgfpathlineto{\pgfpoint{26.505769\du}{10.445363\du}}
\pgfpathlineto{\pgfpoint{26.518547\du}{10.449379\du}}
\pgfpathlineto{\pgfpoint{26.530230\du}{10.454126\du}}
\pgfpathlineto{\pgfpoint{26.542643\du}{10.458142\du}}
\pgfpathlineto{\pgfpoint{26.554691\du}{10.462888\du}}
\pgfpathlineto{\pgfpoint{26.566375\du}{10.467634\du}}
\pgfpathlineto{\pgfpoint{26.577693\du}{10.472380\du}}
\pgfpathlineto{\pgfpoint{26.589011\du}{10.477492\du}}
\pgfpathlineto{\pgfpoint{26.600694\du}{10.482968\du}}
\pgfpathlineto{\pgfpoint{26.611281\du}{10.487714\du}}
\pgfpathlineto{\pgfpoint{26.621869\du}{10.493556\du}}
\pgfpathlineto{\pgfpoint{26.632457\du}{10.498667\du}}
\pgfpathlineto{\pgfpoint{26.642315\du}{10.504509\du}}
\pgfpathlineto{\pgfpoint{26.652172\du}{10.510350\du}}
\pgfpathlineto{\pgfpoint{26.661300\du}{10.516192\du}}
\pgfpathlineto{\pgfpoint{26.670427\du}{10.522034\du}}
\pgfpathlineto{\pgfpoint{26.679920\du}{10.528605\du}}
\pgfpathlineto{\pgfpoint{26.688682\du}{10.534447\du}}
\pgfpathlineto{\pgfpoint{26.697809\du}{10.540654\du}}
\pgfpathlineto{\pgfpoint{26.705842\du}{10.547225\du}}
\pgfpathlineto{\pgfpoint{26.714239\du}{10.553797\du}}
\pgfpathlineto{\pgfpoint{26.721176\du}{10.560734\du}}
\pgfpathlineto{\pgfpoint{26.728478\du}{10.567671\du}}
\pgfpathlineto{\pgfpoint{26.735414\du}{10.574242\du}}
\pgfpathlineto{\pgfpoint{26.742716\du}{10.581179\du}}
\pgfpathlineto{\pgfpoint{26.748558\du}{10.588116\du}}
\pgfpathlineto{\pgfpoint{26.755495\du}{10.595053\du}}
\pgfpathlineto{\pgfpoint{26.760606\du}{10.602720\du}}
\pgfpathlineto{\pgfpoint{26.766083\du}{10.610022\du}}
\pgfpathlineto{\pgfpoint{26.770829\du}{10.617324\du}}
\pgfpathlineto{\pgfpoint{26.775940\du}{10.624991\du}}
\pgfpathlineto{\pgfpoint{26.779956\du}{10.631928\du}}
\pgfpathlineto{\pgfpoint{26.784702\du}{10.639595\du}}
\pgfpathlineto{\pgfpoint{26.788719\du}{10.647262\du}}
\pgfpathlineto{\pgfpoint{26.792004\du}{10.655659\du}}
\pgfpathlineto{\pgfpoint{26.794560\du}{10.662961\du}}
\pgfpathlineto{\pgfpoint{26.797846\du}{10.670628\du}}
\pgfpathlineto{\pgfpoint{26.800037\du}{10.678295\du}}
\pgfpathlineto{\pgfpoint{26.802592\du}{10.686327\du}}
\pgfpathlineto{\pgfpoint{26.804053\du}{10.694724\du}}
\pgfpathlineto{\pgfpoint{26.805513\du}{10.702391\du}}
\pgfpathlineto{\pgfpoint{26.806243\du}{10.710424\du}}
\pgfpathlineto{\pgfpoint{26.806608\du}{10.718821\du}}
\pgfpathlineto{\pgfpoint{26.807338\du}{10.726853\du}}
\pgfpathlineto{\pgfpoint{26.827419\du}{10.726853\du}}
\pgfusepath{fill}
\pgfsetbuttcap
\pgfsetmiterjoin
\pgfsetdash{}{0pt}
\definecolor{dialinecolor}{rgb}{0.074510, 0.082353, 0.086275}
\pgfsetfillcolor{dialinecolor}
\pgfpathmoveto{\pgfpoint{25.881453\du}{10.821048\du}}
\pgfpathlineto{\pgfpoint{26.108908\du}{10.592862\du}}
\pgfpathlineto{\pgfpoint{26.049032\du}{10.531891\du}}
\pgfpathlineto{\pgfpoint{26.229390\du}{10.531891\du}}
\pgfpathlineto{\pgfpoint{26.229390\du}{10.720281\du}}
\pgfpathlineto{\pgfpoint{26.169149\du}{10.660040\du}}
\pgfpathlineto{\pgfpoint{25.948996\du}{10.881289\du}}
\pgfpathlineto{\pgfpoint{25.881453\du}{10.821048\du}}
\pgfusepath{fill}
\pgfsetbuttcap
\pgfsetmiterjoin
\pgfsetdash{}{0pt}
\definecolor{dialinecolor}{rgb}{0.074510, 0.082353, 0.086275}
\pgfsetfillcolor{dialinecolor}
\pgfpathmoveto{\pgfpoint{26.149434\du}{10.941530\du}}
\pgfpathlineto{\pgfpoint{26.376524\du}{10.713344\du}}
\pgfpathlineto{\pgfpoint{26.315918\du}{10.653103\du}}
\pgfpathlineto{\pgfpoint{26.497006\du}{10.653103\du}}
\pgfpathlineto{\pgfpoint{26.497006\du}{10.841128\du}}
\pgfpathlineto{\pgfpoint{26.436400\du}{10.780887\du}}
\pgfpathlineto{\pgfpoint{26.215882\du}{11.001771\du}}
\pgfpathlineto{\pgfpoint{26.149434\du}{10.941530\du}}
\pgfusepath{fill}
\pgfsetbuttcap
\pgfsetmiterjoin
\pgfsetdash{}{0pt}
\definecolor{dialinecolor}{rgb}{1.000000, 1.000000, 1.000000}
\pgfsetfillcolor{dialinecolor}
\pgfpathmoveto{\pgfpoint{25.868310\du}{10.807539\du}}
\pgfpathlineto{\pgfpoint{26.095400\du}{10.579354\du}}
\pgfpathlineto{\pgfpoint{26.035889\du}{10.519113\du}}
\pgfpathlineto{\pgfpoint{26.215882\du}{10.519113\du}}
\pgfpathlineto{\pgfpoint{26.215882\du}{10.707138\du}}
\pgfpathlineto{\pgfpoint{26.156006\du}{10.646166\du}}
\pgfpathlineto{\pgfpoint{25.935487\du}{10.867780\du}}
\pgfpathlineto{\pgfpoint{25.868310\du}{10.807539\du}}
\pgfusepath{fill}
\pgfsetbuttcap
\pgfsetmiterjoin
\pgfsetdash{}{0pt}
\definecolor{dialinecolor}{rgb}{1.000000, 1.000000, 1.000000}
\pgfsetfillcolor{dialinecolor}
\pgfpathmoveto{\pgfpoint{26.135926\du}{10.928021\du}}
\pgfpathlineto{\pgfpoint{26.363016\du}{10.699836\du}}
\pgfpathlineto{\pgfpoint{26.302775\du}{10.639595\du}}
\pgfpathlineto{\pgfpoint{26.483133\du}{10.639595\du}}
\pgfpathlineto{\pgfpoint{26.483133\du}{10.827620\du}}
\pgfpathlineto{\pgfpoint{26.423622\du}{10.767379\du}}
\pgfpathlineto{\pgfpoint{26.202373\du}{10.988262\du}}
\pgfpathlineto{\pgfpoint{26.135926\du}{10.928021\du}}
\pgfusepath{fill}
\pgfsetlinewidth{0.000000\du}
\pgfsetdash{}{0pt}
\pgfsetdash{}{0pt}
\pgfsetbuttcap
\pgfsetmiterjoin
\pgfsetlinewidth{0.000000\du}
\pgfsetbuttcap
\pgfsetmiterjoin
\pgfsetdash{}{0pt}
\definecolor{dialinecolor}{rgb}{0.717647, 0.717647, 0.615686}
\pgfsetfillcolor{dialinecolor}
\pgfpathmoveto{\pgfpoint{34.599416\du}{7.674024\du}}
\pgfpathlineto{\pgfpoint{34.599416\du}{11.050000\du}}
\pgfpathlineto{\pgfpoint{36.410357\du}{11.050000\du}}
\pgfpathlineto{\pgfpoint{36.410357\du}{7.674024\du}}
\pgfpathlineto{\pgfpoint{34.599416\du}{7.674024\du}}
\pgfusepath{fill}
\pgfsetbuttcap
\pgfsetmiterjoin
\pgfsetdash{}{0pt}
\definecolor{dialinecolor}{rgb}{0.286275, 0.286275, 0.211765}
\pgfsetstrokecolor{dialinecolor}
\pgfpathmoveto{\pgfpoint{34.599416\du}{7.674024\du}}
\pgfpathlineto{\pgfpoint{34.599416\du}{11.050000\du}}
\pgfpathlineto{\pgfpoint{36.410357\du}{11.050000\du}}
\pgfpathlineto{\pgfpoint{36.410357\du}{7.674024\du}}
\pgfpathlineto{\pgfpoint{34.599416\du}{7.674024\du}}
\pgfusepath{stroke}
\pgfsetbuttcap
\pgfsetmiterjoin
\pgfsetdash{}{0pt}
\definecolor{dialinecolor}{rgb}{0.000000, 0.000000, 0.000000}
\pgfsetfillcolor{dialinecolor}
\pgfpathmoveto{\pgfpoint{35.180823\du}{8.054322\du}}
\pgfpathlineto{\pgfpoint{35.180823\du}{8.538987\du}}
\pgfpathlineto{\pgfpoint{35.888997\du}{8.538987\du}}
\pgfpathlineto{\pgfpoint{35.888997\du}{8.054322\du}}
\pgfpathlineto{\pgfpoint{35.180823\du}{8.054322\du}}
\pgfusepath{fill}
\pgfsetbuttcap
\pgfsetmiterjoin
\pgfsetdash{}{0pt}
\definecolor{dialinecolor}{rgb}{0.000000, 0.000000, 0.000000}
\pgfsetfillcolor{dialinecolor}
\pgfpathmoveto{\pgfpoint{35.097425\du}{8.886878\du}}
\pgfpathlineto{\pgfpoint{35.097425\du}{8.880206\du}}
\pgfpathlineto{\pgfpoint{35.096948\du}{8.874964\du}}
\pgfpathlineto{\pgfpoint{35.096471\du}{8.869245\du}}
\pgfpathlineto{\pgfpoint{35.095042\du}{8.863527\du}}
\pgfpathlineto{\pgfpoint{35.094089\du}{8.858284\du}}
\pgfpathlineto{\pgfpoint{35.092182\du}{8.852089\du}}
\pgfpathlineto{\pgfpoint{35.090753\du}{8.846847\du}}
\pgfpathlineto{\pgfpoint{35.087893\du}{8.841128\du}}
\pgfpathlineto{\pgfpoint{35.085511\du}{8.835409\du}}
\pgfpathlineto{\pgfpoint{35.083128\du}{8.830644\du}}
\pgfpathlineto{\pgfpoint{35.080268\du}{8.825878\du}}
\pgfpathlineto{\pgfpoint{35.077409\du}{8.820636\du}}
\pgfpathlineto{\pgfpoint{35.073596\du}{8.816347\du}}
\pgfpathlineto{\pgfpoint{35.070260\du}{8.811581\du}}
\pgfpathlineto{\pgfpoint{35.065971\du}{8.807292\du}}
\pgfpathlineto{\pgfpoint{35.061682\du}{8.803003\du}}
\pgfpathlineto{\pgfpoint{35.057870\du}{8.799667\du}}
\pgfpathlineto{\pgfpoint{35.053581\du}{8.795378\du}}
\pgfpathlineto{\pgfpoint{35.048815\du}{8.792042\du}}
\pgfpathlineto{\pgfpoint{35.044049\du}{8.788706\du}}
\pgfpathlineto{\pgfpoint{35.038331\du}{8.785847\du}}
\pgfpathlineto{\pgfpoint{35.034042\du}{8.783464\du}}
\pgfpathlineto{\pgfpoint{35.028323\du}{8.781081\du}}
\pgfpathlineto{\pgfpoint{35.023081\du}{8.778698\du}}
\pgfpathlineto{\pgfpoint{35.016885\du}{8.776792\du}}
\pgfpathlineto{\pgfpoint{35.011643\du}{8.774886\du}}
\pgfpathlineto{\pgfpoint{35.006401\du}{8.773933\du}}
\pgfpathlineto{\pgfpoint{35.000206\du}{8.772503\du}}
\pgfpathlineto{\pgfpoint{34.994487\du}{8.772026\du}}
\pgfpathlineto{\pgfpoint{34.989245\du}{8.771550\du}}
\pgfpathlineto{\pgfpoint{34.983526\du}{8.771550\du}}
\pgfpathlineto{\pgfpoint{34.983526\du}{8.771550\du}}
\pgfpathlineto{\pgfpoint{34.977331\du}{8.771550\du}}
\pgfpathlineto{\pgfpoint{34.971612\du}{8.772026\du}}
\pgfpathlineto{\pgfpoint{34.965893\du}{8.772503\du}}
\pgfpathlineto{\pgfpoint{34.959698\du}{8.773933\du}}
\pgfpathlineto{\pgfpoint{34.954456\du}{8.774886\du}}
\pgfpathlineto{\pgfpoint{34.949213\du}{8.776792\du}}
\pgfpathlineto{\pgfpoint{34.943018\du}{8.778698\du}}
\pgfpathlineto{\pgfpoint{34.937776\du}{8.781081\du}}
\pgfpathlineto{\pgfpoint{34.933010\du}{8.783464\du}}
\pgfpathlineto{\pgfpoint{34.927768\du}{8.785847\du}}
\pgfpathlineto{\pgfpoint{34.922049\du}{8.788706\du}}
\pgfpathlineto{\pgfpoint{34.917760\du}{8.792042\du}}
\pgfpathlineto{\pgfpoint{34.912518\du}{8.795378\du}}
\pgfpathlineto{\pgfpoint{34.909182\du}{8.799667\du}}
\pgfpathlineto{\pgfpoint{34.904416\du}{8.803003\du}}
\pgfpathlineto{\pgfpoint{34.900127\du}{8.807292\du}}
\pgfpathlineto{\pgfpoint{34.895838\du}{8.811581\du}}
\pgfpathlineto{\pgfpoint{34.892502\du}{8.816347\du}}
\pgfpathlineto{\pgfpoint{34.888690\du}{8.820636\du}}
\pgfpathlineto{\pgfpoint{34.885830\du}{8.825878\du}}
\pgfpathlineto{\pgfpoint{34.882971\du}{8.830644\du}}
\pgfpathlineto{\pgfpoint{34.880588\du}{8.835409\du}}
\pgfpathlineto{\pgfpoint{34.878205\du}{8.841128\du}}
\pgfpathlineto{\pgfpoint{34.876299\du}{8.846847\du}}
\pgfpathlineto{\pgfpoint{34.873916\du}{8.852089\du}}
\pgfpathlineto{\pgfpoint{34.872010\du}{8.858284\du}}
\pgfpathlineto{\pgfpoint{34.871057\du}{8.863527\du}}
\pgfpathlineto{\pgfpoint{34.870104\du}{8.869245\du}}
\pgfpathlineto{\pgfpoint{34.869151\du}{8.874964\du}}
\pgfpathlineto{\pgfpoint{34.868674\du}{8.880206\du}}
\pgfpathlineto{\pgfpoint{34.868674\du}{8.886878\du}}
\pgfpathlineto{\pgfpoint{34.868674\du}{8.886878\du}}
\pgfpathlineto{\pgfpoint{34.868674\du}{8.892120\du}}
\pgfpathlineto{\pgfpoint{34.869151\du}{8.898792\du}}
\pgfpathlineto{\pgfpoint{34.870104\du}{8.904034\du}}
\pgfpathlineto{\pgfpoint{34.871057\du}{8.910230\du}}
\pgfpathlineto{\pgfpoint{34.872010\du}{8.915472\du}}
\pgfpathlineto{\pgfpoint{34.873916\du}{8.921667\du}}
\pgfpathlineto{\pgfpoint{34.876299\du}{8.926910\du}}
\pgfpathlineto{\pgfpoint{34.878205\du}{8.932152\du}}
\pgfpathlineto{\pgfpoint{34.880588\du}{8.937870\du}}
\pgfpathlineto{\pgfpoint{34.882971\du}{8.942636\du}}
\pgfpathlineto{\pgfpoint{34.885830\du}{8.947878\du}}
\pgfpathlineto{\pgfpoint{34.888690\du}{8.952644\du}}
\pgfpathlineto{\pgfpoint{34.892502\du}{8.957410\du}}
\pgfpathlineto{\pgfpoint{34.895838\du}{8.962175\du}}
\pgfpathlineto{\pgfpoint{34.900127\du}{8.966464\du}}
\pgfpathlineto{\pgfpoint{34.904416\du}{8.970753\du}}
\pgfpathlineto{\pgfpoint{34.909182\du}{8.974089\du}}
\pgfpathlineto{\pgfpoint{34.912518\du}{8.978378\du}}
\pgfpathlineto{\pgfpoint{34.917760\du}{8.981238\du}}
\pgfpathlineto{\pgfpoint{34.922049\du}{8.984574\du}}
\pgfpathlineto{\pgfpoint{34.927768\du}{8.987433\du}}
\pgfpathlineto{\pgfpoint{34.933010\du}{8.990292\du}}
\pgfpathlineto{\pgfpoint{34.937776\du}{8.992675\du}}
\pgfpathlineto{\pgfpoint{34.943018\du}{8.995058\du}}
\pgfpathlineto{\pgfpoint{34.949213\du}{8.996488\du}}
\pgfpathlineto{\pgfpoint{34.954456\du}{8.998871\du}}
\pgfpathlineto{\pgfpoint{34.959698\du}{8.999347\du}}
\pgfpathlineto{\pgfpoint{34.965893\du}{9.001253\du}}
\pgfpathlineto{\pgfpoint{34.971612\du}{9.001730\du}}
\pgfpathlineto{\pgfpoint{34.977331\du}{9.002207\du}}
\pgfpathlineto{\pgfpoint{34.983526\du}{9.002207\du}}
\pgfpathlineto{\pgfpoint{34.983526\du}{9.002207\du}}
\pgfpathlineto{\pgfpoint{34.989245\du}{9.002207\du}}
\pgfpathlineto{\pgfpoint{34.994487\du}{9.001730\du}}
\pgfpathlineto{\pgfpoint{35.000206\du}{9.001253\du}}
\pgfpathlineto{\pgfpoint{35.006401\du}{8.999347\du}}
\pgfpathlineto{\pgfpoint{35.011643\du}{8.998871\du}}
\pgfpathlineto{\pgfpoint{35.016885\du}{8.996488\du}}
\pgfpathlineto{\pgfpoint{35.023081\du}{8.995058\du}}
\pgfpathlineto{\pgfpoint{35.028323\du}{8.992675\du}}
\pgfpathlineto{\pgfpoint{35.034042\du}{8.990292\du}}
\pgfpathlineto{\pgfpoint{35.038331\du}{8.987433\du}}
\pgfpathlineto{\pgfpoint{35.044049\du}{8.984574\du}}
\pgfpathlineto{\pgfpoint{35.048815\du}{8.981238\du}}
\pgfpathlineto{\pgfpoint{35.053581\du}{8.978378\du}}
\pgfpathlineto{\pgfpoint{35.057870\du}{8.974089\du}}
\pgfpathlineto{\pgfpoint{35.061682\du}{8.970753\du}}
\pgfpathlineto{\pgfpoint{35.065971\du}{8.966464\du}}
\pgfpathlineto{\pgfpoint{35.070260\du}{8.962175\du}}
\pgfpathlineto{\pgfpoint{35.073596\du}{8.957410\du}}
\pgfpathlineto{\pgfpoint{35.077409\du}{8.952644\du}}
\pgfpathlineto{\pgfpoint{35.080268\du}{8.947878\du}}
\pgfpathlineto{\pgfpoint{35.083128\du}{8.942636\du}}
\pgfpathlineto{\pgfpoint{35.085511\du}{8.937870\du}}
\pgfpathlineto{\pgfpoint{35.087893\du}{8.932152\du}}
\pgfpathlineto{\pgfpoint{35.090753\du}{8.926910\du}}
\pgfpathlineto{\pgfpoint{35.092182\du}{8.921667\du}}
\pgfpathlineto{\pgfpoint{35.094089\du}{8.915472\du}}
\pgfpathlineto{\pgfpoint{35.095042\du}{8.910230\du}}
\pgfpathlineto{\pgfpoint{35.096471\du}{8.904034\du}}
\pgfpathlineto{\pgfpoint{35.096948\du}{8.898792\du}}
\pgfpathlineto{\pgfpoint{35.097425\du}{8.892120\du}}
\pgfpathlineto{\pgfpoint{35.097425\du}{8.886878\du}}
\pgfusepath{fill}
\pgfsetbuttcap
\pgfsetmiterjoin
\pgfsetdash{}{0pt}
\definecolor{dialinecolor}{rgb}{0.000000, 0.000000, 0.000000}
\pgfsetfillcolor{dialinecolor}
\pgfpathmoveto{\pgfpoint{36.138239\du}{8.886878\du}}
\pgfpathlineto{\pgfpoint{36.138239\du}{8.880206\du}}
\pgfpathlineto{\pgfpoint{36.137763\du}{8.874964\du}}
\pgfpathlineto{\pgfpoint{36.136810\du}{8.869245\du}}
\pgfpathlineto{\pgfpoint{36.136333\du}{8.863527\du}}
\pgfpathlineto{\pgfpoint{36.134427\du}{8.858284\du}}
\pgfpathlineto{\pgfpoint{36.133474\du}{8.852089\du}}
\pgfpathlineto{\pgfpoint{36.131091\du}{8.846847\du}}
\pgfpathlineto{\pgfpoint{36.129185\du}{8.841128\du}}
\pgfpathlineto{\pgfpoint{36.126325\du}{8.835409\du}}
\pgfpathlineto{\pgfpoint{36.124419\du}{8.830644\du}}
\pgfpathlineto{\pgfpoint{36.121560\du}{8.825878\du}}
\pgfpathlineto{\pgfpoint{36.118224\du}{8.820636\du}}
\pgfpathlineto{\pgfpoint{36.114411\du}{8.816347\du}}
\pgfpathlineto{\pgfpoint{36.110599\du}{8.811581\du}}
\pgfpathlineto{\pgfpoint{36.106786\du}{8.807292\du}}
\pgfpathlineto{\pgfpoint{36.103450\du}{8.803003\du}}
\pgfpathlineto{\pgfpoint{36.098208\du}{8.799667\du}}
\pgfpathlineto{\pgfpoint{36.094395\du}{8.795378\du}}
\pgfpathlineto{\pgfpoint{36.089153\du}{8.792042\du}}
\pgfpathlineto{\pgfpoint{36.084388\du}{8.788706\du}}
\pgfpathlineto{\pgfpoint{36.079622\du}{8.785847\du}}
\pgfpathlineto{\pgfpoint{36.074380\du}{8.783464\du}}
\pgfpathlineto{\pgfpoint{36.069138\du}{8.781081\du}}
\pgfpathlineto{\pgfpoint{36.063419\du}{8.778698\du}}
\pgfpathlineto{\pgfpoint{36.058653\du}{8.776792\du}}
\pgfpathlineto{\pgfpoint{36.052934\du}{8.774886\du}}
\pgfpathlineto{\pgfpoint{36.046739\du}{8.773933\du}}
\pgfpathlineto{\pgfpoint{36.041497\du}{8.772503\du}}
\pgfpathlineto{\pgfpoint{36.035778\du}{8.772026\du}}
\pgfpathlineto{\pgfpoint{36.030059\du}{8.771550\du}}
\pgfpathlineto{\pgfpoint{36.024341\du}{8.771550\du}}
\pgfpathlineto{\pgfpoint{36.024341\du}{8.771550\du}}
\pgfpathlineto{\pgfpoint{36.018145\du}{8.771550\du}}
\pgfpathlineto{\pgfpoint{36.012903\du}{8.772026\du}}
\pgfpathlineto{\pgfpoint{36.007184\du}{8.772503\du}}
\pgfpathlineto{\pgfpoint{36.001942\du}{8.773933\du}}
\pgfpathlineto{\pgfpoint{35.996223\du}{8.774886\du}}
\pgfpathlineto{\pgfpoint{35.990505\du}{8.776792\du}}
\pgfpathlineto{\pgfpoint{35.984786\du}{8.778698\du}}
\pgfpathlineto{\pgfpoint{35.979544\du}{8.781081\du}}
\pgfpathlineto{\pgfpoint{35.974301\du}{8.783464\du}}
\pgfpathlineto{\pgfpoint{35.969536\du}{8.785847\du}}
\pgfpathlineto{\pgfpoint{35.963817\du}{8.788706\du}}
\pgfpathlineto{\pgfpoint{35.959051\du}{8.792042\du}}
\pgfpathlineto{\pgfpoint{35.954762\du}{8.795378\du}}
\pgfpathlineto{\pgfpoint{35.950473\du}{8.799667\du}}
\pgfpathlineto{\pgfpoint{35.945708\du}{8.803003\du}}
\pgfpathlineto{\pgfpoint{35.942372\du}{8.807292\du}}
\pgfpathlineto{\pgfpoint{35.937606\du}{8.811581\du}}
\pgfpathlineto{\pgfpoint{35.934270\du}{8.816347\du}}
\pgfpathlineto{\pgfpoint{35.930934\du}{8.820636\du}}
\pgfpathlineto{\pgfpoint{35.927598\du}{8.825878\du}}
\pgfpathlineto{\pgfpoint{35.924739\du}{8.830644\du}}
\pgfpathlineto{\pgfpoint{35.922356\du}{8.835409\du}}
\pgfpathlineto{\pgfpoint{35.919497\du}{8.841128\du}}
\pgfpathlineto{\pgfpoint{35.917590\du}{8.846847\du}}
\pgfpathlineto{\pgfpoint{35.915684\du}{8.852089\du}}
\pgfpathlineto{\pgfpoint{35.913778\du}{8.858284\du}}
\pgfpathlineto{\pgfpoint{35.912825\du}{8.863527\du}}
\pgfpathlineto{\pgfpoint{35.911395\du}{8.869245\du}}
\pgfpathlineto{\pgfpoint{35.910919\du}{8.874964\du}}
\pgfpathlineto{\pgfpoint{35.910442\du}{8.880206\du}}
\pgfpathlineto{\pgfpoint{35.910442\du}{8.886878\du}}
\pgfpathlineto{\pgfpoint{35.910442\du}{8.886878\du}}
\pgfpathlineto{\pgfpoint{35.910442\du}{8.892120\du}}
\pgfpathlineto{\pgfpoint{35.910919\du}{8.898792\du}}
\pgfpathlineto{\pgfpoint{35.911395\du}{8.904034\du}}
\pgfpathlineto{\pgfpoint{35.912825\du}{8.910230\du}}
\pgfpathlineto{\pgfpoint{35.913778\du}{8.915472\du}}
\pgfpathlineto{\pgfpoint{35.915684\du}{8.921667\du}}
\pgfpathlineto{\pgfpoint{35.917590\du}{8.926910\du}}
\pgfpathlineto{\pgfpoint{35.919497\du}{8.932152\du}}
\pgfpathlineto{\pgfpoint{35.922356\du}{8.937870\du}}
\pgfpathlineto{\pgfpoint{35.924739\du}{8.942636\du}}
\pgfpathlineto{\pgfpoint{35.927598\du}{8.947878\du}}
\pgfpathlineto{\pgfpoint{35.930934\du}{8.952644\du}}
\pgfpathlineto{\pgfpoint{35.934270\du}{8.957410\du}}
\pgfpathlineto{\pgfpoint{35.937606\du}{8.962175\du}}
\pgfpathlineto{\pgfpoint{35.942372\du}{8.966464\du}}
\pgfpathlineto{\pgfpoint{35.945708\du}{8.970753\du}}
\pgfpathlineto{\pgfpoint{35.950473\du}{8.974089\du}}
\pgfpathlineto{\pgfpoint{35.954762\du}{8.978378\du}}
\pgfpathlineto{\pgfpoint{35.959051\du}{8.981238\du}}
\pgfpathlineto{\pgfpoint{35.963817\du}{8.984574\du}}
\pgfpathlineto{\pgfpoint{35.969536\du}{8.987433\du}}
\pgfpathlineto{\pgfpoint{35.974301\du}{8.990292\du}}
\pgfpathlineto{\pgfpoint{35.979544\du}{8.992675\du}}
\pgfpathlineto{\pgfpoint{35.984786\du}{8.995058\du}}
\pgfpathlineto{\pgfpoint{35.990505\du}{8.996488\du}}
\pgfpathlineto{\pgfpoint{35.996223\du}{8.998871\du}}
\pgfpathlineto{\pgfpoint{36.001942\du}{8.999347\du}}
\pgfpathlineto{\pgfpoint{36.007184\du}{9.001253\du}}
\pgfpathlineto{\pgfpoint{36.012903\du}{9.001730\du}}
\pgfpathlineto{\pgfpoint{36.018145\du}{9.002207\du}}
\pgfpathlineto{\pgfpoint{36.024341\du}{9.002207\du}}
\pgfpathlineto{\pgfpoint{36.024341\du}{9.002207\du}}
\pgfpathlineto{\pgfpoint{36.030059\du}{9.002207\du}}
\pgfpathlineto{\pgfpoint{36.035778\du}{9.001730\du}}
\pgfpathlineto{\pgfpoint{36.041497\du}{9.001253\du}}
\pgfpathlineto{\pgfpoint{36.046739\du}{8.999347\du}}
\pgfpathlineto{\pgfpoint{36.052934\du}{8.998871\du}}
\pgfpathlineto{\pgfpoint{36.058653\du}{8.996488\du}}
\pgfpathlineto{\pgfpoint{36.063419\du}{8.995058\du}}
\pgfpathlineto{\pgfpoint{36.069138\du}{8.992675\du}}
\pgfpathlineto{\pgfpoint{36.074380\du}{8.990292\du}}
\pgfpathlineto{\pgfpoint{36.079622\du}{8.987433\du}}
\pgfpathlineto{\pgfpoint{36.084388\du}{8.984574\du}}
\pgfpathlineto{\pgfpoint{36.089153\du}{8.981238\du}}
\pgfpathlineto{\pgfpoint{36.094395\du}{8.978378\du}}
\pgfpathlineto{\pgfpoint{36.098208\du}{8.974089\du}}
\pgfpathlineto{\pgfpoint{36.103450\du}{8.970753\du}}
\pgfpathlineto{\pgfpoint{36.106786\du}{8.966464\du}}
\pgfpathlineto{\pgfpoint{36.110599\du}{8.962175\du}}
\pgfpathlineto{\pgfpoint{36.114411\du}{8.957410\du}}
\pgfpathlineto{\pgfpoint{36.118224\du}{8.952644\du}}
\pgfpathlineto{\pgfpoint{36.121560\du}{8.947878\du}}
\pgfpathlineto{\pgfpoint{36.124419\du}{8.942636\du}}
\pgfpathlineto{\pgfpoint{36.126325\du}{8.937870\du}}
\pgfpathlineto{\pgfpoint{36.129185\du}{8.932152\du}}
\pgfpathlineto{\pgfpoint{36.131091\du}{8.926910\du}}
\pgfpathlineto{\pgfpoint{36.133474\du}{8.921667\du}}
\pgfpathlineto{\pgfpoint{36.134427\du}{8.915472\du}}
\pgfpathlineto{\pgfpoint{36.136333\du}{8.910230\du}}
\pgfpathlineto{\pgfpoint{36.136810\du}{8.904034\du}}
\pgfpathlineto{\pgfpoint{36.137763\du}{8.898792\du}}
\pgfpathlineto{\pgfpoint{36.138239\du}{8.892120\du}}
\pgfpathlineto{\pgfpoint{36.138239\du}{8.886878\du}}
\pgfusepath{fill}
\pgfsetlinewidth{0.000000\du}
\pgfsetbuttcap
\pgfsetmiterjoin
\pgfsetdash{}{0pt}
\definecolor{dialinecolor}{rgb}{0.000000, 0.000000, 0.000000}
\pgfsetstrokecolor{dialinecolor}
\pgfpathmoveto{\pgfpoint{34.994010\du}{8.877347\du}}
\pgfpathlineto{\pgfpoint{35.991934\du}{8.877347\du}}
\pgfusepath{stroke}
\pgfsetlinewidth{0.000000\du}
\pgfsetbuttcap
\pgfsetmiterjoin
\pgfsetdash{}{0pt}
\definecolor{dialinecolor}{rgb}{1.000000, 1.000000, 1.000000}
\pgfsetfillcolor{dialinecolor}
\pgfpathmoveto{\pgfpoint{35.097425\du}{9.309113\du}}
\pgfpathlineto{\pgfpoint{35.097425\du}{9.303395\du}}
\pgfpathlineto{\pgfpoint{35.096948\du}{9.297199\du}}
\pgfpathlineto{\pgfpoint{35.096471\du}{9.291957\du}}
\pgfpathlineto{\pgfpoint{35.095042\du}{9.286238\du}}
\pgfpathlineto{\pgfpoint{35.094089\du}{9.280520\du}}
\pgfpathlineto{\pgfpoint{35.092182\du}{9.274801\du}}
\pgfpathlineto{\pgfpoint{35.090753\du}{9.269082\du}}
\pgfpathlineto{\pgfpoint{35.087893\du}{9.263840\du}}
\pgfpathlineto{\pgfpoint{35.085511\du}{9.258598\du}}
\pgfpathlineto{\pgfpoint{35.083128\du}{9.253356\du}}
\pgfpathlineto{\pgfpoint{35.080268\du}{9.248590\du}}
\pgfpathlineto{\pgfpoint{35.077409\du}{9.243348\du}}
\pgfpathlineto{\pgfpoint{35.073596\du}{9.238582\du}}
\pgfpathlineto{\pgfpoint{35.070260\du}{9.233816\du}}
\pgfpathlineto{\pgfpoint{35.065971\du}{9.230004\du}}
\pgfpathlineto{\pgfpoint{35.061682\du}{9.225715\du}}
\pgfpathlineto{\pgfpoint{35.057870\du}{9.221902\du}}
\pgfpathlineto{\pgfpoint{35.053581\du}{9.217613\du}}
\pgfpathlineto{\pgfpoint{35.048815\du}{9.214754\du}}
\pgfpathlineto{\pgfpoint{35.044049\du}{9.210941\du}}
\pgfpathlineto{\pgfpoint{35.038331\du}{9.208082\du}}
\pgfpathlineto{\pgfpoint{35.034042\du}{9.205699\du}}
\pgfpathlineto{\pgfpoint{35.028323\du}{9.203316\du}}
\pgfpathlineto{\pgfpoint{35.023081\du}{9.200934\du}}
\pgfpathlineto{\pgfpoint{35.016885\du}{9.199027\du}}
\pgfpathlineto{\pgfpoint{35.011643\du}{9.197598\du}}
\pgfpathlineto{\pgfpoint{35.006401\du}{9.196168\du}}
\pgfpathlineto{\pgfpoint{35.000206\du}{9.195215\du}}
\pgfpathlineto{\pgfpoint{34.994487\du}{9.194262\du}}
\pgfpathlineto{\pgfpoint{34.989245\du}{9.193785\du}}
\pgfpathlineto{\pgfpoint{34.983526\du}{9.193785\du}}
\pgfpathlineto{\pgfpoint{34.983526\du}{9.193785\du}}
\pgfpathlineto{\pgfpoint{34.977331\du}{9.193785\du}}
\pgfpathlineto{\pgfpoint{34.971612\du}{9.194262\du}}
\pgfpathlineto{\pgfpoint{34.965893\du}{9.195215\du}}
\pgfpathlineto{\pgfpoint{34.959698\du}{9.196168\du}}
\pgfpathlineto{\pgfpoint{34.954456\du}{9.197598\du}}
\pgfpathlineto{\pgfpoint{34.949213\du}{9.199027\du}}
\pgfpathlineto{\pgfpoint{34.943018\du}{9.200934\du}}
\pgfpathlineto{\pgfpoint{34.937776\du}{9.203316\du}}
\pgfpathlineto{\pgfpoint{34.933010\du}{9.205699\du}}
\pgfpathlineto{\pgfpoint{34.927768\du}{9.208082\du}}
\pgfpathlineto{\pgfpoint{34.922049\du}{9.210941\du}}
\pgfpathlineto{\pgfpoint{34.917760\du}{9.214754\du}}
\pgfpathlineto{\pgfpoint{34.912518\du}{9.217613\du}}
\pgfpathlineto{\pgfpoint{34.909182\du}{9.221902\du}}
\pgfpathlineto{\pgfpoint{34.904416\du}{9.225715\du}}
\pgfpathlineto{\pgfpoint{34.900127\du}{9.230004\du}}
\pgfpathlineto{\pgfpoint{34.895838\du}{9.233816\du}}
\pgfpathlineto{\pgfpoint{34.892502\du}{9.238582\du}}
\pgfpathlineto{\pgfpoint{34.888690\du}{9.243348\du}}
\pgfpathlineto{\pgfpoint{34.885830\du}{9.248590\du}}
\pgfpathlineto{\pgfpoint{34.882971\du}{9.253356\du}}
\pgfpathlineto{\pgfpoint{34.880588\du}{9.258598\du}}
\pgfpathlineto{\pgfpoint{34.878205\du}{9.263840\du}}
\pgfpathlineto{\pgfpoint{34.876299\du}{9.269082\du}}
\pgfpathlineto{\pgfpoint{34.873916\du}{9.274801\du}}
\pgfpathlineto{\pgfpoint{34.872010\du}{9.280520\du}}
\pgfpathlineto{\pgfpoint{34.871057\du}{9.286238\du}}
\pgfpathlineto{\pgfpoint{34.870104\du}{9.291957\du}}
\pgfpathlineto{\pgfpoint{34.869151\du}{9.297199\du}}
\pgfpathlineto{\pgfpoint{34.868674\du}{9.303395\du}}
\pgfpathlineto{\pgfpoint{34.868674\du}{9.309113\du}}
\pgfpathlineto{\pgfpoint{34.868674\du}{9.309113\du}}
\pgfpathlineto{\pgfpoint{34.868674\du}{9.315309\du}}
\pgfpathlineto{\pgfpoint{34.869151\du}{9.321028\du}}
\pgfpathlineto{\pgfpoint{34.870104\du}{9.326746\du}}
\pgfpathlineto{\pgfpoint{34.871057\du}{9.332942\du}}
\pgfpathlineto{\pgfpoint{34.872010\du}{9.338184\du}}
\pgfpathlineto{\pgfpoint{34.873916\du}{9.343903\du}}
\pgfpathlineto{\pgfpoint{34.876299\du}{9.349621\du}}
\pgfpathlineto{\pgfpoint{34.878205\du}{9.354864\du}}
\pgfpathlineto{\pgfpoint{34.880588\du}{9.360106\du}}
\pgfpathlineto{\pgfpoint{34.882971\du}{9.365825\du}}
\pgfpathlineto{\pgfpoint{34.885830\du}{9.370590\du}}
\pgfpathlineto{\pgfpoint{34.888690\du}{9.375356\du}}
\pgfpathlineto{\pgfpoint{34.892502\du}{9.380121\du}}
\pgfpathlineto{\pgfpoint{34.895838\du}{9.384887\du}}
\pgfpathlineto{\pgfpoint{34.900127\du}{9.389176\du}}
\pgfpathlineto{\pgfpoint{34.904416\du}{9.392512\du}}
\pgfpathlineto{\pgfpoint{34.909182\du}{9.396801\du}}
\pgfpathlineto{\pgfpoint{34.912518\du}{9.401090\du}}
\pgfpathlineto{\pgfpoint{34.917760\du}{9.403950\du}}
\pgfpathlineto{\pgfpoint{34.922049\du}{9.407762\du}}
\pgfpathlineto{\pgfpoint{34.927768\du}{9.410621\du}}
\pgfpathlineto{\pgfpoint{34.933010\du}{9.413004\du}}
\pgfpathlineto{\pgfpoint{34.937776\du}{9.415387\du}}
\pgfpathlineto{\pgfpoint{34.943018\du}{9.417770\du}}
\pgfpathlineto{\pgfpoint{34.949213\du}{9.419676\du}}
\pgfpathlineto{\pgfpoint{34.954456\du}{9.421106\du}}
\pgfpathlineto{\pgfpoint{34.959698\du}{9.422536\du}}
\pgfpathlineto{\pgfpoint{34.965893\du}{9.423965\du}}
\pgfpathlineto{\pgfpoint{34.971612\du}{9.424442\du}}
\pgfpathlineto{\pgfpoint{34.977331\du}{9.424918\du}}
\pgfpathlineto{\pgfpoint{34.983526\du}{9.424918\du}}
\pgfpathlineto{\pgfpoint{34.983526\du}{9.424918\du}}
\pgfpathlineto{\pgfpoint{34.989245\du}{9.424918\du}}
\pgfpathlineto{\pgfpoint{34.994487\du}{9.424442\du}}
\pgfpathlineto{\pgfpoint{35.000206\du}{9.423965\du}}
\pgfpathlineto{\pgfpoint{35.006401\du}{9.422536\du}}
\pgfpathlineto{\pgfpoint{35.011643\du}{9.421106\du}}
\pgfpathlineto{\pgfpoint{35.016885\du}{9.419676\du}}
\pgfpathlineto{\pgfpoint{35.023081\du}{9.417770\du}}
\pgfpathlineto{\pgfpoint{35.028323\du}{9.415387\du}}
\pgfpathlineto{\pgfpoint{35.034042\du}{9.413004\du}}
\pgfpathlineto{\pgfpoint{35.038331\du}{9.410621\du}}
\pgfpathlineto{\pgfpoint{35.044049\du}{9.407762\du}}
\pgfpathlineto{\pgfpoint{35.048815\du}{9.403950\du}}
\pgfpathlineto{\pgfpoint{35.053581\du}{9.401090\du}}
\pgfpathlineto{\pgfpoint{35.057870\du}{9.396801\du}}
\pgfpathlineto{\pgfpoint{35.061682\du}{9.392512\du}}
\pgfpathlineto{\pgfpoint{35.065971\du}{9.389176\du}}
\pgfpathlineto{\pgfpoint{35.070260\du}{9.384887\du}}
\pgfpathlineto{\pgfpoint{35.073596\du}{9.380121\du}}
\pgfpathlineto{\pgfpoint{35.077409\du}{9.375356\du}}
\pgfpathlineto{\pgfpoint{35.080268\du}{9.370590\du}}
\pgfpathlineto{\pgfpoint{35.083128\du}{9.365825\du}}
\pgfpathlineto{\pgfpoint{35.085511\du}{9.360106\du}}
\pgfpathlineto{\pgfpoint{35.087893\du}{9.354864\du}}
\pgfpathlineto{\pgfpoint{35.090753\du}{9.349621\du}}
\pgfpathlineto{\pgfpoint{35.092182\du}{9.343903\du}}
\pgfpathlineto{\pgfpoint{35.094089\du}{9.338184\du}}
\pgfpathlineto{\pgfpoint{35.095042\du}{9.332942\du}}
\pgfpathlineto{\pgfpoint{35.096471\du}{9.326746\du}}
\pgfpathlineto{\pgfpoint{35.096948\du}{9.321028\du}}
\pgfpathlineto{\pgfpoint{35.097425\du}{9.315309\du}}
\pgfpathlineto{\pgfpoint{35.097425\du}{9.309113\du}}
\pgfusepath{fill}
\pgfsetbuttcap
\pgfsetmiterjoin
\pgfsetdash{}{0pt}
\definecolor{dialinecolor}{rgb}{1.000000, 1.000000, 1.000000}
\pgfsetfillcolor{dialinecolor}
\pgfpathmoveto{\pgfpoint{36.138239\du}{9.309113\du}}
\pgfpathlineto{\pgfpoint{36.138239\du}{9.303395\du}}
\pgfpathlineto{\pgfpoint{36.137763\du}{9.297199\du}}
\pgfpathlineto{\pgfpoint{36.136810\du}{9.291957\du}}
\pgfpathlineto{\pgfpoint{36.136333\du}{9.286238\du}}
\pgfpathlineto{\pgfpoint{36.134427\du}{9.280520\du}}
\pgfpathlineto{\pgfpoint{36.133474\du}{9.274801\du}}
\pgfpathlineto{\pgfpoint{36.131091\du}{9.269082\du}}
\pgfpathlineto{\pgfpoint{36.129185\du}{9.263840\du}}
\pgfpathlineto{\pgfpoint{36.126325\du}{9.258598\du}}
\pgfpathlineto{\pgfpoint{36.124419\du}{9.253356\du}}
\pgfpathlineto{\pgfpoint{36.121560\du}{9.248590\du}}
\pgfpathlineto{\pgfpoint{36.118224\du}{9.243348\du}}
\pgfpathlineto{\pgfpoint{36.114411\du}{9.238582\du}}
\pgfpathlineto{\pgfpoint{36.110599\du}{9.233816\du}}
\pgfpathlineto{\pgfpoint{36.106786\du}{9.230004\du}}
\pgfpathlineto{\pgfpoint{36.103450\du}{9.225715\du}}
\pgfpathlineto{\pgfpoint{36.098208\du}{9.221902\du}}
\pgfpathlineto{\pgfpoint{36.094395\du}{9.217613\du}}
\pgfpathlineto{\pgfpoint{36.089153\du}{9.214754\du}}
\pgfpathlineto{\pgfpoint{36.084388\du}{9.210941\du}}
\pgfpathlineto{\pgfpoint{36.079622\du}{9.208082\du}}
\pgfpathlineto{\pgfpoint{36.074380\du}{9.205699\du}}
\pgfpathlineto{\pgfpoint{36.069138\du}{9.203316\du}}
\pgfpathlineto{\pgfpoint{36.063419\du}{9.200934\du}}
\pgfpathlineto{\pgfpoint{36.058653\du}{9.199027\du}}
\pgfpathlineto{\pgfpoint{36.052934\du}{9.197598\du}}
\pgfpathlineto{\pgfpoint{36.046739\du}{9.196168\du}}
\pgfpathlineto{\pgfpoint{36.041497\du}{9.195215\du}}
\pgfpathlineto{\pgfpoint{36.035778\du}{9.194262\du}}
\pgfpathlineto{\pgfpoint{36.030059\du}{9.193785\du}}
\pgfpathlineto{\pgfpoint{36.024341\du}{9.193785\du}}
\pgfpathlineto{\pgfpoint{36.024341\du}{9.193785\du}}
\pgfpathlineto{\pgfpoint{36.018145\du}{9.193785\du}}
\pgfpathlineto{\pgfpoint{36.012903\du}{9.194262\du}}
\pgfpathlineto{\pgfpoint{36.007184\du}{9.195215\du}}
\pgfpathlineto{\pgfpoint{36.001942\du}{9.196168\du}}
\pgfpathlineto{\pgfpoint{35.996223\du}{9.197598\du}}
\pgfpathlineto{\pgfpoint{35.990505\du}{9.199027\du}}
\pgfpathlineto{\pgfpoint{35.984786\du}{9.200934\du}}
\pgfpathlineto{\pgfpoint{35.979544\du}{9.203316\du}}
\pgfpathlineto{\pgfpoint{35.974301\du}{9.205699\du}}
\pgfpathlineto{\pgfpoint{35.969536\du}{9.208082\du}}
\pgfpathlineto{\pgfpoint{35.963817\du}{9.210941\du}}
\pgfpathlineto{\pgfpoint{35.959051\du}{9.214754\du}}
\pgfpathlineto{\pgfpoint{35.954762\du}{9.217613\du}}
\pgfpathlineto{\pgfpoint{35.950473\du}{9.221902\du}}
\pgfpathlineto{\pgfpoint{35.945708\du}{9.225715\du}}
\pgfpathlineto{\pgfpoint{35.942372\du}{9.230004\du}}
\pgfpathlineto{\pgfpoint{35.937606\du}{9.233816\du}}
\pgfpathlineto{\pgfpoint{35.934270\du}{9.238582\du}}
\pgfpathlineto{\pgfpoint{35.930934\du}{9.243348\du}}
\pgfpathlineto{\pgfpoint{35.927598\du}{9.248590\du}}
\pgfpathlineto{\pgfpoint{35.924739\du}{9.253356\du}}
\pgfpathlineto{\pgfpoint{35.922356\du}{9.258598\du}}
\pgfpathlineto{\pgfpoint{35.919497\du}{9.263840\du}}
\pgfpathlineto{\pgfpoint{35.917590\du}{9.269082\du}}
\pgfpathlineto{\pgfpoint{35.915684\du}{9.274801\du}}
\pgfpathlineto{\pgfpoint{35.913778\du}{9.280520\du}}
\pgfpathlineto{\pgfpoint{35.912825\du}{9.286238\du}}
\pgfpathlineto{\pgfpoint{35.911395\du}{9.291957\du}}
\pgfpathlineto{\pgfpoint{35.910919\du}{9.297199\du}}
\pgfpathlineto{\pgfpoint{35.910442\du}{9.303395\du}}
\pgfpathlineto{\pgfpoint{35.910442\du}{9.309113\du}}
\pgfpathlineto{\pgfpoint{35.910442\du}{9.309113\du}}
\pgfpathlineto{\pgfpoint{35.910442\du}{9.315309\du}}
\pgfpathlineto{\pgfpoint{35.910919\du}{9.321028\du}}
\pgfpathlineto{\pgfpoint{35.911395\du}{9.326746\du}}
\pgfpathlineto{\pgfpoint{35.912825\du}{9.332942\du}}
\pgfpathlineto{\pgfpoint{35.913778\du}{9.338184\du}}
\pgfpathlineto{\pgfpoint{35.915684\du}{9.343903\du}}
\pgfpathlineto{\pgfpoint{35.917590\du}{9.349621\du}}
\pgfpathlineto{\pgfpoint{35.919497\du}{9.354864\du}}
\pgfpathlineto{\pgfpoint{35.922356\du}{9.360106\du}}
\pgfpathlineto{\pgfpoint{35.924739\du}{9.365825\du}}
\pgfpathlineto{\pgfpoint{35.927598\du}{9.370590\du}}
\pgfpathlineto{\pgfpoint{35.930934\du}{9.375356\du}}
\pgfpathlineto{\pgfpoint{35.934270\du}{9.380121\du}}
\pgfpathlineto{\pgfpoint{35.937606\du}{9.384887\du}}
\pgfpathlineto{\pgfpoint{35.942372\du}{9.389176\du}}
\pgfpathlineto{\pgfpoint{35.945708\du}{9.392512\du}}
\pgfpathlineto{\pgfpoint{35.950473\du}{9.396801\du}}
\pgfpathlineto{\pgfpoint{35.954762\du}{9.401090\du}}
\pgfpathlineto{\pgfpoint{35.959051\du}{9.403950\du}}
\pgfpathlineto{\pgfpoint{35.963817\du}{9.407762\du}}
\pgfpathlineto{\pgfpoint{35.969536\du}{9.410621\du}}
\pgfpathlineto{\pgfpoint{35.974301\du}{9.413004\du}}
\pgfpathlineto{\pgfpoint{35.979544\du}{9.415387\du}}
\pgfpathlineto{\pgfpoint{35.984786\du}{9.417770\du}}
\pgfpathlineto{\pgfpoint{35.990505\du}{9.419676\du}}
\pgfpathlineto{\pgfpoint{35.996223\du}{9.421106\du}}
\pgfpathlineto{\pgfpoint{36.001942\du}{9.422536\du}}
\pgfpathlineto{\pgfpoint{36.007184\du}{9.423965\du}}
\pgfpathlineto{\pgfpoint{36.012903\du}{9.424442\du}}
\pgfpathlineto{\pgfpoint{36.018145\du}{9.424918\du}}
\pgfpathlineto{\pgfpoint{36.024341\du}{9.424918\du}}
\pgfpathlineto{\pgfpoint{36.024341\du}{9.424918\du}}
\pgfpathlineto{\pgfpoint{36.030059\du}{9.424918\du}}
\pgfpathlineto{\pgfpoint{36.035778\du}{9.424442\du}}
\pgfpathlineto{\pgfpoint{36.041497\du}{9.423965\du}}
\pgfpathlineto{\pgfpoint{36.046739\du}{9.422536\du}}
\pgfpathlineto{\pgfpoint{36.052934\du}{9.421106\du}}
\pgfpathlineto{\pgfpoint{36.058653\du}{9.419676\du}}
\pgfpathlineto{\pgfpoint{36.063419\du}{9.417770\du}}
\pgfpathlineto{\pgfpoint{36.069138\du}{9.415387\du}}
\pgfpathlineto{\pgfpoint{36.074380\du}{9.413004\du}}
\pgfpathlineto{\pgfpoint{36.079622\du}{9.410621\du}}
\pgfpathlineto{\pgfpoint{36.084388\du}{9.407762\du}}
\pgfpathlineto{\pgfpoint{36.089153\du}{9.403950\du}}
\pgfpathlineto{\pgfpoint{36.094395\du}{9.401090\du}}
\pgfpathlineto{\pgfpoint{36.098208\du}{9.396801\du}}
\pgfpathlineto{\pgfpoint{36.103450\du}{9.392512\du}}
\pgfpathlineto{\pgfpoint{36.106786\du}{9.389176\du}}
\pgfpathlineto{\pgfpoint{36.110599\du}{9.384887\du}}
\pgfpathlineto{\pgfpoint{36.114411\du}{9.380121\du}}
\pgfpathlineto{\pgfpoint{36.118224\du}{9.375356\du}}
\pgfpathlineto{\pgfpoint{36.121560\du}{9.370590\du}}
\pgfpathlineto{\pgfpoint{36.124419\du}{9.365825\du}}
\pgfpathlineto{\pgfpoint{36.126325\du}{9.360106\du}}
\pgfpathlineto{\pgfpoint{36.129185\du}{9.354864\du}}
\pgfpathlineto{\pgfpoint{36.131091\du}{9.349621\du}}
\pgfpathlineto{\pgfpoint{36.133474\du}{9.343903\du}}
\pgfpathlineto{\pgfpoint{36.134427\du}{9.338184\du}}
\pgfpathlineto{\pgfpoint{36.136333\du}{9.332942\du}}
\pgfpathlineto{\pgfpoint{36.136810\du}{9.326746\du}}
\pgfpathlineto{\pgfpoint{36.137763\du}{9.321028\du}}
\pgfpathlineto{\pgfpoint{36.138239\du}{9.315309\du}}
\pgfpathlineto{\pgfpoint{36.138239\du}{9.309113\du}}
\pgfusepath{fill}
\pgfsetlinewidth{0.000000\du}
\pgfsetbuttcap
\pgfsetmiterjoin
\pgfsetdash{}{0pt}
\definecolor{dialinecolor}{rgb}{0.000000, 0.000000, 0.000000}
\pgfsetstrokecolor{dialinecolor}
\pgfpathmoveto{\pgfpoint{34.994010\du}{9.299106\du}}
\pgfpathlineto{\pgfpoint{35.991934\du}{9.299106\du}}
\pgfusepath{stroke}
\pgfsetlinewidth{0.000000\du}
\pgfsetbuttcap
\pgfsetmiterjoin
\pgfsetdash{}{0pt}
\definecolor{dialinecolor}{rgb}{1.000000, 1.000000, 1.000000}
\pgfsetfillcolor{dialinecolor}
\pgfpathmoveto{\pgfpoint{35.097425\du}{9.731349\du}}
\pgfpathlineto{\pgfpoint{35.097425\du}{9.725630\du}}
\pgfpathlineto{\pgfpoint{35.096948\du}{9.719435\du}}
\pgfpathlineto{\pgfpoint{35.096471\du}{9.714192\du}}
\pgfpathlineto{\pgfpoint{35.095042\du}{9.707997\du}}
\pgfpathlineto{\pgfpoint{35.094089\du}{9.702278\du}}
\pgfpathlineto{\pgfpoint{35.092182\du}{9.696560\du}}
\pgfpathlineto{\pgfpoint{35.090753\du}{9.691317\du}}
\pgfpathlineto{\pgfpoint{35.087893\du}{9.686075\du}}
\pgfpathlineto{\pgfpoint{35.085511\du}{9.680356\du}}
\pgfpathlineto{\pgfpoint{35.083128\du}{9.675114\du}}
\pgfpathlineto{\pgfpoint{35.080268\du}{9.670349\du}}
\pgfpathlineto{\pgfpoint{35.077409\du}{9.665583\du}}
\pgfpathlineto{\pgfpoint{35.073596\du}{9.660817\du}}
\pgfpathlineto{\pgfpoint{35.070260\du}{9.656052\du}}
\pgfpathlineto{\pgfpoint{35.065971\du}{9.651763\du}}
\pgfpathlineto{\pgfpoint{35.061682\du}{9.647474\du}}
\pgfpathlineto{\pgfpoint{35.057870\du}{9.644138\du}}
\pgfpathlineto{\pgfpoint{35.053581\du}{9.639849\du}}
\pgfpathlineto{\pgfpoint{35.048815\du}{9.636989\du}}
\pgfpathlineto{\pgfpoint{35.044049\du}{9.633177\du}}
\pgfpathlineto{\pgfpoint{35.038331\du}{9.630317\du}}
\pgfpathlineto{\pgfpoint{35.034042\du}{9.627934\du}}
\pgfpathlineto{\pgfpoint{35.028323\du}{9.625552\du}}
\pgfpathlineto{\pgfpoint{35.023081\du}{9.623169\du}}
\pgfpathlineto{\pgfpoint{35.016885\du}{9.621263\du}}
\pgfpathlineto{\pgfpoint{35.011643\du}{9.618880\du}}
\pgfpathlineto{\pgfpoint{35.006401\du}{9.618403\du}}
\pgfpathlineto{\pgfpoint{35.000206\du}{9.616973\du}}
\pgfpathlineto{\pgfpoint{34.994487\du}{9.616020\du}}
\pgfpathlineto{\pgfpoint{34.989245\du}{9.616020\du}}
\pgfpathlineto{\pgfpoint{34.983526\du}{9.616020\du}}
\pgfpathlineto{\pgfpoint{34.983526\du}{9.616020\du}}
\pgfpathlineto{\pgfpoint{34.977331\du}{9.616020\du}}
\pgfpathlineto{\pgfpoint{34.971612\du}{9.616020\du}}
\pgfpathlineto{\pgfpoint{34.965893\du}{9.616973\du}}
\pgfpathlineto{\pgfpoint{34.959698\du}{9.618403\du}}
\pgfpathlineto{\pgfpoint{34.954456\du}{9.618880\du}}
\pgfpathlineto{\pgfpoint{34.949213\du}{9.621263\du}}
\pgfpathlineto{\pgfpoint{34.943018\du}{9.623169\du}}
\pgfpathlineto{\pgfpoint{34.937776\du}{9.625552\du}}
\pgfpathlineto{\pgfpoint{34.933010\du}{9.627934\du}}
\pgfpathlineto{\pgfpoint{34.927768\du}{9.630317\du}}
\pgfpathlineto{\pgfpoint{34.922049\du}{9.633177\du}}
\pgfpathlineto{\pgfpoint{34.917760\du}{9.636989\du}}
\pgfpathlineto{\pgfpoint{34.912518\du}{9.639849\du}}
\pgfpathlineto{\pgfpoint{34.909182\du}{9.644138\du}}
\pgfpathlineto{\pgfpoint{34.904416\du}{9.647474\du}}
\pgfpathlineto{\pgfpoint{34.900127\du}{9.651763\du}}
\pgfpathlineto{\pgfpoint{34.895838\du}{9.656052\du}}
\pgfpathlineto{\pgfpoint{34.892502\du}{9.660817\du}}
\pgfpathlineto{\pgfpoint{34.888690\du}{9.665583\du}}
\pgfpathlineto{\pgfpoint{34.885830\du}{9.670349\du}}
\pgfpathlineto{\pgfpoint{34.882971\du}{9.675114\du}}
\pgfpathlineto{\pgfpoint{34.880588\du}{9.680356\du}}
\pgfpathlineto{\pgfpoint{34.878205\du}{9.686075\du}}
\pgfpathlineto{\pgfpoint{34.876299\du}{9.691317\du}}
\pgfpathlineto{\pgfpoint{34.873916\du}{9.696560\du}}
\pgfpathlineto{\pgfpoint{34.872010\du}{9.702278\du}}
\pgfpathlineto{\pgfpoint{34.871057\du}{9.707997\du}}
\pgfpathlineto{\pgfpoint{34.870104\du}{9.714192\du}}
\pgfpathlineto{\pgfpoint{34.869151\du}{9.719435\du}}
\pgfpathlineto{\pgfpoint{34.868674\du}{9.725630\du}}
\pgfpathlineto{\pgfpoint{34.868674\du}{9.731349\du}}
\pgfpathlineto{\pgfpoint{34.868674\du}{9.731349\du}}
\pgfpathlineto{\pgfpoint{34.868674\du}{9.737544\du}}
\pgfpathlineto{\pgfpoint{34.869151\du}{9.743263\du}}
\pgfpathlineto{\pgfpoint{34.870104\du}{9.748982\du}}
\pgfpathlineto{\pgfpoint{34.871057\du}{9.754700\du}}
\pgfpathlineto{\pgfpoint{34.872010\du}{9.759943\du}}
\pgfpathlineto{\pgfpoint{34.873916\du}{9.766138\du}}
\pgfpathlineto{\pgfpoint{34.876299\du}{9.770904\du}}
\pgfpathlineto{\pgfpoint{34.878205\du}{9.777099\du}}
\pgfpathlineto{\pgfpoint{34.880588\du}{9.782341\du}}
\pgfpathlineto{\pgfpoint{34.882971\du}{9.787583\du}}
\pgfpathlineto{\pgfpoint{34.885830\du}{9.792349\du}}
\pgfpathlineto{\pgfpoint{34.888690\du}{9.797115\du}}
\pgfpathlineto{\pgfpoint{34.892502\du}{9.801880\du}}
\pgfpathlineto{\pgfpoint{34.895838\du}{9.806646\du}}
\pgfpathlineto{\pgfpoint{34.900127\du}{9.810935\du}}
\pgfpathlineto{\pgfpoint{34.904416\du}{9.815224\du}}
\pgfpathlineto{\pgfpoint{34.909182\du}{9.818560\du}}
\pgfpathlineto{\pgfpoint{34.912518\du}{9.822849\du}}
\pgfpathlineto{\pgfpoint{34.917760\du}{9.826185\du}}
\pgfpathlineto{\pgfpoint{34.922049\du}{9.829521\du}}
\pgfpathlineto{\pgfpoint{34.927768\du}{9.832380\du}}
\pgfpathlineto{\pgfpoint{34.933010\du}{9.834763\du}}
\pgfpathlineto{\pgfpoint{34.937776\du}{9.837146\du}}
\pgfpathlineto{\pgfpoint{34.943018\du}{9.839529\du}}
\pgfpathlineto{\pgfpoint{34.949213\du}{9.841435\du}}
\pgfpathlineto{\pgfpoint{34.954456\du}{9.843341\du}}
\pgfpathlineto{\pgfpoint{34.959698\du}{9.844294\du}}
\pgfpathlineto{\pgfpoint{34.965893\du}{9.845247\du}}
\pgfpathlineto{\pgfpoint{34.971612\du}{9.846201\du}}
\pgfpathlineto{\pgfpoint{34.977331\du}{9.846677\du}}
\pgfpathlineto{\pgfpoint{34.983526\du}{9.846677\du}}
\pgfpathlineto{\pgfpoint{34.983526\du}{9.846677\du}}
\pgfpathlineto{\pgfpoint{34.989245\du}{9.846677\du}}
\pgfpathlineto{\pgfpoint{34.994487\du}{9.846201\du}}
\pgfpathlineto{\pgfpoint{35.000206\du}{9.845247\du}}
\pgfpathlineto{\pgfpoint{35.006401\du}{9.844294\du}}
\pgfpathlineto{\pgfpoint{35.011643\du}{9.843341\du}}
\pgfpathlineto{\pgfpoint{35.016885\du}{9.841435\du}}
\pgfpathlineto{\pgfpoint{35.023081\du}{9.839529\du}}
\pgfpathlineto{\pgfpoint{35.028323\du}{9.837146\du}}
\pgfpathlineto{\pgfpoint{35.034042\du}{9.834763\du}}
\pgfpathlineto{\pgfpoint{35.038331\du}{9.832380\du}}
\pgfpathlineto{\pgfpoint{35.044049\du}{9.829521\du}}
\pgfpathlineto{\pgfpoint{35.048815\du}{9.826185\du}}
\pgfpathlineto{\pgfpoint{35.053581\du}{9.822849\du}}
\pgfpathlineto{\pgfpoint{35.057870\du}{9.818560\du}}
\pgfpathlineto{\pgfpoint{35.061682\du}{9.815224\du}}
\pgfpathlineto{\pgfpoint{35.065971\du}{9.810935\du}}
\pgfpathlineto{\pgfpoint{35.070260\du}{9.806646\du}}
\pgfpathlineto{\pgfpoint{35.073596\du}{9.801880\du}}
\pgfpathlineto{\pgfpoint{35.077409\du}{9.797115\du}}
\pgfpathlineto{\pgfpoint{35.080268\du}{9.792349\du}}
\pgfpathlineto{\pgfpoint{35.083128\du}{9.787583\du}}
\pgfpathlineto{\pgfpoint{35.085511\du}{9.782341\du}}
\pgfpathlineto{\pgfpoint{35.087893\du}{9.777099\du}}
\pgfpathlineto{\pgfpoint{35.090753\du}{9.770904\du}}
\pgfpathlineto{\pgfpoint{35.092182\du}{9.766138\du}}
\pgfpathlineto{\pgfpoint{35.094089\du}{9.759943\du}}
\pgfpathlineto{\pgfpoint{35.095042\du}{9.754700\du}}
\pgfpathlineto{\pgfpoint{35.096471\du}{9.748982\du}}
\pgfpathlineto{\pgfpoint{35.096948\du}{9.743263\du}}
\pgfpathlineto{\pgfpoint{35.097425\du}{9.737544\du}}
\pgfpathlineto{\pgfpoint{35.097425\du}{9.731349\du}}
\pgfusepath{fill}
\pgfsetbuttcap
\pgfsetmiterjoin
\pgfsetdash{}{0pt}
\definecolor{dialinecolor}{rgb}{1.000000, 1.000000, 1.000000}
\pgfsetfillcolor{dialinecolor}
\pgfpathmoveto{\pgfpoint{36.138239\du}{9.731349\du}}
\pgfpathlineto{\pgfpoint{36.138239\du}{9.725630\du}}
\pgfpathlineto{\pgfpoint{36.137763\du}{9.719435\du}}
\pgfpathlineto{\pgfpoint{36.136810\du}{9.714192\du}}
\pgfpathlineto{\pgfpoint{36.136333\du}{9.707997\du}}
\pgfpathlineto{\pgfpoint{36.134427\du}{9.702278\du}}
\pgfpathlineto{\pgfpoint{36.133474\du}{9.696560\du}}
\pgfpathlineto{\pgfpoint{36.131091\du}{9.691317\du}}
\pgfpathlineto{\pgfpoint{36.129185\du}{9.686075\du}}
\pgfpathlineto{\pgfpoint{36.126325\du}{9.680356\du}}
\pgfpathlineto{\pgfpoint{36.124419\du}{9.675114\du}}
\pgfpathlineto{\pgfpoint{36.121560\du}{9.670349\du}}
\pgfpathlineto{\pgfpoint{36.118224\du}{9.665583\du}}
\pgfpathlineto{\pgfpoint{36.114411\du}{9.660817\du}}
\pgfpathlineto{\pgfpoint{36.110599\du}{9.656052\du}}
\pgfpathlineto{\pgfpoint{36.106786\du}{9.651763\du}}
\pgfpathlineto{\pgfpoint{36.103450\du}{9.647474\du}}
\pgfpathlineto{\pgfpoint{36.098208\du}{9.644138\du}}
\pgfpathlineto{\pgfpoint{36.094395\du}{9.639849\du}}
\pgfpathlineto{\pgfpoint{36.089153\du}{9.636989\du}}
\pgfpathlineto{\pgfpoint{36.084388\du}{9.633177\du}}
\pgfpathlineto{\pgfpoint{36.079622\du}{9.630317\du}}
\pgfpathlineto{\pgfpoint{36.074380\du}{9.627934\du}}
\pgfpathlineto{\pgfpoint{36.069138\du}{9.625552\du}}
\pgfpathlineto{\pgfpoint{36.063419\du}{9.623169\du}}
\pgfpathlineto{\pgfpoint{36.058653\du}{9.621263\du}}
\pgfpathlineto{\pgfpoint{36.052934\du}{9.618880\du}}
\pgfpathlineto{\pgfpoint{36.046739\du}{9.618403\du}}
\pgfpathlineto{\pgfpoint{36.041497\du}{9.616973\du}}
\pgfpathlineto{\pgfpoint{36.035778\du}{9.616020\du}}
\pgfpathlineto{\pgfpoint{36.030059\du}{9.616020\du}}
\pgfpathlineto{\pgfpoint{36.024341\du}{9.616020\du}}
\pgfpathlineto{\pgfpoint{36.024341\du}{9.616020\du}}
\pgfpathlineto{\pgfpoint{36.018145\du}{9.616020\du}}
\pgfpathlineto{\pgfpoint{36.012903\du}{9.616020\du}}
\pgfpathlineto{\pgfpoint{36.007184\du}{9.616973\du}}
\pgfpathlineto{\pgfpoint{36.001942\du}{9.618403\du}}
\pgfpathlineto{\pgfpoint{35.996223\du}{9.618880\du}}
\pgfpathlineto{\pgfpoint{35.990505\du}{9.621263\du}}
\pgfpathlineto{\pgfpoint{35.984786\du}{9.623169\du}}
\pgfpathlineto{\pgfpoint{35.979544\du}{9.625552\du}}
\pgfpathlineto{\pgfpoint{35.974301\du}{9.627934\du}}
\pgfpathlineto{\pgfpoint{35.969536\du}{9.630317\du}}
\pgfpathlineto{\pgfpoint{35.963817\du}{9.633177\du}}
\pgfpathlineto{\pgfpoint{35.959051\du}{9.636989\du}}
\pgfpathlineto{\pgfpoint{35.954762\du}{9.639849\du}}
\pgfpathlineto{\pgfpoint{35.950473\du}{9.644138\du}}
\pgfpathlineto{\pgfpoint{35.945708\du}{9.647474\du}}
\pgfpathlineto{\pgfpoint{35.942372\du}{9.651763\du}}
\pgfpathlineto{\pgfpoint{35.937606\du}{9.656052\du}}
\pgfpathlineto{\pgfpoint{35.934270\du}{9.660817\du}}
\pgfpathlineto{\pgfpoint{35.930934\du}{9.665583\du}}
\pgfpathlineto{\pgfpoint{35.927598\du}{9.670349\du}}
\pgfpathlineto{\pgfpoint{35.924739\du}{9.675114\du}}
\pgfpathlineto{\pgfpoint{35.922356\du}{9.680356\du}}
\pgfpathlineto{\pgfpoint{35.919497\du}{9.686075\du}}
\pgfpathlineto{\pgfpoint{35.917590\du}{9.691317\du}}
\pgfpathlineto{\pgfpoint{35.915684\du}{9.696560\du}}
\pgfpathlineto{\pgfpoint{35.913778\du}{9.702278\du}}
\pgfpathlineto{\pgfpoint{35.912825\du}{9.707997\du}}
\pgfpathlineto{\pgfpoint{35.911395\du}{9.714192\du}}
\pgfpathlineto{\pgfpoint{35.910919\du}{9.719435\du}}
\pgfpathlineto{\pgfpoint{35.910442\du}{9.725630\du}}
\pgfpathlineto{\pgfpoint{35.910442\du}{9.731349\du}}
\pgfpathlineto{\pgfpoint{35.910442\du}{9.731349\du}}
\pgfpathlineto{\pgfpoint{35.910442\du}{9.737544\du}}
\pgfpathlineto{\pgfpoint{35.910919\du}{9.743263\du}}
\pgfpathlineto{\pgfpoint{35.911395\du}{9.748982\du}}
\pgfpathlineto{\pgfpoint{35.912825\du}{9.754700\du}}
\pgfpathlineto{\pgfpoint{35.913778\du}{9.759943\du}}
\pgfpathlineto{\pgfpoint{35.915684\du}{9.766138\du}}
\pgfpathlineto{\pgfpoint{35.917590\du}{9.770904\du}}
\pgfpathlineto{\pgfpoint{35.919497\du}{9.777099\du}}
\pgfpathlineto{\pgfpoint{35.922356\du}{9.782341\du}}
\pgfpathlineto{\pgfpoint{35.924739\du}{9.787583\du}}
\pgfpathlineto{\pgfpoint{35.927598\du}{9.792349\du}}
\pgfpathlineto{\pgfpoint{35.930934\du}{9.797115\du}}
\pgfpathlineto{\pgfpoint{35.934270\du}{9.801880\du}}
\pgfpathlineto{\pgfpoint{35.937606\du}{9.806646\du}}
\pgfpathlineto{\pgfpoint{35.942372\du}{9.810935\du}}
\pgfpathlineto{\pgfpoint{35.945708\du}{9.815224\du}}
\pgfpathlineto{\pgfpoint{35.950473\du}{9.818560\du}}
\pgfpathlineto{\pgfpoint{35.954762\du}{9.822849\du}}
\pgfpathlineto{\pgfpoint{35.959051\du}{9.826185\du}}
\pgfpathlineto{\pgfpoint{35.963817\du}{9.829521\du}}
\pgfpathlineto{\pgfpoint{35.969536\du}{9.832380\du}}
\pgfpathlineto{\pgfpoint{35.974301\du}{9.834763\du}}
\pgfpathlineto{\pgfpoint{35.979544\du}{9.837146\du}}
\pgfpathlineto{\pgfpoint{35.984786\du}{9.839529\du}}
\pgfpathlineto{\pgfpoint{35.990505\du}{9.841435\du}}
\pgfpathlineto{\pgfpoint{35.996223\du}{9.843341\du}}
\pgfpathlineto{\pgfpoint{36.001942\du}{9.844294\du}}
\pgfpathlineto{\pgfpoint{36.007184\du}{9.845247\du}}
\pgfpathlineto{\pgfpoint{36.012903\du}{9.846201\du}}
\pgfpathlineto{\pgfpoint{36.018145\du}{9.846677\du}}
\pgfpathlineto{\pgfpoint{36.024341\du}{9.846677\du}}
\pgfpathlineto{\pgfpoint{36.024341\du}{9.846677\du}}
\pgfpathlineto{\pgfpoint{36.030059\du}{9.846677\du}}
\pgfpathlineto{\pgfpoint{36.035778\du}{9.846201\du}}
\pgfpathlineto{\pgfpoint{36.041497\du}{9.845247\du}}
\pgfpathlineto{\pgfpoint{36.046739\du}{9.844294\du}}
\pgfpathlineto{\pgfpoint{36.052934\du}{9.843341\du}}
\pgfpathlineto{\pgfpoint{36.058653\du}{9.841435\du}}
\pgfpathlineto{\pgfpoint{36.063419\du}{9.839529\du}}
\pgfpathlineto{\pgfpoint{36.069138\du}{9.837146\du}}
\pgfpathlineto{\pgfpoint{36.074380\du}{9.834763\du}}
\pgfpathlineto{\pgfpoint{36.079622\du}{9.832380\du}}
\pgfpathlineto{\pgfpoint{36.084388\du}{9.829521\du}}
\pgfpathlineto{\pgfpoint{36.089153\du}{9.826185\du}}
\pgfpathlineto{\pgfpoint{36.094395\du}{9.822849\du}}
\pgfpathlineto{\pgfpoint{36.098208\du}{9.818560\du}}
\pgfpathlineto{\pgfpoint{36.103450\du}{9.815224\du}}
\pgfpathlineto{\pgfpoint{36.106786\du}{9.810935\du}}
\pgfpathlineto{\pgfpoint{36.110599\du}{9.806646\du}}
\pgfpathlineto{\pgfpoint{36.114411\du}{9.801880\du}}
\pgfpathlineto{\pgfpoint{36.118224\du}{9.797115\du}}
\pgfpathlineto{\pgfpoint{36.121560\du}{9.792349\du}}
\pgfpathlineto{\pgfpoint{36.124419\du}{9.787583\du}}
\pgfpathlineto{\pgfpoint{36.126325\du}{9.782341\du}}
\pgfpathlineto{\pgfpoint{36.129185\du}{9.777099\du}}
\pgfpathlineto{\pgfpoint{36.131091\du}{9.770904\du}}
\pgfpathlineto{\pgfpoint{36.133474\du}{9.766138\du}}
\pgfpathlineto{\pgfpoint{36.134427\du}{9.759943\du}}
\pgfpathlineto{\pgfpoint{36.136333\du}{9.754700\du}}
\pgfpathlineto{\pgfpoint{36.136810\du}{9.748982\du}}
\pgfpathlineto{\pgfpoint{36.137763\du}{9.743263\du}}
\pgfpathlineto{\pgfpoint{36.138239\du}{9.737544\du}}
\pgfpathlineto{\pgfpoint{36.138239\du}{9.731349\du}}
\pgfusepath{fill}
\pgfsetlinewidth{0.000000\du}
\pgfsetbuttcap
\pgfsetmiterjoin
\pgfsetdash{}{0pt}
\definecolor{dialinecolor}{rgb}{0.000000, 0.000000, 0.000000}
\pgfsetstrokecolor{dialinecolor}
\pgfpathmoveto{\pgfpoint{34.994010\du}{9.721341\du}}
\pgfpathlineto{\pgfpoint{35.991934\du}{9.721341\du}}
\pgfusepath{stroke}
\pgfsetlinewidth{0.000000\du}
\pgfsetbuttcap
\pgfsetmiterjoin
\pgfsetdash{}{0pt}
\definecolor{dialinecolor}{rgb}{1.000000, 1.000000, 1.000000}
\pgfsetfillcolor{dialinecolor}
\pgfpathmoveto{\pgfpoint{35.097425\du}{10.153584\du}}
\pgfpathlineto{\pgfpoint{35.097425\du}{10.147389\du}}
\pgfpathlineto{\pgfpoint{35.096948\du}{10.141193\du}}
\pgfpathlineto{\pgfpoint{35.096471\du}{10.135951\du}}
\pgfpathlineto{\pgfpoint{35.095042\du}{10.130232\du}}
\pgfpathlineto{\pgfpoint{35.094089\du}{10.124514\du}}
\pgfpathlineto{\pgfpoint{35.092182\du}{10.118795\du}}
\pgfpathlineto{\pgfpoint{35.090753\du}{10.113076\du}}
\pgfpathlineto{\pgfpoint{35.087893\du}{10.107357\du}}
\pgfpathlineto{\pgfpoint{35.085511\du}{10.102592\du}}
\pgfpathlineto{\pgfpoint{35.083128\du}{10.096873\du}}
\pgfpathlineto{\pgfpoint{35.080268\du}{10.092107\du}}
\pgfpathlineto{\pgfpoint{35.077409\du}{10.087342\du}}
\pgfpathlineto{\pgfpoint{35.073596\du}{10.082576\du}}
\pgfpathlineto{\pgfpoint{35.070260\du}{10.077810\du}}
\pgfpathlineto{\pgfpoint{35.065971\du}{10.073998\du}}
\pgfpathlineto{\pgfpoint{35.061682\du}{10.069709\du}}
\pgfpathlineto{\pgfpoint{35.057870\du}{10.065896\du}}
\pgfpathlineto{\pgfpoint{35.053581\du}{10.061607\du}}
\pgfpathlineto{\pgfpoint{35.048815\du}{10.058748\du}}
\pgfpathlineto{\pgfpoint{35.044049\du}{10.055412\du}}
\pgfpathlineto{\pgfpoint{35.038331\du}{10.052076\du}}
\pgfpathlineto{\pgfpoint{35.034042\du}{10.049693\du}}
\pgfpathlineto{\pgfpoint{35.028323\du}{10.047310\du}}
\pgfpathlineto{\pgfpoint{35.023081\du}{10.044928\du}}
\pgfpathlineto{\pgfpoint{35.016885\du}{10.043021\du}}
\pgfpathlineto{\pgfpoint{35.011643\du}{10.041592\du}}
\pgfpathlineto{\pgfpoint{35.006401\du}{10.040162\du}}
\pgfpathlineto{\pgfpoint{35.000206\du}{10.039209\du}}
\pgfpathlineto{\pgfpoint{34.994487\du}{10.038256\du}}
\pgfpathlineto{\pgfpoint{34.989245\du}{10.037779\du}}
\pgfpathlineto{\pgfpoint{34.983526\du}{10.037779\du}}
\pgfpathlineto{\pgfpoint{34.983526\du}{10.037779\du}}
\pgfpathlineto{\pgfpoint{34.977331\du}{10.037779\du}}
\pgfpathlineto{\pgfpoint{34.971612\du}{10.038256\du}}
\pgfpathlineto{\pgfpoint{34.965893\du}{10.039209\du}}
\pgfpathlineto{\pgfpoint{34.959698\du}{10.040162\du}}
\pgfpathlineto{\pgfpoint{34.954456\du}{10.041592\du}}
\pgfpathlineto{\pgfpoint{34.949213\du}{10.043021\du}}
\pgfpathlineto{\pgfpoint{34.943018\du}{10.044928\du}}
\pgfpathlineto{\pgfpoint{34.937776\du}{10.047310\du}}
\pgfpathlineto{\pgfpoint{34.933010\du}{10.049693\du}}
\pgfpathlineto{\pgfpoint{34.927768\du}{10.052076\du}}
\pgfpathlineto{\pgfpoint{34.922049\du}{10.055412\du}}
\pgfpathlineto{\pgfpoint{34.917760\du}{10.058748\du}}
\pgfpathlineto{\pgfpoint{34.912518\du}{10.061607\du}}
\pgfpathlineto{\pgfpoint{34.909182\du}{10.065896\du}}
\pgfpathlineto{\pgfpoint{34.904416\du}{10.069709\du}}
\pgfpathlineto{\pgfpoint{34.900127\du}{10.073998\du}}
\pgfpathlineto{\pgfpoint{34.895838\du}{10.077810\du}}
\pgfpathlineto{\pgfpoint{34.892502\du}{10.082576\du}}
\pgfpathlineto{\pgfpoint{34.888690\du}{10.087342\du}}
\pgfpathlineto{\pgfpoint{34.885830\du}{10.092107\du}}
\pgfpathlineto{\pgfpoint{34.882971\du}{10.096873\du}}
\pgfpathlineto{\pgfpoint{34.880588\du}{10.102592\du}}
\pgfpathlineto{\pgfpoint{34.878205\du}{10.107357\du}}
\pgfpathlineto{\pgfpoint{34.876299\du}{10.113076\du}}
\pgfpathlineto{\pgfpoint{34.873916\du}{10.118795\du}}
\pgfpathlineto{\pgfpoint{34.872010\du}{10.124514\du}}
\pgfpathlineto{\pgfpoint{34.871057\du}{10.130232\du}}
\pgfpathlineto{\pgfpoint{34.870104\du}{10.135951\du}}
\pgfpathlineto{\pgfpoint{34.869151\du}{10.141193\du}}
\pgfpathlineto{\pgfpoint{34.868674\du}{10.147389\du}}
\pgfpathlineto{\pgfpoint{34.868674\du}{10.153584\du}}
\pgfpathlineto{\pgfpoint{34.868674\du}{10.153584\du}}
\pgfpathlineto{\pgfpoint{34.868674\du}{10.159303\du}}
\pgfpathlineto{\pgfpoint{34.869151\du}{10.165498\du}}
\pgfpathlineto{\pgfpoint{34.870104\du}{10.170740\du}}
\pgfpathlineto{\pgfpoint{34.871057\du}{10.176936\du}}
\pgfpathlineto{\pgfpoint{34.872010\du}{10.182178\du}}
\pgfpathlineto{\pgfpoint{34.873916\du}{10.187897\du}}
\pgfpathlineto{\pgfpoint{34.876299\du}{10.193615\du}}
\pgfpathlineto{\pgfpoint{34.878205\du}{10.198858\du}}
\pgfpathlineto{\pgfpoint{34.880588\du}{10.204100\du}}
\pgfpathlineto{\pgfpoint{34.882971\du}{10.209819\du}}
\pgfpathlineto{\pgfpoint{34.885830\du}{10.214584\du}}
\pgfpathlineto{\pgfpoint{34.888690\du}{10.219350\du}}
\pgfpathlineto{\pgfpoint{34.892502\du}{10.223639\du}}
\pgfpathlineto{\pgfpoint{34.895838\du}{10.228881\du}}
\pgfpathlineto{\pgfpoint{34.900127\du}{10.233170\du}}
\pgfpathlineto{\pgfpoint{34.904416\du}{10.236983\du}}
\pgfpathlineto{\pgfpoint{34.909182\du}{10.240795\du}}
\pgfpathlineto{\pgfpoint{34.912518\du}{10.245084\du}}
\pgfpathlineto{\pgfpoint{34.917760\du}{10.247467\du}}
\pgfpathlineto{\pgfpoint{34.922049\du}{10.251756\du}}
\pgfpathlineto{\pgfpoint{34.927768\du}{10.254615\du}}
\pgfpathlineto{\pgfpoint{34.933010\du}{10.256998\du}}
\pgfpathlineto{\pgfpoint{34.937776\du}{10.259381\du}}
\pgfpathlineto{\pgfpoint{34.943018\du}{10.261764\du}}
\pgfpathlineto{\pgfpoint{34.949213\du}{10.263670\du}}
\pgfpathlineto{\pgfpoint{34.954456\du}{10.265100\du}}
\pgfpathlineto{\pgfpoint{34.959698\du}{10.266530\du}}
\pgfpathlineto{\pgfpoint{34.965893\du}{10.267483\du}}
\pgfpathlineto{\pgfpoint{34.971612\du}{10.268436\du}}
\pgfpathlineto{\pgfpoint{34.977331\du}{10.268912\du}}
\pgfpathlineto{\pgfpoint{34.983526\du}{10.268912\du}}
\pgfpathlineto{\pgfpoint{34.983526\du}{10.268912\du}}
\pgfpathlineto{\pgfpoint{34.989245\du}{10.268912\du}}
\pgfpathlineto{\pgfpoint{34.994487\du}{10.268436\du}}
\pgfpathlineto{\pgfpoint{35.000206\du}{10.267483\du}}
\pgfpathlineto{\pgfpoint{35.006401\du}{10.266530\du}}
\pgfpathlineto{\pgfpoint{35.011643\du}{10.265100\du}}
\pgfpathlineto{\pgfpoint{35.016885\du}{10.263670\du}}
\pgfpathlineto{\pgfpoint{35.023081\du}{10.261764\du}}
\pgfpathlineto{\pgfpoint{35.028323\du}{10.259381\du}}
\pgfpathlineto{\pgfpoint{35.034042\du}{10.256998\du}}
\pgfpathlineto{\pgfpoint{35.038331\du}{10.254615\du}}
\pgfpathlineto{\pgfpoint{35.044049\du}{10.251756\du}}
\pgfpathlineto{\pgfpoint{35.048815\du}{10.247467\du}}
\pgfpathlineto{\pgfpoint{35.053581\du}{10.245084\du}}
\pgfpathlineto{\pgfpoint{35.057870\du}{10.240795\du}}
\pgfpathlineto{\pgfpoint{35.061682\du}{10.236983\du}}
\pgfpathlineto{\pgfpoint{35.065971\du}{10.233170\du}}
\pgfpathlineto{\pgfpoint{35.070260\du}{10.228881\du}}
\pgfpathlineto{\pgfpoint{35.073596\du}{10.223639\du}}
\pgfpathlineto{\pgfpoint{35.077409\du}{10.219350\du}}
\pgfpathlineto{\pgfpoint{35.080268\du}{10.214584\du}}
\pgfpathlineto{\pgfpoint{35.083128\du}{10.209819\du}}
\pgfpathlineto{\pgfpoint{35.085511\du}{10.204100\du}}
\pgfpathlineto{\pgfpoint{35.087893\du}{10.198858\du}}
\pgfpathlineto{\pgfpoint{35.090753\du}{10.193615\du}}
\pgfpathlineto{\pgfpoint{35.092182\du}{10.187897\du}}
\pgfpathlineto{\pgfpoint{35.094089\du}{10.182178\du}}
\pgfpathlineto{\pgfpoint{35.095042\du}{10.176936\du}}
\pgfpathlineto{\pgfpoint{35.096471\du}{10.170740\du}}
\pgfpathlineto{\pgfpoint{35.096948\du}{10.165498\du}}
\pgfpathlineto{\pgfpoint{35.097425\du}{10.159303\du}}
\pgfpathlineto{\pgfpoint{35.097425\du}{10.153584\du}}
\pgfusepath{fill}
\pgfsetbuttcap
\pgfsetmiterjoin
\pgfsetdash{}{0pt}
\definecolor{dialinecolor}{rgb}{1.000000, 1.000000, 1.000000}
\pgfsetfillcolor{dialinecolor}
\pgfpathmoveto{\pgfpoint{36.138239\du}{10.153584\du}}
\pgfpathlineto{\pgfpoint{36.138239\du}{10.147389\du}}
\pgfpathlineto{\pgfpoint{36.137763\du}{10.141193\du}}
\pgfpathlineto{\pgfpoint{36.136810\du}{10.135951\du}}
\pgfpathlineto{\pgfpoint{36.136333\du}{10.130232\du}}
\pgfpathlineto{\pgfpoint{36.134427\du}{10.124514\du}}
\pgfpathlineto{\pgfpoint{36.133474\du}{10.118795\du}}
\pgfpathlineto{\pgfpoint{36.131091\du}{10.113076\du}}
\pgfpathlineto{\pgfpoint{36.129185\du}{10.107357\du}}
\pgfpathlineto{\pgfpoint{36.126325\du}{10.102592\du}}
\pgfpathlineto{\pgfpoint{36.124419\du}{10.096873\du}}
\pgfpathlineto{\pgfpoint{36.121560\du}{10.092107\du}}
\pgfpathlineto{\pgfpoint{36.118224\du}{10.087342\du}}
\pgfpathlineto{\pgfpoint{36.114411\du}{10.082576\du}}
\pgfpathlineto{\pgfpoint{36.110599\du}{10.077810\du}}
\pgfpathlineto{\pgfpoint{36.106786\du}{10.073998\du}}
\pgfpathlineto{\pgfpoint{36.103450\du}{10.069709\du}}
\pgfpathlineto{\pgfpoint{36.098208\du}{10.065896\du}}
\pgfpathlineto{\pgfpoint{36.094395\du}{10.061607\du}}
\pgfpathlineto{\pgfpoint{36.089153\du}{10.058748\du}}
\pgfpathlineto{\pgfpoint{36.084388\du}{10.055412\du}}
\pgfpathlineto{\pgfpoint{36.079622\du}{10.052076\du}}
\pgfpathlineto{\pgfpoint{36.074380\du}{10.049693\du}}
\pgfpathlineto{\pgfpoint{36.069138\du}{10.047310\du}}
\pgfpathlineto{\pgfpoint{36.063419\du}{10.044928\du}}
\pgfpathlineto{\pgfpoint{36.058653\du}{10.043021\du}}
\pgfpathlineto{\pgfpoint{36.052934\du}{10.041592\du}}
\pgfpathlineto{\pgfpoint{36.046739\du}{10.040162\du}}
\pgfpathlineto{\pgfpoint{36.041497\du}{10.039209\du}}
\pgfpathlineto{\pgfpoint{36.035778\du}{10.038256\du}}
\pgfpathlineto{\pgfpoint{36.030059\du}{10.037779\du}}
\pgfpathlineto{\pgfpoint{36.024341\du}{10.037779\du}}
\pgfpathlineto{\pgfpoint{36.024341\du}{10.037779\du}}
\pgfpathlineto{\pgfpoint{36.018145\du}{10.037779\du}}
\pgfpathlineto{\pgfpoint{36.012903\du}{10.038256\du}}
\pgfpathlineto{\pgfpoint{36.007184\du}{10.039209\du}}
\pgfpathlineto{\pgfpoint{36.001942\du}{10.040162\du}}
\pgfpathlineto{\pgfpoint{35.996223\du}{10.041592\du}}
\pgfpathlineto{\pgfpoint{35.990505\du}{10.043021\du}}
\pgfpathlineto{\pgfpoint{35.984786\du}{10.044928\du}}
\pgfpathlineto{\pgfpoint{35.979544\du}{10.047310\du}}
\pgfpathlineto{\pgfpoint{35.974301\du}{10.049693\du}}
\pgfpathlineto{\pgfpoint{35.969536\du}{10.052076\du}}
\pgfpathlineto{\pgfpoint{35.963817\du}{10.055412\du}}
\pgfpathlineto{\pgfpoint{35.959051\du}{10.058748\du}}
\pgfpathlineto{\pgfpoint{35.954762\du}{10.061607\du}}
\pgfpathlineto{\pgfpoint{35.950473\du}{10.065896\du}}
\pgfpathlineto{\pgfpoint{35.945708\du}{10.069709\du}}
\pgfpathlineto{\pgfpoint{35.942372\du}{10.073998\du}}
\pgfpathlineto{\pgfpoint{35.937606\du}{10.077810\du}}
\pgfpathlineto{\pgfpoint{35.934270\du}{10.082576\du}}
\pgfpathlineto{\pgfpoint{35.930934\du}{10.087342\du}}
\pgfpathlineto{\pgfpoint{35.927598\du}{10.092107\du}}
\pgfpathlineto{\pgfpoint{35.924739\du}{10.096873\du}}
\pgfpathlineto{\pgfpoint{35.922356\du}{10.102592\du}}
\pgfpathlineto{\pgfpoint{35.919497\du}{10.107357\du}}
\pgfpathlineto{\pgfpoint{35.917590\du}{10.113076\du}}
\pgfpathlineto{\pgfpoint{35.915684\du}{10.118795\du}}
\pgfpathlineto{\pgfpoint{35.913778\du}{10.124514\du}}
\pgfpathlineto{\pgfpoint{35.912825\du}{10.130232\du}}
\pgfpathlineto{\pgfpoint{35.911395\du}{10.135951\du}}
\pgfpathlineto{\pgfpoint{35.910919\du}{10.141193\du}}
\pgfpathlineto{\pgfpoint{35.910442\du}{10.147389\du}}
\pgfpathlineto{\pgfpoint{35.910442\du}{10.153584\du}}
\pgfpathlineto{\pgfpoint{35.910442\du}{10.153584\du}}
\pgfpathlineto{\pgfpoint{35.910442\du}{10.159303\du}}
\pgfpathlineto{\pgfpoint{35.910919\du}{10.165498\du}}
\pgfpathlineto{\pgfpoint{35.911395\du}{10.170740\du}}
\pgfpathlineto{\pgfpoint{35.912825\du}{10.176936\du}}
\pgfpathlineto{\pgfpoint{35.913778\du}{10.182178\du}}
\pgfpathlineto{\pgfpoint{35.915684\du}{10.187897\du}}
\pgfpathlineto{\pgfpoint{35.917590\du}{10.193615\du}}
\pgfpathlineto{\pgfpoint{35.919497\du}{10.198858\du}}
\pgfpathlineto{\pgfpoint{35.922356\du}{10.204100\du}}
\pgfpathlineto{\pgfpoint{35.924739\du}{10.209819\du}}
\pgfpathlineto{\pgfpoint{35.927598\du}{10.214584\du}}
\pgfpathlineto{\pgfpoint{35.930934\du}{10.219350\du}}
\pgfpathlineto{\pgfpoint{35.934270\du}{10.223639\du}}
\pgfpathlineto{\pgfpoint{35.937606\du}{10.228881\du}}
\pgfpathlineto{\pgfpoint{35.942372\du}{10.233170\du}}
\pgfpathlineto{\pgfpoint{35.945708\du}{10.236983\du}}
\pgfpathlineto{\pgfpoint{35.950473\du}{10.240795\du}}
\pgfpathlineto{\pgfpoint{35.954762\du}{10.245084\du}}
\pgfpathlineto{\pgfpoint{35.959051\du}{10.247467\du}}
\pgfpathlineto{\pgfpoint{35.963817\du}{10.251756\du}}
\pgfpathlineto{\pgfpoint{35.969536\du}{10.254615\du}}
\pgfpathlineto{\pgfpoint{35.974301\du}{10.256998\du}}
\pgfpathlineto{\pgfpoint{35.979544\du}{10.259381\du}}
\pgfpathlineto{\pgfpoint{35.984786\du}{10.261764\du}}
\pgfpathlineto{\pgfpoint{35.990505\du}{10.263670\du}}
\pgfpathlineto{\pgfpoint{35.996223\du}{10.265100\du}}
\pgfpathlineto{\pgfpoint{36.001942\du}{10.266530\du}}
\pgfpathlineto{\pgfpoint{36.007184\du}{10.267483\du}}
\pgfpathlineto{\pgfpoint{36.012903\du}{10.268436\du}}
\pgfpathlineto{\pgfpoint{36.018145\du}{10.268912\du}}
\pgfpathlineto{\pgfpoint{36.024341\du}{10.268912\du}}
\pgfpathlineto{\pgfpoint{36.024341\du}{10.268912\du}}
\pgfpathlineto{\pgfpoint{36.030059\du}{10.268912\du}}
\pgfpathlineto{\pgfpoint{36.035778\du}{10.268436\du}}
\pgfpathlineto{\pgfpoint{36.041497\du}{10.267483\du}}
\pgfpathlineto{\pgfpoint{36.046739\du}{10.266530\du}}
\pgfpathlineto{\pgfpoint{36.052934\du}{10.265100\du}}
\pgfpathlineto{\pgfpoint{36.058653\du}{10.263670\du}}
\pgfpathlineto{\pgfpoint{36.063419\du}{10.261764\du}}
\pgfpathlineto{\pgfpoint{36.069138\du}{10.259381\du}}
\pgfpathlineto{\pgfpoint{36.074380\du}{10.256998\du}}
\pgfpathlineto{\pgfpoint{36.079622\du}{10.254615\du}}
\pgfpathlineto{\pgfpoint{36.084388\du}{10.251756\du}}
\pgfpathlineto{\pgfpoint{36.089153\du}{10.247467\du}}
\pgfpathlineto{\pgfpoint{36.094395\du}{10.245084\du}}
\pgfpathlineto{\pgfpoint{36.098208\du}{10.240795\du}}
\pgfpathlineto{\pgfpoint{36.103450\du}{10.236983\du}}
\pgfpathlineto{\pgfpoint{36.106786\du}{10.233170\du}}
\pgfpathlineto{\pgfpoint{36.110599\du}{10.228881\du}}
\pgfpathlineto{\pgfpoint{36.114411\du}{10.223639\du}}
\pgfpathlineto{\pgfpoint{36.118224\du}{10.219350\du}}
\pgfpathlineto{\pgfpoint{36.121560\du}{10.214584\du}}
\pgfpathlineto{\pgfpoint{36.124419\du}{10.209819\du}}
\pgfpathlineto{\pgfpoint{36.126325\du}{10.204100\du}}
\pgfpathlineto{\pgfpoint{36.129185\du}{10.198858\du}}
\pgfpathlineto{\pgfpoint{36.131091\du}{10.193615\du}}
\pgfpathlineto{\pgfpoint{36.133474\du}{10.187897\du}}
\pgfpathlineto{\pgfpoint{36.134427\du}{10.182178\du}}
\pgfpathlineto{\pgfpoint{36.136333\du}{10.176936\du}}
\pgfpathlineto{\pgfpoint{36.136810\du}{10.170740\du}}
\pgfpathlineto{\pgfpoint{36.137763\du}{10.165498\du}}
\pgfpathlineto{\pgfpoint{36.138239\du}{10.159303\du}}
\pgfpathlineto{\pgfpoint{36.138239\du}{10.153584\du}}
\pgfusepath{fill}
\pgfsetlinewidth{0.000000\du}
\pgfsetbuttcap
\pgfsetmiterjoin
\pgfsetdash{}{0pt}
\definecolor{dialinecolor}{rgb}{0.000000, 0.000000, 0.000000}
\pgfsetstrokecolor{dialinecolor}
\pgfpathmoveto{\pgfpoint{34.994010\du}{10.142623\du}}
\pgfpathlineto{\pgfpoint{35.991934\du}{10.142623\du}}
\pgfusepath{stroke}
\pgfsetlinewidth{0.000000\du}
\pgfsetbuttcap
\pgfsetmiterjoin
\pgfsetdash{}{0pt}
\definecolor{dialinecolor}{rgb}{1.000000, 1.000000, 1.000000}
\pgfsetfillcolor{dialinecolor}
\pgfpathmoveto{\pgfpoint{35.097425\du}{10.574866\du}}
\pgfpathlineto{\pgfpoint{35.097425\du}{10.568671\du}}
\pgfpathlineto{\pgfpoint{35.096948\du}{10.562952\du}}
\pgfpathlineto{\pgfpoint{35.096471\du}{10.557233\du}}
\pgfpathlineto{\pgfpoint{35.095042\du}{10.551515\du}}
\pgfpathlineto{\pgfpoint{35.094089\du}{10.545319\du}}
\pgfpathlineto{\pgfpoint{35.092182\du}{10.540077\du}}
\pgfpathlineto{\pgfpoint{35.090753\du}{10.534835\du}}
\pgfpathlineto{\pgfpoint{35.087893\du}{10.528640\du}}
\pgfpathlineto{\pgfpoint{35.085511\du}{10.523874\du}}
\pgfpathlineto{\pgfpoint{35.083128\du}{10.518632\du}}
\pgfpathlineto{\pgfpoint{35.080268\du}{10.512436\du}}
\pgfpathlineto{\pgfpoint{35.077409\du}{10.508147\du}}
\pgfpathlineto{\pgfpoint{35.073596\du}{10.503382\du}}
\pgfpathlineto{\pgfpoint{35.070260\du}{10.499093\du}}
\pgfpathlineto{\pgfpoint{35.065971\du}{10.494327\du}}
\pgfpathlineto{\pgfpoint{35.061682\du}{10.490514\du}}
\pgfpathlineto{\pgfpoint{35.057870\du}{10.486702\du}}
\pgfpathlineto{\pgfpoint{35.053581\du}{10.483366\du}}
\pgfpathlineto{\pgfpoint{35.048815\du}{10.479553\du}}
\pgfpathlineto{\pgfpoint{35.044049\du}{10.476218\du}}
\pgfpathlineto{\pgfpoint{35.038331\du}{10.472882\du}}
\pgfpathlineto{\pgfpoint{35.034042\du}{10.470499\du}}
\pgfpathlineto{\pgfpoint{35.028323\du}{10.468116\du}}
\pgfpathlineto{\pgfpoint{35.023081\du}{10.465733\du}}
\pgfpathlineto{\pgfpoint{35.016885\du}{10.463827\du}}
\pgfpathlineto{\pgfpoint{35.011643\du}{10.461921\du}}
\pgfpathlineto{\pgfpoint{35.006401\du}{10.460967\du}}
\pgfpathlineto{\pgfpoint{35.000206\du}{10.460014\du}}
\pgfpathlineto{\pgfpoint{34.994487\du}{10.459061\du}}
\pgfpathlineto{\pgfpoint{34.989245\du}{10.458585\du}}
\pgfpathlineto{\pgfpoint{34.983526\du}{10.458585\du}}
\pgfpathlineto{\pgfpoint{34.983526\du}{10.458585\du}}
\pgfpathlineto{\pgfpoint{34.977331\du}{10.458585\du}}
\pgfpathlineto{\pgfpoint{34.971612\du}{10.459061\du}}
\pgfpathlineto{\pgfpoint{34.965893\du}{10.460014\du}}
\pgfpathlineto{\pgfpoint{34.959698\du}{10.460967\du}}
\pgfpathlineto{\pgfpoint{34.954456\du}{10.461921\du}}
\pgfpathlineto{\pgfpoint{34.949213\du}{10.463827\du}}
\pgfpathlineto{\pgfpoint{34.943018\du}{10.465733\du}}
\pgfpathlineto{\pgfpoint{34.937776\du}{10.468116\du}}
\pgfpathlineto{\pgfpoint{34.933010\du}{10.470499\du}}
\pgfpathlineto{\pgfpoint{34.927768\du}{10.472882\du}}
\pgfpathlineto{\pgfpoint{34.922049\du}{10.476218\du}}
\pgfpathlineto{\pgfpoint{34.917760\du}{10.479553\du}}
\pgfpathlineto{\pgfpoint{34.912518\du}{10.483366\du}}
\pgfpathlineto{\pgfpoint{34.909182\du}{10.486702\du}}
\pgfpathlineto{\pgfpoint{34.904416\du}{10.490514\du}}
\pgfpathlineto{\pgfpoint{34.900127\du}{10.494327\du}}
\pgfpathlineto{\pgfpoint{34.895838\du}{10.499093\du}}
\pgfpathlineto{\pgfpoint{34.892502\du}{10.503382\du}}
\pgfpathlineto{\pgfpoint{34.888690\du}{10.508147\du}}
\pgfpathlineto{\pgfpoint{34.885830\du}{10.512436\du}}
\pgfpathlineto{\pgfpoint{34.882971\du}{10.518632\du}}
\pgfpathlineto{\pgfpoint{34.880588\du}{10.523874\du}}
\pgfpathlineto{\pgfpoint{34.878205\du}{10.528640\du}}
\pgfpathlineto{\pgfpoint{34.876299\du}{10.534835\du}}
\pgfpathlineto{\pgfpoint{34.873916\du}{10.540077\du}}
\pgfpathlineto{\pgfpoint{34.872010\du}{10.545319\du}}
\pgfpathlineto{\pgfpoint{34.871057\du}{10.551515\du}}
\pgfpathlineto{\pgfpoint{34.870104\du}{10.557233\du}}
\pgfpathlineto{\pgfpoint{34.869151\du}{10.562952\du}}
\pgfpathlineto{\pgfpoint{34.868674\du}{10.568671\du}}
\pgfpathlineto{\pgfpoint{34.868674\du}{10.574866\du}}
\pgfpathlineto{\pgfpoint{34.868674\du}{10.574866\du}}
\pgfpathlineto{\pgfpoint{34.868674\du}{10.580585\du}}
\pgfpathlineto{\pgfpoint{34.869151\du}{10.586780\du}}
\pgfpathlineto{\pgfpoint{34.870104\du}{10.592022\du}}
\pgfpathlineto{\pgfpoint{34.871057\du}{10.598218\du}}
\pgfpathlineto{\pgfpoint{34.872010\du}{10.603937\du}}
\pgfpathlineto{\pgfpoint{34.873916\du}{10.609655\du}}
\pgfpathlineto{\pgfpoint{34.876299\du}{10.614898\du}}
\pgfpathlineto{\pgfpoint{34.878205\du}{10.620616\du}}
\pgfpathlineto{\pgfpoint{34.880588\du}{10.625858\du}}
\pgfpathlineto{\pgfpoint{34.882971\du}{10.631101\du}}
\pgfpathlineto{\pgfpoint{34.885830\du}{10.636343\du}}
\pgfpathlineto{\pgfpoint{34.888690\du}{10.640632\du}}
\pgfpathlineto{\pgfpoint{34.892502\du}{10.645874\du}}
\pgfpathlineto{\pgfpoint{34.895838\du}{10.650163\du}}
\pgfpathlineto{\pgfpoint{34.900127\du}{10.654929\du}}
\pgfpathlineto{\pgfpoint{34.904416\du}{10.659218\du}}
\pgfpathlineto{\pgfpoint{34.909182\du}{10.662554\du}}
\pgfpathlineto{\pgfpoint{34.912518\du}{10.666366\du}}
\pgfpathlineto{\pgfpoint{34.917760\du}{10.669702\du}}
\pgfpathlineto{\pgfpoint{34.922049\du}{10.673515\du}}
\pgfpathlineto{\pgfpoint{34.927768\du}{10.676374\du}}
\pgfpathlineto{\pgfpoint{34.933010\du}{10.678757\du}}
\pgfpathlineto{\pgfpoint{34.937776\du}{10.681140\du}}
\pgfpathlineto{\pgfpoint{34.943018\du}{10.683523\du}}
\pgfpathlineto{\pgfpoint{34.949213\du}{10.685429\du}}
\pgfpathlineto{\pgfpoint{34.954456\du}{10.687335\du}}
\pgfpathlineto{\pgfpoint{34.959698\du}{10.688288\du}}
\pgfpathlineto{\pgfpoint{34.965893\du}{10.689718\du}}
\pgfpathlineto{\pgfpoint{34.971612\du}{10.690195\du}}
\pgfpathlineto{\pgfpoint{34.977331\du}{10.690671\du}}
\pgfpathlineto{\pgfpoint{34.983526\du}{10.690671\du}}
\pgfpathlineto{\pgfpoint{34.983526\du}{10.690671\du}}
\pgfpathlineto{\pgfpoint{34.989245\du}{10.690671\du}}
\pgfpathlineto{\pgfpoint{34.994487\du}{10.690195\du}}
\pgfpathlineto{\pgfpoint{35.000206\du}{10.689718\du}}
\pgfpathlineto{\pgfpoint{35.006401\du}{10.688288\du}}
\pgfpathlineto{\pgfpoint{35.011643\du}{10.687335\du}}
\pgfpathlineto{\pgfpoint{35.016885\du}{10.685429\du}}
\pgfpathlineto{\pgfpoint{35.023081\du}{10.683523\du}}
\pgfpathlineto{\pgfpoint{35.028323\du}{10.681140\du}}
\pgfpathlineto{\pgfpoint{35.034042\du}{10.678757\du}}
\pgfpathlineto{\pgfpoint{35.038331\du}{10.676374\du}}
\pgfpathlineto{\pgfpoint{35.044049\du}{10.673515\du}}
\pgfpathlineto{\pgfpoint{35.048815\du}{10.669702\du}}
\pgfpathlineto{\pgfpoint{35.053581\du}{10.666366\du}}
\pgfpathlineto{\pgfpoint{35.057870\du}{10.662554\du}}
\pgfpathlineto{\pgfpoint{35.061682\du}{10.659218\du}}
\pgfpathlineto{\pgfpoint{35.065971\du}{10.654929\du}}
\pgfpathlineto{\pgfpoint{35.070260\du}{10.650163\du}}
\pgfpathlineto{\pgfpoint{35.073596\du}{10.645874\du}}
\pgfpathlineto{\pgfpoint{35.077409\du}{10.640632\du}}
\pgfpathlineto{\pgfpoint{35.080268\du}{10.636343\du}}
\pgfpathlineto{\pgfpoint{35.083128\du}{10.631101\du}}
\pgfpathlineto{\pgfpoint{35.085511\du}{10.625858\du}}
\pgfpathlineto{\pgfpoint{35.087893\du}{10.620616\du}}
\pgfpathlineto{\pgfpoint{35.090753\du}{10.614898\du}}
\pgfpathlineto{\pgfpoint{35.092182\du}{10.609655\du}}
\pgfpathlineto{\pgfpoint{35.094089\du}{10.603937\du}}
\pgfpathlineto{\pgfpoint{35.095042\du}{10.598218\du}}
\pgfpathlineto{\pgfpoint{35.096471\du}{10.592022\du}}
\pgfpathlineto{\pgfpoint{35.096948\du}{10.586780\du}}
\pgfpathlineto{\pgfpoint{35.097425\du}{10.580585\du}}
\pgfpathlineto{\pgfpoint{35.097425\du}{10.574866\du}}
\pgfusepath{fill}
\pgfsetbuttcap
\pgfsetmiterjoin
\pgfsetdash{}{0pt}
\definecolor{dialinecolor}{rgb}{1.000000, 1.000000, 1.000000}
\pgfsetfillcolor{dialinecolor}
\pgfpathmoveto{\pgfpoint{36.138239\du}{10.574866\du}}
\pgfpathlineto{\pgfpoint{36.138239\du}{10.568671\du}}
\pgfpathlineto{\pgfpoint{36.137763\du}{10.562952\du}}
\pgfpathlineto{\pgfpoint{36.136810\du}{10.557233\du}}
\pgfpathlineto{\pgfpoint{36.136333\du}{10.551515\du}}
\pgfpathlineto{\pgfpoint{36.134427\du}{10.545319\du}}
\pgfpathlineto{\pgfpoint{36.133474\du}{10.540077\du}}
\pgfpathlineto{\pgfpoint{36.131091\du}{10.534835\du}}
\pgfpathlineto{\pgfpoint{36.129185\du}{10.528640\du}}
\pgfpathlineto{\pgfpoint{36.126325\du}{10.523874\du}}
\pgfpathlineto{\pgfpoint{36.124419\du}{10.518632\du}}
\pgfpathlineto{\pgfpoint{36.121560\du}{10.512436\du}}
\pgfpathlineto{\pgfpoint{36.118224\du}{10.508147\du}}
\pgfpathlineto{\pgfpoint{36.114411\du}{10.503382\du}}
\pgfpathlineto{\pgfpoint{36.110599\du}{10.499093\du}}
\pgfpathlineto{\pgfpoint{36.106786\du}{10.494327\du}}
\pgfpathlineto{\pgfpoint{36.103450\du}{10.490514\du}}
\pgfpathlineto{\pgfpoint{36.098208\du}{10.486702\du}}
\pgfpathlineto{\pgfpoint{36.094395\du}{10.483366\du}}
\pgfpathlineto{\pgfpoint{36.089153\du}{10.479553\du}}
\pgfpathlineto{\pgfpoint{36.084388\du}{10.476218\du}}
\pgfpathlineto{\pgfpoint{36.079622\du}{10.472882\du}}
\pgfpathlineto{\pgfpoint{36.074380\du}{10.470499\du}}
\pgfpathlineto{\pgfpoint{36.069138\du}{10.468116\du}}
\pgfpathlineto{\pgfpoint{36.063419\du}{10.465733\du}}
\pgfpathlineto{\pgfpoint{36.058653\du}{10.463827\du}}
\pgfpathlineto{\pgfpoint{36.052934\du}{10.461921\du}}
\pgfpathlineto{\pgfpoint{36.046739\du}{10.460967\du}}
\pgfpathlineto{\pgfpoint{36.041497\du}{10.460014\du}}
\pgfpathlineto{\pgfpoint{36.035778\du}{10.459061\du}}
\pgfpathlineto{\pgfpoint{36.030059\du}{10.458585\du}}
\pgfpathlineto{\pgfpoint{36.024341\du}{10.458585\du}}
\pgfpathlineto{\pgfpoint{36.024341\du}{10.458585\du}}
\pgfpathlineto{\pgfpoint{36.018145\du}{10.458585\du}}
\pgfpathlineto{\pgfpoint{36.012903\du}{10.459061\du}}
\pgfpathlineto{\pgfpoint{36.007184\du}{10.460014\du}}
\pgfpathlineto{\pgfpoint{36.001942\du}{10.460967\du}}
\pgfpathlineto{\pgfpoint{35.996223\du}{10.461921\du}}
\pgfpathlineto{\pgfpoint{35.990505\du}{10.463827\du}}
\pgfpathlineto{\pgfpoint{35.984786\du}{10.465733\du}}
\pgfpathlineto{\pgfpoint{35.979544\du}{10.468116\du}}
\pgfpathlineto{\pgfpoint{35.974301\du}{10.470499\du}}
\pgfpathlineto{\pgfpoint{35.969536\du}{10.472882\du}}
\pgfpathlineto{\pgfpoint{35.963817\du}{10.476218\du}}
\pgfpathlineto{\pgfpoint{35.959051\du}{10.479553\du}}
\pgfpathlineto{\pgfpoint{35.954762\du}{10.483366\du}}
\pgfpathlineto{\pgfpoint{35.950473\du}{10.486702\du}}
\pgfpathlineto{\pgfpoint{35.945708\du}{10.490514\du}}
\pgfpathlineto{\pgfpoint{35.942372\du}{10.494327\du}}
\pgfpathlineto{\pgfpoint{35.937606\du}{10.499093\du}}
\pgfpathlineto{\pgfpoint{35.934270\du}{10.503382\du}}
\pgfpathlineto{\pgfpoint{35.930934\du}{10.508147\du}}
\pgfpathlineto{\pgfpoint{35.927598\du}{10.512436\du}}
\pgfpathlineto{\pgfpoint{35.924739\du}{10.518632\du}}
\pgfpathlineto{\pgfpoint{35.922356\du}{10.523874\du}}
\pgfpathlineto{\pgfpoint{35.919497\du}{10.528640\du}}
\pgfpathlineto{\pgfpoint{35.917590\du}{10.534835\du}}
\pgfpathlineto{\pgfpoint{35.915684\du}{10.540077\du}}
\pgfpathlineto{\pgfpoint{35.913778\du}{10.545319\du}}
\pgfpathlineto{\pgfpoint{35.912825\du}{10.551515\du}}
\pgfpathlineto{\pgfpoint{35.911395\du}{10.557233\du}}
\pgfpathlineto{\pgfpoint{35.910919\du}{10.562952\du}}
\pgfpathlineto{\pgfpoint{35.910442\du}{10.568671\du}}
\pgfpathlineto{\pgfpoint{35.910442\du}{10.574866\du}}
\pgfpathlineto{\pgfpoint{35.910442\du}{10.574866\du}}
\pgfpathlineto{\pgfpoint{35.910442\du}{10.580585\du}}
\pgfpathlineto{\pgfpoint{35.910919\du}{10.586780\du}}
\pgfpathlineto{\pgfpoint{35.911395\du}{10.592022\du}}
\pgfpathlineto{\pgfpoint{35.912825\du}{10.598218\du}}
\pgfpathlineto{\pgfpoint{35.913778\du}{10.603937\du}}
\pgfpathlineto{\pgfpoint{35.915684\du}{10.609655\du}}
\pgfpathlineto{\pgfpoint{35.917590\du}{10.614898\du}}
\pgfpathlineto{\pgfpoint{35.919497\du}{10.620616\du}}
\pgfpathlineto{\pgfpoint{35.922356\du}{10.625858\du}}
\pgfpathlineto{\pgfpoint{35.924739\du}{10.631101\du}}
\pgfpathlineto{\pgfpoint{35.927598\du}{10.636343\du}}
\pgfpathlineto{\pgfpoint{35.930934\du}{10.640632\du}}
\pgfpathlineto{\pgfpoint{35.934270\du}{10.645874\du}}
\pgfpathlineto{\pgfpoint{35.937606\du}{10.650163\du}}
\pgfpathlineto{\pgfpoint{35.942372\du}{10.654929\du}}
\pgfpathlineto{\pgfpoint{35.945708\du}{10.659218\du}}
\pgfpathlineto{\pgfpoint{35.950473\du}{10.662554\du}}
\pgfpathlineto{\pgfpoint{35.954762\du}{10.666366\du}}
\pgfpathlineto{\pgfpoint{35.959051\du}{10.669702\du}}
\pgfpathlineto{\pgfpoint{35.963817\du}{10.673515\du}}
\pgfpathlineto{\pgfpoint{35.969536\du}{10.676374\du}}
\pgfpathlineto{\pgfpoint{35.974301\du}{10.678757\du}}
\pgfpathlineto{\pgfpoint{35.979544\du}{10.681140\du}}
\pgfpathlineto{\pgfpoint{35.984786\du}{10.683523\du}}
\pgfpathlineto{\pgfpoint{35.990505\du}{10.685429\du}}
\pgfpathlineto{\pgfpoint{35.996223\du}{10.687335\du}}
\pgfpathlineto{\pgfpoint{36.001942\du}{10.688288\du}}
\pgfpathlineto{\pgfpoint{36.007184\du}{10.689718\du}}
\pgfpathlineto{\pgfpoint{36.012903\du}{10.690195\du}}
\pgfpathlineto{\pgfpoint{36.018145\du}{10.690671\du}}
\pgfpathlineto{\pgfpoint{36.024341\du}{10.690671\du}}
\pgfpathlineto{\pgfpoint{36.024341\du}{10.690671\du}}
\pgfpathlineto{\pgfpoint{36.030059\du}{10.690671\du}}
\pgfpathlineto{\pgfpoint{36.035778\du}{10.690195\du}}
\pgfpathlineto{\pgfpoint{36.041497\du}{10.689718\du}}
\pgfpathlineto{\pgfpoint{36.046739\du}{10.688288\du}}
\pgfpathlineto{\pgfpoint{36.052934\du}{10.687335\du}}
\pgfpathlineto{\pgfpoint{36.058653\du}{10.685429\du}}
\pgfpathlineto{\pgfpoint{36.063419\du}{10.683523\du}}
\pgfpathlineto{\pgfpoint{36.069138\du}{10.681140\du}}
\pgfpathlineto{\pgfpoint{36.074380\du}{10.678757\du}}
\pgfpathlineto{\pgfpoint{36.079622\du}{10.676374\du}}
\pgfpathlineto{\pgfpoint{36.084388\du}{10.673515\du}}
\pgfpathlineto{\pgfpoint{36.089153\du}{10.669702\du}}
\pgfpathlineto{\pgfpoint{36.094395\du}{10.666366\du}}
\pgfpathlineto{\pgfpoint{36.098208\du}{10.662554\du}}
\pgfpathlineto{\pgfpoint{36.103450\du}{10.659218\du}}
\pgfpathlineto{\pgfpoint{36.106786\du}{10.654929\du}}
\pgfpathlineto{\pgfpoint{36.110599\du}{10.650163\du}}
\pgfpathlineto{\pgfpoint{36.114411\du}{10.645874\du}}
\pgfpathlineto{\pgfpoint{36.118224\du}{10.640632\du}}
\pgfpathlineto{\pgfpoint{36.121560\du}{10.636343\du}}
\pgfpathlineto{\pgfpoint{36.124419\du}{10.631101\du}}
\pgfpathlineto{\pgfpoint{36.126325\du}{10.625858\du}}
\pgfpathlineto{\pgfpoint{36.129185\du}{10.620616\du}}
\pgfpathlineto{\pgfpoint{36.131091\du}{10.614898\du}}
\pgfpathlineto{\pgfpoint{36.133474\du}{10.609655\du}}
\pgfpathlineto{\pgfpoint{36.134427\du}{10.603937\du}}
\pgfpathlineto{\pgfpoint{36.136333\du}{10.598218\du}}
\pgfpathlineto{\pgfpoint{36.136810\du}{10.592022\du}}
\pgfpathlineto{\pgfpoint{36.137763\du}{10.586780\du}}
\pgfpathlineto{\pgfpoint{36.138239\du}{10.580585\du}}
\pgfpathlineto{\pgfpoint{36.138239\du}{10.574866\du}}
\pgfusepath{fill}
\pgfsetlinewidth{0.000000\du}
\pgfsetbuttcap
\pgfsetmiterjoin
\pgfsetdash{}{0pt}
\definecolor{dialinecolor}{rgb}{0.000000, 0.000000, 0.000000}
\pgfsetstrokecolor{dialinecolor}
\pgfpathmoveto{\pgfpoint{34.994010\du}{10.564382\du}}
\pgfpathlineto{\pgfpoint{35.991934\du}{10.564382\du}}
\pgfusepath{stroke}
\pgfsetlinewidth{0.000000\du}
\pgfsetbuttcap
\pgfsetmiterjoin
\pgfsetdash{}{0pt}
\definecolor{dialinecolor}{rgb}{1.000000, 1.000000, 1.000000}
\pgfsetfillcolor{dialinecolor}
\pgfpathmoveto{\pgfpoint{35.202745\du}{8.075291\du}}
\pgfpathlineto{\pgfpoint{35.202745\du}{8.560432\du}}
\pgfpathlineto{\pgfpoint{35.910442\du}{8.560432\du}}
\pgfpathlineto{\pgfpoint{35.910442\du}{8.075291\du}}
\pgfpathlineto{\pgfpoint{35.202745\du}{8.075291\du}}
\pgfusepath{fill}
\pgfsetbuttcap
\pgfsetmiterjoin
\pgfsetdash{}{0pt}
\definecolor{dialinecolor}{rgb}{1.000000, 1.000000, 1.000000}
\pgfsetfillcolor{dialinecolor}
\pgfpathmoveto{\pgfpoint{35.118393\du}{8.908324\du}}
\pgfpathlineto{\pgfpoint{35.118393\du}{8.902605\du}}
\pgfpathlineto{\pgfpoint{35.117917\du}{8.896409\du}}
\pgfpathlineto{\pgfpoint{35.117440\du}{8.891167\du}}
\pgfpathlineto{\pgfpoint{35.116011\du}{8.884972\du}}
\pgfpathlineto{\pgfpoint{35.115057\du}{8.879730\du}}
\pgfpathlineto{\pgfpoint{35.113151\du}{8.873534\du}}
\pgfpathlineto{\pgfpoint{35.111722\du}{8.868292\du}}
\pgfpathlineto{\pgfpoint{35.109339\du}{8.863050\du}}
\pgfpathlineto{\pgfpoint{35.106479\du}{8.857331\du}}
\pgfpathlineto{\pgfpoint{35.104096\du}{8.852089\du}}
\pgfpathlineto{\pgfpoint{35.101237\du}{8.847323\du}}
\pgfpathlineto{\pgfpoint{35.098378\du}{8.842558\du}}
\pgfpathlineto{\pgfpoint{35.094565\du}{8.837792\du}}
\pgfpathlineto{\pgfpoint{35.091229\du}{8.832550\du}}
\pgfpathlineto{\pgfpoint{35.086940\du}{8.828737\du}}
\pgfpathlineto{\pgfpoint{35.083128\du}{8.824448\du}}
\pgfpathlineto{\pgfpoint{35.078839\du}{8.820636\du}}
\pgfpathlineto{\pgfpoint{35.074073\du}{8.816823\du}}
\pgfpathlineto{\pgfpoint{35.069784\du}{8.813964\du}}
\pgfpathlineto{\pgfpoint{35.064542\du}{8.810151\du}}
\pgfpathlineto{\pgfpoint{35.059776\du}{8.807292\du}}
\pgfpathlineto{\pgfpoint{35.055010\du}{8.804909\du}}
\pgfpathlineto{\pgfpoint{35.049292\du}{8.802526\du}}
\pgfpathlineto{\pgfpoint{35.044049\du}{8.799667\du}}
\pgfpathlineto{\pgfpoint{35.038331\du}{8.798237\du}}
\pgfpathlineto{\pgfpoint{35.033089\du}{8.796331\du}}
\pgfpathlineto{\pgfpoint{35.027370\du}{8.795378\du}}
\pgfpathlineto{\pgfpoint{35.022128\du}{8.793948\du}}
\pgfpathlineto{\pgfpoint{35.016409\du}{8.793472\du}}
\pgfpathlineto{\pgfpoint{35.010690\du}{8.792995\du}}
\pgfpathlineto{\pgfpoint{35.004495\du}{8.792995\du}}
\pgfpathlineto{\pgfpoint{35.004495\du}{8.792995\du}}
\pgfpathlineto{\pgfpoint{34.998776\du}{8.792995\du}}
\pgfpathlineto{\pgfpoint{34.993057\du}{8.793472\du}}
\pgfpathlineto{\pgfpoint{34.987338\du}{8.793948\du}}
\pgfpathlineto{\pgfpoint{34.982096\du}{8.795378\du}}
\pgfpathlineto{\pgfpoint{34.975901\du}{8.796331\du}}
\pgfpathlineto{\pgfpoint{34.970659\du}{8.798237\du}}
\pgfpathlineto{\pgfpoint{34.965417\du}{8.799667\du}}
\pgfpathlineto{\pgfpoint{34.959698\du}{8.802526\du}}
\pgfpathlineto{\pgfpoint{34.954456\du}{8.804909\du}}
\pgfpathlineto{\pgfpoint{34.949213\du}{8.807292\du}}
\pgfpathlineto{\pgfpoint{34.944924\du}{8.810151\du}}
\pgfpathlineto{\pgfpoint{34.939682\du}{8.813964\du}}
\pgfpathlineto{\pgfpoint{34.934916\du}{8.816823\du}}
\pgfpathlineto{\pgfpoint{34.930627\du}{8.820636\du}}
\pgfpathlineto{\pgfpoint{34.925862\du}{8.824448\du}}
\pgfpathlineto{\pgfpoint{34.922049\du}{8.828737\du}}
\pgfpathlineto{\pgfpoint{34.918237\du}{8.832550\du}}
\pgfpathlineto{\pgfpoint{34.914901\du}{8.837792\du}}
\pgfpathlineto{\pgfpoint{34.911088\du}{8.842558\du}}
\pgfpathlineto{\pgfpoint{34.907752\du}{8.847323\du}}
\pgfpathlineto{\pgfpoint{34.904893\du}{8.852089\du}}
\pgfpathlineto{\pgfpoint{34.902987\du}{8.857331\du}}
\pgfpathlineto{\pgfpoint{34.900127\du}{8.863050\du}}
\pgfpathlineto{\pgfpoint{34.897744\du}{8.868292\du}}
\pgfpathlineto{\pgfpoint{34.895838\du}{8.873534\du}}
\pgfpathlineto{\pgfpoint{34.894409\du}{8.879730\du}}
\pgfpathlineto{\pgfpoint{34.892979\du}{8.884972\du}}
\pgfpathlineto{\pgfpoint{34.892026\du}{8.891167\du}}
\pgfpathlineto{\pgfpoint{34.891073\du}{8.896409\du}}
\pgfpathlineto{\pgfpoint{34.891073\du}{8.902605\du}}
\pgfpathlineto{\pgfpoint{34.891073\du}{8.908324\du}}
\pgfpathlineto{\pgfpoint{34.891073\du}{8.908324\du}}
\pgfpathlineto{\pgfpoint{34.891073\du}{8.914519\du}}
\pgfpathlineto{\pgfpoint{34.891073\du}{8.920238\du}}
\pgfpathlineto{\pgfpoint{34.892026\du}{8.925956\du}}
\pgfpathlineto{\pgfpoint{34.892979\du}{8.931675\du}}
\pgfpathlineto{\pgfpoint{34.894409\du}{8.936917\du}}
\pgfpathlineto{\pgfpoint{34.895838\du}{8.942636\du}}
\pgfpathlineto{\pgfpoint{34.897744\du}{8.948355\du}}
\pgfpathlineto{\pgfpoint{34.900127\du}{8.954074\du}}
\pgfpathlineto{\pgfpoint{34.902987\du}{8.959316\du}}
\pgfpathlineto{\pgfpoint{34.904893\du}{8.964558\du}}
\pgfpathlineto{\pgfpoint{34.907752\du}{8.969324\du}}
\pgfpathlineto{\pgfpoint{34.911088\du}{8.974089\du}}
\pgfpathlineto{\pgfpoint{34.914901\du}{8.978378\du}}
\pgfpathlineto{\pgfpoint{34.918237\du}{8.983621\du}}
\pgfpathlineto{\pgfpoint{34.922049\du}{8.987433\du}}
\pgfpathlineto{\pgfpoint{34.925862\du}{8.992199\du}}
\pgfpathlineto{\pgfpoint{34.930627\du}{8.995535\du}}
\pgfpathlineto{\pgfpoint{34.934916\du}{8.999347\du}}
\pgfpathlineto{\pgfpoint{34.939682\du}{9.003160\du}}
\pgfpathlineto{\pgfpoint{34.944924\du}{9.006496\du}}
\pgfpathlineto{\pgfpoint{34.949213\du}{9.009355\du}}
\pgfpathlineto{\pgfpoint{34.954456\du}{9.011261\du}}
\pgfpathlineto{\pgfpoint{34.959698\du}{9.014121\du}}
\pgfpathlineto{\pgfpoint{34.965417\du}{9.016503\du}}
\pgfpathlineto{\pgfpoint{34.970659\du}{9.018410\du}}
\pgfpathlineto{\pgfpoint{34.975901\du}{9.020316\du}}
\pgfpathlineto{\pgfpoint{34.982096\du}{9.021746\du}}
\pgfpathlineto{\pgfpoint{34.987338\du}{9.022699\du}}
\pgfpathlineto{\pgfpoint{34.993057\du}{9.023175\du}}
\pgfpathlineto{\pgfpoint{34.998776\du}{9.024128\du}}
\pgfpathlineto{\pgfpoint{35.004495\du}{9.024128\du}}
\pgfpathlineto{\pgfpoint{35.004495\du}{9.024128\du}}
\pgfpathlineto{\pgfpoint{35.010690\du}{9.024128\du}}
\pgfpathlineto{\pgfpoint{35.016409\du}{9.023175\du}}
\pgfpathlineto{\pgfpoint{35.022128\du}{9.022699\du}}
\pgfpathlineto{\pgfpoint{35.027370\du}{9.021746\du}}
\pgfpathlineto{\pgfpoint{35.033089\du}{9.020316\du}}
\pgfpathlineto{\pgfpoint{35.038331\du}{9.018410\du}}
\pgfpathlineto{\pgfpoint{35.044049\du}{9.016503\du}}
\pgfpathlineto{\pgfpoint{35.049292\du}{9.014121\du}}
\pgfpathlineto{\pgfpoint{35.055010\du}{9.011261\du}}
\pgfpathlineto{\pgfpoint{35.059776\du}{9.009355\du}}
\pgfpathlineto{\pgfpoint{35.064542\du}{9.006496\du}}
\pgfpathlineto{\pgfpoint{35.069784\du}{9.003160\du}}
\pgfpathlineto{\pgfpoint{35.074073\du}{8.999347\du}}
\pgfpathlineto{\pgfpoint{35.078839\du}{8.995535\du}}
\pgfpathlineto{\pgfpoint{35.083128\du}{8.992199\du}}
\pgfpathlineto{\pgfpoint{35.086940\du}{8.987433\du}}
\pgfpathlineto{\pgfpoint{35.091229\du}{8.983621\du}}
\pgfpathlineto{\pgfpoint{35.094565\du}{8.978378\du}}
\pgfpathlineto{\pgfpoint{35.098378\du}{8.974089\du}}
\pgfpathlineto{\pgfpoint{35.101237\du}{8.969324\du}}
\pgfpathlineto{\pgfpoint{35.104096\du}{8.964558\du}}
\pgfpathlineto{\pgfpoint{35.106479\du}{8.959316\du}}
\pgfpathlineto{\pgfpoint{35.109339\du}{8.954074\du}}
\pgfpathlineto{\pgfpoint{35.111722\du}{8.948355\du}}
\pgfpathlineto{\pgfpoint{35.113151\du}{8.942636\du}}
\pgfpathlineto{\pgfpoint{35.115057\du}{8.936917\du}}
\pgfpathlineto{\pgfpoint{35.116011\du}{8.931675\du}}
\pgfpathlineto{\pgfpoint{35.117440\du}{8.925956\du}}
\pgfpathlineto{\pgfpoint{35.117917\du}{8.920238\du}}
\pgfpathlineto{\pgfpoint{35.118393\du}{8.914519\du}}
\pgfpathlineto{\pgfpoint{35.118393\du}{8.908324\du}}
\pgfusepath{fill}
\pgfsetbuttcap
\pgfsetmiterjoin
\pgfsetdash{}{0pt}
\definecolor{dialinecolor}{rgb}{1.000000, 1.000000, 1.000000}
\pgfsetfillcolor{dialinecolor}
\pgfpathmoveto{\pgfpoint{36.159208\du}{8.908324\du}}
\pgfpathlineto{\pgfpoint{36.159208\du}{8.902605\du}}
\pgfpathlineto{\pgfpoint{36.158255\du}{8.896409\du}}
\pgfpathlineto{\pgfpoint{36.157778\du}{8.891167\du}}
\pgfpathlineto{\pgfpoint{36.157302\du}{8.884972\du}}
\pgfpathlineto{\pgfpoint{36.155396\du}{8.879730\du}}
\pgfpathlineto{\pgfpoint{36.153966\du}{8.873534\du}}
\pgfpathlineto{\pgfpoint{36.152060\du}{8.868292\du}}
\pgfpathlineto{\pgfpoint{36.149677\du}{8.863050\du}}
\pgfpathlineto{\pgfpoint{36.147294\du}{8.857331\du}}
\pgfpathlineto{\pgfpoint{36.145388\du}{8.852089\du}}
\pgfpathlineto{\pgfpoint{36.142052\du}{8.847323\du}}
\pgfpathlineto{\pgfpoint{36.139192\du}{8.842558\du}}
\pgfpathlineto{\pgfpoint{36.135380\du}{8.837792\du}}
\pgfpathlineto{\pgfpoint{36.131567\du}{8.832550\du}}
\pgfpathlineto{\pgfpoint{36.127278\du}{8.828737\du}}
\pgfpathlineto{\pgfpoint{36.123466\du}{8.824448\du}}
\pgfpathlineto{\pgfpoint{36.119177\du}{8.820636\du}}
\pgfpathlineto{\pgfpoint{36.115364\du}{8.816823\du}}
\pgfpathlineto{\pgfpoint{36.110122\du}{8.813964\du}}
\pgfpathlineto{\pgfpoint{36.105356\du}{8.810151\du}}
\pgfpathlineto{\pgfpoint{36.100591\du}{8.807292\du}}
\pgfpathlineto{\pgfpoint{36.095349\du}{8.804909\du}}
\pgfpathlineto{\pgfpoint{36.090106\du}{8.802526\du}}
\pgfpathlineto{\pgfpoint{36.084388\du}{8.799667\du}}
\pgfpathlineto{\pgfpoint{36.078669\du}{8.798237\du}}
\pgfpathlineto{\pgfpoint{36.073427\du}{8.796331\du}}
\pgfpathlineto{\pgfpoint{36.067708\du}{8.795378\du}}
\pgfpathlineto{\pgfpoint{36.061989\du}{8.793948\du}}
\pgfpathlineto{\pgfpoint{36.055794\du}{8.793472\du}}
\pgfpathlineto{\pgfpoint{36.050552\du}{8.792995\du}}
\pgfpathlineto{\pgfpoint{36.044356\du}{8.792995\du}}
\pgfpathlineto{\pgfpoint{36.044356\du}{8.792995\du}}
\pgfpathlineto{\pgfpoint{36.038638\du}{8.792995\du}}
\pgfpathlineto{\pgfpoint{36.033395\du}{8.793472\du}}
\pgfpathlineto{\pgfpoint{36.027200\du}{8.793948\du}}
\pgfpathlineto{\pgfpoint{36.021481\du}{8.795378\du}}
\pgfpathlineto{\pgfpoint{36.015763\du}{8.796331\du}}
\pgfpathlineto{\pgfpoint{36.010997\du}{8.798237\du}}
\pgfpathlineto{\pgfpoint{36.005278\du}{8.799667\du}}
\pgfpathlineto{\pgfpoint{35.999083\du}{8.802526\du}}
\pgfpathlineto{\pgfpoint{35.993841\du}{8.804909\du}}
\pgfpathlineto{\pgfpoint{35.989075\du}{8.807292\du}}
\pgfpathlineto{\pgfpoint{35.984309\du}{8.810151\du}}
\pgfpathlineto{\pgfpoint{35.979067\du}{8.813964\du}}
\pgfpathlineto{\pgfpoint{35.974301\du}{8.816823\du}}
\pgfpathlineto{\pgfpoint{35.970012\du}{8.820636\du}}
\pgfpathlineto{\pgfpoint{35.966200\du}{8.824448\du}}
\pgfpathlineto{\pgfpoint{35.961434\du}{8.828737\du}}
\pgfpathlineto{\pgfpoint{35.957622\du}{8.832550\du}}
\pgfpathlineto{\pgfpoint{35.954286\du}{8.837792\du}}
\pgfpathlineto{\pgfpoint{35.950473\du}{8.842558\du}}
\pgfpathlineto{\pgfpoint{35.947137\du}{8.847323\du}}
\pgfpathlineto{\pgfpoint{35.944278\du}{8.852089\du}}
\pgfpathlineto{\pgfpoint{35.942372\du}{8.857331\du}}
\pgfpathlineto{\pgfpoint{35.939512\du}{8.863050\du}}
\pgfpathlineto{\pgfpoint{35.937130\du}{8.868292\du}}
\pgfpathlineto{\pgfpoint{35.935223\du}{8.873534\du}}
\pgfpathlineto{\pgfpoint{35.933794\du}{8.879730\du}}
\pgfpathlineto{\pgfpoint{35.932364\du}{8.884972\du}}
\pgfpathlineto{\pgfpoint{35.931411\du}{8.891167\du}}
\pgfpathlineto{\pgfpoint{35.930934\du}{8.896409\du}}
\pgfpathlineto{\pgfpoint{35.930458\du}{8.902605\du}}
\pgfpathlineto{\pgfpoint{35.930458\du}{8.908324\du}}
\pgfpathlineto{\pgfpoint{35.930458\du}{8.908324\du}}
\pgfpathlineto{\pgfpoint{35.930458\du}{8.914519\du}}
\pgfpathlineto{\pgfpoint{35.930934\du}{8.920238\du}}
\pgfpathlineto{\pgfpoint{35.931411\du}{8.925956\du}}
\pgfpathlineto{\pgfpoint{35.932364\du}{8.931675\du}}
\pgfpathlineto{\pgfpoint{35.933794\du}{8.936917\du}}
\pgfpathlineto{\pgfpoint{35.935223\du}{8.942636\du}}
\pgfpathlineto{\pgfpoint{35.937130\du}{8.948355\du}}
\pgfpathlineto{\pgfpoint{35.939512\du}{8.954074\du}}
\pgfpathlineto{\pgfpoint{35.942372\du}{8.959316\du}}
\pgfpathlineto{\pgfpoint{35.944278\du}{8.964558\du}}
\pgfpathlineto{\pgfpoint{35.947137\du}{8.969324\du}}
\pgfpathlineto{\pgfpoint{35.950473\du}{8.974089\du}}
\pgfpathlineto{\pgfpoint{35.954286\du}{8.978378\du}}
\pgfpathlineto{\pgfpoint{35.957622\du}{8.983621\du}}
\pgfpathlineto{\pgfpoint{35.961434\du}{8.987433\du}}
\pgfpathlineto{\pgfpoint{35.966200\du}{8.992199\du}}
\pgfpathlineto{\pgfpoint{35.970012\du}{8.995535\du}}
\pgfpathlineto{\pgfpoint{35.974301\du}{8.999347\du}}
\pgfpathlineto{\pgfpoint{35.979067\du}{9.003160\du}}
\pgfpathlineto{\pgfpoint{35.984309\du}{9.006496\du}}
\pgfpathlineto{\pgfpoint{35.989075\du}{9.009355\du}}
\pgfpathlineto{\pgfpoint{35.993841\du}{9.011261\du}}
\pgfpathlineto{\pgfpoint{35.999083\du}{9.014121\du}}
\pgfpathlineto{\pgfpoint{36.005278\du}{9.016503\du}}
\pgfpathlineto{\pgfpoint{36.010997\du}{9.018410\du}}
\pgfpathlineto{\pgfpoint{36.015763\du}{9.020316\du}}
\pgfpathlineto{\pgfpoint{36.021481\du}{9.021746\du}}
\pgfpathlineto{\pgfpoint{36.027200\du}{9.022699\du}}
\pgfpathlineto{\pgfpoint{36.033395\du}{9.023175\du}}
\pgfpathlineto{\pgfpoint{36.038638\du}{9.024128\du}}
\pgfpathlineto{\pgfpoint{36.044356\du}{9.024128\du}}
\pgfpathlineto{\pgfpoint{36.044356\du}{9.024128\du}}
\pgfpathlineto{\pgfpoint{36.050552\du}{9.024128\du}}
\pgfpathlineto{\pgfpoint{36.055794\du}{9.023175\du}}
\pgfpathlineto{\pgfpoint{36.061989\du}{9.022699\du}}
\pgfpathlineto{\pgfpoint{36.067708\du}{9.021746\du}}
\pgfpathlineto{\pgfpoint{36.073427\du}{9.020316\du}}
\pgfpathlineto{\pgfpoint{36.078669\du}{9.018410\du}}
\pgfpathlineto{\pgfpoint{36.084388\du}{9.016503\du}}
\pgfpathlineto{\pgfpoint{36.090106\du}{9.014121\du}}
\pgfpathlineto{\pgfpoint{36.095349\du}{9.011261\du}}
\pgfpathlineto{\pgfpoint{36.100591\du}{9.009355\du}}
\pgfpathlineto{\pgfpoint{36.105356\du}{9.006496\du}}
\pgfpathlineto{\pgfpoint{36.110122\du}{9.003160\du}}
\pgfpathlineto{\pgfpoint{36.115364\du}{8.999347\du}}
\pgfpathlineto{\pgfpoint{36.119177\du}{8.995535\du}}
\pgfpathlineto{\pgfpoint{36.123466\du}{8.992199\du}}
\pgfpathlineto{\pgfpoint{36.127278\du}{8.987433\du}}
\pgfpathlineto{\pgfpoint{36.131567\du}{8.983621\du}}
\pgfpathlineto{\pgfpoint{36.135380\du}{8.978378\du}}
\pgfpathlineto{\pgfpoint{36.139192\du}{8.974089\du}}
\pgfpathlineto{\pgfpoint{36.142052\du}{8.969324\du}}
\pgfpathlineto{\pgfpoint{36.145388\du}{8.964558\du}}
\pgfpathlineto{\pgfpoint{36.147294\du}{8.959316\du}}
\pgfpathlineto{\pgfpoint{36.149677\du}{8.954074\du}}
\pgfpathlineto{\pgfpoint{36.152060\du}{8.948355\du}}
\pgfpathlineto{\pgfpoint{36.153966\du}{8.942636\du}}
\pgfpathlineto{\pgfpoint{36.155396\du}{8.936917\du}}
\pgfpathlineto{\pgfpoint{36.157302\du}{8.931675\du}}
\pgfpathlineto{\pgfpoint{36.157778\du}{8.925956\du}}
\pgfpathlineto{\pgfpoint{36.158255\du}{8.920238\du}}
\pgfpathlineto{\pgfpoint{36.159208\du}{8.914519\du}}
\pgfpathlineto{\pgfpoint{36.159208\du}{8.908324\du}}
\pgfusepath{fill}
\pgfsetlinewidth{0.000000\du}
\pgfsetbuttcap
\pgfsetmiterjoin
\pgfsetdash{}{0pt}
\definecolor{dialinecolor}{rgb}{1.000000, 1.000000, 1.000000}
\pgfsetstrokecolor{dialinecolor}
\pgfpathmoveto{\pgfpoint{35.015456\du}{8.898316\du}}
\pgfpathlineto{\pgfpoint{36.013856\du}{8.898316\du}}
\pgfusepath{stroke}
\pgfsetlinewidth{0.000000\du}
\pgfsetbuttcap
\pgfsetmiterjoin
\pgfsetdash{}{0pt}
\definecolor{dialinecolor}{rgb}{1.000000, 1.000000, 1.000000}
\pgfsetfillcolor{dialinecolor}
\pgfpathmoveto{\pgfpoint{35.118393\du}{9.330082\du}}
\pgfpathlineto{\pgfpoint{35.118393\du}{9.323887\du}}
\pgfpathlineto{\pgfpoint{35.117917\du}{9.318168\du}}
\pgfpathlineto{\pgfpoint{35.117440\du}{9.312926\du}}
\pgfpathlineto{\pgfpoint{35.116011\du}{9.307207\du}}
\pgfpathlineto{\pgfpoint{35.115057\du}{9.301488\du}}
\pgfpathlineto{\pgfpoint{35.113151\du}{9.295770\du}}
\pgfpathlineto{\pgfpoint{35.111722\du}{9.290051\du}}
\pgfpathlineto{\pgfpoint{35.109339\du}{9.284809\du}}
\pgfpathlineto{\pgfpoint{35.106479\du}{9.279567\du}}
\pgfpathlineto{\pgfpoint{35.104096\du}{9.274324\du}}
\pgfpathlineto{\pgfpoint{35.101237\du}{9.269082\du}}
\pgfpathlineto{\pgfpoint{35.098378\du}{9.264316\du}}
\pgfpathlineto{\pgfpoint{35.094565\du}{9.259551\du}}
\pgfpathlineto{\pgfpoint{35.091229\du}{9.254785\du}}
\pgfpathlineto{\pgfpoint{35.086940\du}{9.250973\du}}
\pgfpathlineto{\pgfpoint{35.083128\du}{9.246684\du}}
\pgfpathlineto{\pgfpoint{35.078839\du}{9.242871\du}}
\pgfpathlineto{\pgfpoint{35.074073\du}{9.238582\du}}
\pgfpathlineto{\pgfpoint{35.069784\du}{9.235723\du}}
\pgfpathlineto{\pgfpoint{35.064542\du}{9.232387\du}}
\pgfpathlineto{\pgfpoint{35.059776\du}{9.228574\du}}
\pgfpathlineto{\pgfpoint{35.055010\du}{9.226668\du}}
\pgfpathlineto{\pgfpoint{35.049292\du}{9.224285\du}}
\pgfpathlineto{\pgfpoint{35.044049\du}{9.221902\du}}
\pgfpathlineto{\pgfpoint{35.038331\du}{9.219996\du}}
\pgfpathlineto{\pgfpoint{35.033089\du}{9.218566\du}}
\pgfpathlineto{\pgfpoint{35.027370\du}{9.216660\du}}
\pgfpathlineto{\pgfpoint{35.022128\du}{9.216184\du}}
\pgfpathlineto{\pgfpoint{35.016409\du}{9.215230\du}}
\pgfpathlineto{\pgfpoint{35.010690\du}{9.214754\du}}
\pgfpathlineto{\pgfpoint{35.004495\du}{9.214754\du}}
\pgfpathlineto{\pgfpoint{35.004495\du}{9.214754\du}}
\pgfpathlineto{\pgfpoint{34.998776\du}{9.214754\du}}
\pgfpathlineto{\pgfpoint{34.993057\du}{9.215230\du}}
\pgfpathlineto{\pgfpoint{34.987338\du}{9.216184\du}}
\pgfpathlineto{\pgfpoint{34.982096\du}{9.216660\du}}
\pgfpathlineto{\pgfpoint{34.975901\du}{9.218566\du}}
\pgfpathlineto{\pgfpoint{34.970659\du}{9.219996\du}}
\pgfpathlineto{\pgfpoint{34.965417\du}{9.221902\du}}
\pgfpathlineto{\pgfpoint{34.959698\du}{9.224285\du}}
\pgfpathlineto{\pgfpoint{34.954456\du}{9.226668\du}}
\pgfpathlineto{\pgfpoint{34.949213\du}{9.228574\du}}
\pgfpathlineto{\pgfpoint{34.944924\du}{9.232387\du}}
\pgfpathlineto{\pgfpoint{34.939682\du}{9.235723\du}}
\pgfpathlineto{\pgfpoint{34.934916\du}{9.238582\du}}
\pgfpathlineto{\pgfpoint{34.930627\du}{9.242871\du}}
\pgfpathlineto{\pgfpoint{34.925862\du}{9.246684\du}}
\pgfpathlineto{\pgfpoint{34.922049\du}{9.250973\du}}
\pgfpathlineto{\pgfpoint{34.918237\du}{9.254785\du}}
\pgfpathlineto{\pgfpoint{34.914901\du}{9.259551\du}}
\pgfpathlineto{\pgfpoint{34.911088\du}{9.264316\du}}
\pgfpathlineto{\pgfpoint{34.907752\du}{9.269082\du}}
\pgfpathlineto{\pgfpoint{34.904893\du}{9.274324\du}}
\pgfpathlineto{\pgfpoint{34.902987\du}{9.279567\du}}
\pgfpathlineto{\pgfpoint{34.900127\du}{9.284809\du}}
\pgfpathlineto{\pgfpoint{34.897744\du}{9.290051\du}}
\pgfpathlineto{\pgfpoint{34.895838\du}{9.295770\du}}
\pgfpathlineto{\pgfpoint{34.894409\du}{9.301488\du}}
\pgfpathlineto{\pgfpoint{34.892979\du}{9.307207\du}}
\pgfpathlineto{\pgfpoint{34.892026\du}{9.312926\du}}
\pgfpathlineto{\pgfpoint{34.891073\du}{9.318168\du}}
\pgfpathlineto{\pgfpoint{34.891073\du}{9.323887\du}}
\pgfpathlineto{\pgfpoint{34.891073\du}{9.330082\du}}
\pgfpathlineto{\pgfpoint{34.891073\du}{9.330082\du}}
\pgfpathlineto{\pgfpoint{34.891073\du}{9.335801\du}}
\pgfpathlineto{\pgfpoint{34.891073\du}{9.341996\du}}
\pgfpathlineto{\pgfpoint{34.892026\du}{9.347715\du}}
\pgfpathlineto{\pgfpoint{34.892979\du}{9.353910\du}}
\pgfpathlineto{\pgfpoint{34.894409\du}{9.359153\du}}
\pgfpathlineto{\pgfpoint{34.895838\du}{9.364871\du}}
\pgfpathlineto{\pgfpoint{34.897744\du}{9.370590\du}}
\pgfpathlineto{\pgfpoint{34.900127\du}{9.375832\du}}
\pgfpathlineto{\pgfpoint{34.902987\du}{9.380598\du}}
\pgfpathlineto{\pgfpoint{34.904893\du}{9.386793\du}}
\pgfpathlineto{\pgfpoint{34.907752\du}{9.391559\du}}
\pgfpathlineto{\pgfpoint{34.911088\du}{9.396325\du}}
\pgfpathlineto{\pgfpoint{34.914901\du}{9.401090\du}}
\pgfpathlineto{\pgfpoint{34.918237\du}{9.405856\du}}
\pgfpathlineto{\pgfpoint{34.922049\du}{9.410145\du}}
\pgfpathlineto{\pgfpoint{34.925862\du}{9.413481\du}}
\pgfpathlineto{\pgfpoint{34.930627\du}{9.417770\du}}
\pgfpathlineto{\pgfpoint{34.934916\du}{9.422059\du}}
\pgfpathlineto{\pgfpoint{34.939682\du}{9.424918\du}}
\pgfpathlineto{\pgfpoint{34.944924\du}{9.428254\du}}
\pgfpathlineto{\pgfpoint{34.949213\du}{9.431590\du}}
\pgfpathlineto{\pgfpoint{34.954456\du}{9.433973\du}}
\pgfpathlineto{\pgfpoint{34.959698\du}{9.436356\du}}
\pgfpathlineto{\pgfpoint{34.965417\du}{9.438739\du}}
\pgfpathlineto{\pgfpoint{34.970659\du}{9.440168\du}}
\pgfpathlineto{\pgfpoint{34.975901\du}{9.442075\du}}
\pgfpathlineto{\pgfpoint{34.982096\du}{9.443504\du}}
\pgfpathlineto{\pgfpoint{34.987338\du}{9.444458\du}}
\pgfpathlineto{\pgfpoint{34.993057\du}{9.445411\du}}
\pgfpathlineto{\pgfpoint{34.998776\du}{9.445887\du}}
\pgfpathlineto{\pgfpoint{35.004495\du}{9.445887\du}}
\pgfpathlineto{\pgfpoint{35.004495\du}{9.445887\du}}
\pgfpathlineto{\pgfpoint{35.010690\du}{9.445887\du}}
\pgfpathlineto{\pgfpoint{35.016409\du}{9.445411\du}}
\pgfpathlineto{\pgfpoint{35.022128\du}{9.444458\du}}
\pgfpathlineto{\pgfpoint{35.027370\du}{9.443504\du}}
\pgfpathlineto{\pgfpoint{35.033089\du}{9.442075\du}}
\pgfpathlineto{\pgfpoint{35.038331\du}{9.440168\du}}
\pgfpathlineto{\pgfpoint{35.044049\du}{9.438739\du}}
\pgfpathlineto{\pgfpoint{35.049292\du}{9.436356\du}}
\pgfpathlineto{\pgfpoint{35.055010\du}{9.433973\du}}
\pgfpathlineto{\pgfpoint{35.059776\du}{9.431590\du}}
\pgfpathlineto{\pgfpoint{35.064542\du}{9.428254\du}}
\pgfpathlineto{\pgfpoint{35.069784\du}{9.424918\du}}
\pgfpathlineto{\pgfpoint{35.074073\du}{9.422059\du}}
\pgfpathlineto{\pgfpoint{35.078839\du}{9.417770\du}}
\pgfpathlineto{\pgfpoint{35.083128\du}{9.413481\du}}
\pgfpathlineto{\pgfpoint{35.086940\du}{9.410145\du}}
\pgfpathlineto{\pgfpoint{35.091229\du}{9.405856\du}}
\pgfpathlineto{\pgfpoint{35.094565\du}{9.401090\du}}
\pgfpathlineto{\pgfpoint{35.098378\du}{9.396325\du}}
\pgfpathlineto{\pgfpoint{35.101237\du}{9.391559\du}}
\pgfpathlineto{\pgfpoint{35.104096\du}{9.386793\du}}
\pgfpathlineto{\pgfpoint{35.106479\du}{9.380598\du}}
\pgfpathlineto{\pgfpoint{35.109339\du}{9.375832\du}}
\pgfpathlineto{\pgfpoint{35.111722\du}{9.370590\du}}
\pgfpathlineto{\pgfpoint{35.113151\du}{9.364871\du}}
\pgfpathlineto{\pgfpoint{35.115057\du}{9.359153\du}}
\pgfpathlineto{\pgfpoint{35.116011\du}{9.353910\du}}
\pgfpathlineto{\pgfpoint{35.117440\du}{9.347715\du}}
\pgfpathlineto{\pgfpoint{35.117917\du}{9.341996\du}}
\pgfpathlineto{\pgfpoint{35.118393\du}{9.335801\du}}
\pgfpathlineto{\pgfpoint{35.118393\du}{9.330082\du}}
\pgfusepath{fill}
\pgfsetbuttcap
\pgfsetmiterjoin
\pgfsetdash{}{0pt}
\definecolor{dialinecolor}{rgb}{1.000000, 1.000000, 1.000000}
\pgfsetfillcolor{dialinecolor}
\pgfpathmoveto{\pgfpoint{36.159208\du}{9.330082\du}}
\pgfpathlineto{\pgfpoint{36.159208\du}{9.323887\du}}
\pgfpathlineto{\pgfpoint{36.158255\du}{9.318168\du}}
\pgfpathlineto{\pgfpoint{36.157778\du}{9.312926\du}}
\pgfpathlineto{\pgfpoint{36.157302\du}{9.307207\du}}
\pgfpathlineto{\pgfpoint{36.155396\du}{9.301488\du}}
\pgfpathlineto{\pgfpoint{36.153966\du}{9.295770\du}}
\pgfpathlineto{\pgfpoint{36.152060\du}{9.290051\du}}
\pgfpathlineto{\pgfpoint{36.149677\du}{9.284809\du}}
\pgfpathlineto{\pgfpoint{36.147294\du}{9.279567\du}}
\pgfpathlineto{\pgfpoint{36.145388\du}{9.274324\du}}
\pgfpathlineto{\pgfpoint{36.142052\du}{9.269082\du}}
\pgfpathlineto{\pgfpoint{36.139192\du}{9.264316\du}}
\pgfpathlineto{\pgfpoint{36.135380\du}{9.259551\du}}
\pgfpathlineto{\pgfpoint{36.131567\du}{9.254785\du}}
\pgfpathlineto{\pgfpoint{36.127278\du}{9.250973\du}}
\pgfpathlineto{\pgfpoint{36.123466\du}{9.246684\du}}
\pgfpathlineto{\pgfpoint{36.119177\du}{9.242871\du}}
\pgfpathlineto{\pgfpoint{36.115364\du}{9.238582\du}}
\pgfpathlineto{\pgfpoint{36.110122\du}{9.235723\du}}
\pgfpathlineto{\pgfpoint{36.105356\du}{9.232387\du}}
\pgfpathlineto{\pgfpoint{36.100591\du}{9.228574\du}}
\pgfpathlineto{\pgfpoint{36.095349\du}{9.226668\du}}
\pgfpathlineto{\pgfpoint{36.090106\du}{9.224285\du}}
\pgfpathlineto{\pgfpoint{36.084388\du}{9.221902\du}}
\pgfpathlineto{\pgfpoint{36.078669\du}{9.219996\du}}
\pgfpathlineto{\pgfpoint{36.073427\du}{9.218566\du}}
\pgfpathlineto{\pgfpoint{36.067708\du}{9.216660\du}}
\pgfpathlineto{\pgfpoint{36.061989\du}{9.216184\du}}
\pgfpathlineto{\pgfpoint{36.055794\du}{9.215230\du}}
\pgfpathlineto{\pgfpoint{36.050552\du}{9.214754\du}}
\pgfpathlineto{\pgfpoint{36.044356\du}{9.214754\du}}
\pgfpathlineto{\pgfpoint{36.044356\du}{9.214754\du}}
\pgfpathlineto{\pgfpoint{36.038638\du}{9.214754\du}}
\pgfpathlineto{\pgfpoint{36.033395\du}{9.215230\du}}
\pgfpathlineto{\pgfpoint{36.027200\du}{9.216184\du}}
\pgfpathlineto{\pgfpoint{36.021481\du}{9.216660\du}}
\pgfpathlineto{\pgfpoint{36.015763\du}{9.218566\du}}
\pgfpathlineto{\pgfpoint{36.010997\du}{9.219996\du}}
\pgfpathlineto{\pgfpoint{36.005278\du}{9.221902\du}}
\pgfpathlineto{\pgfpoint{35.999083\du}{9.224285\du}}
\pgfpathlineto{\pgfpoint{35.993841\du}{9.226668\du}}
\pgfpathlineto{\pgfpoint{35.989075\du}{9.228574\du}}
\pgfpathlineto{\pgfpoint{35.984309\du}{9.232387\du}}
\pgfpathlineto{\pgfpoint{35.979067\du}{9.235723\du}}
\pgfpathlineto{\pgfpoint{35.974301\du}{9.238582\du}}
\pgfpathlineto{\pgfpoint{35.970012\du}{9.242871\du}}
\pgfpathlineto{\pgfpoint{35.966200\du}{9.246684\du}}
\pgfpathlineto{\pgfpoint{35.961434\du}{9.250973\du}}
\pgfpathlineto{\pgfpoint{35.957622\du}{9.254785\du}}
\pgfpathlineto{\pgfpoint{35.954286\du}{9.259551\du}}
\pgfpathlineto{\pgfpoint{35.950473\du}{9.264316\du}}
\pgfpathlineto{\pgfpoint{35.947137\du}{9.269082\du}}
\pgfpathlineto{\pgfpoint{35.944278\du}{9.274324\du}}
\pgfpathlineto{\pgfpoint{35.942372\du}{9.279567\du}}
\pgfpathlineto{\pgfpoint{35.939512\du}{9.284809\du}}
\pgfpathlineto{\pgfpoint{35.937130\du}{9.290051\du}}
\pgfpathlineto{\pgfpoint{35.935223\du}{9.295770\du}}
\pgfpathlineto{\pgfpoint{35.933794\du}{9.301488\du}}
\pgfpathlineto{\pgfpoint{35.932364\du}{9.307207\du}}
\pgfpathlineto{\pgfpoint{35.931411\du}{9.312926\du}}
\pgfpathlineto{\pgfpoint{35.930934\du}{9.318168\du}}
\pgfpathlineto{\pgfpoint{35.930458\du}{9.323887\du}}
\pgfpathlineto{\pgfpoint{35.930458\du}{9.330082\du}}
\pgfpathlineto{\pgfpoint{35.930458\du}{9.330082\du}}
\pgfpathlineto{\pgfpoint{35.930458\du}{9.335801\du}}
\pgfpathlineto{\pgfpoint{35.930934\du}{9.341996\du}}
\pgfpathlineto{\pgfpoint{35.931411\du}{9.347715\du}}
\pgfpathlineto{\pgfpoint{35.932364\du}{9.353910\du}}
\pgfpathlineto{\pgfpoint{35.933794\du}{9.359153\du}}
\pgfpathlineto{\pgfpoint{35.935223\du}{9.364871\du}}
\pgfpathlineto{\pgfpoint{35.937130\du}{9.370590\du}}
\pgfpathlineto{\pgfpoint{35.939512\du}{9.375832\du}}
\pgfpathlineto{\pgfpoint{35.942372\du}{9.380598\du}}
\pgfpathlineto{\pgfpoint{35.944278\du}{9.386793\du}}
\pgfpathlineto{\pgfpoint{35.947137\du}{9.391559\du}}
\pgfpathlineto{\pgfpoint{35.950473\du}{9.396325\du}}
\pgfpathlineto{\pgfpoint{35.954286\du}{9.401090\du}}
\pgfpathlineto{\pgfpoint{35.957622\du}{9.405856\du}}
\pgfpathlineto{\pgfpoint{35.961434\du}{9.410145\du}}
\pgfpathlineto{\pgfpoint{35.966200\du}{9.413481\du}}
\pgfpathlineto{\pgfpoint{35.970012\du}{9.417770\du}}
\pgfpathlineto{\pgfpoint{35.974301\du}{9.422059\du}}
\pgfpathlineto{\pgfpoint{35.979067\du}{9.424918\du}}
\pgfpathlineto{\pgfpoint{35.984309\du}{9.428254\du}}
\pgfpathlineto{\pgfpoint{35.989075\du}{9.431590\du}}
\pgfpathlineto{\pgfpoint{35.993841\du}{9.433973\du}}
\pgfpathlineto{\pgfpoint{35.999083\du}{9.436356\du}}
\pgfpathlineto{\pgfpoint{36.005278\du}{9.438739\du}}
\pgfpathlineto{\pgfpoint{36.010997\du}{9.440168\du}}
\pgfpathlineto{\pgfpoint{36.015763\du}{9.442075\du}}
\pgfpathlineto{\pgfpoint{36.021481\du}{9.443504\du}}
\pgfpathlineto{\pgfpoint{36.027200\du}{9.444458\du}}
\pgfpathlineto{\pgfpoint{36.033395\du}{9.445411\du}}
\pgfpathlineto{\pgfpoint{36.038638\du}{9.445887\du}}
\pgfpathlineto{\pgfpoint{36.044356\du}{9.445887\du}}
\pgfpathlineto{\pgfpoint{36.044356\du}{9.445887\du}}
\pgfpathlineto{\pgfpoint{36.050552\du}{9.445887\du}}
\pgfpathlineto{\pgfpoint{36.055794\du}{9.445411\du}}
\pgfpathlineto{\pgfpoint{36.061989\du}{9.444458\du}}
\pgfpathlineto{\pgfpoint{36.067708\du}{9.443504\du}}
\pgfpathlineto{\pgfpoint{36.073427\du}{9.442075\du}}
\pgfpathlineto{\pgfpoint{36.078669\du}{9.440168\du}}
\pgfpathlineto{\pgfpoint{36.084388\du}{9.438739\du}}
\pgfpathlineto{\pgfpoint{36.090106\du}{9.436356\du}}
\pgfpathlineto{\pgfpoint{36.095349\du}{9.433973\du}}
\pgfpathlineto{\pgfpoint{36.100591\du}{9.431590\du}}
\pgfpathlineto{\pgfpoint{36.105356\du}{9.428254\du}}
\pgfpathlineto{\pgfpoint{36.110122\du}{9.424918\du}}
\pgfpathlineto{\pgfpoint{36.115364\du}{9.422059\du}}
\pgfpathlineto{\pgfpoint{36.119177\du}{9.417770\du}}
\pgfpathlineto{\pgfpoint{36.123466\du}{9.413481\du}}
\pgfpathlineto{\pgfpoint{36.127278\du}{9.410145\du}}
\pgfpathlineto{\pgfpoint{36.131567\du}{9.405856\du}}
\pgfpathlineto{\pgfpoint{36.135380\du}{9.401090\du}}
\pgfpathlineto{\pgfpoint{36.139192\du}{9.396325\du}}
\pgfpathlineto{\pgfpoint{36.142052\du}{9.391559\du}}
\pgfpathlineto{\pgfpoint{36.145388\du}{9.386793\du}}
\pgfpathlineto{\pgfpoint{36.147294\du}{9.380598\du}}
\pgfpathlineto{\pgfpoint{36.149677\du}{9.375832\du}}
\pgfpathlineto{\pgfpoint{36.152060\du}{9.370590\du}}
\pgfpathlineto{\pgfpoint{36.153966\du}{9.364871\du}}
\pgfpathlineto{\pgfpoint{36.155396\du}{9.359153\du}}
\pgfpathlineto{\pgfpoint{36.157302\du}{9.353910\du}}
\pgfpathlineto{\pgfpoint{36.157778\du}{9.347715\du}}
\pgfpathlineto{\pgfpoint{36.158255\du}{9.341996\du}}
\pgfpathlineto{\pgfpoint{36.159208\du}{9.335801\du}}
\pgfpathlineto{\pgfpoint{36.159208\du}{9.330082\du}}
\pgfusepath{fill}
\pgfsetlinewidth{0.000000\du}
\pgfsetbuttcap
\pgfsetmiterjoin
\pgfsetdash{}{0pt}
\definecolor{dialinecolor}{rgb}{1.000000, 1.000000, 1.000000}
\pgfsetstrokecolor{dialinecolor}
\pgfpathmoveto{\pgfpoint{35.015456\du}{9.320074\du}}
\pgfpathlineto{\pgfpoint{36.013856\du}{9.320074\du}}
\pgfusepath{stroke}
\pgfsetlinewidth{0.000000\du}
\pgfsetbuttcap
\pgfsetmiterjoin
\pgfsetdash{}{0pt}
\definecolor{dialinecolor}{rgb}{1.000000, 1.000000, 1.000000}
\pgfsetfillcolor{dialinecolor}
\pgfpathmoveto{\pgfpoint{35.118393\du}{9.751841\du}}
\pgfpathlineto{\pgfpoint{35.118393\du}{9.745646\du}}
\pgfpathlineto{\pgfpoint{35.117917\du}{9.739927\du}}
\pgfpathlineto{\pgfpoint{35.117440\du}{9.734208\du}}
\pgfpathlineto{\pgfpoint{35.116011\du}{9.728489\du}}
\pgfpathlineto{\pgfpoint{35.115057\du}{9.722294\du}}
\pgfpathlineto{\pgfpoint{35.113151\du}{9.717052\du}}
\pgfpathlineto{\pgfpoint{35.111722\du}{9.711333\du}}
\pgfpathlineto{\pgfpoint{35.109339\du}{9.705614\du}}
\pgfpathlineto{\pgfpoint{35.106479\du}{9.700849\du}}
\pgfpathlineto{\pgfpoint{35.104096\du}{9.695606\du}}
\pgfpathlineto{\pgfpoint{35.101237\du}{9.689888\du}}
\pgfpathlineto{\pgfpoint{35.098378\du}{9.685122\du}}
\pgfpathlineto{\pgfpoint{35.094565\du}{9.680356\du}}
\pgfpathlineto{\pgfpoint{35.091229\du}{9.676544\du}}
\pgfpathlineto{\pgfpoint{35.086940\du}{9.671302\du}}
\pgfpathlineto{\pgfpoint{35.083128\du}{9.667489\du}}
\pgfpathlineto{\pgfpoint{35.078839\du}{9.663677\du}}
\pgfpathlineto{\pgfpoint{35.074073\du}{9.660341\du}}
\pgfpathlineto{\pgfpoint{35.069784\du}{9.656528\du}}
\pgfpathlineto{\pgfpoint{35.064542\du}{9.653192\du}}
\pgfpathlineto{\pgfpoint{35.059776\du}{9.649856\du}}
\pgfpathlineto{\pgfpoint{35.055010\du}{9.647474\du}}
\pgfpathlineto{\pgfpoint{35.049292\du}{9.645091\du}}
\pgfpathlineto{\pgfpoint{35.044049\du}{9.642708\du}}
\pgfpathlineto{\pgfpoint{35.038331\du}{9.640802\du}}
\pgfpathlineto{\pgfpoint{35.033089\du}{9.639372\du}}
\pgfpathlineto{\pgfpoint{35.027370\du}{9.637942\du}}
\pgfpathlineto{\pgfpoint{35.022128\du}{9.636989\du}}
\pgfpathlineto{\pgfpoint{35.016409\du}{9.636036\du}}
\pgfpathlineto{\pgfpoint{35.010690\du}{9.635559\du}}
\pgfpathlineto{\pgfpoint{35.004495\du}{9.635559\du}}
\pgfpathlineto{\pgfpoint{35.004495\du}{9.635559\du}}
\pgfpathlineto{\pgfpoint{34.998776\du}{9.635559\du}}
\pgfpathlineto{\pgfpoint{34.993057\du}{9.636036\du}}
\pgfpathlineto{\pgfpoint{34.987338\du}{9.636989\du}}
\pgfpathlineto{\pgfpoint{34.982096\du}{9.637942\du}}
\pgfpathlineto{\pgfpoint{34.975901\du}{9.639372\du}}
\pgfpathlineto{\pgfpoint{34.970659\du}{9.640802\du}}
\pgfpathlineto{\pgfpoint{34.965417\du}{9.642708\du}}
\pgfpathlineto{\pgfpoint{34.959698\du}{9.645091\du}}
\pgfpathlineto{\pgfpoint{34.954456\du}{9.647474\du}}
\pgfpathlineto{\pgfpoint{34.949213\du}{9.649856\du}}
\pgfpathlineto{\pgfpoint{34.944924\du}{9.653192\du}}
\pgfpathlineto{\pgfpoint{34.939682\du}{9.656528\du}}
\pgfpathlineto{\pgfpoint{34.934916\du}{9.660341\du}}
\pgfpathlineto{\pgfpoint{34.930627\du}{9.663677\du}}
\pgfpathlineto{\pgfpoint{34.925862\du}{9.667489\du}}
\pgfpathlineto{\pgfpoint{34.922049\du}{9.671302\du}}
\pgfpathlineto{\pgfpoint{34.918237\du}{9.676544\du}}
\pgfpathlineto{\pgfpoint{34.914901\du}{9.680356\du}}
\pgfpathlineto{\pgfpoint{34.911088\du}{9.685122\du}}
\pgfpathlineto{\pgfpoint{34.907752\du}{9.689888\du}}
\pgfpathlineto{\pgfpoint{34.904893\du}{9.695606\du}}
\pgfpathlineto{\pgfpoint{34.902987\du}{9.700849\du}}
\pgfpathlineto{\pgfpoint{34.900127\du}{9.705614\du}}
\pgfpathlineto{\pgfpoint{34.897744\du}{9.711333\du}}
\pgfpathlineto{\pgfpoint{34.895838\du}{9.717052\du}}
\pgfpathlineto{\pgfpoint{34.894409\du}{9.722294\du}}
\pgfpathlineto{\pgfpoint{34.892979\du}{9.728489\du}}
\pgfpathlineto{\pgfpoint{34.892026\du}{9.734208\du}}
\pgfpathlineto{\pgfpoint{34.891073\du}{9.739927\du}}
\pgfpathlineto{\pgfpoint{34.891073\du}{9.745646\du}}
\pgfpathlineto{\pgfpoint{34.891073\du}{9.751841\du}}
\pgfpathlineto{\pgfpoint{34.891073\du}{9.751841\du}}
\pgfpathlineto{\pgfpoint{34.891073\du}{9.757560\du}}
\pgfpathlineto{\pgfpoint{34.891073\du}{9.763755\du}}
\pgfpathlineto{\pgfpoint{34.892026\du}{9.768997\du}}
\pgfpathlineto{\pgfpoint{34.892979\du}{9.775193\du}}
\pgfpathlineto{\pgfpoint{34.894409\du}{9.780911\du}}
\pgfpathlineto{\pgfpoint{34.895838\du}{9.786630\du}}
\pgfpathlineto{\pgfpoint{34.897744\du}{9.791872\du}}
\pgfpathlineto{\pgfpoint{34.900127\du}{9.797591\du}}
\pgfpathlineto{\pgfpoint{34.902987\du}{9.802833\du}}
\pgfpathlineto{\pgfpoint{34.904893\du}{9.808075\du}}
\pgfpathlineto{\pgfpoint{34.907752\du}{9.813318\du}}
\pgfpathlineto{\pgfpoint{34.911088\du}{9.818083\du}}
\pgfpathlineto{\pgfpoint{34.914901\du}{9.822849\du}}
\pgfpathlineto{\pgfpoint{34.918237\du}{9.827138\du}}
\pgfpathlineto{\pgfpoint{34.922049\du}{9.831904\du}}
\pgfpathlineto{\pgfpoint{34.925862\du}{9.836193\du}}
\pgfpathlineto{\pgfpoint{34.930627\du}{9.839529\du}}
\pgfpathlineto{\pgfpoint{34.934916\du}{9.843341\du}}
\pgfpathlineto{\pgfpoint{34.939682\du}{9.846677\du}}
\pgfpathlineto{\pgfpoint{34.944924\du}{9.850490\du}}
\pgfpathlineto{\pgfpoint{34.949213\du}{9.853349\du}}
\pgfpathlineto{\pgfpoint{34.954456\du}{9.855732\du}}
\pgfpathlineto{\pgfpoint{34.959698\du}{9.858115\du}}
\pgfpathlineto{\pgfpoint{34.965417\du}{9.860497\du}}
\pgfpathlineto{\pgfpoint{34.970659\du}{9.862404\du}}
\pgfpathlineto{\pgfpoint{34.975901\du}{9.864310\du}}
\pgfpathlineto{\pgfpoint{34.982096\du}{9.865263\du}}
\pgfpathlineto{\pgfpoint{34.987338\du}{9.866216\du}}
\pgfpathlineto{\pgfpoint{34.993057\du}{9.867169\du}}
\pgfpathlineto{\pgfpoint{34.998776\du}{9.867646\du}}
\pgfpathlineto{\pgfpoint{35.004495\du}{9.867646\du}}
\pgfpathlineto{\pgfpoint{35.004495\du}{9.867646\du}}
\pgfpathlineto{\pgfpoint{35.010690\du}{9.867646\du}}
\pgfpathlineto{\pgfpoint{35.016409\du}{9.867169\du}}
\pgfpathlineto{\pgfpoint{35.022128\du}{9.866216\du}}
\pgfpathlineto{\pgfpoint{35.027370\du}{9.865263\du}}
\pgfpathlineto{\pgfpoint{35.033089\du}{9.864310\du}}
\pgfpathlineto{\pgfpoint{35.038331\du}{9.862404\du}}
\pgfpathlineto{\pgfpoint{35.044049\du}{9.860497\du}}
\pgfpathlineto{\pgfpoint{35.049292\du}{9.858115\du}}
\pgfpathlineto{\pgfpoint{35.055010\du}{9.855732\du}}
\pgfpathlineto{\pgfpoint{35.059776\du}{9.853349\du}}
\pgfpathlineto{\pgfpoint{35.064542\du}{9.850490\du}}
\pgfpathlineto{\pgfpoint{35.069784\du}{9.846677\du}}
\pgfpathlineto{\pgfpoint{35.074073\du}{9.843341\du}}
\pgfpathlineto{\pgfpoint{35.078839\du}{9.839529\du}}
\pgfpathlineto{\pgfpoint{35.083128\du}{9.836193\du}}
\pgfpathlineto{\pgfpoint{35.086940\du}{9.831904\du}}
\pgfpathlineto{\pgfpoint{35.091229\du}{9.827138\du}}
\pgfpathlineto{\pgfpoint{35.094565\du}{9.822849\du}}
\pgfpathlineto{\pgfpoint{35.098378\du}{9.818083\du}}
\pgfpathlineto{\pgfpoint{35.101237\du}{9.813318\du}}
\pgfpathlineto{\pgfpoint{35.104096\du}{9.808075\du}}
\pgfpathlineto{\pgfpoint{35.106479\du}{9.802833\du}}
\pgfpathlineto{\pgfpoint{35.109339\du}{9.797591\du}}
\pgfpathlineto{\pgfpoint{35.111722\du}{9.791872\du}}
\pgfpathlineto{\pgfpoint{35.113151\du}{9.786630\du}}
\pgfpathlineto{\pgfpoint{35.115057\du}{9.780911\du}}
\pgfpathlineto{\pgfpoint{35.116011\du}{9.775193\du}}
\pgfpathlineto{\pgfpoint{35.117440\du}{9.768997\du}}
\pgfpathlineto{\pgfpoint{35.117917\du}{9.763755\du}}
\pgfpathlineto{\pgfpoint{35.118393\du}{9.757560\du}}
\pgfpathlineto{\pgfpoint{35.118393\du}{9.751841\du}}
\pgfusepath{fill}
\pgfsetbuttcap
\pgfsetmiterjoin
\pgfsetdash{}{0pt}
\definecolor{dialinecolor}{rgb}{1.000000, 1.000000, 1.000000}
\pgfsetfillcolor{dialinecolor}
\pgfpathmoveto{\pgfpoint{36.159208\du}{9.751841\du}}
\pgfpathlineto{\pgfpoint{36.159208\du}{9.745646\du}}
\pgfpathlineto{\pgfpoint{36.158255\du}{9.739927\du}}
\pgfpathlineto{\pgfpoint{36.157778\du}{9.734208\du}}
\pgfpathlineto{\pgfpoint{36.157302\du}{9.728489\du}}
\pgfpathlineto{\pgfpoint{36.155396\du}{9.722294\du}}
\pgfpathlineto{\pgfpoint{36.153966\du}{9.717052\du}}
\pgfpathlineto{\pgfpoint{36.152060\du}{9.711333\du}}
\pgfpathlineto{\pgfpoint{36.149677\du}{9.705614\du}}
\pgfpathlineto{\pgfpoint{36.147294\du}{9.700849\du}}
\pgfpathlineto{\pgfpoint{36.145388\du}{9.695606\du}}
\pgfpathlineto{\pgfpoint{36.142052\du}{9.689888\du}}
\pgfpathlineto{\pgfpoint{36.139192\du}{9.685122\du}}
\pgfpathlineto{\pgfpoint{36.135380\du}{9.680356\du}}
\pgfpathlineto{\pgfpoint{36.131567\du}{9.676544\du}}
\pgfpathlineto{\pgfpoint{36.127278\du}{9.671302\du}}
\pgfpathlineto{\pgfpoint{36.123466\du}{9.667489\du}}
\pgfpathlineto{\pgfpoint{36.119177\du}{9.663677\du}}
\pgfpathlineto{\pgfpoint{36.115364\du}{9.660341\du}}
\pgfpathlineto{\pgfpoint{36.110122\du}{9.656528\du}}
\pgfpathlineto{\pgfpoint{36.105356\du}{9.653192\du}}
\pgfpathlineto{\pgfpoint{36.100591\du}{9.649856\du}}
\pgfpathlineto{\pgfpoint{36.095349\du}{9.647474\du}}
\pgfpathlineto{\pgfpoint{36.090106\du}{9.645091\du}}
\pgfpathlineto{\pgfpoint{36.084388\du}{9.642708\du}}
\pgfpathlineto{\pgfpoint{36.078669\du}{9.640802\du}}
\pgfpathlineto{\pgfpoint{36.073427\du}{9.639372\du}}
\pgfpathlineto{\pgfpoint{36.067708\du}{9.637942\du}}
\pgfpathlineto{\pgfpoint{36.061989\du}{9.636989\du}}
\pgfpathlineto{\pgfpoint{36.055794\du}{9.636036\du}}
\pgfpathlineto{\pgfpoint{36.050552\du}{9.635559\du}}
\pgfpathlineto{\pgfpoint{36.044356\du}{9.635559\du}}
\pgfpathlineto{\pgfpoint{36.044356\du}{9.635559\du}}
\pgfpathlineto{\pgfpoint{36.038638\du}{9.635559\du}}
\pgfpathlineto{\pgfpoint{36.033395\du}{9.636036\du}}
\pgfpathlineto{\pgfpoint{36.027200\du}{9.636989\du}}
\pgfpathlineto{\pgfpoint{36.021481\du}{9.637942\du}}
\pgfpathlineto{\pgfpoint{36.015763\du}{9.639372\du}}
\pgfpathlineto{\pgfpoint{36.010997\du}{9.640802\du}}
\pgfpathlineto{\pgfpoint{36.005278\du}{9.642708\du}}
\pgfpathlineto{\pgfpoint{35.999083\du}{9.645091\du}}
\pgfpathlineto{\pgfpoint{35.993841\du}{9.647474\du}}
\pgfpathlineto{\pgfpoint{35.989075\du}{9.649856\du}}
\pgfpathlineto{\pgfpoint{35.984309\du}{9.653192\du}}
\pgfpathlineto{\pgfpoint{35.979067\du}{9.656528\du}}
\pgfpathlineto{\pgfpoint{35.974301\du}{9.660341\du}}
\pgfpathlineto{\pgfpoint{35.970012\du}{9.663677\du}}
\pgfpathlineto{\pgfpoint{35.966200\du}{9.667489\du}}
\pgfpathlineto{\pgfpoint{35.961434\du}{9.671302\du}}
\pgfpathlineto{\pgfpoint{35.957622\du}{9.676544\du}}
\pgfpathlineto{\pgfpoint{35.954286\du}{9.680356\du}}
\pgfpathlineto{\pgfpoint{35.950473\du}{9.685122\du}}
\pgfpathlineto{\pgfpoint{35.947137\du}{9.689888\du}}
\pgfpathlineto{\pgfpoint{35.944278\du}{9.695606\du}}
\pgfpathlineto{\pgfpoint{35.942372\du}{9.700849\du}}
\pgfpathlineto{\pgfpoint{35.939512\du}{9.705614\du}}
\pgfpathlineto{\pgfpoint{35.937130\du}{9.711333\du}}
\pgfpathlineto{\pgfpoint{35.935223\du}{9.717052\du}}
\pgfpathlineto{\pgfpoint{35.933794\du}{9.722294\du}}
\pgfpathlineto{\pgfpoint{35.932364\du}{9.728489\du}}
\pgfpathlineto{\pgfpoint{35.931411\du}{9.734208\du}}
\pgfpathlineto{\pgfpoint{35.930934\du}{9.739927\du}}
\pgfpathlineto{\pgfpoint{35.930458\du}{9.745646\du}}
\pgfpathlineto{\pgfpoint{35.930458\du}{9.751841\du}}
\pgfpathlineto{\pgfpoint{35.930458\du}{9.751841\du}}
\pgfpathlineto{\pgfpoint{35.930458\du}{9.757560\du}}
\pgfpathlineto{\pgfpoint{35.930934\du}{9.763755\du}}
\pgfpathlineto{\pgfpoint{35.931411\du}{9.768997\du}}
\pgfpathlineto{\pgfpoint{35.932364\du}{9.775193\du}}
\pgfpathlineto{\pgfpoint{35.933794\du}{9.780911\du}}
\pgfpathlineto{\pgfpoint{35.935223\du}{9.786630\du}}
\pgfpathlineto{\pgfpoint{35.937130\du}{9.791872\du}}
\pgfpathlineto{\pgfpoint{35.939512\du}{9.797591\du}}
\pgfpathlineto{\pgfpoint{35.942372\du}{9.802833\du}}
\pgfpathlineto{\pgfpoint{35.944278\du}{9.808075\du}}
\pgfpathlineto{\pgfpoint{35.947137\du}{9.813318\du}}
\pgfpathlineto{\pgfpoint{35.950473\du}{9.818083\du}}
\pgfpathlineto{\pgfpoint{35.954286\du}{9.822849\du}}
\pgfpathlineto{\pgfpoint{35.957622\du}{9.827138\du}}
\pgfpathlineto{\pgfpoint{35.961434\du}{9.831904\du}}
\pgfpathlineto{\pgfpoint{35.966200\du}{9.836193\du}}
\pgfpathlineto{\pgfpoint{35.970012\du}{9.839529\du}}
\pgfpathlineto{\pgfpoint{35.974301\du}{9.843341\du}}
\pgfpathlineto{\pgfpoint{35.979067\du}{9.846677\du}}
\pgfpathlineto{\pgfpoint{35.984309\du}{9.850490\du}}
\pgfpathlineto{\pgfpoint{35.989075\du}{9.853349\du}}
\pgfpathlineto{\pgfpoint{35.993841\du}{9.855732\du}}
\pgfpathlineto{\pgfpoint{35.999083\du}{9.858115\du}}
\pgfpathlineto{\pgfpoint{36.005278\du}{9.860497\du}}
\pgfpathlineto{\pgfpoint{36.010997\du}{9.862404\du}}
\pgfpathlineto{\pgfpoint{36.015763\du}{9.864310\du}}
\pgfpathlineto{\pgfpoint{36.021481\du}{9.865263\du}}
\pgfpathlineto{\pgfpoint{36.027200\du}{9.866216\du}}
\pgfpathlineto{\pgfpoint{36.033395\du}{9.867169\du}}
\pgfpathlineto{\pgfpoint{36.038638\du}{9.867646\du}}
\pgfpathlineto{\pgfpoint{36.044356\du}{9.867646\du}}
\pgfpathlineto{\pgfpoint{36.044356\du}{9.867646\du}}
\pgfpathlineto{\pgfpoint{36.050552\du}{9.867646\du}}
\pgfpathlineto{\pgfpoint{36.055794\du}{9.867169\du}}
\pgfpathlineto{\pgfpoint{36.061989\du}{9.866216\du}}
\pgfpathlineto{\pgfpoint{36.067708\du}{9.865263\du}}
\pgfpathlineto{\pgfpoint{36.073427\du}{9.864310\du}}
\pgfpathlineto{\pgfpoint{36.078669\du}{9.862404\du}}
\pgfpathlineto{\pgfpoint{36.084388\du}{9.860497\du}}
\pgfpathlineto{\pgfpoint{36.090106\du}{9.858115\du}}
\pgfpathlineto{\pgfpoint{36.095349\du}{9.855732\du}}
\pgfpathlineto{\pgfpoint{36.100591\du}{9.853349\du}}
\pgfpathlineto{\pgfpoint{36.105356\du}{9.850490\du}}
\pgfpathlineto{\pgfpoint{36.110122\du}{9.846677\du}}
\pgfpathlineto{\pgfpoint{36.115364\du}{9.843341\du}}
\pgfpathlineto{\pgfpoint{36.119177\du}{9.839529\du}}
\pgfpathlineto{\pgfpoint{36.123466\du}{9.836193\du}}
\pgfpathlineto{\pgfpoint{36.127278\du}{9.831904\du}}
\pgfpathlineto{\pgfpoint{36.131567\du}{9.827138\du}}
\pgfpathlineto{\pgfpoint{36.135380\du}{9.822849\du}}
\pgfpathlineto{\pgfpoint{36.139192\du}{9.818083\du}}
\pgfpathlineto{\pgfpoint{36.142052\du}{9.813318\du}}
\pgfpathlineto{\pgfpoint{36.145388\du}{9.808075\du}}
\pgfpathlineto{\pgfpoint{36.147294\du}{9.802833\du}}
\pgfpathlineto{\pgfpoint{36.149677\du}{9.797591\du}}
\pgfpathlineto{\pgfpoint{36.152060\du}{9.791872\du}}
\pgfpathlineto{\pgfpoint{36.153966\du}{9.786630\du}}
\pgfpathlineto{\pgfpoint{36.155396\du}{9.780911\du}}
\pgfpathlineto{\pgfpoint{36.157302\du}{9.775193\du}}
\pgfpathlineto{\pgfpoint{36.157778\du}{9.768997\du}}
\pgfpathlineto{\pgfpoint{36.158255\du}{9.763755\du}}
\pgfpathlineto{\pgfpoint{36.159208\du}{9.757560\du}}
\pgfpathlineto{\pgfpoint{36.159208\du}{9.751841\du}}
\pgfusepath{fill}
\pgfsetlinewidth{0.000000\du}
\pgfsetbuttcap
\pgfsetmiterjoin
\pgfsetdash{}{0pt}
\definecolor{dialinecolor}{rgb}{1.000000, 1.000000, 1.000000}
\pgfsetstrokecolor{dialinecolor}
\pgfpathmoveto{\pgfpoint{35.015456\du}{9.742786\du}}
\pgfpathlineto{\pgfpoint{36.013856\du}{9.742786\du}}
\pgfusepath{stroke}
\pgfsetlinewidth{0.000000\du}
\pgfsetbuttcap
\pgfsetmiterjoin
\pgfsetdash{}{0pt}
\definecolor{dialinecolor}{rgb}{1.000000, 1.000000, 1.000000}
\pgfsetfillcolor{dialinecolor}
\pgfpathmoveto{\pgfpoint{35.118393\du}{10.173600\du}}
\pgfpathlineto{\pgfpoint{35.118393\du}{10.167881\du}}
\pgfpathlineto{\pgfpoint{35.117917\du}{10.161686\du}}
\pgfpathlineto{\pgfpoint{35.117440\du}{10.156443\du}}
\pgfpathlineto{\pgfpoint{35.116011\du}{10.150248\du}}
\pgfpathlineto{\pgfpoint{35.115057\du}{10.145006\du}}
\pgfpathlineto{\pgfpoint{35.113151\du}{10.138811\du}}
\pgfpathlineto{\pgfpoint{35.111722\du}{10.133568\du}}
\pgfpathlineto{\pgfpoint{35.109339\du}{10.128326\du}}
\pgfpathlineto{\pgfpoint{35.106479\du}{10.122607\du}}
\pgfpathlineto{\pgfpoint{35.104096\du}{10.117365\du}}
\pgfpathlineto{\pgfpoint{35.101237\du}{10.112600\du}}
\pgfpathlineto{\pgfpoint{35.098378\du}{10.107357\du}}
\pgfpathlineto{\pgfpoint{35.094565\du}{10.103068\du}}
\pgfpathlineto{\pgfpoint{35.091229\du}{10.098303\du}}
\pgfpathlineto{\pgfpoint{35.086940\du}{10.094014\du}}
\pgfpathlineto{\pgfpoint{35.083128\du}{10.089725\du}}
\pgfpathlineto{\pgfpoint{35.078839\du}{10.086389\du}}
\pgfpathlineto{\pgfpoint{35.074073\du}{10.082099\du}}
\pgfpathlineto{\pgfpoint{35.069784\du}{10.079240\du}}
\pgfpathlineto{\pgfpoint{35.064542\du}{10.075428\du}}
\pgfpathlineto{\pgfpoint{35.059776\du}{10.072568\du}}
\pgfpathlineto{\pgfpoint{35.055010\du}{10.070185\du}}
\pgfpathlineto{\pgfpoint{35.049292\du}{10.067803\du}}
\pgfpathlineto{\pgfpoint{35.044049\du}{10.065420\du}}
\pgfpathlineto{\pgfpoint{35.038331\du}{10.063514\du}}
\pgfpathlineto{\pgfpoint{35.033089\du}{10.061607\du}}
\pgfpathlineto{\pgfpoint{35.027370\du}{10.060654\du}}
\pgfpathlineto{\pgfpoint{35.022128\du}{10.059224\du}}
\pgfpathlineto{\pgfpoint{35.016409\du}{10.058748\du}}
\pgfpathlineto{\pgfpoint{35.010690\du}{10.058271\du}}
\pgfpathlineto{\pgfpoint{35.004495\du}{10.058271\du}}
\pgfpathlineto{\pgfpoint{35.004495\du}{10.058271\du}}
\pgfpathlineto{\pgfpoint{34.998776\du}{10.058271\du}}
\pgfpathlineto{\pgfpoint{34.993057\du}{10.058748\du}}
\pgfpathlineto{\pgfpoint{34.987338\du}{10.059224\du}}
\pgfpathlineto{\pgfpoint{34.982096\du}{10.060654\du}}
\pgfpathlineto{\pgfpoint{34.975901\du}{10.061607\du}}
\pgfpathlineto{\pgfpoint{34.970659\du}{10.063514\du}}
\pgfpathlineto{\pgfpoint{34.965417\du}{10.065420\du}}
\pgfpathlineto{\pgfpoint{34.959698\du}{10.067803\du}}
\pgfpathlineto{\pgfpoint{34.954456\du}{10.070185\du}}
\pgfpathlineto{\pgfpoint{34.949213\du}{10.072568\du}}
\pgfpathlineto{\pgfpoint{34.944924\du}{10.075428\du}}
\pgfpathlineto{\pgfpoint{34.939682\du}{10.079240\du}}
\pgfpathlineto{\pgfpoint{34.934916\du}{10.082099\du}}
\pgfpathlineto{\pgfpoint{34.930627\du}{10.086389\du}}
\pgfpathlineto{\pgfpoint{34.925862\du}{10.089725\du}}
\pgfpathlineto{\pgfpoint{34.922049\du}{10.094014\du}}
\pgfpathlineto{\pgfpoint{34.918237\du}{10.098303\du}}
\pgfpathlineto{\pgfpoint{34.914901\du}{10.103068\du}}
\pgfpathlineto{\pgfpoint{34.911088\du}{10.107357\du}}
\pgfpathlineto{\pgfpoint{34.907752\du}{10.112600\du}}
\pgfpathlineto{\pgfpoint{34.904893\du}{10.117365\du}}
\pgfpathlineto{\pgfpoint{34.902987\du}{10.122607\du}}
\pgfpathlineto{\pgfpoint{34.900127\du}{10.128326\du}}
\pgfpathlineto{\pgfpoint{34.897744\du}{10.133568\du}}
\pgfpathlineto{\pgfpoint{34.895838\du}{10.138811\du}}
\pgfpathlineto{\pgfpoint{34.894409\du}{10.145006\du}}
\pgfpathlineto{\pgfpoint{34.892979\du}{10.150248\du}}
\pgfpathlineto{\pgfpoint{34.892026\du}{10.156443\du}}
\pgfpathlineto{\pgfpoint{34.891073\du}{10.161686\du}}
\pgfpathlineto{\pgfpoint{34.891073\du}{10.167881\du}}
\pgfpathlineto{\pgfpoint{34.891073\du}{10.173600\du}}
\pgfpathlineto{\pgfpoint{34.891073\du}{10.173600\du}}
\pgfpathlineto{\pgfpoint{34.891073\du}{10.179795\du}}
\pgfpathlineto{\pgfpoint{34.891073\du}{10.185514\du}}
\pgfpathlineto{\pgfpoint{34.892026\du}{10.190756\du}}
\pgfpathlineto{\pgfpoint{34.892979\du}{10.196951\du}}
\pgfpathlineto{\pgfpoint{34.894409\du}{10.202670\du}}
\pgfpathlineto{\pgfpoint{34.895838\du}{10.208389\du}}
\pgfpathlineto{\pgfpoint{34.897744\du}{10.213631\du}}
\pgfpathlineto{\pgfpoint{34.900127\du}{10.219350\du}}
\pgfpathlineto{\pgfpoint{34.902987\du}{10.224592\du}}
\pgfpathlineto{\pgfpoint{34.904893\du}{10.229834\du}}
\pgfpathlineto{\pgfpoint{34.907752\du}{10.234600\du}}
\pgfpathlineto{\pgfpoint{34.911088\du}{10.239365\du}}
\pgfpathlineto{\pgfpoint{34.914901\du}{10.244131\du}}
\pgfpathlineto{\pgfpoint{34.918237\du}{10.249373\du}}
\pgfpathlineto{\pgfpoint{34.922049\du}{10.253186\du}}
\pgfpathlineto{\pgfpoint{34.925862\du}{10.257475\du}}
\pgfpathlineto{\pgfpoint{34.930627\du}{10.261287\du}}
\pgfpathlineto{\pgfpoint{34.934916\du}{10.265100\du}}
\pgfpathlineto{\pgfpoint{34.939682\du}{10.268436\du}}
\pgfpathlineto{\pgfpoint{34.944924\du}{10.271295\du}}
\pgfpathlineto{\pgfpoint{34.949213\du}{10.275108\du}}
\pgfpathlineto{\pgfpoint{34.954456\du}{10.277491\du}}
\pgfpathlineto{\pgfpoint{34.959698\du}{10.279873\du}}
\pgfpathlineto{\pgfpoint{34.965417\du}{10.282256\du}}
\pgfpathlineto{\pgfpoint{34.970659\du}{10.283209\du}}
\pgfpathlineto{\pgfpoint{34.975901\du}{10.285592\du}}
\pgfpathlineto{\pgfpoint{34.982096\du}{10.287022\du}}
\pgfpathlineto{\pgfpoint{34.987338\du}{10.287975\du}}
\pgfpathlineto{\pgfpoint{34.993057\du}{10.288451\du}}
\pgfpathlineto{\pgfpoint{34.998776\du}{10.289405\du}}
\pgfpathlineto{\pgfpoint{35.004495\du}{10.289405\du}}
\pgfpathlineto{\pgfpoint{35.004495\du}{10.289405\du}}
\pgfpathlineto{\pgfpoint{35.010690\du}{10.289405\du}}
\pgfpathlineto{\pgfpoint{35.016409\du}{10.288451\du}}
\pgfpathlineto{\pgfpoint{35.022128\du}{10.287975\du}}
\pgfpathlineto{\pgfpoint{35.027370\du}{10.287022\du}}
\pgfpathlineto{\pgfpoint{35.033089\du}{10.285592\du}}
\pgfpathlineto{\pgfpoint{35.038331\du}{10.283209\du}}
\pgfpathlineto{\pgfpoint{35.044049\du}{10.282256\du}}
\pgfpathlineto{\pgfpoint{35.049292\du}{10.279873\du}}
\pgfpathlineto{\pgfpoint{35.055010\du}{10.277491\du}}
\pgfpathlineto{\pgfpoint{35.059776\du}{10.275108\du}}
\pgfpathlineto{\pgfpoint{35.064542\du}{10.271295\du}}
\pgfpathlineto{\pgfpoint{35.069784\du}{10.268436\du}}
\pgfpathlineto{\pgfpoint{35.074073\du}{10.265100\du}}
\pgfpathlineto{\pgfpoint{35.078839\du}{10.261287\du}}
\pgfpathlineto{\pgfpoint{35.083128\du}{10.257475\du}}
\pgfpathlineto{\pgfpoint{35.086940\du}{10.253186\du}}
\pgfpathlineto{\pgfpoint{35.091229\du}{10.249373\du}}
\pgfpathlineto{\pgfpoint{35.094565\du}{10.244131\du}}
\pgfpathlineto{\pgfpoint{35.098378\du}{10.239365\du}}
\pgfpathlineto{\pgfpoint{35.101237\du}{10.234600\du}}
\pgfpathlineto{\pgfpoint{35.104096\du}{10.229834\du}}
\pgfpathlineto{\pgfpoint{35.106479\du}{10.224592\du}}
\pgfpathlineto{\pgfpoint{35.109339\du}{10.219350\du}}
\pgfpathlineto{\pgfpoint{35.111722\du}{10.213631\du}}
\pgfpathlineto{\pgfpoint{35.113151\du}{10.208389\du}}
\pgfpathlineto{\pgfpoint{35.115057\du}{10.202670\du}}
\pgfpathlineto{\pgfpoint{35.116011\du}{10.196951\du}}
\pgfpathlineto{\pgfpoint{35.117440\du}{10.190756\du}}
\pgfpathlineto{\pgfpoint{35.117917\du}{10.185514\du}}
\pgfpathlineto{\pgfpoint{35.118393\du}{10.179795\du}}
\pgfpathlineto{\pgfpoint{35.118393\du}{10.173600\du}}
\pgfusepath{fill}
\pgfsetbuttcap
\pgfsetmiterjoin
\pgfsetdash{}{0pt}
\definecolor{dialinecolor}{rgb}{1.000000, 1.000000, 1.000000}
\pgfsetfillcolor{dialinecolor}
\pgfpathmoveto{\pgfpoint{36.159208\du}{10.173600\du}}
\pgfpathlineto{\pgfpoint{36.159208\du}{10.167881\du}}
\pgfpathlineto{\pgfpoint{36.158255\du}{10.161686\du}}
\pgfpathlineto{\pgfpoint{36.157778\du}{10.156443\du}}
\pgfpathlineto{\pgfpoint{36.157302\du}{10.150248\du}}
\pgfpathlineto{\pgfpoint{36.155396\du}{10.145006\du}}
\pgfpathlineto{\pgfpoint{36.153966\du}{10.138811\du}}
\pgfpathlineto{\pgfpoint{36.152060\du}{10.133568\du}}
\pgfpathlineto{\pgfpoint{36.149677\du}{10.128326\du}}
\pgfpathlineto{\pgfpoint{36.147294\du}{10.122607\du}}
\pgfpathlineto{\pgfpoint{36.145388\du}{10.117365\du}}
\pgfpathlineto{\pgfpoint{36.142052\du}{10.112600\du}}
\pgfpathlineto{\pgfpoint{36.139192\du}{10.107357\du}}
\pgfpathlineto{\pgfpoint{36.135380\du}{10.103068\du}}
\pgfpathlineto{\pgfpoint{36.131567\du}{10.098303\du}}
\pgfpathlineto{\pgfpoint{36.127278\du}{10.094014\du}}
\pgfpathlineto{\pgfpoint{36.123466\du}{10.089725\du}}
\pgfpathlineto{\pgfpoint{36.119177\du}{10.086389\du}}
\pgfpathlineto{\pgfpoint{36.115364\du}{10.082099\du}}
\pgfpathlineto{\pgfpoint{36.110122\du}{10.079240\du}}
\pgfpathlineto{\pgfpoint{36.105356\du}{10.075428\du}}
\pgfpathlineto{\pgfpoint{36.100591\du}{10.072568\du}}
\pgfpathlineto{\pgfpoint{36.095349\du}{10.070185\du}}
\pgfpathlineto{\pgfpoint{36.090106\du}{10.067803\du}}
\pgfpathlineto{\pgfpoint{36.084388\du}{10.065420\du}}
\pgfpathlineto{\pgfpoint{36.078669\du}{10.063514\du}}
\pgfpathlineto{\pgfpoint{36.073427\du}{10.061607\du}}
\pgfpathlineto{\pgfpoint{36.067708\du}{10.060654\du}}
\pgfpathlineto{\pgfpoint{36.061989\du}{10.059224\du}}
\pgfpathlineto{\pgfpoint{36.055794\du}{10.058748\du}}
\pgfpathlineto{\pgfpoint{36.050552\du}{10.058271\du}}
\pgfpathlineto{\pgfpoint{36.044356\du}{10.058271\du}}
\pgfpathlineto{\pgfpoint{36.044356\du}{10.058271\du}}
\pgfpathlineto{\pgfpoint{36.038638\du}{10.058271\du}}
\pgfpathlineto{\pgfpoint{36.033395\du}{10.058748\du}}
\pgfpathlineto{\pgfpoint{36.027200\du}{10.059224\du}}
\pgfpathlineto{\pgfpoint{36.021481\du}{10.060654\du}}
\pgfpathlineto{\pgfpoint{36.015763\du}{10.061607\du}}
\pgfpathlineto{\pgfpoint{36.010997\du}{10.063514\du}}
\pgfpathlineto{\pgfpoint{36.005278\du}{10.065420\du}}
\pgfpathlineto{\pgfpoint{35.999083\du}{10.067803\du}}
\pgfpathlineto{\pgfpoint{35.993841\du}{10.070185\du}}
\pgfpathlineto{\pgfpoint{35.989075\du}{10.072568\du}}
\pgfpathlineto{\pgfpoint{35.984309\du}{10.075428\du}}
\pgfpathlineto{\pgfpoint{35.979067\du}{10.079240\du}}
\pgfpathlineto{\pgfpoint{35.974301\du}{10.082099\du}}
\pgfpathlineto{\pgfpoint{35.970012\du}{10.086389\du}}
\pgfpathlineto{\pgfpoint{35.966200\du}{10.089725\du}}
\pgfpathlineto{\pgfpoint{35.961434\du}{10.094014\du}}
\pgfpathlineto{\pgfpoint{35.957622\du}{10.098303\du}}
\pgfpathlineto{\pgfpoint{35.954286\du}{10.103068\du}}
\pgfpathlineto{\pgfpoint{35.950473\du}{10.107357\du}}
\pgfpathlineto{\pgfpoint{35.947137\du}{10.112600\du}}
\pgfpathlineto{\pgfpoint{35.944278\du}{10.117365\du}}
\pgfpathlineto{\pgfpoint{35.942372\du}{10.122607\du}}
\pgfpathlineto{\pgfpoint{35.939512\du}{10.128326\du}}
\pgfpathlineto{\pgfpoint{35.937130\du}{10.133568\du}}
\pgfpathlineto{\pgfpoint{35.935223\du}{10.138811\du}}
\pgfpathlineto{\pgfpoint{35.933794\du}{10.145006\du}}
\pgfpathlineto{\pgfpoint{35.932364\du}{10.150248\du}}
\pgfpathlineto{\pgfpoint{35.931411\du}{10.156443\du}}
\pgfpathlineto{\pgfpoint{35.930934\du}{10.161686\du}}
\pgfpathlineto{\pgfpoint{35.930458\du}{10.167881\du}}
\pgfpathlineto{\pgfpoint{35.930458\du}{10.173600\du}}
\pgfpathlineto{\pgfpoint{35.930458\du}{10.173600\du}}
\pgfpathlineto{\pgfpoint{35.930458\du}{10.179795\du}}
\pgfpathlineto{\pgfpoint{35.930934\du}{10.185514\du}}
\pgfpathlineto{\pgfpoint{35.931411\du}{10.190756\du}}
\pgfpathlineto{\pgfpoint{35.932364\du}{10.196951\du}}
\pgfpathlineto{\pgfpoint{35.933794\du}{10.202670\du}}
\pgfpathlineto{\pgfpoint{35.935223\du}{10.208389\du}}
\pgfpathlineto{\pgfpoint{35.937130\du}{10.213631\du}}
\pgfpathlineto{\pgfpoint{35.939512\du}{10.219350\du}}
\pgfpathlineto{\pgfpoint{35.942372\du}{10.224592\du}}
\pgfpathlineto{\pgfpoint{35.944278\du}{10.229834\du}}
\pgfpathlineto{\pgfpoint{35.947137\du}{10.234600\du}}
\pgfpathlineto{\pgfpoint{35.950473\du}{10.239365\du}}
\pgfpathlineto{\pgfpoint{35.954286\du}{10.244131\du}}
\pgfpathlineto{\pgfpoint{35.957622\du}{10.249373\du}}
\pgfpathlineto{\pgfpoint{35.961434\du}{10.253186\du}}
\pgfpathlineto{\pgfpoint{35.966200\du}{10.257475\du}}
\pgfpathlineto{\pgfpoint{35.970012\du}{10.261287\du}}
\pgfpathlineto{\pgfpoint{35.974301\du}{10.265100\du}}
\pgfpathlineto{\pgfpoint{35.979067\du}{10.268436\du}}
\pgfpathlineto{\pgfpoint{35.984309\du}{10.271295\du}}
\pgfpathlineto{\pgfpoint{35.989075\du}{10.275108\du}}
\pgfpathlineto{\pgfpoint{35.993841\du}{10.277491\du}}
\pgfpathlineto{\pgfpoint{35.999083\du}{10.279873\du}}
\pgfpathlineto{\pgfpoint{36.005278\du}{10.282256\du}}
\pgfpathlineto{\pgfpoint{36.010997\du}{10.283209\du}}
\pgfpathlineto{\pgfpoint{36.015763\du}{10.285592\du}}
\pgfpathlineto{\pgfpoint{36.021481\du}{10.287022\du}}
\pgfpathlineto{\pgfpoint{36.027200\du}{10.287975\du}}
\pgfpathlineto{\pgfpoint{36.033395\du}{10.288451\du}}
\pgfpathlineto{\pgfpoint{36.038638\du}{10.289405\du}}
\pgfpathlineto{\pgfpoint{36.044356\du}{10.289405\du}}
\pgfpathlineto{\pgfpoint{36.044356\du}{10.289405\du}}
\pgfpathlineto{\pgfpoint{36.050552\du}{10.289405\du}}
\pgfpathlineto{\pgfpoint{36.055794\du}{10.288451\du}}
\pgfpathlineto{\pgfpoint{36.061989\du}{10.287975\du}}
\pgfpathlineto{\pgfpoint{36.067708\du}{10.287022\du}}
\pgfpathlineto{\pgfpoint{36.073427\du}{10.285592\du}}
\pgfpathlineto{\pgfpoint{36.078669\du}{10.283209\du}}
\pgfpathlineto{\pgfpoint{36.084388\du}{10.282256\du}}
\pgfpathlineto{\pgfpoint{36.090106\du}{10.279873\du}}
\pgfpathlineto{\pgfpoint{36.095349\du}{10.277491\du}}
\pgfpathlineto{\pgfpoint{36.100591\du}{10.275108\du}}
\pgfpathlineto{\pgfpoint{36.105356\du}{10.271295\du}}
\pgfpathlineto{\pgfpoint{36.110122\du}{10.268436\du}}
\pgfpathlineto{\pgfpoint{36.115364\du}{10.265100\du}}
\pgfpathlineto{\pgfpoint{36.119177\du}{10.261287\du}}
\pgfpathlineto{\pgfpoint{36.123466\du}{10.257475\du}}
\pgfpathlineto{\pgfpoint{36.127278\du}{10.253186\du}}
\pgfpathlineto{\pgfpoint{36.131567\du}{10.249373\du}}
\pgfpathlineto{\pgfpoint{36.135380\du}{10.244131\du}}
\pgfpathlineto{\pgfpoint{36.139192\du}{10.239365\du}}
\pgfpathlineto{\pgfpoint{36.142052\du}{10.234600\du}}
\pgfpathlineto{\pgfpoint{36.145388\du}{10.229834\du}}
\pgfpathlineto{\pgfpoint{36.147294\du}{10.224592\du}}
\pgfpathlineto{\pgfpoint{36.149677\du}{10.219350\du}}
\pgfpathlineto{\pgfpoint{36.152060\du}{10.213631\du}}
\pgfpathlineto{\pgfpoint{36.153966\du}{10.208389\du}}
\pgfpathlineto{\pgfpoint{36.155396\du}{10.202670\du}}
\pgfpathlineto{\pgfpoint{36.157302\du}{10.196951\du}}
\pgfpathlineto{\pgfpoint{36.157778\du}{10.190756\du}}
\pgfpathlineto{\pgfpoint{36.158255\du}{10.185514\du}}
\pgfpathlineto{\pgfpoint{36.159208\du}{10.179795\du}}
\pgfpathlineto{\pgfpoint{36.159208\du}{10.173600\du}}
\pgfusepath{fill}
\pgfsetlinewidth{0.000000\du}
\pgfsetbuttcap
\pgfsetmiterjoin
\pgfsetdash{}{0pt}
\definecolor{dialinecolor}{rgb}{1.000000, 1.000000, 1.000000}
\pgfsetstrokecolor{dialinecolor}
\pgfpathmoveto{\pgfpoint{35.015456\du}{10.164068\du}}
\pgfpathlineto{\pgfpoint{36.013856\du}{10.164068\du}}
\pgfusepath{stroke}
\pgfsetlinewidth{0.000000\du}
\pgfsetbuttcap
\pgfsetmiterjoin
\pgfsetdash{}{0pt}
\definecolor{dialinecolor}{rgb}{1.000000, 1.000000, 1.000000}
\pgfsetfillcolor{dialinecolor}
\pgfpathmoveto{\pgfpoint{35.118393\du}{10.595835\du}}
\pgfpathlineto{\pgfpoint{35.118393\du}{10.590116\du}}
\pgfpathlineto{\pgfpoint{35.117917\du}{10.583921\du}}
\pgfpathlineto{\pgfpoint{35.117440\du}{10.579155\du}}
\pgfpathlineto{\pgfpoint{35.116011\du}{10.572960\du}}
\pgfpathlineto{\pgfpoint{35.115057\du}{10.566765\du}}
\pgfpathlineto{\pgfpoint{35.113151\du}{10.561522\du}}
\pgfpathlineto{\pgfpoint{35.111722\du}{10.556280\du}}
\pgfpathlineto{\pgfpoint{35.109339\du}{10.550085\du}}
\pgfpathlineto{\pgfpoint{35.106479\du}{10.545319\du}}
\pgfpathlineto{\pgfpoint{35.104096\du}{10.540077\du}}
\pgfpathlineto{\pgfpoint{35.101237\du}{10.534835\du}}
\pgfpathlineto{\pgfpoint{35.098378\du}{10.530069\du}}
\pgfpathlineto{\pgfpoint{35.094565\du}{10.525304\du}}
\pgfpathlineto{\pgfpoint{35.091229\du}{10.521014\du}}
\pgfpathlineto{\pgfpoint{35.086940\du}{10.516249\du}}
\pgfpathlineto{\pgfpoint{35.083128\du}{10.511960\du}}
\pgfpathlineto{\pgfpoint{35.078839\du}{10.508147\du}}
\pgfpathlineto{\pgfpoint{35.074073\du}{10.504811\du}}
\pgfpathlineto{\pgfpoint{35.069784\du}{10.500522\du}}
\pgfpathlineto{\pgfpoint{35.064542\du}{10.497663\du}}
\pgfpathlineto{\pgfpoint{35.059776\du}{10.494327\du}}
\pgfpathlineto{\pgfpoint{35.055010\du}{10.491944\du}}
\pgfpathlineto{\pgfpoint{35.049292\du}{10.489561\du}}
\pgfpathlineto{\pgfpoint{35.044049\du}{10.487178\du}}
\pgfpathlineto{\pgfpoint{35.038331\du}{10.485749\du}}
\pgfpathlineto{\pgfpoint{35.033089\du}{10.483843\du}}
\pgfpathlineto{\pgfpoint{35.027370\du}{10.482413\du}}
\pgfpathlineto{\pgfpoint{35.022128\du}{10.481460\du}}
\pgfpathlineto{\pgfpoint{35.016409\du}{10.480983\du}}
\pgfpathlineto{\pgfpoint{35.010690\du}{10.480030\du}}
\pgfpathlineto{\pgfpoint{35.004495\du}{10.480030\du}}
\pgfpathlineto{\pgfpoint{35.004495\du}{10.480030\du}}
\pgfpathlineto{\pgfpoint{34.998776\du}{10.480030\du}}
\pgfpathlineto{\pgfpoint{34.993057\du}{10.480983\du}}
\pgfpathlineto{\pgfpoint{34.987338\du}{10.481460\du}}
\pgfpathlineto{\pgfpoint{34.982096\du}{10.482413\du}}
\pgfpathlineto{\pgfpoint{34.975901\du}{10.483843\du}}
\pgfpathlineto{\pgfpoint{34.970659\du}{10.485749\du}}
\pgfpathlineto{\pgfpoint{34.965417\du}{10.487178\du}}
\pgfpathlineto{\pgfpoint{34.959698\du}{10.489561\du}}
\pgfpathlineto{\pgfpoint{34.954456\du}{10.491944\du}}
\pgfpathlineto{\pgfpoint{34.949213\du}{10.494327\du}}
\pgfpathlineto{\pgfpoint{34.944924\du}{10.497663\du}}
\pgfpathlineto{\pgfpoint{34.939682\du}{10.500522\du}}
\pgfpathlineto{\pgfpoint{34.934916\du}{10.504811\du}}
\pgfpathlineto{\pgfpoint{34.930627\du}{10.508147\du}}
\pgfpathlineto{\pgfpoint{34.925862\du}{10.511960\du}}
\pgfpathlineto{\pgfpoint{34.922049\du}{10.516249\du}}
\pgfpathlineto{\pgfpoint{34.918237\du}{10.521014\du}}
\pgfpathlineto{\pgfpoint{34.914901\du}{10.525304\du}}
\pgfpathlineto{\pgfpoint{34.911088\du}{10.530069\du}}
\pgfpathlineto{\pgfpoint{34.907752\du}{10.534835\du}}
\pgfpathlineto{\pgfpoint{34.904893\du}{10.540077\du}}
\pgfpathlineto{\pgfpoint{34.902987\du}{10.545319\du}}
\pgfpathlineto{\pgfpoint{34.900127\du}{10.550085\du}}
\pgfpathlineto{\pgfpoint{34.897744\du}{10.556280\du}}
\pgfpathlineto{\pgfpoint{34.895838\du}{10.561522\du}}
\pgfpathlineto{\pgfpoint{34.894409\du}{10.566765\du}}
\pgfpathlineto{\pgfpoint{34.892979\du}{10.572960\du}}
\pgfpathlineto{\pgfpoint{34.892026\du}{10.579155\du}}
\pgfpathlineto{\pgfpoint{34.891073\du}{10.583921\du}}
\pgfpathlineto{\pgfpoint{34.891073\du}{10.590116\du}}
\pgfpathlineto{\pgfpoint{34.891073\du}{10.595835\du}}
\pgfpathlineto{\pgfpoint{34.891073\du}{10.595835\du}}
\pgfpathlineto{\pgfpoint{34.891073\du}{10.602507\du}}
\pgfpathlineto{\pgfpoint{34.891073\du}{10.607749\du}}
\pgfpathlineto{\pgfpoint{34.892026\du}{10.613468\du}}
\pgfpathlineto{\pgfpoint{34.892979\du}{10.619663\du}}
\pgfpathlineto{\pgfpoint{34.894409\du}{10.625858\du}}
\pgfpathlineto{\pgfpoint{34.895838\du}{10.631101\du}}
\pgfpathlineto{\pgfpoint{34.897744\du}{10.636343\du}}
\pgfpathlineto{\pgfpoint{34.900127\du}{10.642538\du}}
\pgfpathlineto{\pgfpoint{34.902987\du}{10.647304\du}}
\pgfpathlineto{\pgfpoint{34.904893\du}{10.652546\du}}
\pgfpathlineto{\pgfpoint{34.907752\du}{10.657788\du}}
\pgfpathlineto{\pgfpoint{34.911088\du}{10.662554\du}}
\pgfpathlineto{\pgfpoint{34.914901\du}{10.667319\du}}
\pgfpathlineto{\pgfpoint{34.918237\du}{10.671609\du}}
\pgfpathlineto{\pgfpoint{34.922049\du}{10.676374\du}}
\pgfpathlineto{\pgfpoint{34.925862\du}{10.680663\du}}
\pgfpathlineto{\pgfpoint{34.930627\du}{10.684476\du}}
\pgfpathlineto{\pgfpoint{34.934916\du}{10.687812\du}}
\pgfpathlineto{\pgfpoint{34.939682\du}{10.691148\du}}
\pgfpathlineto{\pgfpoint{34.944924\du}{10.694960\du}}
\pgfpathlineto{\pgfpoint{34.949213\du}{10.697820\du}}
\pgfpathlineto{\pgfpoint{34.954456\du}{10.700202\du}}
\pgfpathlineto{\pgfpoint{34.959698\du}{10.703062\du}}
\pgfpathlineto{\pgfpoint{34.965417\du}{10.705445\du}}
\pgfpathlineto{\pgfpoint{34.970659\du}{10.706874\du}}
\pgfpathlineto{\pgfpoint{34.975901\du}{10.708781\du}}
\pgfpathlineto{\pgfpoint{34.982096\du}{10.710210\du}}
\pgfpathlineto{\pgfpoint{34.987338\du}{10.711163\du}}
\pgfpathlineto{\pgfpoint{34.993057\du}{10.711640\du}}
\pgfpathlineto{\pgfpoint{34.998776\du}{10.712116\du}}
\pgfpathlineto{\pgfpoint{35.004495\du}{10.712116\du}}
\pgfpathlineto{\pgfpoint{35.004495\du}{10.712116\du}}
\pgfpathlineto{\pgfpoint{35.010690\du}{10.712116\du}}
\pgfpathlineto{\pgfpoint{35.016409\du}{10.711640\du}}
\pgfpathlineto{\pgfpoint{35.022128\du}{10.711163\du}}
\pgfpathlineto{\pgfpoint{35.027370\du}{10.710210\du}}
\pgfpathlineto{\pgfpoint{35.033089\du}{10.708781\du}}
\pgfpathlineto{\pgfpoint{35.038331\du}{10.706874\du}}
\pgfpathlineto{\pgfpoint{35.044049\du}{10.705445\du}}
\pgfpathlineto{\pgfpoint{35.049292\du}{10.703062\du}}
\pgfpathlineto{\pgfpoint{35.055010\du}{10.700202\du}}
\pgfpathlineto{\pgfpoint{35.059776\du}{10.697820\du}}
\pgfpathlineto{\pgfpoint{35.064542\du}{10.694960\du}}
\pgfpathlineto{\pgfpoint{35.069784\du}{10.691148\du}}
\pgfpathlineto{\pgfpoint{35.074073\du}{10.687812\du}}
\pgfpathlineto{\pgfpoint{35.078839\du}{10.684476\du}}
\pgfpathlineto{\pgfpoint{35.083128\du}{10.680663\du}}
\pgfpathlineto{\pgfpoint{35.086940\du}{10.676374\du}}
\pgfpathlineto{\pgfpoint{35.091229\du}{10.671609\du}}
\pgfpathlineto{\pgfpoint{35.094565\du}{10.667319\du}}
\pgfpathlineto{\pgfpoint{35.098378\du}{10.662554\du}}
\pgfpathlineto{\pgfpoint{35.101237\du}{10.657788\du}}
\pgfpathlineto{\pgfpoint{35.104096\du}{10.652546\du}}
\pgfpathlineto{\pgfpoint{35.106479\du}{10.647304\du}}
\pgfpathlineto{\pgfpoint{35.109339\du}{10.642538\du}}
\pgfpathlineto{\pgfpoint{35.111722\du}{10.636343\du}}
\pgfpathlineto{\pgfpoint{35.113151\du}{10.631101\du}}
\pgfpathlineto{\pgfpoint{35.115057\du}{10.625858\du}}
\pgfpathlineto{\pgfpoint{35.116011\du}{10.619663\du}}
\pgfpathlineto{\pgfpoint{35.117440\du}{10.613468\du}}
\pgfpathlineto{\pgfpoint{35.117917\du}{10.607749\du}}
\pgfpathlineto{\pgfpoint{35.118393\du}{10.602507\du}}
\pgfpathlineto{\pgfpoint{35.118393\du}{10.595835\du}}
\pgfusepath{fill}
\pgfsetbuttcap
\pgfsetmiterjoin
\pgfsetdash{}{0pt}
\definecolor{dialinecolor}{rgb}{1.000000, 1.000000, 1.000000}
\pgfsetfillcolor{dialinecolor}
\pgfpathmoveto{\pgfpoint{36.159208\du}{10.595835\du}}
\pgfpathlineto{\pgfpoint{36.159208\du}{10.590116\du}}
\pgfpathlineto{\pgfpoint{36.158255\du}{10.583921\du}}
\pgfpathlineto{\pgfpoint{36.157778\du}{10.579155\du}}
\pgfpathlineto{\pgfpoint{36.157302\du}{10.572960\du}}
\pgfpathlineto{\pgfpoint{36.155396\du}{10.566765\du}}
\pgfpathlineto{\pgfpoint{36.153966\du}{10.561522\du}}
\pgfpathlineto{\pgfpoint{36.152060\du}{10.556280\du}}
\pgfpathlineto{\pgfpoint{36.149677\du}{10.550085\du}}
\pgfpathlineto{\pgfpoint{36.147294\du}{10.545319\du}}
\pgfpathlineto{\pgfpoint{36.145388\du}{10.540077\du}}
\pgfpathlineto{\pgfpoint{36.142052\du}{10.534835\du}}
\pgfpathlineto{\pgfpoint{36.139192\du}{10.530069\du}}
\pgfpathlineto{\pgfpoint{36.135380\du}{10.525304\du}}
\pgfpathlineto{\pgfpoint{36.131567\du}{10.521014\du}}
\pgfpathlineto{\pgfpoint{36.127278\du}{10.516249\du}}
\pgfpathlineto{\pgfpoint{36.123466\du}{10.511960\du}}
\pgfpathlineto{\pgfpoint{36.119177\du}{10.508147\du}}
\pgfpathlineto{\pgfpoint{36.115364\du}{10.504811\du}}
\pgfpathlineto{\pgfpoint{36.110122\du}{10.500522\du}}
\pgfpathlineto{\pgfpoint{36.105356\du}{10.497663\du}}
\pgfpathlineto{\pgfpoint{36.100591\du}{10.494327\du}}
\pgfpathlineto{\pgfpoint{36.095349\du}{10.491944\du}}
\pgfpathlineto{\pgfpoint{36.090106\du}{10.489561\du}}
\pgfpathlineto{\pgfpoint{36.084388\du}{10.487178\du}}
\pgfpathlineto{\pgfpoint{36.078669\du}{10.485749\du}}
\pgfpathlineto{\pgfpoint{36.073427\du}{10.483843\du}}
\pgfpathlineto{\pgfpoint{36.067708\du}{10.482413\du}}
\pgfpathlineto{\pgfpoint{36.061989\du}{10.481460\du}}
\pgfpathlineto{\pgfpoint{36.055794\du}{10.480983\du}}
\pgfpathlineto{\pgfpoint{36.050552\du}{10.480030\du}}
\pgfpathlineto{\pgfpoint{36.044356\du}{10.480030\du}}
\pgfpathlineto{\pgfpoint{36.044356\du}{10.480030\du}}
\pgfpathlineto{\pgfpoint{36.038638\du}{10.480030\du}}
\pgfpathlineto{\pgfpoint{36.033395\du}{10.480983\du}}
\pgfpathlineto{\pgfpoint{36.027200\du}{10.481460\du}}
\pgfpathlineto{\pgfpoint{36.021481\du}{10.482413\du}}
\pgfpathlineto{\pgfpoint{36.015763\du}{10.483843\du}}
\pgfpathlineto{\pgfpoint{36.010997\du}{10.485749\du}}
\pgfpathlineto{\pgfpoint{36.005278\du}{10.487178\du}}
\pgfpathlineto{\pgfpoint{35.999083\du}{10.489561\du}}
\pgfpathlineto{\pgfpoint{35.993841\du}{10.491944\du}}
\pgfpathlineto{\pgfpoint{35.989075\du}{10.494327\du}}
\pgfpathlineto{\pgfpoint{35.984309\du}{10.497663\du}}
\pgfpathlineto{\pgfpoint{35.979067\du}{10.500522\du}}
\pgfpathlineto{\pgfpoint{35.974301\du}{10.504811\du}}
\pgfpathlineto{\pgfpoint{35.970012\du}{10.508147\du}}
\pgfpathlineto{\pgfpoint{35.966200\du}{10.511960\du}}
\pgfpathlineto{\pgfpoint{35.961434\du}{10.516249\du}}
\pgfpathlineto{\pgfpoint{35.957622\du}{10.521014\du}}
\pgfpathlineto{\pgfpoint{35.954286\du}{10.525304\du}}
\pgfpathlineto{\pgfpoint{35.950473\du}{10.530069\du}}
\pgfpathlineto{\pgfpoint{35.947137\du}{10.534835\du}}
\pgfpathlineto{\pgfpoint{35.944278\du}{10.540077\du}}
\pgfpathlineto{\pgfpoint{35.942372\du}{10.545319\du}}
\pgfpathlineto{\pgfpoint{35.939512\du}{10.550085\du}}
\pgfpathlineto{\pgfpoint{35.937130\du}{10.556280\du}}
\pgfpathlineto{\pgfpoint{35.935223\du}{10.561522\du}}
\pgfpathlineto{\pgfpoint{35.933794\du}{10.566765\du}}
\pgfpathlineto{\pgfpoint{35.932364\du}{10.572960\du}}
\pgfpathlineto{\pgfpoint{35.931411\du}{10.579155\du}}
\pgfpathlineto{\pgfpoint{35.930934\du}{10.583921\du}}
\pgfpathlineto{\pgfpoint{35.930458\du}{10.590116\du}}
\pgfpathlineto{\pgfpoint{35.930458\du}{10.595835\du}}
\pgfpathlineto{\pgfpoint{35.930458\du}{10.595835\du}}
\pgfpathlineto{\pgfpoint{35.930458\du}{10.602507\du}}
\pgfpathlineto{\pgfpoint{35.930934\du}{10.607749\du}}
\pgfpathlineto{\pgfpoint{35.931411\du}{10.613468\du}}
\pgfpathlineto{\pgfpoint{35.932364\du}{10.619663\du}}
\pgfpathlineto{\pgfpoint{35.933794\du}{10.625858\du}}
\pgfpathlineto{\pgfpoint{35.935223\du}{10.631101\du}}
\pgfpathlineto{\pgfpoint{35.937130\du}{10.636343\du}}
\pgfpathlineto{\pgfpoint{35.939512\du}{10.642538\du}}
\pgfpathlineto{\pgfpoint{35.942372\du}{10.647304\du}}
\pgfpathlineto{\pgfpoint{35.944278\du}{10.652546\du}}
\pgfpathlineto{\pgfpoint{35.947137\du}{10.657788\du}}
\pgfpathlineto{\pgfpoint{35.950473\du}{10.662554\du}}
\pgfpathlineto{\pgfpoint{35.954286\du}{10.667319\du}}
\pgfpathlineto{\pgfpoint{35.957622\du}{10.671609\du}}
\pgfpathlineto{\pgfpoint{35.961434\du}{10.676374\du}}
\pgfpathlineto{\pgfpoint{35.966200\du}{10.680663\du}}
\pgfpathlineto{\pgfpoint{35.970012\du}{10.684476\du}}
\pgfpathlineto{\pgfpoint{35.974301\du}{10.687812\du}}
\pgfpathlineto{\pgfpoint{35.979067\du}{10.691148\du}}
\pgfpathlineto{\pgfpoint{35.984309\du}{10.694960\du}}
\pgfpathlineto{\pgfpoint{35.989075\du}{10.697820\du}}
\pgfpathlineto{\pgfpoint{35.993841\du}{10.700202\du}}
\pgfpathlineto{\pgfpoint{35.999083\du}{10.703062\du}}
\pgfpathlineto{\pgfpoint{36.005278\du}{10.705445\du}}
\pgfpathlineto{\pgfpoint{36.010997\du}{10.706874\du}}
\pgfpathlineto{\pgfpoint{36.015763\du}{10.708781\du}}
\pgfpathlineto{\pgfpoint{36.021481\du}{10.710210\du}}
\pgfpathlineto{\pgfpoint{36.027200\du}{10.711163\du}}
\pgfpathlineto{\pgfpoint{36.033395\du}{10.711640\du}}
\pgfpathlineto{\pgfpoint{36.038638\du}{10.712116\du}}
\pgfpathlineto{\pgfpoint{36.044356\du}{10.712116\du}}
\pgfpathlineto{\pgfpoint{36.044356\du}{10.712116\du}}
\pgfpathlineto{\pgfpoint{36.050552\du}{10.712116\du}}
\pgfpathlineto{\pgfpoint{36.055794\du}{10.711640\du}}
\pgfpathlineto{\pgfpoint{36.061989\du}{10.711163\du}}
\pgfpathlineto{\pgfpoint{36.067708\du}{10.710210\du}}
\pgfpathlineto{\pgfpoint{36.073427\du}{10.708781\du}}
\pgfpathlineto{\pgfpoint{36.078669\du}{10.706874\du}}
\pgfpathlineto{\pgfpoint{36.084388\du}{10.705445\du}}
\pgfpathlineto{\pgfpoint{36.090106\du}{10.703062\du}}
\pgfpathlineto{\pgfpoint{36.095349\du}{10.700202\du}}
\pgfpathlineto{\pgfpoint{36.100591\du}{10.697820\du}}
\pgfpathlineto{\pgfpoint{36.105356\du}{10.694960\du}}
\pgfpathlineto{\pgfpoint{36.110122\du}{10.691148\du}}
\pgfpathlineto{\pgfpoint{36.115364\du}{10.687812\du}}
\pgfpathlineto{\pgfpoint{36.119177\du}{10.684476\du}}
\pgfpathlineto{\pgfpoint{36.123466\du}{10.680663\du}}
\pgfpathlineto{\pgfpoint{36.127278\du}{10.676374\du}}
\pgfpathlineto{\pgfpoint{36.131567\du}{10.671609\du}}
\pgfpathlineto{\pgfpoint{36.135380\du}{10.667319\du}}
\pgfpathlineto{\pgfpoint{36.139192\du}{10.662554\du}}
\pgfpathlineto{\pgfpoint{36.142052\du}{10.657788\du}}
\pgfpathlineto{\pgfpoint{36.145388\du}{10.652546\du}}
\pgfpathlineto{\pgfpoint{36.147294\du}{10.647304\du}}
\pgfpathlineto{\pgfpoint{36.149677\du}{10.642538\du}}
\pgfpathlineto{\pgfpoint{36.152060\du}{10.636343\du}}
\pgfpathlineto{\pgfpoint{36.153966\du}{10.631101\du}}
\pgfpathlineto{\pgfpoint{36.155396\du}{10.625858\du}}
\pgfpathlineto{\pgfpoint{36.157302\du}{10.619663\du}}
\pgfpathlineto{\pgfpoint{36.157778\du}{10.613468\du}}
\pgfpathlineto{\pgfpoint{36.158255\du}{10.607749\du}}
\pgfpathlineto{\pgfpoint{36.159208\du}{10.602507\du}}
\pgfpathlineto{\pgfpoint{36.159208\du}{10.595835\du}}
\pgfusepath{fill}
\pgfsetlinewidth{0.000000\du}
\pgfsetbuttcap
\pgfsetmiterjoin
\pgfsetdash{}{0pt}
\definecolor{dialinecolor}{rgb}{1.000000, 1.000000, 1.000000}
\pgfsetstrokecolor{dialinecolor}
\pgfpathmoveto{\pgfpoint{35.015456\du}{10.586304\du}}
\pgfpathlineto{\pgfpoint{36.013856\du}{10.586304\du}}
\pgfusepath{stroke}
\pgfsetlinewidth{0.000000\du}
\pgfsetbuttcap
\pgfsetmiterjoin
\pgfsetdash{}{0pt}
\definecolor{dialinecolor}{rgb}{0.788235, 0.788235, 0.713726}
\pgfsetfillcolor{dialinecolor}
\pgfpathmoveto{\pgfpoint{34.599416\du}{7.674024\du}}
\pgfpathlineto{\pgfpoint{34.891073\du}{7.400000\du}}
\pgfpathlineto{\pgfpoint{36.700584\du}{7.400000\du}}
\pgfpathlineto{\pgfpoint{36.410357\du}{7.674024\du}}
\pgfpathlineto{\pgfpoint{34.599416\du}{7.674024\du}}
\pgfusepath{fill}
\pgfsetbuttcap
\pgfsetmiterjoin
\pgfsetdash{}{0pt}
\definecolor{dialinecolor}{rgb}{0.286275, 0.286275, 0.211765}
\pgfsetstrokecolor{dialinecolor}
\pgfpathmoveto{\pgfpoint{34.599416\du}{7.674024\du}}
\pgfpathlineto{\pgfpoint{34.891073\du}{7.400000\du}}
\pgfpathlineto{\pgfpoint{36.700584\du}{7.400000\du}}
\pgfpathlineto{\pgfpoint{36.410357\du}{7.674024\du}}
\pgfpathlineto{\pgfpoint{34.599416\du}{7.674024\du}}
\pgfusepath{stroke}
\pgfsetbuttcap
\pgfsetmiterjoin
\pgfsetdash{}{0pt}
\definecolor{dialinecolor}{rgb}{0.478431, 0.478431, 0.352941}
\pgfsetfillcolor{dialinecolor}
\pgfpathmoveto{\pgfpoint{36.410357\du}{11.050000\du}}
\pgfpathlineto{\pgfpoint{36.700584\du}{10.755007\du}}
\pgfpathlineto{\pgfpoint{36.700584\du}{7.400000\du}}
\pgfpathlineto{\pgfpoint{36.410357\du}{7.674024\du}}
\pgfpathlineto{\pgfpoint{36.410357\du}{11.050000\du}}
\pgfusepath{fill}
\pgfsetbuttcap
\pgfsetmiterjoin
\pgfsetdash{}{0pt}
\definecolor{dialinecolor}{rgb}{0.286275, 0.286275, 0.211765}
\pgfsetstrokecolor{dialinecolor}
\pgfpathmoveto{\pgfpoint{36.410357\du}{11.050000\du}}
\pgfpathlineto{\pgfpoint{36.700584\du}{10.755007\du}}
\pgfpathlineto{\pgfpoint{36.700584\du}{7.400000\du}}
\pgfpathlineto{\pgfpoint{36.410357\du}{7.674024\du}}
\pgfpathlineto{\pgfpoint{36.410357\du}{11.050000\du}}
\pgfusepath{stroke}
% setfont left to latex
\definecolor{dialinecolor}{rgb}{0.000000, 0.000000, 0.000000}
\pgfsetstrokecolor{dialinecolor}
\node[anchor=west] at (16.000000\du,5.650000\du){X};
% setfont left to latex
\definecolor{dialinecolor}{rgb}{0.000000, 0.000000, 0.000000}
\pgfsetstrokecolor{dialinecolor}
\node[anchor=west] at (25.335000\du,5.712500\du){Y};
% setfont left to latex
\definecolor{dialinecolor}{rgb}{0.000000, 0.000000, 0.000000}
\pgfsetstrokecolor{dialinecolor}
\node[anchor=west] at (34.970000\du,5.822500\du){D};
% setfont left to latex
\definecolor{dialinecolor}{rgb}{0.000000, 0.000000, 0.000000}
\pgfsetstrokecolor{dialinecolor}
\node[anchor=west] at (7.205000\du,5.532500\du){P};
\end{tikzpicture}

}
	\caption{}
	\label{fig:traceroute}
\end{figure}

\subsection{Paramètres de l'algorithme de la détection}

\begin{itemize}
		\item  Objectif : suivre l'évolution du délais d'un lien au cours du temps en suivant son RTT différentiel par période du temps (\textit{timeWindow}).
	\item Entrées : l'ensemble des traceroutes stockés dans un fichier, date de début de l'analyse \textit{start}, date de la fin de l'analyse \textit{end}, lien à analyser (\textit{link}) et la fenêtre de l'analyse (\textit{timeWindow}).
	\item Sorties : les dates pendant lesquelles des anomalies ont été détectées.
\end{itemize}

Soient $ d_1 $,$  d_2 $, ..., $ d_N $ les périodes entre \textit{start} et \textit{end} où

\begin{center}
	 $  d_{i+1} $ - $  d_{i} $ = $  d_{j+1} $ - $  d_{j} $ = \textit{step} \footnote{\textit{step} est la durée d'une période en secondes, $3600$ pour fenêtre d'une heure.}
\end{center}
 
 pour tout $ i $ et $ j $ dans$  [1,N] $
\subsection{Le processus de la détection des anomalies dans les délais d'un lien}\label{steps-rtt-analysis}

On distingue trois grandes étapes dans le processus de la détection des anomalies, dont les détails sont donnés dans la figure \ref{fig:process-rttanalysis_tex}. La première étape a pour objectif de trier les traceroutes à analyser par période (étape 1). La deuxième étape désigne les opérations à appliquer sur les traceroutes d'une période (étapes entre 2 et 6). A la fin de la préparation des traceroutes de toutes les périodes, la troisième étape est dédiée à la comparaison des délais avec les valeurs de la référence (étape 7).
\paragraph{1. Trier les traceroutes } à analyser par \textit{timeWindow}. En effet, chaque $d_i$ est associé à un ensemble de traceroutes ayant été effectués entre $d_i$ et $d_i + step$ \footnote{Chaque traceroute reprend le temps pendant lequel il était effectué.}. 

\begin{figure}[H]
	\centering
	\captionsetup{justification=centering}
	% Graphic for TeX using PGF
% Title: /home/hayat/RipeAtlasTraceroutesAnalysis/report/illustrations/timing.dia
% Creator: Dia v0.97+git
% CreationDate: Thu Nov 29 15:39:51 2018
% For: hayat
% \usepackage{tikz}
% The following commands are not supported in PSTricks at present
% We define them conditionally, so when they are implemented,
% this pgf file will use them.
\ifx\du\undefined
  \newlength{\du}
\fi
\setlength{\du}{15\unitlength}
\begin{tikzpicture}[even odd rule]
\pgftransformxscale{1.000000}
\pgftransformyscale{-1.000000}
\definecolor{dialinecolor}{rgb}{0.000000, 0.000000, 0.000000}
\pgfsetstrokecolor{dialinecolor}
\pgfsetstrokeopacity{1.000000}
\definecolor{diafillcolor}{rgb}{1.000000, 1.000000, 1.000000}
\pgfsetfillcolor{diafillcolor}
\pgfsetfillopacity{1.000000}
\pgfsetlinewidth{0.100000\du}
\pgfsetdash{}{0pt}
\pgfsetbuttcap
{
\definecolor{diafillcolor}{rgb}{0.000000, 0.000000, 0.000000}
\pgfsetfillcolor{diafillcolor}
\pgfsetfillopacity{1.000000}
% was here!!!
}
\definecolor{dialinecolor}{rgb}{0.000000, 0.000000, 0.000000}
\pgfsetstrokecolor{dialinecolor}
\pgfsetstrokeopacity{1.000000}
\draw (34.600000\du,6.050000\du)--(40.550000\du,6.000000\du);
\pgfsetlinewidth{0.100000\du}
\pgfsetdash{}{0pt}
\pgfsetmiterjoin
\pgfsetbuttcap
\definecolor{dialinecolor}{rgb}{0.000000, 0.000000, 0.000000}
\pgfsetstrokecolor{dialinecolor}
\pgfsetstrokeopacity{1.000000}
\draw (35.097882\du,5.795807\du)--(35.102083\du,6.295790\du);
\pgfsetlinewidth{0.100000\du}
\pgfsetdash{}{0pt}
\pgfsetmiterjoin
\pgfsetbuttcap
\definecolor{dialinecolor}{rgb}{0.000000, 0.000000, 0.000000}
\pgfsetstrokecolor{dialinecolor}
\pgfsetstrokeopacity{1.000000}
\draw (40.052118\du,6.254193\du)--(40.047917\du,5.754210\du);
% setfont left to latex
\definecolor{dialinecolor}{rgb}{0.000000, 0.000000, 0.000000}
\pgfsetstrokecolor{dialinecolor}
\pgfsetstrokeopacity{1.000000}
\definecolor{diafillcolor}{rgb}{0.000000, 0.000000, 0.000000}
\pgfsetfillcolor{diafillcolor}
\pgfsetfillopacity{1.000000}
\node[anchor=base west,inner sep=0pt,outer sep=0pt,color=dialinecolor] at (14.800166\du,8.464650\du){start};
% setfont left to latex
\definecolor{dialinecolor}{rgb}{0.000000, 0.000000, 0.000000}
\pgfsetstrokecolor{dialinecolor}
\pgfsetstrokeopacity{1.000000}
\definecolor{diafillcolor}{rgb}{0.000000, 0.000000, 0.000000}
\pgfsetfillcolor{diafillcolor}
\pgfsetfillopacity{1.000000}
\node[anchor=base west,inner sep=0pt,outer sep=0pt,color=dialinecolor] at (39.375580\du,7.990892\du){end};
% setfont left to latex
\definecolor{dialinecolor}{rgb}{0.000000, 0.000000, 0.000000}
\pgfsetstrokecolor{dialinecolor}
\pgfsetstrokeopacity{1.000000}
\definecolor{diafillcolor}{rgb}{0.000000, 0.000000, 0.000000}
\pgfsetfillcolor{diafillcolor}
\pgfsetfillopacity{1.000000}
\node[anchor=base west,inner sep=0pt,outer sep=0pt,color=dialinecolor] at (20.674752\du,8.627312\du){Périodes de l'analyse (Unix time)};
% setfont left to latex
\definecolor{dialinecolor}{rgb}{0.000000, 0.000000, 0.000000}
\pgfsetstrokecolor{dialinecolor}
\pgfsetstrokeopacity{1.000000}
\definecolor{diafillcolor}{rgb}{0.000000, 0.000000, 0.000000}
\pgfsetfillcolor{diafillcolor}
\pgfsetfillopacity{1.000000}
\node[anchor=base west,inner sep=0pt,outer sep=0pt,color=dialinecolor] at (14.950000\du,7.200000\du){d1};
\pgfsetlinewidth{0.100000\du}
\pgfsetdash{}{0pt}
\pgfsetbuttcap
{
\definecolor{diafillcolor}{rgb}{0.000000, 0.000000, 0.000000}
\pgfsetfillcolor{diafillcolor}
\pgfsetfillopacity{1.000000}
% was here!!!
}
\definecolor{dialinecolor}{rgb}{0.000000, 0.000000, 0.000000}
\pgfsetstrokecolor{dialinecolor}
\pgfsetstrokeopacity{1.000000}
\draw (29.656799\du,6.066268\du)--(35.606799\du,6.016268\du);
\pgfsetlinewidth{0.100000\du}
\pgfsetdash{}{0pt}
\pgfsetmiterjoin
\pgfsetbuttcap
\definecolor{dialinecolor}{rgb}{0.000000, 0.000000, 0.000000}
\pgfsetstrokecolor{dialinecolor}
\pgfsetstrokeopacity{1.000000}
\draw (30.154681\du,5.812076\du)--(30.158883\du,6.312058\du);
\pgfsetlinewidth{0.100000\du}
\pgfsetdash{}{0pt}
\pgfsetmiterjoin
\pgfsetbuttcap
\definecolor{dialinecolor}{rgb}{0.000000, 0.000000, 0.000000}
\pgfsetstrokecolor{dialinecolor}
\pgfsetstrokeopacity{1.000000}
\draw (35.108918\du,6.270461\du)--(35.104716\du,5.770479\du);
\pgfsetlinewidth{0.100000\du}
\pgfsetdash{}{0pt}
\pgfsetbuttcap
{
\definecolor{diafillcolor}{rgb}{0.000000, 0.000000, 0.000000}
\pgfsetfillcolor{diafillcolor}
\pgfsetfillopacity{1.000000}
% was here!!!
}
\definecolor{dialinecolor}{rgb}{0.000000, 0.000000, 0.000000}
\pgfsetstrokecolor{dialinecolor}
\pgfsetstrokeopacity{1.000000}
\draw (24.701799\du,6.106103\du)--(30.651799\du,6.056103\du);
\pgfsetlinewidth{0.100000\du}
\pgfsetdash{}{0pt}
\pgfsetmiterjoin
\pgfsetbuttcap
\definecolor{dialinecolor}{rgb}{0.000000, 0.000000, 0.000000}
\pgfsetstrokecolor{dialinecolor}
\pgfsetstrokeopacity{1.000000}
\draw (25.199681\du,5.851910\du)--(25.203883\du,6.351892\du);
\pgfsetlinewidth{0.100000\du}
\pgfsetdash{}{0pt}
\pgfsetmiterjoin
\pgfsetbuttcap
\definecolor{dialinecolor}{rgb}{0.000000, 0.000000, 0.000000}
\pgfsetstrokecolor{dialinecolor}
\pgfsetstrokeopacity{1.000000}
\draw (30.153918\du,6.310295\du)--(30.149716\du,5.810313\du);
\pgfsetlinewidth{0.100000\du}
\pgfsetdash{}{0pt}
\pgfsetbuttcap
{
\definecolor{diafillcolor}{rgb}{0.000000, 0.000000, 0.000000}
\pgfsetfillcolor{diafillcolor}
\pgfsetfillopacity{1.000000}
% was here!!!
}
\definecolor{dialinecolor}{rgb}{0.000000, 0.000000, 0.000000}
\pgfsetstrokecolor{dialinecolor}
\pgfsetstrokeopacity{1.000000}
\draw (19.796634\du,6.121020\du)--(25.746634\du,6.071020\du);
\pgfsetlinewidth{0.100000\du}
\pgfsetdash{}{0pt}
\pgfsetmiterjoin
\pgfsetbuttcap
\definecolor{dialinecolor}{rgb}{0.000000, 0.000000, 0.000000}
\pgfsetstrokecolor{dialinecolor}
\pgfsetstrokeopacity{1.000000}
\draw (20.294515\du,5.866827\du)--(20.298717\du,6.366810\du);
\pgfsetlinewidth{0.100000\du}
\pgfsetdash{}{0pt}
\pgfsetmiterjoin
\pgfsetbuttcap
\definecolor{dialinecolor}{rgb}{0.000000, 0.000000, 0.000000}
\pgfsetstrokecolor{dialinecolor}
\pgfsetstrokeopacity{1.000000}
\draw (25.248752\du,6.325213\du)--(25.244551\du,5.825230\du);
\pgfsetlinewidth{0.100000\du}
\pgfsetdash{}{0pt}
\pgfsetbuttcap
{
\definecolor{diafillcolor}{rgb}{0.000000, 0.000000, 0.000000}
\pgfsetfillcolor{diafillcolor}
\pgfsetfillopacity{1.000000}
% was here!!!
}
\definecolor{dialinecolor}{rgb}{0.000000, 0.000000, 0.000000}
\pgfsetstrokecolor{dialinecolor}
\pgfsetstrokeopacity{1.000000}
\draw (14.841634\du,6.135937\du)--(20.791634\du,6.085937\du);
\pgfsetlinewidth{0.100000\du}
\pgfsetdash{}{0pt}
\pgfsetmiterjoin
\pgfsetbuttcap
\definecolor{dialinecolor}{rgb}{0.000000, 0.000000, 0.000000}
\pgfsetstrokecolor{dialinecolor}
\pgfsetstrokeopacity{1.000000}
\draw (15.339515\du,5.881744\du)--(15.343717\du,6.381727\du);
\pgfsetlinewidth{0.100000\du}
\pgfsetdash{}{0pt}
\pgfsetmiterjoin
\pgfsetbuttcap
\definecolor{dialinecolor}{rgb}{0.000000, 0.000000, 0.000000}
\pgfsetstrokecolor{dialinecolor}
\pgfsetstrokeopacity{1.000000}
\draw (20.293752\du,6.340130\du)--(20.289551\du,5.840147\du);
% setfont left to latex
\definecolor{dialinecolor}{rgb}{0.000000, 0.000000, 0.000000}
\pgfsetstrokecolor{dialinecolor}
\pgfsetstrokeopacity{1.000000}
\definecolor{diafillcolor}{rgb}{0.000000, 0.000000, 0.000000}
\pgfsetfillcolor{diafillcolor}
\pgfsetfillopacity{1.000000}
\node[anchor=base west,inner sep=0pt,outer sep=0pt,color=dialinecolor] at (19.845000\du,7.235000\du){d2};
% setfont left to latex
\definecolor{dialinecolor}{rgb}{0.000000, 0.000000, 0.000000}
\pgfsetstrokecolor{dialinecolor}
\pgfsetstrokeopacity{1.000000}
\definecolor{diafillcolor}{rgb}{0.000000, 0.000000, 0.000000}
\pgfsetfillcolor{diafillcolor}
\pgfsetfillopacity{1.000000}
\node[anchor=base west,inner sep=0pt,outer sep=0pt,color=dialinecolor] at (24.750000\du,7.300000\du){d3};
% setfont left to latex
\definecolor{dialinecolor}{rgb}{0.000000, 0.000000, 0.000000}
\pgfsetstrokecolor{dialinecolor}
\pgfsetstrokeopacity{1.000000}
\definecolor{diafillcolor}{rgb}{0.000000, 0.000000, 0.000000}
\pgfsetfillcolor{diafillcolor}
\pgfsetfillopacity{1.000000}
\node[anchor=base west,inner sep=0pt,outer sep=0pt,color=dialinecolor] at (29.740000\du,7.175000\du){di};
% setfont left to latex
\definecolor{dialinecolor}{rgb}{0.000000, 0.000000, 0.000000}
\pgfsetstrokecolor{dialinecolor}
\pgfsetstrokeopacity{1.000000}
\definecolor{diafillcolor}{rgb}{0.000000, 0.000000, 0.000000}
\pgfsetfillcolor{diafillcolor}
\pgfsetfillopacity{1.000000}
\node[anchor=base west,inner sep=0pt,outer sep=0pt,color=dialinecolor] at (34.685000\du,7.165000\du){di+1};
\pgfsetlinewidth{0.100000\du}
\pgfsetdash{}{0pt}
\pgfsetbuttcap
\definecolor{dialinecolor}{rgb}{0.000000, 0.000000, 0.000000}
\pgfsetstrokecolor{dialinecolor}
\pgfsetstrokeopacity{1.000000}
\pgfpathmoveto{\pgfpoint{25.180033\du}{4.900033\du}}
\pgfpatharc{315}{226}{3.501250\du and 3.501250\du}
\pgfusepath{stroke}
% setfont left to latex
\definecolor{dialinecolor}{rgb}{0.000000, 0.000000, 0.000000}
\pgfsetstrokecolor{dialinecolor}
\pgfsetstrokeopacity{1.000000}
\definecolor{diafillcolor}{rgb}{0.000000, 0.000000, 0.000000}
\pgfsetfillcolor{diafillcolor}
\pgfsetfillopacity{1.000000}
\node[anchor=base west,inner sep=0pt,outer sep=0pt,color=dialinecolor] at (20.780000\du,3.400000\du){timeWindow};
\pgfsetlinewidth{0.100000\du}
\pgfsetdash{}{0pt}
\pgfsetbuttcap
\definecolor{dialinecolor}{rgb}{0.000000, 0.000000, 0.000000}
\pgfsetstrokecolor{dialinecolor}
\pgfsetstrokeopacity{1.000000}
\pgfpathmoveto{\pgfpoint{29.975033\du}{4.885033\du}}
\pgfpatharc{315}{226}{3.501250\du and 3.501250\du}
\pgfusepath{stroke}
% setfont left to latex
\definecolor{dialinecolor}{rgb}{0.000000, 0.000000, 0.000000}
\pgfsetstrokecolor{dialinecolor}
\pgfsetstrokeopacity{1.000000}
\definecolor{diafillcolor}{rgb}{0.000000, 0.000000, 0.000000}
\pgfsetfillcolor{diafillcolor}
\pgfsetfillopacity{1.000000}
\node[anchor=base west,inner sep=0pt,outer sep=0pt,color=dialinecolor] at (25.675000\du,3.485000\du){timeWindow};
\pgfsetlinewidth{0.100000\du}
\pgfsetdash{}{0pt}
\pgfsetbuttcap
\definecolor{dialinecolor}{rgb}{0.000000, 0.000000, 0.000000}
\pgfsetstrokecolor{dialinecolor}
\pgfsetstrokeopacity{1.000000}
\pgfpathmoveto{\pgfpoint{20.275033\du}{4.785033\du}}
\pgfpatharc{315}{226}{3.501250\du and 3.501250\du}
\pgfusepath{stroke}
% setfont left to latex
\definecolor{dialinecolor}{rgb}{0.000000, 0.000000, 0.000000}
\pgfsetstrokecolor{dialinecolor}
\pgfsetstrokeopacity{1.000000}
\definecolor{diafillcolor}{rgb}{0.000000, 0.000000, 0.000000}
\pgfsetfillcolor{diafillcolor}
\pgfsetfillopacity{1.000000}
\node[anchor=base west,inner sep=0pt,outer sep=0pt,color=dialinecolor] at (16.025000\du,3.435000\du){timeWindow};
\end{tikzpicture}

	\caption{}
	\label{fig:timing_tex}
\end{figure}


Les opérations  (2 à 6) concernent  les traceroutes par tout $d_i$.  

\paragraph{2. Vérification de la validité de chaque traceroute }. Ces vérifications reprennent les points suivants:
\begin{itemize}
	\item élimination des traceroutes échoués complètement;
	\item élimination du signal contenant une adresse IP privée;
	\item élimination du signal qui ne contient pas un RTT ou celui qui contient un RTT négatif;
	\item  élimination du signal échoué.
\end{itemize}

\paragraph{3. Calcul de la médiane des RTTs par saut.} Pour tout saut d'un traceroute,  on calcul la médiane des RTTs par adresse IP. Soit le saut $h =\{s \}$ où $s$ est un  Signal, mediane\_rtt ($h$) =  $\{median(\{s.rtt\})\}$  pour tout signal $s$ ayant la même adresse IP. Autrement dit, le nouveau saut du traceroute est reconstruit en regroupant les signaux par adresse IP et ensuite en calculant leurs RTTs. 




\paragraph{4. Inférence des liens topologiques par traceroute.} Un lien topologique est formé par chaque deux routeurs consécutifs. Ce sont les deux routeurs des deux sauts consécutifs. De manière générale, la figure \ref{fig:link-inference} illustre la constitution des liens possibles  dans un traceroute. Soient  RAi, avec $i \in [1,N]$,  l'ensemble des routeurs pour le saut A et RBj, avec $j \in [1,M]$, l'ensemble  des routeurs pour le saut B, avec N et M deux entiers.



Ainsi, les liens  construits sont ceux partant de tout RAi vers tout RBj, où A et B sont deux sauts consécutifs. A l'issue de cette étape, pour tout traceroute, on obtient la liste des liens possibles tout en reprenant des informations générales de la requête traceroute.
\begin{figure}[H]
	\centering
	\captionsetup{justification=centering}
	%\includegraphics[width=0.5\linewidth]{illustrations/link-inference}
	% Graphic for TeX using PGF
% Title: /home/hayat/RipeAtlasTraceroutesAnalysis/report/illustrations/link-inference.dia
% Creator: Dia v0.97+git
% CreationDate: Thu Nov 29 20:58:09 2018
% For: hayat
% \usepackage{tikz}
% The following commands are not supported in PSTricks at present
% We define them conditionally, so when they are implemented,
% this pgf file will use them.
\ifx\du\undefined
  \newlength{\du}
\fi
\setlength{\du}{15\unitlength}
\begin{tikzpicture}[even odd rule]
\pgftransformxscale{1.000000}
\pgftransformyscale{-1.000000}
\definecolor{dialinecolor}{rgb}{0.000000, 0.000000, 0.000000}
\pgfsetstrokecolor{dialinecolor}
\pgfsetstrokeopacity{1.000000}
\definecolor{diafillcolor}{rgb}{1.000000, 1.000000, 1.000000}
\pgfsetfillcolor{diafillcolor}
\pgfsetfillopacity{1.000000}
\pgfsetlinewidth{0.100000\du}
\pgfsetdash{}{0pt}
\pgfsetmiterjoin
\definecolor{diafillcolor}{rgb}{1.000000, 1.000000, 1.000000}
\pgfsetfillcolor{diafillcolor}
\pgfsetfillopacity{1.000000}
\pgfpathellipse{\pgfpoint{27.430614\du}{13.298442\du}}{\pgfpoint{1.349214\du}{0\du}}{\pgfpoint{0\du}{1.217042\du}}
\pgfusepath{fill}
\definecolor{dialinecolor}{rgb}{0.000000, 0.000000, 0.000000}
\pgfsetstrokecolor{dialinecolor}
\pgfsetstrokeopacity{1.000000}
\pgfpathellipse{\pgfpoint{27.430614\du}{13.298442\du}}{\pgfpoint{1.349214\du}{0\du}}{\pgfpoint{0\du}{1.217042\du}}
\pgfusepath{stroke}
% setfont left to latex
\definecolor{dialinecolor}{rgb}{0.000000, 0.000000, 0.000000}
\pgfsetstrokecolor{dialinecolor}
\pgfsetstrokeopacity{1.000000}
\definecolor{diafillcolor}{rgb}{0.000000, 0.000000, 0.000000}
\pgfsetfillcolor{diafillcolor}
\pgfsetfillopacity{1.000000}
\node[anchor=base,inner sep=0pt, outer sep=0pt,color=dialinecolor] at (27.430614\du,13.493442\du){$R_{a,1}$};
\pgfsetlinewidth{0.100000\du}
\pgfsetdash{}{0pt}
\pgfsetmiterjoin
\definecolor{diafillcolor}{rgb}{1.000000, 1.000000, 1.000000}
\pgfsetfillcolor{diafillcolor}
\pgfsetfillopacity{1.000000}
\pgfpathellipse{\pgfpoint{27.475614\du}{17.238442\du}}{\pgfpoint{1.349214\du}{0\du}}{\pgfpoint{0\du}{1.217042\du}}
\pgfusepath{fill}
\definecolor{dialinecolor}{rgb}{0.000000, 0.000000, 0.000000}
\pgfsetstrokecolor{dialinecolor}
\pgfsetstrokeopacity{1.000000}
\pgfpathellipse{\pgfpoint{27.475614\du}{17.238442\du}}{\pgfpoint{1.349214\du}{0\du}}{\pgfpoint{0\du}{1.217042\du}}
\pgfusepath{stroke}
% setfont left to latex
\definecolor{dialinecolor}{rgb}{0.000000, 0.000000, 0.000000}
\pgfsetstrokecolor{dialinecolor}
\pgfsetstrokeopacity{1.000000}
\definecolor{diafillcolor}{rgb}{0.000000, 0.000000, 0.000000}
\pgfsetfillcolor{diafillcolor}
\pgfsetfillopacity{1.000000}
\node[anchor=base,inner sep=0pt, outer sep=0pt,color=dialinecolor] at (27.475614\du,17.433442\du){$R_{a,2}$};
\pgfsetlinewidth{0.100000\du}
\pgfsetdash{}{0pt}
\pgfsetmiterjoin
\definecolor{diafillcolor}{rgb}{1.000000, 1.000000, 1.000000}
\pgfsetfillcolor{diafillcolor}
\pgfsetfillopacity{1.000000}
\pgfpathellipse{\pgfpoint{27.570578\du}{23.578496\du}}{\pgfpoint{1.376878\du}{0\du}}{\pgfpoint{0\du}{1.241996\du}}
\pgfusepath{fill}
\definecolor{dialinecolor}{rgb}{0.000000, 0.000000, 0.000000}
\pgfsetstrokecolor{dialinecolor}
\pgfsetstrokeopacity{1.000000}
\pgfpathellipse{\pgfpoint{27.570578\du}{23.578496\du}}{\pgfpoint{1.376878\du}{0\du}}{\pgfpoint{0\du}{1.241996\du}}
\pgfusepath{stroke}
% setfont left to latex
\definecolor{dialinecolor}{rgb}{0.000000, 0.000000, 0.000000}
\pgfsetstrokecolor{dialinecolor}
\pgfsetstrokeopacity{1.000000}
\definecolor{diafillcolor}{rgb}{0.000000, 0.000000, 0.000000}
\pgfsetfillcolor{diafillcolor}
\pgfsetfillopacity{1.000000}
\node[anchor=base,inner sep=0pt, outer sep=0pt,color=dialinecolor] at (27.570578\du,23.773496\du){$R_{a,N}$};
\pgfsetlinewidth{0.300000\du}
\pgfsetdash{{\pgflinewidth}{0.200000\du}}{0cm}
\pgfsetbuttcap
{
\definecolor{diafillcolor}{rgb}{0.000000, 0.000000, 0.000000}
\pgfsetfillcolor{diafillcolor}
\pgfsetfillopacity{1.000000}
% was here!!!
\definecolor{dialinecolor}{rgb}{0.000000, 0.000000, 0.000000}
\pgfsetstrokecolor{dialinecolor}
\pgfsetstrokeopacity{1.000000}
\draw (27.540000\du,18.980000\du)--(27.540000\du,21.480000\du);
}
\pgfsetlinewidth{0.100000\du}
\pgfsetdash{}{0pt}
\pgfsetmiterjoin
\definecolor{diafillcolor}{rgb}{1.000000, 1.000000, 1.000000}
\pgfsetfillcolor{diafillcolor}
\pgfsetfillopacity{1.000000}
\pgfpathellipse{\pgfpoint{37.980536\du}{13.298484\du}}{\pgfpoint{1.355136\du}{0\du}}{\pgfpoint{0\du}{1.222384\du}}
\pgfusepath{fill}
\definecolor{dialinecolor}{rgb}{0.000000, 0.000000, 0.000000}
\pgfsetstrokecolor{dialinecolor}
\pgfsetstrokeopacity{1.000000}
\pgfpathellipse{\pgfpoint{37.980536\du}{13.298484\du}}{\pgfpoint{1.355136\du}{0\du}}{\pgfpoint{0\du}{1.222384\du}}
\pgfusepath{stroke}
% setfont left to latex
\definecolor{dialinecolor}{rgb}{0.000000, 0.000000, 0.000000}
\pgfsetstrokecolor{dialinecolor}
\pgfsetstrokeopacity{1.000000}
\definecolor{diafillcolor}{rgb}{0.000000, 0.000000, 0.000000}
\pgfsetfillcolor{diafillcolor}
\pgfsetfillopacity{1.000000}
\node[anchor=base,inner sep=0pt, outer sep=0pt,color=dialinecolor] at (37.980536\du,13.493484\du){$R_{b,1}$};
\pgfsetlinewidth{0.100000\du}
\pgfsetdash{}{0pt}
\pgfsetmiterjoin
\definecolor{diafillcolor}{rgb}{1.000000, 1.000000, 1.000000}
\pgfsetfillcolor{diafillcolor}
\pgfsetfillopacity{1.000000}
\pgfpathellipse{\pgfpoint{37.925536\du}{17.238484\du}}{\pgfpoint{1.355136\du}{0\du}}{\pgfpoint{0\du}{1.222384\du}}
\pgfusepath{fill}
\definecolor{dialinecolor}{rgb}{0.000000, 0.000000, 0.000000}
\pgfsetstrokecolor{dialinecolor}
\pgfsetstrokeopacity{1.000000}
\pgfpathellipse{\pgfpoint{37.925536\du}{17.238484\du}}{\pgfpoint{1.355136\du}{0\du}}{\pgfpoint{0\du}{1.222384\du}}
\pgfusepath{stroke}
% setfont left to latex
\definecolor{dialinecolor}{rgb}{0.000000, 0.000000, 0.000000}
\pgfsetstrokecolor{dialinecolor}
\pgfsetstrokeopacity{1.000000}
\definecolor{diafillcolor}{rgb}{0.000000, 0.000000, 0.000000}
\pgfsetfillcolor{diafillcolor}
\pgfsetfillopacity{1.000000}
\node[anchor=base,inner sep=0pt, outer sep=0pt,color=dialinecolor] at (37.925536\du,17.433484\du){$R_{b,2}$};
\pgfsetlinewidth{0.100000\du}
\pgfsetdash{}{0pt}
\pgfsetmiterjoin
\definecolor{diafillcolor}{rgb}{1.000000, 1.000000, 1.000000}
\pgfsetfillcolor{diafillcolor}
\pgfsetfillopacity{1.000000}
\pgfpathellipse{\pgfpoint{38.170599\du}{23.478445\du}}{\pgfpoint{1.411299\du}{0\du}}{\pgfpoint{0\du}{1.273045\du}}
\pgfusepath{fill}
\definecolor{dialinecolor}{rgb}{0.000000, 0.000000, 0.000000}
\pgfsetstrokecolor{dialinecolor}
\pgfsetstrokeopacity{1.000000}
\pgfpathellipse{\pgfpoint{38.170599\du}{23.478445\du}}{\pgfpoint{1.411299\du}{0\du}}{\pgfpoint{0\du}{1.273045\du}}
\pgfusepath{stroke}
% setfont left to latex
\definecolor{dialinecolor}{rgb}{0.000000, 0.000000, 0.000000}
\pgfsetstrokecolor{dialinecolor}
\pgfsetstrokeopacity{1.000000}
\definecolor{diafillcolor}{rgb}{0.000000, 0.000000, 0.000000}
\pgfsetfillcolor{diafillcolor}
\pgfsetfillopacity{1.000000}
\node[anchor=base,inner sep=0pt, outer sep=0pt,color=dialinecolor] at (38.170599\du,23.673445\du){$R_{b,M}$};
\pgfsetlinewidth{0.300000\du}
\pgfsetdash{{\pgflinewidth}{0.200000\du}}{0cm}
\pgfsetbuttcap
{
\definecolor{diafillcolor}{rgb}{0.000000, 0.000000, 0.000000}
\pgfsetfillcolor{diafillcolor}
\pgfsetfillopacity{1.000000}
% was here!!!
\definecolor{dialinecolor}{rgb}{0.000000, 0.000000, 0.000000}
\pgfsetstrokecolor{dialinecolor}
\pgfsetstrokeopacity{1.000000}
\draw (37.985000\du,18.970000\du)--(37.985000\du,21.470000\du);
}
\pgfsetlinewidth{0.050000\du}
\pgfsetdash{}{0pt}
\pgfsetbuttcap
{
\definecolor{diafillcolor}{rgb}{0.000000, 1.000000, 0.000000}
\pgfsetfillcolor{diafillcolor}
\pgfsetfillopacity{1.000000}
% was here!!!
\pgfsetarrowsstart{stealth}
\definecolor{dialinecolor}{rgb}{0.000000, 1.000000, 0.000000}
\pgfsetstrokecolor{dialinecolor}
\pgfsetstrokeopacity{1.000000}
\draw (28.384600\du,12.437900\du)--(37.022310\du,12.434128\du);
}
\pgfsetlinewidth{0.050000\du}
\pgfsetdash{}{0pt}
\pgfsetbuttcap
{
\definecolor{diafillcolor}{rgb}{1.000000, 0.000000, 0.000000}
\pgfsetfillcolor{diafillcolor}
\pgfsetfillopacity{1.000000}
% was here!!!
\pgfsetarrowsstart{stealth}
\definecolor{dialinecolor}{rgb}{1.000000, 0.000000, 0.000000}
\pgfsetstrokecolor{dialinecolor}
\pgfsetstrokeopacity{1.000000}
\draw (28.842600\du,24.053700\du)--(37.630518\du,24.654584\du);
}
\pgfsetlinewidth{0.100000\du}
\pgfsetdash{{\pgflinewidth}{0.200000\du}}{0cm}
\pgfsetmiterjoin
\pgfsetbuttcap
{\pgfsetcornersarced{\pgfpoint{0.000000\du}{0.000000\du}}\definecolor{dialinecolor}{rgb}{0.000000, 0.000000, 0.000000}
\pgfsetstrokecolor{dialinecolor}
\pgfsetstrokeopacity{1.000000}
\draw (25.638764\du,11.830038\du)--(25.638764\du,25.156536\du)--(29.721269\du,25.156536\du)--(29.721269\du,11.830038\du)--cycle;
}\pgfsetlinewidth{0.100000\du}
\pgfsetdash{{\pgflinewidth}{0.200000\du}}{0cm}
\pgfsetmiterjoin
\pgfsetbuttcap
{\pgfsetcornersarced{\pgfpoint{0.000000\du}{0.000000\du}}\definecolor{dialinecolor}{rgb}{0.000000, 0.000000, 0.000000}
\pgfsetstrokecolor{dialinecolor}
\pgfsetstrokeopacity{1.000000}
\draw (36.096688\du,11.857262\du)--(36.096688\du,25.128574\du)--(39.871607\du,25.128574\du)--(39.871607\du,11.857262\du)--cycle;
}% setfont left to latex
\definecolor{dialinecolor}{rgb}{0.000000, 0.000000, 0.000000}
\pgfsetstrokecolor{dialinecolor}
\pgfsetstrokeopacity{1.000000}
\definecolor{diafillcolor}{rgb}{0.000000, 0.000000, 0.000000}
\pgfsetfillcolor{diafillcolor}
\pgfsetfillopacity{1.000000}
\node[anchor=base west,inner sep=0pt,outer sep=0pt,color=dialinecolor] at (26.470600\du,11.464400\du){Saut a};
% setfont left to latex
\definecolor{dialinecolor}{rgb}{0.000000, 0.000000, 0.000000}
\pgfsetstrokecolor{dialinecolor}
\pgfsetstrokeopacity{1.000000}
\definecolor{diafillcolor}{rgb}{0.000000, 0.000000, 0.000000}
\pgfsetfillcolor{diafillcolor}
\pgfsetfillopacity{1.000000}
\node[anchor=base west,inner sep=0pt,outer sep=0pt,color=dialinecolor] at (36.898700\du,11.464100\du){Saut b};
\pgfsetlinewidth{0.050000\du}
\pgfsetdash{}{0pt}
\pgfsetbuttcap
{
\definecolor{diafillcolor}{rgb}{0.000000, 0.000000, 1.000000}
\pgfsetfillcolor{diafillcolor}
\pgfsetfillopacity{1.000000}
% was here!!!
\pgfsetarrowsstart{stealth}
\definecolor{dialinecolor}{rgb}{0.000000, 0.000000, 1.000000}
\pgfsetstrokecolor{dialinecolor}
\pgfsetstrokeopacity{1.000000}
\draw (28.779828\du,13.298442\du)--(37.406948\du,16.109148\du);
}
\pgfsetlinewidth{0.050000\du}
\pgfsetdash{}{0pt}
\pgfsetbuttcap
{
\definecolor{diafillcolor}{rgb}{0.000000, 1.000000, 0.000000}
\pgfsetfillcolor{diafillcolor}
\pgfsetfillopacity{1.000000}
% was here!!!
\pgfsetarrowsstart{stealth}
\definecolor{dialinecolor}{rgb}{0.000000, 1.000000, 0.000000}
\pgfsetstrokecolor{dialinecolor}
\pgfsetstrokeopacity{1.000000}
\draw (28.722125\du,16.772700\du)--(36.728554\du,12.830698\du);
}
\pgfsetlinewidth{0.050000\du}
\pgfsetdash{}{0pt}
\pgfsetbuttcap
{
\definecolor{diafillcolor}{rgb}{0.000000, 1.000000, 0.000000}
\pgfsetfillcolor{diafillcolor}
\pgfsetfillopacity{1.000000}
% was here!!!
\pgfsetarrowsstart{stealth}
\definecolor{dialinecolor}{rgb}{0.000000, 1.000000, 0.000000}
\pgfsetstrokecolor{dialinecolor}
\pgfsetstrokeopacity{1.000000}
\draw (28.602774\du,20.095348\du)--(36.625400\du,13.298484\du);
}
\pgfsetlinewidth{0.050000\du}
\pgfsetdash{}{0pt}
\pgfsetbuttcap
{
\definecolor{diafillcolor}{rgb}{0.000000, 1.000000, 0.000000}
\pgfsetfillcolor{diafillcolor}
\pgfsetfillopacity{1.000000}
% was here!!!
\pgfsetarrowsstart{stealth}
\definecolor{dialinecolor}{rgb}{0.000000, 1.000000, 0.000000}
\pgfsetstrokecolor{dialinecolor}
\pgfsetstrokeopacity{1.000000}
\draw (28.097487\du,22.431041\du)--(36.728554\du,13.766270\du);
}
\pgfsetlinewidth{0.050000\du}
\pgfsetdash{}{0pt}
\pgfsetbuttcap
{
\definecolor{diafillcolor}{rgb}{0.000000, 0.000000, 1.000000}
\pgfsetfillcolor{diafillcolor}
\pgfsetfillopacity{1.000000}
% was here!!!
\pgfsetarrowsstart{stealth}
\definecolor{dialinecolor}{rgb}{0.000000, 0.000000, 1.000000}
\pgfsetstrokecolor{dialinecolor}
\pgfsetstrokeopacity{1.000000}
\draw (28.824828\du,17.238442\du)--(36.967310\du,16.374128\du);
}
\pgfsetlinewidth{0.050000\du}
\pgfsetdash{}{0pt}
\pgfsetbuttcap
{
\definecolor{diafillcolor}{rgb}{0.000000, 0.000000, 1.000000}
\pgfsetfillcolor{diafillcolor}
\pgfsetfillopacity{1.000000}
% was here!!!
\pgfsetarrowsstart{stealth}
\definecolor{dialinecolor}{rgb}{0.000000, 0.000000, 1.000000}
\pgfsetstrokecolor{dialinecolor}
\pgfsetstrokeopacity{1.000000}
\draw (28.798511\du,20.486821\du)--(36.673554\du,16.770698\du);
}
\pgfsetlinewidth{0.050000\du}
\pgfsetdash{}{0pt}
\pgfsetbuttcap
{
\definecolor{diafillcolor}{rgb}{0.000000, 0.000000, 1.000000}
\pgfsetfillcolor{diafillcolor}
\pgfsetfillopacity{1.000000}
% was here!!!
\pgfsetarrowsstart{stealth}
\definecolor{dialinecolor}{rgb}{0.000000, 0.000000, 1.000000}
\pgfsetstrokecolor{dialinecolor}
\pgfsetstrokeopacity{1.000000}
\draw (28.544178\du,22.700272\du)--(36.516123\du,17.355037\du);
}
\pgfsetlinewidth{0.050000\du}
\pgfsetdash{}{0pt}
\pgfsetbuttcap
{
\definecolor{diafillcolor}{rgb}{1.000000, 0.000000, 0.000000}
\pgfsetfillcolor{diafillcolor}
\pgfsetfillopacity{1.000000}
% was here!!!
\pgfsetarrowsstart{stealth}
\definecolor{dialinecolor}{rgb}{1.000000, 0.000000, 0.000000}
\pgfsetstrokecolor{dialinecolor}
\pgfsetstrokeopacity{1.000000}
\draw (28.938323\du,20.850332\du)--(36.866729\du,23.965618\du);
}
\pgfsetlinewidth{0.050000\du}
\pgfsetdash{}{0pt}
\pgfsetbuttcap
{
\definecolor{diafillcolor}{rgb}{1.000000, 0.000000, 0.000000}
\pgfsetfillcolor{diafillcolor}
\pgfsetfillopacity{1.000000}
% was here!!!
\pgfsetarrowsstart{stealth}
\definecolor{dialinecolor}{rgb}{1.000000, 0.000000, 0.000000}
\pgfsetstrokecolor{dialinecolor}
\pgfsetstrokeopacity{1.000000}
\draw (28.722125\du,17.704183\du)--(36.712688\du,23.381227\du);
}
\pgfsetlinewidth{0.050000\du}
\pgfsetdash{}{0pt}
\pgfsetbuttcap
{
\definecolor{diafillcolor}{rgb}{1.000000, 0.000000, 0.000000}
\pgfsetfillcolor{diafillcolor}
\pgfsetfillopacity{1.000000}
% was here!!!
\pgfsetarrowsstart{stealth}
\definecolor{dialinecolor}{rgb}{1.000000, 0.000000, 0.000000}
\pgfsetstrokecolor{dialinecolor}
\pgfsetstrokeopacity{1.000000}
\draw (28.677125\du,13.764183\du)--(36.866729\du,22.991272\du);
}
\end{tikzpicture}

	\caption{Inférence des liens possibles entre les routeurs des deux sauts consécutifs RAi et RBj}
	\label{fig:link-inference}
\end{figure}
\paragraph{5. Caractérisation des liens} avec leurs RTTs différentiels. A cette étape, on calcul le RTT différentiel d'un lien en calculant la différence entre les RTTs\footnote{C'est la médiane calculée à l'étape 3.} des deux routeurs du lien en question. En plus du RTT différentiel, on note aussi la sonde Atlas ayant effectué la requête traceroute où le lien a été identifié. 

\paragraph{6. Fusion des informations d'un lien. } Etant donné qu'un lien (IP1, IP2) peut être identifié plusieurs fois pendant une même période $d_i$ d'une part, et le lien (IP2, IP1) est similaire\footnote{La similarité est mesurée par le RTT différentiel.} au lien  (IP1, IP2) d'autre part, la fusion permet de construire une nouvelle distribution des RTTs différentiels caractérisant le lien (IP1, IP2) qui reprend les RTTs différentiels du (IP1, IP2) et du (IP2, IP1).


A la fin de l'étape 6, tous les traceroutes sont analysés tout en identifiant leurs liens, et ce par $d_i$. A présent, l'objectif c'est d'identifier les dates pendant lesquelles des anomalies ont été détectées. Pour ce faire, l'idée du travail de référence c'est de conserver, pour un lien donné, une référence du RTT différentiel médian qui sera d'abord comparée avec la médiane courante du RTT différentiel et ensuite mettre à jour cette référence tout au long de la période de l'analyse.

  
  \paragraph{7. Calcul de la médiane des RTTs différentiels et   l'intervalle de confiance courant} du lien analysé. Pour un lien donné, on calcule la médiane des RTTs différentiels d'une $d_i$, ensuite on calcule les deux bornes de l'intervalle de confiance pour $d_i$.
  
  \paragraph{8. Mise à jour de la médiane et de l'intervalle de  référence du lien analysé.} La médiane des RTTs différentiels de référence sont d'abords comparés avec ceux de la période $d_i$ courante. Ensuite, ces références sont mises à jour pour prendre en compte ces nouvelles valeurs. A l'issue de cette comparaison, la liste des dates des anomalies est mise à jour.
  


\subsection{Vue globale des étapes de la détection des anomalies}
La figure 	\ref{fig:process-rttanalysis_tex} présente la succession des étapes de la détection des anomalies dans les délais d'un lien donné. 

\begin{figure}[h]
	\centering
	\resizebox{\textwidth}{\textheight}{
		% Graphic for TeX using PGF
% Title: /home/hayat/RipeAtlasTraceroutesAnalysis/dia/process-rttanalysis.dia
% Creator: Dia v0.97+git
% CreationDate: Thu Nov 29 01:56:23 2018
% For: hayat
% \usepackage{tikz}
% The following commands are not supported in PSTricks at present
% We define them conditionally, so when they are implemented,
% this pgf file will use them.
\ifx\du\undefined
  \newlength{\du}
\fi
\setlength{\du}{15\unitlength}
\begin{tikzpicture}[even odd rule]
\pgftransformxscale{1.000000}
\pgftransformyscale{-1.000000}
\definecolor{dialinecolor}{rgb}{0.000000, 0.000000, 0.000000}
\pgfsetstrokecolor{dialinecolor}
\pgfsetstrokeopacity{1.000000}
\definecolor{diafillcolor}{rgb}{1.000000, 1.000000, 1.000000}
\pgfsetfillcolor{diafillcolor}
\pgfsetfillopacity{1.000000}
\pgfsetlinewidth{0.100000\du}
\pgfsetdash{}{0pt}
\pgfsetmiterjoin
{\pgfsetcornersarced{\pgfpoint{0.000000\du}{0.000000\du}}\definecolor{diafillcolor}{rgb}{1.000000, 1.000000, 1.000000}
\pgfsetfillcolor{diafillcolor}
\pgfsetfillopacity{1.000000}
\fill (8.138750\du,-5.900000\du)--(8.138750\du,-4.000000\du)--(21.861250\du,-4.000000\du)--(21.861250\du,-5.900000\du)--cycle;
}{\pgfsetcornersarced{\pgfpoint{0.000000\du}{0.000000\du}}\definecolor{dialinecolor}{rgb}{0.000000, 0.000000, 0.000000}
\pgfsetstrokecolor{dialinecolor}
\pgfsetstrokeopacity{1.000000}
\draw (8.138750\du,-5.900000\du)--(8.138750\du,-4.000000\du)--(21.861250\du,-4.000000\du)--(21.861250\du,-5.900000\du)--cycle;
}% setfont left to latex
\definecolor{dialinecolor}{rgb}{0.000000, 0.000000, 0.000000}
\pgfsetstrokecolor{dialinecolor}
\pgfsetstrokeopacity{1.000000}
\definecolor{diafillcolor}{rgb}{0.000000, 0.000000, 0.000000}
\pgfsetfillcolor{diafillcolor}
\pgfsetfillopacity{1.000000}
\node[anchor=base,inner sep=0pt, outer sep=0pt,color=dialinecolor] at (15.000000\du,-4.755000\du){\ensuremath{[}traceroute:\{from:"",type:"", result:""\}\ensuremath{]}};
\pgfsetlinewidth{0.100000\du}
\pgfsetdash{}{0pt}
\pgfsetbuttcap
{
\definecolor{diafillcolor}{rgb}{0.000000, 0.000000, 0.000000}
\pgfsetfillcolor{diafillcolor}
\pgfsetfillopacity{1.000000}
% was here!!!
\pgfsetarrowsend{stealth}
\definecolor{dialinecolor}{rgb}{0.000000, 0.000000, 0.000000}
\pgfsetstrokecolor{dialinecolor}
\pgfsetstrokeopacity{1.000000}
\draw (20.060900\du,1.256430\du)--(15.048400\du,4.142720\du);
}
\pgfsetlinewidth{0.100000\du}
\pgfsetdash{}{0pt}
\pgfsetmiterjoin
{\pgfsetcornersarced{\pgfpoint{0.000000\du}{0.000000\du}}\definecolor{diafillcolor}{rgb}{1.000000, 1.000000, 1.000000}
\pgfsetfillcolor{diafillcolor}
\pgfsetfillopacity{1.000000}
\fill (4.175940\du,4.142080\du)--(4.175940\du,6.042080\du)--(9.163440\du,6.042080\du)--(9.163440\du,4.142080\du)--cycle;
}{\pgfsetcornersarced{\pgfpoint{0.000000\du}{0.000000\du}}\definecolor{dialinecolor}{rgb}{0.000000, 0.000000, 0.000000}
\pgfsetstrokecolor{dialinecolor}
\pgfsetstrokeopacity{1.000000}
\draw (4.175940\du,4.142080\du)--(4.175940\du,6.042080\du)--(9.163440\du,6.042080\du)--(9.163440\du,4.142080\du)--cycle;
}% setfont left to latex
\definecolor{dialinecolor}{rgb}{0.000000, 0.000000, 0.000000}
\pgfsetstrokecolor{dialinecolor}
\pgfsetstrokeopacity{1.000000}
\definecolor{diafillcolor}{rgb}{0.000000, 0.000000, 0.000000}
\pgfsetfillcolor{diafillcolor}
\pgfsetfillopacity{1.000000}
\node[anchor=base,inner sep=0pt, outer sep=0pt,color=dialinecolor] at (6.669690\du,5.287080\du){\ensuremath{[}Traceroute\ensuremath{]}};
% setfont left to latex
\definecolor{dialinecolor}{rgb}{0.000000, 0.000000, 0.000000}
\pgfsetstrokecolor{dialinecolor}
\pgfsetstrokeopacity{1.000000}
\definecolor{diafillcolor}{rgb}{0.000000, 0.000000, 0.000000}
\pgfsetfillcolor{diafillcolor}
\pgfsetfillopacity{1.000000}
\node[anchor=base west,inner sep=0pt,outer sep=0pt,color=dialinecolor] at (16.400000\du,9.562040\du){};
\pgfsetlinewidth{0.100000\du}
\pgfsetdash{}{0pt}
\pgfsetbuttcap
\pgfsetmiterjoin
\pgfsetlinewidth{0.100000\du}
\pgfsetbuttcap
\pgfsetmiterjoin
\pgfsetdash{}{0pt}
\definecolor{diafillcolor}{rgb}{1.000000, 1.000000, 1.000000}
\pgfsetfillcolor{diafillcolor}
\pgfsetfillopacity{1.000000}
\definecolor{dialinecolor}{rgb}{0.000000, 0.000000, 0.000000}
\pgfsetstrokecolor{dialinecolor}
\pgfsetstrokeopacity{1.000000}
\pgfpathmoveto{\pgfpoint{24.362725\du}{6.967570\du}}
\pgfpathlineto{\pgfpoint{38.555225\du}{6.967570\du}}
\pgfpathcurveto{\pgfpoint{40.514801\du}{6.967570\du}}{\pgfpoint{42.103350\du}{7.213813\du}}{\pgfpoint{42.103350\du}{7.517570\du}}
\pgfpathcurveto{\pgfpoint{42.103350\du}{7.821327\du}}{\pgfpoint{40.514801\du}{8.067570\du}}{\pgfpoint{38.555225\du}{8.067570\du}}
\pgfpathlineto{\pgfpoint{24.362725\du}{8.067570\du}}
\pgfpathcurveto{\pgfpoint{22.403149\du}{8.067570\du}}{\pgfpoint{20.814600\du}{7.821327\du}}{\pgfpoint{20.814600\du}{7.517570\du}}
\pgfpathcurveto{\pgfpoint{20.814600\du}{7.213813\du}}{\pgfpoint{22.403149\du}{6.967570\du}}{\pgfpoint{24.362725\du}{6.967570\du}}
\pgfpathclose
\pgfusepath{fill,stroke}
% setfont left to latex
\definecolor{dialinecolor}{rgb}{0.000000, 0.000000, 0.000000}
\pgfsetstrokecolor{dialinecolor}
\pgfsetstrokeopacity{1.000000}
\definecolor{diafillcolor}{rgb}{0.000000, 0.000000, 0.000000}
\pgfsetfillcolor{diafillcolor}
\pgfsetfillopacity{1.000000}
\node[anchor=base,inner sep=0pt, outer sep=0pt,color=dialinecolor] at (31.458975\du,7.717570\du){2.Elimination des traceroutes échoués, etc.};
\pgfsetlinewidth{0.100000\du}
\pgfsetdash{}{0pt}
\pgfsetmiterjoin
{\pgfsetcornersarced{\pgfpoint{0.000000\du}{0.000000\du}}\definecolor{diafillcolor}{rgb}{1.000000, 1.000000, 1.000000}
\pgfsetfillcolor{diafillcolor}
\pgfsetfillopacity{1.000000}
\fill (10.748200\du,8.560140\du)--(10.748200\du,10.460140\du)--(19.318200\du,10.460140\du)--(19.318200\du,8.560140\du)--cycle;
}{\pgfsetcornersarced{\pgfpoint{0.000000\du}{0.000000\du}}\definecolor{dialinecolor}{rgb}{0.000000, 0.000000, 0.000000}
\pgfsetstrokecolor{dialinecolor}
\pgfsetstrokeopacity{1.000000}
\draw (10.748200\du,8.560140\du)--(10.748200\du,10.460140\du)--(19.318200\du,10.460140\du)--(19.318200\du,8.560140\du)--cycle;
}% setfont left to latex
\definecolor{dialinecolor}{rgb}{0.000000, 0.000000, 0.000000}
\pgfsetstrokecolor{dialinecolor}
\pgfsetstrokeopacity{1.000000}
\definecolor{diafillcolor}{rgb}{0.000000, 0.000000, 0.000000}
\pgfsetfillcolor{diafillcolor}
\pgfsetfillopacity{1.000000}
\node[anchor=base,inner sep=0pt, outer sep=0pt,color=dialinecolor] at (15.033200\du,9.705140\du){\ensuremath{[}Traceroute\ensuremath{]}};
\pgfsetlinewidth{0.100000\du}
\pgfsetdash{}{0pt}
\pgfsetbuttcap
{
\definecolor{diafillcolor}{rgb}{0.000000, 0.000000, 0.000000}
\pgfsetfillcolor{diafillcolor}
\pgfsetfillopacity{1.000000}
% was here!!!
\pgfsetarrowsend{stealth}
\definecolor{dialinecolor}{rgb}{0.000000, 0.000000, 0.000000}
\pgfsetstrokecolor{dialinecolor}
\pgfsetstrokeopacity{1.000000}
\draw (15.063300\du,6.216320\du)--(15.033200\du,8.560140\du);
}
\pgfsetlinewidth{0.100000\du}
\pgfsetdash{}{0pt}
\pgfsetmiterjoin
{\pgfsetcornersarced{\pgfpoint{0.000000\du}{0.000000\du}}\definecolor{diafillcolor}{rgb}{1.000000, 1.000000, 1.000000}
\pgfsetfillcolor{diafillcolor}
\pgfsetfillopacity{1.000000}
\fill (11.463400\du,17.484800\du)--(11.463400\du,19.384800\du)--(18.483400\du,19.384800\du)--(18.483400\du,17.484800\du)--cycle;
}{\pgfsetcornersarced{\pgfpoint{0.000000\du}{0.000000\du}}\definecolor{dialinecolor}{rgb}{0.000000, 0.000000, 0.000000}
\pgfsetstrokecolor{dialinecolor}
\pgfsetstrokeopacity{1.000000}
\draw (11.463400\du,17.484800\du)--(11.463400\du,19.384800\du)--(18.483400\du,19.384800\du)--(18.483400\du,17.484800\du)--cycle;
}% setfont left to latex
\definecolor{dialinecolor}{rgb}{0.000000, 0.000000, 0.000000}
\pgfsetstrokecolor{dialinecolor}
\pgfsetstrokeopacity{1.000000}
\definecolor{diafillcolor}{rgb}{0.000000, 0.000000, 0.000000}
\pgfsetfillcolor{diafillcolor}
\pgfsetfillopacity{1.000000}
\node[anchor=base,inner sep=0pt, outer sep=0pt,color=dialinecolor] at (14.973400\du,18.629800\du){\ensuremath{[}LinksTraceroute\ensuremath{]}};
\pgfsetlinewidth{0.100000\du}
\pgfsetdash{}{0pt}
\pgfsetbuttcap
{
\definecolor{diafillcolor}{rgb}{0.000000, 0.000000, 0.000000}
\pgfsetfillcolor{diafillcolor}
\pgfsetfillopacity{1.000000}
% was here!!!
\pgfsetarrowsend{stealth}
\definecolor{dialinecolor}{rgb}{0.000000, 0.000000, 0.000000}
\pgfsetstrokecolor{dialinecolor}
\pgfsetstrokeopacity{1.000000}
\draw (14.954000\du,14.821200\du)--(14.973400\du,17.484800\du);
}
\pgfsetlinewidth{0.100000\du}
\pgfsetdash{}{0pt}
\pgfsetbuttcap
{
\definecolor{diafillcolor}{rgb}{0.000000, 0.000000, 0.000000}
\pgfsetfillcolor{diafillcolor}
\pgfsetfillopacity{1.000000}
% was here!!!
\pgfsetarrowsend{stealth}
\definecolor{dialinecolor}{rgb}{0.000000, 0.000000, 0.000000}
\pgfsetstrokecolor{dialinecolor}
\pgfsetstrokeopacity{1.000000}
\draw (15.033200\du,10.460100\du)--(14.954000\du,12.921200\du);
}
\pgfsetlinewidth{0.100000\du}
\pgfsetdash{}{0pt}
\pgfsetmiterjoin
{\pgfsetcornersarced{\pgfpoint{0.000000\du}{0.000000\du}}\definecolor{diafillcolor}{rgb}{1.000000, 1.000000, 1.000000}
\pgfsetfillcolor{diafillcolor}
\pgfsetfillopacity{1.000000}
\fill (10.244000\du,12.921200\du)--(10.244000\du,14.821200\du)--(19.664000\du,14.821200\du)--(19.664000\du,12.921200\du)--cycle;
}{\pgfsetcornersarced{\pgfpoint{0.000000\du}{0.000000\du}}\definecolor{dialinecolor}{rgb}{0.000000, 0.000000, 0.000000}
\pgfsetstrokecolor{dialinecolor}
\pgfsetstrokeopacity{1.000000}
\draw (10.244000\du,12.921200\du)--(10.244000\du,14.821200\du)--(19.664000\du,14.821200\du)--(19.664000\du,12.921200\du)--cycle;
}% setfont left to latex
\definecolor{dialinecolor}{rgb}{0.000000, 0.000000, 0.000000}
\pgfsetstrokecolor{dialinecolor}
\pgfsetstrokeopacity{1.000000}
\definecolor{diafillcolor}{rgb}{0.000000, 0.000000, 0.000000}
\pgfsetfillcolor{diafillcolor}
\pgfsetfillopacity{1.000000}
\node[anchor=base,inner sep=0pt, outer sep=0pt,color=dialinecolor] at (14.954000\du,14.066200\du){\ensuremath{[}MedianByHopTraceroute\ensuremath{]}};
\pgfsetlinewidth{0.100000\du}
\pgfsetdash{}{0pt}
\pgfsetbuttcap
\pgfsetmiterjoin
\pgfsetlinewidth{0.100000\du}
\pgfsetbuttcap
\pgfsetmiterjoin
\pgfsetdash{}{0pt}
\definecolor{diafillcolor}{rgb}{1.000000, 1.000000, 1.000000}
\pgfsetfillcolor{diafillcolor}
\pgfsetfillopacity{1.000000}
\definecolor{dialinecolor}{rgb}{0.000000, 0.000000, 0.000000}
\pgfsetstrokecolor{dialinecolor}
\pgfsetstrokeopacity{1.000000}
\pgfpathmoveto{\pgfpoint{21.475800\du}{11.116500\du}}
\pgfpathlineto{\pgfpoint{31.295800\du}{11.116500\du}}
\pgfpathcurveto{\pgfpoint{32.651660\du}{11.116500\du}}{\pgfpoint{33.750800\du}{11.362743\du}}{\pgfpoint{33.750800\du}{11.666500\du}}
\pgfpathcurveto{\pgfpoint{33.750800\du}{11.970257\du}}{\pgfpoint{32.651660\du}{12.216500\du}}{\pgfpoint{31.295800\du}{12.216500\du}}
\pgfpathlineto{\pgfpoint{21.475800\du}{12.216500\du}}
\pgfpathcurveto{\pgfpoint{20.119940\du}{12.216500\du}}{\pgfpoint{19.020800\du}{11.970257\du}}{\pgfpoint{19.020800\du}{11.666500\du}}
\pgfpathcurveto{\pgfpoint{19.020800\du}{11.362743\du}}{\pgfpoint{20.119940\du}{11.116500\du}}{\pgfpoint{21.475800\du}{11.116500\du}}
\pgfpathclose
\pgfusepath{fill,stroke}
% setfont left to latex
\definecolor{dialinecolor}{rgb}{0.000000, 0.000000, 0.000000}
\pgfsetstrokecolor{dialinecolor}
\pgfsetstrokeopacity{1.000000}
\definecolor{diafillcolor}{rgb}{0.000000, 0.000000, 0.000000}
\pgfsetfillcolor{diafillcolor}
\pgfsetfillopacity{1.000000}
\node[anchor=base,inner sep=0pt, outer sep=0pt,color=dialinecolor] at (26.385800\du,11.866500\du){3. calcul de mediane par saut};
\pgfsetlinewidth{0.100000\du}
\pgfsetdash{}{0pt}
\pgfsetbuttcap
\pgfsetmiterjoin
\pgfsetlinewidth{0.100000\du}
\pgfsetbuttcap
\pgfsetmiterjoin
\pgfsetdash{}{0pt}
\definecolor{diafillcolor}{rgb}{1.000000, 1.000000, 1.000000}
\pgfsetfillcolor{diafillcolor}
\pgfsetfillopacity{1.000000}
\definecolor{dialinecolor}{rgb}{0.000000, 0.000000, 0.000000}
\pgfsetstrokecolor{dialinecolor}
\pgfsetstrokeopacity{1.000000}
\pgfpathmoveto{\pgfpoint{20.956000\du}{15.486500\du}}
\pgfpathlineto{\pgfpoint{28.296000\du}{15.486500\du}}
\pgfpathcurveto{\pgfpoint{29.309443\du}{15.486500\du}}{\pgfpoint{30.131000\du}{15.732743\du}}{\pgfpoint{30.131000\du}{16.036500\du}}
\pgfpathcurveto{\pgfpoint{30.131000\du}{16.340257\du}}{\pgfpoint{29.309443\du}{16.586500\du}}{\pgfpoint{28.296000\du}{16.586500\du}}
\pgfpathlineto{\pgfpoint{20.956000\du}{16.586500\du}}
\pgfpathcurveto{\pgfpoint{19.942557\du}{16.586500\du}}{\pgfpoint{19.121000\du}{16.340257\du}}{\pgfpoint{19.121000\du}{16.036500\du}}
\pgfpathcurveto{\pgfpoint{19.121000\du}{15.732743\du}}{\pgfpoint{19.942557\du}{15.486500\du}}{\pgfpoint{20.956000\du}{15.486500\du}}
\pgfpathclose
\pgfusepath{fill,stroke}
% setfont left to latex
\definecolor{dialinecolor}{rgb}{0.000000, 0.000000, 0.000000}
\pgfsetstrokecolor{dialinecolor}
\pgfsetstrokeopacity{1.000000}
\definecolor{diafillcolor}{rgb}{0.000000, 0.000000, 0.000000}
\pgfsetfillcolor{diafillcolor}
\pgfsetfillopacity{1.000000}
\node[anchor=base,inner sep=0pt, outer sep=0pt,color=dialinecolor] at (24.626000\du,16.236500\du){4. inférence des liens };
\pgfsetlinewidth{0.100000\du}
\pgfsetdash{}{0pt}
\pgfsetmiterjoin
{\pgfsetcornersarced{\pgfpoint{0.000000\du}{0.000000\du}}\definecolor{diafillcolor}{rgb}{1.000000, 1.000000, 1.000000}
\pgfsetfillcolor{diafillcolor}
\pgfsetfillopacity{1.000000}
\fill (13.005900\du,21.418700\du)--(13.005900\du,23.318700\du)--(17.040900\du,23.318700\du)--(17.040900\du,21.418700\du)--cycle;
}{\pgfsetcornersarced{\pgfpoint{0.000000\du}{0.000000\du}}\definecolor{dialinecolor}{rgb}{0.000000, 0.000000, 0.000000}
\pgfsetstrokecolor{dialinecolor}
\pgfsetstrokeopacity{1.000000}
\draw (13.005900\du,21.418700\du)--(13.005900\du,23.318700\du)--(17.040900\du,23.318700\du)--(17.040900\du,21.418700\du)--cycle;
}% setfont left to latex
\definecolor{dialinecolor}{rgb}{0.000000, 0.000000, 0.000000}
\pgfsetstrokecolor{dialinecolor}
\pgfsetstrokeopacity{1.000000}
\definecolor{diafillcolor}{rgb}{0.000000, 0.000000, 0.000000}
\pgfsetfillcolor{diafillcolor}
\pgfsetfillopacity{1.000000}
\node[anchor=base,inner sep=0pt, outer sep=0pt,color=dialinecolor] at (15.023400\du,22.563700\du){\ensuremath{[}DiffRTT\ensuremath{]}};
\pgfsetlinewidth{0.100000\du}
\pgfsetdash{}{0pt}
\pgfsetbuttcap
{
\definecolor{diafillcolor}{rgb}{0.000000, 0.000000, 0.000000}
\pgfsetfillcolor{diafillcolor}
\pgfsetfillopacity{1.000000}
% was here!!!
\pgfsetarrowsend{stealth}
\definecolor{dialinecolor}{rgb}{0.000000, 0.000000, 0.000000}
\pgfsetstrokecolor{dialinecolor}
\pgfsetstrokeopacity{1.000000}
\draw (14.973400\du,19.384800\du)--(15.006640\du,21.368482\du);
}
\pgfsetlinewidth{0.100000\du}
\pgfsetdash{}{0pt}
\pgfsetbuttcap
\pgfsetmiterjoin
\pgfsetlinewidth{0.100000\du}
\pgfsetbuttcap
\pgfsetmiterjoin
\pgfsetdash{}{0pt}
\definecolor{diafillcolor}{rgb}{1.000000, 1.000000, 1.000000}
\pgfsetfillcolor{diafillcolor}
\pgfsetfillopacity{1.000000}
\definecolor{dialinecolor}{rgb}{0.000000, 0.000000, 0.000000}
\pgfsetstrokecolor{dialinecolor}
\pgfsetstrokeopacity{1.000000}
\pgfpathmoveto{\pgfpoint{22.489775\du}{20.027600\du}}
\pgfpathlineto{\pgfpoint{38.917275\du}{20.027600\du}}
\pgfpathcurveto{\pgfpoint{41.185440\du}{20.027600\du}}{\pgfpoint{43.024150\du}{20.273843\du}}{\pgfpoint{43.024150\du}{20.577600\du}}
\pgfpathcurveto{\pgfpoint{43.024150\du}{20.881357\du}}{\pgfpoint{41.185440\du}{21.127600\du}}{\pgfpoint{38.917275\du}{21.127600\du}}
\pgfpathlineto{\pgfpoint{22.489775\du}{21.127600\du}}
\pgfpathcurveto{\pgfpoint{20.221610\du}{21.127600\du}}{\pgfpoint{18.382900\du}{20.881357\du}}{\pgfpoint{18.382900\du}{20.577600\du}}
\pgfpathcurveto{\pgfpoint{18.382900\du}{20.273843\du}}{\pgfpoint{20.221610\du}{20.027600\du}}{\pgfpoint{22.489775\du}{20.027600\du}}
\pgfpathclose
\pgfusepath{fill,stroke}
% setfont left to latex
\definecolor{dialinecolor}{rgb}{0.000000, 0.000000, 0.000000}
\pgfsetstrokecolor{dialinecolor}
\pgfsetstrokeopacity{1.000000}
\definecolor{diafillcolor}{rgb}{0.000000, 0.000000, 0.000000}
\pgfsetfillcolor{diafillcolor}
\pgfsetfillopacity{1.000000}
\node[anchor=base,inner sep=0pt, outer sep=0pt,color=dialinecolor] at (30.703525\du,20.777600\du){5. Caractérisation de chaque lien en objet DiffRTT };
\pgfsetlinewidth{0.100000\du}
\pgfsetdash{}{0pt}
\pgfsetbuttcap
{
\definecolor{diafillcolor}{rgb}{0.000000, 0.000000, 0.000000}
\pgfsetfillcolor{diafillcolor}
\pgfsetfillopacity{1.000000}
% was here!!!
\pgfsetarrowsend{stealth}
\definecolor{dialinecolor}{rgb}{0.000000, 0.000000, 0.000000}
\pgfsetstrokecolor{dialinecolor}
\pgfsetstrokeopacity{1.000000}
\draw (15.035872\du,23.368063\du)--(15.069300\du,26.046500\du);
}
\pgfsetlinewidth{0.100000\du}
\pgfsetdash{}{0pt}
\pgfsetmiterjoin
{\pgfsetcornersarced{\pgfpoint{0.000000\du}{0.000000\du}}\definecolor{diafillcolor}{rgb}{1.000000, 1.000000, 1.000000}
\pgfsetfillcolor{diafillcolor}
\pgfsetfillopacity{1.000000}
\fill (10.881300\du,-0.650000\du)--(10.881300\du,1.250000\du)--(19.218800\du,1.250000\du)--(19.218800\du,-0.650000\du)--cycle;
}{\pgfsetcornersarced{\pgfpoint{0.000000\du}{0.000000\du}}\definecolor{dialinecolor}{rgb}{0.000000, 0.000000, 0.000000}
\pgfsetstrokecolor{dialinecolor}
\pgfsetstrokeopacity{1.000000}
\draw (10.881300\du,-0.650000\du)--(10.881300\du,1.250000\du)--(19.218800\du,1.250000\du)--(19.218800\du,-0.650000\du)--cycle;
}% setfont left to latex
\definecolor{dialinecolor}{rgb}{0.000000, 0.000000, 0.000000}
\pgfsetstrokecolor{dialinecolor}
\pgfsetstrokeopacity{1.000000}
\definecolor{diafillcolor}{rgb}{0.000000, 0.000000, 0.000000}
\pgfsetfillcolor{diafillcolor}
\pgfsetfillopacity{1.000000}
\node[anchor=base,inner sep=0pt, outer sep=0pt,color=dialinecolor] at (15.050050\du,0.495000\du){\ensuremath{[}TraceroutesPerPeriod\ensuremath{]}};
\pgfsetlinewidth{0.100000\du}
\pgfsetdash{}{0pt}
\pgfsetbuttcap
{
\definecolor{diafillcolor}{rgb}{0.000000, 0.000000, 0.000000}
\pgfsetfillcolor{diafillcolor}
\pgfsetfillopacity{1.000000}
% was here!!!
\pgfsetarrowsend{stealth}
\definecolor{dialinecolor}{rgb}{0.000000, 0.000000, 0.000000}
\pgfsetstrokecolor{dialinecolor}
\pgfsetstrokeopacity{1.000000}
\draw (15.000000\du,-4.000000\du)--(15.050100\du,-0.650000\du);
}
\pgfsetlinewidth{0.100000\du}
\pgfsetdash{}{0pt}
\pgfsetbuttcap
\pgfsetmiterjoin
\pgfsetlinewidth{0.100000\du}
\pgfsetbuttcap
\pgfsetmiterjoin
\pgfsetdash{}{0pt}
\definecolor{diafillcolor}{rgb}{1.000000, 1.000000, 1.000000}
\pgfsetfillcolor{diafillcolor}
\pgfsetfillopacity{1.000000}
\definecolor{dialinecolor}{rgb}{0.000000, 0.000000, 0.000000}
\pgfsetstrokecolor{dialinecolor}
\pgfsetstrokeopacity{1.000000}
\pgfpathmoveto{\pgfpoint{23.806850\du}{-2.900000\du}}
\pgfpathlineto{\pgfpoint{36.731850\du}{-2.900000\du}}
\pgfpathcurveto{\pgfpoint{38.516421\du}{-2.900000\du}}{\pgfpoint{39.963100\du}{-2.620178\du}}{\pgfpoint{39.963100\du}{-2.275000\du}}
\pgfpathcurveto{\pgfpoint{39.963100\du}{-1.929822\du}}{\pgfpoint{38.516421\du}{-1.650000\du}}{\pgfpoint{36.731850\du}{-1.650000\du}}
\pgfpathlineto{\pgfpoint{23.806850\du}{-1.650000\du}}
\pgfpathcurveto{\pgfpoint{22.022279\du}{-1.650000\du}}{\pgfpoint{20.575600\du}{-1.929822\du}}{\pgfpoint{20.575600\du}{-2.275000\du}}
\pgfpathcurveto{\pgfpoint{20.575600\du}{-2.620178\du}}{\pgfpoint{22.022279\du}{-2.900000\du}}{\pgfpoint{23.806850\du}{-2.900000\du}}
\pgfpathclose
\pgfusepath{fill,stroke}
% setfont left to latex
\definecolor{dialinecolor}{rgb}{0.000000, 0.000000, 0.000000}
\pgfsetstrokecolor{dialinecolor}
\pgfsetstrokeopacity{1.000000}
\definecolor{diafillcolor}{rgb}{0.000000, 0.000000, 0.000000}
\pgfsetfillcolor{diafillcolor}
\pgfsetfillopacity{1.000000}
\node[anchor=base,inner sep=0pt, outer sep=0pt,color=dialinecolor] at (30.269350\du,-2.075000\du){1. Trier les traceroutes par timeWindow};
% setfont left to latex
\definecolor{dialinecolor}{rgb}{0.000000, 0.000000, 0.000000}
\pgfsetstrokecolor{dialinecolor}
\pgfsetstrokeopacity{1.000000}
\definecolor{diafillcolor}{rgb}{0.000000, 0.000000, 0.000000}
\pgfsetfillcolor{diafillcolor}
\pgfsetfillopacity{1.000000}
\node[anchor=base west,inner sep=0pt,outer sep=0pt,color=dialinecolor] at (21.150000\du,0.700000\du){ Pour chaque instance du TraceroutesPerPeriod appliquer 2. à 6.};
\pgfsetlinewidth{0.100000\du}
\pgfsetdash{}{0pt}
\pgfsetmiterjoin
{\pgfsetcornersarced{\pgfpoint{0.000000\du}{0.000000\du}}\definecolor{diafillcolor}{rgb}{1.000000, 1.000000, 1.000000}
\pgfsetfillcolor{diafillcolor}
\pgfsetfillopacity{1.000000}
\fill (11.031200\du,4.142720\du)--(11.031200\du,6.042720\du)--(19.065613\du,6.042720\du)--(19.065613\du,4.142720\du)--cycle;
}{\pgfsetcornersarced{\pgfpoint{0.000000\du}{0.000000\du}}\definecolor{dialinecolor}{rgb}{0.000000, 0.000000, 0.000000}
\pgfsetstrokecolor{dialinecolor}
\pgfsetstrokeopacity{1.000000}
\draw (11.031200\du,4.142720\du)--(11.031200\du,6.042720\du)--(19.065613\du,6.042720\du)--(19.065613\du,4.142720\du)--cycle;
}% setfont left to latex
\definecolor{dialinecolor}{rgb}{0.000000, 0.000000, 0.000000}
\pgfsetstrokecolor{dialinecolor}
\pgfsetstrokeopacity{1.000000}
\definecolor{diafillcolor}{rgb}{0.000000, 0.000000, 0.000000}
\pgfsetfillcolor{diafillcolor}
\pgfsetfillopacity{1.000000}
\node[anchor=base,inner sep=0pt, outer sep=0pt,color=dialinecolor] at (15.048406\du,5.287720\du){TraceroutesPerPeriod};
\pgfsetlinewidth{0.100000\du}
\pgfsetdash{}{0pt}
\pgfsetmiterjoin
{\pgfsetcornersarced{\pgfpoint{0.000000\du}{0.000000\du}}\definecolor{diafillcolor}{rgb}{1.000000, 1.000000, 1.000000}
\pgfsetfillcolor{diafillcolor}
\pgfsetfillopacity{1.000000}
\fill (12.053100\du,26.046500\du)--(12.053100\du,27.946500\du)--(18.085600\du,27.946500\du)--(18.085600\du,26.046500\du)--cycle;
}{\pgfsetcornersarced{\pgfpoint{0.000000\du}{0.000000\du}}\definecolor{dialinecolor}{rgb}{0.000000, 0.000000, 0.000000}
\pgfsetstrokecolor{dialinecolor}
\pgfsetstrokeopacity{1.000000}
\draw (12.053100\du,26.046500\du)--(12.053100\du,27.946500\du)--(18.085600\du,27.946500\du)--(18.085600\du,26.046500\du)--cycle;
}% setfont left to latex
\definecolor{dialinecolor}{rgb}{0.000000, 0.000000, 0.000000}
\pgfsetstrokecolor{dialinecolor}
\pgfsetstrokeopacity{1.000000}
\definecolor{diafillcolor}{rgb}{0.000000, 0.000000, 0.000000}
\pgfsetfillcolor{diafillcolor}
\pgfsetfillopacity{1.000000}
\node[anchor=base,inner sep=0pt, outer sep=0pt,color=dialinecolor] at (15.069350\du,27.191500\du){\ensuremath{[}DiffRTT\ensuremath{]}};
\pgfsetlinewidth{0.100000\du}
\pgfsetdash{}{0pt}
\pgfsetbuttcap
\pgfsetmiterjoin
\pgfsetlinewidth{0.100000\du}
\pgfsetbuttcap
\pgfsetmiterjoin
\pgfsetdash{}{0pt}
\definecolor{diafillcolor}{rgb}{1.000000, 1.000000, 1.000000}
\pgfsetfillcolor{diafillcolor}
\pgfsetfillopacity{1.000000}
\definecolor{dialinecolor}{rgb}{0.000000, 0.000000, 0.000000}
\pgfsetstrokecolor{dialinecolor}
\pgfsetstrokeopacity{1.000000}
\pgfpathmoveto{\pgfpoint{21.153600\du}{24.632500\du}}
\pgfpathlineto{\pgfpoint{35.633600\du}{24.632500\du}}
\pgfpathcurveto{\pgfpoint{37.632872\du}{24.632500\du}}{\pgfpoint{39.253600\du}{24.878743\du}}{\pgfpoint{39.253600\du}{25.182500\du}}
\pgfpathcurveto{\pgfpoint{39.253600\du}{25.486257\du}}{\pgfpoint{37.632872\du}{25.732500\du}}{\pgfpoint{35.633600\du}{25.732500\du}}
\pgfpathlineto{\pgfpoint{21.153600\du}{25.732500\du}}
\pgfpathcurveto{\pgfpoint{19.154328\du}{25.732500\du}}{\pgfpoint{17.533600\du}{25.486257\du}}{\pgfpoint{17.533600\du}{25.182500\du}}
\pgfpathcurveto{\pgfpoint{17.533600\du}{24.878743\du}}{\pgfpoint{19.154328\du}{24.632500\du}}{\pgfpoint{21.153600\du}{24.632500\du}}
\pgfpathclose
\pgfusepath{fill,stroke}
% setfont left to latex
\definecolor{dialinecolor}{rgb}{0.000000, 0.000000, 0.000000}
\pgfsetstrokecolor{dialinecolor}
\pgfsetstrokeopacity{1.000000}
\definecolor{diafillcolor}{rgb}{0.000000, 0.000000, 0.000000}
\pgfsetfillcolor{diafillcolor}
\pgfsetfillopacity{1.000000}
\node[anchor=base,inner sep=0pt, outer sep=0pt,color=dialinecolor] at (28.393600\du,25.382500\du){6.1. ordonner les adresses IP de chaque lien};
\pgfsetlinewidth{0.100000\du}
\pgfsetdash{}{0pt}
\pgfsetmiterjoin
{\pgfsetcornersarced{\pgfpoint{0.000000\du}{0.000000\du}}\definecolor{diafillcolor}{rgb}{1.000000, 1.000000, 1.000000}
\pgfsetfillcolor{diafillcolor}
\pgfsetfillopacity{1.000000}
\fill (11.729100\du,30.612100\du)--(11.729100\du,32.512100\du)--(18.496600\du,32.512100\du)--(18.496600\du,30.612100\du)--cycle;
}{\pgfsetcornersarced{\pgfpoint{0.000000\du}{0.000000\du}}\definecolor{dialinecolor}{rgb}{0.000000, 0.000000, 0.000000}
\pgfsetstrokecolor{dialinecolor}
\pgfsetstrokeopacity{1.000000}
\draw (11.729100\du,30.612100\du)--(11.729100\du,32.512100\du)--(18.496600\du,32.512100\du)--(18.496600\du,30.612100\du)--cycle;
}% setfont left to latex
\definecolor{dialinecolor}{rgb}{0.000000, 0.000000, 0.000000}
\pgfsetstrokecolor{dialinecolor}
\pgfsetstrokeopacity{1.000000}
\definecolor{diafillcolor}{rgb}{0.000000, 0.000000, 0.000000}
\pgfsetfillcolor{diafillcolor}
\pgfsetfillopacity{1.000000}
\node[anchor=base,inner sep=0pt, outer sep=0pt,color=dialinecolor] at (15.112850\du,31.757100\du){(LinkIPs, \ensuremath{[}DiffRtt\ensuremath{]})};
\pgfsetlinewidth{0.100000\du}
\pgfsetdash{}{0pt}
\pgfsetbuttcap
\pgfsetmiterjoin
\pgfsetlinewidth{0.100000\du}
\pgfsetbuttcap
\pgfsetmiterjoin
\pgfsetdash{}{0pt}
\definecolor{diafillcolor}{rgb}{1.000000, 1.000000, 1.000000}
\pgfsetfillcolor{diafillcolor}
\pgfsetfillopacity{1.000000}
\definecolor{dialinecolor}{rgb}{0.000000, 0.000000, 0.000000}
\pgfsetstrokecolor{dialinecolor}
\pgfsetstrokeopacity{1.000000}
\pgfpathmoveto{\pgfpoint{19.696075\du}{33.034600\du}}
\pgfpathlineto{\pgfpoint{31.713575\du}{33.034600\du}}
\pgfpathcurveto{\pgfpoint{33.372846\du}{33.034600\du}}{\pgfpoint{34.717950\du}{33.280843\du}}{\pgfpoint{34.717950\du}{33.584600\du}}
\pgfpathcurveto{\pgfpoint{34.717950\du}{33.888357\du}}{\pgfpoint{33.372846\du}{34.134600\du}}{\pgfpoint{31.713575\du}{34.134600\du}}
\pgfpathlineto{\pgfpoint{19.696075\du}{34.134600\du}}
\pgfpathcurveto{\pgfpoint{18.036804\du}{34.134600\du}}{\pgfpoint{16.691700\du}{33.888357\du}}{\pgfpoint{16.691700\du}{33.584600\du}}
\pgfpathcurveto{\pgfpoint{16.691700\du}{33.280843\du}}{\pgfpoint{18.036804\du}{33.034600\du}}{\pgfpoint{19.696075\du}{33.034600\du}}
\pgfpathclose
\pgfusepath{fill,stroke}
% setfont left to latex
\definecolor{dialinecolor}{rgb}{0.000000, 0.000000, 0.000000}
\pgfsetstrokecolor{dialinecolor}
\pgfsetstrokeopacity{1.000000}
\definecolor{diafillcolor}{rgb}{0.000000, 0.000000, 0.000000}
\pgfsetfillcolor{diafillcolor}
\pgfsetfillopacity{1.000000}
\node[anchor=base,inner sep=0pt, outer sep=0pt,color=dialinecolor] at (25.704825\du,33.784600\du){6.3. organisation les dates des liens };
\pgfsetlinewidth{0.100000\du}
\pgfsetdash{}{0pt}
\pgfsetbuttcap
{
\definecolor{diafillcolor}{rgb}{0.000000, 0.000000, 0.000000}
\pgfsetfillcolor{diafillcolor}
\pgfsetfillopacity{1.000000}
% was here!!!
\pgfsetarrowsend{stealth}
\definecolor{dialinecolor}{rgb}{0.000000, 0.000000, 0.000000}
\pgfsetstrokecolor{dialinecolor}
\pgfsetstrokeopacity{1.000000}
\draw (15.069300\du,27.946500\du)--(15.112800\du,30.612100\du);
}
\pgfsetlinewidth{0.100000\du}
\pgfsetdash{}{0pt}
\pgfsetmiterjoin
{\pgfsetcornersarced{\pgfpoint{0.000000\du}{0.000000\du}}\definecolor{diafillcolor}{rgb}{1.000000, 1.000000, 1.000000}
\pgfsetfillcolor{diafillcolor}
\pgfsetfillopacity{1.000000}
\fill (12.209300\du,35.321700\du)--(12.209300\du,37.221700\du)--(18.084300\du,37.221700\du)--(18.084300\du,35.321700\du)--cycle;
}{\pgfsetcornersarced{\pgfpoint{0.000000\du}{0.000000\du}}\definecolor{dialinecolor}{rgb}{0.000000, 0.000000, 0.000000}
\pgfsetstrokecolor{dialinecolor}
\pgfsetstrokeopacity{1.000000}
\draw (12.209300\du,35.321700\du)--(12.209300\du,37.221700\du)--(18.084300\du,37.221700\du)--(18.084300\du,35.321700\du)--cycle;
}% setfont left to latex
\definecolor{dialinecolor}{rgb}{0.000000, 0.000000, 0.000000}
\pgfsetstrokecolor{dialinecolor}
\pgfsetstrokeopacity{1.000000}
\definecolor{diafillcolor}{rgb}{0.000000, 0.000000, 0.000000}
\pgfsetfillcolor{diafillcolor}
\pgfsetfillopacity{1.000000}
\node[anchor=base,inner sep=0pt, outer sep=0pt,color=dialinecolor] at (15.146800\du,36.466700\du){\ensuremath{[}DiffRTTPeriod\ensuremath{]}};
\pgfsetlinewidth{0.100000\du}
\pgfsetdash{}{0pt}
\pgfsetbuttcap
{
\definecolor{diafillcolor}{rgb}{0.000000, 0.000000, 0.000000}
\pgfsetfillcolor{diafillcolor}
\pgfsetfillopacity{1.000000}
% was here!!!
\pgfsetarrowsend{stealth}
\definecolor{dialinecolor}{rgb}{0.000000, 0.000000, 0.000000}
\pgfsetstrokecolor{dialinecolor}
\pgfsetstrokeopacity{1.000000}
\draw (15.112800\du,32.512100\du)--(15.146800\du,35.321700\du);
}
\pgfsetlinewidth{0.100000\du}
\pgfsetdash{}{0pt}
\pgfsetbuttcap
\pgfsetmiterjoin
\pgfsetlinewidth{0.100000\du}
\pgfsetbuttcap
\pgfsetmiterjoin
\pgfsetdash{}{0pt}
\definecolor{diafillcolor}{rgb}{1.000000, 1.000000, 1.000000}
\pgfsetfillcolor{diafillcolor}
\pgfsetfillopacity{1.000000}
\definecolor{dialinecolor}{rgb}{0.000000, 0.000000, 0.000000}
\pgfsetstrokecolor{dialinecolor}
\pgfsetstrokeopacity{1.000000}
\pgfpathmoveto{\pgfpoint{20.708450\du}{28.664600\du}}
\pgfpathlineto{\pgfpoint{33.403450\du}{28.664600\du}}
\pgfpathcurveto{\pgfpoint{35.156265\du}{28.664600\du}}{\pgfpoint{36.577200\du}{28.910843\du}}{\pgfpoint{36.577200\du}{29.214600\du}}
\pgfpathcurveto{\pgfpoint{36.577200\du}{29.518357\du}}{\pgfpoint{35.156265\du}{29.764600\du}}{\pgfpoint{33.403450\du}{29.764600\du}}
\pgfpathlineto{\pgfpoint{20.708450\du}{29.764600\du}}
\pgfpathcurveto{\pgfpoint{18.955635\du}{29.764600\du}}{\pgfpoint{17.534700\du}{29.518357\du}}{\pgfpoint{17.534700\du}{29.214600\du}}
\pgfpathcurveto{\pgfpoint{17.534700\du}{28.910843\du}}{\pgfpoint{18.955635\du}{28.664600\du}}{\pgfpoint{20.708450\du}{28.664600\du}}
\pgfpathclose
\pgfusepath{fill,stroke}
% setfont left to latex
\definecolor{dialinecolor}{rgb}{0.000000, 0.000000, 0.000000}
\pgfsetstrokecolor{dialinecolor}
\pgfsetstrokeopacity{1.000000}
\definecolor{diafillcolor}{rgb}{0.000000, 0.000000, 0.000000}
\pgfsetfillcolor{diafillcolor}
\pgfsetfillopacity{1.000000}
\node[anchor=base,inner sep=0pt, outer sep=0pt,color=dialinecolor] at (27.055950\du,29.414600\du){6.2. aggregation des données par lien };
\pgfsetlinewidth{0.100000\du}
\pgfsetdash{}{0pt}
\pgfsetmiterjoin
{\pgfsetcornersarced{\pgfpoint{0.000000\du}{0.000000\du}}\definecolor{diafillcolor}{rgb}{1.000000, 1.000000, 1.000000}
\pgfsetfillcolor{diafillcolor}
\pgfsetfillopacity{1.000000}
\fill (11.630600\du,39.879700\du)--(11.630600\du,41.779700\du)--(18.663100\du,41.779700\du)--(18.663100\du,39.879700\du)--cycle;
}{\pgfsetcornersarced{\pgfpoint{0.000000\du}{0.000000\du}}\definecolor{dialinecolor}{rgb}{0.000000, 0.000000, 0.000000}
\pgfsetstrokecolor{dialinecolor}
\pgfsetstrokeopacity{1.000000}
\draw (11.630600\du,39.879700\du)--(11.630600\du,41.779700\du)--(18.663100\du,41.779700\du)--(18.663100\du,39.879700\du)--cycle;
}% setfont left to latex
\definecolor{dialinecolor}{rgb}{0.000000, 0.000000, 0.000000}
\pgfsetstrokecolor{dialinecolor}
\pgfsetstrokeopacity{1.000000}
\definecolor{diafillcolor}{rgb}{0.000000, 0.000000, 0.000000}
\pgfsetfillcolor{diafillcolor}
\pgfsetfillopacity{1.000000}
\node[anchor=base,inner sep=0pt, outer sep=0pt,color=dialinecolor] at (15.146850\du,41.024700\du){current : LinkState};
\pgfsetlinewidth{0.100000\du}
\pgfsetdash{}{0pt}
\pgfsetmiterjoin
{\pgfsetcornersarced{\pgfpoint{0.000000\du}{0.000000\du}}\definecolor{diafillcolor}{rgb}{1.000000, 1.000000, 1.000000}
\pgfsetfillcolor{diafillcolor}
\pgfsetfillopacity{1.000000}
\fill (11.275000\du,44.391900\du)--(11.275000\du,46.291900\du)--(19.057500\du,46.291900\du)--(19.057500\du,44.391900\du)--cycle;
}{\pgfsetcornersarced{\pgfpoint{0.000000\du}{0.000000\du}}\definecolor{dialinecolor}{rgb}{0.000000, 0.000000, 0.000000}
\pgfsetstrokecolor{dialinecolor}
\pgfsetstrokeopacity{1.000000}
\draw (11.275000\du,44.391900\du)--(11.275000\du,46.291900\du)--(19.057500\du,46.291900\du)--(19.057500\du,44.391900\du)--cycle;
}% setfont left to latex
\definecolor{dialinecolor}{rgb}{0.000000, 0.000000, 0.000000}
\pgfsetstrokecolor{dialinecolor}
\pgfsetstrokeopacity{1.000000}
\definecolor{diafillcolor}{rgb}{0.000000, 0.000000, 0.000000}
\pgfsetfillcolor{diafillcolor}
\pgfsetfillopacity{1.000000}
\node[anchor=base,inner sep=0pt, outer sep=0pt,color=dialinecolor] at (15.166250\du,45.536900\du){reference : LinkState};
\pgfsetlinewidth{0.100000\du}
\pgfsetdash{}{0pt}
\pgfsetbuttcap
\pgfsetmiterjoin
\pgfsetlinewidth{0.100000\du}
\pgfsetbuttcap
\pgfsetmiterjoin
\pgfsetdash{}{0pt}
\definecolor{diafillcolor}{rgb}{1.000000, 1.000000, 1.000000}
\pgfsetfillcolor{diafillcolor}
\pgfsetfillopacity{1.000000}
\definecolor{dialinecolor}{rgb}{0.000000, 0.000000, 0.000000}
\pgfsetstrokecolor{dialinecolor}
\pgfsetstrokeopacity{1.000000}
\pgfpathmoveto{\pgfpoint{22.357225\du}{38.015300\du}}
\pgfpathlineto{\pgfpoint{40.139725\du}{38.015300\du}}
\pgfpathcurveto{\pgfpoint{42.594977\du}{38.015300\du}}{\pgfpoint{44.585350\du}{38.261543\du}}{\pgfpoint{44.585350\du}{38.565300\du}}
\pgfpathcurveto{\pgfpoint{44.585350\du}{38.869057\du}}{\pgfpoint{42.594977\du}{39.115300\du}}{\pgfpoint{40.139725\du}{39.115300\du}}
\pgfpathlineto{\pgfpoint{22.357225\du}{39.115300\du}}
\pgfpathcurveto{\pgfpoint{19.901973\du}{39.115300\du}}{\pgfpoint{17.911600\du}{38.869057\du}}{\pgfpoint{17.911600\du}{38.565300\du}}
\pgfpathcurveto{\pgfpoint{17.911600\du}{38.261543\du}}{\pgfpoint{19.901973\du}{38.015300\du}}{\pgfpoint{22.357225\du}{38.015300\du}}
\pgfpathclose
\pgfusepath{fill,stroke}
% setfont left to latex
\definecolor{dialinecolor}{rgb}{0.000000, 0.000000, 0.000000}
\pgfsetstrokecolor{dialinecolor}
\pgfsetstrokeopacity{1.000000}
\definecolor{diafillcolor}{rgb}{0.000000, 0.000000, 0.000000}
\pgfsetfillcolor{diafillcolor}
\pgfsetfillopacity{1.000000}
\node[anchor=base,inner sep=0pt, outer sep=0pt,color=dialinecolor] at (31.248475\du,38.765300\du){7. pour chaque période calculer l'état courant du lien};
\pgfsetlinewidth{0.100000\du}
\pgfsetdash{}{0pt}
\pgfsetbuttcap
{
\definecolor{diafillcolor}{rgb}{0.000000, 0.000000, 0.000000}
\pgfsetfillcolor{diafillcolor}
\pgfsetfillopacity{1.000000}
% was here!!!
\pgfsetarrowsend{stealth}
\definecolor{dialinecolor}{rgb}{0.000000, 0.000000, 0.000000}
\pgfsetstrokecolor{dialinecolor}
\pgfsetstrokeopacity{1.000000}
\draw (15.146800\du,37.221700\du)--(15.146900\du,39.879700\du);
}
\pgfsetlinewidth{0.100000\du}
\pgfsetdash{}{0pt}
\pgfsetbuttcap
{
\definecolor{diafillcolor}{rgb}{0.000000, 0.000000, 0.000000}
\pgfsetfillcolor{diafillcolor}
\pgfsetfillopacity{1.000000}
% was here!!!
\pgfsetarrowsend{stealth}
\definecolor{dialinecolor}{rgb}{0.000000, 0.000000, 0.000000}
\pgfsetstrokecolor{dialinecolor}
\pgfsetstrokeopacity{1.000000}
\draw (15.146900\du,41.779700\du)--(15.166200\du,44.391900\du);
}
\pgfsetlinewidth{0.100000\du}
\pgfsetdash{}{0pt}
\pgfsetbuttcap
\pgfsetmiterjoin
\pgfsetlinewidth{0.100000\du}
\pgfsetbuttcap
\pgfsetmiterjoin
\pgfsetdash{}{0pt}
\definecolor{diafillcolor}{rgb}{1.000000, 1.000000, 1.000000}
\pgfsetfillcolor{diafillcolor}
\pgfsetfillopacity{1.000000}
\definecolor{dialinecolor}{rgb}{0.000000, 0.000000, 0.000000}
\pgfsetstrokecolor{dialinecolor}
\pgfsetstrokeopacity{1.000000}
\pgfpathmoveto{\pgfpoint{23.185050\du}{42.538700\du}}
\pgfpathlineto{\pgfpoint{43.610050\du}{42.538700\du}}
\pgfpathcurveto{\pgfpoint{46.430155\du}{42.538700\du}}{\pgfpoint{48.716300\du}{42.784943\du}}{\pgfpoint{48.716300\du}{43.088700\du}}
\pgfpathcurveto{\pgfpoint{48.716300\du}{43.392457\du}}{\pgfpoint{46.430155\du}{43.638700\du}}{\pgfpoint{43.610050\du}{43.638700\du}}
\pgfpathlineto{\pgfpoint{23.185050\du}{43.638700\du}}
\pgfpathcurveto{\pgfpoint{20.364945\du}{43.638700\du}}{\pgfpoint{18.078800\du}{43.392457\du}}{\pgfpoint{18.078800\du}{43.088700\du}}
\pgfpathcurveto{\pgfpoint{18.078800\du}{42.784943\du}}{\pgfpoint{20.364945\du}{42.538700\du}}{\pgfpoint{23.185050\du}{42.538700\du}}
\pgfpathclose
\pgfusepath{fill,stroke}
% setfont left to latex
\definecolor{dialinecolor}{rgb}{0.000000, 0.000000, 0.000000}
\pgfsetstrokecolor{dialinecolor}
\pgfsetstrokeopacity{1.000000}
\definecolor{diafillcolor}{rgb}{0.000000, 0.000000, 0.000000}
\pgfsetfillcolor{diafillcolor}
\pgfsetfillopacity{1.000000}
\node[anchor=base,inner sep=0pt, outer sep=0pt,color=dialinecolor] at (33.397550\du,43.288700\du){8. comparaison des états du lien et mise à jour de la référence};
\pgfsetlinewidth{0.100000\du}
\pgfsetdash{}{0pt}
\pgfsetmiterjoin
{\pgfsetcornersarced{\pgfpoint{0.000000\du}{0.000000\du}}\definecolor{diafillcolor}{rgb}{1.000000, 1.000000, 1.000000}
\pgfsetfillcolor{diafillcolor}
\pgfsetfillopacity{1.000000}
\fill (13.357800\du,47.900000\du)--(13.357800\du,49.800000\du)--(17.127800\du,49.800000\du)--(17.127800\du,47.900000\du)--cycle;
}{\pgfsetcornersarced{\pgfpoint{0.000000\du}{0.000000\du}}\definecolor{dialinecolor}{rgb}{0.000000, 0.000000, 0.000000}
\pgfsetstrokecolor{dialinecolor}
\pgfsetstrokeopacity{1.000000}
\draw (13.357800\du,47.900000\du)--(13.357800\du,49.800000\du)--(17.127800\du,49.800000\du)--(17.127800\du,47.900000\du)--cycle;
}% setfont left to latex
\definecolor{dialinecolor}{rgb}{0.000000, 0.000000, 0.000000}
\pgfsetstrokecolor{dialinecolor}
\pgfsetstrokeopacity{1.000000}
\definecolor{diafillcolor}{rgb}{0.000000, 0.000000, 0.000000}
\pgfsetfillcolor{diafillcolor}
\pgfsetfillopacity{1.000000}
\node[anchor=base,inner sep=0pt, outer sep=0pt,color=dialinecolor] at (15.242800\du,49.045000\du){\ensuremath{[}alarm\ensuremath{]}};
\pgfsetlinewidth{0.100000\du}
\pgfsetdash{}{0pt}
\pgfsetbuttcap
{
\definecolor{diafillcolor}{rgb}{0.000000, 0.000000, 0.000000}
\pgfsetfillcolor{diafillcolor}
\pgfsetfillopacity{1.000000}
% was here!!!
\pgfsetarrowsend{stealth}
\definecolor{dialinecolor}{rgb}{0.000000, 0.000000, 0.000000}
\pgfsetstrokecolor{dialinecolor}
\pgfsetstrokeopacity{1.000000}
\draw (15.196171\du,46.341782\du)--(15.242800\du,47.900000\du);
}
\pgfsetlinewidth{0.100000\du}
\pgfsetdash{}{0pt}
\pgfsetbuttcap
\pgfsetmiterjoin
\pgfsetlinewidth{0.100000\du}
\pgfsetbuttcap
\pgfsetmiterjoin
\pgfsetdash{}{0pt}
\definecolor{diafillcolor}{rgb}{1.000000, 1.000000, 1.000000}
\pgfsetfillcolor{diafillcolor}
\pgfsetfillopacity{1.000000}
\definecolor{dialinecolor}{rgb}{0.000000, 0.000000, 0.000000}
\pgfsetstrokecolor{dialinecolor}
\pgfsetstrokeopacity{1.000000}
\pgfpathmoveto{\pgfpoint{20.591200\du}{46.464400\du}}
\pgfpathlineto{\pgfpoint{29.241200\du}{46.464400\du}}
\pgfpathcurveto{\pgfpoint{30.435516\du}{46.464400\du}}{\pgfpoint{31.403700\du}{46.710643\du}}{\pgfpoint{31.403700\du}{47.014400\du}}
\pgfpathcurveto{\pgfpoint{31.403700\du}{47.318157\du}}{\pgfpoint{30.435516\du}{47.564400\du}}{\pgfpoint{29.241200\du}{47.564400\du}}
\pgfpathlineto{\pgfpoint{20.591200\du}{47.564400\du}}
\pgfpathcurveto{\pgfpoint{19.396884\du}{47.564400\du}}{\pgfpoint{18.428700\du}{47.318157\du}}{\pgfpoint{18.428700\du}{47.014400\du}}
\pgfpathcurveto{\pgfpoint{18.428700\du}{46.710643\du}}{\pgfpoint{19.396884\du}{46.464400\du}}{\pgfpoint{20.591200\du}{46.464400\du}}
\pgfpathclose
\pgfusepath{fill,stroke}
% setfont left to latex
\definecolor{dialinecolor}{rgb}{0.000000, 0.000000, 0.000000}
\pgfsetstrokecolor{dialinecolor}
\pgfsetstrokeopacity{1.000000}
\definecolor{diafillcolor}{rgb}{0.000000, 0.000000, 0.000000}
\pgfsetfillcolor{diafillcolor}
\pgfsetfillopacity{1.000000}
\node[anchor=base,inner sep=0pt, outer sep=0pt,color=dialinecolor] at (24.916200\du,47.214400\du){9. détection des alarmes};
% setfont left to latex
\definecolor{dialinecolor}{rgb}{0.000000, 0.000000, 0.000000}
\pgfsetstrokecolor{dialinecolor}
\pgfsetstrokeopacity{1.000000}
\definecolor{diafillcolor}{rgb}{0.000000, 0.000000, 0.000000}
\pgfsetfillcolor{diafillcolor}
\pgfsetfillopacity{1.000000}
\node[anchor=base west,inner sep=0pt,outer sep=0pt,color=dialinecolor] at (19.550000\du,2.533740\du){Chargement des traceroutes de la période concernée};
\pgfsetlinewidth{0.100000\du}
\pgfsetdash{}{0pt}
\pgfsetbuttcap
{
\definecolor{diafillcolor}{rgb}{0.000000, 0.000000, 0.000000}
\pgfsetfillcolor{diafillcolor}
\pgfsetfillopacity{1.000000}
% was here!!!
\pgfsetarrowsend{to}
\definecolor{dialinecolor}{rgb}{0.000000, 0.000000, 0.000000}
\pgfsetstrokecolor{dialinecolor}
\pgfsetstrokeopacity{1.000000}
\pgfpathmoveto{\pgfpoint{19.218821\du}{0.775034\du}}
\pgfpatharc{508}{212}{0.914891\du and 0.914891\du}
\pgfusepath{stroke}
}
% setfont left to latex
\definecolor{dialinecolor}{rgb}{0.000000, 0.000000, 0.000000}
\pgfsetstrokecolor{dialinecolor}
\pgfsetstrokeopacity{1.000000}
\definecolor{diafillcolor}{rgb}{0.000000, 0.000000, 0.000000}
\pgfsetfillcolor{diafillcolor}
\pgfsetfillopacity{1.000000}
\node[anchor=base west,inner sep=0pt,outer sep=0pt,color=dialinecolor] at (20.365600\du,4.846460\du){};
% setfont left to latex
\definecolor{dialinecolor}{rgb}{0.000000, 0.000000, 0.000000}
\pgfsetstrokecolor{dialinecolor}
\pgfsetstrokeopacity{1.000000}
\definecolor{diafillcolor}{rgb}{0.000000, 0.000000, 0.000000}
\pgfsetfillcolor{diafillcolor}
\pgfsetfillopacity{1.000000}
\node[anchor=base west,inner sep=0pt,outer sep=0pt,color=dialinecolor] at (23.265600\du,4.646460\du){Illustration des opérations sur une instance de TraceroutesPerPeriod };
% setfont left to latex
\definecolor{dialinecolor}{rgb}{0.000000, 0.000000, 0.000000}
\pgfsetstrokecolor{dialinecolor}
\pgfsetstrokeopacity{1.000000}
\definecolor{diafillcolor}{rgb}{0.000000, 0.000000, 0.000000}
\pgfsetfillcolor{diafillcolor}
\pgfsetfillopacity{1.000000}
\node[anchor=base west,inner sep=0pt,outer sep=0pt,color=dialinecolor] at (9.738010\du,5.397040\du){=};
\end{tikzpicture}

	}
	\caption{Le processus de la détection des anomalies dans les délais des liens}
	\label{fig:process-rttanalysis_tex}
\end{figure}


\subsection{Complément d'information du processus de la détection avec le langage Scala}
Comme complément à aux étapes décrites dans \ref{steps-rtt-analysis}, on présente les différentes classes permettant de modéliser les données tout au long du processus de l'analyse. La définition de ces classes est liée au langage \textit{Scala}. 

Soient les classes suivantes utilisées : 

\paragraph{La classe Signal} modélise un signal \footnote{Un signal dans le contexte d'un traceroute.}. Ainsi, \textit{from} est l'adresse IP du routeur émettant ce signal, \textit{rtt} est le Round Trip Time entre la sonde Atlas et ce routeur et enfin \textit{x} est un indicateur de l'échec du signal.
\begin{lstlisting}[language=scala]
case class Signal(
	rtt:  Option[Double],
	x:    Option[String],
	from: Option[String])
\end{lstlisting}

\paragraph{La classe Hop} modélise un saut dans un traceroute. On caractérise un saut par son identifiant noté \textit{hop} dont il prend comme valeur un entier à partir de la valeur $1$ et la liste des signaux relatifs à ce saut notée par \textit{result}, généralement un saut est représenté par $3$ signaux.
\begin{lstlisting}[language=scala]
case class Hop(
	var result: Seq[Signal],
	hop:        Int)
\end{lstlisting}
\paragraph{La classe Traceroutes} modélise le résultat d'une requête traceroute effectué par une sonde Atlas. Cette modélisation se limite aux données qui nous intéressent dans la présente analyse. 

\textit{dst\_name} représente l'adresse IP de la destination de la requête traceroute, \textit{from} est l'adresse IP de la sonde Atlas ayant déclenché la requête traceroute, \textit{prb\_id} est l'identifiant de la sonde Atlas ayant déclenché la requête traceroute, \textit{msm\_id} est l'identifiant de mesure Atlas dans le cadre duquel la requête traceroute a été déclenchée, \textit{timestamp} est le temps pendant lequel la requête traceroute a été effectuée et enfin on trouve la liste des sauts qui représentent les routeurs traversés par le trafic entre la source et la destination. 

 \begin{lstlisting}[language=scala]
case class Traceroute(
	dst_name:  String,
	from:      String,
	prb_id:    BigInt,
	msm_id:    BigInt,
	timestamp: BigInt,
	result:    Seq[Hop])
 \end{lstlisting}
\paragraph{La classe TraceroutesPerPeriod} permet de présenter les traceroutes après les avoir trié   suivant la période pendant laquelle ils ont été effectués.  Avec \textit{timeWindow} est le temps unix marquant le début de la période \footnote{Pour précision, la fin de la période peut être inférée en prenant deux débuts de deux périodes car la durée d'une période est fixe tout au long de l'analyse.} et  \textit{traceroutes} est la liste des traceroutes effectués pendant cette période. 


A l'étape 2, l'objectif était d'agréger  les signaux par routeur source et ensuite calculer la médiane des RTTs par ce routeur. Par conséquent, un traceroute est présenté différemment, ce qui est  illustré par la classe \textit{MedianByHopTraceroute}.

\paragraph{La classe ProceedSignal }  est une agrégation de tous les signaux, d'un saut donné, par le routeur \textit{from},  la médiane des RTTs calculée est présentée par \textit{medianRtt}.
\begin{lstlisting}[language=scala]
case class ProceedSignal(
	medianRtt: Double,
	from:      String)
\end{lstlisting}
\paragraph{La classe ProceedHop } modélise un saut après avoir agrégé ses signaux. 
\begin{lstlisting}[language=scala]
case class ProceedHop(
	var result: Seq[ProceedSignal],
	hop:        Int)
\end{lstlisting}


\paragraph{La classe MedianByHopTraceroute } modélise un traceroute après avoir agrégé ses sauts. Par rapport au traceroute d'avant l'agrégation, seulement la liste des sauts qui a subi un changement. 
\begin{lstlisting}[language=scala]
case class MedianByHopTraceroute(
	dst_name:  String,
	from:      String,
	prb_id:    BigInt,
	msm_id:    BigInt,
	timestamp: BigInt,
	result:    Seq[ProceedHop])
\end{lstlisting}


\paragraph{La classe Link} modélise un lien topologique, ce dernier est définit par deux adresses IP  \textit{ip1} et \textit{ip2} et par son RTT différentiel calculé \textit{rttDiff}.
\begin{lstlisting}[language=scala]
case class Link(
	ip1:     String,
	ip2:     String,
	rttDiff: Double)
\end{lstlisting}

\paragraph{La classe LinksTraceroute} permet de modéliser un traceroute après avoir inféré tous les liens de ce dernier. Ainsi, la liste des sauts est remplacée par la liste des liens (\textit{links}). 

\begin{lstlisting}[language=scala]
case class LinksTraceroute(
	dst_name:  String,
	from:      String,
	prb_id:    BigInt,
	msm_id:    BigInt,
	timestamp: BigInt,
	links:     Seq[Link])
\end{lstlisting}


A l'étape 5, l'objectif était de passer d'un traceroute à une liste de liens caractérisés par les informations générales sur la sonde Atlas, la mesure Atlas,etc. Chaque élément de cette liste est présenté par la classe \textit{DiffRtt}, où \textit{LinkIPs} représente les deux adresses IP d'un lien donné.
\paragraph{La classe LinkIPs} permet présenter un lien par seulement ses deux adresses IP \textit{ip1} et \textit{ip2}.
\begin{lstlisting}[language=scala]
case class LinkIPs(
	ip1: String,
	ip2: String)
\end{lstlisting}

\paragraph{La classe DiffRtt} est une représentation plus détaillée d'un lien, en plus de son RTT différentiel, on ajoute d'autres informations.  Les adresses IP d'un lien sont modélisées par la classe \textit{LinkIPs}.

\begin{lstlisting}[language=scala]
case class DiffRtt(
	rtt:      Double,
	var link: LinkIPs,
	probe:    BigInt)
\end{lstlisting}

A l'étape 6.3, on souhaite normaliser les dates de chaque lien; peu importe le moment pendant lequel le traceroute a été effectué durant une période $d_i$, on note seulement le début de cette période. Ainsi,  la classe  \textit{DiffRTTPeriod}  reprend un \textit{lien} donné, les différentes sondes Atlas ayant identifié ce lien (\textit{probes}), les RTTs différentiels de ce lien tout au long de la période et enfin les dates associées à chaque RTT différentiel.
\paragraph{La classe DiffRTTPeriod } ~
\begin{lstlisting}[language=scala]
case class DiffRTTPeriod(
	link:      LinkIPs,
	probes:    Seq[BigInt],
	rtts:      Seq[Double],
	var dates: Seq[Int])
\end{lstlisting}

A la fin des opérations de l'étape 6, on reprend pour chaque période, pour un lien donné, les RTTs différentiels ainsi que leurs dates, ensuite, on construit les bornes de l'intervalle de confiance courants pour ce lien et les bornes de l'intervalle de confiance de référence, et ce afin de comparer ces deux intervalles en vue d'inférer les anomalies possibles du délais de ce lien.


\paragraph{La classe LinkState } permet de modéliser l'état d'un lien en matière de ses intervalles de confiance pendant une période $d_i$ donnée. \textit{valueLow} est la borne inférieur de l'intervalle de confiance, \textit{valueHi} est la borne supérieure de l'intervalle de confiance, \textit{valueMedian} est la médiane des RTTs différentiels et enfin \textit{valueMean} est la moyenne des RTTs différentiels. Pour précision, les données concernant l'état d'un lien est une liste, l'idée c'est de garder l'historique de ces valeurs durant tout la période de l'analyse, toutefois, la données qui se trouve à la i ème position est une valeur qui combine les valeurs précédentes.  Pour toute période, on a une instance de \textit{LinkState} pour 
\begin{lstlisting}[language=scala]
case class LinkState(
	var valueMedian: Seq[Double],
	var valueHi:     Seq[Double],
	var valueLow:    Seq[Double],
	var valueMean:   Seq[Double])
\end{lstlisting}
\paragraph{La classe } 
\begin{lstlisting}[language=scala]

\end{lstlisting}






%\chapter{La détection des anomalies}

\section{Introduction à la détection des anomalies dans les liens}

\subsection{Introduction}

R. Fontugne et al ont exploité la distribution des sondes Atlas dans le monde afin d'étudier un des problèmes relatifs aux performances des réseaux informatiques. C'est le problème des délais des liens topologiques sur Internet.  Ils ont  exploité l'utilitaire traceroute afin d'identifier des liens topologiques sur Internet et ensuite suivre le délai de chaque lien au cours du temps.

Il est difficile d'avoir une idée globale sur la topologie de l'Internet. Toutefois, les opérateurs des réseaux informatiques disposent d'un aperçu de l'état des entités qui forment leurs réseaux, les relations entre ces entités ainsi que les éventuels problèmes. Avec la distribution abondante des sondes Atlas dans le monde en terme de type d'adressage (IPv4 et IPv6), en terme de diversité géographique, de diversité en terme d'ASs hébergeant les sondes, etc, il était possible d'aborder les délais dans les réseaux informatiques à travers de nouvelles approches reposant sur des fondements statistiques. Un des points forts de l'analyse menée par R. Fontugne et al. la possibilité de valider les méthodes proposées avec des événements marquants sur Internet observés par ailleurs.

Le travail de R. Fontugne et al. reprend trois méthodes basées sur les données collectées par les sondes Atlas. Ces méthodes sont les suivantes:


\begin{enumerate}
	\item la détection des changements des délais subissent les paquets;
	
	\item  la conception d'un modèle A pour un routeur donné. Ce modèle prédit l'acheminement du trafic afin d'identifier les routeurs et les liens en panne dans le cas d'un problème de perte de paquets;
	
	\item la création d'un score par Système Autonome afin d'évaluer l'état de ce dernier.
\end{enumerate}

Dans la suite de ce document, nous allons reprendre seulement la première méthode.  Il s'agit d'étudier le délai d'un lien topologique.







\paragraph{RTT différentiel}~

Il est indispensable de présenter la définition du RTT différentiel d'un lien avant de procéder à la description de l'algorithme de la détection des anomalies. 


\begin{tcolorbox}
	
	\textbf{ICMP (Internet Control Message Protocol)}  est un protocole utilisé pour véhiculer des messages de contrôle sur Internet.
	
	\textbf{RTT (Round Trip Time)} est obtenu en calculant la différence entre le timestamp associé à l'envoi du paquet sondé  et le timestamp associé à la réception de la réponse ICMP. RTT est une métrique pour évaluer les performances d'un réseau en matière de temps de réponse. Les mesures du RTT sont fournies par les utilitaires traceroute et ping. Par exemple, traceroute fournit les sauts impliqués dans le  chemin de forwarding, c'est le chemin parcouru par le trafic entre la source et la destination.  RTT inclut le temps de transmission, du quering et  du traitement. 
\end{tcolorbox}

La figure 	\ref{fig:rtt-differ} (a)  illustre le RTT entre la sonde P et les deux routeurs B et C. Le RTT différentiel  entre deux routeurs $B$ et $C$ adjacents, noté $\Delta_{PBC}$, est la différence entre le RTT entre la sonde $P$ et $B$ (bleu) d'une part, et le RTT entre la sonde $P$ et $C$ (rouge) dans la figure 	\ref{fig:rtt-differ} (b). 

\begin{align*}
\Delta_{PBC} &= RTT_{PC} - RTT_{PB} \\
&= \delta_{BC} + \delta_{CD} + \delta_{DA}  - \delta_{BA} \\
&= \delta_{BC} + \varepsilon_{PBC}
\end{align*}

où $\delta_{BC}$ est le délai du lien $BC$ et $\varepsilon_{PBC}$ est la différence entre les deux chemins de retour ($B$ vers $P$ et $C$ vers $P$). 
%La première composante dépend de l'état des routeurs $B$ et $C$. La deuxième composante dépend de la sonde $P$. L'analyse du RTT différentiel repose sur la variation de valeurs qu'il prend, au lieu des valeurs exactes, dans  le cas des valeurs exactes, elles peuvent dévier l'interprétation.
\begin{figure}[H]
	\centering
	\includegraphics[width=0.7\linewidth]{illustrations/rtt-differ}
	\caption{}
	\label{fig:rtt-differ}
\end{figure}



\subsection{Le principe de la détection des changements des délais}

L'évolution du délai d'un lien est déduit  de l'évolution de son RTT différentiel. Reprenons la formule du RTT différentiel du lien BC :  $\delta_{BC}$ + $\varepsilon_{PBC}$. Supposons qu'on dispose d'un nombre $n$ de sondes Atlas P$_i$, $i$ $\in$ [$1$, $n$], telles que toutes les sondes ont un chemin de retour différent depuis B et depuis C.  En effet, les RTTs différentiels pour chacune des sondes Atlas $\Delta_{P{_i}BC}$ partagent la même composante $\delta_{BC}$, toutefois, ces RTTs ont des valeurs des  $\varepsilon_{P_{i}BC}$ indépendantes. L'indépendance de ces valeurs implique que la distribution $\Delta_{P_{i}BC}$ est estimé d'être stable au cours du temps si $\delta_{BC}$ est constant. Cependant, un changement significatif de la valeur de $\delta_{BC}$ influence les valeurs des RTTs différentiels. Dans ce cas, la distribution des RTTs différentiels change si $\delta_{BC}$ change. Enfin, les changements des délais sont déduits des changements des RTTs différentiels qu'on peut les quantifier.

La détection des anomalies des délais repose sur un théorème très important en statistiques, c'est le théorème  central limite (TCL). Ce théorème  annonce que si on a une suite de variables aléatoires $X_i$ indépendantes ayant la même espérance $µ$ et la même variance $\sigma^2$, la moyenne de ces variables aléatoires est une variable aléatoire qui suit une loi normale. 

%De manière générale, le théorème central limite explique la distribution des moyennes des échantillons. Ce théorème peut être appliquer aux différents lois. Par exemple la loi normale \footnote{Un exemple illustratif dans \ref{appendix:clt-exemple}.}, binomiale, etc. 


\subsection{Quelques précisions}

Le travail de référence \cite{DBLP:journals/corr/FontugneAPB16} implique principalement les mesures de type  traceroute. Deux catégories de mesures sont utilisées :

\begin{itemize}
	\item \textit{builtin} : ce sont les traceroutes effectués par toutes les sondes Atlas vers les instances des  $13$ serveurs DNS racines. Les traceroutes sont effectués chaque $30$ minutes. En pratique, certains serveurs racines DNS déploient l'anycast. Au moment de la réalisation du travail de référence\cite{DBLP:journals/corr/FontugneAPB16}, c'étaient des traceroutes vers $ 500 $ instances des serveurs DNS racines;
	\begin{tcolorbox}
		\textbf{DNS Anycast} est une solution   utilisée pour accélérer le fonctionnement  des serveurs DNS. Les serveurs DNS adoptant cette approche fournissent des temps de réponse plus courts, et ce partout dans le monde. Les requêtes en provenance de l'utilisateur sont redirigées vers un n\oe{}ud adéquat suivant un routage prédéfini. 
	\end{tcolorbox}
	
	\item \textit{anchoring} : ce sont les traceroutes effectués par environ $400$ sondes Atlas à destination de $189$ serveurs\footnote{Sondes Atlas ayant des fonctionnalités avancées.} et ce chaque $15$ minutes.
\end{itemize}

 Le tableau \ref{tab:dataset} reprend plus de détails  concernant les traceroutes analysés dans le cadre du travail de référence. Par exemple $2.8$ traceroutes effectués par des sondes Atlas ayant IPv4 comme adressage IP  ont été exploités dans le travail de référence, y inclus les deux autres méthodes. Ces traceroutes ont été effectués par $11.538$ sondes Atlas.

\begin{table}[H]
	\centering
	\begin{tabular}{|l|l|l|}
		\hline
		& \textbf{Nombre de traceroutes}& \textbf{Nombre de sondes Atlas }\\ \hline
		IPv4		&$ 2.8 $ billion & $ 11.538 $\\ \hline
		IPv6	&	$ 1.2 $ billion & $ 4.30 $ \\ \hline
	\end{tabular}
	\caption{Récapitulatif des traceroutes utilisés dans le travail de référence }
	\label{tab:dataset}
\end{table}

La méthode conçue pour la détection des changements des délais se base sur des fondements statistiques. Ces derniers sont capables de montrer leurs performances si la taille des échantillons des RTTs différentiel caractérisant un lien considérés est grande.   Afin de surveiller un grand nombre de liens sur Internet, il faut avoir un grand nombre de sondes Atlas avec une certaine diversité et  qui sont capables de collecter une quantité importante de données relatives aux performances des réseaux informatiques.



%L'étude des délais ne concerne pas  les adresses privées, ainsi, le suivi des délais ne concerne pas les réseaux privés.  De plus, ce  suivi  se base sur les requêtes de type traceroute, et traceroute reprend une partie de la topologie de l'Internet. En effet, les liens considérés sont ceux topologiques et ne sont pas  les liens physiques. 

\section{Description de la détection des anomalies dans liens}

La détection des anomalies dans les délais des liens passe par deux étapes principales. La première étape consiste à préparer les traceroutes. Quant à la deuxième étape, elle sert à calculer une référence pour ensuite comparer les nouvelles valeurs avec la valeur de référence. Cette  référence représente un intervalle de valeurs des RTTs différentiel ou un intervalle de confiance, toute nouvelle valeur de RTT est comparée avec l'intervalle de confiance et suite à cette comparaison, on caractérise le changement par rapport à la référence.  

\subsection{Vue générale du processus de la détection}

 La figure 	\ref{fig:process-detection} reprend le processus de la détection des anomalies dans le délais d'un lien. On distingue les deux étapes principales : la préparation des traceroutes et l'exploitation des données issues de l'étape de la préparation.  Dans chacune des étapes, on distingue un succession d'opérations. 
 
 L'objectif de l'étape de préparation des traceroutes est de traiter chaque traceroute, indépendamment des autres traceroutes. Toutefois, ces traceroutes sont regroupés par période, où une période est une durée marquée par un début et une fin. En ce qui concerne la deuxième étape, elle exploite les données de la première étape pour identifier les liens et les différents changements que l'ont subi.
  
\begin{figure}[H]
	\centering
	\includegraphics[width=1\linewidth]{illustrations/process-detection}
	\caption{Le processus de l'analyse des délais d'un lien.}
	\label{fig:process-detection}
\end{figure}

\subsection{Préparation des données}
 A l'étape de la préparation des traceroutes, on distingue trois opérations, premièrement c'est l'élimination des données inutiles dans un traceroute, voire le traceroute même s'il a échoué. Deuxièmement, c'est l'identification des liens topologiques et leurs caractérisation. Enfin, c'est la fusion des données si c'est nécessaire. 
 
\paragraph{L'élimination des données inutiles}~

  Un traceroute est sauvegardé en étant un object JSON, la figure \ref{annexe:traceroute-attributes} décrit les attributs qu'un traceroute peut contenir. D'une part, il y a les attributs obligatoires dont chaque traceroute doit révéler. D'autre part, il existe les attributs optionnels.  La conservation de tel ou tel attribut dépend de l'objectif pour lequel un traceroute est analysé.  

\paragraph{Vérification de la validation des données utiles}~

 Les données utiles dans l'analyse des délais doivent être validées suivant les objectifs spécifiques prédéfinis. Etant donné que l'analyse des délais ne concerne pas les réseaux informatiques privés, les adresses IP privées sont éliminées. Comme une sonde Atlas peut recevoir jusqu'à trois signaux différents pour un saut donné, chaque signal doit reprendre un RTT et une adresse IP source du signal. Le RTT doit être valide; positif non nul.
 
 
 \paragraph{Inférence des liens à partir d'un traceroute } ~
 
 
 La figure \ref{fig:links-inference} montre tous les  liens possibles qu'on peut retrouver en se basant sur l'approche adoptée dans le  travail de référence. Les routeurs impliqués dans le trafic lors d'une requête traceroute entre une  source et une destination forment les liens; chaque deux routeurs (R1, R2) consécutifs forment un lien topologique comme illustré dans la figure \ref{fig:links-inference}. Le RTT différentiel d'un lien est la différence entre le RTT de ces deux routeurs et la sonde Atlas (la source).
 
 
     \begin{landscape}
 	\begin{figure}[H]
 		\centering
 		\resizebox{20cm}{!}{
 			% Graphic for TeX using PGF
% Title: /home/bellafkih/Documents/2018-2019/memoire/rapport_memoire/dia/traceroute-exemple.dia
% Creator: Dia v0.97.3
% CreationDate: Fri Oct  5 16:42:09 2018
% For: bellafkih
% \usepackage{tikz}
% The following commands are not supported in PSTricks at present
% We define them conditionally, so when they are implemented,
% this pgf file will use them.
\ifx\du\undefined
  \newlength{\du}
\fi
\setlength{\du}{15\unitlength}
\begin{tikzpicture}
\pgftransformxscale{1.000000}
\pgftransformyscale{-1.000000}
\definecolor{dialinecolor}{rgb}{0.000000, 0.000000, 0.000000}
\pgfsetstrokecolor{dialinecolor}
\definecolor{dialinecolor}{rgb}{1.000000, 1.000000, 1.000000}
\pgfsetfillcolor{dialinecolor}
\pgfsetlinewidth{0.000000\du}
\pgfsetdash{}{0pt}
\pgfsetdash{}{0pt}
\pgfsetbuttcap
\pgfsetmiterjoin
\pgfsetlinewidth{0.000000\du}
\pgfsetbuttcap
\pgfsetmiterjoin
\pgfsetdash{}{0pt}
\definecolor{dialinecolor}{rgb}{0.027451, 0.486275, 0.682353}
\pgfsetfillcolor{dialinecolor}
\pgfpathmoveto{\pgfpoint{-6.844583\du}{14.308835\du}}
\pgfpathlineto{\pgfpoint{-6.846043\du}{14.338043\du}}
\pgfpathlineto{\pgfpoint{-6.853345\du}{14.367981\du}}
\pgfpathlineto{\pgfpoint{-6.863568\du}{14.396458\du}}
\pgfpathlineto{\pgfpoint{-6.878171\du}{14.424571\du}}
\pgfpathlineto{\pgfpoint{-6.897522\du}{14.452683\du}}
\pgfpathlineto{\pgfpoint{-6.919427\du}{14.480065\du}}
\pgfpathlineto{\pgfpoint{-6.946079\du}{14.507082\du}}
\pgfpathlineto{\pgfpoint{-6.976748\du}{14.533369\du}}
\pgfpathlineto{\pgfpoint{-7.009606\du}{14.558561\du}}
\pgfpathlineto{\pgfpoint{-7.046846\du}{14.583753\du}}
\pgfpathlineto{\pgfpoint{-7.087737\du}{14.607849\du}}
\pgfpathlineto{\pgfpoint{-7.131549\du}{14.631215\du}}
\pgfpathlineto{\pgfpoint{-7.177916\du}{14.653851\du}}
\pgfpathlineto{\pgfpoint{-7.228299\du}{14.675757\du}}
\pgfpathlineto{\pgfpoint{-7.281238\du}{14.696568\du}}
\pgfpathlineto{\pgfpoint{-7.336733\du}{14.716648\du}}
\pgfpathlineto{\pgfpoint{-7.394783\du}{14.735998\du}}
\pgfpathlineto{\pgfpoint{-7.455389\du}{14.753888\du}}
\pgfpathlineto{\pgfpoint{-7.519281\du}{14.771047\du}}
\pgfpathlineto{\pgfpoint{-7.584269\du}{14.787477\du}}
\pgfpathlineto{\pgfpoint{-7.652907\du}{14.802446\du}}
\pgfpathlineto{\pgfpoint{-7.723736\du}{14.815954\du}}
\pgfpathlineto{\pgfpoint{-7.795660\du}{14.828733\du}}
\pgfpathlineto{\pgfpoint{-7.870504\du}{14.840781\du}}
\pgfpathlineto{\pgfpoint{-7.946445\du}{14.850639\du}}
\pgfpathlineto{\pgfpoint{-8.024940\du}{14.859766\du}}
\pgfpathlineto{\pgfpoint{-8.104531\du}{14.867433\du}}
\pgfpathlineto{\pgfpoint{-8.185583\du}{14.874005\du}}
\pgfpathlineto{\pgfpoint{-8.268825\du}{14.879116\du}}
\pgfpathlineto{\pgfpoint{-8.352432\du}{14.882767\du}}
\pgfpathlineto{\pgfpoint{-8.437865\du}{14.884958\du}}
\pgfpathlineto{\pgfpoint{-8.524393\du}{14.885688\du}}
\pgfpathlineto{\pgfpoint{-8.610556\du}{14.884958\du}}
\pgfpathlineto{\pgfpoint{-8.696353\du}{14.882767\du}}
\pgfpathlineto{\pgfpoint{-8.779960\du}{14.879116\du}}
\pgfpathlineto{\pgfpoint{-8.862837\du}{14.874005\du}}
\pgfpathlineto{\pgfpoint{-8.944254\du}{14.867433\du}}
\pgfpathlineto{\pgfpoint{-9.023845\du}{14.859766\du}}
\pgfpathlineto{\pgfpoint{-9.101611\du}{14.850639\du}}
\pgfpathlineto{\pgfpoint{-9.178281\du}{14.840781\du}}
\pgfpathlineto{\pgfpoint{-9.252396\du}{14.828733\du}}
\pgfpathlineto{\pgfpoint{-9.325050\du}{14.815954\du}}
\pgfpathlineto{\pgfpoint{-9.395514\du}{14.802446\du}}
\pgfpathlineto{\pgfpoint{-9.464152\du}{14.787477\du}}
\pgfpathlineto{\pgfpoint{-9.529869\du}{14.771047\du}}
\pgfpathlineto{\pgfpoint{-9.593396\du}{14.753888\du}}
\pgfpathlineto{\pgfpoint{-9.654367\du}{14.735998\du}}
\pgfpathlineto{\pgfpoint{-9.712783\du}{14.716648\du}}
\pgfpathlineto{\pgfpoint{-9.767912\du}{14.696568\du}}
\pgfpathlineto{\pgfpoint{-9.820851\du}{14.675757\du}}
\pgfpathlineto{\pgfpoint{-9.870870\du}{14.653851\du}}
\pgfpathlineto{\pgfpoint{-9.917967\du}{14.631215\du}}
\pgfpathlineto{\pgfpoint{-9.961414\du}{14.607849\du}}
\pgfpathlineto{\pgfpoint{-10.002304\du}{14.583753\du}}
\pgfpathlineto{\pgfpoint{-10.039544\du}{14.558561\du}}
\pgfpathlineto{\pgfpoint{-10.072768\du}{14.533369\du}}
\pgfpathlineto{\pgfpoint{-10.103071\du}{14.507082\du}}
\pgfpathlineto{\pgfpoint{-10.129723\du}{14.480065\du}}
\pgfpathlineto{\pgfpoint{-10.151629\du}{14.452683\du}}
\pgfpathlineto{\pgfpoint{-10.170979\du}{14.424571\du}}
\pgfpathlineto{\pgfpoint{-10.185583\du}{14.396458\du}}
\pgfpathlineto{\pgfpoint{-10.196171\du}{14.367981\du}}
\pgfpathlineto{\pgfpoint{-10.203108\du}{14.338043\du}}
\pgfpathlineto{\pgfpoint{-10.204933\du}{14.308835\du}}
\pgfpathlineto{\pgfpoint{-10.203108\du}{14.278897\du}}
\pgfpathlineto{\pgfpoint{-10.196171\du}{14.249689\du}}
\pgfpathlineto{\pgfpoint{-10.185583\du}{14.220482\du}}
\pgfpathlineto{\pgfpoint{-10.170979\du}{14.192369\du}}
\pgfpathlineto{\pgfpoint{-10.151629\du}{14.164257\du}}
\pgfpathlineto{\pgfpoint{-10.129723\du}{14.136874\du}}
\pgfpathlineto{\pgfpoint{-10.103071\du}{14.110222\du}}
\pgfpathlineto{\pgfpoint{-10.072768\du}{14.083935\du}}
\pgfpathlineto{\pgfpoint{-10.039544\du}{14.058379\du}}
\pgfpathlineto{\pgfpoint{-10.002304\du}{14.033552\du}}
\pgfpathlineto{\pgfpoint{-9.961414\du}{14.009091\du}}
\pgfpathlineto{\pgfpoint{-9.917967\du}{13.985724\du}}
\pgfpathlineto{\pgfpoint{-9.870870\du}{13.963453\du}}
\pgfpathlineto{\pgfpoint{-9.820851\du}{13.941183\du}}
\pgfpathlineto{\pgfpoint{-9.767912\du}{13.920372\du}}
\pgfpathlineto{\pgfpoint{-9.712783\du}{13.900292\du}}
\pgfpathlineto{\pgfpoint{-9.654367\du}{13.881672\du}}
\pgfpathlineto{\pgfpoint{-9.593396\du}{13.862687\du}}
\pgfpathlineto{\pgfpoint{-9.529869\du}{13.845892\du}}
\pgfpathlineto{\pgfpoint{-9.464152\du}{13.830193\du}}
\pgfpathlineto{\pgfpoint{-9.395514\du}{13.814859\du}}
\pgfpathlineto{\pgfpoint{-9.325050\du}{13.800985\du}}
\pgfpathlineto{\pgfpoint{-9.252396\du}{13.787842\du}}
\pgfpathlineto{\pgfpoint{-9.178281\du}{13.776889\du}}
\pgfpathlineto{\pgfpoint{-9.101611\du}{13.766301\du}}
\pgfpathlineto{\pgfpoint{-9.023845\du}{13.756809\du}}
\pgfpathlineto{\pgfpoint{-8.944254\du}{13.749507\du}}
\pgfpathlineto{\pgfpoint{-8.862837\du}{13.742935\du}}
\pgfpathlineto{\pgfpoint{-8.779960\du}{13.737824\du}}
\pgfpathlineto{\pgfpoint{-8.696353\du}{13.734173\du}}
\pgfpathlineto{\pgfpoint{-8.610556\du}{13.732347\du}}
\pgfpathlineto{\pgfpoint{-8.524393\du}{13.731252\du}}
\pgfpathlineto{\pgfpoint{-8.437865\du}{13.732347\du}}
\pgfpathlineto{\pgfpoint{-8.352432\du}{13.734173\du}}
\pgfpathlineto{\pgfpoint{-8.268825\du}{13.737824\du}}
\pgfpathlineto{\pgfpoint{-8.185583\du}{13.742935\du}}
\pgfpathlineto{\pgfpoint{-8.104531\du}{13.749507\du}}
\pgfpathlineto{\pgfpoint{-8.024940\du}{13.756809\du}}
\pgfpathlineto{\pgfpoint{-7.946445\du}{13.766301\du}}
\pgfpathlineto{\pgfpoint{-7.870504\du}{13.776889\du}}
\pgfpathlineto{\pgfpoint{-7.795660\du}{13.787842\du}}
\pgfpathlineto{\pgfpoint{-7.723736\du}{13.800985\du}}
\pgfpathlineto{\pgfpoint{-7.652907\du}{13.814859\du}}
\pgfpathlineto{\pgfpoint{-7.584269\du}{13.830193\du}}
\pgfpathlineto{\pgfpoint{-7.519281\du}{13.845892\du}}
\pgfpathlineto{\pgfpoint{-7.455389\du}{13.862687\du}}
\pgfpathlineto{\pgfpoint{-7.394783\du}{13.881672\du}}
\pgfpathlineto{\pgfpoint{-7.336733\du}{13.900292\du}}
\pgfpathlineto{\pgfpoint{-7.281238\du}{13.920372\du}}
\pgfpathlineto{\pgfpoint{-7.228299\du}{13.941183\du}}
\pgfpathlineto{\pgfpoint{-7.177916\du}{13.963453\du}}
\pgfpathlineto{\pgfpoint{-7.131549\du}{13.985724\du}}
\pgfpathlineto{\pgfpoint{-7.087737\du}{14.009091\du}}
\pgfpathlineto{\pgfpoint{-7.046846\du}{14.033552\du}}
\pgfpathlineto{\pgfpoint{-7.009606\du}{14.058379\du}}
\pgfpathlineto{\pgfpoint{-6.976748\du}{14.083935\du}}
\pgfpathlineto{\pgfpoint{-6.946079\du}{14.110222\du}}
\pgfpathlineto{\pgfpoint{-6.919427\du}{14.136874\du}}
\pgfpathlineto{\pgfpoint{-6.897522\du}{14.164257\du}}
\pgfpathlineto{\pgfpoint{-6.878171\du}{14.192369\du}}
\pgfpathlineto{\pgfpoint{-6.863568\du}{14.220482\du}}
\pgfpathlineto{\pgfpoint{-6.853345\du}{14.249689\du}}
\pgfpathlineto{\pgfpoint{-6.846043\du}{14.278897\du}}
\pgfpathlineto{\pgfpoint{-6.844583\du}{14.308835\du}}
\pgfusepath{fill}
\pgfsetlinewidth{0.000000\du}
\pgfsetbuttcap
\pgfsetmiterjoin
\pgfsetdash{}{0pt}
\definecolor{dialinecolor}{rgb}{0.678431, 0.839216, 0.905882}
\pgfsetfillcolor{dialinecolor}
\pgfpathmoveto{\pgfpoint{-8.524393\du}{14.896276\du}}
\pgfpathlineto{\pgfpoint{-8.524393\du}{14.896276\du}}
\pgfpathlineto{\pgfpoint{-8.480946\du}{14.896276\du}}
\pgfpathlineto{\pgfpoint{-8.437500\du}{14.895545\du}}
\pgfpathlineto{\pgfpoint{-8.394418\du}{14.894450\du}}
\pgfpathlineto{\pgfpoint{-8.352432\du}{14.893355\du}}
\pgfpathlineto{\pgfpoint{-8.309716\du}{14.891529\du}}
\pgfpathlineto{\pgfpoint{-8.268095\du}{14.889339\du}}
\pgfpathlineto{\pgfpoint{-8.226474\du}{14.886783\du}}
\pgfpathlineto{\pgfpoint{-8.184853\du}{14.884593\du}}
\pgfpathlineto{\pgfpoint{-8.144327\du}{14.881672\du}}
\pgfpathlineto{\pgfpoint{-8.103436\du}{14.878021\du}}
\pgfpathlineto{\pgfpoint{-8.063641\du}{14.874005\du}}
\pgfpathlineto{\pgfpoint{-8.023480\du}{14.869989\du}}
\pgfpathlineto{\pgfpoint{-7.984780\du}{14.865607\du}}
\pgfpathlineto{\pgfpoint{-7.945349\du}{14.861226\du}}
\pgfpathlineto{\pgfpoint{-7.907379\du}{14.855750\du}}
\pgfpathlineto{\pgfpoint{-7.868314\du}{14.850639\du}}
\pgfpathlineto{\pgfpoint{-7.831074\du}{14.845162\du}}
\pgfpathlineto{\pgfpoint{-7.794199\du}{14.838955\du}}
\pgfpathlineto{\pgfpoint{-7.757690\du}{14.833114\du}}
\pgfpathlineto{\pgfpoint{-7.721180\du}{14.826542\du}}
\pgfpathlineto{\pgfpoint{-7.685765\du}{14.819605\du}}
\pgfpathlineto{\pgfpoint{-7.651081\du}{14.812668\du}}
\pgfpathlineto{\pgfpoint{-7.616397\du}{14.805001\du}}
\pgfpathlineto{\pgfpoint{-7.582443\du}{14.797334\du}}
\pgfpathlineto{\pgfpoint{-7.548854\du}{14.788937\du}}
\pgfpathlineto{\pgfpoint{-7.515995\du}{14.780905\du}}
\pgfpathlineto{\pgfpoint{-7.484232\du}{14.772873\du}}
\pgfpathlineto{\pgfpoint{-7.452834\du}{14.764111\du}}
\pgfpathlineto{\pgfpoint{-7.437500\du}{14.759364\du}}
\pgfpathlineto{\pgfpoint{-7.422166\du}{14.755348\du}}
\pgfpathlineto{\pgfpoint{-7.406101\du}{14.750602\du}}
\pgfpathlineto{\pgfpoint{-7.391497\du}{14.745856\du}}
\pgfpathlineto{\pgfpoint{-7.377259\du}{14.741110\du}}
\pgfpathlineto{\pgfpoint{-7.362290\du}{14.735998\du}}
\pgfpathlineto{\pgfpoint{-7.347321\du}{14.731252\du}}
\pgfpathlineto{\pgfpoint{-7.333447\du}{14.726506\du}}
\pgfpathlineto{\pgfpoint{-7.319208\du}{14.721394\du}}
\pgfpathlineto{\pgfpoint{-7.305335\du}{14.716648\du}}
\pgfpathlineto{\pgfpoint{-7.291096\du}{14.711172\du}}
\pgfpathlineto{\pgfpoint{-7.277952\du}{14.706790\du}}
\pgfpathlineto{\pgfpoint{-7.263714\du}{14.701314\du}}
\pgfpathlineto{\pgfpoint{-7.250570\du}{14.696203\du}}
\pgfpathlineto{\pgfpoint{-7.237427\du}{14.690726\du}}
\pgfpathlineto{\pgfpoint{-7.223918\du}{14.684885\du}}
\pgfpathlineto{\pgfpoint{-7.211140\du}{14.679773\du}}
\pgfpathlineto{\pgfpoint{-7.199092\du}{14.674297\du}}
\pgfpathlineto{\pgfpoint{-7.186313\du}{14.668455\du}}
\pgfpathlineto{\pgfpoint{-7.173900\du}{14.663344\du}}
\pgfpathlineto{\pgfpoint{-7.161487\du}{14.657502\du}}
\pgfpathlineto{\pgfpoint{-7.150169\du}{14.652026\du}}
\pgfpathlineto{\pgfpoint{-7.138120\du}{14.646184\du}}
\pgfpathlineto{\pgfpoint{-7.127167\du}{14.640343\du}}
\pgfpathlineto{\pgfpoint{-7.115484\du}{14.634501\du}}
\pgfpathlineto{\pgfpoint{-7.104166\du}{14.628660\du}}
\pgfpathlineto{\pgfpoint{-7.092848\du}{14.622818\du}}
\pgfpathlineto{\pgfpoint{-7.082261\du}{14.616612\du}}
\pgfpathlineto{\pgfpoint{-7.072403\du}{14.610770\du}}
\pgfpathlineto{\pgfpoint{-7.061815\du}{14.604928\du}}
\pgfpathlineto{\pgfpoint{-7.051592\du}{14.598357\du}}
\pgfpathlineto{\pgfpoint{-7.041735\du}{14.592515\du}}
\pgfpathlineto{\pgfpoint{-7.031877\du}{14.585943\du}}
\pgfpathlineto{\pgfpoint{-7.022750\du}{14.579737\du}}
\pgfpathlineto{\pgfpoint{-7.013622\du}{14.573165\du}}
\pgfpathlineto{\pgfpoint{-7.004130\du}{14.567323\du}}
\pgfpathlineto{\pgfpoint{-6.995368\du}{14.560752\du}}
\pgfpathlineto{\pgfpoint{-6.986240\du}{14.554545\du}}
\pgfpathlineto{\pgfpoint{-6.978208\du}{14.547973\du}}
\pgfpathlineto{\pgfpoint{-6.969446\du}{14.541036\du}}
\pgfpathlineto{\pgfpoint{-6.962144\du}{14.534465\du}}
\pgfpathlineto{\pgfpoint{-6.954112\du}{14.528258\du}}
\pgfpathlineto{\pgfpoint{-6.946079\du}{14.520956\du}}
\pgfpathlineto{\pgfpoint{-6.939508\du}{14.514750\du}}
\pgfpathlineto{\pgfpoint{-6.932206\du}{14.507813\du}}
\pgfpathlineto{\pgfpoint{-6.925634\du}{14.500511\du}}
\pgfpathlineto{\pgfpoint{-6.918697\du}{14.494304\du}}
\pgfpathlineto{\pgfpoint{-6.912125\du}{14.487367\du}}
\pgfpathlineto{\pgfpoint{-6.905919\du}{14.480065\du}}
\pgfpathlineto{\pgfpoint{-6.900077\du}{14.473128\du}}
\pgfpathlineto{\pgfpoint{-6.894601\du}{14.466192\du}}
\pgfpathlineto{\pgfpoint{-6.888759\du}{14.459255\du}}
\pgfpathlineto{\pgfpoint{-6.884013\du}{14.451953\du}}
\pgfpathlineto{\pgfpoint{-6.878537\du}{14.444651\du}}
\pgfpathlineto{\pgfpoint{-6.873790\du}{14.437349\du}}
\pgfpathlineto{\pgfpoint{-6.869409\du}{14.430412\du}}
\pgfpathlineto{\pgfpoint{-6.865393\du}{14.422745\du}}
\pgfpathlineto{\pgfpoint{-6.861377\du}{14.415808\du}}
\pgfpathlineto{\pgfpoint{-6.858091\du}{14.408141\du}}
\pgfpathlineto{\pgfpoint{-6.854075\du}{14.400474\du}}
\pgfpathlineto{\pgfpoint{-6.850789\du}{14.392807\du}}
\pgfpathlineto{\pgfpoint{-6.848234\du}{14.385505\du}}
\pgfpathlineto{\pgfpoint{-6.845313\du}{14.378203\du}}
\pgfpathlineto{\pgfpoint{-6.843487\du}{14.370901\du}}
\pgfpathlineto{\pgfpoint{-6.840566\du}{14.363234\du}}
\pgfpathlineto{\pgfpoint{-6.839471\du}{14.354837\du}}
\pgfpathlineto{\pgfpoint{-6.837281\du}{14.347170\du}}
\pgfpathlineto{\pgfpoint{-6.836185\du}{14.339868\du}}
\pgfpathlineto{\pgfpoint{-6.835455\du}{14.332201\du}}
\pgfpathlineto{\pgfpoint{-6.834725\du}{14.323804\du}}
\pgfpathlineto{\pgfpoint{-6.833995\du}{14.316502\du}}
\pgfpathlineto{\pgfpoint{-6.833995\du}{14.308835\du}}
\pgfpathlineto{\pgfpoint{-6.854075\du}{14.308835\du}}
\pgfpathlineto{\pgfpoint{-6.854805\du}{14.315772\du}}
\pgfpathlineto{\pgfpoint{-6.854805\du}{14.322709\du}}
\pgfpathlineto{\pgfpoint{-6.855170\du}{14.329645\du}}
\pgfpathlineto{\pgfpoint{-6.856996\du}{14.336947\du}}
\pgfpathlineto{\pgfpoint{-6.858091\du}{14.343884\du}}
\pgfpathlineto{\pgfpoint{-6.858821\du}{14.350821\du}}
\pgfpathlineto{\pgfpoint{-6.861012\du}{14.357758\du}}
\pgfpathlineto{\pgfpoint{-6.862472\du}{14.365060\du}}
\pgfpathlineto{\pgfpoint{-6.864663\du}{14.371266\du}}
\pgfpathlineto{\pgfpoint{-6.867219\du}{14.378203\du}}
\pgfpathlineto{\pgfpoint{-6.870139\du}{14.385505\du}}
\pgfpathlineto{\pgfpoint{-6.872695\du}{14.392442\du}}
\pgfpathlineto{\pgfpoint{-6.876711\du}{14.399379\du}}
\pgfpathlineto{\pgfpoint{-6.879632\du}{14.405951\du}}
\pgfpathlineto{\pgfpoint{-6.883648\du}{14.412888\du}}
\pgfpathlineto{\pgfpoint{-6.886569\du}{14.419459\du}}
\pgfpathlineto{\pgfpoint{-6.890950\du}{14.426396\du}}
\pgfpathlineto{\pgfpoint{-6.895331\du}{14.433333\du}}
\pgfpathlineto{\pgfpoint{-6.900077\du}{14.439905\du}}
\pgfpathlineto{\pgfpoint{-6.905189\du}{14.446111\du}}
\pgfpathlineto{\pgfpoint{-6.909935\du}{14.453413\du}}
\pgfpathlineto{\pgfpoint{-6.916142\du}{14.460350\du}}
\pgfpathlineto{\pgfpoint{-6.921618\du}{14.466557\du}}
\pgfpathlineto{\pgfpoint{-6.926729\du}{14.473128\du}}
\pgfpathlineto{\pgfpoint{-6.933301\du}{14.479700\du}}
\pgfpathlineto{\pgfpoint{-6.940238\du}{14.486637\du}}
\pgfpathlineto{\pgfpoint{-6.946079\du}{14.493209\du}}
\pgfpathlineto{\pgfpoint{-6.953381\du}{14.499415\du}}
\pgfpathlineto{\pgfpoint{-6.959953\du}{14.505987\du}}
\pgfpathlineto{\pgfpoint{-6.967620\du}{14.512194\du}}
\pgfpathlineto{\pgfpoint{-6.974922\du}{14.518766\du}}
\pgfpathlineto{\pgfpoint{-6.982954\du}{14.525337\du}}
\pgfpathlineto{\pgfpoint{-6.990621\du}{14.531544\du}}
\pgfpathlineto{\pgfpoint{-6.999384\du}{14.538116\du}}
\pgfpathlineto{\pgfpoint{-7.007781\du}{14.543957\du}}
\pgfpathlineto{\pgfpoint{-7.016543\du}{14.550529\du}}
\pgfpathlineto{\pgfpoint{-7.024575\du}{14.556736\du}}
\pgfpathlineto{\pgfpoint{-7.033338\du}{14.562577\du}}
\pgfpathlineto{\pgfpoint{-7.042830\du}{14.569149\du}}
\pgfpathlineto{\pgfpoint{-7.053053\du}{14.574990\du}}
\pgfpathlineto{\pgfpoint{-7.062180\du}{14.581562\du}}
\pgfpathlineto{\pgfpoint{-7.072403\du}{14.587404\du}}
\pgfpathlineto{\pgfpoint{-7.082261\du}{14.593245\du}}
\pgfpathlineto{\pgfpoint{-7.092118\du}{14.599087\du}}
\pgfpathlineto{\pgfpoint{-7.103071\du}{14.604928\du}}
\pgfpathlineto{\pgfpoint{-7.114024\du}{14.610770\du}}
\pgfpathlineto{\pgfpoint{-7.124247\du}{14.616612\du}}
\pgfpathlineto{\pgfpoint{-7.136295\du}{14.622453\du}}
\pgfpathlineto{\pgfpoint{-7.146883\du}{14.627564\du}}
\pgfpathlineto{\pgfpoint{-7.158931\du}{14.633406\du}}
\pgfpathlineto{\pgfpoint{-7.170249\du}{14.639248\du}}
\pgfpathlineto{\pgfpoint{-7.182662\du}{14.644724\du}}
\pgfpathlineto{\pgfpoint{-7.194710\du}{14.649835\du}}
\pgfpathlineto{\pgfpoint{-7.207124\du}{14.655677\du}}
\pgfpathlineto{\pgfpoint{-7.219172\du}{14.661153\du}}
\pgfpathlineto{\pgfpoint{-7.232315\du}{14.666265\du}}
\pgfpathlineto{\pgfpoint{-7.245094\du}{14.671376\du}}
\pgfpathlineto{\pgfpoint{-7.257872\du}{14.676852\du}}
\pgfpathlineto{\pgfpoint{-7.271016\du}{14.681964\du}}
\pgfpathlineto{\pgfpoint{-7.284524\du}{14.687440\du}}
\pgfpathlineto{\pgfpoint{-7.297668\du}{14.691821\du}}
\pgfpathlineto{\pgfpoint{-7.311906\du}{14.697298\du}}
\pgfpathlineto{\pgfpoint{-7.325780\du}{14.702044\du}}
\pgfpathlineto{\pgfpoint{-7.340019\du}{14.707156\du}}
\pgfpathlineto{\pgfpoint{-7.354623\du}{14.711902\du}}
\pgfpathlineto{\pgfpoint{-7.368496\du}{14.716648\du}}
\pgfpathlineto{\pgfpoint{-7.383100\du}{14.721394\du}}
\pgfpathlineto{\pgfpoint{-7.397339\du}{14.725775\du}}
\pgfpathlineto{\pgfpoint{-7.413038\du}{14.730522\du}}
\pgfpathlineto{\pgfpoint{-7.427277\du}{14.735268\du}}
\pgfpathlineto{\pgfpoint{-7.442976\du}{14.740014\du}}
\pgfpathlineto{\pgfpoint{-7.457945\du}{14.744030\du}}
\pgfpathlineto{\pgfpoint{-7.489343\du}{14.752793\du}}
\pgfpathlineto{\pgfpoint{-7.521472\du}{14.761190\du}}
\pgfpathlineto{\pgfpoint{-7.553600\du}{14.769222\du}}
\pgfpathlineto{\pgfpoint{-7.586824\du}{14.777254\du}}
\pgfpathlineto{\pgfpoint{-7.621143\du}{14.784921\du}}
\pgfpathlineto{\pgfpoint{-7.655097\du}{14.792223\du}}
\pgfpathlineto{\pgfpoint{-7.689782\du}{14.799160\du}}
\pgfpathlineto{\pgfpoint{-7.725561\du}{14.806097\du}}
\pgfpathlineto{\pgfpoint{-7.760975\du}{14.812668\du}}
\pgfpathlineto{\pgfpoint{-7.797850\du}{14.818875\du}}
\pgfpathlineto{\pgfpoint{-7.834360\du}{14.824717\du}}
\pgfpathlineto{\pgfpoint{-7.871600\du}{14.830558\du}}
\pgfpathlineto{\pgfpoint{-7.909205\du}{14.836035\du}}
\pgfpathlineto{\pgfpoint{-7.947905\du}{14.840781\du}}
\pgfpathlineto{\pgfpoint{-7.986605\du}{14.845162\du}}
\pgfpathlineto{\pgfpoint{-8.026036\du}{14.849908\du}}
\pgfpathlineto{\pgfpoint{-8.065101\du}{14.853559\du}}
\pgfpathlineto{\pgfpoint{-8.105262\du}{14.857575\du}}
\pgfpathlineto{\pgfpoint{-8.145422\du}{14.860496\du}}
\pgfpathlineto{\pgfpoint{-8.186678\du}{14.864147\du}}
\pgfpathlineto{\pgfpoint{-8.227569\du}{14.867068\du}}
\pgfpathlineto{\pgfpoint{-8.268825\du}{14.869258\du}}
\pgfpathlineto{\pgfpoint{-8.310446\du}{14.871084\du}}
\pgfpathlineto{\pgfpoint{-8.353162\du}{14.872909\du}}
\pgfpathlineto{\pgfpoint{-8.395879\du}{14.874005\du}}
\pgfpathlineto{\pgfpoint{-8.437865\du}{14.875100\du}}
\pgfpathlineto{\pgfpoint{-8.481311\du}{14.875100\du}}
\pgfpathlineto{\pgfpoint{-8.524393\du}{14.875830\du}}
\pgfpathlineto{\pgfpoint{-8.524393\du}{14.875830\du}}
\pgfpathlineto{\pgfpoint{-8.524393\du}{14.875830\du}}
\pgfpathlineto{\pgfpoint{-8.525123\du}{14.875830\du}}
\pgfpathlineto{\pgfpoint{-8.526948\du}{14.875830\du}}
\pgfpathlineto{\pgfpoint{-8.528044\du}{14.876195\du}}
\pgfpathlineto{\pgfpoint{-8.528774\du}{14.876195\du}}
\pgfpathlineto{\pgfpoint{-8.529139\du}{14.876925\du}}
\pgfpathlineto{\pgfpoint{-8.530599\du}{14.877291\du}}
\pgfpathlineto{\pgfpoint{-8.531330\du}{14.878021\du}}
\pgfpathlineto{\pgfpoint{-8.532060\du}{14.878751\du}}
\pgfpathlineto{\pgfpoint{-8.533155\du}{14.880576\du}}
\pgfpathlineto{\pgfpoint{-8.533885\du}{14.882037\du}}
\pgfpathlineto{\pgfpoint{-8.533885\du}{14.883862\du}}
\pgfpathlineto{\pgfpoint{-8.534615\du}{14.885688\du}}
\pgfpathlineto{\pgfpoint{-8.533885\du}{14.887878\du}}
\pgfpathlineto{\pgfpoint{-8.533885\du}{14.889704\du}}
\pgfpathlineto{\pgfpoint{-8.533155\du}{14.891529\du}}
\pgfpathlineto{\pgfpoint{-8.532060\du}{14.893355\du}}
\pgfpathlineto{\pgfpoint{-8.531330\du}{14.893720\du}}
\pgfpathlineto{\pgfpoint{-8.530599\du}{14.894450\du}}
\pgfpathlineto{\pgfpoint{-8.529139\du}{14.895180\du}}
\pgfpathlineto{\pgfpoint{-8.528774\du}{14.895545\du}}
\pgfpathlineto{\pgfpoint{-8.528044\du}{14.895545\du}}
\pgfpathlineto{\pgfpoint{-8.526948\du}{14.896276\du}}
\pgfpathlineto{\pgfpoint{-8.525123\du}{14.896276\du}}
\pgfpathlineto{\pgfpoint{-8.524393\du}{14.896276\du}}
\pgfusepath{fill}
\pgfsetbuttcap
\pgfsetmiterjoin
\pgfsetdash{}{0pt}
\definecolor{dialinecolor}{rgb}{0.678431, 0.839216, 0.905882}
\pgfsetfillcolor{dialinecolor}
\pgfpathmoveto{\pgfpoint{-10.215156\du}{14.308835\du}}
\pgfpathlineto{\pgfpoint{-10.215156\du}{14.308835\du}}
\pgfpathlineto{\pgfpoint{-10.215156\du}{14.316502\du}}
\pgfpathlineto{\pgfpoint{-10.214791\du}{14.323804\du}}
\pgfpathlineto{\pgfpoint{-10.214060\du}{14.332201\du}}
\pgfpathlineto{\pgfpoint{-10.212965\du}{14.339868\du}}
\pgfpathlineto{\pgfpoint{-10.211870\du}{14.347170\du}}
\pgfpathlineto{\pgfpoint{-10.210044\du}{14.354837\du}}
\pgfpathlineto{\pgfpoint{-10.208219\du}{14.363234\du}}
\pgfpathlineto{\pgfpoint{-10.206028\du}{14.370901\du}}
\pgfpathlineto{\pgfpoint{-10.203838\du}{14.378203\du}}
\pgfpathlineto{\pgfpoint{-10.201282\du}{14.385505\du}}
\pgfpathlineto{\pgfpoint{-10.198361\du}{14.392807\du}}
\pgfpathlineto{\pgfpoint{-10.194710\du}{14.400474\du}}
\pgfpathlineto{\pgfpoint{-10.191424\du}{14.408141\du}}
\pgfpathlineto{\pgfpoint{-10.187774\du}{14.415808\du}}
\pgfpathlineto{\pgfpoint{-10.183392\du}{14.422745\du}}
\pgfpathlineto{\pgfpoint{-10.179741\du}{14.430412\du}}
\pgfpathlineto{\pgfpoint{-10.174630\du}{14.437349\du}}
\pgfpathlineto{\pgfpoint{-10.170614\du}{14.444651\du}}
\pgfpathlineto{\pgfpoint{-10.165138\du}{14.451953\du}}
\pgfpathlineto{\pgfpoint{-10.160391\du}{14.459255\du}}
\pgfpathlineto{\pgfpoint{-10.154915\du}{14.466192\du}}
\pgfpathlineto{\pgfpoint{-10.149073\du}{14.473128\du}}
\pgfpathlineto{\pgfpoint{-10.143232\du}{14.480065\du}}
\pgfpathlineto{\pgfpoint{-10.137390\du}{14.487367\du}}
\pgfpathlineto{\pgfpoint{-10.130453\du}{14.494304\du}}
\pgfpathlineto{\pgfpoint{-10.123882\du}{14.500511\du}}
\pgfpathlineto{\pgfpoint{-10.116945\du}{14.507813\du}}
\pgfpathlineto{\pgfpoint{-10.110008\du}{14.514750\du}}
\pgfpathlineto{\pgfpoint{-10.103071\du}{14.520956\du}}
\pgfpathlineto{\pgfpoint{-10.094309\du}{14.528258\du}}
\pgfpathlineto{\pgfpoint{-10.087372\du}{14.534465\du}}
\pgfpathlineto{\pgfpoint{-10.079705\du}{14.541036\du}}
\pgfpathlineto{\pgfpoint{-10.070943\du}{14.547973\du}}
\pgfpathlineto{\pgfpoint{-10.062545\du}{14.554545\du}}
\pgfpathlineto{\pgfpoint{-10.053783\du}{14.560752\du}}
\pgfpathlineto{\pgfpoint{-10.045386\du}{14.567323\du}}
\pgfpathlineto{\pgfpoint{-10.035528\du}{14.573165\du}}
\pgfpathlineto{\pgfpoint{-10.026401\du}{14.579737\du}}
\pgfpathlineto{\pgfpoint{-10.016908\du}{14.585943\du}}
\pgfpathlineto{\pgfpoint{-10.007416\du}{14.592515\du}}
\pgfpathlineto{\pgfpoint{-9.997558\du}{14.598357\du}}
\pgfpathlineto{\pgfpoint{-9.987335\du}{14.604928\du}}
\pgfpathlineto{\pgfpoint{-9.977113\du}{14.610770\du}}
\pgfpathlineto{\pgfpoint{-9.966525\du}{14.616612\du}}
\pgfpathlineto{\pgfpoint{-9.956302\du}{14.622818\du}}
\pgfpathlineto{\pgfpoint{-9.945349\du}{14.628660\du}}
\pgfpathlineto{\pgfpoint{-9.933666\du}{14.634501\du}}
\pgfpathlineto{\pgfpoint{-9.922348\du}{14.640343\du}}
\pgfpathlineto{\pgfpoint{-9.910665\du}{14.646184\du}}
\pgfpathlineto{\pgfpoint{-9.899347\du}{14.652026\du}}
\pgfpathlineto{\pgfpoint{-9.887664\du}{14.657502\du}}
\pgfpathlineto{\pgfpoint{-9.875251\du}{14.663344\du}}
\pgfpathlineto{\pgfpoint{-9.863202\du}{14.668455\du}}
\pgfpathlineto{\pgfpoint{-9.850059\du}{14.674297\du}}
\pgfpathlineto{\pgfpoint{-9.838011\du}{14.679773\du}}
\pgfpathlineto{\pgfpoint{-9.825232\du}{14.684885\du}}
\pgfpathlineto{\pgfpoint{-9.811359\du}{14.690726\du}}
\pgfpathlineto{\pgfpoint{-9.798945\du}{14.696203\du}}
\pgfpathlineto{\pgfpoint{-9.785802\du}{14.701314\du}}
\pgfpathlineto{\pgfpoint{-9.771563\du}{14.706790\du}}
\pgfpathlineto{\pgfpoint{-9.758420\du}{14.711172\du}}
\pgfpathlineto{\pgfpoint{-9.744181\du}{14.716648\du}}
\pgfpathlineto{\pgfpoint{-9.730307\du}{14.721394\du}}
\pgfpathlineto{\pgfpoint{-9.715703\du}{14.726506\du}}
\pgfpathlineto{\pgfpoint{-9.701830\du}{14.731252\du}}
\pgfpathlineto{\pgfpoint{-9.686131\du}{14.735998\du}}
\pgfpathlineto{\pgfpoint{-9.672257\du}{14.741110\du}}
\pgfpathlineto{\pgfpoint{-9.657653\du}{14.745856\du}}
\pgfpathlineto{\pgfpoint{-9.642319\du}{14.750602\du}}
\pgfpathlineto{\pgfpoint{-9.626620\du}{14.755348\du}}
\pgfpathlineto{\pgfpoint{-9.611651\du}{14.759364\du}}
\pgfpathlineto{\pgfpoint{-9.595952\du}{14.764111\du}}
\pgfpathlineto{\pgfpoint{-9.564188\du}{14.772873\du}}
\pgfpathlineto{\pgfpoint{-9.532425\du}{14.780905\du}}
\pgfpathlineto{\pgfpoint{-9.499201\du}{14.788937\du}}
\pgfpathlineto{\pgfpoint{-9.466707\du}{14.797334\du}}
\pgfpathlineto{\pgfpoint{-9.432023\du}{14.805001\du}}
\pgfpathlineto{\pgfpoint{-9.397704\du}{14.812668\du}}
\pgfpathlineto{\pgfpoint{-9.363020\du}{14.819605\du}}
\pgfpathlineto{\pgfpoint{-9.327240\du}{14.826542\du}}
\pgfpathlineto{\pgfpoint{-9.291096\du}{14.833114\du}}
\pgfpathlineto{\pgfpoint{-9.254586\du}{14.838955\du}}
\pgfpathlineto{\pgfpoint{-9.217346\du}{14.845162\du}}
\pgfpathlineto{\pgfpoint{-9.179741\du}{14.850639\du}}
\pgfpathlineto{\pgfpoint{-9.141771\du}{14.855750\du}}
\pgfpathlineto{\pgfpoint{-9.103071\du}{14.861226\du}}
\pgfpathlineto{\pgfpoint{-9.064371\du}{14.865607\du}}
\pgfpathlineto{\pgfpoint{-9.025305\du}{14.869989\du}}
\pgfpathlineto{\pgfpoint{-8.985145\du}{14.874005\du}}
\pgfpathlineto{\pgfpoint{-8.945349\du}{14.878021\du}}
\pgfpathlineto{\pgfpoint{-8.904458\du}{14.881672\du}}
\pgfpathlineto{\pgfpoint{-8.863568\du}{14.884593\du}}
\pgfpathlineto{\pgfpoint{-8.822312\du}{14.886783\du}}
\pgfpathlineto{\pgfpoint{-8.780326\du}{14.889339\du}}
\pgfpathlineto{\pgfpoint{-8.738705\du}{14.891529\du}}
\pgfpathlineto{\pgfpoint{-8.696353\du}{14.893355\du}}
\pgfpathlineto{\pgfpoint{-8.654367\du}{14.894450\du}}
\pgfpathlineto{\pgfpoint{-8.610921\du}{14.895545\du}}
\pgfpathlineto{\pgfpoint{-8.568204\du}{14.896276\du}}
\pgfpathlineto{\pgfpoint{-8.524393\du}{14.896276\du}}
\pgfpathlineto{\pgfpoint{-8.524393\du}{14.875830\du}}
\pgfpathlineto{\pgfpoint{-8.567109\du}{14.875100\du}}
\pgfpathlineto{\pgfpoint{-8.610556\du}{14.875100\du}}
\pgfpathlineto{\pgfpoint{-8.652907\du}{14.874005\du}}
\pgfpathlineto{\pgfpoint{-8.695623\du}{14.872909\du}}
\pgfpathlineto{\pgfpoint{-8.737974\du}{14.871084\du}}
\pgfpathlineto{\pgfpoint{-8.779960\du}{14.869258\du}}
\pgfpathlineto{\pgfpoint{-8.820851\du}{14.867068\du}}
\pgfpathlineto{\pgfpoint{-8.862107\du}{14.864147\du}}
\pgfpathlineto{\pgfpoint{-8.902998\du}{14.860496\du}}
\pgfpathlineto{\pgfpoint{-8.943159\du}{14.857575\du}}
\pgfpathlineto{\pgfpoint{-8.982954\du}{14.853559\du}}
\pgfpathlineto{\pgfpoint{-9.022385\du}{14.849908\du}}
\pgfpathlineto{\pgfpoint{-9.061815\du}{14.845162\du}}
\pgfpathlineto{\pgfpoint{-9.100880\du}{14.840781\du}}
\pgfpathlineto{\pgfpoint{-9.139216\du}{14.836035\du}}
\pgfpathlineto{\pgfpoint{-9.176821\du}{14.830558\du}}
\pgfpathlineto{\pgfpoint{-9.214060\du}{14.824717\du}}
\pgfpathlineto{\pgfpoint{-9.250570\du}{14.818875\du}}
\pgfpathlineto{\pgfpoint{-9.287810\du}{14.812668\du}}
\pgfpathlineto{\pgfpoint{-9.322859\du}{14.806097\du}}
\pgfpathlineto{\pgfpoint{-9.359004\du}{14.799160\du}}
\pgfpathlineto{\pgfpoint{-9.393688\du}{14.792223\du}}
\pgfpathlineto{\pgfpoint{-9.428007\du}{14.784921\du}}
\pgfpathlineto{\pgfpoint{-9.461961\du}{14.777254\du}}
\pgfpathlineto{\pgfpoint{-9.495185\du}{14.769222\du}}
\pgfpathlineto{\pgfpoint{-9.526583\du}{14.761190\du}}
\pgfpathlineto{\pgfpoint{-9.559077\du}{14.752793\du}}
\pgfpathlineto{\pgfpoint{-9.590110\du}{14.744030\du}}
\pgfpathlineto{\pgfpoint{-9.605809\du}{14.740014\du}}
\pgfpathlineto{\pgfpoint{-9.621508\du}{14.735268\du}}
\pgfpathlineto{\pgfpoint{-9.636112\du}{14.730522\du}}
\pgfpathlineto{\pgfpoint{-9.651081\du}{14.725775\du}}
\pgfpathlineto{\pgfpoint{-9.666415\du}{14.721394\du}}
\pgfpathlineto{\pgfpoint{-9.680654\du}{14.716648\du}}
\pgfpathlineto{\pgfpoint{-9.694893\du}{14.711902\du}}
\pgfpathlineto{\pgfpoint{-9.709132\du}{14.707156\du}}
\pgfpathlineto{\pgfpoint{-9.723370\du}{14.702044\du}}
\pgfpathlineto{\pgfpoint{-9.737244\du}{14.697298\du}}
\pgfpathlineto{\pgfpoint{-9.751118\du}{14.691821\du}}
\pgfpathlineto{\pgfpoint{-9.764626\du}{14.687440\du}}
\pgfpathlineto{\pgfpoint{-9.778135\du}{14.681964\du}}
\pgfpathlineto{\pgfpoint{-9.790913\du}{14.676852\du}}
\pgfpathlineto{\pgfpoint{-9.804422\du}{14.671376\du}}
\pgfpathlineto{\pgfpoint{-9.817200\du}{14.666265\du}}
\pgfpathlineto{\pgfpoint{-9.829614\du}{14.661153\du}}
\pgfpathlineto{\pgfpoint{-9.842757\du}{14.655677\du}}
\pgfpathlineto{\pgfpoint{-9.854805\du}{14.649835\du}}
\pgfpathlineto{\pgfpoint{-9.866123\du}{14.644724\du}}
\pgfpathlineto{\pgfpoint{-9.878902\du}{14.639248\du}}
\pgfpathlineto{\pgfpoint{-9.890585\du}{14.633406\du}}
\pgfpathlineto{\pgfpoint{-9.902268\du}{14.627564\du}}
\pgfpathlineto{\pgfpoint{-9.913221\du}{14.622453\du}}
\pgfpathlineto{\pgfpoint{-9.924904\du}{14.616612\du}}
\pgfpathlineto{\pgfpoint{-9.934761\du}{14.610770\du}}
\pgfpathlineto{\pgfpoint{-9.946079\du}{14.604928\du}}
\pgfpathlineto{\pgfpoint{-9.956667\du}{14.599087\du}}
\pgfpathlineto{\pgfpoint{-9.966525\du}{14.593245\du}}
\pgfpathlineto{\pgfpoint{-9.977113\du}{14.587404\du}}
\pgfpathlineto{\pgfpoint{-9.986970\du}{14.581562\du}}
\pgfpathlineto{\pgfpoint{-9.996463\du}{14.574990\du}}
\pgfpathlineto{\pgfpoint{-10.006320\du}{14.569149\du}}
\pgfpathlineto{\pgfpoint{-10.015448\du}{14.562577\du}}
\pgfpathlineto{\pgfpoint{-10.023845\du}{14.556736\du}}
\pgfpathlineto{\pgfpoint{-10.032607\du}{14.550529\du}}
\pgfpathlineto{\pgfpoint{-10.041735\du}{14.543957\du}}
\pgfpathlineto{\pgfpoint{-10.050497\du}{14.538116\du}}
\pgfpathlineto{\pgfpoint{-10.058894\du}{14.531544\du}}
\pgfpathlineto{\pgfpoint{-10.066196\du}{14.525337\du}}
\pgfpathlineto{\pgfpoint{-10.074228\du}{14.518766\du}}
\pgfpathlineto{\pgfpoint{-10.081895\du}{14.512194\du}}
\pgfpathlineto{\pgfpoint{-10.088832\du}{14.505987\du}}
\pgfpathlineto{\pgfpoint{-10.096134\du}{14.499415\du}}
\pgfpathlineto{\pgfpoint{-10.103071\du}{14.493209\du}}
\pgfpathlineto{\pgfpoint{-10.109278\du}{14.486637\du}}
\pgfpathlineto{\pgfpoint{-10.115849\du}{14.479700\du}}
\pgfpathlineto{\pgfpoint{-10.121691\du}{14.473128\du}}
\pgfpathlineto{\pgfpoint{-10.127898\du}{14.466557\du}}
\pgfpathlineto{\pgfpoint{-10.133009\du}{14.460350\du}}
\pgfpathlineto{\pgfpoint{-10.139216\du}{14.453413\du}}
\pgfpathlineto{\pgfpoint{-10.143597\du}{14.446842\du}}
\pgfpathlineto{\pgfpoint{-10.149073\du}{14.439905\du}}
\pgfpathlineto{\pgfpoint{-10.153454\du}{14.433333\du}}
\pgfpathlineto{\pgfpoint{-10.157836\du}{14.426396\du}}
\pgfpathlineto{\pgfpoint{-10.162217\du}{14.419459\du}}
\pgfpathlineto{\pgfpoint{-10.165503\du}{14.412888\du}}
\pgfpathlineto{\pgfpoint{-10.169519\du}{14.405951\du}}
\pgfpathlineto{\pgfpoint{-10.173535\du}{14.399379\du}}
\pgfpathlineto{\pgfpoint{-10.176456\du}{14.392442\du}}
\pgfpathlineto{\pgfpoint{-10.179011\du}{14.385505\du}}
\pgfpathlineto{\pgfpoint{-10.182297\du}{14.378203\du}}
\pgfpathlineto{\pgfpoint{-10.184488\du}{14.371266\du}}
\pgfpathlineto{\pgfpoint{-10.186678\du}{14.365060\du}}
\pgfpathlineto{\pgfpoint{-10.188139\du}{14.357758\du}}
\pgfpathlineto{\pgfpoint{-10.190329\du}{14.350821\du}}
\pgfpathlineto{\pgfpoint{-10.191424\du}{14.343884\du}}
\pgfpathlineto{\pgfpoint{-10.192520\du}{14.336947\du}}
\pgfpathlineto{\pgfpoint{-10.193980\du}{14.329645\du}}
\pgfpathlineto{\pgfpoint{-10.194345\du}{14.322709\du}}
\pgfpathlineto{\pgfpoint{-10.194345\du}{14.315772\du}}
\pgfpathlineto{\pgfpoint{-10.194710\du}{14.308835\du}}
\pgfpathlineto{\pgfpoint{-10.194710\du}{14.308835\du}}
\pgfpathlineto{\pgfpoint{-10.194710\du}{14.308835\du}}
\pgfpathlineto{\pgfpoint{-10.194710\du}{14.307009\du}}
\pgfpathlineto{\pgfpoint{-10.194710\du}{14.306279\du}}
\pgfpathlineto{\pgfpoint{-10.195075\du}{14.305184\du}}
\pgfpathlineto{\pgfpoint{-10.195075\du}{14.304089\du}}
\pgfpathlineto{\pgfpoint{-10.196171\du}{14.303359\du}}
\pgfpathlineto{\pgfpoint{-10.196536\du}{14.302263\du}}
\pgfpathlineto{\pgfpoint{-10.196901\du}{14.301533\du}}
\pgfpathlineto{\pgfpoint{-10.197996\du}{14.301168\du}}
\pgfpathlineto{\pgfpoint{-10.199457\du}{14.300073\du}}
\pgfpathlineto{\pgfpoint{-10.201282\du}{14.298612\du}}
\pgfpathlineto{\pgfpoint{-10.203108\du}{14.298247\du}}
\pgfpathlineto{\pgfpoint{-10.204933\du}{14.298247\du}}
\pgfpathlineto{\pgfpoint{-10.207124\du}{14.298247\du}}
\pgfpathlineto{\pgfpoint{-10.208584\du}{14.298612\du}}
\pgfpathlineto{\pgfpoint{-10.210775\du}{14.300073\du}}
\pgfpathlineto{\pgfpoint{-10.212600\du}{14.301168\du}}
\pgfpathlineto{\pgfpoint{-10.212965\du}{14.301533\du}}
\pgfpathlineto{\pgfpoint{-10.213330\du}{14.302263\du}}
\pgfpathlineto{\pgfpoint{-10.214060\du}{14.303359\du}}
\pgfpathlineto{\pgfpoint{-10.214791\du}{14.304089\du}}
\pgfpathlineto{\pgfpoint{-10.214791\du}{14.305184\du}}
\pgfpathlineto{\pgfpoint{-10.215156\du}{14.306279\du}}
\pgfpathlineto{\pgfpoint{-10.215156\du}{14.307009\du}}
\pgfpathlineto{\pgfpoint{-10.215156\du}{14.308835\du}}
\pgfusepath{fill}
\pgfsetbuttcap
\pgfsetmiterjoin
\pgfsetdash{}{0pt}
\definecolor{dialinecolor}{rgb}{0.678431, 0.839216, 0.905882}
\pgfsetfillcolor{dialinecolor}
\pgfpathmoveto{\pgfpoint{-8.524393\du}{13.721394\du}}
\pgfpathlineto{\pgfpoint{-8.524393\du}{13.721394\du}}
\pgfpathlineto{\pgfpoint{-8.568204\du}{13.721394\du}}
\pgfpathlineto{\pgfpoint{-8.610921\du}{13.721759\du}}
\pgfpathlineto{\pgfpoint{-8.654367\du}{13.722855\du}}
\pgfpathlineto{\pgfpoint{-8.696353\du}{13.724315\du}}
\pgfpathlineto{\pgfpoint{-8.738705\du}{13.725775\du}}
\pgfpathlineto{\pgfpoint{-8.780326\du}{13.727601\du}}
\pgfpathlineto{\pgfpoint{-8.822312\du}{13.730157\du}}
\pgfpathlineto{\pgfpoint{-8.863568\du}{13.733077\du}}
\pgfpathlineto{\pgfpoint{-8.904458\du}{13.735998\du}}
\pgfpathlineto{\pgfpoint{-8.945349\du}{13.738919\du}}
\pgfpathlineto{\pgfpoint{-8.985145\du}{13.742935\du}}
\pgfpathlineto{\pgfpoint{-9.025305\du}{13.746951\du}}
\pgfpathlineto{\pgfpoint{-9.064371\du}{13.751697\du}}
\pgfpathlineto{\pgfpoint{-9.103071\du}{13.756444\du}}
\pgfpathlineto{\pgfpoint{-9.141771\du}{13.761190\du}}
\pgfpathlineto{\pgfpoint{-9.179741\du}{13.766301\du}}
\pgfpathlineto{\pgfpoint{-9.217346\du}{13.772143\du}}
\pgfpathlineto{\pgfpoint{-9.254586\du}{13.777984\du}}
\pgfpathlineto{\pgfpoint{-9.291096\du}{13.784556\du}}
\pgfpathlineto{\pgfpoint{-9.327240\du}{13.790763\du}}
\pgfpathlineto{\pgfpoint{-9.363020\du}{13.798065\du}}
\pgfpathlineto{\pgfpoint{-9.397704\du}{13.805001\du}}
\pgfpathlineto{\pgfpoint{-9.432023\du}{13.811938\du}}
\pgfpathlineto{\pgfpoint{-9.466707\du}{13.819970\du}}
\pgfpathlineto{\pgfpoint{-9.499201\du}{13.827637\du}}
\pgfpathlineto{\pgfpoint{-9.532425\du}{13.836035\du}}
\pgfpathlineto{\pgfpoint{-9.564188\du}{13.844797\du}}
\pgfpathlineto{\pgfpoint{-9.595952\du}{13.853559\du}}
\pgfpathlineto{\pgfpoint{-9.626620\du}{13.862322\du}}
\pgfpathlineto{\pgfpoint{-9.657653\du}{13.871814\du}}
\pgfpathlineto{\pgfpoint{-9.672257\du}{13.876195\du}}
\pgfpathlineto{\pgfpoint{-9.686131\du}{13.880942\du}}
\pgfpathlineto{\pgfpoint{-9.701830\du}{13.885688\du}}
\pgfpathlineto{\pgfpoint{-9.715703\du}{13.890799\du}}
\pgfpathlineto{\pgfpoint{-9.730307\du}{13.895545\du}}
\pgfpathlineto{\pgfpoint{-9.744181\du}{13.901022\du}}
\pgfpathlineto{\pgfpoint{-9.758420\du}{13.905403\du}}
\pgfpathlineto{\pgfpoint{-9.771563\du}{13.910879\du}}
\pgfpathlineto{\pgfpoint{-9.785802\du}{13.915991\du}}
\pgfpathlineto{\pgfpoint{-9.798945\du}{13.921467\du}}
\pgfpathlineto{\pgfpoint{-9.811359\du}{13.926579\du}}
\pgfpathlineto{\pgfpoint{-9.825232\du}{13.932055\du}}
\pgfpathlineto{\pgfpoint{-9.838011\du}{13.937166\du}}
\pgfpathlineto{\pgfpoint{-9.850059\du}{13.942278\du}}
\pgfpathlineto{\pgfpoint{-9.863202\du}{13.948119\du}}
\pgfpathlineto{\pgfpoint{-9.875251\du}{13.954326\du}}
\pgfpathlineto{\pgfpoint{-9.887664\du}{13.959437\du}}
\pgfpathlineto{\pgfpoint{-9.899347\du}{13.965279\du}}
\pgfpathlineto{\pgfpoint{-9.910665\du}{13.971120\du}}
\pgfpathlineto{\pgfpoint{-9.922348\du}{13.976962\du}}
\pgfpathlineto{\pgfpoint{-9.933666\du}{13.982804\du}}
\pgfpathlineto{\pgfpoint{-9.945349\du}{13.987915\du}}
\pgfpathlineto{\pgfpoint{-9.956302\du}{13.994487\du}}
\pgfpathlineto{\pgfpoint{-9.966525\du}{14.000328\du}}
\pgfpathlineto{\pgfpoint{-9.977113\du}{14.006170\du}}
\pgfpathlineto{\pgfpoint{-9.987335\du}{14.012011\du}}
\pgfpathlineto{\pgfpoint{-9.997558\du}{14.018583\du}}
\pgfpathlineto{\pgfpoint{-10.007416\du}{14.024790\du}}
\pgfpathlineto{\pgfpoint{-10.016908\du}{14.030631\du}}
\pgfpathlineto{\pgfpoint{-10.026401\du}{14.037203\du}}
\pgfpathlineto{\pgfpoint{-10.035528\du}{14.043775\du}}
\pgfpathlineto{\pgfpoint{-10.045386\du}{14.049981\du}}
\pgfpathlineto{\pgfpoint{-10.053783\du}{14.056553\du}}
\pgfpathlineto{\pgfpoint{-10.062545\du}{14.063125\du}}
\pgfpathlineto{\pgfpoint{-10.070943\du}{14.069331\du}}
\pgfpathlineto{\pgfpoint{-10.079705\du}{14.075903\du}}
\pgfpathlineto{\pgfpoint{-10.087372\du}{14.082840\du}}
\pgfpathlineto{\pgfpoint{-10.094309\du}{14.089412\du}}
\pgfpathlineto{\pgfpoint{-10.103071\du}{14.095618\du}}
\pgfpathlineto{\pgfpoint{-10.110008\du}{14.102920\du}}
\pgfpathlineto{\pgfpoint{-10.116945\du}{14.109127\du}}
\pgfpathlineto{\pgfpoint{-10.123882\du}{14.116064\du}}
\pgfpathlineto{\pgfpoint{-10.130453\du}{14.123366\du}}
\pgfpathlineto{\pgfpoint{-10.137390\du}{14.129572\du}}
\pgfpathlineto{\pgfpoint{-10.143232\du}{14.136874\du}}
\pgfpathlineto{\pgfpoint{-10.149073\du}{14.143811\du}}
\pgfpathlineto{\pgfpoint{-10.154915\du}{14.151478\du}}
\pgfpathlineto{\pgfpoint{-10.160391\du}{14.158415\du}}
\pgfpathlineto{\pgfpoint{-10.165138\du}{14.165352\du}}
\pgfpathlineto{\pgfpoint{-10.170614\du}{14.172289\du}}
\pgfpathlineto{\pgfpoint{-10.174630\du}{14.179591\du}}
\pgfpathlineto{\pgfpoint{-10.179741\du}{14.186893\du}}
\pgfpathlineto{\pgfpoint{-10.183392\du}{14.194195\du}}
\pgfpathlineto{\pgfpoint{-10.187774\du}{14.201497\du}}
\pgfpathlineto{\pgfpoint{-10.191424\du}{14.209164\du}}
\pgfpathlineto{\pgfpoint{-10.194710\du}{14.216831\du}}
\pgfpathlineto{\pgfpoint{-10.198361\du}{14.223767\du}}
\pgfpathlineto{\pgfpoint{-10.201282\du}{14.231434\du}}
\pgfpathlineto{\pgfpoint{-10.203838\du}{14.239101\du}}
\pgfpathlineto{\pgfpoint{-10.206028\du}{14.246769\du}}
\pgfpathlineto{\pgfpoint{-10.208219\du}{14.254436\du}}
\pgfpathlineto{\pgfpoint{-10.210044\du}{14.261737\du}}
\pgfpathlineto{\pgfpoint{-10.211870\du}{14.269405\du}}
\pgfpathlineto{\pgfpoint{-10.212965\du}{14.277072\du}}
\pgfpathlineto{\pgfpoint{-10.214060\du}{14.285469\du}}
\pgfpathlineto{\pgfpoint{-10.214791\du}{14.292771\du}}
\pgfpathlineto{\pgfpoint{-10.215156\du}{14.300438\du}}
\pgfpathlineto{\pgfpoint{-10.215156\du}{14.308835\du}}
\pgfpathlineto{\pgfpoint{-10.194710\du}{14.308835\du}}
\pgfpathlineto{\pgfpoint{-10.194345\du}{14.301533\du}}
\pgfpathlineto{\pgfpoint{-10.194345\du}{14.294596\du}}
\pgfpathlineto{\pgfpoint{-10.193980\du}{14.287659\du}}
\pgfpathlineto{\pgfpoint{-10.192520\du}{14.279992\du}}
\pgfpathlineto{\pgfpoint{-10.191424\du}{14.273786\du}}
\pgfpathlineto{\pgfpoint{-10.190329\du}{14.266484\du}}
\pgfpathlineto{\pgfpoint{-10.188139\du}{14.259547\du}}
\pgfpathlineto{\pgfpoint{-10.186678\du}{14.252610\du}}
\pgfpathlineto{\pgfpoint{-10.184488\du}{14.245673\du}}
\pgfpathlineto{\pgfpoint{-10.182297\du}{14.238371\du}}
\pgfpathlineto{\pgfpoint{-10.179011\du}{14.232165\du}}
\pgfpathlineto{\pgfpoint{-10.176456\du}{14.225228\du}}
\pgfpathlineto{\pgfpoint{-10.173535\du}{14.217926\du}}
\pgfpathlineto{\pgfpoint{-10.169519\du}{14.210989\du}}
\pgfpathlineto{\pgfpoint{-10.165503\du}{14.204052\du}}
\pgfpathlineto{\pgfpoint{-10.162217\du}{14.197480\du}}
\pgfpathlineto{\pgfpoint{-10.158201\du}{14.190544\du}}
\pgfpathlineto{\pgfpoint{-10.153454\du}{14.183972\du}}
\pgfpathlineto{\pgfpoint{-10.149073\du}{14.177035\du}}
\pgfpathlineto{\pgfpoint{-10.143597\du}{14.170463\du}}
\pgfpathlineto{\pgfpoint{-10.139216\du}{14.164257\du}}
\pgfpathlineto{\pgfpoint{-10.133009\du}{14.157320\du}}
\pgfpathlineto{\pgfpoint{-10.127898\du}{14.150748\du}}
\pgfpathlineto{\pgfpoint{-10.121691\du}{14.143811\du}}
\pgfpathlineto{\pgfpoint{-10.115849\du}{14.137239\du}}
\pgfpathlineto{\pgfpoint{-10.109278\du}{14.131033\du}}
\pgfpathlineto{\pgfpoint{-10.103071\du}{14.124461\du}}
\pgfpathlineto{\pgfpoint{-10.096134\du}{14.117524\du}}
\pgfpathlineto{\pgfpoint{-10.088832\du}{14.111683\du}}
\pgfpathlineto{\pgfpoint{-10.081895\du}{14.104381\du}}
\pgfpathlineto{\pgfpoint{-10.074228\du}{14.098174\du}}
\pgfpathlineto{\pgfpoint{-10.066196\du}{14.092333\du}}
\pgfpathlineto{\pgfpoint{-10.058894\du}{14.085761\du}}
\pgfpathlineto{\pgfpoint{-10.050497\du}{14.079189\du}}
\pgfpathlineto{\pgfpoint{-10.041735\du}{14.072982\du}}
\pgfpathlineto{\pgfpoint{-10.032607\du}{14.066411\du}}
\pgfpathlineto{\pgfpoint{-10.023845\du}{14.060569\du}}
\pgfpathlineto{\pgfpoint{-10.015448\du}{14.054363\du}}
\pgfpathlineto{\pgfpoint{-10.006320\du}{14.048521\du}}
\pgfpathlineto{\pgfpoint{-9.996463\du}{14.041949\du}}
\pgfpathlineto{\pgfpoint{-9.986970\du}{14.036108\du}}
\pgfpathlineto{\pgfpoint{-9.977113\du}{14.030266\du}}
\pgfpathlineto{\pgfpoint{-9.966525\du}{14.024425\du}}
\pgfpathlineto{\pgfpoint{-9.956667\du}{14.018583\du}}
\pgfpathlineto{\pgfpoint{-9.946079\du}{14.012011\du}}
\pgfpathlineto{\pgfpoint{-9.934761\du}{14.006900\du}}
\pgfpathlineto{\pgfpoint{-9.924904\du}{14.001058\du}}
\pgfpathlineto{\pgfpoint{-9.913221\du}{13.995217\du}}
\pgfpathlineto{\pgfpoint{-9.902268\du}{13.989375\du}}
\pgfpathlineto{\pgfpoint{-9.890585\du}{13.983534\du}}
\pgfpathlineto{\pgfpoint{-9.878902\du}{13.978057\du}}
\pgfpathlineto{\pgfpoint{-9.866123\du}{13.972946\du}}
\pgfpathlineto{\pgfpoint{-9.854805\du}{13.967104\du}}
\pgfpathlineto{\pgfpoint{-9.842757\du}{13.961628\du}}
\pgfpathlineto{\pgfpoint{-9.829614\du}{13.956517\du}}
\pgfpathlineto{\pgfpoint{-9.817200\du}{13.951040\du}}
\pgfpathlineto{\pgfpoint{-9.804422\du}{13.945929\du}}
\pgfpathlineto{\pgfpoint{-9.790913\du}{13.940087\du}}
\pgfpathlineto{\pgfpoint{-9.778135\du}{13.935341\du}}
\pgfpathlineto{\pgfpoint{-9.764626\du}{13.930230\du}}
\pgfpathlineto{\pgfpoint{-9.751118\du}{13.924753\du}}
\pgfpathlineto{\pgfpoint{-9.737244\du}{13.920372\du}}
\pgfpathlineto{\pgfpoint{-9.723370\du}{13.914896\du}}
\pgfpathlineto{\pgfpoint{-9.709132\du}{13.910149\du}}
\pgfpathlineto{\pgfpoint{-9.694893\du}{13.905403\du}}
\pgfpathlineto{\pgfpoint{-9.680654\du}{13.900292\du}}
\pgfpathlineto{\pgfpoint{-9.666415\du}{13.895545\du}}
\pgfpathlineto{\pgfpoint{-9.651081\du}{13.890799\du}}
\pgfpathlineto{\pgfpoint{-9.621508\du}{13.882037\du}}
\pgfpathlineto{\pgfpoint{-9.590110\du}{13.872909\du}}
\pgfpathlineto{\pgfpoint{-9.559077\du}{13.864512\du}}
\pgfpathlineto{\pgfpoint{-9.526583\du}{13.855750\du}}
\pgfpathlineto{\pgfpoint{-9.495185\du}{13.847718\du}}
\pgfpathlineto{\pgfpoint{-9.461961\du}{13.840051\du}}
\pgfpathlineto{\pgfpoint{-9.428007\du}{13.832384\du}}
\pgfpathlineto{\pgfpoint{-9.393688\du}{13.824717\du}}
\pgfpathlineto{\pgfpoint{-9.359004\du}{13.817780\du}}
\pgfpathlineto{\pgfpoint{-9.322859\du}{13.810843\du}}
\pgfpathlineto{\pgfpoint{-9.287810\du}{13.805001\du}}
\pgfpathlineto{\pgfpoint{-9.250570\du}{13.798430\du}}
\pgfpathlineto{\pgfpoint{-9.214060\du}{13.792588\du}}
\pgfpathlineto{\pgfpoint{-9.176821\du}{13.786747\du}}
\pgfpathlineto{\pgfpoint{-9.139216\du}{13.781635\du}}
\pgfpathlineto{\pgfpoint{-9.100880\du}{13.776159\du}}
\pgfpathlineto{\pgfpoint{-9.061815\du}{13.771413\du}}
\pgfpathlineto{\pgfpoint{-9.022385\du}{13.767396\du}}
\pgfpathlineto{\pgfpoint{-8.982954\du}{13.763380\du}}
\pgfpathlineto{\pgfpoint{-8.943159\du}{13.759729\du}}
\pgfpathlineto{\pgfpoint{-8.902998\du}{13.756444\du}}
\pgfpathlineto{\pgfpoint{-8.862107\du}{13.753523\du}}
\pgfpathlineto{\pgfpoint{-8.820851\du}{13.750602\du}}
\pgfpathlineto{\pgfpoint{-8.779960\du}{13.748046\du}}
\pgfpathlineto{\pgfpoint{-8.737974\du}{13.745856\du}}
\pgfpathlineto{\pgfpoint{-8.695623\du}{13.744760\du}}
\pgfpathlineto{\pgfpoint{-8.652907\du}{13.742935\du}}
\pgfpathlineto{\pgfpoint{-8.610556\du}{13.742205\du}}
\pgfpathlineto{\pgfpoint{-8.567109\du}{13.741840\du}}
\pgfpathlineto{\pgfpoint{-8.524393\du}{13.741840\du}}
\pgfpathlineto{\pgfpoint{-8.524393\du}{13.741840\du}}
\pgfpathlineto{\pgfpoint{-8.524393\du}{13.741840\du}}
\pgfpathlineto{\pgfpoint{-8.523297\du}{13.741110\du}}
\pgfpathlineto{\pgfpoint{-8.522202\du}{13.741110\du}}
\pgfpathlineto{\pgfpoint{-8.520742\du}{13.741110\du}}
\pgfpathlineto{\pgfpoint{-8.519646\du}{13.740744\du}}
\pgfpathlineto{\pgfpoint{-8.519281\du}{13.740014\du}}
\pgfpathlineto{\pgfpoint{-8.518186\du}{13.740014\du}}
\pgfpathlineto{\pgfpoint{-8.517456\du}{13.738919\du}}
\pgfpathlineto{\pgfpoint{-8.516361\du}{13.738189\du}}
\pgfpathlineto{\pgfpoint{-8.515265\du}{13.737093\du}}
\pgfpathlineto{\pgfpoint{-8.514535\du}{13.735268\du}}
\pgfpathlineto{\pgfpoint{-8.514535\du}{13.733077\du}}
\pgfpathlineto{\pgfpoint{-8.513805\du}{13.731252\du}}
\pgfpathlineto{\pgfpoint{-8.514535\du}{13.729426\du}}
\pgfpathlineto{\pgfpoint{-8.514535\du}{13.727601\du}}
\pgfpathlineto{\pgfpoint{-8.515265\du}{13.725775\du}}
\pgfpathlineto{\pgfpoint{-8.516361\du}{13.724315\du}}
\pgfpathlineto{\pgfpoint{-8.517456\du}{13.723585\du}}
\pgfpathlineto{\pgfpoint{-8.518186\du}{13.722855\du}}
\pgfpathlineto{\pgfpoint{-8.519281\du}{13.722490\du}}
\pgfpathlineto{\pgfpoint{-8.519646\du}{13.721759\du}}
\pgfpathlineto{\pgfpoint{-8.520742\du}{13.721394\du}}
\pgfpathlineto{\pgfpoint{-8.522202\du}{13.721394\du}}
\pgfpathlineto{\pgfpoint{-8.523297\du}{13.721394\du}}
\pgfpathlineto{\pgfpoint{-8.524393\du}{13.721394\du}}
\pgfusepath{fill}
\pgfsetbuttcap
\pgfsetmiterjoin
\pgfsetdash{}{0pt}
\definecolor{dialinecolor}{rgb}{0.678431, 0.839216, 0.905882}
\pgfsetfillcolor{dialinecolor}
\pgfpathmoveto{\pgfpoint{-6.833995\du}{14.308835\du}}
\pgfpathlineto{\pgfpoint{-6.833995\du}{14.300438\du}}
\pgfpathlineto{\pgfpoint{-6.834725\du}{14.292771\du}}
\pgfpathlineto{\pgfpoint{-6.835455\du}{14.285469\du}}
\pgfpathlineto{\pgfpoint{-6.836185\du}{14.277072\du}}
\pgfpathlineto{\pgfpoint{-6.837281\du}{14.269405\du}}
\pgfpathlineto{\pgfpoint{-6.839471\du}{14.261737\du}}
\pgfpathlineto{\pgfpoint{-6.840566\du}{14.254436\du}}
\pgfpathlineto{\pgfpoint{-6.843487\du}{14.246769\du}}
\pgfpathlineto{\pgfpoint{-6.845313\du}{14.239101\du}}
\pgfpathlineto{\pgfpoint{-6.848234\du}{14.231434\du}}
\pgfpathlineto{\pgfpoint{-6.850789\du}{14.223767\du}}
\pgfpathlineto{\pgfpoint{-6.854075\du}{14.216831\du}}
\pgfpathlineto{\pgfpoint{-6.858091\du}{14.209164\du}}
\pgfpathlineto{\pgfpoint{-6.861377\du}{14.201497\du}}
\pgfpathlineto{\pgfpoint{-6.865393\du}{14.194195\du}}
\pgfpathlineto{\pgfpoint{-6.869409\du}{14.186893\du}}
\pgfpathlineto{\pgfpoint{-6.873790\du}{14.179591\du}}
\pgfpathlineto{\pgfpoint{-6.878537\du}{14.172289\du}}
\pgfpathlineto{\pgfpoint{-6.884013\du}{14.165352\du}}
\pgfpathlineto{\pgfpoint{-6.888759\du}{14.158415\du}}
\pgfpathlineto{\pgfpoint{-6.894601\du}{14.150748\du}}
\pgfpathlineto{\pgfpoint{-6.900077\du}{14.143811\du}}
\pgfpathlineto{\pgfpoint{-6.905919\du}{14.136874\du}}
\pgfpathlineto{\pgfpoint{-6.912125\du}{14.129572\du}}
\pgfpathlineto{\pgfpoint{-6.918697\du}{14.123366\du}}
\pgfpathlineto{\pgfpoint{-6.925634\du}{14.116064\du}}
\pgfpathlineto{\pgfpoint{-6.932206\du}{14.109127\du}}
\pgfpathlineto{\pgfpoint{-6.939508\du}{14.102920\du}}
\pgfpathlineto{\pgfpoint{-6.946079\du}{14.095618\du}}
\pgfpathlineto{\pgfpoint{-6.954112\du}{14.089412\du}}
\pgfpathlineto{\pgfpoint{-6.962144\du}{14.082840\du}}
\pgfpathlineto{\pgfpoint{-6.969446\du}{14.075903\du}}
\pgfpathlineto{\pgfpoint{-6.978208\du}{14.069331\du}}
\pgfpathlineto{\pgfpoint{-6.986240\du}{14.063125\du}}
\pgfpathlineto{\pgfpoint{-6.995368\du}{14.056553\du}}
\pgfpathlineto{\pgfpoint{-7.004130\du}{14.049981\du}}
\pgfpathlineto{\pgfpoint{-7.013622\du}{14.043775\du}}
\pgfpathlineto{\pgfpoint{-7.022750\du}{14.037203\du}}
\pgfpathlineto{\pgfpoint{-7.031877\du}{14.030631\du}}
\pgfpathlineto{\pgfpoint{-7.041735\du}{14.024790\du}}
\pgfpathlineto{\pgfpoint{-7.051592\du}{14.018583\du}}
\pgfpathlineto{\pgfpoint{-7.061815\du}{14.012011\du}}
\pgfpathlineto{\pgfpoint{-7.072403\du}{14.006170\du}}
\pgfpathlineto{\pgfpoint{-7.082261\du}{14.000328\du}}
\pgfpathlineto{\pgfpoint{-7.092848\du}{13.994487\du}}
\pgfpathlineto{\pgfpoint{-7.104166\du}{13.987915\du}}
\pgfpathlineto{\pgfpoint{-7.115484\du}{13.982804\du}}
\pgfpathlineto{\pgfpoint{-7.127167\du}{13.976962\du}}
\pgfpathlineto{\pgfpoint{-7.138120\du}{13.971120\du}}
\pgfpathlineto{\pgfpoint{-7.150169\du}{13.965279\du}}
\pgfpathlineto{\pgfpoint{-7.161487\du}{13.959437\du}}
\pgfpathlineto{\pgfpoint{-7.173900\du}{13.954326\du}}
\pgfpathlineto{\pgfpoint{-7.186313\du}{13.948119\du}}
\pgfpathlineto{\pgfpoint{-7.199092\du}{13.942278\du}}
\pgfpathlineto{\pgfpoint{-7.211140\du}{13.937166\du}}
\pgfpathlineto{\pgfpoint{-7.223918\du}{13.932055\du}}
\pgfpathlineto{\pgfpoint{-7.237427\du}{13.926579\du}}
\pgfpathlineto{\pgfpoint{-7.250570\du}{13.921467\du}}
\pgfpathlineto{\pgfpoint{-7.263714\du}{13.915991\du}}
\pgfpathlineto{\pgfpoint{-7.277952\du}{13.910879\du}}
\pgfpathlineto{\pgfpoint{-7.291096\du}{13.905403\du}}
\pgfpathlineto{\pgfpoint{-7.305335\du}{13.901022\du}}
\pgfpathlineto{\pgfpoint{-7.319208\du}{13.895545\du}}
\pgfpathlineto{\pgfpoint{-7.333447\du}{13.890799\du}}
\pgfpathlineto{\pgfpoint{-7.347321\du}{13.885688\du}}
\pgfpathlineto{\pgfpoint{-7.362290\du}{13.880942\du}}
\pgfpathlineto{\pgfpoint{-7.377259\du}{13.876195\du}}
\pgfpathlineto{\pgfpoint{-7.391497\du}{13.871814\du}}
\pgfpathlineto{\pgfpoint{-7.422166\du}{13.862322\du}}
\pgfpathlineto{\pgfpoint{-7.452834\du}{13.853559\du}}
\pgfpathlineto{\pgfpoint{-7.484232\du}{13.844797\du}}
\pgfpathlineto{\pgfpoint{-7.515995\du}{13.836035\du}}
\pgfpathlineto{\pgfpoint{-7.548854\du}{13.827637\du}}
\pgfpathlineto{\pgfpoint{-7.582443\du}{13.819970\du}}
\pgfpathlineto{\pgfpoint{-7.616397\du}{13.811938\du}}
\pgfpathlineto{\pgfpoint{-7.651081\du}{13.805001\du}}
\pgfpathlineto{\pgfpoint{-7.685765\du}{13.798065\du}}
\pgfpathlineto{\pgfpoint{-7.721180\du}{13.790763\du}}
\pgfpathlineto{\pgfpoint{-7.757690\du}{13.784556\du}}
\pgfpathlineto{\pgfpoint{-7.794199\du}{13.777984\du}}
\pgfpathlineto{\pgfpoint{-7.831074\du}{13.772143\du}}
\pgfpathlineto{\pgfpoint{-7.868314\du}{13.766301\du}}
\pgfpathlineto{\pgfpoint{-7.907379\du}{13.761190\du}}
\pgfpathlineto{\pgfpoint{-7.945349\du}{13.756444\du}}
\pgfpathlineto{\pgfpoint{-7.984780\du}{13.751697\du}}
\pgfpathlineto{\pgfpoint{-8.023480\du}{13.746951\du}}
\pgfpathlineto{\pgfpoint{-8.063641\du}{13.742935\du}}
\pgfpathlineto{\pgfpoint{-8.103436\du}{13.738919\du}}
\pgfpathlineto{\pgfpoint{-8.144327\du}{13.735998\du}}
\pgfpathlineto{\pgfpoint{-8.184853\du}{13.733077\du}}
\pgfpathlineto{\pgfpoint{-8.226474\du}{13.730157\du}}
\pgfpathlineto{\pgfpoint{-8.268095\du}{13.727601\du}}
\pgfpathlineto{\pgfpoint{-8.309716\du}{13.725775\du}}
\pgfpathlineto{\pgfpoint{-8.352432\du}{13.724315\du}}
\pgfpathlineto{\pgfpoint{-8.394418\du}{13.722855\du}}
\pgfpathlineto{\pgfpoint{-8.437500\du}{13.721759\du}}
\pgfpathlineto{\pgfpoint{-8.480946\du}{13.721394\du}}
\pgfpathlineto{\pgfpoint{-8.524393\du}{13.721394\du}}
\pgfpathlineto{\pgfpoint{-8.524393\du}{13.741840\du}}
\pgfpathlineto{\pgfpoint{-8.481311\du}{13.741840\du}}
\pgfpathlineto{\pgfpoint{-8.437865\du}{13.742205\du}}
\pgfpathlineto{\pgfpoint{-8.395879\du}{13.742935\du}}
\pgfpathlineto{\pgfpoint{-8.353162\du}{13.744760\du}}
\pgfpathlineto{\pgfpoint{-8.310446\du}{13.745856\du}}
\pgfpathlineto{\pgfpoint{-8.268825\du}{13.748046\du}}
\pgfpathlineto{\pgfpoint{-8.227569\du}{13.750602\du}}
\pgfpathlineto{\pgfpoint{-8.186678\du}{13.753523\du}}
\pgfpathlineto{\pgfpoint{-8.145422\du}{13.756444\du}}
\pgfpathlineto{\pgfpoint{-8.105262\du}{13.759729\du}}
\pgfpathlineto{\pgfpoint{-8.065101\du}{13.763380\du}}
\pgfpathlineto{\pgfpoint{-8.026036\du}{13.767396\du}}
\pgfpathlineto{\pgfpoint{-7.986605\du}{13.771413\du}}
\pgfpathlineto{\pgfpoint{-7.947905\du}{13.776159\du}}
\pgfpathlineto{\pgfpoint{-7.909205\du}{13.781635\du}}
\pgfpathlineto{\pgfpoint{-7.871600\du}{13.786747\du}}
\pgfpathlineto{\pgfpoint{-7.834360\du}{13.792588\du}}
\pgfpathlineto{\pgfpoint{-7.797850\du}{13.798430\du}}
\pgfpathlineto{\pgfpoint{-7.760975\du}{13.805001\du}}
\pgfpathlineto{\pgfpoint{-7.725561\du}{13.810843\du}}
\pgfpathlineto{\pgfpoint{-7.689782\du}{13.817780\du}}
\pgfpathlineto{\pgfpoint{-7.655097\du}{13.824717\du}}
\pgfpathlineto{\pgfpoint{-7.621143\du}{13.832384\du}}
\pgfpathlineto{\pgfpoint{-7.586824\du}{13.840051\du}}
\pgfpathlineto{\pgfpoint{-7.553600\du}{13.847718\du}}
\pgfpathlineto{\pgfpoint{-7.521472\du}{13.855750\du}}
\pgfpathlineto{\pgfpoint{-7.489343\du}{13.864512\du}}
\pgfpathlineto{\pgfpoint{-7.457945\du}{13.872909\du}}
\pgfpathlineto{\pgfpoint{-7.427277\du}{13.882037\du}}
\pgfpathlineto{\pgfpoint{-7.397339\du}{13.890799\du}}
\pgfpathlineto{\pgfpoint{-7.383100\du}{13.895545\du}}
\pgfpathlineto{\pgfpoint{-7.368496\du}{13.900292\du}}
\pgfpathlineto{\pgfpoint{-7.354623\du}{13.905403\du}}
\pgfpathlineto{\pgfpoint{-7.340019\du}{13.910149\du}}
\pgfpathlineto{\pgfpoint{-7.325780\du}{13.914896\du}}
\pgfpathlineto{\pgfpoint{-7.311906\du}{13.920372\du}}
\pgfpathlineto{\pgfpoint{-7.297668\du}{13.924753\du}}
\pgfpathlineto{\pgfpoint{-7.284524\du}{13.930230\du}}
\pgfpathlineto{\pgfpoint{-7.271016\du}{13.935341\du}}
\pgfpathlineto{\pgfpoint{-7.257872\du}{13.940087\du}}
\pgfpathlineto{\pgfpoint{-7.245094\du}{13.945929\du}}
\pgfpathlineto{\pgfpoint{-7.232315\du}{13.951040\du}}
\pgfpathlineto{\pgfpoint{-7.219172\du}{13.956517\du}}
\pgfpathlineto{\pgfpoint{-7.207124\du}{13.961628\du}}
\pgfpathlineto{\pgfpoint{-7.194710\du}{13.967104\du}}
\pgfpathlineto{\pgfpoint{-7.182662\du}{13.972946\du}}
\pgfpathlineto{\pgfpoint{-7.170249\du}{13.978057\du}}
\pgfpathlineto{\pgfpoint{-7.158931\du}{13.983534\du}}
\pgfpathlineto{\pgfpoint{-7.146883\du}{13.989375\du}}
\pgfpathlineto{\pgfpoint{-7.136295\du}{13.995217\du}}
\pgfpathlineto{\pgfpoint{-7.124247\du}{14.001058\du}}
\pgfpathlineto{\pgfpoint{-7.114024\du}{14.006900\du}}
\pgfpathlineto{\pgfpoint{-7.103071\du}{14.012011\du}}
\pgfpathlineto{\pgfpoint{-7.092118\du}{14.018583\du}}
\pgfpathlineto{\pgfpoint{-7.082261\du}{14.024425\du}}
\pgfpathlineto{\pgfpoint{-7.072403\du}{14.030266\du}}
\pgfpathlineto{\pgfpoint{-7.062180\du}{14.036108\du}}
\pgfpathlineto{\pgfpoint{-7.053053\du}{14.041949\du}}
\pgfpathlineto{\pgfpoint{-7.042830\du}{14.048521\du}}
\pgfpathlineto{\pgfpoint{-7.033338\du}{14.054363\du}}
\pgfpathlineto{\pgfpoint{-7.024575\du}{14.060569\du}}
\pgfpathlineto{\pgfpoint{-7.016543\du}{14.066411\du}}
\pgfpathlineto{\pgfpoint{-7.007781\du}{14.072982\du}}
\pgfpathlineto{\pgfpoint{-6.999384\du}{14.079189\du}}
\pgfpathlineto{\pgfpoint{-6.990621\du}{14.085761\du}}
\pgfpathlineto{\pgfpoint{-6.982954\du}{14.092333\du}}
\pgfpathlineto{\pgfpoint{-6.974922\du}{14.098174\du}}
\pgfpathlineto{\pgfpoint{-6.967620\du}{14.104381\du}}
\pgfpathlineto{\pgfpoint{-6.959953\du}{14.111683\du}}
\pgfpathlineto{\pgfpoint{-6.953381\du}{14.117524\du}}
\pgfpathlineto{\pgfpoint{-6.946079\du}{14.124461\du}}
\pgfpathlineto{\pgfpoint{-6.940238\du}{14.131033\du}}
\pgfpathlineto{\pgfpoint{-6.933301\du}{14.137239\du}}
\pgfpathlineto{\pgfpoint{-6.926729\du}{14.143811\du}}
\pgfpathlineto{\pgfpoint{-6.921618\du}{14.150748\du}}
\pgfpathlineto{\pgfpoint{-6.916142\du}{14.157320\du}}
\pgfpathlineto{\pgfpoint{-6.909935\du}{14.164257\du}}
\pgfpathlineto{\pgfpoint{-6.905189\du}{14.170463\du}}
\pgfpathlineto{\pgfpoint{-6.900077\du}{14.177035\du}}
\pgfpathlineto{\pgfpoint{-6.895331\du}{14.183972\du}}
\pgfpathlineto{\pgfpoint{-6.890950\du}{14.190544\du}}
\pgfpathlineto{\pgfpoint{-6.886569\du}{14.197480\du}}
\pgfpathlineto{\pgfpoint{-6.883648\du}{14.204052\du}}
\pgfpathlineto{\pgfpoint{-6.879632\du}{14.210989\du}}
\pgfpathlineto{\pgfpoint{-6.876711\du}{14.217926\du}}
\pgfpathlineto{\pgfpoint{-6.872695\du}{14.225228\du}}
\pgfpathlineto{\pgfpoint{-6.870139\du}{14.231434\du}}
\pgfpathlineto{\pgfpoint{-6.867219\du}{14.238371\du}}
\pgfpathlineto{\pgfpoint{-6.864663\du}{14.245673\du}}
\pgfpathlineto{\pgfpoint{-6.862472\du}{14.252610\du}}
\pgfpathlineto{\pgfpoint{-6.861012\du}{14.259547\du}}
\pgfpathlineto{\pgfpoint{-6.858821\du}{14.266484\du}}
\pgfpathlineto{\pgfpoint{-6.858091\du}{14.273786\du}}
\pgfpathlineto{\pgfpoint{-6.856996\du}{14.279992\du}}
\pgfpathlineto{\pgfpoint{-6.855170\du}{14.287659\du}}
\pgfpathlineto{\pgfpoint{-6.854805\du}{14.294596\du}}
\pgfpathlineto{\pgfpoint{-6.854805\du}{14.301533\du}}
\pgfpathlineto{\pgfpoint{-6.854075\du}{14.308835\du}}
\pgfpathlineto{\pgfpoint{-6.833995\du}{14.308835\du}}
\pgfusepath{fill}
\pgfsetbuttcap
\pgfsetmiterjoin
\pgfsetdash{}{0pt}
\definecolor{dialinecolor}{rgb}{0.027451, 0.486275, 0.682353}
\pgfsetfillcolor{dialinecolor}
\pgfpathmoveto{\pgfpoint{-10.210044\du}{13.499415\du}}
\pgfpathlineto{\pgfpoint{-10.210044\du}{14.323804\du}}
\pgfpathlineto{\pgfpoint{-6.844583\du}{14.323804\du}}
\pgfpathlineto{\pgfpoint{-6.843852\du}{13.500146\du}}
\pgfpathlineto{\pgfpoint{-10.210044\du}{13.499415\du}}
\pgfusepath{fill}
\pgfsetbuttcap
\pgfsetmiterjoin
\pgfsetdash{}{0pt}
\definecolor{dialinecolor}{rgb}{0.235294, 0.686275, 0.894118}
\pgfsetfillcolor{dialinecolor}
\pgfpathmoveto{\pgfpoint{-6.844583\du}{13.483716\du}}
\pgfpathlineto{\pgfpoint{-6.846043\du}{13.513654\du}}
\pgfpathlineto{\pgfpoint{-6.853345\du}{13.542862\du}}
\pgfpathlineto{\pgfpoint{-6.863568\du}{13.572070\du}}
\pgfpathlineto{\pgfpoint{-6.878171\du}{13.600182\du}}
\pgfpathlineto{\pgfpoint{-6.897522\du}{13.628295\du}}
\pgfpathlineto{\pgfpoint{-6.919427\du}{13.655677\du}}
\pgfpathlineto{\pgfpoint{-6.946079\du}{13.681964\du}}
\pgfpathlineto{\pgfpoint{-6.976748\du}{13.708251\du}}
\pgfpathlineto{\pgfpoint{-7.009606\du}{13.734173\du}}
\pgfpathlineto{\pgfpoint{-7.046846\du}{13.759364\du}}
\pgfpathlineto{\pgfpoint{-7.087737\du}{13.783461\du}}
\pgfpathlineto{\pgfpoint{-7.131549\du}{13.806827\du}}
\pgfpathlineto{\pgfpoint{-7.177916\du}{13.829098\du}}
\pgfpathlineto{\pgfpoint{-7.228299\du}{13.851004\du}}
\pgfpathlineto{\pgfpoint{-7.281238\du}{13.872179\du}}
\pgfpathlineto{\pgfpoint{-7.336733\du}{13.892260\du}}
\pgfpathlineto{\pgfpoint{-7.394783\du}{13.910879\du}}
\pgfpathlineto{\pgfpoint{-7.455389\du}{13.929499\du}}
\pgfpathlineto{\pgfpoint{-7.519281\du}{13.946659\du}}
\pgfpathlineto{\pgfpoint{-7.584269\du}{13.962358\du}}
\pgfpathlineto{\pgfpoint{-7.652907\du}{13.977692\du}}
\pgfpathlineto{\pgfpoint{-7.723736\du}{13.991566\du}}
\pgfpathlineto{\pgfpoint{-7.795660\du}{14.004344\du}}
\pgfpathlineto{\pgfpoint{-7.870504\du}{14.015662\du}}
\pgfpathlineto{\pgfpoint{-7.946445\du}{14.026250\du}}
\pgfpathlineto{\pgfpoint{-8.024940\du}{14.035377\du}}
\pgfpathlineto{\pgfpoint{-8.104531\du}{14.043045\du}}
\pgfpathlineto{\pgfpoint{-8.185583\du}{14.049616\du}}
\pgfpathlineto{\pgfpoint{-8.268825\du}{14.054728\du}}
\pgfpathlineto{\pgfpoint{-8.352432\du}{14.058379\du}}
\pgfpathlineto{\pgfpoint{-8.437865\du}{14.060204\du}}
\pgfpathlineto{\pgfpoint{-8.524393\du}{14.061299\du}}
\pgfpathlineto{\pgfpoint{-8.610556\du}{14.060204\du}}
\pgfpathlineto{\pgfpoint{-8.696353\du}{14.058379\du}}
\pgfpathlineto{\pgfpoint{-8.779960\du}{14.054728\du}}
\pgfpathlineto{\pgfpoint{-8.862837\du}{14.049616\du}}
\pgfpathlineto{\pgfpoint{-8.944254\du}{14.043045\du}}
\pgfpathlineto{\pgfpoint{-9.023845\du}{14.035377\du}}
\pgfpathlineto{\pgfpoint{-9.101611\du}{14.026250\du}}
\pgfpathlineto{\pgfpoint{-9.178281\du}{14.015662\du}}
\pgfpathlineto{\pgfpoint{-9.252396\du}{14.004344\du}}
\pgfpathlineto{\pgfpoint{-9.325050\du}{13.991566\du}}
\pgfpathlineto{\pgfpoint{-9.395514\du}{13.977692\du}}
\pgfpathlineto{\pgfpoint{-9.464152\du}{13.962358\du}}
\pgfpathlineto{\pgfpoint{-9.529869\du}{13.946659\du}}
\pgfpathlineto{\pgfpoint{-9.593396\du}{13.929499\du}}
\pgfpathlineto{\pgfpoint{-9.654367\du}{13.910879\du}}
\pgfpathlineto{\pgfpoint{-9.712783\du}{13.892260\du}}
\pgfpathlineto{\pgfpoint{-9.767912\du}{13.872179\du}}
\pgfpathlineto{\pgfpoint{-9.820851\du}{13.851004\du}}
\pgfpathlineto{\pgfpoint{-9.870870\du}{13.829098\du}}
\pgfpathlineto{\pgfpoint{-9.917967\du}{13.806827\du}}
\pgfpathlineto{\pgfpoint{-9.961414\du}{13.783461\du}}
\pgfpathlineto{\pgfpoint{-10.002304\du}{13.759364\du}}
\pgfpathlineto{\pgfpoint{-10.039544\du}{13.734173\du}}
\pgfpathlineto{\pgfpoint{-10.072768\du}{13.708251\du}}
\pgfpathlineto{\pgfpoint{-10.103071\du}{13.681964\du}}
\pgfpathlineto{\pgfpoint{-10.129723\du}{13.655677\du}}
\pgfpathlineto{\pgfpoint{-10.151629\du}{13.628295\du}}
\pgfpathlineto{\pgfpoint{-10.170979\du}{13.600182\du}}
\pgfpathlineto{\pgfpoint{-10.185583\du}{13.572070\du}}
\pgfpathlineto{\pgfpoint{-10.196171\du}{13.542862\du}}
\pgfpathlineto{\pgfpoint{-10.203108\du}{13.513654\du}}
\pgfpathlineto{\pgfpoint{-10.204933\du}{13.483716\du}}
\pgfpathlineto{\pgfpoint{-10.203108\du}{13.454509\du}}
\pgfpathlineto{\pgfpoint{-10.196171\du}{13.424571\du}}
\pgfpathlineto{\pgfpoint{-10.185583\du}{13.396093\du}}
\pgfpathlineto{\pgfpoint{-10.170979\du}{13.367981\du}}
\pgfpathlineto{\pgfpoint{-10.151629\du}{13.339868\du}}
\pgfpathlineto{\pgfpoint{-10.129723\du}{13.312121\du}}
\pgfpathlineto{\pgfpoint{-10.103071\du}{13.285469\du}}
\pgfpathlineto{\pgfpoint{-10.072768\du}{13.259182\du}}
\pgfpathlineto{\pgfpoint{-10.039544\du}{13.233990\du}}
\pgfpathlineto{\pgfpoint{-10.002304\du}{13.208798\du}}
\pgfpathlineto{\pgfpoint{-9.961414\du}{13.184702\du}}
\pgfpathlineto{\pgfpoint{-9.917967\du}{13.161336\du}}
\pgfpathlineto{\pgfpoint{-9.870870\du}{13.138335\du}}
\pgfpathlineto{\pgfpoint{-9.820851\du}{13.116794\du}}
\pgfpathlineto{\pgfpoint{-9.767912\du}{13.095618\du}}
\pgfpathlineto{\pgfpoint{-9.712783\du}{13.075903\du}}
\pgfpathlineto{\pgfpoint{-9.654367\du}{13.056553\du}}
\pgfpathlineto{\pgfpoint{-9.593396\du}{13.038298\du}}
\pgfpathlineto{\pgfpoint{-9.529869\du}{13.021504\du}}
\pgfpathlineto{\pgfpoint{-9.464152\du}{13.005074\du}}
\pgfpathlineto{\pgfpoint{-9.395514\du}{12.989740\du}}
\pgfpathlineto{\pgfpoint{-9.325050\du}{12.976232\du}}
\pgfpathlineto{\pgfpoint{-9.252396\du}{12.963453\du}}
\pgfpathlineto{\pgfpoint{-9.178281\du}{12.951770\du}}
\pgfpathlineto{\pgfpoint{-9.101611\du}{12.941913\du}}
\pgfpathlineto{\pgfpoint{-9.023845\du}{12.932420\du}}
\pgfpathlineto{\pgfpoint{-8.944254\du}{12.924753\du}}
\pgfpathlineto{\pgfpoint{-8.862837\du}{12.918181\du}}
\pgfpathlineto{\pgfpoint{-8.779960\du}{12.913070\du}}
\pgfpathlineto{\pgfpoint{-8.696353\du}{12.909784\du}}
\pgfpathlineto{\pgfpoint{-8.610556\du}{12.907229\du}}
\pgfpathlineto{\pgfpoint{-8.524393\du}{12.906863\du}}
\pgfpathlineto{\pgfpoint{-8.437865\du}{12.907229\du}}
\pgfpathlineto{\pgfpoint{-8.352432\du}{12.909784\du}}
\pgfpathlineto{\pgfpoint{-8.268825\du}{12.913070\du}}
\pgfpathlineto{\pgfpoint{-8.185583\du}{12.918181\du}}
\pgfpathlineto{\pgfpoint{-8.104531\du}{12.924753\du}}
\pgfpathlineto{\pgfpoint{-8.024940\du}{12.932420\du}}
\pgfpathlineto{\pgfpoint{-7.946445\du}{12.941913\du}}
\pgfpathlineto{\pgfpoint{-7.870504\du}{12.951770\du}}
\pgfpathlineto{\pgfpoint{-7.795660\du}{12.963453\du}}
\pgfpathlineto{\pgfpoint{-7.723736\du}{12.976232\du}}
\pgfpathlineto{\pgfpoint{-7.652907\du}{12.989740\du}}
\pgfpathlineto{\pgfpoint{-7.584269\du}{13.005074\du}}
\pgfpathlineto{\pgfpoint{-7.519281\du}{13.021504\du}}
\pgfpathlineto{\pgfpoint{-7.455389\du}{13.038298\du}}
\pgfpathlineto{\pgfpoint{-7.394783\du}{13.056553\du}}
\pgfpathlineto{\pgfpoint{-7.336733\du}{13.075903\du}}
\pgfpathlineto{\pgfpoint{-7.281238\du}{13.095618\du}}
\pgfpathlineto{\pgfpoint{-7.228299\du}{13.116794\du}}
\pgfpathlineto{\pgfpoint{-7.177916\du}{13.138335\du}}
\pgfpathlineto{\pgfpoint{-7.131549\du}{13.161336\du}}
\pgfpathlineto{\pgfpoint{-7.087737\du}{13.184702\du}}
\pgfpathlineto{\pgfpoint{-7.046846\du}{13.208798\du}}
\pgfpathlineto{\pgfpoint{-7.009606\du}{13.233990\du}}
\pgfpathlineto{\pgfpoint{-6.976748\du}{13.259182\du}}
\pgfpathlineto{\pgfpoint{-6.946079\du}{13.285469\du}}
\pgfpathlineto{\pgfpoint{-6.919427\du}{13.312121\du}}
\pgfpathlineto{\pgfpoint{-6.897522\du}{13.339868\du}}
\pgfpathlineto{\pgfpoint{-6.878171\du}{13.367981\du}}
\pgfpathlineto{\pgfpoint{-6.863568\du}{13.396093\du}}
\pgfpathlineto{\pgfpoint{-6.853345\du}{13.424571\du}}
\pgfpathlineto{\pgfpoint{-6.846043\du}{13.454509\du}}
\pgfpathlineto{\pgfpoint{-6.844583\du}{13.483716\du}}
\pgfusepath{fill}
\pgfsetbuttcap
\pgfsetmiterjoin
\pgfsetdash{}{0pt}
\definecolor{dialinecolor}{rgb}{0.678431, 0.839216, 0.905882}
\pgfsetfillcolor{dialinecolor}
\pgfpathmoveto{\pgfpoint{-8.524393\du}{14.071157\du}}
\pgfpathlineto{\pgfpoint{-8.524393\du}{14.071157\du}}
\pgfpathlineto{\pgfpoint{-8.480946\du}{14.071157\du}}
\pgfpathlineto{\pgfpoint{-8.437500\du}{14.070427\du}}
\pgfpathlineto{\pgfpoint{-8.394418\du}{14.069331\du}}
\pgfpathlineto{\pgfpoint{-8.352432\du}{14.068236\du}}
\pgfpathlineto{\pgfpoint{-8.309716\du}{14.066411\du}}
\pgfpathlineto{\pgfpoint{-8.268095\du}{14.064585\du}}
\pgfpathlineto{\pgfpoint{-8.226474\du}{14.062395\du}}
\pgfpathlineto{\pgfpoint{-8.184853\du}{14.059474\du}}
\pgfpathlineto{\pgfpoint{-8.144327\du}{14.056553\du}}
\pgfpathlineto{\pgfpoint{-8.103436\du}{14.053632\du}}
\pgfpathlineto{\pgfpoint{-8.063641\du}{14.049616\du}}
\pgfpathlineto{\pgfpoint{-8.023480\du}{14.045600\du}}
\pgfpathlineto{\pgfpoint{-7.984780\du}{14.040854\du}}
\pgfpathlineto{\pgfpoint{-7.945349\du}{14.036108\du}}
\pgfpathlineto{\pgfpoint{-7.907379\du}{14.031361\du}}
\pgfpathlineto{\pgfpoint{-7.868314\du}{14.026250\du}}
\pgfpathlineto{\pgfpoint{-7.831074\du}{14.020409\du}}
\pgfpathlineto{\pgfpoint{-7.794199\du}{14.014567\du}}
\pgfpathlineto{\pgfpoint{-7.757690\du}{14.007995\du}}
\pgfpathlineto{\pgfpoint{-7.721180\du}{14.001423\du}}
\pgfpathlineto{\pgfpoint{-7.685765\du}{13.994487\du}}
\pgfpathlineto{\pgfpoint{-7.651081\du}{13.987550\du}}
\pgfpathlineto{\pgfpoint{-7.616397\du}{13.980613\du}}
\pgfpathlineto{\pgfpoint{-7.582443\du}{13.972946\du}}
\pgfpathlineto{\pgfpoint{-7.548854\du}{13.964549\du}}
\pgfpathlineto{\pgfpoint{-7.515995\du}{13.956517\du}}
\pgfpathlineto{\pgfpoint{-7.484232\du}{13.948119\du}}
\pgfpathlineto{\pgfpoint{-7.452834\du}{13.938992\du}}
\pgfpathlineto{\pgfpoint{-7.437500\du}{13.934976\du}}
\pgfpathlineto{\pgfpoint{-7.422166\du}{13.930230\du}}
\pgfpathlineto{\pgfpoint{-7.406101\du}{13.925483\du}}
\pgfpathlineto{\pgfpoint{-7.391497\du}{13.920737\du}}
\pgfpathlineto{\pgfpoint{-7.377259\du}{13.915991\du}}
\pgfpathlineto{\pgfpoint{-7.362290\du}{13.911610\du}}
\pgfpathlineto{\pgfpoint{-7.347321\du}{13.906863\du}}
\pgfpathlineto{\pgfpoint{-7.333447\du}{13.901387\du}}
\pgfpathlineto{\pgfpoint{-7.319208\du}{13.896641\du}}
\pgfpathlineto{\pgfpoint{-7.305335\du}{13.891529\du}}
\pgfpathlineto{\pgfpoint{-7.291096\du}{13.886783\du}}
\pgfpathlineto{\pgfpoint{-7.277952\du}{13.881672\du}}
\pgfpathlineto{\pgfpoint{-7.263714\du}{13.876195\du}}
\pgfpathlineto{\pgfpoint{-7.250570\du}{13.871084\du}}
\pgfpathlineto{\pgfpoint{-7.237427\du}{13.865973\du}}
\pgfpathlineto{\pgfpoint{-7.223918\du}{13.860496\du}}
\pgfpathlineto{\pgfpoint{-7.211140\du}{13.855385\du}}
\pgfpathlineto{\pgfpoint{-7.199092\du}{13.849908\du}}
\pgfpathlineto{\pgfpoint{-7.186313\du}{13.844067\du}}
\pgfpathlineto{\pgfpoint{-7.173900\du}{13.838225\du}}
\pgfpathlineto{\pgfpoint{-7.161487\du}{13.833114\du}}
\pgfpathlineto{\pgfpoint{-7.150169\du}{13.827272\du}}
\pgfpathlineto{\pgfpoint{-7.138120\du}{13.821431\du}}
\pgfpathlineto{\pgfpoint{-7.127167\du}{13.815589\du}}
\pgfpathlineto{\pgfpoint{-7.115484\du}{13.810113\du}}
\pgfpathlineto{\pgfpoint{-7.104166\du}{13.804271\du}}
\pgfpathlineto{\pgfpoint{-7.092848\du}{13.798065\du}}
\pgfpathlineto{\pgfpoint{-7.082261\du}{13.792223\du}}
\pgfpathlineto{\pgfpoint{-7.072403\du}{13.786382\du}}
\pgfpathlineto{\pgfpoint{-7.061815\du}{13.780540\du}}
\pgfpathlineto{\pgfpoint{-7.051592\du}{13.773968\du}}
\pgfpathlineto{\pgfpoint{-7.041735\du}{13.767396\du}}
\pgfpathlineto{\pgfpoint{-7.031877\du}{13.761555\du}}
\pgfpathlineto{\pgfpoint{-7.022750\du}{13.755348\du}}
\pgfpathlineto{\pgfpoint{-7.013622\du}{13.748777\du}}
\pgfpathlineto{\pgfpoint{-7.004130\du}{13.742205\du}}
\pgfpathlineto{\pgfpoint{-6.995368\du}{13.735998\du}}
\pgfpathlineto{\pgfpoint{-6.986240\du}{13.729426\du}}
\pgfpathlineto{\pgfpoint{-6.978208\du}{13.722855\du}}
\pgfpathlineto{\pgfpoint{-6.969446\du}{13.716648\du}}
\pgfpathlineto{\pgfpoint{-6.962144\du}{13.710076\du}}
\pgfpathlineto{\pgfpoint{-6.954112\du}{13.703139\du}}
\pgfpathlineto{\pgfpoint{-6.946079\du}{13.696568\du}}
\pgfpathlineto{\pgfpoint{-6.939508\du}{13.689631\du}}
\pgfpathlineto{\pgfpoint{-6.932206\du}{13.683059\du}}
\pgfpathlineto{\pgfpoint{-6.925634\du}{13.676122\du}}
\pgfpathlineto{\pgfpoint{-6.918697\du}{13.669185\du}}
\pgfpathlineto{\pgfpoint{-6.912125\du}{13.662614\du}}
\pgfpathlineto{\pgfpoint{-6.905919\du}{13.655677\du}}
\pgfpathlineto{\pgfpoint{-6.900077\du}{13.648740\du}}
\pgfpathlineto{\pgfpoint{-6.894601\du}{13.641803\du}}
\pgfpathlineto{\pgfpoint{-6.888759\du}{13.634136\du}}
\pgfpathlineto{\pgfpoint{-6.884013\du}{13.627199\du}}
\pgfpathlineto{\pgfpoint{-6.878537\du}{13.619897\du}}
\pgfpathlineto{\pgfpoint{-6.873790\du}{13.612961\du}}
\pgfpathlineto{\pgfpoint{-6.869409\du}{13.605294\du}}
\pgfpathlineto{\pgfpoint{-6.865393\du}{13.598357\du}}
\pgfpathlineto{\pgfpoint{-6.861377\du}{13.590690\du}}
\pgfpathlineto{\pgfpoint{-6.858091\du}{13.583023\du}}
\pgfpathlineto{\pgfpoint{-6.854075\du}{13.575721\du}}
\pgfpathlineto{\pgfpoint{-6.850789\du}{13.568419\du}}
\pgfpathlineto{\pgfpoint{-6.848234\du}{13.561117\du}}
\pgfpathlineto{\pgfpoint{-6.845313\du}{13.553450\du}}
\pgfpathlineto{\pgfpoint{-6.843487\du}{13.545783\du}}
\pgfpathlineto{\pgfpoint{-6.840566\du}{13.538116\du}}
\pgfpathlineto{\pgfpoint{-6.839471\du}{13.530449\du}}
\pgfpathlineto{\pgfpoint{-6.837281\du}{13.522782\du}}
\pgfpathlineto{\pgfpoint{-6.836185\du}{13.515480\du}}
\pgfpathlineto{\pgfpoint{-6.835455\du}{13.507082\du}}
\pgfpathlineto{\pgfpoint{-6.834725\du}{13.499415\du}}
\pgfpathlineto{\pgfpoint{-6.833995\du}{13.491748\du}}
\pgfpathlineto{\pgfpoint{-6.833995\du}{13.483716\du}}
\pgfpathlineto{\pgfpoint{-6.854075\du}{13.483716\du}}
\pgfpathlineto{\pgfpoint{-6.854805\du}{13.490653\du}}
\pgfpathlineto{\pgfpoint{-6.854805\du}{13.498320\du}}
\pgfpathlineto{\pgfpoint{-6.855170\du}{13.505257\du}}
\pgfpathlineto{\pgfpoint{-6.856996\du}{13.512194\du}}
\pgfpathlineto{\pgfpoint{-6.858091\du}{13.519496\du}}
\pgfpathlineto{\pgfpoint{-6.858821\du}{13.525702\du}}
\pgfpathlineto{\pgfpoint{-6.861012\du}{13.533004\du}}
\pgfpathlineto{\pgfpoint{-6.862472\du}{13.539941\du}}
\pgfpathlineto{\pgfpoint{-6.864663\du}{13.546878\du}}
\pgfpathlineto{\pgfpoint{-6.867219\du}{13.553815\du}}
\pgfpathlineto{\pgfpoint{-6.870139\du}{13.561117\du}}
\pgfpathlineto{\pgfpoint{-6.872695\du}{13.567323\du}}
\pgfpathlineto{\pgfpoint{-6.876711\du}{13.574260\du}}
\pgfpathlineto{\pgfpoint{-6.879632\du}{13.581562\du}}
\pgfpathlineto{\pgfpoint{-6.883648\du}{13.588499\du}}
\pgfpathlineto{\pgfpoint{-6.886569\du}{13.594706\du}}
\pgfpathlineto{\pgfpoint{-6.890950\du}{13.602008\du}}
\pgfpathlineto{\pgfpoint{-6.895331\du}{13.608214\du}}
\pgfpathlineto{\pgfpoint{-6.900077\du}{13.615516\du}}
\pgfpathlineto{\pgfpoint{-6.905189\du}{13.621723\du}}
\pgfpathlineto{\pgfpoint{-6.909935\du}{13.628660\du}}
\pgfpathlineto{\pgfpoint{-6.916142\du}{13.635231\du}}
\pgfpathlineto{\pgfpoint{-6.921618\du}{13.641803\du}}
\pgfpathlineto{\pgfpoint{-6.926729\du}{13.648740\du}}
\pgfpathlineto{\pgfpoint{-6.933301\du}{13.655312\du}}
\pgfpathlineto{\pgfpoint{-6.940238\du}{13.661518\du}}
\pgfpathlineto{\pgfpoint{-6.946079\du}{13.668090\du}}
\pgfpathlineto{\pgfpoint{-6.953381\du}{13.675027\du}}
\pgfpathlineto{\pgfpoint{-6.959953\du}{13.681599\du}}
\pgfpathlineto{\pgfpoint{-6.967620\du}{13.687805\du}}
\pgfpathlineto{\pgfpoint{-6.974922\du}{13.694377\du}}
\pgfpathlineto{\pgfpoint{-6.982954\du}{13.700219\du}}
\pgfpathlineto{\pgfpoint{-6.990621\du}{13.706790\du}}
\pgfpathlineto{\pgfpoint{-6.999384\du}{13.712997\du}}
\pgfpathlineto{\pgfpoint{-7.007781\du}{13.719569\du}}
\pgfpathlineto{\pgfpoint{-7.016543\du}{13.725775\du}}
\pgfpathlineto{\pgfpoint{-7.024575\du}{13.731617\du}}
\pgfpathlineto{\pgfpoint{-7.033338\du}{13.738189\du}}
\pgfpathlineto{\pgfpoint{-7.042830\du}{13.744030\du}}
\pgfpathlineto{\pgfpoint{-7.053053\du}{13.750602\du}}
\pgfpathlineto{\pgfpoint{-7.062180\du}{13.756444\du}}
\pgfpathlineto{\pgfpoint{-7.072403\du}{13.762285\du}}
\pgfpathlineto{\pgfpoint{-7.082261\du}{13.768127\du}}
\pgfpathlineto{\pgfpoint{-7.092118\du}{13.773968\du}}
\pgfpathlineto{\pgfpoint{-7.103071\du}{13.780540\du}}
\pgfpathlineto{\pgfpoint{-7.114024\du}{13.785651\du}}
\pgfpathlineto{\pgfpoint{-7.124247\du}{13.791493\du}}
\pgfpathlineto{\pgfpoint{-7.136295\du}{13.797334\du}}
\pgfpathlineto{\pgfpoint{-7.146883\du}{13.803176\du}}
\pgfpathlineto{\pgfpoint{-7.158931\du}{13.809018\du}}
\pgfpathlineto{\pgfpoint{-7.170249\du}{13.814129\du}}
\pgfpathlineto{\pgfpoint{-7.182662\du}{13.819605\du}}
\pgfpathlineto{\pgfpoint{-7.194710\du}{13.825447\du}}
\pgfpathlineto{\pgfpoint{-7.207124\du}{13.830558\du}}
\pgfpathlineto{\pgfpoint{-7.219172\du}{13.836035\du}}
\pgfpathlineto{\pgfpoint{-7.232315\du}{13.841146\du}}
\pgfpathlineto{\pgfpoint{-7.245094\du}{13.846622\du}}
\pgfpathlineto{\pgfpoint{-7.257872\du}{13.852464\du}}
\pgfpathlineto{\pgfpoint{-7.271016\du}{13.856845\du}}
\pgfpathlineto{\pgfpoint{-7.284524\du}{13.862322\du}}
\pgfpathlineto{\pgfpoint{-7.297668\du}{13.867433\du}}
\pgfpathlineto{\pgfpoint{-7.311906\du}{13.872179\du}}
\pgfpathlineto{\pgfpoint{-7.325780\du}{13.877656\du}}
\pgfpathlineto{\pgfpoint{-7.340019\du}{13.882037\du}}
\pgfpathlineto{\pgfpoint{-7.354623\du}{13.886783\du}}
\pgfpathlineto{\pgfpoint{-7.368496\du}{13.892260\du}}
\pgfpathlineto{\pgfpoint{-7.383100\du}{13.896641\du}}
\pgfpathlineto{\pgfpoint{-7.397339\du}{13.901387\du}}
\pgfpathlineto{\pgfpoint{-7.413038\du}{13.906133\du}}
\pgfpathlineto{\pgfpoint{-7.427277\du}{13.910149\du}}
\pgfpathlineto{\pgfpoint{-7.442976\du}{13.914896\du}}
\pgfpathlineto{\pgfpoint{-7.457945\du}{13.919642\du}}
\pgfpathlineto{\pgfpoint{-7.489343\du}{13.927674\du}}
\pgfpathlineto{\pgfpoint{-7.521472\du}{13.936436\du}}
\pgfpathlineto{\pgfpoint{-7.553600\du}{13.944833\du}}
\pgfpathlineto{\pgfpoint{-7.586824\du}{13.952501\du}}
\pgfpathlineto{\pgfpoint{-7.621143\du}{13.960168\du}}
\pgfpathlineto{\pgfpoint{-7.655097\du}{13.967469\du}}
\pgfpathlineto{\pgfpoint{-7.689782\du}{13.974771\du}}
\pgfpathlineto{\pgfpoint{-7.725561\du}{13.981708\du}}
\pgfpathlineto{\pgfpoint{-7.760975\du}{13.987915\du}}
\pgfpathlineto{\pgfpoint{-7.797850\du}{13.993756\du}}
\pgfpathlineto{\pgfpoint{-7.834360\du}{13.999963\du}}
\pgfpathlineto{\pgfpoint{-7.871600\du}{14.005805\du}}
\pgfpathlineto{\pgfpoint{-7.909205\du}{14.010916\du}}
\pgfpathlineto{\pgfpoint{-7.947905\du}{14.016027\du}}
\pgfpathlineto{\pgfpoint{-7.986605\du}{14.020774\du}}
\pgfpathlineto{\pgfpoint{-8.026036\du}{14.024790\du}}
\pgfpathlineto{\pgfpoint{-8.065101\du}{14.029171\du}}
\pgfpathlineto{\pgfpoint{-8.105262\du}{14.032457\du}}
\pgfpathlineto{\pgfpoint{-8.145422\du}{14.036108\du}}
\pgfpathlineto{\pgfpoint{-8.186678\du}{14.039028\du}}
\pgfpathlineto{\pgfpoint{-8.227569\du}{14.041949\du}}
\pgfpathlineto{\pgfpoint{-8.268825\du}{14.044140\du}}
\pgfpathlineto{\pgfpoint{-8.310446\du}{14.046695\du}}
\pgfpathlineto{\pgfpoint{-8.353162\du}{14.047791\du}}
\pgfpathlineto{\pgfpoint{-8.395879\du}{14.049616\du}}
\pgfpathlineto{\pgfpoint{-8.437865\du}{14.049981\du}}
\pgfpathlineto{\pgfpoint{-8.481311\du}{14.050712\du}}
\pgfpathlineto{\pgfpoint{-8.524393\du}{14.050712\du}}
\pgfpathlineto{\pgfpoint{-8.524393\du}{14.050712\du}}
\pgfpathlineto{\pgfpoint{-8.524393\du}{14.050712\du}}
\pgfpathlineto{\pgfpoint{-8.525123\du}{14.051442\du}}
\pgfpathlineto{\pgfpoint{-8.526948\du}{14.051442\du}}
\pgfpathlineto{\pgfpoint{-8.528044\du}{14.051442\du}}
\pgfpathlineto{\pgfpoint{-8.528774\du}{14.051807\du}}
\pgfpathlineto{\pgfpoint{-8.529139\du}{14.052537\du}}
\pgfpathlineto{\pgfpoint{-8.530599\du}{14.052537\du}}
\pgfpathlineto{\pgfpoint{-8.531330\du}{14.053632\du}}
\pgfpathlineto{\pgfpoint{-8.532060\du}{14.054363\du}}
\pgfpathlineto{\pgfpoint{-8.533155\du}{14.055458\du}}
\pgfpathlineto{\pgfpoint{-8.533885\du}{14.057283\du}}
\pgfpathlineto{\pgfpoint{-8.533885\du}{14.059474\du}}
\pgfpathlineto{\pgfpoint{-8.534615\du}{14.061299\du}}
\pgfpathlineto{\pgfpoint{-8.533885\du}{14.063125\du}}
\pgfpathlineto{\pgfpoint{-8.533885\du}{14.064585\du}}
\pgfpathlineto{\pgfpoint{-8.533155\du}{14.066411\du}}
\pgfpathlineto{\pgfpoint{-8.532060\du}{14.068236\du}}
\pgfpathlineto{\pgfpoint{-8.531330\du}{14.068966\du}}
\pgfpathlineto{\pgfpoint{-8.530599\du}{14.069331\du}}
\pgfpathlineto{\pgfpoint{-8.529139\du}{14.070062\du}}
\pgfpathlineto{\pgfpoint{-8.528774\du}{14.070427\du}}
\pgfpathlineto{\pgfpoint{-8.528044\du}{14.071157\du}}
\pgfpathlineto{\pgfpoint{-8.526948\du}{14.071157\du}}
\pgfpathlineto{\pgfpoint{-8.525123\du}{14.071157\du}}
\pgfpathlineto{\pgfpoint{-8.524393\du}{14.071157\du}}
\pgfusepath{fill}
\pgfsetbuttcap
\pgfsetmiterjoin
\pgfsetdash{}{0pt}
\definecolor{dialinecolor}{rgb}{0.678431, 0.839216, 0.905882}
\pgfsetfillcolor{dialinecolor}
\pgfpathmoveto{\pgfpoint{-10.215156\du}{13.483716\du}}
\pgfpathlineto{\pgfpoint{-10.215156\du}{13.483716\du}}
\pgfpathlineto{\pgfpoint{-10.215156\du}{13.491748\du}}
\pgfpathlineto{\pgfpoint{-10.214791\du}{13.499415\du}}
\pgfpathlineto{\pgfpoint{-10.214060\du}{13.507082\du}}
\pgfpathlineto{\pgfpoint{-10.212965\du}{13.515480\du}}
\pgfpathlineto{\pgfpoint{-10.211870\du}{13.522782\du}}
\pgfpathlineto{\pgfpoint{-10.210044\du}{13.530449\du}}
\pgfpathlineto{\pgfpoint{-10.208219\du}{13.538116\du}}
\pgfpathlineto{\pgfpoint{-10.206028\du}{13.545783\du}}
\pgfpathlineto{\pgfpoint{-10.203838\du}{13.553450\du}}
\pgfpathlineto{\pgfpoint{-10.201282\du}{13.561117\du}}
\pgfpathlineto{\pgfpoint{-10.198361\du}{13.568419\du}}
\pgfpathlineto{\pgfpoint{-10.194710\du}{13.575721\du}}
\pgfpathlineto{\pgfpoint{-10.191424\du}{13.583023\du}}
\pgfpathlineto{\pgfpoint{-10.187774\du}{13.590690\du}}
\pgfpathlineto{\pgfpoint{-10.183392\du}{13.598357\du}}
\pgfpathlineto{\pgfpoint{-10.179741\du}{13.605294\du}}
\pgfpathlineto{\pgfpoint{-10.174630\du}{13.612961\du}}
\pgfpathlineto{\pgfpoint{-10.170614\du}{13.619897\du}}
\pgfpathlineto{\pgfpoint{-10.165138\du}{13.627199\du}}
\pgfpathlineto{\pgfpoint{-10.160391\du}{13.634136\du}}
\pgfpathlineto{\pgfpoint{-10.154915\du}{13.641803\du}}
\pgfpathlineto{\pgfpoint{-10.149073\du}{13.648740\du}}
\pgfpathlineto{\pgfpoint{-10.143232\du}{13.655677\du}}
\pgfpathlineto{\pgfpoint{-10.137390\du}{13.662614\du}}
\pgfpathlineto{\pgfpoint{-10.130453\du}{13.669185\du}}
\pgfpathlineto{\pgfpoint{-10.123882\du}{13.676122\du}}
\pgfpathlineto{\pgfpoint{-10.116945\du}{13.683059\du}}
\pgfpathlineto{\pgfpoint{-10.110008\du}{13.689631\du}}
\pgfpathlineto{\pgfpoint{-10.103071\du}{13.696568\du}}
\pgfpathlineto{\pgfpoint{-10.094309\du}{13.703139\du}}
\pgfpathlineto{\pgfpoint{-10.087372\du}{13.710076\du}}
\pgfpathlineto{\pgfpoint{-10.079705\du}{13.716648\du}}
\pgfpathlineto{\pgfpoint{-10.070943\du}{13.722855\du}}
\pgfpathlineto{\pgfpoint{-10.062545\du}{13.729426\du}}
\pgfpathlineto{\pgfpoint{-10.053783\du}{13.735998\du}}
\pgfpathlineto{\pgfpoint{-10.045386\du}{13.742205\du}}
\pgfpathlineto{\pgfpoint{-10.035528\du}{13.748777\du}}
\pgfpathlineto{\pgfpoint{-10.026401\du}{13.755348\du}}
\pgfpathlineto{\pgfpoint{-10.016908\du}{13.761555\du}}
\pgfpathlineto{\pgfpoint{-10.007416\du}{13.767396\du}}
\pgfpathlineto{\pgfpoint{-9.997558\du}{13.773968\du}}
\pgfpathlineto{\pgfpoint{-9.987335\du}{13.780540\du}}
\pgfpathlineto{\pgfpoint{-9.977113\du}{13.786382\du}}
\pgfpathlineto{\pgfpoint{-9.966525\du}{13.792223\du}}
\pgfpathlineto{\pgfpoint{-9.956302\du}{13.798065\du}}
\pgfpathlineto{\pgfpoint{-9.945349\du}{13.804271\du}}
\pgfpathlineto{\pgfpoint{-9.933666\du}{13.810113\du}}
\pgfpathlineto{\pgfpoint{-9.922348\du}{13.815589\du}}
\pgfpathlineto{\pgfpoint{-9.910665\du}{13.821431\du}}
\pgfpathlineto{\pgfpoint{-9.899347\du}{13.827272\du}}
\pgfpathlineto{\pgfpoint{-9.887664\du}{13.833114\du}}
\pgfpathlineto{\pgfpoint{-9.875251\du}{13.838225\du}}
\pgfpathlineto{\pgfpoint{-9.863202\du}{13.844067\du}}
\pgfpathlineto{\pgfpoint{-9.850059\du}{13.849908\du}}
\pgfpathlineto{\pgfpoint{-9.838011\du}{13.855385\du}}
\pgfpathlineto{\pgfpoint{-9.825232\du}{13.860496\du}}
\pgfpathlineto{\pgfpoint{-9.811359\du}{13.865973\du}}
\pgfpathlineto{\pgfpoint{-9.798945\du}{13.871084\du}}
\pgfpathlineto{\pgfpoint{-9.785802\du}{13.876195\du}}
\pgfpathlineto{\pgfpoint{-9.771563\du}{13.881672\du}}
\pgfpathlineto{\pgfpoint{-9.758420\du}{13.886783\du}}
\pgfpathlineto{\pgfpoint{-9.744181\du}{13.891529\du}}
\pgfpathlineto{\pgfpoint{-9.730307\du}{13.896641\du}}
\pgfpathlineto{\pgfpoint{-9.715703\du}{13.901387\du}}
\pgfpathlineto{\pgfpoint{-9.701830\du}{13.906863\du}}
\pgfpathlineto{\pgfpoint{-9.686131\du}{13.911610\du}}
\pgfpathlineto{\pgfpoint{-9.672257\du}{13.915991\du}}
\pgfpathlineto{\pgfpoint{-9.657653\du}{13.920737\du}}
\pgfpathlineto{\pgfpoint{-9.642319\du}{13.925483\du}}
\pgfpathlineto{\pgfpoint{-9.626620\du}{13.930230\du}}
\pgfpathlineto{\pgfpoint{-9.611651\du}{13.934976\du}}
\pgfpathlineto{\pgfpoint{-9.595952\du}{13.938992\du}}
\pgfpathlineto{\pgfpoint{-9.564188\du}{13.948119\du}}
\pgfpathlineto{\pgfpoint{-9.532425\du}{13.956517\du}}
\pgfpathlineto{\pgfpoint{-9.499201\du}{13.964549\du}}
\pgfpathlineto{\pgfpoint{-9.466707\du}{13.972946\du}}
\pgfpathlineto{\pgfpoint{-9.432023\du}{13.980613\du}}
\pgfpathlineto{\pgfpoint{-9.397704\du}{13.987550\du}}
\pgfpathlineto{\pgfpoint{-9.363020\du}{13.994487\du}}
\pgfpathlineto{\pgfpoint{-9.327240\du}{14.001423\du}}
\pgfpathlineto{\pgfpoint{-9.291096\du}{14.007995\du}}
\pgfpathlineto{\pgfpoint{-9.254586\du}{14.014567\du}}
\pgfpathlineto{\pgfpoint{-9.217346\du}{14.020409\du}}
\pgfpathlineto{\pgfpoint{-9.179741\du}{14.026250\du}}
\pgfpathlineto{\pgfpoint{-9.141771\du}{14.031361\du}}
\pgfpathlineto{\pgfpoint{-9.103071\du}{14.036108\du}}
\pgfpathlineto{\pgfpoint{-9.064371\du}{14.040854\du}}
\pgfpathlineto{\pgfpoint{-9.025305\du}{14.045600\du}}
\pgfpathlineto{\pgfpoint{-8.985145\du}{14.049616\du}}
\pgfpathlineto{\pgfpoint{-8.945349\du}{14.053632\du}}
\pgfpathlineto{\pgfpoint{-8.904458\du}{14.056553\du}}
\pgfpathlineto{\pgfpoint{-8.863568\du}{14.059474\du}}
\pgfpathlineto{\pgfpoint{-8.822312\du}{14.062395\du}}
\pgfpathlineto{\pgfpoint{-8.780326\du}{14.064585\du}}
\pgfpathlineto{\pgfpoint{-8.738705\du}{14.066411\du}}
\pgfpathlineto{\pgfpoint{-8.696353\du}{14.068236\du}}
\pgfpathlineto{\pgfpoint{-8.654367\du}{14.069331\du}}
\pgfpathlineto{\pgfpoint{-8.610921\du}{14.070427\du}}
\pgfpathlineto{\pgfpoint{-8.568204\du}{14.071157\du}}
\pgfpathlineto{\pgfpoint{-8.524393\du}{14.071157\du}}
\pgfpathlineto{\pgfpoint{-8.524393\du}{14.050712\du}}
\pgfpathlineto{\pgfpoint{-8.567109\du}{14.050712\du}}
\pgfpathlineto{\pgfpoint{-8.610556\du}{14.049981\du}}
\pgfpathlineto{\pgfpoint{-8.652907\du}{14.049616\du}}
\pgfpathlineto{\pgfpoint{-8.695623\du}{14.047791\du}}
\pgfpathlineto{\pgfpoint{-8.737974\du}{14.046695\du}}
\pgfpathlineto{\pgfpoint{-8.779960\du}{14.044140\du}}
\pgfpathlineto{\pgfpoint{-8.820851\du}{14.041949\du}}
\pgfpathlineto{\pgfpoint{-8.862107\du}{14.039028\du}}
\pgfpathlineto{\pgfpoint{-8.902998\du}{14.036108\du}}
\pgfpathlineto{\pgfpoint{-8.943159\du}{14.032457\du}}
\pgfpathlineto{\pgfpoint{-8.982954\du}{14.029171\du}}
\pgfpathlineto{\pgfpoint{-9.022385\du}{14.024790\du}}
\pgfpathlineto{\pgfpoint{-9.061815\du}{14.020774\du}}
\pgfpathlineto{\pgfpoint{-9.100880\du}{14.016027\du}}
\pgfpathlineto{\pgfpoint{-9.139216\du}{14.010916\du}}
\pgfpathlineto{\pgfpoint{-9.176821\du}{14.005805\du}}
\pgfpathlineto{\pgfpoint{-9.214060\du}{13.999963\du}}
\pgfpathlineto{\pgfpoint{-9.250570\du}{13.993756\du}}
\pgfpathlineto{\pgfpoint{-9.287810\du}{13.987915\du}}
\pgfpathlineto{\pgfpoint{-9.322859\du}{13.981708\du}}
\pgfpathlineto{\pgfpoint{-9.359004\du}{13.974771\du}}
\pgfpathlineto{\pgfpoint{-9.393688\du}{13.967469\du}}
\pgfpathlineto{\pgfpoint{-9.428007\du}{13.960168\du}}
\pgfpathlineto{\pgfpoint{-9.461961\du}{13.952501\du}}
\pgfpathlineto{\pgfpoint{-9.495185\du}{13.944833\du}}
\pgfpathlineto{\pgfpoint{-9.526583\du}{13.936436\du}}
\pgfpathlineto{\pgfpoint{-9.559077\du}{13.927674\du}}
\pgfpathlineto{\pgfpoint{-9.590110\du}{13.919642\du}}
\pgfpathlineto{\pgfpoint{-9.605809\du}{13.914896\du}}
\pgfpathlineto{\pgfpoint{-9.621508\du}{13.910149\du}}
\pgfpathlineto{\pgfpoint{-9.636112\du}{13.906133\du}}
\pgfpathlineto{\pgfpoint{-9.651081\du}{13.901387\du}}
\pgfpathlineto{\pgfpoint{-9.666415\du}{13.896641\du}}
\pgfpathlineto{\pgfpoint{-9.680654\du}{13.892260\du}}
\pgfpathlineto{\pgfpoint{-9.694893\du}{13.886783\du}}
\pgfpathlineto{\pgfpoint{-9.709132\du}{13.882037\du}}
\pgfpathlineto{\pgfpoint{-9.723370\du}{13.877656\du}}
\pgfpathlineto{\pgfpoint{-9.737244\du}{13.872179\du}}
\pgfpathlineto{\pgfpoint{-9.751118\du}{13.867433\du}}
\pgfpathlineto{\pgfpoint{-9.764626\du}{13.862322\du}}
\pgfpathlineto{\pgfpoint{-9.778135\du}{13.856845\du}}
\pgfpathlineto{\pgfpoint{-9.790913\du}{13.852464\du}}
\pgfpathlineto{\pgfpoint{-9.804422\du}{13.846622\du}}
\pgfpathlineto{\pgfpoint{-9.817200\du}{13.841146\du}}
\pgfpathlineto{\pgfpoint{-9.829614\du}{13.836035\du}}
\pgfpathlineto{\pgfpoint{-9.842757\du}{13.830558\du}}
\pgfpathlineto{\pgfpoint{-9.854805\du}{13.825447\du}}
\pgfpathlineto{\pgfpoint{-9.866123\du}{13.819605\du}}
\pgfpathlineto{\pgfpoint{-9.878902\du}{13.814129\du}}
\pgfpathlineto{\pgfpoint{-9.890585\du}{13.809018\du}}
\pgfpathlineto{\pgfpoint{-9.902268\du}{13.803176\du}}
\pgfpathlineto{\pgfpoint{-9.913221\du}{13.797334\du}}
\pgfpathlineto{\pgfpoint{-9.924904\du}{13.791493\du}}
\pgfpathlineto{\pgfpoint{-9.934761\du}{13.785651\du}}
\pgfpathlineto{\pgfpoint{-9.946079\du}{13.780540\du}}
\pgfpathlineto{\pgfpoint{-9.956667\du}{13.773968\du}}
\pgfpathlineto{\pgfpoint{-9.966525\du}{13.768127\du}}
\pgfpathlineto{\pgfpoint{-9.977113\du}{13.762285\du}}
\pgfpathlineto{\pgfpoint{-9.986970\du}{13.756444\du}}
\pgfpathlineto{\pgfpoint{-9.996463\du}{13.750602\du}}
\pgfpathlineto{\pgfpoint{-10.006320\du}{13.744030\du}}
\pgfpathlineto{\pgfpoint{-10.015448\du}{13.738189\du}}
\pgfpathlineto{\pgfpoint{-10.023845\du}{13.731617\du}}
\pgfpathlineto{\pgfpoint{-10.032607\du}{13.725775\du}}
\pgfpathlineto{\pgfpoint{-10.041735\du}{13.719569\du}}
\pgfpathlineto{\pgfpoint{-10.050497\du}{13.712997\du}}
\pgfpathlineto{\pgfpoint{-10.058894\du}{13.706790\du}}
\pgfpathlineto{\pgfpoint{-10.066196\du}{13.700219\du}}
\pgfpathlineto{\pgfpoint{-10.074228\du}{13.694377\du}}
\pgfpathlineto{\pgfpoint{-10.081895\du}{13.687805\du}}
\pgfpathlineto{\pgfpoint{-10.088832\du}{13.681599\du}}
\pgfpathlineto{\pgfpoint{-10.096134\du}{13.675027\du}}
\pgfpathlineto{\pgfpoint{-10.103071\du}{13.668090\du}}
\pgfpathlineto{\pgfpoint{-10.109278\du}{13.661518\du}}
\pgfpathlineto{\pgfpoint{-10.115849\du}{13.655312\du}}
\pgfpathlineto{\pgfpoint{-10.121691\du}{13.648740\du}}
\pgfpathlineto{\pgfpoint{-10.127898\du}{13.641803\du}}
\pgfpathlineto{\pgfpoint{-10.133009\du}{13.635231\du}}
\pgfpathlineto{\pgfpoint{-10.139216\du}{13.628660\du}}
\pgfpathlineto{\pgfpoint{-10.143597\du}{13.621723\du}}
\pgfpathlineto{\pgfpoint{-10.149073\du}{13.615516\du}}
\pgfpathlineto{\pgfpoint{-10.153454\du}{13.608214\du}}
\pgfpathlineto{\pgfpoint{-10.157836\du}{13.602008\du}}
\pgfpathlineto{\pgfpoint{-10.162217\du}{13.594706\du}}
\pgfpathlineto{\pgfpoint{-10.165503\du}{13.588499\du}}
\pgfpathlineto{\pgfpoint{-10.169519\du}{13.581562\du}}
\pgfpathlineto{\pgfpoint{-10.173535\du}{13.574260\du}}
\pgfpathlineto{\pgfpoint{-10.176456\du}{13.567323\du}}
\pgfpathlineto{\pgfpoint{-10.179011\du}{13.561117\du}}
\pgfpathlineto{\pgfpoint{-10.182297\du}{13.553815\du}}
\pgfpathlineto{\pgfpoint{-10.184488\du}{13.546878\du}}
\pgfpathlineto{\pgfpoint{-10.186678\du}{13.539941\du}}
\pgfpathlineto{\pgfpoint{-10.188139\du}{13.533004\du}}
\pgfpathlineto{\pgfpoint{-10.190329\du}{13.525702\du}}
\pgfpathlineto{\pgfpoint{-10.191424\du}{13.519496\du}}
\pgfpathlineto{\pgfpoint{-10.192520\du}{13.512194\du}}
\pgfpathlineto{\pgfpoint{-10.193980\du}{13.505257\du}}
\pgfpathlineto{\pgfpoint{-10.194345\du}{13.498320\du}}
\pgfpathlineto{\pgfpoint{-10.194345\du}{13.490653\du}}
\pgfpathlineto{\pgfpoint{-10.194710\du}{13.483716\du}}
\pgfpathlineto{\pgfpoint{-10.194710\du}{13.483716\du}}
\pgfpathlineto{\pgfpoint{-10.194710\du}{13.483716\du}}
\pgfpathlineto{\pgfpoint{-10.194710\du}{13.482621\du}}
\pgfpathlineto{\pgfpoint{-10.194710\du}{13.481526\du}}
\pgfpathlineto{\pgfpoint{-10.195075\du}{13.480065\du}}
\pgfpathlineto{\pgfpoint{-10.195075\du}{13.479700\du}}
\pgfpathlineto{\pgfpoint{-10.196171\du}{13.478605\du}}
\pgfpathlineto{\pgfpoint{-10.196536\du}{13.477145\du}}
\pgfpathlineto{\pgfpoint{-10.196901\du}{13.476779\du}}
\pgfpathlineto{\pgfpoint{-10.197996\du}{13.476049\du}}
\pgfpathlineto{\pgfpoint{-10.199457\du}{13.474954\du}}
\pgfpathlineto{\pgfpoint{-10.201282\du}{13.474224\du}}
\pgfpathlineto{\pgfpoint{-10.203108\du}{13.473859\du}}
\pgfpathlineto{\pgfpoint{-10.204933\du}{13.473859\du}}
\pgfpathlineto{\pgfpoint{-10.207124\du}{13.473859\du}}
\pgfpathlineto{\pgfpoint{-10.208584\du}{13.474224\du}}
\pgfpathlineto{\pgfpoint{-10.210775\du}{13.474954\du}}
\pgfpathlineto{\pgfpoint{-10.212600\du}{13.476049\du}}
\pgfpathlineto{\pgfpoint{-10.212965\du}{13.476779\du}}
\pgfpathlineto{\pgfpoint{-10.213330\du}{13.477145\du}}
\pgfpathlineto{\pgfpoint{-10.214060\du}{13.478605\du}}
\pgfpathlineto{\pgfpoint{-10.214791\du}{13.479700\du}}
\pgfpathlineto{\pgfpoint{-10.214791\du}{13.480065\du}}
\pgfpathlineto{\pgfpoint{-10.215156\du}{13.481526\du}}
\pgfpathlineto{\pgfpoint{-10.215156\du}{13.482621\du}}
\pgfpathlineto{\pgfpoint{-10.215156\du}{13.483716\du}}
\pgfusepath{fill}
\pgfsetbuttcap
\pgfsetmiterjoin
\pgfsetdash{}{0pt}
\definecolor{dialinecolor}{rgb}{0.678431, 0.839216, 0.905882}
\pgfsetfillcolor{dialinecolor}
\pgfpathmoveto{\pgfpoint{-8.524393\du}{12.896276\du}}
\pgfpathlineto{\pgfpoint{-8.524393\du}{12.896276\du}}
\pgfpathlineto{\pgfpoint{-8.568204\du}{12.896276\du}}
\pgfpathlineto{\pgfpoint{-8.610921\du}{12.897006\du}}
\pgfpathlineto{\pgfpoint{-8.654367\du}{12.898101\du}}
\pgfpathlineto{\pgfpoint{-8.696353\du}{12.899196\du}}
\pgfpathlineto{\pgfpoint{-8.738705\du}{12.901022\du}}
\pgfpathlineto{\pgfpoint{-8.780326\du}{12.903212\du}}
\pgfpathlineto{\pgfpoint{-8.822312\du}{12.905768\du}}
\pgfpathlineto{\pgfpoint{-8.863568\du}{12.907959\du}}
\pgfpathlineto{\pgfpoint{-8.904458\du}{12.910879\du}}
\pgfpathlineto{\pgfpoint{-8.945349\du}{12.914530\du}}
\pgfpathlineto{\pgfpoint{-8.985145\du}{12.918181\du}}
\pgfpathlineto{\pgfpoint{-9.025305\du}{12.922197\du}}
\pgfpathlineto{\pgfpoint{-9.064371\du}{12.926579\du}}
\pgfpathlineto{\pgfpoint{-9.103071\du}{12.931325\du}}
\pgfpathlineto{\pgfpoint{-9.141771\du}{12.936436\du}}
\pgfpathlineto{\pgfpoint{-9.179741\du}{12.941913\du}}
\pgfpathlineto{\pgfpoint{-9.217346\du}{12.947024\du}}
\pgfpathlineto{\pgfpoint{-9.254586\du}{12.953596\du}}
\pgfpathlineto{\pgfpoint{-9.291096\du}{12.959437\du}}
\pgfpathlineto{\pgfpoint{-9.327240\du}{12.966009\du}}
\pgfpathlineto{\pgfpoint{-9.363020\du}{12.972946\du}}
\pgfpathlineto{\pgfpoint{-9.397704\du}{12.979883\du}}
\pgfpathlineto{\pgfpoint{-9.432023\du}{12.987550\du}}
\pgfpathlineto{\pgfpoint{-9.466707\du}{12.995217\du}}
\pgfpathlineto{\pgfpoint{-9.499201\du}{13.003249\du}}
\pgfpathlineto{\pgfpoint{-9.532425\du}{13.011646\du}}
\pgfpathlineto{\pgfpoint{-9.564188\du}{13.019678\du}}
\pgfpathlineto{\pgfpoint{-9.595952\du}{13.028441\du}}
\pgfpathlineto{\pgfpoint{-9.626620\du}{13.037933\du}}
\pgfpathlineto{\pgfpoint{-9.657653\du}{13.046695\du}}
\pgfpathlineto{\pgfpoint{-9.672257\du}{13.051442\du}}
\pgfpathlineto{\pgfpoint{-9.686131\du}{13.056553\du}}
\pgfpathlineto{\pgfpoint{-9.701830\du}{13.061299\du}}
\pgfpathlineto{\pgfpoint{-9.715703\du}{13.066046\du}}
\pgfpathlineto{\pgfpoint{-9.730307\du}{13.071157\du}}
\pgfpathlineto{\pgfpoint{-9.744181\du}{13.075903\du}}
\pgfpathlineto{\pgfpoint{-9.758420\du}{13.081015\du}}
\pgfpathlineto{\pgfpoint{-9.771563\du}{13.085761\du}}
\pgfpathlineto{\pgfpoint{-9.785802\du}{13.091237\du}}
\pgfpathlineto{\pgfpoint{-9.798945\du}{13.096349\du}}
\pgfpathlineto{\pgfpoint{-9.811359\du}{13.101460\du}}
\pgfpathlineto{\pgfpoint{-9.825232\du}{13.107302\du}}
\pgfpathlineto{\pgfpoint{-9.838011\du}{13.112778\du}}
\pgfpathlineto{\pgfpoint{-9.850059\du}{13.117889\du}}
\pgfpathlineto{\pgfpoint{-9.863202\du}{13.123731\du}}
\pgfpathlineto{\pgfpoint{-9.875251\du}{13.129207\du}}
\pgfpathlineto{\pgfpoint{-9.887664\du}{13.135049\du}}
\pgfpathlineto{\pgfpoint{-9.899347\du}{13.140160\du}}
\pgfpathlineto{\pgfpoint{-9.910665\du}{13.146002\du}}
\pgfpathlineto{\pgfpoint{-9.922348\du}{13.151843\du}}
\pgfpathlineto{\pgfpoint{-9.933666\du}{13.157685\du}}
\pgfpathlineto{\pgfpoint{-9.945349\du}{13.163526\du}}
\pgfpathlineto{\pgfpoint{-9.956302\du}{13.169368\du}}
\pgfpathlineto{\pgfpoint{-9.966525\du}{13.175940\du}}
\pgfpathlineto{\pgfpoint{-9.977113\du}{13.181781\du}}
\pgfpathlineto{\pgfpoint{-9.987335\du}{13.187623\du}}
\pgfpathlineto{\pgfpoint{-9.997558\du}{13.194195\du}}
\pgfpathlineto{\pgfpoint{-10.007416\du}{13.200036\du}}
\pgfpathlineto{\pgfpoint{-10.016908\du}{13.206243\du}}
\pgfpathlineto{\pgfpoint{-10.026401\du}{13.212815\du}}
\pgfpathlineto{\pgfpoint{-10.035528\du}{13.219386\du}}
\pgfpathlineto{\pgfpoint{-10.045386\du}{13.225228\du}}
\pgfpathlineto{\pgfpoint{-10.053783\du}{13.231434\du}}
\pgfpathlineto{\pgfpoint{-10.062545\du}{13.238006\du}}
\pgfpathlineto{\pgfpoint{-10.070943\du}{13.244213\du}}
\pgfpathlineto{\pgfpoint{-10.079705\du}{13.251515\du}}
\pgfpathlineto{\pgfpoint{-10.087372\du}{13.257721\du}}
\pgfpathlineto{\pgfpoint{-10.094309\du}{13.264293\du}}
\pgfpathlineto{\pgfpoint{-10.103071\du}{13.271230\du}}
\pgfpathlineto{\pgfpoint{-10.110008\du}{13.277802\du}}
\pgfpathlineto{\pgfpoint{-10.116945\du}{13.284739\du}}
\pgfpathlineto{\pgfpoint{-10.123882\du}{13.291675\du}}
\pgfpathlineto{\pgfpoint{-10.130453\du}{13.298247\du}}
\pgfpathlineto{\pgfpoint{-10.137390\du}{13.305184\du}}
\pgfpathlineto{\pgfpoint{-10.143232\du}{13.312121\du}}
\pgfpathlineto{\pgfpoint{-10.149073\du}{13.319423\du}}
\pgfpathlineto{\pgfpoint{-10.154915\du}{13.326360\du}}
\pgfpathlineto{\pgfpoint{-10.160391\du}{13.333296\du}}
\pgfpathlineto{\pgfpoint{-10.165138\du}{13.340233\du}}
\pgfpathlineto{\pgfpoint{-10.170614\du}{13.347900\du}}
\pgfpathlineto{\pgfpoint{-10.174630\du}{13.354837\du}}
\pgfpathlineto{\pgfpoint{-10.179741\du}{13.362139\du}}
\pgfpathlineto{\pgfpoint{-10.183392\du}{13.369441\du}}
\pgfpathlineto{\pgfpoint{-10.187774\du}{13.376743\du}}
\pgfpathlineto{\pgfpoint{-10.191424\du}{13.384410\du}}
\pgfpathlineto{\pgfpoint{-10.194710\du}{13.391712\du}}
\pgfpathlineto{\pgfpoint{-10.198361\du}{13.399379\du}}
\pgfpathlineto{\pgfpoint{-10.201282\du}{13.407046\du}}
\pgfpathlineto{\pgfpoint{-10.203838\du}{13.413983\du}}
\pgfpathlineto{\pgfpoint{-10.206028\du}{13.421650\du}}
\pgfpathlineto{\pgfpoint{-10.208219\du}{13.429317\du}}
\pgfpathlineto{\pgfpoint{-10.210044\du}{13.437349\du}}
\pgfpathlineto{\pgfpoint{-10.211870\du}{13.445016\du}}
\pgfpathlineto{\pgfpoint{-10.212965\du}{13.452683\du}}
\pgfpathlineto{\pgfpoint{-10.214060\du}{13.460350\du}}
\pgfpathlineto{\pgfpoint{-10.214791\du}{13.468382\du}}
\pgfpathlineto{\pgfpoint{-10.215156\du}{13.476049\du}}
\pgfpathlineto{\pgfpoint{-10.215156\du}{13.483716\du}}
\pgfpathlineto{\pgfpoint{-10.194710\du}{13.483716\du}}
\pgfpathlineto{\pgfpoint{-10.194345\du}{13.476779\du}}
\pgfpathlineto{\pgfpoint{-10.194345\du}{13.469843\du}}
\pgfpathlineto{\pgfpoint{-10.193980\du}{13.462541\du}}
\pgfpathlineto{\pgfpoint{-10.192520\du}{13.455604\du}}
\pgfpathlineto{\pgfpoint{-10.191424\du}{13.448667\du}}
\pgfpathlineto{\pgfpoint{-10.190329\du}{13.441730\du}}
\pgfpathlineto{\pgfpoint{-10.188139\du}{13.434428\du}}
\pgfpathlineto{\pgfpoint{-10.186678\du}{13.428222\du}}
\pgfpathlineto{\pgfpoint{-10.184488\du}{13.420920\du}}
\pgfpathlineto{\pgfpoint{-10.182297\du}{13.413983\du}}
\pgfpathlineto{\pgfpoint{-10.179011\du}{13.407046\du}}
\pgfpathlineto{\pgfpoint{-10.176456\du}{13.400109\du}}
\pgfpathlineto{\pgfpoint{-10.173535\du}{13.393172\du}}
\pgfpathlineto{\pgfpoint{-10.169519\du}{13.386601\du}}
\pgfpathlineto{\pgfpoint{-10.165503\du}{13.379664\du}}
\pgfpathlineto{\pgfpoint{-10.162217\du}{13.373092\du}}
\pgfpathlineto{\pgfpoint{-10.158201\du}{13.366155\du}}
\pgfpathlineto{\pgfpoint{-10.153454\du}{13.359218\du}}
\pgfpathlineto{\pgfpoint{-10.149073\du}{13.352647\du}}
\pgfpathlineto{\pgfpoint{-10.143597\du}{13.346075\du}}
\pgfpathlineto{\pgfpoint{-10.139216\du}{13.339138\du}}
\pgfpathlineto{\pgfpoint{-10.133009\du}{13.332566\du}}
\pgfpathlineto{\pgfpoint{-10.127898\du}{13.325629\du}}
\pgfpathlineto{\pgfpoint{-10.121691\du}{13.319423\du}}
\pgfpathlineto{\pgfpoint{-10.115849\du}{13.312851\du}}
\pgfpathlineto{\pgfpoint{-10.109278\du}{13.305914\du}}
\pgfpathlineto{\pgfpoint{-10.103071\du}{13.299342\du}}
\pgfpathlineto{\pgfpoint{-10.096134\du}{13.293136\du}}
\pgfpathlineto{\pgfpoint{-10.088832\du}{13.286564\du}}
\pgfpathlineto{\pgfpoint{-10.081895\du}{13.279992\du}}
\pgfpathlineto{\pgfpoint{-10.074228\du}{13.273786\du}}
\pgfpathlineto{\pgfpoint{-10.066196\du}{13.267214\du}}
\pgfpathlineto{\pgfpoint{-10.058894\du}{13.260642\du}}
\pgfpathlineto{\pgfpoint{-10.050497\du}{13.254436\du}}
\pgfpathlineto{\pgfpoint{-10.041735\du}{13.248594\du}}
\pgfpathlineto{\pgfpoint{-10.032607\du}{13.241657\du}}
\pgfpathlineto{\pgfpoint{-10.023845\du}{13.236181\du}}
\pgfpathlineto{\pgfpoint{-10.015448\du}{13.229609\du}}
\pgfpathlineto{\pgfpoint{-10.006320\du}{13.223402\du}}
\pgfpathlineto{\pgfpoint{-9.996463\du}{13.217561\du}}
\pgfpathlineto{\pgfpoint{-9.986970\du}{13.211719\du}}
\pgfpathlineto{\pgfpoint{-9.977113\du}{13.205147\du}}
\pgfpathlineto{\pgfpoint{-9.966525\du}{13.199306\du}}
\pgfpathlineto{\pgfpoint{-9.956667\du}{13.193464\du}}
\pgfpathlineto{\pgfpoint{-9.946079\du}{13.187623\du}}
\pgfpathlineto{\pgfpoint{-9.934761\du}{13.181781\du}}
\pgfpathlineto{\pgfpoint{-9.924904\du}{13.175940\du}}
\pgfpathlineto{\pgfpoint{-9.913221\du}{13.170098\du}}
\pgfpathlineto{\pgfpoint{-9.902268\du}{13.164987\du}}
\pgfpathlineto{\pgfpoint{-9.890585\du}{13.158780\du}}
\pgfpathlineto{\pgfpoint{-9.878902\du}{13.152939\du}}
\pgfpathlineto{\pgfpoint{-9.866123\du}{13.147827\du}}
\pgfpathlineto{\pgfpoint{-9.854805\du}{13.142716\du}}
\pgfpathlineto{\pgfpoint{-9.842757\du}{13.136874\du}}
\pgfpathlineto{\pgfpoint{-9.829614\du}{13.131398\du}}
\pgfpathlineto{\pgfpoint{-9.817200\du}{13.126287\du}}
\pgfpathlineto{\pgfpoint{-9.804422\du}{13.120810\du}}
\pgfpathlineto{\pgfpoint{-9.790913\du}{13.115699\du}}
\pgfpathlineto{\pgfpoint{-9.778135\du}{13.110222\du}}
\pgfpathlineto{\pgfpoint{-9.764626\du}{13.104746\du}}
\pgfpathlineto{\pgfpoint{-9.751118\du}{13.100365\du}}
\pgfpathlineto{\pgfpoint{-9.737244\du}{13.095253\du}}
\pgfpathlineto{\pgfpoint{-9.723370\du}{13.090507\du}}
\pgfpathlineto{\pgfpoint{-9.709132\du}{13.085396\du}}
\pgfpathlineto{\pgfpoint{-9.694893\du}{13.080649\du}}
\pgfpathlineto{\pgfpoint{-9.680654\du}{13.075903\du}}
\pgfpathlineto{\pgfpoint{-9.666415\du}{13.071157\du}}
\pgfpathlineto{\pgfpoint{-9.651081\du}{13.066411\du}}
\pgfpathlineto{\pgfpoint{-9.621508\du}{13.057283\du}}
\pgfpathlineto{\pgfpoint{-9.590110\du}{13.048521\du}}
\pgfpathlineto{\pgfpoint{-9.559077\du}{13.039759\du}}
\pgfpathlineto{\pgfpoint{-9.526583\du}{13.031361\du}}
\pgfpathlineto{\pgfpoint{-9.495185\du}{13.023329\du}}
\pgfpathlineto{\pgfpoint{-9.461961\du}{13.014932\du}}
\pgfpathlineto{\pgfpoint{-9.428007\du}{13.007265\du}}
\pgfpathlineto{\pgfpoint{-9.393688\du}{13.000328\du}}
\pgfpathlineto{\pgfpoint{-9.359004\du}{12.993391\du}}
\pgfpathlineto{\pgfpoint{-9.322859\du}{12.986089\du}}
\pgfpathlineto{\pgfpoint{-9.287810\du}{12.979883\du}}
\pgfpathlineto{\pgfpoint{-9.250570\du}{12.972946\du}}
\pgfpathlineto{\pgfpoint{-9.214060\du}{12.967469\du}}
\pgfpathlineto{\pgfpoint{-9.176821\du}{12.961628\du}}
\pgfpathlineto{\pgfpoint{-9.139216\du}{12.956517\du}}
\pgfpathlineto{\pgfpoint{-9.100880\du}{12.951770\du}}
\pgfpathlineto{\pgfpoint{-9.061815\du}{12.947024\du}}
\pgfpathlineto{\pgfpoint{-9.022385\du}{12.942643\du}}
\pgfpathlineto{\pgfpoint{-8.982954\du}{12.938992\du}}
\pgfpathlineto{\pgfpoint{-8.943159\du}{12.934976\du}}
\pgfpathlineto{\pgfpoint{-8.902998\du}{12.932055\du}}
\pgfpathlineto{\pgfpoint{-8.862107\du}{12.928404\du}}
\pgfpathlineto{\pgfpoint{-8.820851\du}{12.925483\du}}
\pgfpathlineto{\pgfpoint{-8.779960\du}{12.923293\du}}
\pgfpathlineto{\pgfpoint{-8.737974\du}{12.921467\du}}
\pgfpathlineto{\pgfpoint{-8.695623\du}{12.919642\du}}
\pgfpathlineto{\pgfpoint{-8.652907\du}{12.918181\du}}
\pgfpathlineto{\pgfpoint{-8.610556\du}{12.917816\du}}
\pgfpathlineto{\pgfpoint{-8.567109\du}{12.917451\du}}
\pgfpathlineto{\pgfpoint{-8.524393\du}{12.916721\du}}
\pgfpathlineto{\pgfpoint{-8.524393\du}{12.916721\du}}
\pgfpathlineto{\pgfpoint{-8.524393\du}{12.916721\du}}
\pgfpathlineto{\pgfpoint{-8.523297\du}{12.916721\du}}
\pgfpathlineto{\pgfpoint{-8.522202\du}{12.916721\du}}
\pgfpathlineto{\pgfpoint{-8.520742\du}{12.915991\du}}
\pgfpathlineto{\pgfpoint{-8.519646\du}{12.915991\du}}
\pgfpathlineto{\pgfpoint{-8.519281\du}{12.915626\du}}
\pgfpathlineto{\pgfpoint{-8.518186\du}{12.914896\du}}
\pgfpathlineto{\pgfpoint{-8.517456\du}{12.914530\du}}
\pgfpathlineto{\pgfpoint{-8.516361\du}{12.913800\du}}
\pgfpathlineto{\pgfpoint{-8.515265\du}{12.911975\du}}
\pgfpathlineto{\pgfpoint{-8.514535\du}{12.910149\du}}
\pgfpathlineto{\pgfpoint{-8.514535\du}{12.908689\du}}
\pgfpathlineto{\pgfpoint{-8.513805\du}{12.906863\du}}
\pgfpathlineto{\pgfpoint{-8.514535\du}{12.905038\du}}
\pgfpathlineto{\pgfpoint{-8.514535\du}{12.902847\du}}
\pgfpathlineto{\pgfpoint{-8.515265\du}{12.901022\du}}
\pgfpathlineto{\pgfpoint{-8.516361\du}{12.899196\du}}
\pgfpathlineto{\pgfpoint{-8.517456\du}{12.898466\du}}
\pgfpathlineto{\pgfpoint{-8.518186\du}{12.898101\du}}
\pgfpathlineto{\pgfpoint{-8.519281\du}{12.897371\du}}
\pgfpathlineto{\pgfpoint{-8.519646\du}{12.897006\du}}
\pgfpathlineto{\pgfpoint{-8.520742\du}{12.897006\du}}
\pgfpathlineto{\pgfpoint{-8.522202\du}{12.896276\du}}
\pgfpathlineto{\pgfpoint{-8.523297\du}{12.896276\du}}
\pgfpathlineto{\pgfpoint{-8.524393\du}{12.896276\du}}
\pgfusepath{fill}
\pgfsetbuttcap
\pgfsetmiterjoin
\pgfsetdash{}{0pt}
\definecolor{dialinecolor}{rgb}{0.678431, 0.839216, 0.905882}
\pgfsetfillcolor{dialinecolor}
\pgfpathmoveto{\pgfpoint{-6.833995\du}{13.483716\du}}
\pgfpathlineto{\pgfpoint{-6.833995\du}{13.476049\du}}
\pgfpathlineto{\pgfpoint{-6.834725\du}{13.468382\du}}
\pgfpathlineto{\pgfpoint{-6.835455\du}{13.460350\du}}
\pgfpathlineto{\pgfpoint{-6.836185\du}{13.452683\du}}
\pgfpathlineto{\pgfpoint{-6.837281\du}{13.445016\du}}
\pgfpathlineto{\pgfpoint{-6.839471\du}{13.437349\du}}
\pgfpathlineto{\pgfpoint{-6.840566\du}{13.429317\du}}
\pgfpathlineto{\pgfpoint{-6.843487\du}{13.421650\du}}
\pgfpathlineto{\pgfpoint{-6.845313\du}{13.413983\du}}
\pgfpathlineto{\pgfpoint{-6.848234\du}{13.407046\du}}
\pgfpathlineto{\pgfpoint{-6.850789\du}{13.399379\du}}
\pgfpathlineto{\pgfpoint{-6.854075\du}{13.391712\du}}
\pgfpathlineto{\pgfpoint{-6.858091\du}{13.384410\du}}
\pgfpathlineto{\pgfpoint{-6.861377\du}{13.376743\du}}
\pgfpathlineto{\pgfpoint{-6.865393\du}{13.369441\du}}
\pgfpathlineto{\pgfpoint{-6.869409\du}{13.362139\du}}
\pgfpathlineto{\pgfpoint{-6.873790\du}{13.354837\du}}
\pgfpathlineto{\pgfpoint{-6.878537\du}{13.347900\du}}
\pgfpathlineto{\pgfpoint{-6.884013\du}{13.340233\du}}
\pgfpathlineto{\pgfpoint{-6.888759\du}{13.333296\du}}
\pgfpathlineto{\pgfpoint{-6.894601\du}{13.326360\du}}
\pgfpathlineto{\pgfpoint{-6.900077\du}{13.319423\du}}
\pgfpathlineto{\pgfpoint{-6.905919\du}{13.312121\du}}
\pgfpathlineto{\pgfpoint{-6.912125\du}{13.305184\du}}
\pgfpathlineto{\pgfpoint{-6.918697\du}{13.298247\du}}
\pgfpathlineto{\pgfpoint{-6.925634\du}{13.291675\du}}
\pgfpathlineto{\pgfpoint{-6.932206\du}{13.284739\du}}
\pgfpathlineto{\pgfpoint{-6.939508\du}{13.277802\du}}
\pgfpathlineto{\pgfpoint{-6.946079\du}{13.271230\du}}
\pgfpathlineto{\pgfpoint{-6.954112\du}{13.264293\du}}
\pgfpathlineto{\pgfpoint{-6.962144\du}{13.257721\du}}
\pgfpathlineto{\pgfpoint{-6.969446\du}{13.251515\du}}
\pgfpathlineto{\pgfpoint{-6.978208\du}{13.244213\du}}
\pgfpathlineto{\pgfpoint{-6.986240\du}{13.238006\du}}
\pgfpathlineto{\pgfpoint{-6.995368\du}{13.231434\du}}
\pgfpathlineto{\pgfpoint{-7.004130\du}{13.225228\du}}
\pgfpathlineto{\pgfpoint{-7.013622\du}{13.219386\du}}
\pgfpathlineto{\pgfpoint{-7.022750\du}{13.212815\du}}
\pgfpathlineto{\pgfpoint{-7.031877\du}{13.206243\du}}
\pgfpathlineto{\pgfpoint{-7.041735\du}{13.200036\du}}
\pgfpathlineto{\pgfpoint{-7.051592\du}{13.194195\du}}
\pgfpathlineto{\pgfpoint{-7.061815\du}{13.187623\du}}
\pgfpathlineto{\pgfpoint{-7.072403\du}{13.181781\du}}
\pgfpathlineto{\pgfpoint{-7.082261\du}{13.175940\du}}
\pgfpathlineto{\pgfpoint{-7.092848\du}{13.169368\du}}
\pgfpathlineto{\pgfpoint{-7.104166\du}{13.163526\du}}
\pgfpathlineto{\pgfpoint{-7.115484\du}{13.157685\du}}
\pgfpathlineto{\pgfpoint{-7.127167\du}{13.151843\du}}
\pgfpathlineto{\pgfpoint{-7.138120\du}{13.146002\du}}
\pgfpathlineto{\pgfpoint{-7.150169\du}{13.140160\du}}
\pgfpathlineto{\pgfpoint{-7.161487\du}{13.135049\du}}
\pgfpathlineto{\pgfpoint{-7.173900\du}{13.129207\du}}
\pgfpathlineto{\pgfpoint{-7.186313\du}{13.123731\du}}
\pgfpathlineto{\pgfpoint{-7.199092\du}{13.117889\du}}
\pgfpathlineto{\pgfpoint{-7.211140\du}{13.112778\du}}
\pgfpathlineto{\pgfpoint{-7.223918\du}{13.107302\du}}
\pgfpathlineto{\pgfpoint{-7.237427\du}{13.101460\du}}
\pgfpathlineto{\pgfpoint{-7.250570\du}{13.096349\du}}
\pgfpathlineto{\pgfpoint{-7.263714\du}{13.091237\du}}
\pgfpathlineto{\pgfpoint{-7.277952\du}{13.085761\du}}
\pgfpathlineto{\pgfpoint{-7.291096\du}{13.081015\du}}
\pgfpathlineto{\pgfpoint{-7.305335\du}{13.075903\du}}
\pgfpathlineto{\pgfpoint{-7.319208\du}{13.071157\du}}
\pgfpathlineto{\pgfpoint{-7.333447\du}{13.066046\du}}
\pgfpathlineto{\pgfpoint{-7.347321\du}{13.061299\du}}
\pgfpathlineto{\pgfpoint{-7.362290\du}{13.056553\du}}
\pgfpathlineto{\pgfpoint{-7.377259\du}{13.051442\du}}
\pgfpathlineto{\pgfpoint{-7.391497\du}{13.046695\du}}
\pgfpathlineto{\pgfpoint{-7.422166\du}{13.037933\du}}
\pgfpathlineto{\pgfpoint{-7.452834\du}{13.028441\du}}
\pgfpathlineto{\pgfpoint{-7.484232\du}{13.019678\du}}
\pgfpathlineto{\pgfpoint{-7.515995\du}{13.011646\du}}
\pgfpathlineto{\pgfpoint{-7.548854\du}{13.003249\du}}
\pgfpathlineto{\pgfpoint{-7.582443\du}{12.995217\du}}
\pgfpathlineto{\pgfpoint{-7.616397\du}{12.987550\du}}
\pgfpathlineto{\pgfpoint{-7.651081\du}{12.979883\du}}
\pgfpathlineto{\pgfpoint{-7.685765\du}{12.972946\du}}
\pgfpathlineto{\pgfpoint{-7.721180\du}{12.966009\du}}
\pgfpathlineto{\pgfpoint{-7.757690\du}{12.959437\du}}
\pgfpathlineto{\pgfpoint{-7.794199\du}{12.953596\du}}
\pgfpathlineto{\pgfpoint{-7.831074\du}{12.947024\du}}
\pgfpathlineto{\pgfpoint{-7.868314\du}{12.941913\du}}
\pgfpathlineto{\pgfpoint{-7.907379\du}{12.936436\du}}
\pgfpathlineto{\pgfpoint{-7.945349\du}{12.931325\du}}
\pgfpathlineto{\pgfpoint{-7.984780\du}{12.926579\du}}
\pgfpathlineto{\pgfpoint{-8.023480\du}{12.922197\du}}
\pgfpathlineto{\pgfpoint{-8.063641\du}{12.918181\du}}
\pgfpathlineto{\pgfpoint{-8.103436\du}{12.914530\du}}
\pgfpathlineto{\pgfpoint{-8.144327\du}{12.910879\du}}
\pgfpathlineto{\pgfpoint{-8.184853\du}{12.907959\du}}
\pgfpathlineto{\pgfpoint{-8.226474\du}{12.905768\du}}
\pgfpathlineto{\pgfpoint{-8.268095\du}{12.903212\du}}
\pgfpathlineto{\pgfpoint{-8.309716\du}{12.901022\du}}
\pgfpathlineto{\pgfpoint{-8.352432\du}{12.899196\du}}
\pgfpathlineto{\pgfpoint{-8.394418\du}{12.898101\du}}
\pgfpathlineto{\pgfpoint{-8.437500\du}{12.897006\du}}
\pgfpathlineto{\pgfpoint{-8.480946\du}{12.896276\du}}
\pgfpathlineto{\pgfpoint{-8.524393\du}{12.896276\du}}
\pgfpathlineto{\pgfpoint{-8.524393\du}{12.916721\du}}
\pgfpathlineto{\pgfpoint{-8.481311\du}{12.917451\du}}
\pgfpathlineto{\pgfpoint{-8.437865\du}{12.917816\du}}
\pgfpathlineto{\pgfpoint{-8.395879\du}{12.918181\du}}
\pgfpathlineto{\pgfpoint{-8.353162\du}{12.919642\du}}
\pgfpathlineto{\pgfpoint{-8.310446\du}{12.921467\du}}
\pgfpathlineto{\pgfpoint{-8.268825\du}{12.923293\du}}
\pgfpathlineto{\pgfpoint{-8.227569\du}{12.925483\du}}
\pgfpathlineto{\pgfpoint{-8.186678\du}{12.928404\du}}
\pgfpathlineto{\pgfpoint{-8.145422\du}{12.932055\du}}
\pgfpathlineto{\pgfpoint{-8.105262\du}{12.934976\du}}
\pgfpathlineto{\pgfpoint{-8.065101\du}{12.938992\du}}
\pgfpathlineto{\pgfpoint{-8.026036\du}{12.942643\du}}
\pgfpathlineto{\pgfpoint{-7.986605\du}{12.947024\du}}
\pgfpathlineto{\pgfpoint{-7.947905\du}{12.951770\du}}
\pgfpathlineto{\pgfpoint{-7.909205\du}{12.956517\du}}
\pgfpathlineto{\pgfpoint{-7.871600\du}{12.961628\du}}
\pgfpathlineto{\pgfpoint{-7.834360\du}{12.967469\du}}
\pgfpathlineto{\pgfpoint{-7.797850\du}{12.972946\du}}
\pgfpathlineto{\pgfpoint{-7.760975\du}{12.979883\du}}
\pgfpathlineto{\pgfpoint{-7.725561\du}{12.986089\du}}
\pgfpathlineto{\pgfpoint{-7.689782\du}{12.993391\du}}
\pgfpathlineto{\pgfpoint{-7.655097\du}{13.000328\du}}
\pgfpathlineto{\pgfpoint{-7.621143\du}{13.007265\du}}
\pgfpathlineto{\pgfpoint{-7.586824\du}{13.014932\du}}
\pgfpathlineto{\pgfpoint{-7.553600\du}{13.023329\du}}
\pgfpathlineto{\pgfpoint{-7.521472\du}{13.031361\du}}
\pgfpathlineto{\pgfpoint{-7.489343\du}{13.039759\du}}
\pgfpathlineto{\pgfpoint{-7.457945\du}{13.048521\du}}
\pgfpathlineto{\pgfpoint{-7.427277\du}{13.057283\du}}
\pgfpathlineto{\pgfpoint{-7.397339\du}{13.066411\du}}
\pgfpathlineto{\pgfpoint{-7.383100\du}{13.071157\du}}
\pgfpathlineto{\pgfpoint{-7.368496\du}{13.075903\du}}
\pgfpathlineto{\pgfpoint{-7.354623\du}{13.080649\du}}
\pgfpathlineto{\pgfpoint{-7.340019\du}{13.085396\du}}
\pgfpathlineto{\pgfpoint{-7.325780\du}{13.090507\du}}
\pgfpathlineto{\pgfpoint{-7.311906\du}{13.095253\du}}
\pgfpathlineto{\pgfpoint{-7.297668\du}{13.100365\du}}
\pgfpathlineto{\pgfpoint{-7.284524\du}{13.104746\du}}
\pgfpathlineto{\pgfpoint{-7.271016\du}{13.110222\du}}
\pgfpathlineto{\pgfpoint{-7.257872\du}{13.115699\du}}
\pgfpathlineto{\pgfpoint{-7.245094\du}{13.120810\du}}
\pgfpathlineto{\pgfpoint{-7.232315\du}{13.126287\du}}
\pgfpathlineto{\pgfpoint{-7.219172\du}{13.131398\du}}
\pgfpathlineto{\pgfpoint{-7.207124\du}{13.136874\du}}
\pgfpathlineto{\pgfpoint{-7.194710\du}{13.142716\du}}
\pgfpathlineto{\pgfpoint{-7.182662\du}{13.147827\du}}
\pgfpathlineto{\pgfpoint{-7.170249\du}{13.152939\du}}
\pgfpathlineto{\pgfpoint{-7.158931\du}{13.158780\du}}
\pgfpathlineto{\pgfpoint{-7.146883\du}{13.164987\du}}
\pgfpathlineto{\pgfpoint{-7.136295\du}{13.170098\du}}
\pgfpathlineto{\pgfpoint{-7.124247\du}{13.175940\du}}
\pgfpathlineto{\pgfpoint{-7.114024\du}{13.181781\du}}
\pgfpathlineto{\pgfpoint{-7.103071\du}{13.187623\du}}
\pgfpathlineto{\pgfpoint{-7.092118\du}{13.193464\du}}
\pgfpathlineto{\pgfpoint{-7.082261\du}{13.199306\du}}
\pgfpathlineto{\pgfpoint{-7.072403\du}{13.205147\du}}
\pgfpathlineto{\pgfpoint{-7.062180\du}{13.211719\du}}
\pgfpathlineto{\pgfpoint{-7.053053\du}{13.217561\du}}
\pgfpathlineto{\pgfpoint{-7.042830\du}{13.223402\du}}
\pgfpathlineto{\pgfpoint{-7.033338\du}{13.229609\du}}
\pgfpathlineto{\pgfpoint{-7.024575\du}{13.236181\du}}
\pgfpathlineto{\pgfpoint{-7.016543\du}{13.241657\du}}
\pgfpathlineto{\pgfpoint{-7.007781\du}{13.248594\du}}
\pgfpathlineto{\pgfpoint{-6.999384\du}{13.254436\du}}
\pgfpathlineto{\pgfpoint{-6.990621\du}{13.260642\du}}
\pgfpathlineto{\pgfpoint{-6.982954\du}{13.267214\du}}
\pgfpathlineto{\pgfpoint{-6.974922\du}{13.273786\du}}
\pgfpathlineto{\pgfpoint{-6.967620\du}{13.279992\du}}
\pgfpathlineto{\pgfpoint{-6.959953\du}{13.286564\du}}
\pgfpathlineto{\pgfpoint{-6.953381\du}{13.293136\du}}
\pgfpathlineto{\pgfpoint{-6.946079\du}{13.299342\du}}
\pgfpathlineto{\pgfpoint{-6.940238\du}{13.305914\du}}
\pgfpathlineto{\pgfpoint{-6.933301\du}{13.312851\du}}
\pgfpathlineto{\pgfpoint{-6.926729\du}{13.319423\du}}
\pgfpathlineto{\pgfpoint{-6.921618\du}{13.325629\du}}
\pgfpathlineto{\pgfpoint{-6.916142\du}{13.332566\du}}
\pgfpathlineto{\pgfpoint{-6.909935\du}{13.339138\du}}
\pgfpathlineto{\pgfpoint{-6.905189\du}{13.346075\du}}
\pgfpathlineto{\pgfpoint{-6.900077\du}{13.352647\du}}
\pgfpathlineto{\pgfpoint{-6.895331\du}{13.359218\du}}
\pgfpathlineto{\pgfpoint{-6.890950\du}{13.366155\du}}
\pgfpathlineto{\pgfpoint{-6.886569\du}{13.373092\du}}
\pgfpathlineto{\pgfpoint{-6.883648\du}{13.379664\du}}
\pgfpathlineto{\pgfpoint{-6.879632\du}{13.386601\du}}
\pgfpathlineto{\pgfpoint{-6.876711\du}{13.393172\du}}
\pgfpathlineto{\pgfpoint{-6.872695\du}{13.400109\du}}
\pgfpathlineto{\pgfpoint{-6.870139\du}{13.407046\du}}
\pgfpathlineto{\pgfpoint{-6.867219\du}{13.413983\du}}
\pgfpathlineto{\pgfpoint{-6.864663\du}{13.420920\du}}
\pgfpathlineto{\pgfpoint{-6.862472\du}{13.428222\du}}
\pgfpathlineto{\pgfpoint{-6.861012\du}{13.434428\du}}
\pgfpathlineto{\pgfpoint{-6.858821\du}{13.441730\du}}
\pgfpathlineto{\pgfpoint{-6.858091\du}{13.448667\du}}
\pgfpathlineto{\pgfpoint{-6.856996\du}{13.455604\du}}
\pgfpathlineto{\pgfpoint{-6.855170\du}{13.462541\du}}
\pgfpathlineto{\pgfpoint{-6.854805\du}{13.469843\du}}
\pgfpathlineto{\pgfpoint{-6.854805\du}{13.476779\du}}
\pgfpathlineto{\pgfpoint{-6.854075\du}{13.483716\du}}
\pgfpathlineto{\pgfpoint{-6.833995\du}{13.483716\du}}
\pgfusepath{fill}
\pgfsetbuttcap
\pgfsetmiterjoin
\pgfsetdash{}{0pt}
\definecolor{dialinecolor}{rgb}{0.074510, 0.082353, 0.086275}
\pgfsetfillcolor{dialinecolor}
\pgfpathmoveto{\pgfpoint{-8.481311\du}{13.356298\du}}
\pgfpathlineto{\pgfpoint{-8.233411\du}{13.438809\du}}
\pgfpathlineto{\pgfpoint{-7.648161\du}{13.204417\du}}
\pgfpathlineto{\pgfpoint{-7.375433\du}{13.271960\du}}
\pgfpathlineto{\pgfpoint{-7.519281\du}{13.063490\du}}
\pgfpathlineto{\pgfpoint{-8.223553\du}{13.063490\du}}
\pgfpathlineto{\pgfpoint{-7.929285\du}{13.136144\du}}
\pgfpathlineto{\pgfpoint{-8.481311\du}{13.356298\du}}
\pgfusepath{fill}
\pgfsetbuttcap
\pgfsetmiterjoin
\pgfsetdash{}{0pt}
\definecolor{dialinecolor}{rgb}{0.074510, 0.082353, 0.086275}
\pgfsetfillcolor{dialinecolor}
\pgfpathmoveto{\pgfpoint{-8.583173\du}{13.594341\du}}
\pgfpathlineto{\pgfpoint{-8.831074\du}{13.512194\du}}
\pgfpathlineto{\pgfpoint{-9.416324\du}{13.745856\du}}
\pgfpathlineto{\pgfpoint{-9.689416\du}{13.679043\du}}
\pgfpathlineto{\pgfpoint{-9.545568\du}{13.886783\du}}
\pgfpathlineto{\pgfpoint{-8.840201\du}{13.886783\du}}
\pgfpathlineto{\pgfpoint{-9.135200\du}{13.814859\du}}
\pgfpathlineto{\pgfpoint{-8.583173\du}{13.594341\du}}
\pgfusepath{fill}
\pgfsetbuttcap
\pgfsetmiterjoin
\pgfsetdash{}{0pt}
\definecolor{dialinecolor}{rgb}{0.074510, 0.082353, 0.086275}
\pgfsetfillcolor{dialinecolor}
\pgfpathmoveto{\pgfpoint{-9.629175\du}{13.135414\du}}
\pgfpathlineto{\pgfpoint{-9.381640\du}{13.053632\du}}
\pgfpathlineto{\pgfpoint{-8.796390\du}{13.286929\du}}
\pgfpathlineto{\pgfpoint{-8.523297\du}{13.220482\du}}
\pgfpathlineto{\pgfpoint{-8.667146\du}{13.428222\du}}
\pgfpathlineto{\pgfpoint{-9.372147\du}{13.428222\du}}
\pgfpathlineto{\pgfpoint{-9.077149\du}{13.356298\du}}
\pgfpathlineto{\pgfpoint{-9.629175\du}{13.135414\du}}
\pgfusepath{fill}
\pgfsetbuttcap
\pgfsetmiterjoin
\pgfsetdash{}{0pt}
\definecolor{dialinecolor}{rgb}{0.074510, 0.082353, 0.086275}
\pgfsetfillcolor{dialinecolor}
\pgfpathmoveto{\pgfpoint{-7.410848\du}{13.830558\du}}
\pgfpathlineto{\pgfpoint{-7.658383\du}{13.912705\du}}
\pgfpathlineto{\pgfpoint{-8.243633\du}{13.679043\du}}
\pgfpathlineto{\pgfpoint{-8.517456\du}{13.745856\du}}
\pgfpathlineto{\pgfpoint{-8.372878\du}{13.538116\du}}
\pgfpathlineto{\pgfpoint{-7.667511\du}{13.538116\du}}
\pgfpathlineto{\pgfpoint{-7.962874\du}{13.610040\du}}
\pgfpathlineto{\pgfpoint{-7.410848\du}{13.830558\du}}
\pgfusepath{fill}
\pgfsetbuttcap
\pgfsetmiterjoin
\pgfsetdash{}{0pt}
\definecolor{dialinecolor}{rgb}{1.000000, 1.000000, 1.000000}
\pgfsetfillcolor{dialinecolor}
\pgfpathmoveto{\pgfpoint{-8.460501\du}{13.376743\du}}
\pgfpathlineto{\pgfpoint{-8.212965\du}{13.459255\du}}
\pgfpathlineto{\pgfpoint{-7.627715\du}{13.225228\du}}
\pgfpathlineto{\pgfpoint{-7.355353\du}{13.292406\du}}
\pgfpathlineto{\pgfpoint{-7.498106\du}{13.083935\du}}
\pgfpathlineto{\pgfpoint{-8.203473\du}{13.083935\du}}
\pgfpathlineto{\pgfpoint{-7.908474\du}{13.156590\du}}
\pgfpathlineto{\pgfpoint{-8.460501\du}{13.376743\du}}
\pgfusepath{fill}
\pgfsetbuttcap
\pgfsetmiterjoin
\pgfsetdash{}{0pt}
\definecolor{dialinecolor}{rgb}{1.000000, 1.000000, 1.000000}
\pgfsetfillcolor{dialinecolor}
\pgfpathmoveto{\pgfpoint{-8.562363\du}{13.615516\du}}
\pgfpathlineto{\pgfpoint{-8.810994\du}{13.533004\du}}
\pgfpathlineto{\pgfpoint{-9.395879\du}{13.767031\du}}
\pgfpathlineto{\pgfpoint{-9.669336\du}{13.699488\du}}
\pgfpathlineto{\pgfpoint{-9.524393\du}{13.907959\du}}
\pgfpathlineto{\pgfpoint{-8.819756\du}{13.907959\du}}
\pgfpathlineto{\pgfpoint{-9.114389\du}{13.835304\du}}
\pgfpathlineto{\pgfpoint{-8.562363\du}{13.615516\du}}
\pgfusepath{fill}
\pgfsetbuttcap
\pgfsetmiterjoin
\pgfsetdash{}{0pt}
\definecolor{dialinecolor}{rgb}{1.000000, 1.000000, 1.000000}
\pgfsetfillcolor{dialinecolor}
\pgfpathmoveto{\pgfpoint{-9.608730\du}{13.155859\du}}
\pgfpathlineto{\pgfpoint{-9.361194\du}{13.074078\du}}
\pgfpathlineto{\pgfpoint{-8.775579\du}{13.308105\du}}
\pgfpathlineto{\pgfpoint{-8.502487\du}{13.241292\du}}
\pgfpathlineto{\pgfpoint{-8.647430\du}{13.448667\du}}
\pgfpathlineto{\pgfpoint{-9.351702\du}{13.448667\du}}
\pgfpathlineto{\pgfpoint{-9.057069\du}{13.376743\du}}
\pgfpathlineto{\pgfpoint{-9.608730\du}{13.155859\du}}
\pgfusepath{fill}
\pgfsetbuttcap
\pgfsetmiterjoin
\pgfsetdash{}{0pt}
\definecolor{dialinecolor}{rgb}{1.000000, 1.000000, 1.000000}
\pgfsetfillcolor{dialinecolor}
\pgfpathmoveto{\pgfpoint{-7.390767\du}{13.851004\du}}
\pgfpathlineto{\pgfpoint{-7.638303\du}{13.933150\du}}
\pgfpathlineto{\pgfpoint{-8.223188\du}{13.699488\du}}
\pgfpathlineto{\pgfpoint{-8.496645\du}{13.766301\du}}
\pgfpathlineto{\pgfpoint{-8.352432\du}{13.558561\du}}
\pgfpathlineto{\pgfpoint{-7.647430\du}{13.558561\du}}
\pgfpathlineto{\pgfpoint{-7.941698\du}{13.630485\du}}
\pgfpathlineto{\pgfpoint{-7.390767\du}{13.851004\du}}
\pgfusepath{fill}
\pgfsetbuttcap
\pgfsetmiterjoin
\pgfsetdash{}{0pt}
\definecolor{dialinecolor}{rgb}{0.678431, 0.839216, 0.905882}
\pgfsetfillcolor{dialinecolor}
\pgfpathmoveto{\pgfpoint{-10.194710\du}{13.494304\du}}
\pgfpathlineto{\pgfpoint{-10.194710\du}{13.483716\du}}
\pgfpathlineto{\pgfpoint{-10.215156\du}{13.483716\du}}
\pgfpathlineto{\pgfpoint{-10.215156\du}{13.494304\du}}
\pgfpathlineto{\pgfpoint{-10.194710\du}{13.494304\du}}
\pgfusepath{fill}
\pgfsetbuttcap
\pgfsetmiterjoin
\pgfsetdash{}{0pt}
\definecolor{dialinecolor}{rgb}{0.678431, 0.839216, 0.905882}
\pgfsetfillcolor{dialinecolor}
\pgfpathmoveto{\pgfpoint{-10.194710\du}{14.323804\du}}
\pgfpathlineto{\pgfpoint{-10.194710\du}{13.494304\du}}
\pgfpathlineto{\pgfpoint{-10.215156\du}{13.494304\du}}
\pgfpathlineto{\pgfpoint{-10.215156\du}{14.323804\du}}
\pgfpathlineto{\pgfpoint{-10.194710\du}{14.323804\du}}
\pgfusepath{fill}
\pgfsetbuttcap
\pgfsetmiterjoin
\pgfsetdash{}{0pt}
\definecolor{dialinecolor}{rgb}{0.678431, 0.839216, 0.905882}
\pgfsetfillcolor{dialinecolor}
\pgfpathmoveto{\pgfpoint{-10.215156\du}{14.323804\du}}
\pgfpathlineto{\pgfpoint{-10.215156\du}{14.334392\du}}
\pgfpathlineto{\pgfpoint{-10.194710\du}{14.334392\du}}
\pgfpathlineto{\pgfpoint{-10.194710\du}{14.323804\du}}
\pgfpathlineto{\pgfpoint{-10.215156\du}{14.323804\du}}
\pgfusepath{fill}
\pgfsetbuttcap
\pgfsetmiterjoin
\pgfsetdash{}{0pt}
\definecolor{dialinecolor}{rgb}{0.678431, 0.839216, 0.905882}
\pgfsetfillcolor{dialinecolor}
\pgfpathmoveto{\pgfpoint{-6.833995\du}{13.494304\du}}
\pgfpathlineto{\pgfpoint{-6.833995\du}{13.483716\du}}
\pgfpathlineto{\pgfpoint{-6.854075\du}{13.483716\du}}
\pgfpathlineto{\pgfpoint{-6.854075\du}{13.494304\du}}
\pgfpathlineto{\pgfpoint{-6.833995\du}{13.494304\du}}
\pgfusepath{fill}
\pgfsetbuttcap
\pgfsetmiterjoin
\pgfsetdash{}{0pt}
\definecolor{dialinecolor}{rgb}{0.678431, 0.839216, 0.905882}
\pgfsetfillcolor{dialinecolor}
\pgfpathmoveto{\pgfpoint{-6.833995\du}{14.323804\du}}
\pgfpathlineto{\pgfpoint{-6.833995\du}{13.494304\du}}
\pgfpathlineto{\pgfpoint{-6.854075\du}{13.494304\du}}
\pgfpathlineto{\pgfpoint{-6.854075\du}{14.323804\du}}
\pgfpathlineto{\pgfpoint{-6.833995\du}{14.323804\du}}
\pgfusepath{fill}
\pgfsetbuttcap
\pgfsetmiterjoin
\pgfsetdash{}{0pt}
\definecolor{dialinecolor}{rgb}{0.678431, 0.839216, 0.905882}
\pgfsetfillcolor{dialinecolor}
\pgfpathmoveto{\pgfpoint{-6.854075\du}{14.323804\du}}
\pgfpathlineto{\pgfpoint{-6.854075\du}{14.334392\du}}
\pgfpathlineto{\pgfpoint{-6.833995\du}{14.334392\du}}
\pgfpathlineto{\pgfpoint{-6.833995\du}{14.323804\du}}
\pgfpathlineto{\pgfpoint{-6.854075\du}{14.323804\du}}
\pgfusepath{fill}
\pgfsetbuttcap
\pgfsetmiterjoin
\pgfsetdash{}{0pt}
\definecolor{dialinecolor}{rgb}{0.027451, 0.372549, 0.529412}
\pgfsetfillcolor{dialinecolor}
\pgfpathmoveto{\pgfpoint{-7.907744\du}{14.473128\du}}
\pgfpathlineto{\pgfpoint{-7.908109\du}{14.490653\du}}
\pgfpathlineto{\pgfpoint{-7.910665\du}{14.507813\du}}
\pgfpathlineto{\pgfpoint{-7.914316\du}{14.523877\du}}
\pgfpathlineto{\pgfpoint{-7.920158\du}{14.541036\du}}
\pgfpathlineto{\pgfpoint{-7.926729\du}{14.556736\du}}
\pgfpathlineto{\pgfpoint{-7.935127\du}{14.572800\du}}
\pgfpathlineto{\pgfpoint{-7.944619\du}{14.588499\du}}
\pgfpathlineto{\pgfpoint{-7.956302\du}{14.603468\du}}
\pgfpathlineto{\pgfpoint{-7.967985\du}{14.618437\du}}
\pgfpathlineto{\pgfpoint{-7.981494\du}{14.633041\du}}
\pgfpathlineto{\pgfpoint{-7.996463\du}{14.646915\du}}
\pgfpathlineto{\pgfpoint{-8.012527\du}{14.661153\du}}
\pgfpathlineto{\pgfpoint{-8.030052\du}{14.673932\du}}
\pgfpathlineto{\pgfpoint{-8.048307\du}{14.686710\du}}
\pgfpathlineto{\pgfpoint{-8.067292\du}{14.699123\du}}
\pgfpathlineto{\pgfpoint{-8.087372\du}{14.710806\du}}
\pgfpathlineto{\pgfpoint{-8.108913\du}{14.721394\du}}
\pgfpathlineto{\pgfpoint{-8.131549\du}{14.732347\du}}
\pgfpathlineto{\pgfpoint{-8.154550\du}{14.742205\du}}
\pgfpathlineto{\pgfpoint{-8.178281\du}{14.751697\du}}
\pgfpathlineto{\pgfpoint{-8.203473\du}{14.759729\du}}
\pgfpathlineto{\pgfpoint{-8.229029\du}{14.768127\du}}
\pgfpathlineto{\pgfpoint{-8.255681\du}{14.775794\du}}
\pgfpathlineto{\pgfpoint{-8.282334\du}{14.782731\du}}
\pgfpathlineto{\pgfpoint{-8.310446\du}{14.788572\du}}
\pgfpathlineto{\pgfpoint{-8.338924\du}{14.793683\du}}
\pgfpathlineto{\pgfpoint{-8.368496\du}{14.798065\du}}
\pgfpathlineto{\pgfpoint{-8.398799\du}{14.802081\du}}
\pgfpathlineto{\pgfpoint{-8.428737\du}{14.805001\du}}
\pgfpathlineto{\pgfpoint{-8.459405\du}{14.807192\du}}
\pgfpathlineto{\pgfpoint{-8.490804\du}{14.808287\du}}
\pgfpathlineto{\pgfpoint{-8.522202\du}{14.809018\du}}
\pgfpathlineto{\pgfpoint{-8.553235\du}{14.808287\du}}
\pgfpathlineto{\pgfpoint{-8.584634\du}{14.807192\du}}
\pgfpathlineto{\pgfpoint{-8.615667\du}{14.805001\du}}
\pgfpathlineto{\pgfpoint{-8.645970\du}{14.802081\du}}
\pgfpathlineto{\pgfpoint{-8.675543\du}{14.798065\du}}
\pgfpathlineto{\pgfpoint{-8.705116\du}{14.793683\du}}
\pgfpathlineto{\pgfpoint{-8.733593\du}{14.788572\du}}
\pgfpathlineto{\pgfpoint{-8.761340\du}{14.782731\du}}
\pgfpathlineto{\pgfpoint{-8.788723\du}{14.775794\du}}
\pgfpathlineto{\pgfpoint{-8.815010\du}{14.768127\du}}
\pgfpathlineto{\pgfpoint{-8.840201\du}{14.759729\du}}
\pgfpathlineto{\pgfpoint{-8.865758\du}{14.751697\du}}
\pgfpathlineto{\pgfpoint{-8.889855\du}{14.742205\du}}
\pgfpathlineto{\pgfpoint{-8.912491\du}{14.732347\du}}
\pgfpathlineto{\pgfpoint{-8.935127\du}{14.721394\du}}
\pgfpathlineto{\pgfpoint{-8.955937\du}{14.710806\du}}
\pgfpathlineto{\pgfpoint{-8.976382\du}{14.699123\du}}
\pgfpathlineto{\pgfpoint{-8.996828\du}{14.686710\du}}
\pgfpathlineto{\pgfpoint{-9.014718\du}{14.673932\du}}
\pgfpathlineto{\pgfpoint{-9.031147\du}{14.661153\du}}
\pgfpathlineto{\pgfpoint{-9.047941\du}{14.646915\du}}
\pgfpathlineto{\pgfpoint{-9.062545\du}{14.633041\du}}
\pgfpathlineto{\pgfpoint{-9.075689\du}{14.618437\du}}
\pgfpathlineto{\pgfpoint{-9.088102\du}{14.603468\du}}
\pgfpathlineto{\pgfpoint{-9.099055\du}{14.588499\du}}
\pgfpathlineto{\pgfpoint{-9.108548\du}{14.572800\du}}
\pgfpathlineto{\pgfpoint{-9.117310\du}{14.556736\du}}
\pgfpathlineto{\pgfpoint{-9.124247\du}{14.541036\du}}
\pgfpathlineto{\pgfpoint{-9.128993\du}{14.523877\du}}
\pgfpathlineto{\pgfpoint{-9.133374\du}{14.507813\du}}
\pgfpathlineto{\pgfpoint{-9.135930\du}{14.490653\du}}
\pgfpathlineto{\pgfpoint{-9.137025\du}{14.473128\du}}
\pgfpathlineto{\pgfpoint{-9.135930\du}{14.455604\du}}
\pgfpathlineto{\pgfpoint{-9.133374\du}{14.438444\du}}
\pgfpathlineto{\pgfpoint{-9.128993\du}{14.422380\du}}
\pgfpathlineto{\pgfpoint{-9.124247\du}{14.405220\du}}
\pgfpathlineto{\pgfpoint{-9.117310\du}{14.389521\du}}
\pgfpathlineto{\pgfpoint{-9.108548\du}{14.373822\du}}
\pgfpathlineto{\pgfpoint{-9.099055\du}{14.357758\du}}
\pgfpathlineto{\pgfpoint{-9.088102\du}{14.342789\du}}
\pgfpathlineto{\pgfpoint{-9.075689\du}{14.328185\du}}
\pgfpathlineto{\pgfpoint{-9.062545\du}{14.313581\du}}
\pgfpathlineto{\pgfpoint{-9.047941\du}{14.299342\du}}
\pgfpathlineto{\pgfpoint{-9.031147\du}{14.285469\du}}
\pgfpathlineto{\pgfpoint{-9.014718\du}{14.272325\du}}
\pgfpathlineto{\pgfpoint{-8.996828\du}{14.260277\du}}
\pgfpathlineto{\pgfpoint{-8.976382\du}{14.247864\du}}
\pgfpathlineto{\pgfpoint{-8.955937\du}{14.236181\du}}
\pgfpathlineto{\pgfpoint{-8.935127\du}{14.225228\du}}
\pgfpathlineto{\pgfpoint{-8.912491\du}{14.214640\du}}
\pgfpathlineto{\pgfpoint{-8.889855\du}{14.204052\du}}
\pgfpathlineto{\pgfpoint{-8.865758\du}{14.195290\du}}
\pgfpathlineto{\pgfpoint{-8.840201\du}{14.186528\du}}
\pgfpathlineto{\pgfpoint{-8.815010\du}{14.178130\du}}
\pgfpathlineto{\pgfpoint{-8.788723\du}{14.170463\du}}
\pgfpathlineto{\pgfpoint{-8.761340\du}{14.164257\du}}
\pgfpathlineto{\pgfpoint{-8.733593\du}{14.157685\du}}
\pgfpathlineto{\pgfpoint{-8.705116\du}{14.152574\du}}
\pgfpathlineto{\pgfpoint{-8.675543\du}{14.148557\du}}
\pgfpathlineto{\pgfpoint{-8.645970\du}{14.144176\du}}
\pgfpathlineto{\pgfpoint{-8.615667\du}{14.141256\du}}
\pgfpathlineto{\pgfpoint{-8.584634\du}{14.139795\du}}
\pgfpathlineto{\pgfpoint{-8.553235\du}{14.137970\du}}
\pgfpathlineto{\pgfpoint{-8.522202\du}{14.137970\du}}
\pgfpathlineto{\pgfpoint{-8.490804\du}{14.137970\du}}
\pgfpathlineto{\pgfpoint{-8.459405\du}{14.139795\du}}
\pgfpathlineto{\pgfpoint{-8.428737\du}{14.141256\du}}
\pgfpathlineto{\pgfpoint{-8.398799\du}{14.144176\du}}
\pgfpathlineto{\pgfpoint{-8.368496\du}{14.148557\du}}
\pgfpathlineto{\pgfpoint{-8.338924\du}{14.152574\du}}
\pgfpathlineto{\pgfpoint{-8.310446\du}{14.157685\du}}
\pgfpathlineto{\pgfpoint{-8.282334\du}{14.164257\du}}
\pgfpathlineto{\pgfpoint{-8.255681\du}{14.170463\du}}
\pgfpathlineto{\pgfpoint{-8.229029\du}{14.178130\du}}
\pgfpathlineto{\pgfpoint{-8.203473\du}{14.186528\du}}
\pgfpathlineto{\pgfpoint{-8.178281\du}{14.195290\du}}
\pgfpathlineto{\pgfpoint{-8.154550\du}{14.204052\du}}
\pgfpathlineto{\pgfpoint{-8.131549\du}{14.214640\du}}
\pgfpathlineto{\pgfpoint{-8.108913\du}{14.225228\du}}
\pgfpathlineto{\pgfpoint{-8.087372\du}{14.236181\du}}
\pgfpathlineto{\pgfpoint{-8.067292\du}{14.247864\du}}
\pgfpathlineto{\pgfpoint{-8.048307\du}{14.260277\du}}
\pgfpathlineto{\pgfpoint{-8.030052\du}{14.272325\du}}
\pgfpathlineto{\pgfpoint{-8.012527\du}{14.285469\du}}
\pgfpathlineto{\pgfpoint{-7.996463\du}{14.299342\du}}
\pgfpathlineto{\pgfpoint{-7.981494\du}{14.313581\du}}
\pgfpathlineto{\pgfpoint{-7.967985\du}{14.328185\du}}
\pgfpathlineto{\pgfpoint{-7.956302\du}{14.342789\du}}
\pgfpathlineto{\pgfpoint{-7.944619\du}{14.357758\du}}
\pgfpathlineto{\pgfpoint{-7.935127\du}{14.373822\du}}
\pgfpathlineto{\pgfpoint{-7.926729\du}{14.389521\du}}
\pgfpathlineto{\pgfpoint{-7.920158\du}{14.405220\du}}
\pgfpathlineto{\pgfpoint{-7.914316\du}{14.422380\du}}
\pgfpathlineto{\pgfpoint{-7.910665\du}{14.438444\du}}
\pgfpathlineto{\pgfpoint{-7.908109\du}{14.455604\du}}
\pgfpathlineto{\pgfpoint{-7.907744\du}{14.473128\du}}
\pgfusepath{fill}
\pgfsetbuttcap
\pgfsetmiterjoin
\pgfsetdash{}{0pt}
\definecolor{dialinecolor}{rgb}{0.678431, 0.839216, 0.905882}
\pgfsetfillcolor{dialinecolor}
\pgfpathmoveto{\pgfpoint{-8.522202\du}{14.818875\du}}
\pgfpathlineto{\pgfpoint{-8.522202\du}{14.818875\du}}
\pgfpathlineto{\pgfpoint{-8.506138\du}{14.818875\du}}
\pgfpathlineto{\pgfpoint{-8.490074\du}{14.818510\du}}
\pgfpathlineto{\pgfpoint{-8.474009\du}{14.817780\du}}
\pgfpathlineto{\pgfpoint{-8.458675\du}{14.817050\du}}
\pgfpathlineto{\pgfpoint{-8.442976\du}{14.815954\du}}
\pgfpathlineto{\pgfpoint{-8.427642\du}{14.814859\du}}
\pgfpathlineto{\pgfpoint{-8.412673\du}{14.813764\du}}
\pgfpathlineto{\pgfpoint{-8.396609\du}{14.811938\du}}
\pgfpathlineto{\pgfpoint{-8.382370\du}{14.810113\du}}
\pgfpathlineto{\pgfpoint{-8.367401\du}{14.808287\du}}
\pgfpathlineto{\pgfpoint{-8.352432\du}{14.806097\du}}
\pgfpathlineto{\pgfpoint{-8.337098\du}{14.803906\du}}
\pgfpathlineto{\pgfpoint{-8.322859\du}{14.800985\du}}
\pgfpathlineto{\pgfpoint{-8.308986\du}{14.798430\du}}
\pgfpathlineto{\pgfpoint{-8.294382\du}{14.795509\du}}
\pgfpathlineto{\pgfpoint{-8.280508\du}{14.792588\du}}
\pgfpathlineto{\pgfpoint{-8.266999\du}{14.788937\du}}
\pgfpathlineto{\pgfpoint{-8.253126\du}{14.785651\du}}
\pgfpathlineto{\pgfpoint{-8.239617\du}{14.782000\du}}
\pgfpathlineto{\pgfpoint{-8.226474\du}{14.777984\du}}
\pgfpathlineto{\pgfpoint{-8.213330\du}{14.773968\du}}
\pgfpathlineto{\pgfpoint{-8.200187\du}{14.769952\du}}
\pgfpathlineto{\pgfpoint{-8.187408\du}{14.765206\du}}
\pgfpathlineto{\pgfpoint{-8.174630\du}{14.761190\du}}
\pgfpathlineto{\pgfpoint{-8.162947\du}{14.756444\du}}
\pgfpathlineto{\pgfpoint{-8.150169\du}{14.751697\du}}
\pgfpathlineto{\pgfpoint{-8.138851\du}{14.746221\du}}
\pgfpathlineto{\pgfpoint{-8.127167\du}{14.741110\du}}
\pgfpathlineto{\pgfpoint{-8.115849\du}{14.735998\du}}
\pgfpathlineto{\pgfpoint{-8.104531\du}{14.730522\du}}
\pgfpathlineto{\pgfpoint{-8.093579\du}{14.725410\du}}
\pgfpathlineto{\pgfpoint{-8.082261\du}{14.719569\du}}
\pgfpathlineto{\pgfpoint{-8.072403\du}{14.713727\du}}
\pgfpathlineto{\pgfpoint{-8.062180\du}{14.707156\du}}
\pgfpathlineto{\pgfpoint{-8.052323\du}{14.701314\du}}
\pgfpathlineto{\pgfpoint{-8.042100\du}{14.694742\du}}
\pgfpathlineto{\pgfpoint{-8.032607\du}{14.688536\du}}
\pgfpathlineto{\pgfpoint{-8.023480\du}{14.681964\du}}
\pgfpathlineto{\pgfpoint{-8.014718\du}{14.675757\du}}
\pgfpathlineto{\pgfpoint{-8.006320\du}{14.668455\du}}
\pgfpathlineto{\pgfpoint{-7.997923\du}{14.661518\du}}
\pgfpathlineto{\pgfpoint{-7.989891\du}{14.654582\du}}
\pgfpathlineto{\pgfpoint{-7.981859\du}{14.647645\du}}
\pgfpathlineto{\pgfpoint{-7.974922\du}{14.640343\du}}
\pgfpathlineto{\pgfpoint{-7.967255\du}{14.633041\du}}
\pgfpathlineto{\pgfpoint{-7.960683\du}{14.625374\du}}
\pgfpathlineto{\pgfpoint{-7.954112\du}{14.617707\du}}
\pgfpathlineto{\pgfpoint{-7.947905\du}{14.610040\du}}
\pgfpathlineto{\pgfpoint{-7.941698\du}{14.602008\du}}
\pgfpathlineto{\pgfpoint{-7.935857\du}{14.594341\du}}
\pgfpathlineto{\pgfpoint{-7.931110\du}{14.585943\du}}
\pgfpathlineto{\pgfpoint{-7.926364\du}{14.577911\du}}
\pgfpathlineto{\pgfpoint{-7.921618\du}{14.569514\du}}
\pgfpathlineto{\pgfpoint{-7.917237\du}{14.561482\du}}
\pgfpathlineto{\pgfpoint{-7.913951\du}{14.552720\du}}
\pgfpathlineto{\pgfpoint{-7.909935\du}{14.544687\du}}
\pgfpathlineto{\pgfpoint{-7.907379\du}{14.535925\du}}
\pgfpathlineto{\pgfpoint{-7.904458\du}{14.527163\du}}
\pgfpathlineto{\pgfpoint{-7.902633\du}{14.518035\du}}
\pgfpathlineto{\pgfpoint{-7.900442\du}{14.509273\du}}
\pgfpathlineto{\pgfpoint{-7.898982\du}{14.500511\du}}
\pgfpathlineto{\pgfpoint{-7.897887\du}{14.491383\du}}
\pgfpathlineto{\pgfpoint{-7.897156\du}{14.482621\du}}
\pgfpathlineto{\pgfpoint{-7.897156\du}{14.473128\du}}
\pgfpathlineto{\pgfpoint{-7.917237\du}{14.473128\du}}
\pgfpathlineto{\pgfpoint{-7.917967\du}{14.481161\du}}
\pgfpathlineto{\pgfpoint{-7.918332\du}{14.489558\du}}
\pgfpathlineto{\pgfpoint{-7.919062\du}{14.497590\du}}
\pgfpathlineto{\pgfpoint{-7.920523\du}{14.505257\du}}
\pgfpathlineto{\pgfpoint{-7.921983\du}{14.513654\du}}
\pgfpathlineto{\pgfpoint{-7.924539\du}{14.521686\du}}
\pgfpathlineto{\pgfpoint{-7.926729\du}{14.529353\du}}
\pgfpathlineto{\pgfpoint{-7.930015\du}{14.537020\du}}
\pgfpathlineto{\pgfpoint{-7.932571\du}{14.544687\du}}
\pgfpathlineto{\pgfpoint{-7.935857\du}{14.552720\du}}
\pgfpathlineto{\pgfpoint{-7.939873\du}{14.560387\du}}
\pgfpathlineto{\pgfpoint{-7.944619\du}{14.568054\du}}
\pgfpathlineto{\pgfpoint{-7.948635\du}{14.575721\du}}
\pgfpathlineto{\pgfpoint{-7.953746\du}{14.582658\du}}
\pgfpathlineto{\pgfpoint{-7.958493\du}{14.590325\du}}
\pgfpathlineto{\pgfpoint{-7.963969\du}{14.597261\du}}
\pgfpathlineto{\pgfpoint{-7.969081\du}{14.604928\du}}
\pgfpathlineto{\pgfpoint{-7.976017\du}{14.611865\du}}
\pgfpathlineto{\pgfpoint{-7.981859\du}{14.618802\du}}
\pgfpathlineto{\pgfpoint{-7.989161\du}{14.625739\du}}
\pgfpathlineto{\pgfpoint{-7.996098\du}{14.633041\du}}
\pgfpathlineto{\pgfpoint{-8.003400\du}{14.639248\du}}
\pgfpathlineto{\pgfpoint{-8.010336\du}{14.646184\du}}
\pgfpathlineto{\pgfpoint{-8.018734\du}{14.652756\du}}
\pgfpathlineto{\pgfpoint{-8.026766\du}{14.659328\du}}
\pgfpathlineto{\pgfpoint{-8.035893\du}{14.665534\du}}
\pgfpathlineto{\pgfpoint{-8.044656\du}{14.671376\du}}
\pgfpathlineto{\pgfpoint{-8.054148\du}{14.677948\du}}
\pgfpathlineto{\pgfpoint{-8.063276\du}{14.684520\du}}
\pgfpathlineto{\pgfpoint{-8.072403\du}{14.689631\du}}
\pgfpathlineto{\pgfpoint{-8.082261\du}{14.695472\du}}
\pgfpathlineto{\pgfpoint{-8.092118\du}{14.701314\du}}
\pgfpathlineto{\pgfpoint{-8.102706\du}{14.706790\du}}
\pgfpathlineto{\pgfpoint{-8.113294\du}{14.712632\du}}
\pgfpathlineto{\pgfpoint{-8.123882\du}{14.717013\du}}
\pgfpathlineto{\pgfpoint{-8.135565\du}{14.722855\du}}
\pgfpathlineto{\pgfpoint{-8.146883\du}{14.727601\du}}
\pgfpathlineto{\pgfpoint{-8.158201\du}{14.732347\du}}
\pgfpathlineto{\pgfpoint{-8.169884\du}{14.737093\du}}
\pgfpathlineto{\pgfpoint{-8.181932\du}{14.741840\du}}
\pgfpathlineto{\pgfpoint{-8.194345\du}{14.745856\du}}
\pgfpathlineto{\pgfpoint{-8.206028\du}{14.750602\du}}
\pgfpathlineto{\pgfpoint{-8.218807\du}{14.754618\du}}
\pgfpathlineto{\pgfpoint{-8.232315\du}{14.758269\du}}
\pgfpathlineto{\pgfpoint{-8.245094\du}{14.762285\du}}
\pgfpathlineto{\pgfpoint{-8.258237\du}{14.765571\du}}
\pgfpathlineto{\pgfpoint{-8.271746\du}{14.769222\du}}
\pgfpathlineto{\pgfpoint{-8.285254\du}{14.772143\du}}
\pgfpathlineto{\pgfpoint{-8.299128\du}{14.775794\du}}
\pgfpathlineto{\pgfpoint{-8.313002\du}{14.777984\du}}
\pgfpathlineto{\pgfpoint{-8.326875\du}{14.780905\du}}
\pgfpathlineto{\pgfpoint{-8.341114\du}{14.783096\du}}
\pgfpathlineto{\pgfpoint{-8.355353\du}{14.785651\du}}
\pgfpathlineto{\pgfpoint{-8.369592\du}{14.787842\du}}
\pgfpathlineto{\pgfpoint{-8.384196\du}{14.789667\du}}
\pgfpathlineto{\pgfpoint{-8.399165\du}{14.791493\du}}
\pgfpathlineto{\pgfpoint{-8.414499\du}{14.793318\du}}
\pgfpathlineto{\pgfpoint{-8.429102\du}{14.794414\du}}
\pgfpathlineto{\pgfpoint{-8.444802\du}{14.795509\du}}
\pgfpathlineto{\pgfpoint{-8.459771\du}{14.796604\du}}
\pgfpathlineto{\pgfpoint{-8.475105\du}{14.797334\du}}
\pgfpathlineto{\pgfpoint{-8.490804\du}{14.798065\du}}
\pgfpathlineto{\pgfpoint{-8.506138\du}{14.798430\du}}
\pgfpathlineto{\pgfpoint{-8.522202\du}{14.798430\du}}
\pgfpathlineto{\pgfpoint{-8.522202\du}{14.798430\du}}
\pgfpathlineto{\pgfpoint{-8.522202\du}{14.798430\du}}
\pgfpathlineto{\pgfpoint{-8.523297\du}{14.798430\du}}
\pgfpathlineto{\pgfpoint{-8.524393\du}{14.798430\du}}
\pgfpathlineto{\pgfpoint{-8.525123\du}{14.799160\du}}
\pgfpathlineto{\pgfpoint{-8.526948\du}{14.799160\du}}
\pgfpathlineto{\pgfpoint{-8.527313\du}{14.799525\du}}
\pgfpathlineto{\pgfpoint{-8.528409\du}{14.800255\du}}
\pgfpathlineto{\pgfpoint{-8.528774\du}{14.800985\du}}
\pgfpathlineto{\pgfpoint{-8.529139\du}{14.801350\du}}
\pgfpathlineto{\pgfpoint{-8.530599\du}{14.803176\du}}
\pgfpathlineto{\pgfpoint{-8.532060\du}{14.805001\du}}
\pgfpathlineto{\pgfpoint{-8.532060\du}{14.806827\du}}
\pgfpathlineto{\pgfpoint{-8.532425\du}{14.809018\du}}
\pgfpathlineto{\pgfpoint{-8.532060\du}{14.810843\du}}
\pgfpathlineto{\pgfpoint{-8.532060\du}{14.812668\du}}
\pgfpathlineto{\pgfpoint{-8.530599\du}{14.814129\du}}
\pgfpathlineto{\pgfpoint{-8.529139\du}{14.815589\du}}
\pgfpathlineto{\pgfpoint{-8.528774\du}{14.816685\du}}
\pgfpathlineto{\pgfpoint{-8.528409\du}{14.817050\du}}
\pgfpathlineto{\pgfpoint{-8.527313\du}{14.817780\du}}
\pgfpathlineto{\pgfpoint{-8.526948\du}{14.817780\du}}
\pgfpathlineto{\pgfpoint{-8.525123\du}{14.818510\du}}
\pgfpathlineto{\pgfpoint{-8.524393\du}{14.818875\du}}
\pgfpathlineto{\pgfpoint{-8.523297\du}{14.818875\du}}
\pgfpathlineto{\pgfpoint{-8.522202\du}{14.818875\du}}
\pgfusepath{fill}
\pgfsetbuttcap
\pgfsetmiterjoin
\pgfsetdash{}{0pt}
\definecolor{dialinecolor}{rgb}{0.678431, 0.839216, 0.905882}
\pgfsetfillcolor{dialinecolor}
\pgfpathmoveto{\pgfpoint{-9.146883\du}{14.473128\du}}
\pgfpathlineto{\pgfpoint{-9.146883\du}{14.473128\du}}
\pgfpathlineto{\pgfpoint{-9.146883\du}{14.481891\du}}
\pgfpathlineto{\pgfpoint{-9.146152\du}{14.491383\du}}
\pgfpathlineto{\pgfpoint{-9.145057\du}{14.500511\du}}
\pgfpathlineto{\pgfpoint{-9.142867\du}{14.509273\du}}
\pgfpathlineto{\pgfpoint{-9.141771\du}{14.518035\du}}
\pgfpathlineto{\pgfpoint{-9.139216\du}{14.527163\du}}
\pgfpathlineto{\pgfpoint{-9.137025\du}{14.535925\du}}
\pgfpathlineto{\pgfpoint{-9.133739\du}{14.544687\du}}
\pgfpathlineto{\pgfpoint{-9.130453\du}{14.552720\du}}
\pgfpathlineto{\pgfpoint{-9.126437\du}{14.561482\du}}
\pgfpathlineto{\pgfpoint{-9.122056\du}{14.569514\du}}
\pgfpathlineto{\pgfpoint{-9.117675\du}{14.577911\du}}
\pgfpathlineto{\pgfpoint{-9.112929\du}{14.585943\du}}
\pgfpathlineto{\pgfpoint{-9.107817\du}{14.594341\du}}
\pgfpathlineto{\pgfpoint{-9.101611\du}{14.602008\du}}
\pgfpathlineto{\pgfpoint{-9.096864\du}{14.610040\du}}
\pgfpathlineto{\pgfpoint{-9.090293\du}{14.617707\du}}
\pgfpathlineto{\pgfpoint{-9.083356\du}{14.625374\du}}
\pgfpathlineto{\pgfpoint{-9.076784\du}{14.633041\du}}
\pgfpathlineto{\pgfpoint{-9.069847\du}{14.640343\du}}
\pgfpathlineto{\pgfpoint{-9.062545\du}{14.647645\du}}
\pgfpathlineto{\pgfpoint{-9.053783\du}{14.654582\du}}
\pgfpathlineto{\pgfpoint{-9.046846\du}{14.661518\du}}
\pgfpathlineto{\pgfpoint{-9.037719\du}{14.668455\du}}
\pgfpathlineto{\pgfpoint{-9.029687\du}{14.675757\du}}
\pgfpathlineto{\pgfpoint{-9.020559\du}{14.681964\du}}
\pgfpathlineto{\pgfpoint{-9.010702\du}{14.688536\du}}
\pgfpathlineto{\pgfpoint{-9.001574\du}{14.694742\du}}
\pgfpathlineto{\pgfpoint{-8.992082\du}{14.701314\du}}
\pgfpathlineto{\pgfpoint{-8.982224\du}{14.707156\du}}
\pgfpathlineto{\pgfpoint{-8.972001\du}{14.713727\du}}
\pgfpathlineto{\pgfpoint{-8.961779\du}{14.719569\du}}
\pgfpathlineto{\pgfpoint{-8.950826\du}{14.725410\du}}
\pgfpathlineto{\pgfpoint{-8.939508\du}{14.730522\du}}
\pgfpathlineto{\pgfpoint{-8.928190\du}{14.735998\du}}
\pgfpathlineto{\pgfpoint{-8.916872\du}{14.741110\du}}
\pgfpathlineto{\pgfpoint{-8.904824\du}{14.746221\du}}
\pgfpathlineto{\pgfpoint{-8.893506\du}{14.751697\du}}
\pgfpathlineto{\pgfpoint{-8.881092\du}{14.756444\du}}
\pgfpathlineto{\pgfpoint{-8.869044\du}{14.761190\du}}
\pgfpathlineto{\pgfpoint{-8.856996\du}{14.765206\du}}
\pgfpathlineto{\pgfpoint{-8.843852\du}{14.769952\du}}
\pgfpathlineto{\pgfpoint{-8.831074\du}{14.773968\du}}
\pgfpathlineto{\pgfpoint{-8.817565\du}{14.777984\du}}
\pgfpathlineto{\pgfpoint{-8.804057\du}{14.782000\du}}
\pgfpathlineto{\pgfpoint{-8.791278\du}{14.785651\du}}
\pgfpathlineto{\pgfpoint{-8.777770\du}{14.788937\du}}
\pgfpathlineto{\pgfpoint{-8.763896\du}{14.792588\du}}
\pgfpathlineto{\pgfpoint{-8.749292\du}{14.795509\du}}
\pgfpathlineto{\pgfpoint{-8.734688\du}{14.798430\du}}
\pgfpathlineto{\pgfpoint{-8.720815\du}{14.800985\du}}
\pgfpathlineto{\pgfpoint{-8.706576\du}{14.803906\du}}
\pgfpathlineto{\pgfpoint{-8.691607\du}{14.806097\du}}
\pgfpathlineto{\pgfpoint{-8.676638\du}{14.808287\du}}
\pgfpathlineto{\pgfpoint{-8.662034\du}{14.810113\du}}
\pgfpathlineto{\pgfpoint{-8.647430\du}{14.811938\du}}
\pgfpathlineto{\pgfpoint{-8.631731\du}{14.813764\du}}
\pgfpathlineto{\pgfpoint{-8.616032\du}{14.814859\du}}
\pgfpathlineto{\pgfpoint{-8.601428\du}{14.815954\du}}
\pgfpathlineto{\pgfpoint{-8.585364\du}{14.817050\du}}
\pgfpathlineto{\pgfpoint{-8.570030\du}{14.817780\du}}
\pgfpathlineto{\pgfpoint{-8.553966\du}{14.818510\du}}
\pgfpathlineto{\pgfpoint{-8.537901\du}{14.818875\du}}
\pgfpathlineto{\pgfpoint{-8.522202\du}{14.818875\du}}
\pgfpathlineto{\pgfpoint{-8.522202\du}{14.798430\du}}
\pgfpathlineto{\pgfpoint{-8.537901\du}{14.798430\du}}
\pgfpathlineto{\pgfpoint{-8.553235\du}{14.798065\du}}
\pgfpathlineto{\pgfpoint{-8.569300\du}{14.797334\du}}
\pgfpathlineto{\pgfpoint{-8.583903\du}{14.796604\du}}
\pgfpathlineto{\pgfpoint{-8.599603\du}{14.795509\du}}
\pgfpathlineto{\pgfpoint{-8.614572\du}{14.794414\du}}
\pgfpathlineto{\pgfpoint{-8.629541\du}{14.793318\du}}
\pgfpathlineto{\pgfpoint{-8.644510\du}{14.791493\du}}
\pgfpathlineto{\pgfpoint{-8.659479\du}{14.789667\du}}
\pgfpathlineto{\pgfpoint{-8.674082\du}{14.787842\du}}
\pgfpathlineto{\pgfpoint{-8.688686\du}{14.785651\du}}
\pgfpathlineto{\pgfpoint{-8.702925\du}{14.783096\du}}
\pgfpathlineto{\pgfpoint{-8.716799\du}{14.780905\du}}
\pgfpathlineto{\pgfpoint{-8.730672\du}{14.777984\du}}
\pgfpathlineto{\pgfpoint{-8.744911\du}{14.775794\du}}
\pgfpathlineto{\pgfpoint{-8.758785\du}{14.772143\du}}
\pgfpathlineto{\pgfpoint{-8.772293\du}{14.769222\du}}
\pgfpathlineto{\pgfpoint{-8.785437\du}{14.765571\du}}
\pgfpathlineto{\pgfpoint{-8.798945\du}{14.762285\du}}
\pgfpathlineto{\pgfpoint{-8.811724\du}{14.758269\du}}
\pgfpathlineto{\pgfpoint{-8.824867\du}{14.754618\du}}
\pgfpathlineto{\pgfpoint{-8.837646\du}{14.750602\du}}
\pgfpathlineto{\pgfpoint{-8.849694\du}{14.745856\du}}
\pgfpathlineto{\pgfpoint{-8.862107\du}{14.741840\du}}
\pgfpathlineto{\pgfpoint{-8.874520\du}{14.737093\du}}
\pgfpathlineto{\pgfpoint{-8.885838\du}{14.732347\du}}
\pgfpathlineto{\pgfpoint{-8.897522\du}{14.727601\du}}
\pgfpathlineto{\pgfpoint{-8.908474\du}{14.722855\du}}
\pgfpathlineto{\pgfpoint{-8.919792\du}{14.717013\du}}
\pgfpathlineto{\pgfpoint{-8.930380\du}{14.712632\du}}
\pgfpathlineto{\pgfpoint{-8.941333\du}{14.706790\du}}
\pgfpathlineto{\pgfpoint{-8.951556\du}{14.701314\du}}
\pgfpathlineto{\pgfpoint{-8.961779\du}{14.695472\du}}
\pgfpathlineto{\pgfpoint{-8.971636\du}{14.689631\du}}
\pgfpathlineto{\pgfpoint{-8.981129\du}{14.684520\du}}
\pgfpathlineto{\pgfpoint{-8.990986\du}{14.677948\du}}
\pgfpathlineto{\pgfpoint{-8.999384\du}{14.671376\du}}
\pgfpathlineto{\pgfpoint{-9.008511\du}{14.665534\du}}
\pgfpathlineto{\pgfpoint{-9.016908\du}{14.659328\du}}
\pgfpathlineto{\pgfpoint{-9.025305\du}{14.652756\du}}
\pgfpathlineto{\pgfpoint{-9.033338\du}{14.646184\du}}
\pgfpathlineto{\pgfpoint{-9.040640\du}{14.639248\du}}
\pgfpathlineto{\pgfpoint{-9.048307\du}{14.633041\du}}
\pgfpathlineto{\pgfpoint{-9.054878\du}{14.625739\du}}
\pgfpathlineto{\pgfpoint{-9.061815\du}{14.618802\du}}
\pgfpathlineto{\pgfpoint{-9.068022\du}{14.611865\du}}
\pgfpathlineto{\pgfpoint{-9.074228\du}{14.604928\du}}
\pgfpathlineto{\pgfpoint{-9.080435\du}{14.597261\du}}
\pgfpathlineto{\pgfpoint{-9.085546\du}{14.590325\du}}
\pgfpathlineto{\pgfpoint{-9.090658\du}{14.582658\du}}
\pgfpathlineto{\pgfpoint{-9.095404\du}{14.575721\du}}
\pgfpathlineto{\pgfpoint{-9.099785\du}{14.568054\du}}
\pgfpathlineto{\pgfpoint{-9.104166\du}{14.560387\du}}
\pgfpathlineto{\pgfpoint{-9.107817\du}{14.552720\du}}
\pgfpathlineto{\pgfpoint{-9.111103\du}{14.544687\du}}
\pgfpathlineto{\pgfpoint{-9.114389\du}{14.537020\du}}
\pgfpathlineto{\pgfpoint{-9.117310\du}{14.529353\du}}
\pgfpathlineto{\pgfpoint{-9.119866\du}{14.521686\du}}
\pgfpathlineto{\pgfpoint{-9.121691\du}{14.513654\du}}
\pgfpathlineto{\pgfpoint{-9.123516\du}{14.505257\du}}
\pgfpathlineto{\pgfpoint{-9.124612\du}{14.497590\du}}
\pgfpathlineto{\pgfpoint{-9.126072\du}{14.489558\du}}
\pgfpathlineto{\pgfpoint{-9.126437\du}{14.481161\du}}
\pgfpathlineto{\pgfpoint{-9.126437\du}{14.473128\du}}
\pgfpathlineto{\pgfpoint{-9.126437\du}{14.473128\du}}
\pgfpathlineto{\pgfpoint{-9.126437\du}{14.473128\du}}
\pgfpathlineto{\pgfpoint{-9.126437\du}{14.472033\du}}
\pgfpathlineto{\pgfpoint{-9.126437\du}{14.470938\du}}
\pgfpathlineto{\pgfpoint{-9.126802\du}{14.469478\du}}
\pgfpathlineto{\pgfpoint{-9.126802\du}{14.468382\du}}
\pgfpathlineto{\pgfpoint{-9.127167\du}{14.468017\du}}
\pgfpathlineto{\pgfpoint{-9.128263\du}{14.466557\du}}
\pgfpathlineto{\pgfpoint{-9.128993\du}{14.466192\du}}
\pgfpathlineto{\pgfpoint{-9.128993\du}{14.465461\du}}
\pgfpathlineto{\pgfpoint{-9.131184\du}{14.464366\du}}
\pgfpathlineto{\pgfpoint{-9.133009\du}{14.463636\du}}
\pgfpathlineto{\pgfpoint{-9.134834\du}{14.463271\du}}
\pgfpathlineto{\pgfpoint{-9.137025\du}{14.462541\du}}
\pgfpathlineto{\pgfpoint{-9.138485\du}{14.463271\du}}
\pgfpathlineto{\pgfpoint{-9.140311\du}{14.463636\du}}
\pgfpathlineto{\pgfpoint{-9.142136\du}{14.464366\du}}
\pgfpathlineto{\pgfpoint{-9.143962\du}{14.465461\du}}
\pgfpathlineto{\pgfpoint{-9.144692\du}{14.466192\du}}
\pgfpathlineto{\pgfpoint{-9.145057\du}{14.466557\du}}
\pgfpathlineto{\pgfpoint{-9.145422\du}{14.468017\du}}
\pgfpathlineto{\pgfpoint{-9.146152\du}{14.468382\du}}
\pgfpathlineto{\pgfpoint{-9.146152\du}{14.469478\du}}
\pgfpathlineto{\pgfpoint{-9.146883\du}{14.470938\du}}
\pgfpathlineto{\pgfpoint{-9.146883\du}{14.472033\du}}
\pgfpathlineto{\pgfpoint{-9.146883\du}{14.473128\du}}
\pgfusepath{fill}
\pgfsetbuttcap
\pgfsetmiterjoin
\pgfsetdash{}{0pt}
\definecolor{dialinecolor}{rgb}{0.678431, 0.839216, 0.905882}
\pgfsetfillcolor{dialinecolor}
\pgfpathmoveto{\pgfpoint{-8.522202\du}{14.127382\du}}
\pgfpathlineto{\pgfpoint{-8.522202\du}{14.127382\du}}
\pgfpathlineto{\pgfpoint{-8.537901\du}{14.127382\du}}
\pgfpathlineto{\pgfpoint{-8.553966\du}{14.127747\du}}
\pgfpathlineto{\pgfpoint{-8.570030\du}{14.128477\du}}
\pgfpathlineto{\pgfpoint{-8.585364\du}{14.129207\du}}
\pgfpathlineto{\pgfpoint{-8.601428\du}{14.130303\du}}
\pgfpathlineto{\pgfpoint{-8.616032\du}{14.131398\du}}
\pgfpathlineto{\pgfpoint{-8.631731\du}{14.132493\du}}
\pgfpathlineto{\pgfpoint{-8.647430\du}{14.134319\du}}
\pgfpathlineto{\pgfpoint{-8.662034\du}{14.136144\du}}
\pgfpathlineto{\pgfpoint{-8.676638\du}{14.137970\du}}
\pgfpathlineto{\pgfpoint{-8.691607\du}{14.140160\du}}
\pgfpathlineto{\pgfpoint{-8.706576\du}{14.142716\du}}
\pgfpathlineto{\pgfpoint{-8.720815\du}{14.145637\du}}
\pgfpathlineto{\pgfpoint{-8.734688\du}{14.147827\du}}
\pgfpathlineto{\pgfpoint{-8.749292\du}{14.150748\du}}
\pgfpathlineto{\pgfpoint{-8.763896\du}{14.153669\du}}
\pgfpathlineto{\pgfpoint{-8.777770\du}{14.157320\du}}
\pgfpathlineto{\pgfpoint{-8.791278\du}{14.160606\du}}
\pgfpathlineto{\pgfpoint{-8.804057\du}{14.164622\du}}
\pgfpathlineto{\pgfpoint{-8.817565\du}{14.168273\du}}
\pgfpathlineto{\pgfpoint{-8.831074\du}{14.172289\du}}
\pgfpathlineto{\pgfpoint{-8.843852\du}{14.176670\du}}
\pgfpathlineto{\pgfpoint{-8.856996\du}{14.181051\du}}
\pgfpathlineto{\pgfpoint{-8.869044\du}{14.185432\du}}
\pgfpathlineto{\pgfpoint{-8.881092\du}{14.189813\du}}
\pgfpathlineto{\pgfpoint{-8.893506\du}{14.195290\du}}
\pgfpathlineto{\pgfpoint{-8.904824\du}{14.200036\du}}
\pgfpathlineto{\pgfpoint{-8.916872\du}{14.205147\du}}
\pgfpathlineto{\pgfpoint{-8.928190\du}{14.210259\du}}
\pgfpathlineto{\pgfpoint{-8.939508\du}{14.215735\du}}
\pgfpathlineto{\pgfpoint{-8.950826\du}{14.221577\du}}
\pgfpathlineto{\pgfpoint{-8.961779\du}{14.226688\du}}
\pgfpathlineto{\pgfpoint{-8.972001\du}{14.232530\du}}
\pgfpathlineto{\pgfpoint{-8.982224\du}{14.239101\du}}
\pgfpathlineto{\pgfpoint{-8.992082\du}{14.244943\du}}
\pgfpathlineto{\pgfpoint{-9.001574\du}{14.251515\du}}
\pgfpathlineto{\pgfpoint{-9.010702\du}{14.257721\du}}
\pgfpathlineto{\pgfpoint{-9.020559\du}{14.264293\du}}
\pgfpathlineto{\pgfpoint{-9.029687\du}{14.270865\du}}
\pgfpathlineto{\pgfpoint{-9.037719\du}{14.277802\du}}
\pgfpathlineto{\pgfpoint{-9.046846\du}{14.284739\du}}
\pgfpathlineto{\pgfpoint{-9.053783\du}{14.291675\du}}
\pgfpathlineto{\pgfpoint{-9.062545\du}{14.298612\du}}
\pgfpathlineto{\pgfpoint{-9.069847\du}{14.305914\du}}
\pgfpathlineto{\pgfpoint{-9.076784\du}{14.313581\du}}
\pgfpathlineto{\pgfpoint{-9.083356\du}{14.320883\du}}
\pgfpathlineto{\pgfpoint{-9.090293\du}{14.328550\du}}
\pgfpathlineto{\pgfpoint{-9.096864\du}{14.336217\du}}
\pgfpathlineto{\pgfpoint{-9.101611\du}{14.344249\du}}
\pgfpathlineto{\pgfpoint{-9.107817\du}{14.351916\du}}
\pgfpathlineto{\pgfpoint{-9.112929\du}{14.360314\du}}
\pgfpathlineto{\pgfpoint{-9.117675\du}{14.368346\du}}
\pgfpathlineto{\pgfpoint{-9.122056\du}{14.376743\du}}
\pgfpathlineto{\pgfpoint{-9.126437\du}{14.384775\du}}
\pgfpathlineto{\pgfpoint{-9.130453\du}{14.393537\du}}
\pgfpathlineto{\pgfpoint{-9.133739\du}{14.401935\du}}
\pgfpathlineto{\pgfpoint{-9.137025\du}{14.410697\du}}
\pgfpathlineto{\pgfpoint{-9.139216\du}{14.419459\du}}
\pgfpathlineto{\pgfpoint{-9.141771\du}{14.428222\du}}
\pgfpathlineto{\pgfpoint{-9.142867\du}{14.436984\du}}
\pgfpathlineto{\pgfpoint{-9.145057\du}{14.445746\du}}
\pgfpathlineto{\pgfpoint{-9.146152\du}{14.454874\du}}
\pgfpathlineto{\pgfpoint{-9.146883\du}{14.464366\du}}
\pgfpathlineto{\pgfpoint{-9.146883\du}{14.473128\du}}
\pgfpathlineto{\pgfpoint{-9.126437\du}{14.473128\du}}
\pgfpathlineto{\pgfpoint{-9.126437\du}{14.465096\du}}
\pgfpathlineto{\pgfpoint{-9.126072\du}{14.456699\du}}
\pgfpathlineto{\pgfpoint{-9.124612\du}{14.448667\du}}
\pgfpathlineto{\pgfpoint{-9.123516\du}{14.441000\du}}
\pgfpathlineto{\pgfpoint{-9.121691\du}{14.432603\du}}
\pgfpathlineto{\pgfpoint{-9.119866\du}{14.424571\du}}
\pgfpathlineto{\pgfpoint{-9.117310\du}{14.416904\du}}
\pgfpathlineto{\pgfpoint{-9.114389\du}{14.409237\du}}
\pgfpathlineto{\pgfpoint{-9.111103\du}{14.401935\du}}
\pgfpathlineto{\pgfpoint{-9.107817\du}{14.393537\du}}
\pgfpathlineto{\pgfpoint{-9.104166\du}{14.385870\du}}
\pgfpathlineto{\pgfpoint{-9.099785\du}{14.378203\du}}
\pgfpathlineto{\pgfpoint{-9.095404\du}{14.371266\du}}
\pgfpathlineto{\pgfpoint{-9.090658\du}{14.363599\du}}
\pgfpathlineto{\pgfpoint{-9.085546\du}{14.356298\du}}
\pgfpathlineto{\pgfpoint{-9.080435\du}{14.348996\du}}
\pgfpathlineto{\pgfpoint{-9.074228\du}{14.341329\du}}
\pgfpathlineto{\pgfpoint{-9.068022\du}{14.334392\du}}
\pgfpathlineto{\pgfpoint{-9.061815\du}{14.327455\du}}
\pgfpathlineto{\pgfpoint{-9.054878\du}{14.320518\du}}
\pgfpathlineto{\pgfpoint{-9.048307\du}{14.313946\du}}
\pgfpathlineto{\pgfpoint{-9.040640\du}{14.307009\du}}
\pgfpathlineto{\pgfpoint{-9.033338\du}{14.300073\du}}
\pgfpathlineto{\pgfpoint{-9.025305\du}{14.293501\du}}
\pgfpathlineto{\pgfpoint{-9.016908\du}{14.286929\du}}
\pgfpathlineto{\pgfpoint{-9.008511\du}{14.280723\du}}
\pgfpathlineto{\pgfpoint{-8.999384\du}{14.274881\du}}
\pgfpathlineto{\pgfpoint{-8.990986\du}{14.268309\du}}
\pgfpathlineto{\pgfpoint{-8.981129\du}{14.262468\du}}
\pgfpathlineto{\pgfpoint{-8.971636\du}{14.256626\du}}
\pgfpathlineto{\pgfpoint{-8.961779\du}{14.250785\du}}
\pgfpathlineto{\pgfpoint{-8.951556\du}{14.244943\du}}
\pgfpathlineto{\pgfpoint{-8.941333\du}{14.239832\du}}
\pgfpathlineto{\pgfpoint{-8.930380\du}{14.233990\du}}
\pgfpathlineto{\pgfpoint{-8.919792\du}{14.229244\du}}
\pgfpathlineto{\pgfpoint{-8.908474\du}{14.223767\du}}
\pgfpathlineto{\pgfpoint{-8.897522\du}{14.218656\du}}
\pgfpathlineto{\pgfpoint{-8.885838\du}{14.213910\du}}
\pgfpathlineto{\pgfpoint{-8.874520\du}{14.209164\du}}
\pgfpathlineto{\pgfpoint{-8.862107\du}{14.204417\du}}
\pgfpathlineto{\pgfpoint{-8.849694\du}{14.200401\du}}
\pgfpathlineto{\pgfpoint{-8.837646\du}{14.195655\du}}
\pgfpathlineto{\pgfpoint{-8.824867\du}{14.191639\du}}
\pgfpathlineto{\pgfpoint{-8.811724\du}{14.188353\du}}
\pgfpathlineto{\pgfpoint{-8.798945\du}{14.183972\du}}
\pgfpathlineto{\pgfpoint{-8.785437\du}{14.180686\du}}
\pgfpathlineto{\pgfpoint{-8.772293\du}{14.177035\du}}
\pgfpathlineto{\pgfpoint{-8.758785\du}{14.174114\du}}
\pgfpathlineto{\pgfpoint{-8.744911\du}{14.171193\du}}
\pgfpathlineto{\pgfpoint{-8.730672\du}{14.168273\du}}
\pgfpathlineto{\pgfpoint{-8.716799\du}{14.165352\du}}
\pgfpathlineto{\pgfpoint{-8.702925\du}{14.163161\du}}
\pgfpathlineto{\pgfpoint{-8.688686\du}{14.160606\du}}
\pgfpathlineto{\pgfpoint{-8.674082\du}{14.158415\du}}
\pgfpathlineto{\pgfpoint{-8.659479\du}{14.156590\du}}
\pgfpathlineto{\pgfpoint{-8.644510\du}{14.154764\du}}
\pgfpathlineto{\pgfpoint{-8.629541\du}{14.152939\du}}
\pgfpathlineto{\pgfpoint{-8.614572\du}{14.151843\du}}
\pgfpathlineto{\pgfpoint{-8.599603\du}{14.150748\du}}
\pgfpathlineto{\pgfpoint{-8.583903\du}{14.149653\du}}
\pgfpathlineto{\pgfpoint{-8.569300\du}{14.148923\du}}
\pgfpathlineto{\pgfpoint{-8.553235\du}{14.148557\du}}
\pgfpathlineto{\pgfpoint{-8.537901\du}{14.147827\du}}
\pgfpathlineto{\pgfpoint{-8.522202\du}{14.147827\du}}
\pgfpathlineto{\pgfpoint{-8.522202\du}{14.147827\du}}
\pgfpathlineto{\pgfpoint{-8.522202\du}{14.147827\du}}
\pgfpathlineto{\pgfpoint{-8.520742\du}{14.147827\du}}
\pgfpathlineto{\pgfpoint{-8.519646\du}{14.147827\du}}
\pgfpathlineto{\pgfpoint{-8.518551\du}{14.147827\du}}
\pgfpathlineto{\pgfpoint{-8.517456\du}{14.147097\du}}
\pgfpathlineto{\pgfpoint{-8.516361\du}{14.146732\du}}
\pgfpathlineto{\pgfpoint{-8.515265\du}{14.146002\du}}
\pgfpathlineto{\pgfpoint{-8.515265\du}{14.145637\du}}
\pgfpathlineto{\pgfpoint{-8.514535\du}{14.144907\du}}
\pgfpathlineto{\pgfpoint{-8.513440\du}{14.143081\du}}
\pgfpathlineto{\pgfpoint{-8.512345\du}{14.141256\du}}
\pgfpathlineto{\pgfpoint{-8.511979\du}{14.139795\du}}
\pgfpathlineto{\pgfpoint{-8.511979\du}{14.137970\du}}
\pgfpathlineto{\pgfpoint{-8.511979\du}{14.135414\du}}
\pgfpathlineto{\pgfpoint{-8.512345\du}{14.133954\du}}
\pgfpathlineto{\pgfpoint{-8.513440\du}{14.132128\du}}
\pgfpathlineto{\pgfpoint{-8.514535\du}{14.131033\du}}
\pgfpathlineto{\pgfpoint{-8.515265\du}{14.129572\du}}
\pgfpathlineto{\pgfpoint{-8.515265\du}{14.129207\du}}
\pgfpathlineto{\pgfpoint{-8.516361\du}{14.128477\du}}
\pgfpathlineto{\pgfpoint{-8.517456\du}{14.128477\du}}
\pgfpathlineto{\pgfpoint{-8.518551\du}{14.127747\du}}
\pgfpathlineto{\pgfpoint{-8.519646\du}{14.127747\du}}
\pgfpathlineto{\pgfpoint{-8.520742\du}{14.127382\du}}
\pgfpathlineto{\pgfpoint{-8.522202\du}{14.127382\du}}
\pgfusepath{fill}
\pgfsetbuttcap
\pgfsetmiterjoin
\pgfsetdash{}{0pt}
\definecolor{dialinecolor}{rgb}{0.678431, 0.839216, 0.905882}
\pgfsetfillcolor{dialinecolor}
\pgfpathmoveto{\pgfpoint{-7.897156\du}{14.473128\du}}
\pgfpathlineto{\pgfpoint{-7.897156\du}{14.463636\du}}
\pgfpathlineto{\pgfpoint{-7.897887\du}{14.454874\du}}
\pgfpathlineto{\pgfpoint{-7.898982\du}{14.445746\du}}
\pgfpathlineto{\pgfpoint{-7.900442\du}{14.436984\du}}
\pgfpathlineto{\pgfpoint{-7.902633\du}{14.428222\du}}
\pgfpathlineto{\pgfpoint{-7.904458\du}{14.419459\du}}
\pgfpathlineto{\pgfpoint{-7.907379\du}{14.410697\du}}
\pgfpathlineto{\pgfpoint{-7.909935\du}{14.401935\du}}
\pgfpathlineto{\pgfpoint{-7.913951\du}{14.393537\du}}
\pgfpathlineto{\pgfpoint{-7.917237\du}{14.384775\du}}
\pgfpathlineto{\pgfpoint{-7.921618\du}{14.376743\du}}
\pgfpathlineto{\pgfpoint{-7.926364\du}{14.368346\du}}
\pgfpathlineto{\pgfpoint{-7.931110\du}{14.360314\du}}
\pgfpathlineto{\pgfpoint{-7.935857\du}{14.351916\du}}
\pgfpathlineto{\pgfpoint{-7.941698\du}{14.344249\du}}
\pgfpathlineto{\pgfpoint{-7.947905\du}{14.336217\du}}
\pgfpathlineto{\pgfpoint{-7.954112\du}{14.328550\du}}
\pgfpathlineto{\pgfpoint{-7.960683\du}{14.320883\du}}
\pgfpathlineto{\pgfpoint{-7.967255\du}{14.313581\du}}
\pgfpathlineto{\pgfpoint{-7.974922\du}{14.305914\du}}
\pgfpathlineto{\pgfpoint{-7.981859\du}{14.298612\du}}
\pgfpathlineto{\pgfpoint{-7.989891\du}{14.291675\du}}
\pgfpathlineto{\pgfpoint{-7.997923\du}{14.284739\du}}
\pgfpathlineto{\pgfpoint{-8.006320\du}{14.277802\du}}
\pgfpathlineto{\pgfpoint{-8.014718\du}{14.270865\du}}
\pgfpathlineto{\pgfpoint{-8.023480\du}{14.264293\du}}
\pgfpathlineto{\pgfpoint{-8.032607\du}{14.257721\du}}
\pgfpathlineto{\pgfpoint{-8.042100\du}{14.251515\du}}
\pgfpathlineto{\pgfpoint{-8.052323\du}{14.244943\du}}
\pgfpathlineto{\pgfpoint{-8.062180\du}{14.239101\du}}
\pgfpathlineto{\pgfpoint{-8.072403\du}{14.232530\du}}
\pgfpathlineto{\pgfpoint{-8.082261\du}{14.226688\du}}
\pgfpathlineto{\pgfpoint{-8.093579\du}{14.221577\du}}
\pgfpathlineto{\pgfpoint{-8.104531\du}{14.215735\du}}
\pgfpathlineto{\pgfpoint{-8.115849\du}{14.210259\du}}
\pgfpathlineto{\pgfpoint{-8.127167\du}{14.205147\du}}
\pgfpathlineto{\pgfpoint{-8.138851\du}{14.200036\du}}
\pgfpathlineto{\pgfpoint{-8.150169\du}{14.195290\du}}
\pgfpathlineto{\pgfpoint{-8.162947\du}{14.189813\du}}
\pgfpathlineto{\pgfpoint{-8.174630\du}{14.185432\du}}
\pgfpathlineto{\pgfpoint{-8.187408\du}{14.181051\du}}
\pgfpathlineto{\pgfpoint{-8.200187\du}{14.176670\du}}
\pgfpathlineto{\pgfpoint{-8.213330\du}{14.172289\du}}
\pgfpathlineto{\pgfpoint{-8.226474\du}{14.168273\du}}
\pgfpathlineto{\pgfpoint{-8.239617\du}{14.164622\du}}
\pgfpathlineto{\pgfpoint{-8.253126\du}{14.160606\du}}
\pgfpathlineto{\pgfpoint{-8.266999\du}{14.157320\du}}
\pgfpathlineto{\pgfpoint{-8.280508\du}{14.153669\du}}
\pgfpathlineto{\pgfpoint{-8.294382\du}{14.150748\du}}
\pgfpathlineto{\pgfpoint{-8.308986\du}{14.147827\du}}
\pgfpathlineto{\pgfpoint{-8.322859\du}{14.145637\du}}
\pgfpathlineto{\pgfpoint{-8.337098\du}{14.142716\du}}
\pgfpathlineto{\pgfpoint{-8.352432\du}{14.140160\du}}
\pgfpathlineto{\pgfpoint{-8.367401\du}{14.137970\du}}
\pgfpathlineto{\pgfpoint{-8.382370\du}{14.136144\du}}
\pgfpathlineto{\pgfpoint{-8.396609\du}{14.134319\du}}
\pgfpathlineto{\pgfpoint{-8.412673\du}{14.132493\du}}
\pgfpathlineto{\pgfpoint{-8.427642\du}{14.131398\du}}
\pgfpathlineto{\pgfpoint{-8.442976\du}{14.130303\du}}
\pgfpathlineto{\pgfpoint{-8.458675\du}{14.129207\du}}
\pgfpathlineto{\pgfpoint{-8.474009\du}{14.128477\du}}
\pgfpathlineto{\pgfpoint{-8.490074\du}{14.127747\du}}
\pgfpathlineto{\pgfpoint{-8.506138\du}{14.127382\du}}
\pgfpathlineto{\pgfpoint{-8.522202\du}{14.127382\du}}
\pgfpathlineto{\pgfpoint{-8.522202\du}{14.147827\du}}
\pgfpathlineto{\pgfpoint{-8.506138\du}{14.147827\du}}
\pgfpathlineto{\pgfpoint{-8.490804\du}{14.148557\du}}
\pgfpathlineto{\pgfpoint{-8.475105\du}{14.148923\du}}
\pgfpathlineto{\pgfpoint{-8.459771\du}{14.149653\du}}
\pgfpathlineto{\pgfpoint{-8.444802\du}{14.150748\du}}
\pgfpathlineto{\pgfpoint{-8.429102\du}{14.151843\du}}
\pgfpathlineto{\pgfpoint{-8.414499\du}{14.152939\du}}
\pgfpathlineto{\pgfpoint{-8.399165\du}{14.154764\du}}
\pgfpathlineto{\pgfpoint{-8.384196\du}{14.156590\du}}
\pgfpathlineto{\pgfpoint{-8.369592\du}{14.158415\du}}
\pgfpathlineto{\pgfpoint{-8.355353\du}{14.160606\du}}
\pgfpathlineto{\pgfpoint{-8.341114\du}{14.163161\du}}
\pgfpathlineto{\pgfpoint{-8.326875\du}{14.165352\du}}
\pgfpathlineto{\pgfpoint{-8.313002\du}{14.168273\du}}
\pgfpathlineto{\pgfpoint{-8.299128\du}{14.171193\du}}
\pgfpathlineto{\pgfpoint{-8.285254\du}{14.174114\du}}
\pgfpathlineto{\pgfpoint{-8.271746\du}{14.177035\du}}
\pgfpathlineto{\pgfpoint{-8.258237\du}{14.180686\du}}
\pgfpathlineto{\pgfpoint{-8.245094\du}{14.183972\du}}
\pgfpathlineto{\pgfpoint{-8.232315\du}{14.188353\du}}
\pgfpathlineto{\pgfpoint{-8.218807\du}{14.191639\du}}
\pgfpathlineto{\pgfpoint{-8.206028\du}{14.195655\du}}
\pgfpathlineto{\pgfpoint{-8.194345\du}{14.200401\du}}
\pgfpathlineto{\pgfpoint{-8.181932\du}{14.204417\du}}
\pgfpathlineto{\pgfpoint{-8.169884\du}{14.209164\du}}
\pgfpathlineto{\pgfpoint{-8.158201\du}{14.213910\du}}
\pgfpathlineto{\pgfpoint{-8.146883\du}{14.218656\du}}
\pgfpathlineto{\pgfpoint{-8.135565\du}{14.223767\du}}
\pgfpathlineto{\pgfpoint{-8.123882\du}{14.229244\du}}
\pgfpathlineto{\pgfpoint{-8.113294\du}{14.233990\du}}
\pgfpathlineto{\pgfpoint{-8.102706\du}{14.239832\du}}
\pgfpathlineto{\pgfpoint{-8.092118\du}{14.244943\du}}
\pgfpathlineto{\pgfpoint{-8.082261\du}{14.250785\du}}
\pgfpathlineto{\pgfpoint{-8.072403\du}{14.256626\du}}
\pgfpathlineto{\pgfpoint{-8.063276\du}{14.262468\du}}
\pgfpathlineto{\pgfpoint{-8.054148\du}{14.268309\du}}
\pgfpathlineto{\pgfpoint{-8.044656\du}{14.274881\du}}
\pgfpathlineto{\pgfpoint{-8.035893\du}{14.280723\du}}
\pgfpathlineto{\pgfpoint{-8.026766\du}{14.286929\du}}
\pgfpathlineto{\pgfpoint{-8.018734\du}{14.293501\du}}
\pgfpathlineto{\pgfpoint{-8.010336\du}{14.300073\du}}
\pgfpathlineto{\pgfpoint{-8.003400\du}{14.307009\du}}
\pgfpathlineto{\pgfpoint{-7.996098\du}{14.313946\du}}
\pgfpathlineto{\pgfpoint{-7.989161\du}{14.320518\du}}
\pgfpathlineto{\pgfpoint{-7.981859\du}{14.327455\du}}
\pgfpathlineto{\pgfpoint{-7.976017\du}{14.334392\du}}
\pgfpathlineto{\pgfpoint{-7.969081\du}{14.341329\du}}
\pgfpathlineto{\pgfpoint{-7.963969\du}{14.348996\du}}
\pgfpathlineto{\pgfpoint{-7.958493\du}{14.356298\du}}
\pgfpathlineto{\pgfpoint{-7.953746\du}{14.363599\du}}
\pgfpathlineto{\pgfpoint{-7.948635\du}{14.371266\du}}
\pgfpathlineto{\pgfpoint{-7.944619\du}{14.378203\du}}
\pgfpathlineto{\pgfpoint{-7.939873\du}{14.385870\du}}
\pgfpathlineto{\pgfpoint{-7.935857\du}{14.393537\du}}
\pgfpathlineto{\pgfpoint{-7.932571\du}{14.401935\du}}
\pgfpathlineto{\pgfpoint{-7.930015\du}{14.409237\du}}
\pgfpathlineto{\pgfpoint{-7.926729\du}{14.416904\du}}
\pgfpathlineto{\pgfpoint{-7.924539\du}{14.424571\du}}
\pgfpathlineto{\pgfpoint{-7.921983\du}{14.432603\du}}
\pgfpathlineto{\pgfpoint{-7.920523\du}{14.441000\du}}
\pgfpathlineto{\pgfpoint{-7.919062\du}{14.448667\du}}
\pgfpathlineto{\pgfpoint{-7.918332\du}{14.456699\du}}
\pgfpathlineto{\pgfpoint{-7.917967\du}{14.465096\du}}
\pgfpathlineto{\pgfpoint{-7.917237\du}{14.473128\du}}
\pgfpathlineto{\pgfpoint{-7.897156\du}{14.473128\du}}
\pgfusepath{fill}
\pgfsetbuttcap
\pgfsetmiterjoin
\pgfsetdash{}{0pt}
\definecolor{dialinecolor}{rgb}{0.074510, 0.082353, 0.086275}
\pgfsetfillcolor{dialinecolor}
\pgfpathmoveto{\pgfpoint{-8.843122\du}{14.567323\du}}
\pgfpathlineto{\pgfpoint{-8.615667\du}{14.339138\du}}
\pgfpathlineto{\pgfpoint{-8.675543\du}{14.278167\du}}
\pgfpathlineto{\pgfpoint{-8.495185\du}{14.278167\du}}
\pgfpathlineto{\pgfpoint{-8.495185\du}{14.466557\du}}
\pgfpathlineto{\pgfpoint{-8.555426\du}{14.406316\du}}
\pgfpathlineto{\pgfpoint{-8.775579\du}{14.627564\du}}
\pgfpathlineto{\pgfpoint{-8.843122\du}{14.567323\du}}
\pgfusepath{fill}
\pgfsetbuttcap
\pgfsetmiterjoin
\pgfsetdash{}{0pt}
\definecolor{dialinecolor}{rgb}{0.074510, 0.082353, 0.086275}
\pgfsetfillcolor{dialinecolor}
\pgfpathmoveto{\pgfpoint{-8.575141\du}{14.687805\du}}
\pgfpathlineto{\pgfpoint{-8.348051\du}{14.459620\du}}
\pgfpathlineto{\pgfpoint{-8.408657\du}{14.399379\du}}
\pgfpathlineto{\pgfpoint{-8.227569\du}{14.399379\du}}
\pgfpathlineto{\pgfpoint{-8.227569\du}{14.587404\du}}
\pgfpathlineto{\pgfpoint{-8.288175\du}{14.527163\du}}
\pgfpathlineto{\pgfpoint{-8.508694\du}{14.748046\du}}
\pgfpathlineto{\pgfpoint{-8.575141\du}{14.687805\du}}
\pgfusepath{fill}
\pgfsetbuttcap
\pgfsetmiterjoin
\pgfsetdash{}{0pt}
\definecolor{dialinecolor}{rgb}{1.000000, 1.000000, 1.000000}
\pgfsetfillcolor{dialinecolor}
\pgfpathmoveto{\pgfpoint{-8.856266\du}{14.553815\du}}
\pgfpathlineto{\pgfpoint{-8.629175\du}{14.325629\du}}
\pgfpathlineto{\pgfpoint{-8.688686\du}{14.265388\du}}
\pgfpathlineto{\pgfpoint{-8.508694\du}{14.265388\du}}
\pgfpathlineto{\pgfpoint{-8.508694\du}{14.453413\du}}
\pgfpathlineto{\pgfpoint{-8.568569\du}{14.392442\du}}
\pgfpathlineto{\pgfpoint{-8.789088\du}{14.614056\du}}
\pgfpathlineto{\pgfpoint{-8.856266\du}{14.553815\du}}
\pgfusepath{fill}
\pgfsetbuttcap
\pgfsetmiterjoin
\pgfsetdash{}{0pt}
\definecolor{dialinecolor}{rgb}{1.000000, 1.000000, 1.000000}
\pgfsetfillcolor{dialinecolor}
\pgfpathmoveto{\pgfpoint{-8.588650\du}{14.674297\du}}
\pgfpathlineto{\pgfpoint{-8.361560\du}{14.446111\du}}
\pgfpathlineto{\pgfpoint{-8.421801\du}{14.385870\du}}
\pgfpathlineto{\pgfpoint{-8.241443\du}{14.385870\du}}
\pgfpathlineto{\pgfpoint{-8.241443\du}{14.573895\du}}
\pgfpathlineto{\pgfpoint{-8.300953\du}{14.513654\du}}
\pgfpathlineto{\pgfpoint{-8.522202\du}{14.734538\du}}
\pgfpathlineto{\pgfpoint{-8.588650\du}{14.674297\du}}
\pgfusepath{fill}
\pgfsetlinewidth{0.000000\du}
\pgfsetdash{}{0pt}
\pgfsetdash{}{0pt}
\pgfsetbuttcap
\pgfsetmiterjoin
\pgfsetlinewidth{0.000000\du}
\pgfsetbuttcap
\pgfsetmiterjoin
\pgfsetdash{}{0pt}
\definecolor{dialinecolor}{rgb}{0.027451, 0.486275, 0.682353}
\pgfsetfillcolor{dialinecolor}
\pgfpathmoveto{\pgfpoint{3.261674\du}{14.264349\du}}
\pgfpathlineto{\pgfpoint{3.260213\du}{14.293557\du}}
\pgfpathlineto{\pgfpoint{3.252911\du}{14.323495\du}}
\pgfpathlineto{\pgfpoint{3.242689\du}{14.351972\du}}
\pgfpathlineto{\pgfpoint{3.228085\du}{14.380085\du}}
\pgfpathlineto{\pgfpoint{3.208735\du}{14.408197\du}}
\pgfpathlineto{\pgfpoint{3.186829\du}{14.435579\du}}
\pgfpathlineto{\pgfpoint{3.160177\du}{14.462597\du}}
\pgfpathlineto{\pgfpoint{3.129509\du}{14.488883\du}}
\pgfpathlineto{\pgfpoint{3.096650\du}{14.514075\du}}
\pgfpathlineto{\pgfpoint{3.059410\du}{14.539267\du}}
\pgfpathlineto{\pgfpoint{3.018519\du}{14.563363\du}}
\pgfpathlineto{\pgfpoint{2.974708\du}{14.586729\du}}
\pgfpathlineto{\pgfpoint{2.928340\du}{14.609365\du}}
\pgfpathlineto{\pgfpoint{2.877957\du}{14.631271\du}}
\pgfpathlineto{\pgfpoint{2.825018\du}{14.652082\du}}
\pgfpathlineto{\pgfpoint{2.769523\du}{14.672162\du}}
\pgfpathlineto{\pgfpoint{2.711473\du}{14.691512\du}}
\pgfpathlineto{\pgfpoint{2.650867\du}{14.709402\du}}
\pgfpathlineto{\pgfpoint{2.586975\du}{14.726561\du}}
\pgfpathlineto{\pgfpoint{2.521988\du}{14.742991\du}}
\pgfpathlineto{\pgfpoint{2.453349\du}{14.757960\du}}
\pgfpathlineto{\pgfpoint{2.382521\du}{14.771468\du}}
\pgfpathlineto{\pgfpoint{2.310597\du}{14.784247\du}}
\pgfpathlineto{\pgfpoint{2.235752\du}{14.796295\du}}
\pgfpathlineto{\pgfpoint{2.159812\du}{14.806153\du}}
\pgfpathlineto{\pgfpoint{2.081316\du}{14.815280\du}}
\pgfpathlineto{\pgfpoint{2.001725\du}{14.822947\du}}
\pgfpathlineto{\pgfpoint{1.920673\du}{14.829519\du}}
\pgfpathlineto{\pgfpoint{1.837431\du}{14.834630\du}}
\pgfpathlineto{\pgfpoint{1.753824\du}{14.838281\du}}
\pgfpathlineto{\pgfpoint{1.668391\du}{14.840472\du}}
\pgfpathlineto{\pgfpoint{1.581863\du}{14.841202\du}}
\pgfpathlineto{\pgfpoint{1.495701\du}{14.840472\du}}
\pgfpathlineto{\pgfpoint{1.409903\du}{14.838281\du}}
\pgfpathlineto{\pgfpoint{1.326296\du}{14.834630\du}}
\pgfpathlineto{\pgfpoint{1.243419\du}{14.829519\du}}
\pgfpathlineto{\pgfpoint{1.162002\du}{14.822947\du}}
\pgfpathlineto{\pgfpoint{1.082411\du}{14.815280\du}}
\pgfpathlineto{\pgfpoint{1.004646\du}{14.806153\du}}
\pgfpathlineto{\pgfpoint{0.927975\du}{14.796295\du}}
\pgfpathlineto{\pgfpoint{0.853861\du}{14.784247\du}}
\pgfpathlineto{\pgfpoint{0.781206\du}{14.771468\du}}
\pgfpathlineto{\pgfpoint{0.710743\du}{14.757960\du}}
\pgfpathlineto{\pgfpoint{0.642104\du}{14.742991\du}}
\pgfpathlineto{\pgfpoint{0.576387\du}{14.726561\du}}
\pgfpathlineto{\pgfpoint{0.512860\du}{14.709402\du}}
\pgfpathlineto{\pgfpoint{0.451889\du}{14.691512\du}}
\pgfpathlineto{\pgfpoint{0.393474\du}{14.672162\du}}
\pgfpathlineto{\pgfpoint{0.338344\du}{14.652082\du}}
\pgfpathlineto{\pgfpoint{0.285405\du}{14.631271\du}}
\pgfpathlineto{\pgfpoint{0.235387\du}{14.609365\du}}
\pgfpathlineto{\pgfpoint{0.188289\du}{14.586729\du}}
\pgfpathlineto{\pgfpoint{0.144843\du}{14.563363\du}}
\pgfpathlineto{\pgfpoint{0.103952\du}{14.539267\du}}
\pgfpathlineto{\pgfpoint{0.066712\du}{14.514075\du}}
\pgfpathlineto{\pgfpoint{0.033488\du}{14.488883\du}}
\pgfpathlineto{\pgfpoint{0.003185\du}{14.462597\du}}
\pgfpathlineto{\pgfpoint{-0.023467\du}{14.435579\du}}
\pgfpathlineto{\pgfpoint{-0.045373\du}{14.408197\du}}
\pgfpathlineto{\pgfpoint{-0.064723\du}{14.380085\du}}
\pgfpathlineto{\pgfpoint{-0.079327\du}{14.351972\du}}
\pgfpathlineto{\pgfpoint{-0.089915\du}{14.323495\du}}
\pgfpathlineto{\pgfpoint{-0.096851\du}{14.293557\du}}
\pgfpathlineto{\pgfpoint{-0.098677\du}{14.264349\du}}
\pgfpathlineto{\pgfpoint{-0.096851\du}{14.234411\du}}
\pgfpathlineto{\pgfpoint{-0.089915\du}{14.205203\du}}
\pgfpathlineto{\pgfpoint{-0.079327\du}{14.175996\du}}
\pgfpathlineto{\pgfpoint{-0.064723\du}{14.147883\du}}
\pgfpathlineto{\pgfpoint{-0.045373\du}{14.119771\du}}
\pgfpathlineto{\pgfpoint{-0.023467\du}{14.092388\du}}
\pgfpathlineto{\pgfpoint{0.003185\du}{14.065736\du}}
\pgfpathlineto{\pgfpoint{0.033488\du}{14.039449\du}}
\pgfpathlineto{\pgfpoint{0.066712\du}{14.013893\du}}
\pgfpathlineto{\pgfpoint{0.103952\du}{13.989066\du}}
\pgfpathlineto{\pgfpoint{0.144843\du}{13.964605\du}}
\pgfpathlineto{\pgfpoint{0.188289\du}{13.941238\du}}
\pgfpathlineto{\pgfpoint{0.235387\du}{13.918967\du}}
\pgfpathlineto{\pgfpoint{0.285405\du}{13.896697\du}}
\pgfpathlineto{\pgfpoint{0.338344\du}{13.875886\du}}
\pgfpathlineto{\pgfpoint{0.393474\du}{13.855806\du}}
\pgfpathlineto{\pgfpoint{0.451889\du}{13.837186\du}}
\pgfpathlineto{\pgfpoint{0.512860\du}{13.818201\du}}
\pgfpathlineto{\pgfpoint{0.576387\du}{13.801406\du}}
\pgfpathlineto{\pgfpoint{0.642104\du}{13.785707\du}}
\pgfpathlineto{\pgfpoint{0.710743\du}{13.770373\du}}
\pgfpathlineto{\pgfpoint{0.781206\du}{13.756499\du}}
\pgfpathlineto{\pgfpoint{0.853861\du}{13.743356\du}}
\pgfpathlineto{\pgfpoint{0.927975\du}{13.732403\du}}
\pgfpathlineto{\pgfpoint{1.004646\du}{13.721815\du}}
\pgfpathlineto{\pgfpoint{1.082411\du}{13.712323\du}}
\pgfpathlineto{\pgfpoint{1.162002\du}{13.705021\du}}
\pgfpathlineto{\pgfpoint{1.243419\du}{13.698449\du}}
\pgfpathlineto{\pgfpoint{1.326296\du}{13.693338\du}}
\pgfpathlineto{\pgfpoint{1.409903\du}{13.689687\du}}
\pgfpathlineto{\pgfpoint{1.495701\du}{13.687861\du}}
\pgfpathlineto{\pgfpoint{1.581863\du}{13.686766\du}}
\pgfpathlineto{\pgfpoint{1.668391\du}{13.687861\du}}
\pgfpathlineto{\pgfpoint{1.753824\du}{13.689687\du}}
\pgfpathlineto{\pgfpoint{1.837431\du}{13.693338\du}}
\pgfpathlineto{\pgfpoint{1.920673\du}{13.698449\du}}
\pgfpathlineto{\pgfpoint{2.001725\du}{13.705021\du}}
\pgfpathlineto{\pgfpoint{2.081316\du}{13.712323\du}}
\pgfpathlineto{\pgfpoint{2.159812\du}{13.721815\du}}
\pgfpathlineto{\pgfpoint{2.235752\du}{13.732403\du}}
\pgfpathlineto{\pgfpoint{2.310597\du}{13.743356\du}}
\pgfpathlineto{\pgfpoint{2.382521\du}{13.756499\du}}
\pgfpathlineto{\pgfpoint{2.453349\du}{13.770373\du}}
\pgfpathlineto{\pgfpoint{2.521988\du}{13.785707\du}}
\pgfpathlineto{\pgfpoint{2.586975\du}{13.801406\du}}
\pgfpathlineto{\pgfpoint{2.650867\du}{13.818201\du}}
\pgfpathlineto{\pgfpoint{2.711473\du}{13.837186\du}}
\pgfpathlineto{\pgfpoint{2.769523\du}{13.855806\du}}
\pgfpathlineto{\pgfpoint{2.825018\du}{13.875886\du}}
\pgfpathlineto{\pgfpoint{2.877957\du}{13.896697\du}}
\pgfpathlineto{\pgfpoint{2.928340\du}{13.918967\du}}
\pgfpathlineto{\pgfpoint{2.974708\du}{13.941238\du}}
\pgfpathlineto{\pgfpoint{3.018519\du}{13.964605\du}}
\pgfpathlineto{\pgfpoint{3.059410\du}{13.989066\du}}
\pgfpathlineto{\pgfpoint{3.096650\du}{14.013893\du}}
\pgfpathlineto{\pgfpoint{3.129509\du}{14.039449\du}}
\pgfpathlineto{\pgfpoint{3.160177\du}{14.065736\du}}
\pgfpathlineto{\pgfpoint{3.186829\du}{14.092388\du}}
\pgfpathlineto{\pgfpoint{3.208735\du}{14.119771\du}}
\pgfpathlineto{\pgfpoint{3.228085\du}{14.147883\du}}
\pgfpathlineto{\pgfpoint{3.242689\du}{14.175996\du}}
\pgfpathlineto{\pgfpoint{3.252911\du}{14.205203\du}}
\pgfpathlineto{\pgfpoint{3.260213\du}{14.234411\du}}
\pgfpathlineto{\pgfpoint{3.261674\du}{14.264349\du}}
\pgfusepath{fill}
\pgfsetlinewidth{0.000000\du}
\pgfsetbuttcap
\pgfsetmiterjoin
\pgfsetdash{}{0pt}
\definecolor{dialinecolor}{rgb}{0.678431, 0.839216, 0.905882}
\pgfsetfillcolor{dialinecolor}
\pgfpathmoveto{\pgfpoint{1.581863\du}{14.851790\du}}
\pgfpathlineto{\pgfpoint{1.581863\du}{14.851790\du}}
\pgfpathlineto{\pgfpoint{1.625310\du}{14.851790\du}}
\pgfpathlineto{\pgfpoint{1.668756\du}{14.851059\du}}
\pgfpathlineto{\pgfpoint{1.711838\du}{14.849964\du}}
\pgfpathlineto{\pgfpoint{1.753824\du}{14.848869\du}}
\pgfpathlineto{\pgfpoint{1.796540\du}{14.847043\du}}
\pgfpathlineto{\pgfpoint{1.838161\du}{14.844853\du}}
\pgfpathlineto{\pgfpoint{1.879782\du}{14.842297\du}}
\pgfpathlineto{\pgfpoint{1.921403\du}{14.840107\du}}
\pgfpathlineto{\pgfpoint{1.961929\du}{14.837186\du}}
\pgfpathlineto{\pgfpoint{2.002820\du}{14.833535\du}}
\pgfpathlineto{\pgfpoint{2.042616\du}{14.829519\du}}
\pgfpathlineto{\pgfpoint{2.082776\du}{14.825503\du}}
\pgfpathlineto{\pgfpoint{2.121476\du}{14.821122\du}}
\pgfpathlineto{\pgfpoint{2.160907\du}{14.816740\du}}
\pgfpathlineto{\pgfpoint{2.198877\du}{14.811264\du}}
\pgfpathlineto{\pgfpoint{2.237942\du}{14.806153\du}}
\pgfpathlineto{\pgfpoint{2.275182\du}{14.800676\du}}
\pgfpathlineto{\pgfpoint{2.312057\du}{14.794469\du}}
\pgfpathlineto{\pgfpoint{2.348567\du}{14.788628\du}}
\pgfpathlineto{\pgfpoint{2.385076\du}{14.782056\du}}
\pgfpathlineto{\pgfpoint{2.420491\du}{14.775119\du}}
\pgfpathlineto{\pgfpoint{2.455175\du}{14.768182\du}}
\pgfpathlineto{\pgfpoint{2.489859\du}{14.760515\du}}
\pgfpathlineto{\pgfpoint{2.523813\du}{14.752848\du}}
\pgfpathlineto{\pgfpoint{2.557402\du}{14.744451\du}}
\pgfpathlineto{\pgfpoint{2.590261\du}{14.736419\du}}
\pgfpathlineto{\pgfpoint{2.622024\du}{14.728387\du}}
\pgfpathlineto{\pgfpoint{2.653422\du}{14.719625\du}}
\pgfpathlineto{\pgfpoint{2.668756\du}{14.714878\du}}
\pgfpathlineto{\pgfpoint{2.684091\du}{14.710862\du}}
\pgfpathlineto{\pgfpoint{2.700155\du}{14.706116\du}}
\pgfpathlineto{\pgfpoint{2.714759\du}{14.701370\du}}
\pgfpathlineto{\pgfpoint{2.728997\du}{14.696624\du}}
\pgfpathlineto{\pgfpoint{2.743966\du}{14.691512\du}}
\pgfpathlineto{\pgfpoint{2.758935\du}{14.686766\du}}
\pgfpathlineto{\pgfpoint{2.772809\du}{14.682020\du}}
\pgfpathlineto{\pgfpoint{2.787048\du}{14.676908\du}}
\pgfpathlineto{\pgfpoint{2.800922\du}{14.672162\du}}
\pgfpathlineto{\pgfpoint{2.815160\du}{14.666686\du}}
\pgfpathlineto{\pgfpoint{2.828304\du}{14.662304\du}}
\pgfpathlineto{\pgfpoint{2.842543\du}{14.656828\du}}
\pgfpathlineto{\pgfpoint{2.855686\du}{14.651717\du}}
\pgfpathlineto{\pgfpoint{2.868830\du}{14.646240\du}}
\pgfpathlineto{\pgfpoint{2.882338\du}{14.640399\du}}
\pgfpathlineto{\pgfpoint{2.895116\du}{14.635287\du}}
\pgfpathlineto{\pgfpoint{2.907165\du}{14.629811\du}}
\pgfpathlineto{\pgfpoint{2.919943\du}{14.623969\du}}
\pgfpathlineto{\pgfpoint{2.932356\du}{14.618858\du}}
\pgfpathlineto{\pgfpoint{2.944770\du}{14.613016\du}}
\pgfpathlineto{\pgfpoint{2.956088\du}{14.607540\du}}
\pgfpathlineto{\pgfpoint{2.968136\du}{14.601698\du}}
\pgfpathlineto{\pgfpoint{2.979089\du}{14.595857\du}}
\pgfpathlineto{\pgfpoint{2.990772\du}{14.590015\du}}
\pgfpathlineto{\pgfpoint{3.002090\du}{14.584174\du}}
\pgfpathlineto{\pgfpoint{3.013408\du}{14.578332\du}}
\pgfpathlineto{\pgfpoint{3.023996\du}{14.572126\du}}
\pgfpathlineto{\pgfpoint{3.033853\du}{14.566284\du}}
\pgfpathlineto{\pgfpoint{3.044441\du}{14.560442\du}}
\pgfpathlineto{\pgfpoint{3.054664\du}{14.553871\du}}
\pgfpathlineto{\pgfpoint{3.064521\du}{14.548029\du}}
\pgfpathlineto{\pgfpoint{3.074379\du}{14.541457\du}}
\pgfpathlineto{\pgfpoint{3.083506\du}{14.535251\du}}
\pgfpathlineto{\pgfpoint{3.092634\du}{14.528679\du}}
\pgfpathlineto{\pgfpoint{3.102126\du}{14.522837\du}}
\pgfpathlineto{\pgfpoint{3.110889\du}{14.516266\du}}
\pgfpathlineto{\pgfpoint{3.120016\du}{14.510059\du}}
\pgfpathlineto{\pgfpoint{3.128048\du}{14.503487\du}}
\pgfpathlineto{\pgfpoint{3.136811\du}{14.496551\du}}
\pgfpathlineto{\pgfpoint{3.144112\du}{14.489979\du}}
\pgfpathlineto{\pgfpoint{3.152145\du}{14.483772\du}}
\pgfpathlineto{\pgfpoint{3.160177\du}{14.476470\du}}
\pgfpathlineto{\pgfpoint{3.166748\du}{14.470264\du}}
\pgfpathlineto{\pgfpoint{3.174050\du}{14.463327\du}}
\pgfpathlineto{\pgfpoint{3.180622\du}{14.456025\du}}
\pgfpathlineto{\pgfpoint{3.187559\du}{14.449818\du}}
\pgfpathlineto{\pgfpoint{3.194131\du}{14.442881\du}}
\pgfpathlineto{\pgfpoint{3.200337\du}{14.435579\du}}
\pgfpathlineto{\pgfpoint{3.206179\du}{14.428643\du}}
\pgfpathlineto{\pgfpoint{3.211655\du}{14.421706\du}}
\pgfpathlineto{\pgfpoint{3.217497\du}{14.414769\du}}
\pgfpathlineto{\pgfpoint{3.222243\du}{14.407467\du}}
\pgfpathlineto{\pgfpoint{3.227720\du}{14.400165\du}}
\pgfpathlineto{\pgfpoint{3.232466\du}{14.392863\du}}
\pgfpathlineto{\pgfpoint{3.236847\du}{14.385926\du}}
\pgfpathlineto{\pgfpoint{3.240863\du}{14.378259\du}}
\pgfpathlineto{\pgfpoint{3.244879\du}{14.371322\du}}
\pgfpathlineto{\pgfpoint{3.248165\du}{14.363655\du}}
\pgfpathlineto{\pgfpoint{3.252181\du}{14.355988\du}}
\pgfpathlineto{\pgfpoint{3.255467\du}{14.348321\du}}
\pgfpathlineto{\pgfpoint{3.258023\du}{14.341019\du}}
\pgfpathlineto{\pgfpoint{3.260943\du}{14.333717\du}}
\pgfpathlineto{\pgfpoint{3.262769\du}{14.326415\du}}
\pgfpathlineto{\pgfpoint{3.265690\du}{14.318748\du}}
\pgfpathlineto{\pgfpoint{3.266785\du}{14.310351\du}}
\pgfpathlineto{\pgfpoint{3.268976\du}{14.302684\du}}
\pgfpathlineto{\pgfpoint{3.270071\du}{14.295382\du}}
\pgfpathlineto{\pgfpoint{3.270801\du}{14.287715\du}}
\pgfpathlineto{\pgfpoint{3.271531\du}{14.279318\du}}
\pgfpathlineto{\pgfpoint{3.272261\du}{14.272016\du}}
\pgfpathlineto{\pgfpoint{3.272261\du}{14.264349\du}}
\pgfpathlineto{\pgfpoint{3.252181\du}{14.264349\du}}
\pgfpathlineto{\pgfpoint{3.251451\du}{14.271286\du}}
\pgfpathlineto{\pgfpoint{3.251451\du}{14.278223\du}}
\pgfpathlineto{\pgfpoint{3.251086\du}{14.285159\du}}
\pgfpathlineto{\pgfpoint{3.249260\du}{14.292461\du}}
\pgfpathlineto{\pgfpoint{3.248165\du}{14.299398\du}}
\pgfpathlineto{\pgfpoint{3.247435\du}{14.306335\du}}
\pgfpathlineto{\pgfpoint{3.245244\du}{14.313272\du}}
\pgfpathlineto{\pgfpoint{3.243784\du}{14.320574\du}}
\pgfpathlineto{\pgfpoint{3.241593\du}{14.326781\du}}
\pgfpathlineto{\pgfpoint{3.239038\du}{14.333717\du}}
\pgfpathlineto{\pgfpoint{3.236117\du}{14.341019\du}}
\pgfpathlineto{\pgfpoint{3.233561\du}{14.347956\du}}
\pgfpathlineto{\pgfpoint{3.229545\du}{14.354893\du}}
\pgfpathlineto{\pgfpoint{3.226624\du}{14.361465\du}}
\pgfpathlineto{\pgfpoint{3.222608\du}{14.368402\du}}
\pgfpathlineto{\pgfpoint{3.219687\du}{14.374973\du}}
\pgfpathlineto{\pgfpoint{3.215306\du}{14.381910\du}}
\pgfpathlineto{\pgfpoint{3.210925\du}{14.388847\du}}
\pgfpathlineto{\pgfpoint{3.206179\du}{14.395419\du}}
\pgfpathlineto{\pgfpoint{3.201068\du}{14.401625\du}}
\pgfpathlineto{\pgfpoint{3.196321\du}{14.408927\du}}
\pgfpathlineto{\pgfpoint{3.190115\du}{14.415864\du}}
\pgfpathlineto{\pgfpoint{3.184638\du}{14.422071\du}}
\pgfpathlineto{\pgfpoint{3.179527\du}{14.428643\du}}
\pgfpathlineto{\pgfpoint{3.172955\du}{14.435214\du}}
\pgfpathlineto{\pgfpoint{3.166018\du}{14.442151\du}}
\pgfpathlineto{\pgfpoint{3.160177\du}{14.448723\du}}
\pgfpathlineto{\pgfpoint{3.152875\du}{14.454929\du}}
\pgfpathlineto{\pgfpoint{3.146303\du}{14.461501\du}}
\pgfpathlineto{\pgfpoint{3.138636\du}{14.467708\du}}
\pgfpathlineto{\pgfpoint{3.131334\du}{14.474280\du}}
\pgfpathlineto{\pgfpoint{3.123302\du}{14.480851\du}}
\pgfpathlineto{\pgfpoint{3.115635\du}{14.487058\du}}
\pgfpathlineto{\pgfpoint{3.106873\du}{14.493630\du}}
\pgfpathlineto{\pgfpoint{3.098475\du}{14.499471\du}}
\pgfpathlineto{\pgfpoint{3.089713\du}{14.506043\du}}
\pgfpathlineto{\pgfpoint{3.081681\du}{14.512250\du}}
\pgfpathlineto{\pgfpoint{3.072919\du}{14.518091\du}}
\pgfpathlineto{\pgfpoint{3.063426\du}{14.524663\du}}
\pgfpathlineto{\pgfpoint{3.053203\du}{14.530504\du}}
\pgfpathlineto{\pgfpoint{3.044076\du}{14.537076\du}}
\pgfpathlineto{\pgfpoint{3.033853\du}{14.542918\du}}
\pgfpathlineto{\pgfpoint{3.023996\du}{14.548759\du}}
\pgfpathlineto{\pgfpoint{3.014138\du}{14.554601\du}}
\pgfpathlineto{\pgfpoint{3.003185\du}{14.560442\du}}
\pgfpathlineto{\pgfpoint{2.992232\du}{14.566284\du}}
\pgfpathlineto{\pgfpoint{2.982010\du}{14.572126\du}}
\pgfpathlineto{\pgfpoint{2.969961\du}{14.577967\du}}
\pgfpathlineto{\pgfpoint{2.959374\du}{14.583078\du}}
\pgfpathlineto{\pgfpoint{2.947325\du}{14.588920\du}}
\pgfpathlineto{\pgfpoint{2.936007\du}{14.594762\du}}
\pgfpathlineto{\pgfpoint{2.923594\du}{14.600238\du}}
\pgfpathlineto{\pgfpoint{2.911546\du}{14.605349\du}}
\pgfpathlineto{\pgfpoint{2.899133\du}{14.611191\du}}
\pgfpathlineto{\pgfpoint{2.887084\du}{14.616667\du}}
\pgfpathlineto{\pgfpoint{2.873941\du}{14.621779\du}}
\pgfpathlineto{\pgfpoint{2.861162\du}{14.626890\du}}
\pgfpathlineto{\pgfpoint{2.848384\du}{14.632366\du}}
\pgfpathlineto{\pgfpoint{2.835241\du}{14.637478\du}}
\pgfpathlineto{\pgfpoint{2.821732\du}{14.642954\du}}
\pgfpathlineto{\pgfpoint{2.808589\du}{14.647335\du}}
\pgfpathlineto{\pgfpoint{2.794350\du}{14.652812\du}}
\pgfpathlineto{\pgfpoint{2.780476\du}{14.657558\du}}
\pgfpathlineto{\pgfpoint{2.766237\du}{14.662670\du}}
\pgfpathlineto{\pgfpoint{2.751633\du}{14.667416\du}}
\pgfpathlineto{\pgfpoint{2.737760\du}{14.672162\du}}
\pgfpathlineto{\pgfpoint{2.723156\du}{14.676908\du}}
\pgfpathlineto{\pgfpoint{2.708917\du}{14.681289\du}}
\pgfpathlineto{\pgfpoint{2.693218\du}{14.686036\du}}
\pgfpathlineto{\pgfpoint{2.678979\du}{14.690782\du}}
\pgfpathlineto{\pgfpoint{2.663280\du}{14.695528\du}}
\pgfpathlineto{\pgfpoint{2.648311\du}{14.699544\du}}
\pgfpathlineto{\pgfpoint{2.616913\du}{14.708307\du}}
\pgfpathlineto{\pgfpoint{2.584784\du}{14.716704\du}}
\pgfpathlineto{\pgfpoint{2.552656\du}{14.724736\du}}
\pgfpathlineto{\pgfpoint{2.519432\du}{14.732768\du}}
\pgfpathlineto{\pgfpoint{2.485113\du}{14.740435\du}}
\pgfpathlineto{\pgfpoint{2.451159\du}{14.747737\du}}
\pgfpathlineto{\pgfpoint{2.416475\du}{14.754674\du}}
\pgfpathlineto{\pgfpoint{2.380695\du}{14.761611\du}}
\pgfpathlineto{\pgfpoint{2.345281\du}{14.768182\du}}
\pgfpathlineto{\pgfpoint{2.308406\du}{14.774389\du}}
\pgfpathlineto{\pgfpoint{2.271896\du}{14.780231\du}}
\pgfpathlineto{\pgfpoint{2.234656\du}{14.786072\du}}
\pgfpathlineto{\pgfpoint{2.197051\du}{14.791549\du}}
\pgfpathlineto{\pgfpoint{2.158351\du}{14.796295\du}}
\pgfpathlineto{\pgfpoint{2.119651\du}{14.800676\du}}
\pgfpathlineto{\pgfpoint{2.080221\du}{14.805422\du}}
\pgfpathlineto{\pgfpoint{2.041155\du}{14.809073\du}}
\pgfpathlineto{\pgfpoint{2.000995\du}{14.813089\du}}
\pgfpathlineto{\pgfpoint{1.960834\du}{14.816010\du}}
\pgfpathlineto{\pgfpoint{1.919578\du}{14.819661\du}}
\pgfpathlineto{\pgfpoint{1.878687\du}{14.822582\du}}
\pgfpathlineto{\pgfpoint{1.837431\du}{14.824772\du}}
\pgfpathlineto{\pgfpoint{1.795810\du}{14.826598\du}}
\pgfpathlineto{\pgfpoint{1.753094\du}{14.828423\du}}
\pgfpathlineto{\pgfpoint{1.710378\du}{14.829519\du}}
\pgfpathlineto{\pgfpoint{1.668391\du}{14.830614\du}}
\pgfpathlineto{\pgfpoint{1.624945\du}{14.830614\du}}
\pgfpathlineto{\pgfpoint{1.581863\du}{14.831344\du}}
\pgfpathlineto{\pgfpoint{1.581863\du}{14.831344\du}}
\pgfpathlineto{\pgfpoint{1.581863\du}{14.831344\du}}
\pgfpathlineto{\pgfpoint{1.581133\du}{14.831344\du}}
\pgfpathlineto{\pgfpoint{1.579308\du}{14.831344\du}}
\pgfpathlineto{\pgfpoint{1.578212\du}{14.831709\du}}
\pgfpathlineto{\pgfpoint{1.577482\du}{14.831709\du}}
\pgfpathlineto{\pgfpoint{1.577117\du}{14.832440\du}}
\pgfpathlineto{\pgfpoint{1.575657\du}{14.832805\du}}
\pgfpathlineto{\pgfpoint{1.574927\du}{14.833535\du}}
\pgfpathlineto{\pgfpoint{1.574196\du}{14.834265\du}}
\pgfpathlineto{\pgfpoint{1.573101\du}{14.836090\du}}
\pgfpathlineto{\pgfpoint{1.572371\du}{14.837551\du}}
\pgfpathlineto{\pgfpoint{1.572371\du}{14.839376\du}}
\pgfpathlineto{\pgfpoint{1.571641\du}{14.841202\du}}
\pgfpathlineto{\pgfpoint{1.572371\du}{14.843392\du}}
\pgfpathlineto{\pgfpoint{1.572371\du}{14.845218\du}}
\pgfpathlineto{\pgfpoint{1.573101\du}{14.847043\du}}
\pgfpathlineto{\pgfpoint{1.574196\du}{14.848869\du}}
\pgfpathlineto{\pgfpoint{1.574927\du}{14.849234\du}}
\pgfpathlineto{\pgfpoint{1.575657\du}{14.849964\du}}
\pgfpathlineto{\pgfpoint{1.577117\du}{14.850694\du}}
\pgfpathlineto{\pgfpoint{1.577482\du}{14.851059\du}}
\pgfpathlineto{\pgfpoint{1.578212\du}{14.851059\du}}
\pgfpathlineto{\pgfpoint{1.579308\du}{14.851790\du}}
\pgfpathlineto{\pgfpoint{1.581133\du}{14.851790\du}}
\pgfpathlineto{\pgfpoint{1.581863\du}{14.851790\du}}
\pgfusepath{fill}
\pgfsetbuttcap
\pgfsetmiterjoin
\pgfsetdash{}{0pt}
\definecolor{dialinecolor}{rgb}{0.678431, 0.839216, 0.905882}
\pgfsetfillcolor{dialinecolor}
\pgfpathmoveto{\pgfpoint{-0.108900\du}{14.264349\du}}
\pgfpathlineto{\pgfpoint{-0.108900\du}{14.264349\du}}
\pgfpathlineto{\pgfpoint{-0.108900\du}{14.272016\du}}
\pgfpathlineto{\pgfpoint{-0.108534\du}{14.279318\du}}
\pgfpathlineto{\pgfpoint{-0.107804\du}{14.287715\du}}
\pgfpathlineto{\pgfpoint{-0.106709\du}{14.295382\du}}
\pgfpathlineto{\pgfpoint{-0.105614\du}{14.302684\du}}
\pgfpathlineto{\pgfpoint{-0.103788\du}{14.310351\du}}
\pgfpathlineto{\pgfpoint{-0.101963\du}{14.318748\du}}
\pgfpathlineto{\pgfpoint{-0.099772\du}{14.326415\du}}
\pgfpathlineto{\pgfpoint{-0.097582\du}{14.333717\du}}
\pgfpathlineto{\pgfpoint{-0.095026\du}{14.341019\du}}
\pgfpathlineto{\pgfpoint{-0.092105\du}{14.348321\du}}
\pgfpathlineto{\pgfpoint{-0.088454\du}{14.355988\du}}
\pgfpathlineto{\pgfpoint{-0.085168\du}{14.363655\du}}
\pgfpathlineto{\pgfpoint{-0.081517\du}{14.371322\du}}
\pgfpathlineto{\pgfpoint{-0.077136\du}{14.378259\du}}
\pgfpathlineto{\pgfpoint{-0.073485\du}{14.385926\du}}
\pgfpathlineto{\pgfpoint{-0.068374\du}{14.392863\du}}
\pgfpathlineto{\pgfpoint{-0.064358\du}{14.400165\du}}
\pgfpathlineto{\pgfpoint{-0.058881\du}{14.407467\du}}
\pgfpathlineto{\pgfpoint{-0.054135\du}{14.414769\du}}
\pgfpathlineto{\pgfpoint{-0.048659\du}{14.421706\du}}
\pgfpathlineto{\pgfpoint{-0.042817\du}{14.428643\du}}
\pgfpathlineto{\pgfpoint{-0.036976\du}{14.435579\du}}
\pgfpathlineto{\pgfpoint{-0.031134\du}{14.442881\du}}
\pgfpathlineto{\pgfpoint{-0.024197\du}{14.449818\du}}
\pgfpathlineto{\pgfpoint{-0.017625\du}{14.456025\du}}
\pgfpathlineto{\pgfpoint{-0.010689\du}{14.463327\du}}
\pgfpathlineto{\pgfpoint{-0.003752\du}{14.470264\du}}
\pgfpathlineto{\pgfpoint{0.003185\du}{14.476470\du}}
\pgfpathlineto{\pgfpoint{0.011947\du}{14.483772\du}}
\pgfpathlineto{\pgfpoint{0.018884\du}{14.489979\du}}
\pgfpathlineto{\pgfpoint{0.026551\du}{14.496551\du}}
\pgfpathlineto{\pgfpoint{0.035314\du}{14.503487\du}}
\pgfpathlineto{\pgfpoint{0.043711\du}{14.510059\du}}
\pgfpathlineto{\pgfpoint{0.052473\du}{14.516266\du}}
\pgfpathlineto{\pgfpoint{0.060870\du}{14.522837\du}}
\pgfpathlineto{\pgfpoint{0.070728\du}{14.528679\du}}
\pgfpathlineto{\pgfpoint{0.079855\du}{14.535251\du}}
\pgfpathlineto{\pgfpoint{0.089348\du}{14.541457\du}}
\pgfpathlineto{\pgfpoint{0.098840\du}{14.548029\du}}
\pgfpathlineto{\pgfpoint{0.108698\du}{14.553871\du}}
\pgfpathlineto{\pgfpoint{0.118921\du}{14.560442\du}}
\pgfpathlineto{\pgfpoint{0.129143\du}{14.566284\du}}
\pgfpathlineto{\pgfpoint{0.139731\du}{14.572126\du}}
\pgfpathlineto{\pgfpoint{0.149954\du}{14.578332\du}}
\pgfpathlineto{\pgfpoint{0.160907\du}{14.584174\du}}
\pgfpathlineto{\pgfpoint{0.172590\du}{14.590015\du}}
\pgfpathlineto{\pgfpoint{0.183908\du}{14.595857\du}}
\pgfpathlineto{\pgfpoint{0.195591\du}{14.601698\du}}
\pgfpathlineto{\pgfpoint{0.206909\du}{14.607540\du}}
\pgfpathlineto{\pgfpoint{0.218592\du}{14.613016\du}}
\pgfpathlineto{\pgfpoint{0.231005\du}{14.618858\du}}
\pgfpathlineto{\pgfpoint{0.243054\du}{14.623969\du}}
\pgfpathlineto{\pgfpoint{0.256197\du}{14.629811\du}}
\pgfpathlineto{\pgfpoint{0.268245\du}{14.635287\du}}
\pgfpathlineto{\pgfpoint{0.281024\du}{14.640399\du}}
\pgfpathlineto{\pgfpoint{0.294897\du}{14.646240\du}}
\pgfpathlineto{\pgfpoint{0.307311\du}{14.651717\du}}
\pgfpathlineto{\pgfpoint{0.320454\du}{14.656828\du}}
\pgfpathlineto{\pgfpoint{0.334693\du}{14.662304\du}}
\pgfpathlineto{\pgfpoint{0.347836\du}{14.666686\du}}
\pgfpathlineto{\pgfpoint{0.362075\du}{14.672162\du}}
\pgfpathlineto{\pgfpoint{0.375949\du}{14.676908\du}}
\pgfpathlineto{\pgfpoint{0.390553\du}{14.682020\du}}
\pgfpathlineto{\pgfpoint{0.404426\du}{14.686766\du}}
\pgfpathlineto{\pgfpoint{0.420126\du}{14.691512\du}}
\pgfpathlineto{\pgfpoint{0.433999\du}{14.696624\du}}
\pgfpathlineto{\pgfpoint{0.448603\du}{14.701370\du}}
\pgfpathlineto{\pgfpoint{0.463937\du}{14.706116\du}}
\pgfpathlineto{\pgfpoint{0.479636\du}{14.710862\du}}
\pgfpathlineto{\pgfpoint{0.494605\du}{14.714878\du}}
\pgfpathlineto{\pgfpoint{0.510305\du}{14.719625\du}}
\pgfpathlineto{\pgfpoint{0.542068\du}{14.728387\du}}
\pgfpathlineto{\pgfpoint{0.573831\du}{14.736419\du}}
\pgfpathlineto{\pgfpoint{0.607055\du}{14.744451\du}}
\pgfpathlineto{\pgfpoint{0.639549\du}{14.752848\du}}
\pgfpathlineto{\pgfpoint{0.674233\du}{14.760515\du}}
\pgfpathlineto{\pgfpoint{0.708552\du}{14.768182\du}}
\pgfpathlineto{\pgfpoint{0.743236\du}{14.775119\du}}
\pgfpathlineto{\pgfpoint{0.779016\du}{14.782056\du}}
\pgfpathlineto{\pgfpoint{0.815160\du}{14.788628\du}}
\pgfpathlineto{\pgfpoint{0.851670\du}{14.794469\du}}
\pgfpathlineto{\pgfpoint{0.888910\du}{14.800676\du}}
\pgfpathlineto{\pgfpoint{0.926515\du}{14.806153\du}}
\pgfpathlineto{\pgfpoint{0.964485\du}{14.811264\du}}
\pgfpathlineto{\pgfpoint{1.003185\du}{14.816740\du}}
\pgfpathlineto{\pgfpoint{1.041885\du}{14.821122\du}}
\pgfpathlineto{\pgfpoint{1.080951\du}{14.825503\du}}
\pgfpathlineto{\pgfpoint{1.121111\du}{14.829519\du}}
\pgfpathlineto{\pgfpoint{1.160907\du}{14.833535\du}}
\pgfpathlineto{\pgfpoint{1.201798\du}{14.837186\du}}
\pgfpathlineto{\pgfpoint{1.242689\du}{14.840107\du}}
\pgfpathlineto{\pgfpoint{1.283945\du}{14.842297\du}}
\pgfpathlineto{\pgfpoint{1.325931\du}{14.844853\du}}
\pgfpathlineto{\pgfpoint{1.367552\du}{14.847043\du}}
\pgfpathlineto{\pgfpoint{1.409903\du}{14.848869\du}}
\pgfpathlineto{\pgfpoint{1.451889\du}{14.849964\du}}
\pgfpathlineto{\pgfpoint{1.495336\du}{14.851059\du}}
\pgfpathlineto{\pgfpoint{1.538052\du}{14.851790\du}}
\pgfpathlineto{\pgfpoint{1.581863\du}{14.851790\du}}
\pgfpathlineto{\pgfpoint{1.581863\du}{14.831344\du}}
\pgfpathlineto{\pgfpoint{1.539147\du}{14.830614\du}}
\pgfpathlineto{\pgfpoint{1.495701\du}{14.830614\du}}
\pgfpathlineto{\pgfpoint{1.453349\du}{14.829519\du}}
\pgfpathlineto{\pgfpoint{1.410633\du}{14.828423\du}}
\pgfpathlineto{\pgfpoint{1.368282\du}{14.826598\du}}
\pgfpathlineto{\pgfpoint{1.326296\du}{14.824772\du}}
\pgfpathlineto{\pgfpoint{1.285405\du}{14.822582\du}}
\pgfpathlineto{\pgfpoint{1.244149\du}{14.819661\du}}
\pgfpathlineto{\pgfpoint{1.203258\du}{14.816010\du}}
\pgfpathlineto{\pgfpoint{1.163097\du}{14.813089\du}}
\pgfpathlineto{\pgfpoint{1.123302\du}{14.809073\du}}
\pgfpathlineto{\pgfpoint{1.083871\du}{14.805422\du}}
\pgfpathlineto{\pgfpoint{1.044441\du}{14.800676\du}}
\pgfpathlineto{\pgfpoint{1.005376\du}{14.796295\du}}
\pgfpathlineto{\pgfpoint{0.967041\du}{14.791549\du}}
\pgfpathlineto{\pgfpoint{0.929436\du}{14.786072\du}}
\pgfpathlineto{\pgfpoint{0.892196\du}{14.780231\du}}
\pgfpathlineto{\pgfpoint{0.855686\du}{14.774389\du}}
\pgfpathlineto{\pgfpoint{0.818446\du}{14.768182\du}}
\pgfpathlineto{\pgfpoint{0.783397\du}{14.761611\du}}
\pgfpathlineto{\pgfpoint{0.747252\du}{14.754674\du}}
\pgfpathlineto{\pgfpoint{0.712568\du}{14.747737\du}}
\pgfpathlineto{\pgfpoint{0.678249\du}{14.740435\du}}
\pgfpathlineto{\pgfpoint{0.644295\du}{14.732768\du}}
\pgfpathlineto{\pgfpoint{0.611071\du}{14.724736\du}}
\pgfpathlineto{\pgfpoint{0.579673\du}{14.716704\du}}
\pgfpathlineto{\pgfpoint{0.547179\du}{14.708307\du}}
\pgfpathlineto{\pgfpoint{0.516146\du}{14.699544\du}}
\pgfpathlineto{\pgfpoint{0.500447\du}{14.695528\du}}
\pgfpathlineto{\pgfpoint{0.484748\du}{14.690782\du}}
\pgfpathlineto{\pgfpoint{0.470144\du}{14.686036\du}}
\pgfpathlineto{\pgfpoint{0.455175\du}{14.681289\du}}
\pgfpathlineto{\pgfpoint{0.439841\du}{14.676908\du}}
\pgfpathlineto{\pgfpoint{0.425602\du}{14.672162\du}}
\pgfpathlineto{\pgfpoint{0.411363\du}{14.667416\du}}
\pgfpathlineto{\pgfpoint{0.397125\du}{14.662670\du}}
\pgfpathlineto{\pgfpoint{0.382886\du}{14.657558\du}}
\pgfpathlineto{\pgfpoint{0.369012\du}{14.652812\du}}
\pgfpathlineto{\pgfpoint{0.355138\du}{14.647335\du}}
\pgfpathlineto{\pgfpoint{0.341630\du}{14.642954\du}}
\pgfpathlineto{\pgfpoint{0.328121\du}{14.637478\du}}
\pgfpathlineto{\pgfpoint{0.315343\du}{14.632366\du}}
\pgfpathlineto{\pgfpoint{0.301834\du}{14.626890\du}}
\pgfpathlineto{\pgfpoint{0.289056\du}{14.621779\du}}
\pgfpathlineto{\pgfpoint{0.276643\du}{14.616667\du}}
\pgfpathlineto{\pgfpoint{0.263499\du}{14.611191\du}}
\pgfpathlineto{\pgfpoint{0.251451\du}{14.605349\du}}
\pgfpathlineto{\pgfpoint{0.240133\du}{14.600238\du}}
\pgfpathlineto{\pgfpoint{0.227355\du}{14.594762\du}}
\pgfpathlineto{\pgfpoint{0.215671\du}{14.588920\du}}
\pgfpathlineto{\pgfpoint{0.203988\du}{14.583078\du}}
\pgfpathlineto{\pgfpoint{0.193035\du}{14.577967\du}}
\pgfpathlineto{\pgfpoint{0.181352\du}{14.572126\du}}
\pgfpathlineto{\pgfpoint{0.171495\du}{14.566284\du}}
\pgfpathlineto{\pgfpoint{0.160177\du}{14.560442\du}}
\pgfpathlineto{\pgfpoint{0.149589\du}{14.554601\du}}
\pgfpathlineto{\pgfpoint{0.139731\du}{14.548759\du}}
\pgfpathlineto{\pgfpoint{0.129143\du}{14.542918\du}}
\pgfpathlineto{\pgfpoint{0.119286\du}{14.537076\du}}
\pgfpathlineto{\pgfpoint{0.109793\du}{14.530504\du}}
\pgfpathlineto{\pgfpoint{0.099936\du}{14.524663\du}}
\pgfpathlineto{\pgfpoint{0.090808\du}{14.518091\du}}
\pgfpathlineto{\pgfpoint{0.082411\du}{14.512250\du}}
\pgfpathlineto{\pgfpoint{0.073649\du}{14.506043\du}}
\pgfpathlineto{\pgfpoint{0.064521\du}{14.499471\du}}
\pgfpathlineto{\pgfpoint{0.055759\du}{14.493630\du}}
\pgfpathlineto{\pgfpoint{0.047362\du}{14.487058\du}}
\pgfpathlineto{\pgfpoint{0.040060\du}{14.480851\du}}
\pgfpathlineto{\pgfpoint{0.032028\du}{14.474280\du}}
\pgfpathlineto{\pgfpoint{0.024361\du}{14.467708\du}}
\pgfpathlineto{\pgfpoint{0.017424\du}{14.461501\du}}
\pgfpathlineto{\pgfpoint{0.010122\du}{14.454929\du}}
\pgfpathlineto{\pgfpoint{0.003185\du}{14.448723\du}}
\pgfpathlineto{\pgfpoint{-0.003022\du}{14.442151\du}}
\pgfpathlineto{\pgfpoint{-0.009593\du}{14.435214\du}}
\pgfpathlineto{\pgfpoint{-0.015435\du}{14.428643\du}}
\pgfpathlineto{\pgfpoint{-0.021641\du}{14.422071\du}}
\pgfpathlineto{\pgfpoint{-0.026753\du}{14.415864\du}}
\pgfpathlineto{\pgfpoint{-0.032959\du}{14.408927\du}}
\pgfpathlineto{\pgfpoint{-0.037341\du}{14.402356\du}}
\pgfpathlineto{\pgfpoint{-0.042817\du}{14.395419\du}}
\pgfpathlineto{\pgfpoint{-0.047198\du}{14.388847\du}}
\pgfpathlineto{\pgfpoint{-0.051579\du}{14.381910\du}}
\pgfpathlineto{\pgfpoint{-0.055961\du}{14.374973\du}}
\pgfpathlineto{\pgfpoint{-0.059246\du}{14.368402\du}}
\pgfpathlineto{\pgfpoint{-0.063262\du}{14.361465\du}}
\pgfpathlineto{\pgfpoint{-0.067279\du}{14.354893\du}}
\pgfpathlineto{\pgfpoint{-0.070199\du}{14.347956\du}}
\pgfpathlineto{\pgfpoint{-0.072755\du}{14.341019\du}}
\pgfpathlineto{\pgfpoint{-0.076041\du}{14.333717\du}}
\pgfpathlineto{\pgfpoint{-0.078231\du}{14.326781\du}}
\pgfpathlineto{\pgfpoint{-0.080422\du}{14.320574\du}}
\pgfpathlineto{\pgfpoint{-0.081882\du}{14.313272\du}}
\pgfpathlineto{\pgfpoint{-0.084073\du}{14.306335\du}}
\pgfpathlineto{\pgfpoint{-0.085168\du}{14.299398\du}}
\pgfpathlineto{\pgfpoint{-0.086264\du}{14.292461\du}}
\pgfpathlineto{\pgfpoint{-0.087724\du}{14.285159\du}}
\pgfpathlineto{\pgfpoint{-0.088089\du}{14.278223\du}}
\pgfpathlineto{\pgfpoint{-0.088089\du}{14.271286\du}}
\pgfpathlineto{\pgfpoint{-0.088454\du}{14.264349\du}}
\pgfpathlineto{\pgfpoint{-0.088454\du}{14.264349\du}}
\pgfpathlineto{\pgfpoint{-0.088454\du}{14.264349\du}}
\pgfpathlineto{\pgfpoint{-0.088454\du}{14.262523\du}}
\pgfpathlineto{\pgfpoint{-0.088454\du}{14.261793\du}}
\pgfpathlineto{\pgfpoint{-0.088819\du}{14.260698\du}}
\pgfpathlineto{\pgfpoint{-0.088819\du}{14.259603\du}}
\pgfpathlineto{\pgfpoint{-0.089915\du}{14.258873\du}}
\pgfpathlineto{\pgfpoint{-0.090280\du}{14.257777\du}}
\pgfpathlineto{\pgfpoint{-0.090645\du}{14.257047\du}}
\pgfpathlineto{\pgfpoint{-0.091740\du}{14.256682\du}}
\pgfpathlineto{\pgfpoint{-0.093200\du}{14.255587\du}}
\pgfpathlineto{\pgfpoint{-0.095026\du}{14.254126\du}}
\pgfpathlineto{\pgfpoint{-0.096851\du}{14.253761\du}}
\pgfpathlineto{\pgfpoint{-0.098677\du}{14.253761\du}}
\pgfpathlineto{\pgfpoint{-0.100867\du}{14.253761\du}}
\pgfpathlineto{\pgfpoint{-0.102328\du}{14.254126\du}}
\pgfpathlineto{\pgfpoint{-0.104518\du}{14.255587\du}}
\pgfpathlineto{\pgfpoint{-0.106344\du}{14.256682\du}}
\pgfpathlineto{\pgfpoint{-0.106709\du}{14.257047\du}}
\pgfpathlineto{\pgfpoint{-0.107074\du}{14.257777\du}}
\pgfpathlineto{\pgfpoint{-0.107804\du}{14.258873\du}}
\pgfpathlineto{\pgfpoint{-0.108534\du}{14.259603\du}}
\pgfpathlineto{\pgfpoint{-0.108534\du}{14.260698\du}}
\pgfpathlineto{\pgfpoint{-0.108900\du}{14.261793\du}}
\pgfpathlineto{\pgfpoint{-0.108900\du}{14.262523\du}}
\pgfpathlineto{\pgfpoint{-0.108900\du}{14.264349\du}}
\pgfusepath{fill}
\pgfsetbuttcap
\pgfsetmiterjoin
\pgfsetdash{}{0pt}
\definecolor{dialinecolor}{rgb}{0.678431, 0.839216, 0.905882}
\pgfsetfillcolor{dialinecolor}
\pgfpathmoveto{\pgfpoint{1.581863\du}{13.676908\du}}
\pgfpathlineto{\pgfpoint{1.581863\du}{13.676908\du}}
\pgfpathlineto{\pgfpoint{1.538052\du}{13.676908\du}}
\pgfpathlineto{\pgfpoint{1.495336\du}{13.677273\du}}
\pgfpathlineto{\pgfpoint{1.451889\du}{13.678369\du}}
\pgfpathlineto{\pgfpoint{1.409903\du}{13.679829\du}}
\pgfpathlineto{\pgfpoint{1.367552\du}{13.681289\du}}
\pgfpathlineto{\pgfpoint{1.325931\du}{13.683115\du}}
\pgfpathlineto{\pgfpoint{1.283945\du}{13.685671\du}}
\pgfpathlineto{\pgfpoint{1.242689\du}{13.688591\du}}
\pgfpathlineto{\pgfpoint{1.201798\du}{13.691512\du}}
\pgfpathlineto{\pgfpoint{1.160907\du}{13.694433\du}}
\pgfpathlineto{\pgfpoint{1.121111\du}{13.698449\du}}
\pgfpathlineto{\pgfpoint{1.080951\du}{13.702465\du}}
\pgfpathlineto{\pgfpoint{1.041885\du}{13.707211\du}}
\pgfpathlineto{\pgfpoint{1.003185\du}{13.711958\du}}
\pgfpathlineto{\pgfpoint{0.964485\du}{13.716704\du}}
\pgfpathlineto{\pgfpoint{0.926515\du}{13.721815\du}}
\pgfpathlineto{\pgfpoint{0.888910\du}{13.727657\du}}
\pgfpathlineto{\pgfpoint{0.851670\du}{13.733498\du}}
\pgfpathlineto{\pgfpoint{0.815160\du}{13.740070\du}}
\pgfpathlineto{\pgfpoint{0.779016\du}{13.746277\du}}
\pgfpathlineto{\pgfpoint{0.743236\du}{13.753579\du}}
\pgfpathlineto{\pgfpoint{0.708552\du}{13.760515\du}}
\pgfpathlineto{\pgfpoint{0.674233\du}{13.767452\du}}
\pgfpathlineto{\pgfpoint{0.639549\du}{13.775484\du}}
\pgfpathlineto{\pgfpoint{0.607055\du}{13.783151\du}}
\pgfpathlineto{\pgfpoint{0.573831\du}{13.791549\du}}
\pgfpathlineto{\pgfpoint{0.542068\du}{13.800311\du}}
\pgfpathlineto{\pgfpoint{0.510305\du}{13.809073\du}}
\pgfpathlineto{\pgfpoint{0.479636\du}{13.817836\du}}
\pgfpathlineto{\pgfpoint{0.448603\du}{13.827328\du}}
\pgfpathlineto{\pgfpoint{0.433999\du}{13.831709\du}}
\pgfpathlineto{\pgfpoint{0.420126\du}{13.836456\du}}
\pgfpathlineto{\pgfpoint{0.404426\du}{13.841202\du}}
\pgfpathlineto{\pgfpoint{0.390553\du}{13.846313\du}}
\pgfpathlineto{\pgfpoint{0.375949\du}{13.851059\du}}
\pgfpathlineto{\pgfpoint{0.362075\du}{13.856536\du}}
\pgfpathlineto{\pgfpoint{0.347836\du}{13.860917\du}}
\pgfpathlineto{\pgfpoint{0.334693\du}{13.866394\du}}
\pgfpathlineto{\pgfpoint{0.320454\du}{13.871505\du}}
\pgfpathlineto{\pgfpoint{0.307311\du}{13.876981\du}}
\pgfpathlineto{\pgfpoint{0.294897\du}{13.882093\du}}
\pgfpathlineto{\pgfpoint{0.281024\du}{13.887569\du}}
\pgfpathlineto{\pgfpoint{0.268245\du}{13.892680\du}}
\pgfpathlineto{\pgfpoint{0.256197\du}{13.897792\du}}
\pgfpathlineto{\pgfpoint{0.243054\du}{13.903633\du}}
\pgfpathlineto{\pgfpoint{0.231005\du}{13.909840\du}}
\pgfpathlineto{\pgfpoint{0.218592\du}{13.914951\du}}
\pgfpathlineto{\pgfpoint{0.206909\du}{13.920793\du}}
\pgfpathlineto{\pgfpoint{0.195591\du}{13.926634\du}}
\pgfpathlineto{\pgfpoint{0.183908\du}{13.932476\du}}
\pgfpathlineto{\pgfpoint{0.172590\du}{13.938318\du}}
\pgfpathlineto{\pgfpoint{0.160907\du}{13.943429\du}}
\pgfpathlineto{\pgfpoint{0.149954\du}{13.950001\du}}
\pgfpathlineto{\pgfpoint{0.139731\du}{13.955842\du}}
\pgfpathlineto{\pgfpoint{0.129143\du}{13.961684\du}}
\pgfpathlineto{\pgfpoint{0.118921\du}{13.967525\du}}
\pgfpathlineto{\pgfpoint{0.108698\du}{13.974097\du}}
\pgfpathlineto{\pgfpoint{0.098840\du}{13.980304\du}}
\pgfpathlineto{\pgfpoint{0.089348\du}{13.986145\du}}
\pgfpathlineto{\pgfpoint{0.079855\du}{13.992717\du}}
\pgfpathlineto{\pgfpoint{0.070728\du}{13.999289\du}}
\pgfpathlineto{\pgfpoint{0.060870\du}{14.005495\du}}
\pgfpathlineto{\pgfpoint{0.052473\du}{14.012067\du}}
\pgfpathlineto{\pgfpoint{0.043711\du}{14.018639\du}}
\pgfpathlineto{\pgfpoint{0.035314\du}{14.024846\du}}
\pgfpathlineto{\pgfpoint{0.026551\du}{14.031417\du}}
\pgfpathlineto{\pgfpoint{0.018884\du}{14.038354\du}}
\pgfpathlineto{\pgfpoint{0.011947\du}{14.044926\du}}
\pgfpathlineto{\pgfpoint{0.003185\du}{14.051132\du}}
\pgfpathlineto{\pgfpoint{-0.003752\du}{14.058434\du}}
\pgfpathlineto{\pgfpoint{-0.010689\du}{14.064641\du}}
\pgfpathlineto{\pgfpoint{-0.017625\du}{14.071578\du}}
\pgfpathlineto{\pgfpoint{-0.024197\du}{14.078880\du}}
\pgfpathlineto{\pgfpoint{-0.031134\du}{14.085086\du}}
\pgfpathlineto{\pgfpoint{-0.036976\du}{14.092388\du}}
\pgfpathlineto{\pgfpoint{-0.042817\du}{14.099325\du}}
\pgfpathlineto{\pgfpoint{-0.048659\du}{14.106992\du}}
\pgfpathlineto{\pgfpoint{-0.054135\du}{14.113929\du}}
\pgfpathlineto{\pgfpoint{-0.058881\du}{14.120866\du}}
\pgfpathlineto{\pgfpoint{-0.064358\du}{14.127803\du}}
\pgfpathlineto{\pgfpoint{-0.068374\du}{14.135105\du}}
\pgfpathlineto{\pgfpoint{-0.073485\du}{14.142407\du}}
\pgfpathlineto{\pgfpoint{-0.077136\du}{14.149709\du}}
\pgfpathlineto{\pgfpoint{-0.081517\du}{14.157011\du}}
\pgfpathlineto{\pgfpoint{-0.085168\du}{14.164678\du}}
\pgfpathlineto{\pgfpoint{-0.088454\du}{14.172345\du}}
\pgfpathlineto{\pgfpoint{-0.092105\du}{14.179281\du}}
\pgfpathlineto{\pgfpoint{-0.095026\du}{14.186948\du}}
\pgfpathlineto{\pgfpoint{-0.097582\du}{14.194615\du}}
\pgfpathlineto{\pgfpoint{-0.099772\du}{14.202283\du}}
\pgfpathlineto{\pgfpoint{-0.101963\du}{14.209950\du}}
\pgfpathlineto{\pgfpoint{-0.103788\du}{14.217251\du}}
\pgfpathlineto{\pgfpoint{-0.105614\du}{14.224919\du}}
\pgfpathlineto{\pgfpoint{-0.106709\du}{14.232586\du}}
\pgfpathlineto{\pgfpoint{-0.107804\du}{14.240983\du}}
\pgfpathlineto{\pgfpoint{-0.108534\du}{14.248285\du}}
\pgfpathlineto{\pgfpoint{-0.108900\du}{14.255952\du}}
\pgfpathlineto{\pgfpoint{-0.108900\du}{14.264349\du}}
\pgfpathlineto{\pgfpoint{-0.088454\du}{14.264349\du}}
\pgfpathlineto{\pgfpoint{-0.088089\du}{14.257047\du}}
\pgfpathlineto{\pgfpoint{-0.088089\du}{14.250110\du}}
\pgfpathlineto{\pgfpoint{-0.087724\du}{14.243173\du}}
\pgfpathlineto{\pgfpoint{-0.086264\du}{14.235506\du}}
\pgfpathlineto{\pgfpoint{-0.085168\du}{14.229300\du}}
\pgfpathlineto{\pgfpoint{-0.084073\du}{14.221998\du}}
\pgfpathlineto{\pgfpoint{-0.081882\du}{14.215061\du}}
\pgfpathlineto{\pgfpoint{-0.080422\du}{14.208124\du}}
\pgfpathlineto{\pgfpoint{-0.078231\du}{14.201187\du}}
\pgfpathlineto{\pgfpoint{-0.076041\du}{14.193885\du}}
\pgfpathlineto{\pgfpoint{-0.072755\du}{14.187679\du}}
\pgfpathlineto{\pgfpoint{-0.070199\du}{14.180742\du}}
\pgfpathlineto{\pgfpoint{-0.067279\du}{14.173440\du}}
\pgfpathlineto{\pgfpoint{-0.063262\du}{14.166503\du}}
\pgfpathlineto{\pgfpoint{-0.059246\du}{14.159566\du}}
\pgfpathlineto{\pgfpoint{-0.055961\du}{14.152994\du}}
\pgfpathlineto{\pgfpoint{-0.051944\du}{14.146058\du}}
\pgfpathlineto{\pgfpoint{-0.047198\du}{14.139486\du}}
\pgfpathlineto{\pgfpoint{-0.042817\du}{14.132549\du}}
\pgfpathlineto{\pgfpoint{-0.037341\du}{14.125977\du}}
\pgfpathlineto{\pgfpoint{-0.032959\du}{14.119771\du}}
\pgfpathlineto{\pgfpoint{-0.026753\du}{14.112834\du}}
\pgfpathlineto{\pgfpoint{-0.021641\du}{14.106262\du}}
\pgfpathlineto{\pgfpoint{-0.015435\du}{14.099325\du}}
\pgfpathlineto{\pgfpoint{-0.009593\du}{14.092753\du}}
\pgfpathlineto{\pgfpoint{-0.003022\du}{14.086547\du}}
\pgfpathlineto{\pgfpoint{0.003185\du}{14.079975\du}}
\pgfpathlineto{\pgfpoint{0.010122\du}{14.073038\du}}
\pgfpathlineto{\pgfpoint{0.017424\du}{14.067197\du}}
\pgfpathlineto{\pgfpoint{0.024361\du}{14.059895\du}}
\pgfpathlineto{\pgfpoint{0.032028\du}{14.053688\du}}
\pgfpathlineto{\pgfpoint{0.040060\du}{14.047847\du}}
\pgfpathlineto{\pgfpoint{0.047362\du}{14.041275\du}}
\pgfpathlineto{\pgfpoint{0.055759\du}{14.034703\du}}
\pgfpathlineto{\pgfpoint{0.064521\du}{14.028496\du}}
\pgfpathlineto{\pgfpoint{0.073649\du}{14.021925\du}}
\pgfpathlineto{\pgfpoint{0.082411\du}{14.016083\du}}
\pgfpathlineto{\pgfpoint{0.090808\du}{14.009877\du}}
\pgfpathlineto{\pgfpoint{0.099936\du}{14.004035\du}}
\pgfpathlineto{\pgfpoint{0.109793\du}{13.997463\du}}
\pgfpathlineto{\pgfpoint{0.119286\du}{13.991622\du}}
\pgfpathlineto{\pgfpoint{0.129143\du}{13.985780\du}}
\pgfpathlineto{\pgfpoint{0.139731\du}{13.979939\du}}
\pgfpathlineto{\pgfpoint{0.149589\du}{13.974097\du}}
\pgfpathlineto{\pgfpoint{0.160177\du}{13.967525\du}}
\pgfpathlineto{\pgfpoint{0.171495\du}{13.962414\du}}
\pgfpathlineto{\pgfpoint{0.181352\du}{13.956572\du}}
\pgfpathlineto{\pgfpoint{0.193035\du}{13.950731\du}}
\pgfpathlineto{\pgfpoint{0.203988\du}{13.944889\du}}
\pgfpathlineto{\pgfpoint{0.215671\du}{13.939048\du}}
\pgfpathlineto{\pgfpoint{0.227355\du}{13.933571\du}}
\pgfpathlineto{\pgfpoint{0.240133\du}{13.928460\du}}
\pgfpathlineto{\pgfpoint{0.251451\du}{13.922618\du}}
\pgfpathlineto{\pgfpoint{0.263499\du}{13.917142\du}}
\pgfpathlineto{\pgfpoint{0.276643\du}{13.912031\du}}
\pgfpathlineto{\pgfpoint{0.289056\du}{13.906554\du}}
\pgfpathlineto{\pgfpoint{0.301834\du}{13.901443\du}}
\pgfpathlineto{\pgfpoint{0.315343\du}{13.895601\du}}
\pgfpathlineto{\pgfpoint{0.328121\du}{13.890855\du}}
\pgfpathlineto{\pgfpoint{0.341630\du}{13.885744\du}}
\pgfpathlineto{\pgfpoint{0.355138\du}{13.880267\du}}
\pgfpathlineto{\pgfpoint{0.369012\du}{13.875886\du}}
\pgfpathlineto{\pgfpoint{0.382886\du}{13.870410\du}}
\pgfpathlineto{\pgfpoint{0.397125\du}{13.865663\du}}
\pgfpathlineto{\pgfpoint{0.411363\du}{13.860917\du}}
\pgfpathlineto{\pgfpoint{0.425602\du}{13.855806\du}}
\pgfpathlineto{\pgfpoint{0.439841\du}{13.851059\du}}
\pgfpathlineto{\pgfpoint{0.455175\du}{13.846313\du}}
\pgfpathlineto{\pgfpoint{0.484748\du}{13.837551\du}}
\pgfpathlineto{\pgfpoint{0.516146\du}{13.828423\du}}
\pgfpathlineto{\pgfpoint{0.547179\du}{13.820026\du}}
\pgfpathlineto{\pgfpoint{0.579673\du}{13.811264\du}}
\pgfpathlineto{\pgfpoint{0.611071\du}{13.803232\du}}
\pgfpathlineto{\pgfpoint{0.644295\du}{13.795565\du}}
\pgfpathlineto{\pgfpoint{0.678249\du}{13.787898\du}}
\pgfpathlineto{\pgfpoint{0.712568\du}{13.780231\du}}
\pgfpathlineto{\pgfpoint{0.747252\du}{13.773294\du}}
\pgfpathlineto{\pgfpoint{0.783397\du}{13.766357\du}}
\pgfpathlineto{\pgfpoint{0.818446\du}{13.760515\du}}
\pgfpathlineto{\pgfpoint{0.855686\du}{13.753944\du}}
\pgfpathlineto{\pgfpoint{0.892196\du}{13.748102\du}}
\pgfpathlineto{\pgfpoint{0.929436\du}{13.742261\du}}
\pgfpathlineto{\pgfpoint{0.967041\du}{13.737149\du}}
\pgfpathlineto{\pgfpoint{1.005376\du}{13.731673\du}}
\pgfpathlineto{\pgfpoint{1.044441\du}{13.726927\du}}
\pgfpathlineto{\pgfpoint{1.083871\du}{13.722910\du}}
\pgfpathlineto{\pgfpoint{1.123302\du}{13.718894\du}}
\pgfpathlineto{\pgfpoint{1.163097\du}{13.715243\du}}
\pgfpathlineto{\pgfpoint{1.203258\du}{13.711958\du}}
\pgfpathlineto{\pgfpoint{1.244149\du}{13.709037\du}}
\pgfpathlineto{\pgfpoint{1.285405\du}{13.706116\du}}
\pgfpathlineto{\pgfpoint{1.326296\du}{13.703560\du}}
\pgfpathlineto{\pgfpoint{1.368282\du}{13.701370\du}}
\pgfpathlineto{\pgfpoint{1.410633\du}{13.700274\du}}
\pgfpathlineto{\pgfpoint{1.453349\du}{13.698449\du}}
\pgfpathlineto{\pgfpoint{1.495701\du}{13.697719\du}}
\pgfpathlineto{\pgfpoint{1.539147\du}{13.697354\du}}
\pgfpathlineto{\pgfpoint{1.581863\du}{13.697354\du}}
\pgfpathlineto{\pgfpoint{1.581863\du}{13.697354\du}}
\pgfpathlineto{\pgfpoint{1.581863\du}{13.697354\du}}
\pgfpathlineto{\pgfpoint{1.582959\du}{13.696624\du}}
\pgfpathlineto{\pgfpoint{1.584054\du}{13.696624\du}}
\pgfpathlineto{\pgfpoint{1.585514\du}{13.696624\du}}
\pgfpathlineto{\pgfpoint{1.586610\du}{13.696258\du}}
\pgfpathlineto{\pgfpoint{1.586975\du}{13.695528\du}}
\pgfpathlineto{\pgfpoint{1.588070\du}{13.695528\du}}
\pgfpathlineto{\pgfpoint{1.588800\du}{13.694433\du}}
\pgfpathlineto{\pgfpoint{1.589896\du}{13.693703\du}}
\pgfpathlineto{\pgfpoint{1.590991\du}{13.692607\du}}
\pgfpathlineto{\pgfpoint{1.591721\du}{13.690782\du}}
\pgfpathlineto{\pgfpoint{1.591721\du}{13.688591\du}}
\pgfpathlineto{\pgfpoint{1.592451\du}{13.686766\du}}
\pgfpathlineto{\pgfpoint{1.591721\du}{13.684940\du}}
\pgfpathlineto{\pgfpoint{1.591721\du}{13.683115\du}}
\pgfpathlineto{\pgfpoint{1.590991\du}{13.681289\du}}
\pgfpathlineto{\pgfpoint{1.589896\du}{13.679829\du}}
\pgfpathlineto{\pgfpoint{1.588800\du}{13.679099\du}}
\pgfpathlineto{\pgfpoint{1.588070\du}{13.678369\du}}
\pgfpathlineto{\pgfpoint{1.586975\du}{13.678004\du}}
\pgfpathlineto{\pgfpoint{1.586610\du}{13.677273\du}}
\pgfpathlineto{\pgfpoint{1.585514\du}{13.676908\du}}
\pgfpathlineto{\pgfpoint{1.584054\du}{13.676908\du}}
\pgfpathlineto{\pgfpoint{1.582959\du}{13.676908\du}}
\pgfpathlineto{\pgfpoint{1.581863\du}{13.676908\du}}
\pgfusepath{fill}
\pgfsetbuttcap
\pgfsetmiterjoin
\pgfsetdash{}{0pt}
\definecolor{dialinecolor}{rgb}{0.678431, 0.839216, 0.905882}
\pgfsetfillcolor{dialinecolor}
\pgfpathmoveto{\pgfpoint{3.272261\du}{14.264349\du}}
\pgfpathlineto{\pgfpoint{3.272261\du}{14.255952\du}}
\pgfpathlineto{\pgfpoint{3.271531\du}{14.248285\du}}
\pgfpathlineto{\pgfpoint{3.270801\du}{14.240983\du}}
\pgfpathlineto{\pgfpoint{3.270071\du}{14.232586\du}}
\pgfpathlineto{\pgfpoint{3.268976\du}{14.224919\du}}
\pgfpathlineto{\pgfpoint{3.266785\du}{14.217251\du}}
\pgfpathlineto{\pgfpoint{3.265690\du}{14.209950\du}}
\pgfpathlineto{\pgfpoint{3.262769\du}{14.202283\du}}
\pgfpathlineto{\pgfpoint{3.260943\du}{14.194615\du}}
\pgfpathlineto{\pgfpoint{3.258023\du}{14.186948\du}}
\pgfpathlineto{\pgfpoint{3.255467\du}{14.179281\du}}
\pgfpathlineto{\pgfpoint{3.252181\du}{14.172345\du}}
\pgfpathlineto{\pgfpoint{3.248165\du}{14.164678\du}}
\pgfpathlineto{\pgfpoint{3.244879\du}{14.157011\du}}
\pgfpathlineto{\pgfpoint{3.240863\du}{14.149709\du}}
\pgfpathlineto{\pgfpoint{3.236847\du}{14.142407\du}}
\pgfpathlineto{\pgfpoint{3.232466\du}{14.135105\du}}
\pgfpathlineto{\pgfpoint{3.227720\du}{14.127803\du}}
\pgfpathlineto{\pgfpoint{3.222243\du}{14.120866\du}}
\pgfpathlineto{\pgfpoint{3.217497\du}{14.113929\du}}
\pgfpathlineto{\pgfpoint{3.211655\du}{14.106262\du}}
\pgfpathlineto{\pgfpoint{3.206179\du}{14.099325\du}}
\pgfpathlineto{\pgfpoint{3.200337\du}{14.092388\du}}
\pgfpathlineto{\pgfpoint{3.194131\du}{14.085086\du}}
\pgfpathlineto{\pgfpoint{3.187559\du}{14.078880\du}}
\pgfpathlineto{\pgfpoint{3.180622\du}{14.071578\du}}
\pgfpathlineto{\pgfpoint{3.174050\du}{14.064641\du}}
\pgfpathlineto{\pgfpoint{3.166748\du}{14.058434\du}}
\pgfpathlineto{\pgfpoint{3.160177\du}{14.051132\du}}
\pgfpathlineto{\pgfpoint{3.152145\du}{14.044926\du}}
\pgfpathlineto{\pgfpoint{3.144112\du}{14.038354\du}}
\pgfpathlineto{\pgfpoint{3.136811\du}{14.031417\du}}
\pgfpathlineto{\pgfpoint{3.128048\du}{14.024846\du}}
\pgfpathlineto{\pgfpoint{3.120016\du}{14.018639\du}}
\pgfpathlineto{\pgfpoint{3.110889\du}{14.012067\du}}
\pgfpathlineto{\pgfpoint{3.102126\du}{14.005495\du}}
\pgfpathlineto{\pgfpoint{3.092634\du}{13.999289\du}}
\pgfpathlineto{\pgfpoint{3.083506\du}{13.992717\du}}
\pgfpathlineto{\pgfpoint{3.074379\du}{13.986145\du}}
\pgfpathlineto{\pgfpoint{3.064521\du}{13.980304\du}}
\pgfpathlineto{\pgfpoint{3.054664\du}{13.974097\du}}
\pgfpathlineto{\pgfpoint{3.044441\du}{13.967525\du}}
\pgfpathlineto{\pgfpoint{3.033853\du}{13.961684\du}}
\pgfpathlineto{\pgfpoint{3.023996\du}{13.955842\du}}
\pgfpathlineto{\pgfpoint{3.013408\du}{13.950001\du}}
\pgfpathlineto{\pgfpoint{3.002090\du}{13.943429\du}}
\pgfpathlineto{\pgfpoint{2.990772\du}{13.938318\du}}
\pgfpathlineto{\pgfpoint{2.979089\du}{13.932476\du}}
\pgfpathlineto{\pgfpoint{2.968136\du}{13.926634\du}}
\pgfpathlineto{\pgfpoint{2.956088\du}{13.920793\du}}
\pgfpathlineto{\pgfpoint{2.944770\du}{13.914951\du}}
\pgfpathlineto{\pgfpoint{2.932356\du}{13.909840\du}}
\pgfpathlineto{\pgfpoint{2.919943\du}{13.903633\du}}
\pgfpathlineto{\pgfpoint{2.907165\du}{13.897792\du}}
\pgfpathlineto{\pgfpoint{2.895116\du}{13.892680\du}}
\pgfpathlineto{\pgfpoint{2.882338\du}{13.887569\du}}
\pgfpathlineto{\pgfpoint{2.868830\du}{13.882093\du}}
\pgfpathlineto{\pgfpoint{2.855686\du}{13.876981\du}}
\pgfpathlineto{\pgfpoint{2.842543\du}{13.871505\du}}
\pgfpathlineto{\pgfpoint{2.828304\du}{13.866394\du}}
\pgfpathlineto{\pgfpoint{2.815160\du}{13.860917\du}}
\pgfpathlineto{\pgfpoint{2.800922\du}{13.856536\du}}
\pgfpathlineto{\pgfpoint{2.787048\du}{13.851059\du}}
\pgfpathlineto{\pgfpoint{2.772809\du}{13.846313\du}}
\pgfpathlineto{\pgfpoint{2.758935\du}{13.841202\du}}
\pgfpathlineto{\pgfpoint{2.743966\du}{13.836456\du}}
\pgfpathlineto{\pgfpoint{2.728997\du}{13.831709\du}}
\pgfpathlineto{\pgfpoint{2.714759\du}{13.827328\du}}
\pgfpathlineto{\pgfpoint{2.684091\du}{13.817836\du}}
\pgfpathlineto{\pgfpoint{2.653422\du}{13.809073\du}}
\pgfpathlineto{\pgfpoint{2.622024\du}{13.800311\du}}
\pgfpathlineto{\pgfpoint{2.590261\du}{13.791549\du}}
\pgfpathlineto{\pgfpoint{2.557402\du}{13.783151\du}}
\pgfpathlineto{\pgfpoint{2.523813\du}{13.775484\du}}
\pgfpathlineto{\pgfpoint{2.489859\du}{13.767452\du}}
\pgfpathlineto{\pgfpoint{2.455175\du}{13.760515\du}}
\pgfpathlineto{\pgfpoint{2.420491\du}{13.753579\du}}
\pgfpathlineto{\pgfpoint{2.385076\du}{13.746277\du}}
\pgfpathlineto{\pgfpoint{2.348567\du}{13.740070\du}}
\pgfpathlineto{\pgfpoint{2.312057\du}{13.733498\du}}
\pgfpathlineto{\pgfpoint{2.275182\du}{13.727657\du}}
\pgfpathlineto{\pgfpoint{2.237942\du}{13.721815\du}}
\pgfpathlineto{\pgfpoint{2.198877\du}{13.716704\du}}
\pgfpathlineto{\pgfpoint{2.160907\du}{13.711958\du}}
\pgfpathlineto{\pgfpoint{2.121476\du}{13.707211\du}}
\pgfpathlineto{\pgfpoint{2.082776\du}{13.702465\du}}
\pgfpathlineto{\pgfpoint{2.042616\du}{13.698449\du}}
\pgfpathlineto{\pgfpoint{2.002820\du}{13.694433\du}}
\pgfpathlineto{\pgfpoint{1.961929\du}{13.691512\du}}
\pgfpathlineto{\pgfpoint{1.921403\du}{13.688591\du}}
\pgfpathlineto{\pgfpoint{1.879782\du}{13.685671\du}}
\pgfpathlineto{\pgfpoint{1.838161\du}{13.683115\du}}
\pgfpathlineto{\pgfpoint{1.796540\du}{13.681289\du}}
\pgfpathlineto{\pgfpoint{1.753824\du}{13.679829\du}}
\pgfpathlineto{\pgfpoint{1.711838\du}{13.678369\du}}
\pgfpathlineto{\pgfpoint{1.668756\du}{13.677273\du}}
\pgfpathlineto{\pgfpoint{1.625310\du}{13.676908\du}}
\pgfpathlineto{\pgfpoint{1.581863\du}{13.676908\du}}
\pgfpathlineto{\pgfpoint{1.581863\du}{13.697354\du}}
\pgfpathlineto{\pgfpoint{1.624945\du}{13.697354\du}}
\pgfpathlineto{\pgfpoint{1.668391\du}{13.697719\du}}
\pgfpathlineto{\pgfpoint{1.710378\du}{13.698449\du}}
\pgfpathlineto{\pgfpoint{1.753094\du}{13.700274\du}}
\pgfpathlineto{\pgfpoint{1.795810\du}{13.701370\du}}
\pgfpathlineto{\pgfpoint{1.837431\du}{13.703560\du}}
\pgfpathlineto{\pgfpoint{1.878687\du}{13.706116\du}}
\pgfpathlineto{\pgfpoint{1.919578\du}{13.709037\du}}
\pgfpathlineto{\pgfpoint{1.960834\du}{13.711958\du}}
\pgfpathlineto{\pgfpoint{2.000995\du}{13.715243\du}}
\pgfpathlineto{\pgfpoint{2.041155\du}{13.718894\du}}
\pgfpathlineto{\pgfpoint{2.080221\du}{13.722910\du}}
\pgfpathlineto{\pgfpoint{2.119651\du}{13.726927\du}}
\pgfpathlineto{\pgfpoint{2.158351\du}{13.731673\du}}
\pgfpathlineto{\pgfpoint{2.197051\du}{13.737149\du}}
\pgfpathlineto{\pgfpoint{2.234656\du}{13.742261\du}}
\pgfpathlineto{\pgfpoint{2.271896\du}{13.748102\du}}
\pgfpathlineto{\pgfpoint{2.308406\du}{13.753944\du}}
\pgfpathlineto{\pgfpoint{2.345281\du}{13.760515\du}}
\pgfpathlineto{\pgfpoint{2.380695\du}{13.766357\du}}
\pgfpathlineto{\pgfpoint{2.416475\du}{13.773294\du}}
\pgfpathlineto{\pgfpoint{2.451159\du}{13.780231\du}}
\pgfpathlineto{\pgfpoint{2.485113\du}{13.787898\du}}
\pgfpathlineto{\pgfpoint{2.519432\du}{13.795565\du}}
\pgfpathlineto{\pgfpoint{2.552656\du}{13.803232\du}}
\pgfpathlineto{\pgfpoint{2.584784\du}{13.811264\du}}
\pgfpathlineto{\pgfpoint{2.616913\du}{13.820026\du}}
\pgfpathlineto{\pgfpoint{2.648311\du}{13.828423\du}}
\pgfpathlineto{\pgfpoint{2.678979\du}{13.837551\du}}
\pgfpathlineto{\pgfpoint{2.708917\du}{13.846313\du}}
\pgfpathlineto{\pgfpoint{2.723156\du}{13.851059\du}}
\pgfpathlineto{\pgfpoint{2.737760\du}{13.855806\du}}
\pgfpathlineto{\pgfpoint{2.751633\du}{13.860917\du}}
\pgfpathlineto{\pgfpoint{2.766237\du}{13.865663\du}}
\pgfpathlineto{\pgfpoint{2.780476\du}{13.870410\du}}
\pgfpathlineto{\pgfpoint{2.794350\du}{13.875886\du}}
\pgfpathlineto{\pgfpoint{2.808589\du}{13.880267\du}}
\pgfpathlineto{\pgfpoint{2.821732\du}{13.885744\du}}
\pgfpathlineto{\pgfpoint{2.835241\du}{13.890855\du}}
\pgfpathlineto{\pgfpoint{2.848384\du}{13.895601\du}}
\pgfpathlineto{\pgfpoint{2.861162\du}{13.901443\du}}
\pgfpathlineto{\pgfpoint{2.873941\du}{13.906554\du}}
\pgfpathlineto{\pgfpoint{2.887084\du}{13.912031\du}}
\pgfpathlineto{\pgfpoint{2.899133\du}{13.917142\du}}
\pgfpathlineto{\pgfpoint{2.911546\du}{13.922618\du}}
\pgfpathlineto{\pgfpoint{2.923594\du}{13.928460\du}}
\pgfpathlineto{\pgfpoint{2.936007\du}{13.933571\du}}
\pgfpathlineto{\pgfpoint{2.947325\du}{13.939048\du}}
\pgfpathlineto{\pgfpoint{2.959374\du}{13.944889\du}}
\pgfpathlineto{\pgfpoint{2.969961\du}{13.950731\du}}
\pgfpathlineto{\pgfpoint{2.982010\du}{13.956572\du}}
\pgfpathlineto{\pgfpoint{2.992232\du}{13.962414\du}}
\pgfpathlineto{\pgfpoint{3.003185\du}{13.967525\du}}
\pgfpathlineto{\pgfpoint{3.014138\du}{13.974097\du}}
\pgfpathlineto{\pgfpoint{3.023996\du}{13.979939\du}}
\pgfpathlineto{\pgfpoint{3.033853\du}{13.985780\du}}
\pgfpathlineto{\pgfpoint{3.044076\du}{13.991622\du}}
\pgfpathlineto{\pgfpoint{3.053203\du}{13.997463\du}}
\pgfpathlineto{\pgfpoint{3.063426\du}{14.004035\du}}
\pgfpathlineto{\pgfpoint{3.072919\du}{14.009877\du}}
\pgfpathlineto{\pgfpoint{3.081681\du}{14.016083\du}}
\pgfpathlineto{\pgfpoint{3.089713\du}{14.021925\du}}
\pgfpathlineto{\pgfpoint{3.098475\du}{14.028496\du}}
\pgfpathlineto{\pgfpoint{3.106873\du}{14.034703\du}}
\pgfpathlineto{\pgfpoint{3.115635\du}{14.041275\du}}
\pgfpathlineto{\pgfpoint{3.123302\du}{14.047847\du}}
\pgfpathlineto{\pgfpoint{3.131334\du}{14.053688\du}}
\pgfpathlineto{\pgfpoint{3.138636\du}{14.059895\du}}
\pgfpathlineto{\pgfpoint{3.146303\du}{14.067197\du}}
\pgfpathlineto{\pgfpoint{3.152875\du}{14.073038\du}}
\pgfpathlineto{\pgfpoint{3.160177\du}{14.079975\du}}
\pgfpathlineto{\pgfpoint{3.166018\du}{14.086547\du}}
\pgfpathlineto{\pgfpoint{3.172955\du}{14.092753\du}}
\pgfpathlineto{\pgfpoint{3.179527\du}{14.099325\du}}
\pgfpathlineto{\pgfpoint{3.184638\du}{14.106262\du}}
\pgfpathlineto{\pgfpoint{3.190115\du}{14.112834\du}}
\pgfpathlineto{\pgfpoint{3.196321\du}{14.119771\du}}
\pgfpathlineto{\pgfpoint{3.201068\du}{14.125977\du}}
\pgfpathlineto{\pgfpoint{3.206179\du}{14.132549\du}}
\pgfpathlineto{\pgfpoint{3.210925\du}{14.139486\du}}
\pgfpathlineto{\pgfpoint{3.215306\du}{14.146058\du}}
\pgfpathlineto{\pgfpoint{3.219687\du}{14.152994\du}}
\pgfpathlineto{\pgfpoint{3.222608\du}{14.159566\du}}
\pgfpathlineto{\pgfpoint{3.226624\du}{14.166503\du}}
\pgfpathlineto{\pgfpoint{3.229545\du}{14.173440\du}}
\pgfpathlineto{\pgfpoint{3.233561\du}{14.180742\du}}
\pgfpathlineto{\pgfpoint{3.236117\du}{14.186948\du}}
\pgfpathlineto{\pgfpoint{3.239038\du}{14.193885\du}}
\pgfpathlineto{\pgfpoint{3.241593\du}{14.201187\du}}
\pgfpathlineto{\pgfpoint{3.243784\du}{14.208124\du}}
\pgfpathlineto{\pgfpoint{3.245244\du}{14.215061\du}}
\pgfpathlineto{\pgfpoint{3.247435\du}{14.221998\du}}
\pgfpathlineto{\pgfpoint{3.248165\du}{14.229300\du}}
\pgfpathlineto{\pgfpoint{3.249260\du}{14.235506\du}}
\pgfpathlineto{\pgfpoint{3.251086\du}{14.243173\du}}
\pgfpathlineto{\pgfpoint{3.251451\du}{14.250110\du}}
\pgfpathlineto{\pgfpoint{3.251451\du}{14.257047\du}}
\pgfpathlineto{\pgfpoint{3.252181\du}{14.264349\du}}
\pgfpathlineto{\pgfpoint{3.272261\du}{14.264349\du}}
\pgfusepath{fill}
\pgfsetbuttcap
\pgfsetmiterjoin
\pgfsetdash{}{0pt}
\definecolor{dialinecolor}{rgb}{0.027451, 0.486275, 0.682353}
\pgfsetfillcolor{dialinecolor}
\pgfpathmoveto{\pgfpoint{-0.103788\du}{13.454929\du}}
\pgfpathlineto{\pgfpoint{-0.103788\du}{14.279318\du}}
\pgfpathlineto{\pgfpoint{3.261674\du}{14.279318\du}}
\pgfpathlineto{\pgfpoint{3.262404\du}{13.455660\du}}
\pgfpathlineto{\pgfpoint{-0.103788\du}{13.454929\du}}
\pgfusepath{fill}
\pgfsetbuttcap
\pgfsetmiterjoin
\pgfsetdash{}{0pt}
\definecolor{dialinecolor}{rgb}{0.235294, 0.686275, 0.894118}
\pgfsetfillcolor{dialinecolor}
\pgfpathmoveto{\pgfpoint{3.261674\du}{13.439230\du}}
\pgfpathlineto{\pgfpoint{3.260213\du}{13.469168\du}}
\pgfpathlineto{\pgfpoint{3.252911\du}{13.498376\du}}
\pgfpathlineto{\pgfpoint{3.242689\du}{13.527584\du}}
\pgfpathlineto{\pgfpoint{3.228085\du}{13.555696\du}}
\pgfpathlineto{\pgfpoint{3.208735\du}{13.583809\du}}
\pgfpathlineto{\pgfpoint{3.186829\du}{13.611191\du}}
\pgfpathlineto{\pgfpoint{3.160177\du}{13.637478\du}}
\pgfpathlineto{\pgfpoint{3.129509\du}{13.663765\du}}
\pgfpathlineto{\pgfpoint{3.096650\du}{13.689687\du}}
\pgfpathlineto{\pgfpoint{3.059410\du}{13.714878\du}}
\pgfpathlineto{\pgfpoint{3.018519\du}{13.738975\du}}
\pgfpathlineto{\pgfpoint{2.974708\du}{13.762341\du}}
\pgfpathlineto{\pgfpoint{2.928340\du}{13.784612\du}}
\pgfpathlineto{\pgfpoint{2.877957\du}{13.806518\du}}
\pgfpathlineto{\pgfpoint{2.825018\du}{13.827693\du}}
\pgfpathlineto{\pgfpoint{2.769523\du}{13.847774\du}}
\pgfpathlineto{\pgfpoint{2.711473\du}{13.866394\du}}
\pgfpathlineto{\pgfpoint{2.650867\du}{13.885013\du}}
\pgfpathlineto{\pgfpoint{2.586975\du}{13.902173\du}}
\pgfpathlineto{\pgfpoint{2.521988\du}{13.917872\du}}
\pgfpathlineto{\pgfpoint{2.453349\du}{13.933206\du}}
\pgfpathlineto{\pgfpoint{2.382521\du}{13.947080\du}}
\pgfpathlineto{\pgfpoint{2.310597\du}{13.959858\du}}
\pgfpathlineto{\pgfpoint{2.235752\du}{13.971176\du}}
\pgfpathlineto{\pgfpoint{2.159812\du}{13.981764\du}}
\pgfpathlineto{\pgfpoint{2.081316\du}{13.990892\du}}
\pgfpathlineto{\pgfpoint{2.001725\du}{13.998559\du}}
\pgfpathlineto{\pgfpoint{1.920673\du}{14.005130\du}}
\pgfpathlineto{\pgfpoint{1.837431\du}{14.010242\du}}
\pgfpathlineto{\pgfpoint{1.753824\du}{14.013893\du}}
\pgfpathlineto{\pgfpoint{1.668391\du}{14.015718\du}}
\pgfpathlineto{\pgfpoint{1.581863\du}{14.016813\du}}
\pgfpathlineto{\pgfpoint{1.495701\du}{14.015718\du}}
\pgfpathlineto{\pgfpoint{1.409903\du}{14.013893\du}}
\pgfpathlineto{\pgfpoint{1.326296\du}{14.010242\du}}
\pgfpathlineto{\pgfpoint{1.243419\du}{14.005130\du}}
\pgfpathlineto{\pgfpoint{1.162002\du}{13.998559\du}}
\pgfpathlineto{\pgfpoint{1.082411\du}{13.990892\du}}
\pgfpathlineto{\pgfpoint{1.004646\du}{13.981764\du}}
\pgfpathlineto{\pgfpoint{0.927975\du}{13.971176\du}}
\pgfpathlineto{\pgfpoint{0.853861\du}{13.959858\du}}
\pgfpathlineto{\pgfpoint{0.781206\du}{13.947080\du}}
\pgfpathlineto{\pgfpoint{0.710743\du}{13.933206\du}}
\pgfpathlineto{\pgfpoint{0.642104\du}{13.917872\du}}
\pgfpathlineto{\pgfpoint{0.576387\du}{13.902173\du}}
\pgfpathlineto{\pgfpoint{0.512860\du}{13.885013\du}}
\pgfpathlineto{\pgfpoint{0.451889\du}{13.866394\du}}
\pgfpathlineto{\pgfpoint{0.393474\du}{13.847774\du}}
\pgfpathlineto{\pgfpoint{0.338344\du}{13.827693\du}}
\pgfpathlineto{\pgfpoint{0.285405\du}{13.806518\du}}
\pgfpathlineto{\pgfpoint{0.235387\du}{13.784612\du}}
\pgfpathlineto{\pgfpoint{0.188289\du}{13.762341\du}}
\pgfpathlineto{\pgfpoint{0.144843\du}{13.738975\du}}
\pgfpathlineto{\pgfpoint{0.103952\du}{13.714878\du}}
\pgfpathlineto{\pgfpoint{0.066712\du}{13.689687\du}}
\pgfpathlineto{\pgfpoint{0.033488\du}{13.663765\du}}
\pgfpathlineto{\pgfpoint{0.003185\du}{13.637478\du}}
\pgfpathlineto{\pgfpoint{-0.023467\du}{13.611191\du}}
\pgfpathlineto{\pgfpoint{-0.045373\du}{13.583809\du}}
\pgfpathlineto{\pgfpoint{-0.064723\du}{13.555696\du}}
\pgfpathlineto{\pgfpoint{-0.079327\du}{13.527584\du}}
\pgfpathlineto{\pgfpoint{-0.089915\du}{13.498376\du}}
\pgfpathlineto{\pgfpoint{-0.096851\du}{13.469168\du}}
\pgfpathlineto{\pgfpoint{-0.098677\du}{13.439230\du}}
\pgfpathlineto{\pgfpoint{-0.096851\du}{13.410023\du}}
\pgfpathlineto{\pgfpoint{-0.089915\du}{13.380085\du}}
\pgfpathlineto{\pgfpoint{-0.079327\du}{13.351607\du}}
\pgfpathlineto{\pgfpoint{-0.064723\du}{13.323495\du}}
\pgfpathlineto{\pgfpoint{-0.045373\du}{13.295382\du}}
\pgfpathlineto{\pgfpoint{-0.023467\du}{13.267635\du}}
\pgfpathlineto{\pgfpoint{0.003185\du}{13.240983\du}}
\pgfpathlineto{\pgfpoint{0.033488\du}{13.214696\du}}
\pgfpathlineto{\pgfpoint{0.066712\du}{13.189504\du}}
\pgfpathlineto{\pgfpoint{0.103952\du}{13.164312\du}}
\pgfpathlineto{\pgfpoint{0.144843\du}{13.140216\du}}
\pgfpathlineto{\pgfpoint{0.188289\du}{13.116850\du}}
\pgfpathlineto{\pgfpoint{0.235387\du}{13.093849\du}}
\pgfpathlineto{\pgfpoint{0.285405\du}{13.072308\du}}
\pgfpathlineto{\pgfpoint{0.338344\du}{13.051132\du}}
\pgfpathlineto{\pgfpoint{0.393474\du}{13.031417\du}}
\pgfpathlineto{\pgfpoint{0.451889\du}{13.012067\du}}
\pgfpathlineto{\pgfpoint{0.512860\du}{12.993812\du}}
\pgfpathlineto{\pgfpoint{0.576387\du}{12.977018\du}}
\pgfpathlineto{\pgfpoint{0.642104\du}{12.960588\du}}
\pgfpathlineto{\pgfpoint{0.710743\du}{12.945254\du}}
\pgfpathlineto{\pgfpoint{0.781206\du}{12.931746\du}}
\pgfpathlineto{\pgfpoint{0.853861\du}{12.918967\du}}
\pgfpathlineto{\pgfpoint{0.927975\du}{12.907284\du}}
\pgfpathlineto{\pgfpoint{1.004646\du}{12.897427\du}}
\pgfpathlineto{\pgfpoint{1.082411\du}{12.887934\du}}
\pgfpathlineto{\pgfpoint{1.162002\du}{12.880267\du}}
\pgfpathlineto{\pgfpoint{1.243419\du}{12.873695\du}}
\pgfpathlineto{\pgfpoint{1.326296\du}{12.868584\du}}
\pgfpathlineto{\pgfpoint{1.409903\du}{12.865298\du}}
\pgfpathlineto{\pgfpoint{1.495701\du}{12.862743\du}}
\pgfpathlineto{\pgfpoint{1.581863\du}{12.862377\du}}
\pgfpathlineto{\pgfpoint{1.668391\du}{12.862743\du}}
\pgfpathlineto{\pgfpoint{1.753824\du}{12.865298\du}}
\pgfpathlineto{\pgfpoint{1.837431\du}{12.868584\du}}
\pgfpathlineto{\pgfpoint{1.920673\du}{12.873695\du}}
\pgfpathlineto{\pgfpoint{2.001725\du}{12.880267\du}}
\pgfpathlineto{\pgfpoint{2.081316\du}{12.887934\du}}
\pgfpathlineto{\pgfpoint{2.159812\du}{12.897427\du}}
\pgfpathlineto{\pgfpoint{2.235752\du}{12.907284\du}}
\pgfpathlineto{\pgfpoint{2.310597\du}{12.918967\du}}
\pgfpathlineto{\pgfpoint{2.382521\du}{12.931746\du}}
\pgfpathlineto{\pgfpoint{2.453349\du}{12.945254\du}}
\pgfpathlineto{\pgfpoint{2.521988\du}{12.960588\du}}
\pgfpathlineto{\pgfpoint{2.586975\du}{12.977018\du}}
\pgfpathlineto{\pgfpoint{2.650867\du}{12.993812\du}}
\pgfpathlineto{\pgfpoint{2.711473\du}{13.012067\du}}
\pgfpathlineto{\pgfpoint{2.769523\du}{13.031417\du}}
\pgfpathlineto{\pgfpoint{2.825018\du}{13.051132\du}}
\pgfpathlineto{\pgfpoint{2.877957\du}{13.072308\du}}
\pgfpathlineto{\pgfpoint{2.928340\du}{13.093849\du}}
\pgfpathlineto{\pgfpoint{2.974708\du}{13.116850\du}}
\pgfpathlineto{\pgfpoint{3.018519\du}{13.140216\du}}
\pgfpathlineto{\pgfpoint{3.059410\du}{13.164312\du}}
\pgfpathlineto{\pgfpoint{3.096650\du}{13.189504\du}}
\pgfpathlineto{\pgfpoint{3.129509\du}{13.214696\du}}
\pgfpathlineto{\pgfpoint{3.160177\du}{13.240983\du}}
\pgfpathlineto{\pgfpoint{3.186829\du}{13.267635\du}}
\pgfpathlineto{\pgfpoint{3.208735\du}{13.295382\du}}
\pgfpathlineto{\pgfpoint{3.228085\du}{13.323495\du}}
\pgfpathlineto{\pgfpoint{3.242689\du}{13.351607\du}}
\pgfpathlineto{\pgfpoint{3.252911\du}{13.380085\du}}
\pgfpathlineto{\pgfpoint{3.260213\du}{13.410023\du}}
\pgfpathlineto{\pgfpoint{3.261674\du}{13.439230\du}}
\pgfusepath{fill}
\pgfsetbuttcap
\pgfsetmiterjoin
\pgfsetdash{}{0pt}
\definecolor{dialinecolor}{rgb}{0.678431, 0.839216, 0.905882}
\pgfsetfillcolor{dialinecolor}
\pgfpathmoveto{\pgfpoint{1.581863\du}{14.026671\du}}
\pgfpathlineto{\pgfpoint{1.581863\du}{14.026671\du}}
\pgfpathlineto{\pgfpoint{1.625310\du}{14.026671\du}}
\pgfpathlineto{\pgfpoint{1.668756\du}{14.025941\du}}
\pgfpathlineto{\pgfpoint{1.711838\du}{14.024846\du}}
\pgfpathlineto{\pgfpoint{1.753824\du}{14.023750\du}}
\pgfpathlineto{\pgfpoint{1.796540\du}{14.021925\du}}
\pgfpathlineto{\pgfpoint{1.838161\du}{14.020099\du}}
\pgfpathlineto{\pgfpoint{1.879782\du}{14.017909\du}}
\pgfpathlineto{\pgfpoint{1.921403\du}{14.014988\du}}
\pgfpathlineto{\pgfpoint{1.961929\du}{14.012067\du}}
\pgfpathlineto{\pgfpoint{2.002820\du}{14.009146\du}}
\pgfpathlineto{\pgfpoint{2.042616\du}{14.005130\du}}
\pgfpathlineto{\pgfpoint{2.082776\du}{14.001114\du}}
\pgfpathlineto{\pgfpoint{2.121476\du}{13.996368\du}}
\pgfpathlineto{\pgfpoint{2.160907\du}{13.991622\du}}
\pgfpathlineto{\pgfpoint{2.198877\du}{13.986875\du}}
\pgfpathlineto{\pgfpoint{2.237942\du}{13.981764\du}}
\pgfpathlineto{\pgfpoint{2.275182\du}{13.975923\du}}
\pgfpathlineto{\pgfpoint{2.312057\du}{13.970081\du}}
\pgfpathlineto{\pgfpoint{2.348567\du}{13.963509\du}}
\pgfpathlineto{\pgfpoint{2.385076\du}{13.956938\du}}
\pgfpathlineto{\pgfpoint{2.420491\du}{13.950001\du}}
\pgfpathlineto{\pgfpoint{2.455175\du}{13.943064\du}}
\pgfpathlineto{\pgfpoint{2.489859\du}{13.936127\du}}
\pgfpathlineto{\pgfpoint{2.523813\du}{13.928460\du}}
\pgfpathlineto{\pgfpoint{2.557402\du}{13.920063\du}}
\pgfpathlineto{\pgfpoint{2.590261\du}{13.912031\du}}
\pgfpathlineto{\pgfpoint{2.622024\du}{13.903633\du}}
\pgfpathlineto{\pgfpoint{2.653422\du}{13.894506\du}}
\pgfpathlineto{\pgfpoint{2.668756\du}{13.890490\du}}
\pgfpathlineto{\pgfpoint{2.684091\du}{13.885744\du}}
\pgfpathlineto{\pgfpoint{2.700155\du}{13.880997\du}}
\pgfpathlineto{\pgfpoint{2.714759\du}{13.876251\du}}
\pgfpathlineto{\pgfpoint{2.728997\du}{13.871505\du}}
\pgfpathlineto{\pgfpoint{2.743966\du}{13.867124\du}}
\pgfpathlineto{\pgfpoint{2.758935\du}{13.862377\du}}
\pgfpathlineto{\pgfpoint{2.772809\du}{13.856901\du}}
\pgfpathlineto{\pgfpoint{2.787048\du}{13.852155\du}}
\pgfpathlineto{\pgfpoint{2.800922\du}{13.847043\du}}
\pgfpathlineto{\pgfpoint{2.815160\du}{13.842297\du}}
\pgfpathlineto{\pgfpoint{2.828304\du}{13.837186\du}}
\pgfpathlineto{\pgfpoint{2.842543\du}{13.831709\du}}
\pgfpathlineto{\pgfpoint{2.855686\du}{13.826598\du}}
\pgfpathlineto{\pgfpoint{2.868830\du}{13.821487\du}}
\pgfpathlineto{\pgfpoint{2.882338\du}{13.816010\du}}
\pgfpathlineto{\pgfpoint{2.895116\du}{13.810899\du}}
\pgfpathlineto{\pgfpoint{2.907165\du}{13.805422\du}}
\pgfpathlineto{\pgfpoint{2.919943\du}{13.799581\du}}
\pgfpathlineto{\pgfpoint{2.932356\du}{13.793739\du}}
\pgfpathlineto{\pgfpoint{2.944770\du}{13.788628\du}}
\pgfpathlineto{\pgfpoint{2.956088\du}{13.782786\du}}
\pgfpathlineto{\pgfpoint{2.968136\du}{13.776945\du}}
\pgfpathlineto{\pgfpoint{2.979089\du}{13.771103\du}}
\pgfpathlineto{\pgfpoint{2.990772\du}{13.765627\du}}
\pgfpathlineto{\pgfpoint{3.002090\du}{13.759785\du}}
\pgfpathlineto{\pgfpoint{3.013408\du}{13.753579\du}}
\pgfpathlineto{\pgfpoint{3.023996\du}{13.747737\du}}
\pgfpathlineto{\pgfpoint{3.033853\du}{13.741896\du}}
\pgfpathlineto{\pgfpoint{3.044441\du}{13.736054\du}}
\pgfpathlineto{\pgfpoint{3.054664\du}{13.729482\du}}
\pgfpathlineto{\pgfpoint{3.064521\du}{13.722910\du}}
\pgfpathlineto{\pgfpoint{3.074379\du}{13.717069\du}}
\pgfpathlineto{\pgfpoint{3.083506\du}{13.710862\du}}
\pgfpathlineto{\pgfpoint{3.092634\du}{13.704291\du}}
\pgfpathlineto{\pgfpoint{3.102126\du}{13.697719\du}}
\pgfpathlineto{\pgfpoint{3.110889\du}{13.691512\du}}
\pgfpathlineto{\pgfpoint{3.120016\du}{13.684940\du}}
\pgfpathlineto{\pgfpoint{3.128048\du}{13.678369\du}}
\pgfpathlineto{\pgfpoint{3.136811\du}{13.672162\du}}
\pgfpathlineto{\pgfpoint{3.144112\du}{13.665590\du}}
\pgfpathlineto{\pgfpoint{3.152145\du}{13.658653\du}}
\pgfpathlineto{\pgfpoint{3.160177\du}{13.652082\du}}
\pgfpathlineto{\pgfpoint{3.166748\du}{13.645145\du}}
\pgfpathlineto{\pgfpoint{3.174050\du}{13.638573\du}}
\pgfpathlineto{\pgfpoint{3.180622\du}{13.631636\du}}
\pgfpathlineto{\pgfpoint{3.187559\du}{13.624699\du}}
\pgfpathlineto{\pgfpoint{3.194131\du}{13.618128\du}}
\pgfpathlineto{\pgfpoint{3.200337\du}{13.611191\du}}
\pgfpathlineto{\pgfpoint{3.206179\du}{13.604254\du}}
\pgfpathlineto{\pgfpoint{3.211655\du}{13.597317\du}}
\pgfpathlineto{\pgfpoint{3.217497\du}{13.589650\du}}
\pgfpathlineto{\pgfpoint{3.222243\du}{13.582713\du}}
\pgfpathlineto{\pgfpoint{3.227720\du}{13.575411\du}}
\pgfpathlineto{\pgfpoint{3.232466\du}{13.568475\du}}
\pgfpathlineto{\pgfpoint{3.236847\du}{13.560808\du}}
\pgfpathlineto{\pgfpoint{3.240863\du}{13.553871\du}}
\pgfpathlineto{\pgfpoint{3.244879\du}{13.546204\du}}
\pgfpathlineto{\pgfpoint{3.248165\du}{13.538537\du}}
\pgfpathlineto{\pgfpoint{3.252181\du}{13.531235\du}}
\pgfpathlineto{\pgfpoint{3.255467\du}{13.523933\du}}
\pgfpathlineto{\pgfpoint{3.258023\du}{13.516631\du}}
\pgfpathlineto{\pgfpoint{3.260943\du}{13.508964\du}}
\pgfpathlineto{\pgfpoint{3.262769\du}{13.501297\du}}
\pgfpathlineto{\pgfpoint{3.265690\du}{13.493630\du}}
\pgfpathlineto{\pgfpoint{3.266785\du}{13.485963\du}}
\pgfpathlineto{\pgfpoint{3.268976\du}{13.478296\du}}
\pgfpathlineto{\pgfpoint{3.270071\du}{13.470994\du}}
\pgfpathlineto{\pgfpoint{3.270801\du}{13.462597\du}}
\pgfpathlineto{\pgfpoint{3.271531\du}{13.454929\du}}
\pgfpathlineto{\pgfpoint{3.272261\du}{13.447262\du}}
\pgfpathlineto{\pgfpoint{3.272261\du}{13.439230\du}}
\pgfpathlineto{\pgfpoint{3.252181\du}{13.439230\du}}
\pgfpathlineto{\pgfpoint{3.251451\du}{13.446167\du}}
\pgfpathlineto{\pgfpoint{3.251451\du}{13.453834\du}}
\pgfpathlineto{\pgfpoint{3.251086\du}{13.460771\du}}
\pgfpathlineto{\pgfpoint{3.249260\du}{13.467708\du}}
\pgfpathlineto{\pgfpoint{3.248165\du}{13.475010\du}}
\pgfpathlineto{\pgfpoint{3.247435\du}{13.481216\du}}
\pgfpathlineto{\pgfpoint{3.245244\du}{13.488518\du}}
\pgfpathlineto{\pgfpoint{3.243784\du}{13.495455\du}}
\pgfpathlineto{\pgfpoint{3.241593\du}{13.502392\du}}
\pgfpathlineto{\pgfpoint{3.239038\du}{13.509329\du}}
\pgfpathlineto{\pgfpoint{3.236117\du}{13.516631\du}}
\pgfpathlineto{\pgfpoint{3.233561\du}{13.522837\du}}
\pgfpathlineto{\pgfpoint{3.229545\du}{13.529774\du}}
\pgfpathlineto{\pgfpoint{3.226624\du}{13.537076\du}}
\pgfpathlineto{\pgfpoint{3.222608\du}{13.544013\du}}
\pgfpathlineto{\pgfpoint{3.219687\du}{13.550220\du}}
\pgfpathlineto{\pgfpoint{3.215306\du}{13.557522\du}}
\pgfpathlineto{\pgfpoint{3.210925\du}{13.563728\du}}
\pgfpathlineto{\pgfpoint{3.206179\du}{13.571030\du}}
\pgfpathlineto{\pgfpoint{3.201068\du}{13.577237\du}}
\pgfpathlineto{\pgfpoint{3.196321\du}{13.584174\du}}
\pgfpathlineto{\pgfpoint{3.190115\du}{13.590745\du}}
\pgfpathlineto{\pgfpoint{3.184638\du}{13.597317\du}}
\pgfpathlineto{\pgfpoint{3.179527\du}{13.604254\du}}
\pgfpathlineto{\pgfpoint{3.172955\du}{13.610826\du}}
\pgfpathlineto{\pgfpoint{3.166018\du}{13.617032\du}}
\pgfpathlineto{\pgfpoint{3.160177\du}{13.623604\du}}
\pgfpathlineto{\pgfpoint{3.152875\du}{13.630541\du}}
\pgfpathlineto{\pgfpoint{3.146303\du}{13.637113\du}}
\pgfpathlineto{\pgfpoint{3.138636\du}{13.643319\du}}
\pgfpathlineto{\pgfpoint{3.131334\du}{13.649891\du}}
\pgfpathlineto{\pgfpoint{3.123302\du}{13.655733\du}}
\pgfpathlineto{\pgfpoint{3.115635\du}{13.662304\du}}
\pgfpathlineto{\pgfpoint{3.106873\du}{13.668511\du}}
\pgfpathlineto{\pgfpoint{3.098475\du}{13.675083\du}}
\pgfpathlineto{\pgfpoint{3.089713\du}{13.681289\du}}
\pgfpathlineto{\pgfpoint{3.081681\du}{13.687131\du}}
\pgfpathlineto{\pgfpoint{3.072919\du}{13.693703\du}}
\pgfpathlineto{\pgfpoint{3.063426\du}{13.699544\du}}
\pgfpathlineto{\pgfpoint{3.053203\du}{13.706116\du}}
\pgfpathlineto{\pgfpoint{3.044076\du}{13.711958\du}}
\pgfpathlineto{\pgfpoint{3.033853\du}{13.717799\du}}
\pgfpathlineto{\pgfpoint{3.023996\du}{13.723641\du}}
\pgfpathlineto{\pgfpoint{3.014138\du}{13.729482\du}}
\pgfpathlineto{\pgfpoint{3.003185\du}{13.736054\du}}
\pgfpathlineto{\pgfpoint{2.992232\du}{13.741165\du}}
\pgfpathlineto{\pgfpoint{2.982010\du}{13.747007\du}}
\pgfpathlineto{\pgfpoint{2.969961\du}{13.752848\du}}
\pgfpathlineto{\pgfpoint{2.959374\du}{13.758690\du}}
\pgfpathlineto{\pgfpoint{2.947325\du}{13.764532\du}}
\pgfpathlineto{\pgfpoint{2.936007\du}{13.769643\du}}
\pgfpathlineto{\pgfpoint{2.923594\du}{13.775119\du}}
\pgfpathlineto{\pgfpoint{2.911546\du}{13.780961\du}}
\pgfpathlineto{\pgfpoint{2.899133\du}{13.786072\du}}
\pgfpathlineto{\pgfpoint{2.887084\du}{13.791549\du}}
\pgfpathlineto{\pgfpoint{2.873941\du}{13.796660\du}}
\pgfpathlineto{\pgfpoint{2.861162\du}{13.802136\du}}
\pgfpathlineto{\pgfpoint{2.848384\du}{13.807978\du}}
\pgfpathlineto{\pgfpoint{2.835241\du}{13.812359\du}}
\pgfpathlineto{\pgfpoint{2.821732\du}{13.817836\du}}
\pgfpathlineto{\pgfpoint{2.808589\du}{13.822947\du}}
\pgfpathlineto{\pgfpoint{2.794350\du}{13.827693\du}}
\pgfpathlineto{\pgfpoint{2.780476\du}{13.833170\du}}
\pgfpathlineto{\pgfpoint{2.766237\du}{13.837551\du}}
\pgfpathlineto{\pgfpoint{2.751633\du}{13.842297\du}}
\pgfpathlineto{\pgfpoint{2.737760\du}{13.847774\du}}
\pgfpathlineto{\pgfpoint{2.723156\du}{13.852155\du}}
\pgfpathlineto{\pgfpoint{2.708917\du}{13.856901\du}}
\pgfpathlineto{\pgfpoint{2.693218\du}{13.861647\du}}
\pgfpathlineto{\pgfpoint{2.678979\du}{13.865663\du}}
\pgfpathlineto{\pgfpoint{2.663280\du}{13.870410\du}}
\pgfpathlineto{\pgfpoint{2.648311\du}{13.875156\du}}
\pgfpathlineto{\pgfpoint{2.616913\du}{13.883188\du}}
\pgfpathlineto{\pgfpoint{2.584784\du}{13.891950\du}}
\pgfpathlineto{\pgfpoint{2.552656\du}{13.900348\du}}
\pgfpathlineto{\pgfpoint{2.519432\du}{13.908015\du}}
\pgfpathlineto{\pgfpoint{2.485113\du}{13.915682\du}}
\pgfpathlineto{\pgfpoint{2.451159\du}{13.922984\du}}
\pgfpathlineto{\pgfpoint{2.416475\du}{13.930285\du}}
\pgfpathlineto{\pgfpoint{2.380695\du}{13.937222\du}}
\pgfpathlineto{\pgfpoint{2.345281\du}{13.943429\du}}
\pgfpathlineto{\pgfpoint{2.308406\du}{13.949270\du}}
\pgfpathlineto{\pgfpoint{2.271896\du}{13.955477\du}}
\pgfpathlineto{\pgfpoint{2.234656\du}{13.961319\du}}
\pgfpathlineto{\pgfpoint{2.197051\du}{13.966430\du}}
\pgfpathlineto{\pgfpoint{2.158351\du}{13.971541\du}}
\pgfpathlineto{\pgfpoint{2.119651\du}{13.976288\du}}
\pgfpathlineto{\pgfpoint{2.080221\du}{13.980304\du}}
\pgfpathlineto{\pgfpoint{2.041155\du}{13.984685\du}}
\pgfpathlineto{\pgfpoint{2.000995\du}{13.987971\du}}
\pgfpathlineto{\pgfpoint{1.960834\du}{13.991622\du}}
\pgfpathlineto{\pgfpoint{1.919578\du}{13.994542\du}}
\pgfpathlineto{\pgfpoint{1.878687\du}{13.997463\du}}
\pgfpathlineto{\pgfpoint{1.837431\du}{13.999654\du}}
\pgfpathlineto{\pgfpoint{1.795810\du}{14.002210\du}}
\pgfpathlineto{\pgfpoint{1.753094\du}{14.003305\du}}
\pgfpathlineto{\pgfpoint{1.710378\du}{14.005130\du}}
\pgfpathlineto{\pgfpoint{1.668391\du}{14.005495\du}}
\pgfpathlineto{\pgfpoint{1.624945\du}{14.006226\du}}
\pgfpathlineto{\pgfpoint{1.581863\du}{14.006226\du}}
\pgfpathlineto{\pgfpoint{1.581863\du}{14.006226\du}}
\pgfpathlineto{\pgfpoint{1.581863\du}{14.006226\du}}
\pgfpathlineto{\pgfpoint{1.581133\du}{14.006956\du}}
\pgfpathlineto{\pgfpoint{1.579308\du}{14.006956\du}}
\pgfpathlineto{\pgfpoint{1.578212\du}{14.006956\du}}
\pgfpathlineto{\pgfpoint{1.577482\du}{14.007321\du}}
\pgfpathlineto{\pgfpoint{1.577117\du}{14.008051\du}}
\pgfpathlineto{\pgfpoint{1.575657\du}{14.008051\du}}
\pgfpathlineto{\pgfpoint{1.574927\du}{14.009146\du}}
\pgfpathlineto{\pgfpoint{1.574196\du}{14.009877\du}}
\pgfpathlineto{\pgfpoint{1.573101\du}{14.010972\du}}
\pgfpathlineto{\pgfpoint{1.572371\du}{14.012797\du}}
\pgfpathlineto{\pgfpoint{1.572371\du}{14.014988\du}}
\pgfpathlineto{\pgfpoint{1.571641\du}{14.016813\du}}
\pgfpathlineto{\pgfpoint{1.572371\du}{14.018639\du}}
\pgfpathlineto{\pgfpoint{1.572371\du}{14.020099\du}}
\pgfpathlineto{\pgfpoint{1.573101\du}{14.021925\du}}
\pgfpathlineto{\pgfpoint{1.574196\du}{14.023750\du}}
\pgfpathlineto{\pgfpoint{1.574927\du}{14.024480\du}}
\pgfpathlineto{\pgfpoint{1.575657\du}{14.024846\du}}
\pgfpathlineto{\pgfpoint{1.577117\du}{14.025576\du}}
\pgfpathlineto{\pgfpoint{1.577482\du}{14.025941\du}}
\pgfpathlineto{\pgfpoint{1.578212\du}{14.026671\du}}
\pgfpathlineto{\pgfpoint{1.579308\du}{14.026671\du}}
\pgfpathlineto{\pgfpoint{1.581133\du}{14.026671\du}}
\pgfpathlineto{\pgfpoint{1.581863\du}{14.026671\du}}
\pgfusepath{fill}
\pgfsetbuttcap
\pgfsetmiterjoin
\pgfsetdash{}{0pt}
\definecolor{dialinecolor}{rgb}{0.678431, 0.839216, 0.905882}
\pgfsetfillcolor{dialinecolor}
\pgfpathmoveto{\pgfpoint{-0.108900\du}{13.439230\du}}
\pgfpathlineto{\pgfpoint{-0.108900\du}{13.439230\du}}
\pgfpathlineto{\pgfpoint{-0.108900\du}{13.447262\du}}
\pgfpathlineto{\pgfpoint{-0.108534\du}{13.454929\du}}
\pgfpathlineto{\pgfpoint{-0.107804\du}{13.462597\du}}
\pgfpathlineto{\pgfpoint{-0.106709\du}{13.470994\du}}
\pgfpathlineto{\pgfpoint{-0.105614\du}{13.478296\du}}
\pgfpathlineto{\pgfpoint{-0.103788\du}{13.485963\du}}
\pgfpathlineto{\pgfpoint{-0.101963\du}{13.493630\du}}
\pgfpathlineto{\pgfpoint{-0.099772\du}{13.501297\du}}
\pgfpathlineto{\pgfpoint{-0.097582\du}{13.508964\du}}
\pgfpathlineto{\pgfpoint{-0.095026\du}{13.516631\du}}
\pgfpathlineto{\pgfpoint{-0.092105\du}{13.523933\du}}
\pgfpathlineto{\pgfpoint{-0.088454\du}{13.531235\du}}
\pgfpathlineto{\pgfpoint{-0.085168\du}{13.538537\du}}
\pgfpathlineto{\pgfpoint{-0.081517\du}{13.546204\du}}
\pgfpathlineto{\pgfpoint{-0.077136\du}{13.553871\du}}
\pgfpathlineto{\pgfpoint{-0.073485\du}{13.560808\du}}
\pgfpathlineto{\pgfpoint{-0.068374\du}{13.568475\du}}
\pgfpathlineto{\pgfpoint{-0.064358\du}{13.575411\du}}
\pgfpathlineto{\pgfpoint{-0.058881\du}{13.582713\du}}
\pgfpathlineto{\pgfpoint{-0.054135\du}{13.589650\du}}
\pgfpathlineto{\pgfpoint{-0.048659\du}{13.597317\du}}
\pgfpathlineto{\pgfpoint{-0.042817\du}{13.604254\du}}
\pgfpathlineto{\pgfpoint{-0.036976\du}{13.611191\du}}
\pgfpathlineto{\pgfpoint{-0.031134\du}{13.618128\du}}
\pgfpathlineto{\pgfpoint{-0.024197\du}{13.624699\du}}
\pgfpathlineto{\pgfpoint{-0.017625\du}{13.631636\du}}
\pgfpathlineto{\pgfpoint{-0.010689\du}{13.638573\du}}
\pgfpathlineto{\pgfpoint{-0.003752\du}{13.645145\du}}
\pgfpathlineto{\pgfpoint{0.003185\du}{13.652082\du}}
\pgfpathlineto{\pgfpoint{0.011947\du}{13.658653\du}}
\pgfpathlineto{\pgfpoint{0.018884\du}{13.665590\du}}
\pgfpathlineto{\pgfpoint{0.026551\du}{13.672162\du}}
\pgfpathlineto{\pgfpoint{0.035314\du}{13.678369\du}}
\pgfpathlineto{\pgfpoint{0.043711\du}{13.684940\du}}
\pgfpathlineto{\pgfpoint{0.052473\du}{13.691512\du}}
\pgfpathlineto{\pgfpoint{0.060870\du}{13.697719\du}}
\pgfpathlineto{\pgfpoint{0.070728\du}{13.704291\du}}
\pgfpathlineto{\pgfpoint{0.079855\du}{13.710862\du}}
\pgfpathlineto{\pgfpoint{0.089348\du}{13.717069\du}}
\pgfpathlineto{\pgfpoint{0.098840\du}{13.722910\du}}
\pgfpathlineto{\pgfpoint{0.108698\du}{13.729482\du}}
\pgfpathlineto{\pgfpoint{0.118921\du}{13.736054\du}}
\pgfpathlineto{\pgfpoint{0.129143\du}{13.741896\du}}
\pgfpathlineto{\pgfpoint{0.139731\du}{13.747737\du}}
\pgfpathlineto{\pgfpoint{0.149954\du}{13.753579\du}}
\pgfpathlineto{\pgfpoint{0.160907\du}{13.759785\du}}
\pgfpathlineto{\pgfpoint{0.172590\du}{13.765627\du}}
\pgfpathlineto{\pgfpoint{0.183908\du}{13.771103\du}}
\pgfpathlineto{\pgfpoint{0.195591\du}{13.776945\du}}
\pgfpathlineto{\pgfpoint{0.206909\du}{13.782786\du}}
\pgfpathlineto{\pgfpoint{0.218592\du}{13.788628\du}}
\pgfpathlineto{\pgfpoint{0.231005\du}{13.793739\du}}
\pgfpathlineto{\pgfpoint{0.243054\du}{13.799581\du}}
\pgfpathlineto{\pgfpoint{0.256197\du}{13.805422\du}}
\pgfpathlineto{\pgfpoint{0.268245\du}{13.810899\du}}
\pgfpathlineto{\pgfpoint{0.281024\du}{13.816010\du}}
\pgfpathlineto{\pgfpoint{0.294897\du}{13.821487\du}}
\pgfpathlineto{\pgfpoint{0.307311\du}{13.826598\du}}
\pgfpathlineto{\pgfpoint{0.320454\du}{13.831709\du}}
\pgfpathlineto{\pgfpoint{0.334693\du}{13.837186\du}}
\pgfpathlineto{\pgfpoint{0.347836\du}{13.842297\du}}
\pgfpathlineto{\pgfpoint{0.362075\du}{13.847043\du}}
\pgfpathlineto{\pgfpoint{0.375949\du}{13.852155\du}}
\pgfpathlineto{\pgfpoint{0.390553\du}{13.856901\du}}
\pgfpathlineto{\pgfpoint{0.404426\du}{13.862377\du}}
\pgfpathlineto{\pgfpoint{0.420126\du}{13.867124\du}}
\pgfpathlineto{\pgfpoint{0.433999\du}{13.871505\du}}
\pgfpathlineto{\pgfpoint{0.448603\du}{13.876251\du}}
\pgfpathlineto{\pgfpoint{0.463937\du}{13.880997\du}}
\pgfpathlineto{\pgfpoint{0.479636\du}{13.885744\du}}
\pgfpathlineto{\pgfpoint{0.494605\du}{13.890490\du}}
\pgfpathlineto{\pgfpoint{0.510305\du}{13.894506\du}}
\pgfpathlineto{\pgfpoint{0.542068\du}{13.903633\du}}
\pgfpathlineto{\pgfpoint{0.573831\du}{13.912031\du}}
\pgfpathlineto{\pgfpoint{0.607055\du}{13.920063\du}}
\pgfpathlineto{\pgfpoint{0.639549\du}{13.928460\du}}
\pgfpathlineto{\pgfpoint{0.674233\du}{13.936127\du}}
\pgfpathlineto{\pgfpoint{0.708552\du}{13.943064\du}}
\pgfpathlineto{\pgfpoint{0.743236\du}{13.950001\du}}
\pgfpathlineto{\pgfpoint{0.779016\du}{13.956938\du}}
\pgfpathlineto{\pgfpoint{0.815160\du}{13.963509\du}}
\pgfpathlineto{\pgfpoint{0.851670\du}{13.970081\du}}
\pgfpathlineto{\pgfpoint{0.888910\du}{13.975923\du}}
\pgfpathlineto{\pgfpoint{0.926515\du}{13.981764\du}}
\pgfpathlineto{\pgfpoint{0.964485\du}{13.986875\du}}
\pgfpathlineto{\pgfpoint{1.003185\du}{13.991622\du}}
\pgfpathlineto{\pgfpoint{1.041885\du}{13.996368\du}}
\pgfpathlineto{\pgfpoint{1.080951\du}{14.001114\du}}
\pgfpathlineto{\pgfpoint{1.121111\du}{14.005130\du}}
\pgfpathlineto{\pgfpoint{1.160907\du}{14.009146\du}}
\pgfpathlineto{\pgfpoint{1.201798\du}{14.012067\du}}
\pgfpathlineto{\pgfpoint{1.242689\du}{14.014988\du}}
\pgfpathlineto{\pgfpoint{1.283945\du}{14.017909\du}}
\pgfpathlineto{\pgfpoint{1.325931\du}{14.020099\du}}
\pgfpathlineto{\pgfpoint{1.367552\du}{14.021925\du}}
\pgfpathlineto{\pgfpoint{1.409903\du}{14.023750\du}}
\pgfpathlineto{\pgfpoint{1.451889\du}{14.024846\du}}
\pgfpathlineto{\pgfpoint{1.495336\du}{14.025941\du}}
\pgfpathlineto{\pgfpoint{1.538052\du}{14.026671\du}}
\pgfpathlineto{\pgfpoint{1.581863\du}{14.026671\du}}
\pgfpathlineto{\pgfpoint{1.581863\du}{14.006226\du}}
\pgfpathlineto{\pgfpoint{1.539147\du}{14.006226\du}}
\pgfpathlineto{\pgfpoint{1.495701\du}{14.005495\du}}
\pgfpathlineto{\pgfpoint{1.453349\du}{14.005130\du}}
\pgfpathlineto{\pgfpoint{1.410633\du}{14.003305\du}}
\pgfpathlineto{\pgfpoint{1.368282\du}{14.002210\du}}
\pgfpathlineto{\pgfpoint{1.326296\du}{13.999654\du}}
\pgfpathlineto{\pgfpoint{1.285405\du}{13.997463\du}}
\pgfpathlineto{\pgfpoint{1.244149\du}{13.994542\du}}
\pgfpathlineto{\pgfpoint{1.203258\du}{13.991622\du}}
\pgfpathlineto{\pgfpoint{1.163097\du}{13.987971\du}}
\pgfpathlineto{\pgfpoint{1.123302\du}{13.984685\du}}
\pgfpathlineto{\pgfpoint{1.083871\du}{13.980304\du}}
\pgfpathlineto{\pgfpoint{1.044441\du}{13.976288\du}}
\pgfpathlineto{\pgfpoint{1.005376\du}{13.971541\du}}
\pgfpathlineto{\pgfpoint{0.967041\du}{13.966430\du}}
\pgfpathlineto{\pgfpoint{0.929436\du}{13.961319\du}}
\pgfpathlineto{\pgfpoint{0.892196\du}{13.955477\du}}
\pgfpathlineto{\pgfpoint{0.855686\du}{13.949270\du}}
\pgfpathlineto{\pgfpoint{0.818446\du}{13.943429\du}}
\pgfpathlineto{\pgfpoint{0.783397\du}{13.937222\du}}
\pgfpathlineto{\pgfpoint{0.747252\du}{13.930285\du}}
\pgfpathlineto{\pgfpoint{0.712568\du}{13.922984\du}}
\pgfpathlineto{\pgfpoint{0.678249\du}{13.915682\du}}
\pgfpathlineto{\pgfpoint{0.644295\du}{13.908015\du}}
\pgfpathlineto{\pgfpoint{0.611071\du}{13.900348\du}}
\pgfpathlineto{\pgfpoint{0.579673\du}{13.891950\du}}
\pgfpathlineto{\pgfpoint{0.547179\du}{13.883188\du}}
\pgfpathlineto{\pgfpoint{0.516146\du}{13.875156\du}}
\pgfpathlineto{\pgfpoint{0.500447\du}{13.870410\du}}
\pgfpathlineto{\pgfpoint{0.484748\du}{13.865663\du}}
\pgfpathlineto{\pgfpoint{0.470144\du}{13.861647\du}}
\pgfpathlineto{\pgfpoint{0.455175\du}{13.856901\du}}
\pgfpathlineto{\pgfpoint{0.439841\du}{13.852155\du}}
\pgfpathlineto{\pgfpoint{0.425602\du}{13.847774\du}}
\pgfpathlineto{\pgfpoint{0.411363\du}{13.842297\du}}
\pgfpathlineto{\pgfpoint{0.397125\du}{13.837551\du}}
\pgfpathlineto{\pgfpoint{0.382886\du}{13.833170\du}}
\pgfpathlineto{\pgfpoint{0.369012\du}{13.827693\du}}
\pgfpathlineto{\pgfpoint{0.355138\du}{13.822947\du}}
\pgfpathlineto{\pgfpoint{0.341630\du}{13.817836\du}}
\pgfpathlineto{\pgfpoint{0.328121\du}{13.812359\du}}
\pgfpathlineto{\pgfpoint{0.315343\du}{13.807978\du}}
\pgfpathlineto{\pgfpoint{0.301834\du}{13.802136\du}}
\pgfpathlineto{\pgfpoint{0.289056\du}{13.796660\du}}
\pgfpathlineto{\pgfpoint{0.276643\du}{13.791549\du}}
\pgfpathlineto{\pgfpoint{0.263499\du}{13.786072\du}}
\pgfpathlineto{\pgfpoint{0.251451\du}{13.780961\du}}
\pgfpathlineto{\pgfpoint{0.240133\du}{13.775119\du}}
\pgfpathlineto{\pgfpoint{0.227355\du}{13.769643\du}}
\pgfpathlineto{\pgfpoint{0.215671\du}{13.764532\du}}
\pgfpathlineto{\pgfpoint{0.203988\du}{13.758690\du}}
\pgfpathlineto{\pgfpoint{0.193035\du}{13.752848\du}}
\pgfpathlineto{\pgfpoint{0.181352\du}{13.747007\du}}
\pgfpathlineto{\pgfpoint{0.171495\du}{13.741165\du}}
\pgfpathlineto{\pgfpoint{0.160177\du}{13.736054\du}}
\pgfpathlineto{\pgfpoint{0.149589\du}{13.729482\du}}
\pgfpathlineto{\pgfpoint{0.139731\du}{13.723641\du}}
\pgfpathlineto{\pgfpoint{0.129143\du}{13.717799\du}}
\pgfpathlineto{\pgfpoint{0.119286\du}{13.711958\du}}
\pgfpathlineto{\pgfpoint{0.109793\du}{13.706116\du}}
\pgfpathlineto{\pgfpoint{0.099936\du}{13.699544\du}}
\pgfpathlineto{\pgfpoint{0.090808\du}{13.693703\du}}
\pgfpathlineto{\pgfpoint{0.082411\du}{13.687131\du}}
\pgfpathlineto{\pgfpoint{0.073649\du}{13.681289\du}}
\pgfpathlineto{\pgfpoint{0.064521\du}{13.675083\du}}
\pgfpathlineto{\pgfpoint{0.055759\du}{13.668511\du}}
\pgfpathlineto{\pgfpoint{0.047362\du}{13.662304\du}}
\pgfpathlineto{\pgfpoint{0.040060\du}{13.655733\du}}
\pgfpathlineto{\pgfpoint{0.032028\du}{13.649891\du}}
\pgfpathlineto{\pgfpoint{0.024361\du}{13.643319\du}}
\pgfpathlineto{\pgfpoint{0.017424\du}{13.637113\du}}
\pgfpathlineto{\pgfpoint{0.010122\du}{13.630541\du}}
\pgfpathlineto{\pgfpoint{0.003185\du}{13.623604\du}}
\pgfpathlineto{\pgfpoint{-0.003022\du}{13.617032\du}}
\pgfpathlineto{\pgfpoint{-0.009593\du}{13.610826\du}}
\pgfpathlineto{\pgfpoint{-0.015435\du}{13.604254\du}}
\pgfpathlineto{\pgfpoint{-0.021641\du}{13.597317\du}}
\pgfpathlineto{\pgfpoint{-0.026753\du}{13.590745\du}}
\pgfpathlineto{\pgfpoint{-0.032959\du}{13.584174\du}}
\pgfpathlineto{\pgfpoint{-0.037341\du}{13.577237\du}}
\pgfpathlineto{\pgfpoint{-0.042817\du}{13.571030\du}}
\pgfpathlineto{\pgfpoint{-0.047198\du}{13.563728\du}}
\pgfpathlineto{\pgfpoint{-0.051579\du}{13.557522\du}}
\pgfpathlineto{\pgfpoint{-0.055961\du}{13.550220\du}}
\pgfpathlineto{\pgfpoint{-0.059246\du}{13.544013\du}}
\pgfpathlineto{\pgfpoint{-0.063262\du}{13.537076\du}}
\pgfpathlineto{\pgfpoint{-0.067279\du}{13.529774\du}}
\pgfpathlineto{\pgfpoint{-0.070199\du}{13.522837\du}}
\pgfpathlineto{\pgfpoint{-0.072755\du}{13.516631\du}}
\pgfpathlineto{\pgfpoint{-0.076041\du}{13.509329\du}}
\pgfpathlineto{\pgfpoint{-0.078231\du}{13.502392\du}}
\pgfpathlineto{\pgfpoint{-0.080422\du}{13.495455\du}}
\pgfpathlineto{\pgfpoint{-0.081882\du}{13.488518\du}}
\pgfpathlineto{\pgfpoint{-0.084073\du}{13.481216\du}}
\pgfpathlineto{\pgfpoint{-0.085168\du}{13.475010\du}}
\pgfpathlineto{\pgfpoint{-0.086264\du}{13.467708\du}}
\pgfpathlineto{\pgfpoint{-0.087724\du}{13.460771\du}}
\pgfpathlineto{\pgfpoint{-0.088089\du}{13.453834\du}}
\pgfpathlineto{\pgfpoint{-0.088089\du}{13.446167\du}}
\pgfpathlineto{\pgfpoint{-0.088454\du}{13.439230\du}}
\pgfpathlineto{\pgfpoint{-0.088454\du}{13.439230\du}}
\pgfpathlineto{\pgfpoint{-0.088454\du}{13.439230\du}}
\pgfpathlineto{\pgfpoint{-0.088454\du}{13.438135\du}}
\pgfpathlineto{\pgfpoint{-0.088454\du}{13.437040\du}}
\pgfpathlineto{\pgfpoint{-0.088819\du}{13.435579\du}}
\pgfpathlineto{\pgfpoint{-0.088819\du}{13.435214\du}}
\pgfpathlineto{\pgfpoint{-0.089915\du}{13.434119\du}}
\pgfpathlineto{\pgfpoint{-0.090280\du}{13.432659\du}}
\pgfpathlineto{\pgfpoint{-0.090645\du}{13.432293\du}}
\pgfpathlineto{\pgfpoint{-0.091740\du}{13.431563\du}}
\pgfpathlineto{\pgfpoint{-0.093200\du}{13.430468\du}}
\pgfpathlineto{\pgfpoint{-0.095026\du}{13.429738\du}}
\pgfpathlineto{\pgfpoint{-0.096851\du}{13.429373\du}}
\pgfpathlineto{\pgfpoint{-0.098677\du}{13.429373\du}}
\pgfpathlineto{\pgfpoint{-0.100867\du}{13.429373\du}}
\pgfpathlineto{\pgfpoint{-0.102328\du}{13.429738\du}}
\pgfpathlineto{\pgfpoint{-0.104518\du}{13.430468\du}}
\pgfpathlineto{\pgfpoint{-0.106344\du}{13.431563\du}}
\pgfpathlineto{\pgfpoint{-0.106709\du}{13.432293\du}}
\pgfpathlineto{\pgfpoint{-0.107074\du}{13.432659\du}}
\pgfpathlineto{\pgfpoint{-0.107804\du}{13.434119\du}}
\pgfpathlineto{\pgfpoint{-0.108534\du}{13.435214\du}}
\pgfpathlineto{\pgfpoint{-0.108534\du}{13.435579\du}}
\pgfpathlineto{\pgfpoint{-0.108900\du}{13.437040\du}}
\pgfpathlineto{\pgfpoint{-0.108900\du}{13.438135\du}}
\pgfpathlineto{\pgfpoint{-0.108900\du}{13.439230\du}}
\pgfusepath{fill}
\pgfsetbuttcap
\pgfsetmiterjoin
\pgfsetdash{}{0pt}
\definecolor{dialinecolor}{rgb}{0.678431, 0.839216, 0.905882}
\pgfsetfillcolor{dialinecolor}
\pgfpathmoveto{\pgfpoint{1.581863\du}{12.851790\du}}
\pgfpathlineto{\pgfpoint{1.581863\du}{12.851790\du}}
\pgfpathlineto{\pgfpoint{1.538052\du}{12.851790\du}}
\pgfpathlineto{\pgfpoint{1.495336\du}{12.852520\du}}
\pgfpathlineto{\pgfpoint{1.451889\du}{12.853615\du}}
\pgfpathlineto{\pgfpoint{1.409903\du}{12.854710\du}}
\pgfpathlineto{\pgfpoint{1.367552\du}{12.856536\du}}
\pgfpathlineto{\pgfpoint{1.325931\du}{12.858726\du}}
\pgfpathlineto{\pgfpoint{1.283945\du}{12.861282\du}}
\pgfpathlineto{\pgfpoint{1.242689\du}{12.863473\du}}
\pgfpathlineto{\pgfpoint{1.201798\du}{12.866394\du}}
\pgfpathlineto{\pgfpoint{1.160907\du}{12.870044\du}}
\pgfpathlineto{\pgfpoint{1.121111\du}{12.873695\du}}
\pgfpathlineto{\pgfpoint{1.080951\du}{12.877712\du}}
\pgfpathlineto{\pgfpoint{1.041885\du}{12.882093\du}}
\pgfpathlineto{\pgfpoint{1.003185\du}{12.886839\du}}
\pgfpathlineto{\pgfpoint{0.964485\du}{12.891950\du}}
\pgfpathlineto{\pgfpoint{0.926515\du}{12.897427\du}}
\pgfpathlineto{\pgfpoint{0.888910\du}{12.902538\du}}
\pgfpathlineto{\pgfpoint{0.851670\du}{12.909110\du}}
\pgfpathlineto{\pgfpoint{0.815160\du}{12.914951\du}}
\pgfpathlineto{\pgfpoint{0.779016\du}{12.921523\du}}
\pgfpathlineto{\pgfpoint{0.743236\du}{12.928460\du}}
\pgfpathlineto{\pgfpoint{0.708552\du}{12.935397\du}}
\pgfpathlineto{\pgfpoint{0.674233\du}{12.943064\du}}
\pgfpathlineto{\pgfpoint{0.639549\du}{12.950731\du}}
\pgfpathlineto{\pgfpoint{0.607055\du}{12.958763\du}}
\pgfpathlineto{\pgfpoint{0.573831\du}{12.967160\du}}
\pgfpathlineto{\pgfpoint{0.542068\du}{12.975192\du}}
\pgfpathlineto{\pgfpoint{0.510305\du}{12.983955\du}}
\pgfpathlineto{\pgfpoint{0.479636\du}{12.993447\du}}
\pgfpathlineto{\pgfpoint{0.448603\du}{13.002210\du}}
\pgfpathlineto{\pgfpoint{0.433999\du}{13.006956\du}}
\pgfpathlineto{\pgfpoint{0.420126\du}{13.012067\du}}
\pgfpathlineto{\pgfpoint{0.404426\du}{13.016813\du}}
\pgfpathlineto{\pgfpoint{0.390553\du}{13.021560\du}}
\pgfpathlineto{\pgfpoint{0.375949\du}{13.026671\du}}
\pgfpathlineto{\pgfpoint{0.362075\du}{13.031417\du}}
\pgfpathlineto{\pgfpoint{0.347836\du}{13.036529\du}}
\pgfpathlineto{\pgfpoint{0.334693\du}{13.041275\du}}
\pgfpathlineto{\pgfpoint{0.320454\du}{13.046751\du}}
\pgfpathlineto{\pgfpoint{0.307311\du}{13.051863\du}}
\pgfpathlineto{\pgfpoint{0.294897\du}{13.056974\du}}
\pgfpathlineto{\pgfpoint{0.281024\du}{13.062816\du}}
\pgfpathlineto{\pgfpoint{0.268245\du}{13.068292\du}}
\pgfpathlineto{\pgfpoint{0.256197\du}{13.073403\du}}
\pgfpathlineto{\pgfpoint{0.243054\du}{13.079245\du}}
\pgfpathlineto{\pgfpoint{0.231005\du}{13.084721\du}}
\pgfpathlineto{\pgfpoint{0.218592\du}{13.090563\du}}
\pgfpathlineto{\pgfpoint{0.206909\du}{13.095674\du}}
\pgfpathlineto{\pgfpoint{0.195591\du}{13.101516\du}}
\pgfpathlineto{\pgfpoint{0.183908\du}{13.107357\du}}
\pgfpathlineto{\pgfpoint{0.172590\du}{13.113199\du}}
\pgfpathlineto{\pgfpoint{0.160907\du}{13.119040\du}}
\pgfpathlineto{\pgfpoint{0.149954\du}{13.124882\du}}
\pgfpathlineto{\pgfpoint{0.139731\du}{13.131454\du}}
\pgfpathlineto{\pgfpoint{0.129143\du}{13.137295\du}}
\pgfpathlineto{\pgfpoint{0.118921\du}{13.143137\du}}
\pgfpathlineto{\pgfpoint{0.108698\du}{13.149709\du}}
\pgfpathlineto{\pgfpoint{0.098840\du}{13.155550\du}}
\pgfpathlineto{\pgfpoint{0.089348\du}{13.161757\du}}
\pgfpathlineto{\pgfpoint{0.079855\du}{13.168329\du}}
\pgfpathlineto{\pgfpoint{0.070728\du}{13.174900\du}}
\pgfpathlineto{\pgfpoint{0.060870\du}{13.180742\du}}
\pgfpathlineto{\pgfpoint{0.052473\du}{13.186948\du}}
\pgfpathlineto{\pgfpoint{0.043711\du}{13.193520\du}}
\pgfpathlineto{\pgfpoint{0.035314\du}{13.199727\du}}
\pgfpathlineto{\pgfpoint{0.026551\du}{13.207029\du}}
\pgfpathlineto{\pgfpoint{0.018884\du}{13.213235\du}}
\pgfpathlineto{\pgfpoint{0.011947\du}{13.219807\du}}
\pgfpathlineto{\pgfpoint{0.003185\du}{13.226744\du}}
\pgfpathlineto{\pgfpoint{-0.003752\du}{13.233316\du}}
\pgfpathlineto{\pgfpoint{-0.010689\du}{13.240253\du}}
\pgfpathlineto{\pgfpoint{-0.017625\du}{13.247189\du}}
\pgfpathlineto{\pgfpoint{-0.024197\du}{13.253761\du}}
\pgfpathlineto{\pgfpoint{-0.031134\du}{13.260698\du}}
\pgfpathlineto{\pgfpoint{-0.036976\du}{13.267635\du}}
\pgfpathlineto{\pgfpoint{-0.042817\du}{13.274937\du}}
\pgfpathlineto{\pgfpoint{-0.048659\du}{13.281874\du}}
\pgfpathlineto{\pgfpoint{-0.054135\du}{13.288810\du}}
\pgfpathlineto{\pgfpoint{-0.058881\du}{13.295747\du}}
\pgfpathlineto{\pgfpoint{-0.064358\du}{13.303414\du}}
\pgfpathlineto{\pgfpoint{-0.068374\du}{13.310351\du}}
\pgfpathlineto{\pgfpoint{-0.073485\du}{13.317653\du}}
\pgfpathlineto{\pgfpoint{-0.077136\du}{13.324955\du}}
\pgfpathlineto{\pgfpoint{-0.081517\du}{13.332257\du}}
\pgfpathlineto{\pgfpoint{-0.085168\du}{13.339924\du}}
\pgfpathlineto{\pgfpoint{-0.088454\du}{13.347226\du}}
\pgfpathlineto{\pgfpoint{-0.092105\du}{13.354893\du}}
\pgfpathlineto{\pgfpoint{-0.095026\du}{13.362560\du}}
\pgfpathlineto{\pgfpoint{-0.097582\du}{13.369497\du}}
\pgfpathlineto{\pgfpoint{-0.099772\du}{13.377164\du}}
\pgfpathlineto{\pgfpoint{-0.101963\du}{13.384831\du}}
\pgfpathlineto{\pgfpoint{-0.103788\du}{13.392863\du}}
\pgfpathlineto{\pgfpoint{-0.105614\du}{13.400530\du}}
\pgfpathlineto{\pgfpoint{-0.106709\du}{13.408197\du}}
\pgfpathlineto{\pgfpoint{-0.107804\du}{13.415864\du}}
\pgfpathlineto{\pgfpoint{-0.108534\du}{13.423896\du}}
\pgfpathlineto{\pgfpoint{-0.108900\du}{13.431563\du}}
\pgfpathlineto{\pgfpoint{-0.108900\du}{13.439230\du}}
\pgfpathlineto{\pgfpoint{-0.088454\du}{13.439230\du}}
\pgfpathlineto{\pgfpoint{-0.088089\du}{13.432293\du}}
\pgfpathlineto{\pgfpoint{-0.088089\du}{13.425357\du}}
\pgfpathlineto{\pgfpoint{-0.087724\du}{13.418055\du}}
\pgfpathlineto{\pgfpoint{-0.086264\du}{13.411118\du}}
\pgfpathlineto{\pgfpoint{-0.085168\du}{13.404181\du}}
\pgfpathlineto{\pgfpoint{-0.084073\du}{13.397244\du}}
\pgfpathlineto{\pgfpoint{-0.081882\du}{13.389942\du}}
\pgfpathlineto{\pgfpoint{-0.080422\du}{13.383736\du}}
\pgfpathlineto{\pgfpoint{-0.078231\du}{13.376434\du}}
\pgfpathlineto{\pgfpoint{-0.076041\du}{13.369497\du}}
\pgfpathlineto{\pgfpoint{-0.072755\du}{13.362560\du}}
\pgfpathlineto{\pgfpoint{-0.070199\du}{13.355623\du}}
\pgfpathlineto{\pgfpoint{-0.067279\du}{13.348686\du}}
\pgfpathlineto{\pgfpoint{-0.063262\du}{13.342115\du}}
\pgfpathlineto{\pgfpoint{-0.059246\du}{13.335178\du}}
\pgfpathlineto{\pgfpoint{-0.055961\du}{13.328606\du}}
\pgfpathlineto{\pgfpoint{-0.051944\du}{13.321669\du}}
\pgfpathlineto{\pgfpoint{-0.047198\du}{13.314732\du}}
\pgfpathlineto{\pgfpoint{-0.042817\du}{13.308161\du}}
\pgfpathlineto{\pgfpoint{-0.037341\du}{13.301589\du}}
\pgfpathlineto{\pgfpoint{-0.032959\du}{13.294652\du}}
\pgfpathlineto{\pgfpoint{-0.026753\du}{13.288080\du}}
\pgfpathlineto{\pgfpoint{-0.021641\du}{13.281143\du}}
\pgfpathlineto{\pgfpoint{-0.015435\du}{13.274937\du}}
\pgfpathlineto{\pgfpoint{-0.009593\du}{13.268365\du}}
\pgfpathlineto{\pgfpoint{-0.003022\du}{13.261428\du}}
\pgfpathlineto{\pgfpoint{0.003185\du}{13.254856\du}}
\pgfpathlineto{\pgfpoint{0.010122\du}{13.248650\du}}
\pgfpathlineto{\pgfpoint{0.017424\du}{13.242078\du}}
\pgfpathlineto{\pgfpoint{0.024361\du}{13.235506\du}}
\pgfpathlineto{\pgfpoint{0.032028\du}{13.229300\du}}
\pgfpathlineto{\pgfpoint{0.040060\du}{13.222728\du}}
\pgfpathlineto{\pgfpoint{0.047362\du}{13.216156\du}}
\pgfpathlineto{\pgfpoint{0.055759\du}{13.209950\du}}
\pgfpathlineto{\pgfpoint{0.064521\du}{13.204108\du}}
\pgfpathlineto{\pgfpoint{0.073649\du}{13.197171\du}}
\pgfpathlineto{\pgfpoint{0.082411\du}{13.191695\du}}
\pgfpathlineto{\pgfpoint{0.090808\du}{13.185123\du}}
\pgfpathlineto{\pgfpoint{0.099936\du}{13.178916\du}}
\pgfpathlineto{\pgfpoint{0.109793\du}{13.173075\du}}
\pgfpathlineto{\pgfpoint{0.119286\du}{13.167233\du}}
\pgfpathlineto{\pgfpoint{0.129143\du}{13.160661\du}}
\pgfpathlineto{\pgfpoint{0.139731\du}{13.154820\du}}
\pgfpathlineto{\pgfpoint{0.149589\du}{13.148978\du}}
\pgfpathlineto{\pgfpoint{0.160177\du}{13.143137\du}}
\pgfpathlineto{\pgfpoint{0.171495\du}{13.137295\du}}
\pgfpathlineto{\pgfpoint{0.181352\du}{13.131454\du}}
\pgfpathlineto{\pgfpoint{0.193035\du}{13.125612\du}}
\pgfpathlineto{\pgfpoint{0.203988\du}{13.120501\du}}
\pgfpathlineto{\pgfpoint{0.215671\du}{13.114294\du}}
\pgfpathlineto{\pgfpoint{0.227355\du}{13.108453\du}}
\pgfpathlineto{\pgfpoint{0.240133\du}{13.103341\du}}
\pgfpathlineto{\pgfpoint{0.251451\du}{13.098230\du}}
\pgfpathlineto{\pgfpoint{0.263499\du}{13.092388\du}}
\pgfpathlineto{\pgfpoint{0.276643\du}{13.086912\du}}
\pgfpathlineto{\pgfpoint{0.289056\du}{13.081801\du}}
\pgfpathlineto{\pgfpoint{0.301834\du}{13.076324\du}}
\pgfpathlineto{\pgfpoint{0.315343\du}{13.071213\du}}
\pgfpathlineto{\pgfpoint{0.328121\du}{13.065736\du}}
\pgfpathlineto{\pgfpoint{0.341630\du}{13.060260\du}}
\pgfpathlineto{\pgfpoint{0.355138\du}{13.055879\du}}
\pgfpathlineto{\pgfpoint{0.369012\du}{13.050767\du}}
\pgfpathlineto{\pgfpoint{0.382886\du}{13.046021\du}}
\pgfpathlineto{\pgfpoint{0.397125\du}{13.040910\du}}
\pgfpathlineto{\pgfpoint{0.411363\du}{13.036163\du}}
\pgfpathlineto{\pgfpoint{0.425602\du}{13.031417\du}}
\pgfpathlineto{\pgfpoint{0.439841\du}{13.026671\du}}
\pgfpathlineto{\pgfpoint{0.455175\du}{13.021925\du}}
\pgfpathlineto{\pgfpoint{0.484748\du}{13.012797\du}}
\pgfpathlineto{\pgfpoint{0.516146\du}{13.004035\du}}
\pgfpathlineto{\pgfpoint{0.547179\du}{12.995273\du}}
\pgfpathlineto{\pgfpoint{0.579673\du}{12.986875\du}}
\pgfpathlineto{\pgfpoint{0.611071\du}{12.978843\du}}
\pgfpathlineto{\pgfpoint{0.644295\du}{12.970446\du}}
\pgfpathlineto{\pgfpoint{0.678249\du}{12.962779\du}}
\pgfpathlineto{\pgfpoint{0.712568\du}{12.955842\du}}
\pgfpathlineto{\pgfpoint{0.747252\du}{12.948905\du}}
\pgfpathlineto{\pgfpoint{0.783397\du}{12.941603\du}}
\pgfpathlineto{\pgfpoint{0.818446\du}{12.935397\du}}
\pgfpathlineto{\pgfpoint{0.855686\du}{12.928460\du}}
\pgfpathlineto{\pgfpoint{0.892196\du}{12.922984\du}}
\pgfpathlineto{\pgfpoint{0.929436\du}{12.917142\du}}
\pgfpathlineto{\pgfpoint{0.967041\du}{12.912031\du}}
\pgfpathlineto{\pgfpoint{1.005376\du}{12.907284\du}}
\pgfpathlineto{\pgfpoint{1.044441\du}{12.902538\du}}
\pgfpathlineto{\pgfpoint{1.083871\du}{12.898157\du}}
\pgfpathlineto{\pgfpoint{1.123302\du}{12.894506\du}}
\pgfpathlineto{\pgfpoint{1.163097\du}{12.890490\du}}
\pgfpathlineto{\pgfpoint{1.203258\du}{12.887569\du}}
\pgfpathlineto{\pgfpoint{1.244149\du}{12.883918\du}}
\pgfpathlineto{\pgfpoint{1.285405\du}{12.880997\du}}
\pgfpathlineto{\pgfpoint{1.326296\du}{12.878807\du}}
\pgfpathlineto{\pgfpoint{1.368282\du}{12.876981\du}}
\pgfpathlineto{\pgfpoint{1.410633\du}{12.875156\du}}
\pgfpathlineto{\pgfpoint{1.453349\du}{12.873695\du}}
\pgfpathlineto{\pgfpoint{1.495701\du}{12.873330\du}}
\pgfpathlineto{\pgfpoint{1.539147\du}{12.872965\du}}
\pgfpathlineto{\pgfpoint{1.581863\du}{12.872235\du}}
\pgfpathlineto{\pgfpoint{1.581863\du}{12.872235\du}}
\pgfpathlineto{\pgfpoint{1.581863\du}{12.872235\du}}
\pgfpathlineto{\pgfpoint{1.582959\du}{12.872235\du}}
\pgfpathlineto{\pgfpoint{1.584054\du}{12.872235\du}}
\pgfpathlineto{\pgfpoint{1.585514\du}{12.871505\du}}
\pgfpathlineto{\pgfpoint{1.586610\du}{12.871505\du}}
\pgfpathlineto{\pgfpoint{1.586975\du}{12.871140\du}}
\pgfpathlineto{\pgfpoint{1.588070\du}{12.870410\du}}
\pgfpathlineto{\pgfpoint{1.588800\du}{12.870044\du}}
\pgfpathlineto{\pgfpoint{1.589896\du}{12.869314\du}}
\pgfpathlineto{\pgfpoint{1.590991\du}{12.867489\du}}
\pgfpathlineto{\pgfpoint{1.591721\du}{12.865663\du}}
\pgfpathlineto{\pgfpoint{1.591721\du}{12.864203\du}}
\pgfpathlineto{\pgfpoint{1.592451\du}{12.862377\du}}
\pgfpathlineto{\pgfpoint{1.591721\du}{12.860552\du}}
\pgfpathlineto{\pgfpoint{1.591721\du}{12.858361\du}}
\pgfpathlineto{\pgfpoint{1.590991\du}{12.856536\du}}
\pgfpathlineto{\pgfpoint{1.589896\du}{12.854710\du}}
\pgfpathlineto{\pgfpoint{1.588800\du}{12.853980\du}}
\pgfpathlineto{\pgfpoint{1.588070\du}{12.853615\du}}
\pgfpathlineto{\pgfpoint{1.586975\du}{12.852885\du}}
\pgfpathlineto{\pgfpoint{1.586610\du}{12.852520\du}}
\pgfpathlineto{\pgfpoint{1.585514\du}{12.852520\du}}
\pgfpathlineto{\pgfpoint{1.584054\du}{12.851790\du}}
\pgfpathlineto{\pgfpoint{1.582959\du}{12.851790\du}}
\pgfpathlineto{\pgfpoint{1.581863\du}{12.851790\du}}
\pgfusepath{fill}
\pgfsetbuttcap
\pgfsetmiterjoin
\pgfsetdash{}{0pt}
\definecolor{dialinecolor}{rgb}{0.678431, 0.839216, 0.905882}
\pgfsetfillcolor{dialinecolor}
\pgfpathmoveto{\pgfpoint{3.272261\du}{13.439230\du}}
\pgfpathlineto{\pgfpoint{3.272261\du}{13.431563\du}}
\pgfpathlineto{\pgfpoint{3.271531\du}{13.423896\du}}
\pgfpathlineto{\pgfpoint{3.270801\du}{13.415864\du}}
\pgfpathlineto{\pgfpoint{3.270071\du}{13.408197\du}}
\pgfpathlineto{\pgfpoint{3.268976\du}{13.400530\du}}
\pgfpathlineto{\pgfpoint{3.266785\du}{13.392863\du}}
\pgfpathlineto{\pgfpoint{3.265690\du}{13.384831\du}}
\pgfpathlineto{\pgfpoint{3.262769\du}{13.377164\du}}
\pgfpathlineto{\pgfpoint{3.260943\du}{13.369497\du}}
\pgfpathlineto{\pgfpoint{3.258023\du}{13.362560\du}}
\pgfpathlineto{\pgfpoint{3.255467\du}{13.354893\du}}
\pgfpathlineto{\pgfpoint{3.252181\du}{13.347226\du}}
\pgfpathlineto{\pgfpoint{3.248165\du}{13.339924\du}}
\pgfpathlineto{\pgfpoint{3.244879\du}{13.332257\du}}
\pgfpathlineto{\pgfpoint{3.240863\du}{13.324955\du}}
\pgfpathlineto{\pgfpoint{3.236847\du}{13.317653\du}}
\pgfpathlineto{\pgfpoint{3.232466\du}{13.310351\du}}
\pgfpathlineto{\pgfpoint{3.227720\du}{13.303414\du}}
\pgfpathlineto{\pgfpoint{3.222243\du}{13.295747\du}}
\pgfpathlineto{\pgfpoint{3.217497\du}{13.288810\du}}
\pgfpathlineto{\pgfpoint{3.211655\du}{13.281874\du}}
\pgfpathlineto{\pgfpoint{3.206179\du}{13.274937\du}}
\pgfpathlineto{\pgfpoint{3.200337\du}{13.267635\du}}
\pgfpathlineto{\pgfpoint{3.194131\du}{13.260698\du}}
\pgfpathlineto{\pgfpoint{3.187559\du}{13.253761\du}}
\pgfpathlineto{\pgfpoint{3.180622\du}{13.247189\du}}
\pgfpathlineto{\pgfpoint{3.174050\du}{13.240253\du}}
\pgfpathlineto{\pgfpoint{3.166748\du}{13.233316\du}}
\pgfpathlineto{\pgfpoint{3.160177\du}{13.226744\du}}
\pgfpathlineto{\pgfpoint{3.152145\du}{13.219807\du}}
\pgfpathlineto{\pgfpoint{3.144112\du}{13.213235\du}}
\pgfpathlineto{\pgfpoint{3.136811\du}{13.207029\du}}
\pgfpathlineto{\pgfpoint{3.128048\du}{13.199727\du}}
\pgfpathlineto{\pgfpoint{3.120016\du}{13.193520\du}}
\pgfpathlineto{\pgfpoint{3.110889\du}{13.186948\du}}
\pgfpathlineto{\pgfpoint{3.102126\du}{13.180742\du}}
\pgfpathlineto{\pgfpoint{3.092634\du}{13.174900\du}}
\pgfpathlineto{\pgfpoint{3.083506\du}{13.168329\du}}
\pgfpathlineto{\pgfpoint{3.074379\du}{13.161757\du}}
\pgfpathlineto{\pgfpoint{3.064521\du}{13.155550\du}}
\pgfpathlineto{\pgfpoint{3.054664\du}{13.149709\du}}
\pgfpathlineto{\pgfpoint{3.044441\du}{13.143137\du}}
\pgfpathlineto{\pgfpoint{3.033853\du}{13.137295\du}}
\pgfpathlineto{\pgfpoint{3.023996\du}{13.131454\du}}
\pgfpathlineto{\pgfpoint{3.013408\du}{13.124882\du}}
\pgfpathlineto{\pgfpoint{3.002090\du}{13.119040\du}}
\pgfpathlineto{\pgfpoint{2.990772\du}{13.113199\du}}
\pgfpathlineto{\pgfpoint{2.979089\du}{13.107357\du}}
\pgfpathlineto{\pgfpoint{2.968136\du}{13.101516\du}}
\pgfpathlineto{\pgfpoint{2.956088\du}{13.095674\du}}
\pgfpathlineto{\pgfpoint{2.944770\du}{13.090563\du}}
\pgfpathlineto{\pgfpoint{2.932356\du}{13.084721\du}}
\pgfpathlineto{\pgfpoint{2.919943\du}{13.079245\du}}
\pgfpathlineto{\pgfpoint{2.907165\du}{13.073403\du}}
\pgfpathlineto{\pgfpoint{2.895116\du}{13.068292\du}}
\pgfpathlineto{\pgfpoint{2.882338\du}{13.062816\du}}
\pgfpathlineto{\pgfpoint{2.868830\du}{13.056974\du}}
\pgfpathlineto{\pgfpoint{2.855686\du}{13.051863\du}}
\pgfpathlineto{\pgfpoint{2.842543\du}{13.046751\du}}
\pgfpathlineto{\pgfpoint{2.828304\du}{13.041275\du}}
\pgfpathlineto{\pgfpoint{2.815160\du}{13.036529\du}}
\pgfpathlineto{\pgfpoint{2.800922\du}{13.031417\du}}
\pgfpathlineto{\pgfpoint{2.787048\du}{13.026671\du}}
\pgfpathlineto{\pgfpoint{2.772809\du}{13.021560\du}}
\pgfpathlineto{\pgfpoint{2.758935\du}{13.016813\du}}
\pgfpathlineto{\pgfpoint{2.743966\du}{13.012067\du}}
\pgfpathlineto{\pgfpoint{2.728997\du}{13.006956\du}}
\pgfpathlineto{\pgfpoint{2.714759\du}{13.002210\du}}
\pgfpathlineto{\pgfpoint{2.684091\du}{12.993447\du}}
\pgfpathlineto{\pgfpoint{2.653422\du}{12.983955\du}}
\pgfpathlineto{\pgfpoint{2.622024\du}{12.975192\du}}
\pgfpathlineto{\pgfpoint{2.590261\du}{12.967160\du}}
\pgfpathlineto{\pgfpoint{2.557402\du}{12.958763\du}}
\pgfpathlineto{\pgfpoint{2.523813\du}{12.950731\du}}
\pgfpathlineto{\pgfpoint{2.489859\du}{12.943064\du}}
\pgfpathlineto{\pgfpoint{2.455175\du}{12.935397\du}}
\pgfpathlineto{\pgfpoint{2.420491\du}{12.928460\du}}
\pgfpathlineto{\pgfpoint{2.385076\du}{12.921523\du}}
\pgfpathlineto{\pgfpoint{2.348567\du}{12.914951\du}}
\pgfpathlineto{\pgfpoint{2.312057\du}{12.909110\du}}
\pgfpathlineto{\pgfpoint{2.275182\du}{12.902538\du}}
\pgfpathlineto{\pgfpoint{2.237942\du}{12.897427\du}}
\pgfpathlineto{\pgfpoint{2.198877\du}{12.891950\du}}
\pgfpathlineto{\pgfpoint{2.160907\du}{12.886839\du}}
\pgfpathlineto{\pgfpoint{2.121476\du}{12.882093\du}}
\pgfpathlineto{\pgfpoint{2.082776\du}{12.877712\du}}
\pgfpathlineto{\pgfpoint{2.042616\du}{12.873695\du}}
\pgfpathlineto{\pgfpoint{2.002820\du}{12.870044\du}}
\pgfpathlineto{\pgfpoint{1.961929\du}{12.866394\du}}
\pgfpathlineto{\pgfpoint{1.921403\du}{12.863473\du}}
\pgfpathlineto{\pgfpoint{1.879782\du}{12.861282\du}}
\pgfpathlineto{\pgfpoint{1.838161\du}{12.858726\du}}
\pgfpathlineto{\pgfpoint{1.796540\du}{12.856536\du}}
\pgfpathlineto{\pgfpoint{1.753824\du}{12.854710\du}}
\pgfpathlineto{\pgfpoint{1.711838\du}{12.853615\du}}
\pgfpathlineto{\pgfpoint{1.668756\du}{12.852520\du}}
\pgfpathlineto{\pgfpoint{1.625310\du}{12.851790\du}}
\pgfpathlineto{\pgfpoint{1.581863\du}{12.851790\du}}
\pgfpathlineto{\pgfpoint{1.581863\du}{12.872235\du}}
\pgfpathlineto{\pgfpoint{1.624945\du}{12.872965\du}}
\pgfpathlineto{\pgfpoint{1.668391\du}{12.873330\du}}
\pgfpathlineto{\pgfpoint{1.710378\du}{12.873695\du}}
\pgfpathlineto{\pgfpoint{1.753094\du}{12.875156\du}}
\pgfpathlineto{\pgfpoint{1.795810\du}{12.876981\du}}
\pgfpathlineto{\pgfpoint{1.837431\du}{12.878807\du}}
\pgfpathlineto{\pgfpoint{1.878687\du}{12.880997\du}}
\pgfpathlineto{\pgfpoint{1.919578\du}{12.883918\du}}
\pgfpathlineto{\pgfpoint{1.960834\du}{12.887569\du}}
\pgfpathlineto{\pgfpoint{2.000995\du}{12.890490\du}}
\pgfpathlineto{\pgfpoint{2.041155\du}{12.894506\du}}
\pgfpathlineto{\pgfpoint{2.080221\du}{12.898157\du}}
\pgfpathlineto{\pgfpoint{2.119651\du}{12.902538\du}}
\pgfpathlineto{\pgfpoint{2.158351\du}{12.907284\du}}
\pgfpathlineto{\pgfpoint{2.197051\du}{12.912031\du}}
\pgfpathlineto{\pgfpoint{2.234656\du}{12.917142\du}}
\pgfpathlineto{\pgfpoint{2.271896\du}{12.922984\du}}
\pgfpathlineto{\pgfpoint{2.308406\du}{12.928460\du}}
\pgfpathlineto{\pgfpoint{2.345281\du}{12.935397\du}}
\pgfpathlineto{\pgfpoint{2.380695\du}{12.941603\du}}
\pgfpathlineto{\pgfpoint{2.416475\du}{12.948905\du}}
\pgfpathlineto{\pgfpoint{2.451159\du}{12.955842\du}}
\pgfpathlineto{\pgfpoint{2.485113\du}{12.962779\du}}
\pgfpathlineto{\pgfpoint{2.519432\du}{12.970446\du}}
\pgfpathlineto{\pgfpoint{2.552656\du}{12.978843\du}}
\pgfpathlineto{\pgfpoint{2.584784\du}{12.986875\du}}
\pgfpathlineto{\pgfpoint{2.616913\du}{12.995273\du}}
\pgfpathlineto{\pgfpoint{2.648311\du}{13.004035\du}}
\pgfpathlineto{\pgfpoint{2.678979\du}{13.012797\du}}
\pgfpathlineto{\pgfpoint{2.708917\du}{13.021925\du}}
\pgfpathlineto{\pgfpoint{2.723156\du}{13.026671\du}}
\pgfpathlineto{\pgfpoint{2.737760\du}{13.031417\du}}
\pgfpathlineto{\pgfpoint{2.751633\du}{13.036163\du}}
\pgfpathlineto{\pgfpoint{2.766237\du}{13.040910\du}}
\pgfpathlineto{\pgfpoint{2.780476\du}{13.046021\du}}
\pgfpathlineto{\pgfpoint{2.794350\du}{13.050767\du}}
\pgfpathlineto{\pgfpoint{2.808589\du}{13.055879\du}}
\pgfpathlineto{\pgfpoint{2.821732\du}{13.060260\du}}
\pgfpathlineto{\pgfpoint{2.835241\du}{13.065736\du}}
\pgfpathlineto{\pgfpoint{2.848384\du}{13.071213\du}}
\pgfpathlineto{\pgfpoint{2.861162\du}{13.076324\du}}
\pgfpathlineto{\pgfpoint{2.873941\du}{13.081801\du}}
\pgfpathlineto{\pgfpoint{2.887084\du}{13.086912\du}}
\pgfpathlineto{\pgfpoint{2.899133\du}{13.092388\du}}
\pgfpathlineto{\pgfpoint{2.911546\du}{13.098230\du}}
\pgfpathlineto{\pgfpoint{2.923594\du}{13.103341\du}}
\pgfpathlineto{\pgfpoint{2.936007\du}{13.108453\du}}
\pgfpathlineto{\pgfpoint{2.947325\du}{13.114294\du}}
\pgfpathlineto{\pgfpoint{2.959374\du}{13.120501\du}}
\pgfpathlineto{\pgfpoint{2.969961\du}{13.125612\du}}
\pgfpathlineto{\pgfpoint{2.982010\du}{13.131454\du}}
\pgfpathlineto{\pgfpoint{2.992232\du}{13.137295\du}}
\pgfpathlineto{\pgfpoint{3.003185\du}{13.143137\du}}
\pgfpathlineto{\pgfpoint{3.014138\du}{13.148978\du}}
\pgfpathlineto{\pgfpoint{3.023996\du}{13.154820\du}}
\pgfpathlineto{\pgfpoint{3.033853\du}{13.160661\du}}
\pgfpathlineto{\pgfpoint{3.044076\du}{13.167233\du}}
\pgfpathlineto{\pgfpoint{3.053203\du}{13.173075\du}}
\pgfpathlineto{\pgfpoint{3.063426\du}{13.178916\du}}
\pgfpathlineto{\pgfpoint{3.072919\du}{13.185123\du}}
\pgfpathlineto{\pgfpoint{3.081681\du}{13.191695\du}}
\pgfpathlineto{\pgfpoint{3.089713\du}{13.197171\du}}
\pgfpathlineto{\pgfpoint{3.098475\du}{13.204108\du}}
\pgfpathlineto{\pgfpoint{3.106873\du}{13.209950\du}}
\pgfpathlineto{\pgfpoint{3.115635\du}{13.216156\du}}
\pgfpathlineto{\pgfpoint{3.123302\du}{13.222728\du}}
\pgfpathlineto{\pgfpoint{3.131334\du}{13.229300\du}}
\pgfpathlineto{\pgfpoint{3.138636\du}{13.235506\du}}
\pgfpathlineto{\pgfpoint{3.146303\du}{13.242078\du}}
\pgfpathlineto{\pgfpoint{3.152875\du}{13.248650\du}}
\pgfpathlineto{\pgfpoint{3.160177\du}{13.254856\du}}
\pgfpathlineto{\pgfpoint{3.166018\du}{13.261428\du}}
\pgfpathlineto{\pgfpoint{3.172955\du}{13.268365\du}}
\pgfpathlineto{\pgfpoint{3.179527\du}{13.274937\du}}
\pgfpathlineto{\pgfpoint{3.184638\du}{13.281143\du}}
\pgfpathlineto{\pgfpoint{3.190115\du}{13.288080\du}}
\pgfpathlineto{\pgfpoint{3.196321\du}{13.294652\du}}
\pgfpathlineto{\pgfpoint{3.201068\du}{13.301589\du}}
\pgfpathlineto{\pgfpoint{3.206179\du}{13.308161\du}}
\pgfpathlineto{\pgfpoint{3.210925\du}{13.314732\du}}
\pgfpathlineto{\pgfpoint{3.215306\du}{13.321669\du}}
\pgfpathlineto{\pgfpoint{3.219687\du}{13.328606\du}}
\pgfpathlineto{\pgfpoint{3.222608\du}{13.335178\du}}
\pgfpathlineto{\pgfpoint{3.226624\du}{13.342115\du}}
\pgfpathlineto{\pgfpoint{3.229545\du}{13.348686\du}}
\pgfpathlineto{\pgfpoint{3.233561\du}{13.355623\du}}
\pgfpathlineto{\pgfpoint{3.236117\du}{13.362560\du}}
\pgfpathlineto{\pgfpoint{3.239038\du}{13.369497\du}}
\pgfpathlineto{\pgfpoint{3.241593\du}{13.376434\du}}
\pgfpathlineto{\pgfpoint{3.243784\du}{13.383736\du}}
\pgfpathlineto{\pgfpoint{3.245244\du}{13.389942\du}}
\pgfpathlineto{\pgfpoint{3.247435\du}{13.397244\du}}
\pgfpathlineto{\pgfpoint{3.248165\du}{13.404181\du}}
\pgfpathlineto{\pgfpoint{3.249260\du}{13.411118\du}}
\pgfpathlineto{\pgfpoint{3.251086\du}{13.418055\du}}
\pgfpathlineto{\pgfpoint{3.251451\du}{13.425357\du}}
\pgfpathlineto{\pgfpoint{3.251451\du}{13.432293\du}}
\pgfpathlineto{\pgfpoint{3.252181\du}{13.439230\du}}
\pgfpathlineto{\pgfpoint{3.272261\du}{13.439230\du}}
\pgfusepath{fill}
\pgfsetbuttcap
\pgfsetmiterjoin
\pgfsetdash{}{0pt}
\definecolor{dialinecolor}{rgb}{0.074510, 0.082353, 0.086275}
\pgfsetfillcolor{dialinecolor}
\pgfpathmoveto{\pgfpoint{1.624945\du}{13.311812\du}}
\pgfpathlineto{\pgfpoint{1.872846\du}{13.394323\du}}
\pgfpathlineto{\pgfpoint{2.458096\du}{13.159931\du}}
\pgfpathlineto{\pgfpoint{2.730823\du}{13.227474\du}}
\pgfpathlineto{\pgfpoint{2.586975\du}{13.019004\du}}
\pgfpathlineto{\pgfpoint{1.882703\du}{13.019004\du}}
\pgfpathlineto{\pgfpoint{2.176971\du}{13.091658\du}}
\pgfpathlineto{\pgfpoint{1.624945\du}{13.311812\du}}
\pgfusepath{fill}
\pgfsetbuttcap
\pgfsetmiterjoin
\pgfsetdash{}{0pt}
\definecolor{dialinecolor}{rgb}{0.074510, 0.082353, 0.086275}
\pgfsetfillcolor{dialinecolor}
\pgfpathmoveto{\pgfpoint{1.523083\du}{13.549855\du}}
\pgfpathlineto{\pgfpoint{1.275182\du}{13.467708\du}}
\pgfpathlineto{\pgfpoint{0.689932\du}{13.701370\du}}
\pgfpathlineto{\pgfpoint{0.416840\du}{13.634557\du}}
\pgfpathlineto{\pgfpoint{0.560688\du}{13.842297\du}}
\pgfpathlineto{\pgfpoint{1.266055\du}{13.842297\du}}
\pgfpathlineto{\pgfpoint{0.971057\du}{13.770373\du}}
\pgfpathlineto{\pgfpoint{1.523083\du}{13.549855\du}}
\pgfusepath{fill}
\pgfsetbuttcap
\pgfsetmiterjoin
\pgfsetdash{}{0pt}
\definecolor{dialinecolor}{rgb}{0.074510, 0.082353, 0.086275}
\pgfsetfillcolor{dialinecolor}
\pgfpathmoveto{\pgfpoint{0.477081\du}{13.090928\du}}
\pgfpathlineto{\pgfpoint{0.724616\du}{13.009146\du}}
\pgfpathlineto{\pgfpoint{1.309866\du}{13.242443\du}}
\pgfpathlineto{\pgfpoint{1.582959\du}{13.175996\du}}
\pgfpathlineto{\pgfpoint{1.439111\du}{13.383736\du}}
\pgfpathlineto{\pgfpoint{0.734109\du}{13.383736\du}}
\pgfpathlineto{\pgfpoint{1.029107\du}{13.311812\du}}
\pgfpathlineto{\pgfpoint{0.477081\du}{13.090928\du}}
\pgfusepath{fill}
\pgfsetbuttcap
\pgfsetmiterjoin
\pgfsetdash{}{0pt}
\definecolor{dialinecolor}{rgb}{0.074510, 0.082353, 0.086275}
\pgfsetfillcolor{dialinecolor}
\pgfpathmoveto{\pgfpoint{2.695409\du}{13.786072\du}}
\pgfpathlineto{\pgfpoint{2.447873\du}{13.868219\du}}
\pgfpathlineto{\pgfpoint{1.862623\du}{13.634557\du}}
\pgfpathlineto{\pgfpoint{1.588800\du}{13.701370\du}}
\pgfpathlineto{\pgfpoint{1.733379\du}{13.493630\du}}
\pgfpathlineto{\pgfpoint{2.438746\du}{13.493630\du}}
\pgfpathlineto{\pgfpoint{2.143382\du}{13.565554\du}}
\pgfpathlineto{\pgfpoint{2.695409\du}{13.786072\du}}
\pgfusepath{fill}
\pgfsetbuttcap
\pgfsetmiterjoin
\pgfsetdash{}{0pt}
\definecolor{dialinecolor}{rgb}{1.000000, 1.000000, 1.000000}
\pgfsetfillcolor{dialinecolor}
\pgfpathmoveto{\pgfpoint{1.645755\du}{13.332257\du}}
\pgfpathlineto{\pgfpoint{1.893291\du}{13.414769\du}}
\pgfpathlineto{\pgfpoint{2.478541\du}{13.180742\du}}
\pgfpathlineto{\pgfpoint{2.750903\du}{13.247920\du}}
\pgfpathlineto{\pgfpoint{2.608150\du}{13.039449\du}}
\pgfpathlineto{\pgfpoint{1.902784\du}{13.039449\du}}
\pgfpathlineto{\pgfpoint{2.197782\du}{13.112104\du}}
\pgfpathlineto{\pgfpoint{1.645755\du}{13.332257\du}}
\pgfusepath{fill}
\pgfsetbuttcap
\pgfsetmiterjoin
\pgfsetdash{}{0pt}
\definecolor{dialinecolor}{rgb}{1.000000, 1.000000, 1.000000}
\pgfsetfillcolor{dialinecolor}
\pgfpathmoveto{\pgfpoint{1.543893\du}{13.571030\du}}
\pgfpathlineto{\pgfpoint{1.295263\du}{13.488518\du}}
\pgfpathlineto{\pgfpoint{0.710378\du}{13.722545\du}}
\pgfpathlineto{\pgfpoint{0.436920\du}{13.655002\du}}
\pgfpathlineto{\pgfpoint{0.581863\du}{13.863473\du}}
\pgfpathlineto{\pgfpoint{1.286500\du}{13.863473\du}}
\pgfpathlineto{\pgfpoint{0.991867\du}{13.790818\du}}
\pgfpathlineto{\pgfpoint{1.543893\du}{13.571030\du}}
\pgfusepath{fill}
\pgfsetbuttcap
\pgfsetmiterjoin
\pgfsetdash{}{0pt}
\definecolor{dialinecolor}{rgb}{1.000000, 1.000000, 1.000000}
\pgfsetfillcolor{dialinecolor}
\pgfpathmoveto{\pgfpoint{0.497526\du}{13.111373\du}}
\pgfpathlineto{\pgfpoint{0.745062\du}{13.029592\du}}
\pgfpathlineto{\pgfpoint{1.330677\du}{13.263619\du}}
\pgfpathlineto{\pgfpoint{1.603769\du}{13.196806\du}}
\pgfpathlineto{\pgfpoint{1.458826\du}{13.404181\du}}
\pgfpathlineto{\pgfpoint{0.754554\du}{13.404181\du}}
\pgfpathlineto{\pgfpoint{1.049187\du}{13.332257\du}}
\pgfpathlineto{\pgfpoint{0.497526\du}{13.111373\du}}
\pgfusepath{fill}
\pgfsetbuttcap
\pgfsetmiterjoin
\pgfsetdash{}{0pt}
\definecolor{dialinecolor}{rgb}{1.000000, 1.000000, 1.000000}
\pgfsetfillcolor{dialinecolor}
\pgfpathmoveto{\pgfpoint{2.715489\du}{13.806518\du}}
\pgfpathlineto{\pgfpoint{2.467953\du}{13.888664\du}}
\pgfpathlineto{\pgfpoint{1.883068\du}{13.655002\du}}
\pgfpathlineto{\pgfpoint{1.609611\du}{13.721815\du}}
\pgfpathlineto{\pgfpoint{1.753824\du}{13.514075\du}}
\pgfpathlineto{\pgfpoint{2.458826\du}{13.514075\du}}
\pgfpathlineto{\pgfpoint{2.164558\du}{13.585999\du}}
\pgfpathlineto{\pgfpoint{2.715489\du}{13.806518\du}}
\pgfusepath{fill}
\pgfsetbuttcap
\pgfsetmiterjoin
\pgfsetdash{}{0pt}
\definecolor{dialinecolor}{rgb}{0.678431, 0.839216, 0.905882}
\pgfsetfillcolor{dialinecolor}
\pgfpathmoveto{\pgfpoint{-0.088454\du}{13.449818\du}}
\pgfpathlineto{\pgfpoint{-0.088454\du}{13.439230\du}}
\pgfpathlineto{\pgfpoint{-0.108900\du}{13.439230\du}}
\pgfpathlineto{\pgfpoint{-0.108900\du}{13.449818\du}}
\pgfpathlineto{\pgfpoint{-0.088454\du}{13.449818\du}}
\pgfusepath{fill}
\pgfsetbuttcap
\pgfsetmiterjoin
\pgfsetdash{}{0pt}
\definecolor{dialinecolor}{rgb}{0.678431, 0.839216, 0.905882}
\pgfsetfillcolor{dialinecolor}
\pgfpathmoveto{\pgfpoint{-0.088454\du}{14.279318\du}}
\pgfpathlineto{\pgfpoint{-0.088454\du}{13.449818\du}}
\pgfpathlineto{\pgfpoint{-0.108900\du}{13.449818\du}}
\pgfpathlineto{\pgfpoint{-0.108900\du}{14.279318\du}}
\pgfpathlineto{\pgfpoint{-0.088454\du}{14.279318\du}}
\pgfusepath{fill}
\pgfsetbuttcap
\pgfsetmiterjoin
\pgfsetdash{}{0pt}
\definecolor{dialinecolor}{rgb}{0.678431, 0.839216, 0.905882}
\pgfsetfillcolor{dialinecolor}
\pgfpathmoveto{\pgfpoint{-0.108900\du}{14.279318\du}}
\pgfpathlineto{\pgfpoint{-0.108900\du}{14.289906\du}}
\pgfpathlineto{\pgfpoint{-0.088454\du}{14.289906\du}}
\pgfpathlineto{\pgfpoint{-0.088454\du}{14.279318\du}}
\pgfpathlineto{\pgfpoint{-0.108900\du}{14.279318\du}}
\pgfusepath{fill}
\pgfsetbuttcap
\pgfsetmiterjoin
\pgfsetdash{}{0pt}
\definecolor{dialinecolor}{rgb}{0.678431, 0.839216, 0.905882}
\pgfsetfillcolor{dialinecolor}
\pgfpathmoveto{\pgfpoint{3.272261\du}{13.449818\du}}
\pgfpathlineto{\pgfpoint{3.272261\du}{13.439230\du}}
\pgfpathlineto{\pgfpoint{3.252181\du}{13.439230\du}}
\pgfpathlineto{\pgfpoint{3.252181\du}{13.449818\du}}
\pgfpathlineto{\pgfpoint{3.272261\du}{13.449818\du}}
\pgfusepath{fill}
\pgfsetbuttcap
\pgfsetmiterjoin
\pgfsetdash{}{0pt}
\definecolor{dialinecolor}{rgb}{0.678431, 0.839216, 0.905882}
\pgfsetfillcolor{dialinecolor}
\pgfpathmoveto{\pgfpoint{3.272261\du}{14.279318\du}}
\pgfpathlineto{\pgfpoint{3.272261\du}{13.449818\du}}
\pgfpathlineto{\pgfpoint{3.252181\du}{13.449818\du}}
\pgfpathlineto{\pgfpoint{3.252181\du}{14.279318\du}}
\pgfpathlineto{\pgfpoint{3.272261\du}{14.279318\du}}
\pgfusepath{fill}
\pgfsetbuttcap
\pgfsetmiterjoin
\pgfsetdash{}{0pt}
\definecolor{dialinecolor}{rgb}{0.678431, 0.839216, 0.905882}
\pgfsetfillcolor{dialinecolor}
\pgfpathmoveto{\pgfpoint{3.252181\du}{14.279318\du}}
\pgfpathlineto{\pgfpoint{3.252181\du}{14.289906\du}}
\pgfpathlineto{\pgfpoint{3.272261\du}{14.289906\du}}
\pgfpathlineto{\pgfpoint{3.272261\du}{14.279318\du}}
\pgfpathlineto{\pgfpoint{3.252181\du}{14.279318\du}}
\pgfusepath{fill}
\pgfsetbuttcap
\pgfsetmiterjoin
\pgfsetdash{}{0pt}
\definecolor{dialinecolor}{rgb}{0.027451, 0.372549, 0.529412}
\pgfsetfillcolor{dialinecolor}
\pgfpathmoveto{\pgfpoint{2.198512\du}{14.428643\du}}
\pgfpathlineto{\pgfpoint{2.198147\du}{14.446167\du}}
\pgfpathlineto{\pgfpoint{2.195591\du}{14.463327\du}}
\pgfpathlineto{\pgfpoint{2.191940\du}{14.479391\du}}
\pgfpathlineto{\pgfpoint{2.186099\du}{14.496551\du}}
\pgfpathlineto{\pgfpoint{2.179527\du}{14.512250\du}}
\pgfpathlineto{\pgfpoint{2.171130\du}{14.528314\du}}
\pgfpathlineto{\pgfpoint{2.161637\du}{14.544013\du}}
\pgfpathlineto{\pgfpoint{2.149954\du}{14.558982\du}}
\pgfpathlineto{\pgfpoint{2.138271\du}{14.573951\du}}
\pgfpathlineto{\pgfpoint{2.124762\du}{14.588555\du}}
\pgfpathlineto{\pgfpoint{2.109793\du}{14.602429\du}}
\pgfpathlineto{\pgfpoint{2.093729\du}{14.616667\du}}
\pgfpathlineto{\pgfpoint{2.076204\du}{14.629446\du}}
\pgfpathlineto{\pgfpoint{2.057950\du}{14.642224\du}}
\pgfpathlineto{\pgfpoint{2.038965\du}{14.654637\du}}
\pgfpathlineto{\pgfpoint{2.018884\du}{14.666320\du}}
\pgfpathlineto{\pgfpoint{1.997344\du}{14.676908\du}}
\pgfpathlineto{\pgfpoint{1.974708\du}{14.687861\du}}
\pgfpathlineto{\pgfpoint{1.951706\du}{14.697719\du}}
\pgfpathlineto{\pgfpoint{1.927975\du}{14.707211\du}}
\pgfpathlineto{\pgfpoint{1.902784\du}{14.715243\du}}
\pgfpathlineto{\pgfpoint{1.877227\du}{14.723641\du}}
\pgfpathlineto{\pgfpoint{1.850575\du}{14.731308\du}}
\pgfpathlineto{\pgfpoint{1.823923\du}{14.738245\du}}
\pgfpathlineto{\pgfpoint{1.795810\du}{14.744086\du}}
\pgfpathlineto{\pgfpoint{1.767333\du}{14.749197\du}}
\pgfpathlineto{\pgfpoint{1.737760\du}{14.753579\du}}
\pgfpathlineto{\pgfpoint{1.707457\du}{14.757595\du}}
\pgfpathlineto{\pgfpoint{1.677519\du}{14.760515\du}}
\pgfpathlineto{\pgfpoint{1.646851\du}{14.762706\du}}
\pgfpathlineto{\pgfpoint{1.615452\du}{14.763801\du}}
\pgfpathlineto{\pgfpoint{1.584054\du}{14.764532\du}}
\pgfpathlineto{\pgfpoint{1.553021\du}{14.763801\du}}
\pgfpathlineto{\pgfpoint{1.521622\du}{14.762706\du}}
\pgfpathlineto{\pgfpoint{1.490589\du}{14.760515\du}}
\pgfpathlineto{\pgfpoint{1.460286\du}{14.757595\du}}
\pgfpathlineto{\pgfpoint{1.430713\du}{14.753579\du}}
\pgfpathlineto{\pgfpoint{1.401141\du}{14.749197\du}}
\pgfpathlineto{\pgfpoint{1.372663\du}{14.744086\du}}
\pgfpathlineto{\pgfpoint{1.344916\du}{14.738245\du}}
\pgfpathlineto{\pgfpoint{1.317533\du}{14.731308\du}}
\pgfpathlineto{\pgfpoint{1.291246\du}{14.723641\du}}
\pgfpathlineto{\pgfpoint{1.266055\du}{14.715243\du}}
\pgfpathlineto{\pgfpoint{1.240498\du}{14.707211\du}}
\pgfpathlineto{\pgfpoint{1.216402\du}{14.697719\du}}
\pgfpathlineto{\pgfpoint{1.193766\du}{14.687861\du}}
\pgfpathlineto{\pgfpoint{1.171130\du}{14.676908\du}}
\pgfpathlineto{\pgfpoint{1.150319\du}{14.666320\du}}
\pgfpathlineto{\pgfpoint{1.129874\du}{14.654637\du}}
\pgfpathlineto{\pgfpoint{1.109428\du}{14.642224\du}}
\pgfpathlineto{\pgfpoint{1.091539\du}{14.629446\du}}
\pgfpathlineto{\pgfpoint{1.075109\du}{14.616667\du}}
\pgfpathlineto{\pgfpoint{1.058315\du}{14.602429\du}}
\pgfpathlineto{\pgfpoint{1.043711\du}{14.588555\du}}
\pgfpathlineto{\pgfpoint{1.030567\du}{14.573951\du}}
\pgfpathlineto{\pgfpoint{1.018154\du}{14.558982\du}}
\pgfpathlineto{\pgfpoint{1.007201\du}{14.544013\du}}
\pgfpathlineto{\pgfpoint{0.997709\du}{14.528314\du}}
\pgfpathlineto{\pgfpoint{0.988946\du}{14.512250\du}}
\pgfpathlineto{\pgfpoint{0.982010\du}{14.496551\du}}
\pgfpathlineto{\pgfpoint{0.977263\du}{14.479391\du}}
\pgfpathlineto{\pgfpoint{0.972882\du}{14.463327\du}}
\pgfpathlineto{\pgfpoint{0.970326\du}{14.446167\du}}
\pgfpathlineto{\pgfpoint{0.969231\du}{14.428643\du}}
\pgfpathlineto{\pgfpoint{0.970326\du}{14.411118\du}}
\pgfpathlineto{\pgfpoint{0.972882\du}{14.393958\du}}
\pgfpathlineto{\pgfpoint{0.977263\du}{14.377894\du}}
\pgfpathlineto{\pgfpoint{0.982010\du}{14.360735\du}}
\pgfpathlineto{\pgfpoint{0.988946\du}{14.345035\du}}
\pgfpathlineto{\pgfpoint{0.997709\du}{14.329336\du}}
\pgfpathlineto{\pgfpoint{1.007201\du}{14.313272\du}}
\pgfpathlineto{\pgfpoint{1.018154\du}{14.298303\du}}
\pgfpathlineto{\pgfpoint{1.030567\du}{14.283699\du}}
\pgfpathlineto{\pgfpoint{1.043711\du}{14.269095\du}}
\pgfpathlineto{\pgfpoint{1.058315\du}{14.254856\du}}
\pgfpathlineto{\pgfpoint{1.075109\du}{14.240983\du}}
\pgfpathlineto{\pgfpoint{1.091539\du}{14.227839\du}}
\pgfpathlineto{\pgfpoint{1.109428\du}{14.215791\du}}
\pgfpathlineto{\pgfpoint{1.129874\du}{14.203378\du}}
\pgfpathlineto{\pgfpoint{1.150319\du}{14.191695\du}}
\pgfpathlineto{\pgfpoint{1.171130\du}{14.180742\du}}
\pgfpathlineto{\pgfpoint{1.193766\du}{14.170154\du}}
\pgfpathlineto{\pgfpoint{1.216402\du}{14.159566\du}}
\pgfpathlineto{\pgfpoint{1.240498\du}{14.150804\du}}
\pgfpathlineto{\pgfpoint{1.266055\du}{14.142042\du}}
\pgfpathlineto{\pgfpoint{1.291246\du}{14.133644\du}}
\pgfpathlineto{\pgfpoint{1.317533\du}{14.125977\du}}
\pgfpathlineto{\pgfpoint{1.344916\du}{14.119771\du}}
\pgfpathlineto{\pgfpoint{1.372663\du}{14.113199\du}}
\pgfpathlineto{\pgfpoint{1.401141\du}{14.108088\du}}
\pgfpathlineto{\pgfpoint{1.430713\du}{14.104071\du}}
\pgfpathlineto{\pgfpoint{1.460286\du}{14.099690\du}}
\pgfpathlineto{\pgfpoint{1.490589\du}{14.096770\du}}
\pgfpathlineto{\pgfpoint{1.521622\du}{14.095309\du}}
\pgfpathlineto{\pgfpoint{1.553021\du}{14.093484\du}}
\pgfpathlineto{\pgfpoint{1.584054\du}{14.093484\du}}
\pgfpathlineto{\pgfpoint{1.615452\du}{14.093484\du}}
\pgfpathlineto{\pgfpoint{1.646851\du}{14.095309\du}}
\pgfpathlineto{\pgfpoint{1.677519\du}{14.096770\du}}
\pgfpathlineto{\pgfpoint{1.707457\du}{14.099690\du}}
\pgfpathlineto{\pgfpoint{1.737760\du}{14.104071\du}}
\pgfpathlineto{\pgfpoint{1.767333\du}{14.108088\du}}
\pgfpathlineto{\pgfpoint{1.795810\du}{14.113199\du}}
\pgfpathlineto{\pgfpoint{1.823923\du}{14.119771\du}}
\pgfpathlineto{\pgfpoint{1.850575\du}{14.125977\du}}
\pgfpathlineto{\pgfpoint{1.877227\du}{14.133644\du}}
\pgfpathlineto{\pgfpoint{1.902784\du}{14.142042\du}}
\pgfpathlineto{\pgfpoint{1.927975\du}{14.150804\du}}
\pgfpathlineto{\pgfpoint{1.951706\du}{14.159566\du}}
\pgfpathlineto{\pgfpoint{1.974708\du}{14.170154\du}}
\pgfpathlineto{\pgfpoint{1.997344\du}{14.180742\du}}
\pgfpathlineto{\pgfpoint{2.018884\du}{14.191695\du}}
\pgfpathlineto{\pgfpoint{2.038965\du}{14.203378\du}}
\pgfpathlineto{\pgfpoint{2.057950\du}{14.215791\du}}
\pgfpathlineto{\pgfpoint{2.076204\du}{14.227839\du}}
\pgfpathlineto{\pgfpoint{2.093729\du}{14.240983\du}}
\pgfpathlineto{\pgfpoint{2.109793\du}{14.254856\du}}
\pgfpathlineto{\pgfpoint{2.124762\du}{14.269095\du}}
\pgfpathlineto{\pgfpoint{2.138271\du}{14.283699\du}}
\pgfpathlineto{\pgfpoint{2.149954\du}{14.298303\du}}
\pgfpathlineto{\pgfpoint{2.161637\du}{14.313272\du}}
\pgfpathlineto{\pgfpoint{2.171130\du}{14.329336\du}}
\pgfpathlineto{\pgfpoint{2.179527\du}{14.345035\du}}
\pgfpathlineto{\pgfpoint{2.186099\du}{14.360735\du}}
\pgfpathlineto{\pgfpoint{2.191940\du}{14.377894\du}}
\pgfpathlineto{\pgfpoint{2.195591\du}{14.393958\du}}
\pgfpathlineto{\pgfpoint{2.198147\du}{14.411118\du}}
\pgfpathlineto{\pgfpoint{2.198512\du}{14.428643\du}}
\pgfusepath{fill}
\pgfsetbuttcap
\pgfsetmiterjoin
\pgfsetdash{}{0pt}
\definecolor{dialinecolor}{rgb}{0.678431, 0.839216, 0.905882}
\pgfsetfillcolor{dialinecolor}
\pgfpathmoveto{\pgfpoint{1.584054\du}{14.774389\du}}
\pgfpathlineto{\pgfpoint{1.584054\du}{14.774389\du}}
\pgfpathlineto{\pgfpoint{1.600118\du}{14.774389\du}}
\pgfpathlineto{\pgfpoint{1.616183\du}{14.774024\du}}
\pgfpathlineto{\pgfpoint{1.632247\du}{14.773294\du}}
\pgfpathlineto{\pgfpoint{1.647581\du}{14.772564\du}}
\pgfpathlineto{\pgfpoint{1.663280\du}{14.771468\du}}
\pgfpathlineto{\pgfpoint{1.678614\du}{14.770373\du}}
\pgfpathlineto{\pgfpoint{1.693583\du}{14.769278\du}}
\pgfpathlineto{\pgfpoint{1.709647\du}{14.767452\du}}
\pgfpathlineto{\pgfpoint{1.723886\du}{14.765627\du}}
\pgfpathlineto{\pgfpoint{1.738855\du}{14.763801\du}}
\pgfpathlineto{\pgfpoint{1.753824\du}{14.761611\du}}
\pgfpathlineto{\pgfpoint{1.769158\du}{14.759420\du}}
\pgfpathlineto{\pgfpoint{1.783397\du}{14.756499\du}}
\pgfpathlineto{\pgfpoint{1.797271\du}{14.753944\du}}
\pgfpathlineto{\pgfpoint{1.811874\du}{14.751023\du}}
\pgfpathlineto{\pgfpoint{1.825748\du}{14.748102\du}}
\pgfpathlineto{\pgfpoint{1.839257\du}{14.744451\du}}
\pgfpathlineto{\pgfpoint{1.853130\du}{14.741165\du}}
\pgfpathlineto{\pgfpoint{1.866639\du}{14.737514\du}}
\pgfpathlineto{\pgfpoint{1.879782\du}{14.733498\du}}
\pgfpathlineto{\pgfpoint{1.892926\du}{14.729482\du}}
\pgfpathlineto{\pgfpoint{1.906069\du}{14.725466\du}}
\pgfpathlineto{\pgfpoint{1.918848\du}{14.720720\du}}
\pgfpathlineto{\pgfpoint{1.931626\du}{14.716704\du}}
\pgfpathlineto{\pgfpoint{1.943309\du}{14.711958\du}}
\pgfpathlineto{\pgfpoint{1.956088\du}{14.707211\du}}
\pgfpathlineto{\pgfpoint{1.967406\du}{14.701735\du}}
\pgfpathlineto{\pgfpoint{1.979089\du}{14.696624\du}}
\pgfpathlineto{\pgfpoint{1.990407\du}{14.691512\du}}
\pgfpathlineto{\pgfpoint{2.001725\du}{14.686036\du}}
\pgfpathlineto{\pgfpoint{2.012678\du}{14.680924\du}}
\pgfpathlineto{\pgfpoint{2.023996\du}{14.675083\du}}
\pgfpathlineto{\pgfpoint{2.033853\du}{14.669241\du}}
\pgfpathlineto{\pgfpoint{2.044076\du}{14.662670\du}}
\pgfpathlineto{\pgfpoint{2.053934\du}{14.656828\du}}
\pgfpathlineto{\pgfpoint{2.064156\du}{14.650256\du}}
\pgfpathlineto{\pgfpoint{2.073649\du}{14.644050\du}}
\pgfpathlineto{\pgfpoint{2.082776\du}{14.637478\du}}
\pgfpathlineto{\pgfpoint{2.091539\du}{14.631271\du}}
\pgfpathlineto{\pgfpoint{2.099936\du}{14.623969\du}}
\pgfpathlineto{\pgfpoint{2.108333\du}{14.617032\du}}
\pgfpathlineto{\pgfpoint{2.116365\du}{14.610096\du}}
\pgfpathlineto{\pgfpoint{2.124397\du}{14.603159\du}}
\pgfpathlineto{\pgfpoint{2.131334\du}{14.595857\du}}
\pgfpathlineto{\pgfpoint{2.139001\du}{14.588555\du}}
\pgfpathlineto{\pgfpoint{2.145573\du}{14.580888\du}}
\pgfpathlineto{\pgfpoint{2.152145\du}{14.573221\du}}
\pgfpathlineto{\pgfpoint{2.158351\du}{14.565554\du}}
\pgfpathlineto{\pgfpoint{2.164558\du}{14.557522\du}}
\pgfpathlineto{\pgfpoint{2.170399\du}{14.549855\du}}
\pgfpathlineto{\pgfpoint{2.175146\du}{14.541457\du}}
\pgfpathlineto{\pgfpoint{2.179892\du}{14.533425\du}}
\pgfpathlineto{\pgfpoint{2.184638\du}{14.525028\du}}
\pgfpathlineto{\pgfpoint{2.189019\du}{14.516996\du}}
\pgfpathlineto{\pgfpoint{2.192305\du}{14.508234\du}}
\pgfpathlineto{\pgfpoint{2.196321\du}{14.500201\du}}
\pgfpathlineto{\pgfpoint{2.198877\du}{14.491439\du}}
\pgfpathlineto{\pgfpoint{2.201798\du}{14.482677\du}}
\pgfpathlineto{\pgfpoint{2.203623\du}{14.473549\du}}
\pgfpathlineto{\pgfpoint{2.205814\du}{14.464787\du}}
\pgfpathlineto{\pgfpoint{2.207274\du}{14.456025\du}}
\pgfpathlineto{\pgfpoint{2.208369\du}{14.446897\du}}
\pgfpathlineto{\pgfpoint{2.209100\du}{14.438135\du}}
\pgfpathlineto{\pgfpoint{2.209100\du}{14.428643\du}}
\pgfpathlineto{\pgfpoint{2.189019\du}{14.428643\du}}
\pgfpathlineto{\pgfpoint{2.188289\du}{14.436675\du}}
\pgfpathlineto{\pgfpoint{2.187924\du}{14.445072\du}}
\pgfpathlineto{\pgfpoint{2.187194\du}{14.453104\du}}
\pgfpathlineto{\pgfpoint{2.185733\du}{14.460771\du}}
\pgfpathlineto{\pgfpoint{2.184273\du}{14.469168\du}}
\pgfpathlineto{\pgfpoint{2.181717\du}{14.477200\du}}
\pgfpathlineto{\pgfpoint{2.179527\du}{14.484867\du}}
\pgfpathlineto{\pgfpoint{2.176241\du}{14.492534\du}}
\pgfpathlineto{\pgfpoint{2.173685\du}{14.500201\du}}
\pgfpathlineto{\pgfpoint{2.170399\du}{14.508234\du}}
\pgfpathlineto{\pgfpoint{2.166383\du}{14.515901\du}}
\pgfpathlineto{\pgfpoint{2.161637\du}{14.523568\du}}
\pgfpathlineto{\pgfpoint{2.157621\du}{14.531235\du}}
\pgfpathlineto{\pgfpoint{2.152510\du}{14.538172\du}}
\pgfpathlineto{\pgfpoint{2.147763\du}{14.545839\du}}
\pgfpathlineto{\pgfpoint{2.142287\du}{14.552775\du}}
\pgfpathlineto{\pgfpoint{2.137176\du}{14.560442\du}}
\pgfpathlineto{\pgfpoint{2.130239\du}{14.567379\du}}
\pgfpathlineto{\pgfpoint{2.124397\du}{14.574316\du}}
\pgfpathlineto{\pgfpoint{2.117095\du}{14.581253\du}}
\pgfpathlineto{\pgfpoint{2.110158\du}{14.588555\du}}
\pgfpathlineto{\pgfpoint{2.102857\du}{14.594762\du}}
\pgfpathlineto{\pgfpoint{2.095920\du}{14.601698\du}}
\pgfpathlineto{\pgfpoint{2.087522\du}{14.608270\du}}
\pgfpathlineto{\pgfpoint{2.079490\du}{14.614842\du}}
\pgfpathlineto{\pgfpoint{2.070363\du}{14.621048\du}}
\pgfpathlineto{\pgfpoint{2.061601\du}{14.626890\du}}
\pgfpathlineto{\pgfpoint{2.052108\du}{14.633462\du}}
\pgfpathlineto{\pgfpoint{2.042981\du}{14.640034\du}}
\pgfpathlineto{\pgfpoint{2.033853\du}{14.645145\du}}
\pgfpathlineto{\pgfpoint{2.023996\du}{14.650986\du}}
\pgfpathlineto{\pgfpoint{2.014138\du}{14.656828\du}}
\pgfpathlineto{\pgfpoint{2.003550\du}{14.662304\du}}
\pgfpathlineto{\pgfpoint{1.992962\du}{14.668146\du}}
\pgfpathlineto{\pgfpoint{1.982375\du}{14.672527\du}}
\pgfpathlineto{\pgfpoint{1.970692\du}{14.678369\du}}
\pgfpathlineto{\pgfpoint{1.959374\du}{14.683115\du}}
\pgfpathlineto{\pgfpoint{1.948056\du}{14.687861\du}}
\pgfpathlineto{\pgfpoint{1.936372\du}{14.692607\du}}
\pgfpathlineto{\pgfpoint{1.924324\du}{14.697354\du}}
\pgfpathlineto{\pgfpoint{1.911911\du}{14.701370\du}}
\pgfpathlineto{\pgfpoint{1.900228\du}{14.706116\du}}
\pgfpathlineto{\pgfpoint{1.887449\du}{14.710132\du}}
\pgfpathlineto{\pgfpoint{1.873941\du}{14.713783\du}}
\pgfpathlineto{\pgfpoint{1.861162\du}{14.717799\du}}
\pgfpathlineto{\pgfpoint{1.848019\du}{14.721085\du}}
\pgfpathlineto{\pgfpoint{1.834510\du}{14.724736\du}}
\pgfpathlineto{\pgfpoint{1.821002\du}{14.727657\du}}
\pgfpathlineto{\pgfpoint{1.807128\du}{14.731308\du}}
\pgfpathlineto{\pgfpoint{1.793254\du}{14.733498\du}}
\pgfpathlineto{\pgfpoint{1.779381\du}{14.736419\du}}
\pgfpathlineto{\pgfpoint{1.765142\du}{14.738610\du}}
\pgfpathlineto{\pgfpoint{1.750903\du}{14.741165\du}}
\pgfpathlineto{\pgfpoint{1.736664\du}{14.743356\du}}
\pgfpathlineto{\pgfpoint{1.722061\du}{14.745181\du}}
\pgfpathlineto{\pgfpoint{1.707092\du}{14.747007\du}}
\pgfpathlineto{\pgfpoint{1.691758\du}{14.748832\du}}
\pgfpathlineto{\pgfpoint{1.677154\du}{14.749928\du}}
\pgfpathlineto{\pgfpoint{1.661455\du}{14.751023\du}}
\pgfpathlineto{\pgfpoint{1.646486\du}{14.752118\du}}
\pgfpathlineto{\pgfpoint{1.631152\du}{14.752848\du}}
\pgfpathlineto{\pgfpoint{1.615452\du}{14.753579\du}}
\pgfpathlineto{\pgfpoint{1.600118\du}{14.753944\du}}
\pgfpathlineto{\pgfpoint{1.584054\du}{14.753944\du}}
\pgfpathlineto{\pgfpoint{1.584054\du}{14.753944\du}}
\pgfpathlineto{\pgfpoint{1.584054\du}{14.753944\du}}
\pgfpathlineto{\pgfpoint{1.582959\du}{14.753944\du}}
\pgfpathlineto{\pgfpoint{1.581863\du}{14.753944\du}}
\pgfpathlineto{\pgfpoint{1.581133\du}{14.754674\du}}
\pgfpathlineto{\pgfpoint{1.579308\du}{14.754674\du}}
\pgfpathlineto{\pgfpoint{1.578943\du}{14.755039\du}}
\pgfpathlineto{\pgfpoint{1.577847\du}{14.755769\du}}
\pgfpathlineto{\pgfpoint{1.577482\du}{14.756499\du}}
\pgfpathlineto{\pgfpoint{1.577117\du}{14.756864\du}}
\pgfpathlineto{\pgfpoint{1.575657\du}{14.758690\du}}
\pgfpathlineto{\pgfpoint{1.574196\du}{14.760515\du}}
\pgfpathlineto{\pgfpoint{1.574196\du}{14.762341\du}}
\pgfpathlineto{\pgfpoint{1.573831\du}{14.764532\du}}
\pgfpathlineto{\pgfpoint{1.574196\du}{14.766357\du}}
\pgfpathlineto{\pgfpoint{1.574196\du}{14.768182\du}}
\pgfpathlineto{\pgfpoint{1.575657\du}{14.769643\du}}
\pgfpathlineto{\pgfpoint{1.577117\du}{14.771103\du}}
\pgfpathlineto{\pgfpoint{1.577482\du}{14.772199\du}}
\pgfpathlineto{\pgfpoint{1.577847\du}{14.772564\du}}
\pgfpathlineto{\pgfpoint{1.578943\du}{14.773294\du}}
\pgfpathlineto{\pgfpoint{1.579308\du}{14.773294\du}}
\pgfpathlineto{\pgfpoint{1.581133\du}{14.774024\du}}
\pgfpathlineto{\pgfpoint{1.581863\du}{14.774389\du}}
\pgfpathlineto{\pgfpoint{1.582959\du}{14.774389\du}}
\pgfpathlineto{\pgfpoint{1.584054\du}{14.774389\du}}
\pgfusepath{fill}
\pgfsetbuttcap
\pgfsetmiterjoin
\pgfsetdash{}{0pt}
\definecolor{dialinecolor}{rgb}{0.678431, 0.839216, 0.905882}
\pgfsetfillcolor{dialinecolor}
\pgfpathmoveto{\pgfpoint{0.959374\du}{14.428643\du}}
\pgfpathlineto{\pgfpoint{0.959374\du}{14.428643\du}}
\pgfpathlineto{\pgfpoint{0.959374\du}{14.437405\du}}
\pgfpathlineto{\pgfpoint{0.960104\du}{14.446897\du}}
\pgfpathlineto{\pgfpoint{0.961199\du}{14.456025\du}}
\pgfpathlineto{\pgfpoint{0.963390\du}{14.464787\du}}
\pgfpathlineto{\pgfpoint{0.964485\du}{14.473549\du}}
\pgfpathlineto{\pgfpoint{0.967041\du}{14.482677\du}}
\pgfpathlineto{\pgfpoint{0.969231\du}{14.491439\du}}
\pgfpathlineto{\pgfpoint{0.972517\du}{14.500201\du}}
\pgfpathlineto{\pgfpoint{0.975803\du}{14.508234\du}}
\pgfpathlineto{\pgfpoint{0.979819\du}{14.516996\du}}
\pgfpathlineto{\pgfpoint{0.984200\du}{14.525028\du}}
\pgfpathlineto{\pgfpoint{0.988581\du}{14.533425\du}}
\pgfpathlineto{\pgfpoint{0.993328\du}{14.541457\du}}
\pgfpathlineto{\pgfpoint{0.998439\du}{14.549855\du}}
\pgfpathlineto{\pgfpoint{1.004646\du}{14.557522\du}}
\pgfpathlineto{\pgfpoint{1.009392\du}{14.565554\du}}
\pgfpathlineto{\pgfpoint{1.015963\du}{14.573221\du}}
\pgfpathlineto{\pgfpoint{1.022900\du}{14.580888\du}}
\pgfpathlineto{\pgfpoint{1.029472\du}{14.588555\du}}
\pgfpathlineto{\pgfpoint{1.036409\du}{14.595857\du}}
\pgfpathlineto{\pgfpoint{1.043711\du}{14.603159\du}}
\pgfpathlineto{\pgfpoint{1.052473\du}{14.610096\du}}
\pgfpathlineto{\pgfpoint{1.059410\du}{14.617032\du}}
\pgfpathlineto{\pgfpoint{1.068537\du}{14.623969\du}}
\pgfpathlineto{\pgfpoint{1.076570\du}{14.631271\du}}
\pgfpathlineto{\pgfpoint{1.085697\du}{14.637478\du}}
\pgfpathlineto{\pgfpoint{1.095555\du}{14.644050\du}}
\pgfpathlineto{\pgfpoint{1.104682\du}{14.650256\du}}
\pgfpathlineto{\pgfpoint{1.114175\du}{14.656828\du}}
\pgfpathlineto{\pgfpoint{1.124032\du}{14.662670\du}}
\pgfpathlineto{\pgfpoint{1.134255\du}{14.669241\du}}
\pgfpathlineto{\pgfpoint{1.144478\du}{14.675083\du}}
\pgfpathlineto{\pgfpoint{1.155430\du}{14.680924\du}}
\pgfpathlineto{\pgfpoint{1.166748\du}{14.686036\du}}
\pgfpathlineto{\pgfpoint{1.178066\du}{14.691512\du}}
\pgfpathlineto{\pgfpoint{1.189384\du}{14.696624\du}}
\pgfpathlineto{\pgfpoint{1.201433\du}{14.701735\du}}
\pgfpathlineto{\pgfpoint{1.212751\du}{14.707211\du}}
\pgfpathlineto{\pgfpoint{1.225164\du}{14.711958\du}}
\pgfpathlineto{\pgfpoint{1.237212\du}{14.716704\du}}
\pgfpathlineto{\pgfpoint{1.249260\du}{14.720720\du}}
\pgfpathlineto{\pgfpoint{1.262404\du}{14.725466\du}}
\pgfpathlineto{\pgfpoint{1.275182\du}{14.729482\du}}
\pgfpathlineto{\pgfpoint{1.288691\du}{14.733498\du}}
\pgfpathlineto{\pgfpoint{1.302199\du}{14.737514\du}}
\pgfpathlineto{\pgfpoint{1.314978\du}{14.741165\du}}
\pgfpathlineto{\pgfpoint{1.328486\du}{14.744451\du}}
\pgfpathlineto{\pgfpoint{1.342360\du}{14.748102\du}}
\pgfpathlineto{\pgfpoint{1.356964\du}{14.751023\du}}
\pgfpathlineto{\pgfpoint{1.371568\du}{14.753944\du}}
\pgfpathlineto{\pgfpoint{1.385441\du}{14.756499\du}}
\pgfpathlineto{\pgfpoint{1.399680\du}{14.759420\du}}
\pgfpathlineto{\pgfpoint{1.414649\du}{14.761611\du}}
\pgfpathlineto{\pgfpoint{1.429618\du}{14.763801\du}}
\pgfpathlineto{\pgfpoint{1.444222\du}{14.765627\du}}
\pgfpathlineto{\pgfpoint{1.458826\du}{14.767452\du}}
\pgfpathlineto{\pgfpoint{1.474525\du}{14.769278\du}}
\pgfpathlineto{\pgfpoint{1.490224\du}{14.770373\du}}
\pgfpathlineto{\pgfpoint{1.504828\du}{14.771468\du}}
\pgfpathlineto{\pgfpoint{1.520892\du}{14.772564\du}}
\pgfpathlineto{\pgfpoint{1.536226\du}{14.773294\du}}
\pgfpathlineto{\pgfpoint{1.552291\du}{14.774024\du}}
\pgfpathlineto{\pgfpoint{1.568355\du}{14.774389\du}}
\pgfpathlineto{\pgfpoint{1.584054\du}{14.774389\du}}
\pgfpathlineto{\pgfpoint{1.584054\du}{14.753944\du}}
\pgfpathlineto{\pgfpoint{1.568355\du}{14.753944\du}}
\pgfpathlineto{\pgfpoint{1.553021\du}{14.753579\du}}
\pgfpathlineto{\pgfpoint{1.536957\du}{14.752848\du}}
\pgfpathlineto{\pgfpoint{1.522353\du}{14.752118\du}}
\pgfpathlineto{\pgfpoint{1.506654\du}{14.751023\du}}
\pgfpathlineto{\pgfpoint{1.491685\du}{14.749928\du}}
\pgfpathlineto{\pgfpoint{1.476716\du}{14.748832\du}}
\pgfpathlineto{\pgfpoint{1.461747\du}{14.747007\du}}
\pgfpathlineto{\pgfpoint{1.446778\du}{14.745181\du}}
\pgfpathlineto{\pgfpoint{1.432174\du}{14.743356\du}}
\pgfpathlineto{\pgfpoint{1.417570\du}{14.741165\du}}
\pgfpathlineto{\pgfpoint{1.403331\du}{14.738610\du}}
\pgfpathlineto{\pgfpoint{1.389457\du}{14.736419\du}}
\pgfpathlineto{\pgfpoint{1.375584\du}{14.733498\du}}
\pgfpathlineto{\pgfpoint{1.361345\du}{14.731308\du}}
\pgfpathlineto{\pgfpoint{1.347471\du}{14.727657\du}}
\pgfpathlineto{\pgfpoint{1.333963\du}{14.724736\du}}
\pgfpathlineto{\pgfpoint{1.320819\du}{14.721085\du}}
\pgfpathlineto{\pgfpoint{1.307311\du}{14.717799\du}}
\pgfpathlineto{\pgfpoint{1.294532\du}{14.713783\du}}
\pgfpathlineto{\pgfpoint{1.281389\du}{14.710132\du}}
\pgfpathlineto{\pgfpoint{1.268610\du}{14.706116\du}}
\pgfpathlineto{\pgfpoint{1.256562\du}{14.701370\du}}
\pgfpathlineto{\pgfpoint{1.244149\du}{14.697354\du}}
\pgfpathlineto{\pgfpoint{1.231736\du}{14.692607\du}}
\pgfpathlineto{\pgfpoint{1.220418\du}{14.687861\du}}
\pgfpathlineto{\pgfpoint{1.208735\du}{14.683115\du}}
\pgfpathlineto{\pgfpoint{1.197782\du}{14.678369\du}}
\pgfpathlineto{\pgfpoint{1.186464\du}{14.672527\du}}
\pgfpathlineto{\pgfpoint{1.175876\du}{14.668146\du}}
\pgfpathlineto{\pgfpoint{1.164923\du}{14.662304\du}}
\pgfpathlineto{\pgfpoint{1.154700\du}{14.656828\du}}
\pgfpathlineto{\pgfpoint{1.144478\du}{14.650986\du}}
\pgfpathlineto{\pgfpoint{1.134620\du}{14.645145\du}}
\pgfpathlineto{\pgfpoint{1.125127\du}{14.640034\du}}
\pgfpathlineto{\pgfpoint{1.115270\du}{14.633462\du}}
\pgfpathlineto{\pgfpoint{1.106873\du}{14.626890\du}}
\pgfpathlineto{\pgfpoint{1.097745\du}{14.621048\du}}
\pgfpathlineto{\pgfpoint{1.089348\du}{14.614842\du}}
\pgfpathlineto{\pgfpoint{1.080951\du}{14.608270\du}}
\pgfpathlineto{\pgfpoint{1.072919\du}{14.601698\du}}
\pgfpathlineto{\pgfpoint{1.065617\du}{14.594762\du}}
\pgfpathlineto{\pgfpoint{1.057950\du}{14.588555\du}}
\pgfpathlineto{\pgfpoint{1.051378\du}{14.581253\du}}
\pgfpathlineto{\pgfpoint{1.044441\du}{14.574316\du}}
\pgfpathlineto{\pgfpoint{1.038234\du}{14.567379\du}}
\pgfpathlineto{\pgfpoint{1.032028\du}{14.560442\du}}
\pgfpathlineto{\pgfpoint{1.025821\du}{14.552775\du}}
\pgfpathlineto{\pgfpoint{1.020710\du}{14.545839\du}}
\pgfpathlineto{\pgfpoint{1.015598\du}{14.538172\du}}
\pgfpathlineto{\pgfpoint{1.010852\du}{14.531235\du}}
\pgfpathlineto{\pgfpoint{1.006471\du}{14.523568\du}}
\pgfpathlineto{\pgfpoint{1.002090\du}{14.515901\du}}
\pgfpathlineto{\pgfpoint{0.998439\du}{14.508234\du}}
\pgfpathlineto{\pgfpoint{0.995153\du}{14.500201\du}}
\pgfpathlineto{\pgfpoint{0.991867\du}{14.492534\du}}
\pgfpathlineto{\pgfpoint{0.988946\du}{14.484867\du}}
\pgfpathlineto{\pgfpoint{0.986391\du}{14.477200\du}}
\pgfpathlineto{\pgfpoint{0.984565\du}{14.469168\du}}
\pgfpathlineto{\pgfpoint{0.982740\du}{14.460771\du}}
\pgfpathlineto{\pgfpoint{0.981644\du}{14.453104\du}}
\pgfpathlineto{\pgfpoint{0.980184\du}{14.445072\du}}
\pgfpathlineto{\pgfpoint{0.979819\du}{14.436675\du}}
\pgfpathlineto{\pgfpoint{0.979819\du}{14.428643\du}}
\pgfpathlineto{\pgfpoint{0.979819\du}{14.428643\du}}
\pgfpathlineto{\pgfpoint{0.979819\du}{14.428643\du}}
\pgfpathlineto{\pgfpoint{0.979819\du}{14.427547\du}}
\pgfpathlineto{\pgfpoint{0.979819\du}{14.426452\du}}
\pgfpathlineto{\pgfpoint{0.979454\du}{14.424992\du}}
\pgfpathlineto{\pgfpoint{0.979454\du}{14.423896\du}}
\pgfpathlineto{\pgfpoint{0.979089\du}{14.423531\du}}
\pgfpathlineto{\pgfpoint{0.977993\du}{14.422071\du}}
\pgfpathlineto{\pgfpoint{0.977263\du}{14.421706\du}}
\pgfpathlineto{\pgfpoint{0.977263\du}{14.420975\du}}
\pgfpathlineto{\pgfpoint{0.975073\du}{14.419880\du}}
\pgfpathlineto{\pgfpoint{0.973247\du}{14.419150\du}}
\pgfpathlineto{\pgfpoint{0.971422\du}{14.418785\du}}
\pgfpathlineto{\pgfpoint{0.969231\du}{14.418055\du}}
\pgfpathlineto{\pgfpoint{0.967771\du}{14.418785\du}}
\pgfpathlineto{\pgfpoint{0.965945\du}{14.419150\du}}
\pgfpathlineto{\pgfpoint{0.964120\du}{14.419880\du}}
\pgfpathlineto{\pgfpoint{0.962294\du}{14.420975\du}}
\pgfpathlineto{\pgfpoint{0.961564\du}{14.421706\du}}
\pgfpathlineto{\pgfpoint{0.961199\du}{14.422071\du}}
\pgfpathlineto{\pgfpoint{0.960834\du}{14.423531\du}}
\pgfpathlineto{\pgfpoint{0.960104\du}{14.423896\du}}
\pgfpathlineto{\pgfpoint{0.960104\du}{14.424992\du}}
\pgfpathlineto{\pgfpoint{0.959374\du}{14.426452\du}}
\pgfpathlineto{\pgfpoint{0.959374\du}{14.427547\du}}
\pgfpathlineto{\pgfpoint{0.959374\du}{14.428643\du}}
\pgfusepath{fill}
\pgfsetbuttcap
\pgfsetmiterjoin
\pgfsetdash{}{0pt}
\definecolor{dialinecolor}{rgb}{0.678431, 0.839216, 0.905882}
\pgfsetfillcolor{dialinecolor}
\pgfpathmoveto{\pgfpoint{1.584054\du}{14.082896\du}}
\pgfpathlineto{\pgfpoint{1.584054\du}{14.082896\du}}
\pgfpathlineto{\pgfpoint{1.568355\du}{14.082896\du}}
\pgfpathlineto{\pgfpoint{1.552291\du}{14.083261\du}}
\pgfpathlineto{\pgfpoint{1.536226\du}{14.083991\du}}
\pgfpathlineto{\pgfpoint{1.520892\du}{14.084721\du}}
\pgfpathlineto{\pgfpoint{1.504828\du}{14.085817\du}}
\pgfpathlineto{\pgfpoint{1.490224\du}{14.086912\du}}
\pgfpathlineto{\pgfpoint{1.474525\du}{14.088007\du}}
\pgfpathlineto{\pgfpoint{1.458826\du}{14.089833\du}}
\pgfpathlineto{\pgfpoint{1.444222\du}{14.091658\du}}
\pgfpathlineto{\pgfpoint{1.429618\du}{14.093484\du}}
\pgfpathlineto{\pgfpoint{1.414649\du}{14.095674\du}}
\pgfpathlineto{\pgfpoint{1.399680\du}{14.098230\du}}
\pgfpathlineto{\pgfpoint{1.385441\du}{14.101151\du}}
\pgfpathlineto{\pgfpoint{1.371568\du}{14.103341\du}}
\pgfpathlineto{\pgfpoint{1.356964\du}{14.106262\du}}
\pgfpathlineto{\pgfpoint{1.342360\du}{14.109183\du}}
\pgfpathlineto{\pgfpoint{1.328486\du}{14.112834\du}}
\pgfpathlineto{\pgfpoint{1.314978\du}{14.116120\du}}
\pgfpathlineto{\pgfpoint{1.302199\du}{14.120136\du}}
\pgfpathlineto{\pgfpoint{1.288691\du}{14.123787\du}}
\pgfpathlineto{\pgfpoint{1.275182\du}{14.127803\du}}
\pgfpathlineto{\pgfpoint{1.262404\du}{14.132184\du}}
\pgfpathlineto{\pgfpoint{1.249260\du}{14.136565\du}}
\pgfpathlineto{\pgfpoint{1.237212\du}{14.140946\du}}
\pgfpathlineto{\pgfpoint{1.225164\du}{14.145327\du}}
\pgfpathlineto{\pgfpoint{1.212751\du}{14.150804\du}}
\pgfpathlineto{\pgfpoint{1.201433\du}{14.155550\du}}
\pgfpathlineto{\pgfpoint{1.189384\du}{14.160661\du}}
\pgfpathlineto{\pgfpoint{1.178066\du}{14.165773\du}}
\pgfpathlineto{\pgfpoint{1.166748\du}{14.171249\du}}
\pgfpathlineto{\pgfpoint{1.155430\du}{14.177091\du}}
\pgfpathlineto{\pgfpoint{1.144478\du}{14.182202\du}}
\pgfpathlineto{\pgfpoint{1.134255\du}{14.188044\du}}
\pgfpathlineto{\pgfpoint{1.124032\du}{14.194615\du}}
\pgfpathlineto{\pgfpoint{1.114175\du}{14.200457\du}}
\pgfpathlineto{\pgfpoint{1.104682\du}{14.207029\du}}
\pgfpathlineto{\pgfpoint{1.095555\du}{14.213235\du}}
\pgfpathlineto{\pgfpoint{1.085697\du}{14.219807\du}}
\pgfpathlineto{\pgfpoint{1.076570\du}{14.226379\du}}
\pgfpathlineto{\pgfpoint{1.068537\du}{14.233316\du}}
\pgfpathlineto{\pgfpoint{1.059410\du}{14.240253\du}}
\pgfpathlineto{\pgfpoint{1.052473\du}{14.247189\du}}
\pgfpathlineto{\pgfpoint{1.043711\du}{14.254126\du}}
\pgfpathlineto{\pgfpoint{1.036409\du}{14.261428\du}}
\pgfpathlineto{\pgfpoint{1.029472\du}{14.269095\du}}
\pgfpathlineto{\pgfpoint{1.022900\du}{14.276397\du}}
\pgfpathlineto{\pgfpoint{1.015963\du}{14.284064\du}}
\pgfpathlineto{\pgfpoint{1.009392\du}{14.291731\du}}
\pgfpathlineto{\pgfpoint{1.004646\du}{14.299763\du}}
\pgfpathlineto{\pgfpoint{0.998439\du}{14.307430\du}}
\pgfpathlineto{\pgfpoint{0.993328\du}{14.315828\du}}
\pgfpathlineto{\pgfpoint{0.988581\du}{14.323860\du}}
\pgfpathlineto{\pgfpoint{0.984200\du}{14.332257\du}}
\pgfpathlineto{\pgfpoint{0.979819\du}{14.340289\du}}
\pgfpathlineto{\pgfpoint{0.975803\du}{14.349051\du}}
\pgfpathlineto{\pgfpoint{0.972517\du}{14.357449\du}}
\pgfpathlineto{\pgfpoint{0.969231\du}{14.366211\du}}
\pgfpathlineto{\pgfpoint{0.967041\du}{14.374973\du}}
\pgfpathlineto{\pgfpoint{0.964485\du}{14.383736\du}}
\pgfpathlineto{\pgfpoint{0.963390\du}{14.392498\du}}
\pgfpathlineto{\pgfpoint{0.961199\du}{14.401260\du}}
\pgfpathlineto{\pgfpoint{0.960104\du}{14.410388\du}}
\pgfpathlineto{\pgfpoint{0.959374\du}{14.419880\du}}
\pgfpathlineto{\pgfpoint{0.959374\du}{14.428643\du}}
\pgfpathlineto{\pgfpoint{0.979819\du}{14.428643\du}}
\pgfpathlineto{\pgfpoint{0.979819\du}{14.420610\du}}
\pgfpathlineto{\pgfpoint{0.980184\du}{14.412213\du}}
\pgfpathlineto{\pgfpoint{0.981644\du}{14.404181\du}}
\pgfpathlineto{\pgfpoint{0.982740\du}{14.396514\du}}
\pgfpathlineto{\pgfpoint{0.984565\du}{14.388117\du}}
\pgfpathlineto{\pgfpoint{0.986391\du}{14.380085\du}}
\pgfpathlineto{\pgfpoint{0.988946\du}{14.372418\du}}
\pgfpathlineto{\pgfpoint{0.991867\du}{14.364751\du}}
\pgfpathlineto{\pgfpoint{0.995153\du}{14.357449\du}}
\pgfpathlineto{\pgfpoint{0.998439\du}{14.349051\du}}
\pgfpathlineto{\pgfpoint{1.002090\du}{14.341384\du}}
\pgfpathlineto{\pgfpoint{1.006471\du}{14.333717\du}}
\pgfpathlineto{\pgfpoint{1.010852\du}{14.326781\du}}
\pgfpathlineto{\pgfpoint{1.015598\du}{14.319113\du}}
\pgfpathlineto{\pgfpoint{1.020710\du}{14.311812\du}}
\pgfpathlineto{\pgfpoint{1.025821\du}{14.304510\du}}
\pgfpathlineto{\pgfpoint{1.032028\du}{14.296843\du}}
\pgfpathlineto{\pgfpoint{1.038234\du}{14.289906\du}}
\pgfpathlineto{\pgfpoint{1.044441\du}{14.282969\du}}
\pgfpathlineto{\pgfpoint{1.051378\du}{14.276032\du}}
\pgfpathlineto{\pgfpoint{1.057950\du}{14.269460\du}}
\pgfpathlineto{\pgfpoint{1.065617\du}{14.262523\du}}
\pgfpathlineto{\pgfpoint{1.072919\du}{14.255587\du}}
\pgfpathlineto{\pgfpoint{1.080951\du}{14.249015\du}}
\pgfpathlineto{\pgfpoint{1.089348\du}{14.242443\du}}
\pgfpathlineto{\pgfpoint{1.097745\du}{14.236237\du}}
\pgfpathlineto{\pgfpoint{1.106873\du}{14.230395\du}}
\pgfpathlineto{\pgfpoint{1.115270\du}{14.223823\du}}
\pgfpathlineto{\pgfpoint{1.125127\du}{14.217982\du}}
\pgfpathlineto{\pgfpoint{1.134620\du}{14.212140\du}}
\pgfpathlineto{\pgfpoint{1.144478\du}{14.206299\du}}
\pgfpathlineto{\pgfpoint{1.154700\du}{14.200457\du}}
\pgfpathlineto{\pgfpoint{1.164923\du}{14.195346\du}}
\pgfpathlineto{\pgfpoint{1.175876\du}{14.189504\du}}
\pgfpathlineto{\pgfpoint{1.186464\du}{14.184758\du}}
\pgfpathlineto{\pgfpoint{1.197782\du}{14.179281\du}}
\pgfpathlineto{\pgfpoint{1.208735\du}{14.174170\du}}
\pgfpathlineto{\pgfpoint{1.220418\du}{14.169424\du}}
\pgfpathlineto{\pgfpoint{1.231736\du}{14.164678\du}}
\pgfpathlineto{\pgfpoint{1.244149\du}{14.159931\du}}
\pgfpathlineto{\pgfpoint{1.256562\du}{14.155915\du}}
\pgfpathlineto{\pgfpoint{1.268610\du}{14.151169\du}}
\pgfpathlineto{\pgfpoint{1.281389\du}{14.147153\du}}
\pgfpathlineto{\pgfpoint{1.294532\du}{14.143867\du}}
\pgfpathlineto{\pgfpoint{1.307311\du}{14.139486\du}}
\pgfpathlineto{\pgfpoint{1.320819\du}{14.136200\du}}
\pgfpathlineto{\pgfpoint{1.333963\du}{14.132549\du}}
\pgfpathlineto{\pgfpoint{1.347471\du}{14.129628\du}}
\pgfpathlineto{\pgfpoint{1.361345\du}{14.126707\du}}
\pgfpathlineto{\pgfpoint{1.375584\du}{14.123787\du}}
\pgfpathlineto{\pgfpoint{1.389457\du}{14.120866\du}}
\pgfpathlineto{\pgfpoint{1.403331\du}{14.118675\du}}
\pgfpathlineto{\pgfpoint{1.417570\du}{14.116120\du}}
\pgfpathlineto{\pgfpoint{1.432174\du}{14.113929\du}}
\pgfpathlineto{\pgfpoint{1.446778\du}{14.112104\du}}
\pgfpathlineto{\pgfpoint{1.461747\du}{14.110278\du}}
\pgfpathlineto{\pgfpoint{1.476716\du}{14.108453\du}}
\pgfpathlineto{\pgfpoint{1.491685\du}{14.107357\du}}
\pgfpathlineto{\pgfpoint{1.506654\du}{14.106262\du}}
\pgfpathlineto{\pgfpoint{1.522353\du}{14.105167\du}}
\pgfpathlineto{\pgfpoint{1.536957\du}{14.104437\du}}
\pgfpathlineto{\pgfpoint{1.553021\du}{14.104071\du}}
\pgfpathlineto{\pgfpoint{1.568355\du}{14.103341\du}}
\pgfpathlineto{\pgfpoint{1.584054\du}{14.103341\du}}
\pgfpathlineto{\pgfpoint{1.584054\du}{14.103341\du}}
\pgfpathlineto{\pgfpoint{1.584054\du}{14.103341\du}}
\pgfpathlineto{\pgfpoint{1.585514\du}{14.103341\du}}
\pgfpathlineto{\pgfpoint{1.586610\du}{14.103341\du}}
\pgfpathlineto{\pgfpoint{1.587705\du}{14.103341\du}}
\pgfpathlineto{\pgfpoint{1.588800\du}{14.102611\du}}
\pgfpathlineto{\pgfpoint{1.589896\du}{14.102246\du}}
\pgfpathlineto{\pgfpoint{1.590991\du}{14.101516\du}}
\pgfpathlineto{\pgfpoint{1.590991\du}{14.101151\du}}
\pgfpathlineto{\pgfpoint{1.591721\du}{14.100421\du}}
\pgfpathlineto{\pgfpoint{1.592816\du}{14.098595\du}}
\pgfpathlineto{\pgfpoint{1.593912\du}{14.096770\du}}
\pgfpathlineto{\pgfpoint{1.594277\du}{14.095309\du}}
\pgfpathlineto{\pgfpoint{1.594277\du}{14.093484\du}}
\pgfpathlineto{\pgfpoint{1.594277\du}{14.090928\du}}
\pgfpathlineto{\pgfpoint{1.593912\du}{14.089468\du}}
\pgfpathlineto{\pgfpoint{1.592816\du}{14.087642\du}}
\pgfpathlineto{\pgfpoint{1.591721\du}{14.086547\du}}
\pgfpathlineto{\pgfpoint{1.590991\du}{14.085086\du}}
\pgfpathlineto{\pgfpoint{1.590991\du}{14.084721\du}}
\pgfpathlineto{\pgfpoint{1.589896\du}{14.083991\du}}
\pgfpathlineto{\pgfpoint{1.588800\du}{14.083991\du}}
\pgfpathlineto{\pgfpoint{1.587705\du}{14.083261\du}}
\pgfpathlineto{\pgfpoint{1.586610\du}{14.083261\du}}
\pgfpathlineto{\pgfpoint{1.585514\du}{14.082896\du}}
\pgfpathlineto{\pgfpoint{1.584054\du}{14.082896\du}}
\pgfusepath{fill}
\pgfsetbuttcap
\pgfsetmiterjoin
\pgfsetdash{}{0pt}
\definecolor{dialinecolor}{rgb}{0.678431, 0.839216, 0.905882}
\pgfsetfillcolor{dialinecolor}
\pgfpathmoveto{\pgfpoint{2.209100\du}{14.428643\du}}
\pgfpathlineto{\pgfpoint{2.209100\du}{14.419150\du}}
\pgfpathlineto{\pgfpoint{2.208369\du}{14.410388\du}}
\pgfpathlineto{\pgfpoint{2.207274\du}{14.401260\du}}
\pgfpathlineto{\pgfpoint{2.205814\du}{14.392498\du}}
\pgfpathlineto{\pgfpoint{2.203623\du}{14.383736\du}}
\pgfpathlineto{\pgfpoint{2.201798\du}{14.374973\du}}
\pgfpathlineto{\pgfpoint{2.198877\du}{14.366211\du}}
\pgfpathlineto{\pgfpoint{2.196321\du}{14.357449\du}}
\pgfpathlineto{\pgfpoint{2.192305\du}{14.349051\du}}
\pgfpathlineto{\pgfpoint{2.189019\du}{14.340289\du}}
\pgfpathlineto{\pgfpoint{2.184638\du}{14.332257\du}}
\pgfpathlineto{\pgfpoint{2.179892\du}{14.323860\du}}
\pgfpathlineto{\pgfpoint{2.175146\du}{14.315828\du}}
\pgfpathlineto{\pgfpoint{2.170399\du}{14.307430\du}}
\pgfpathlineto{\pgfpoint{2.164558\du}{14.299763\du}}
\pgfpathlineto{\pgfpoint{2.158351\du}{14.291731\du}}
\pgfpathlineto{\pgfpoint{2.152145\du}{14.284064\du}}
\pgfpathlineto{\pgfpoint{2.145573\du}{14.276397\du}}
\pgfpathlineto{\pgfpoint{2.139001\du}{14.269095\du}}
\pgfpathlineto{\pgfpoint{2.131334\du}{14.261428\du}}
\pgfpathlineto{\pgfpoint{2.124397\du}{14.254126\du}}
\pgfpathlineto{\pgfpoint{2.116365\du}{14.247189\du}}
\pgfpathlineto{\pgfpoint{2.108333\du}{14.240253\du}}
\pgfpathlineto{\pgfpoint{2.099936\du}{14.233316\du}}
\pgfpathlineto{\pgfpoint{2.091539\du}{14.226379\du}}
\pgfpathlineto{\pgfpoint{2.082776\du}{14.219807\du}}
\pgfpathlineto{\pgfpoint{2.073649\du}{14.213235\du}}
\pgfpathlineto{\pgfpoint{2.064156\du}{14.207029\du}}
\pgfpathlineto{\pgfpoint{2.053934\du}{14.200457\du}}
\pgfpathlineto{\pgfpoint{2.044076\du}{14.194615\du}}
\pgfpathlineto{\pgfpoint{2.033853\du}{14.188044\du}}
\pgfpathlineto{\pgfpoint{2.023996\du}{14.182202\du}}
\pgfpathlineto{\pgfpoint{2.012678\du}{14.177091\du}}
\pgfpathlineto{\pgfpoint{2.001725\du}{14.171249\du}}
\pgfpathlineto{\pgfpoint{1.990407\du}{14.165773\du}}
\pgfpathlineto{\pgfpoint{1.979089\du}{14.160661\du}}
\pgfpathlineto{\pgfpoint{1.967406\du}{14.155550\du}}
\pgfpathlineto{\pgfpoint{1.956088\du}{14.150804\du}}
\pgfpathlineto{\pgfpoint{1.943309\du}{14.145327\du}}
\pgfpathlineto{\pgfpoint{1.931626\du}{14.140946\du}}
\pgfpathlineto{\pgfpoint{1.918848\du}{14.136565\du}}
\pgfpathlineto{\pgfpoint{1.906069\du}{14.132184\du}}
\pgfpathlineto{\pgfpoint{1.892926\du}{14.127803\du}}
\pgfpathlineto{\pgfpoint{1.879782\du}{14.123787\du}}
\pgfpathlineto{\pgfpoint{1.866639\du}{14.120136\du}}
\pgfpathlineto{\pgfpoint{1.853130\du}{14.116120\du}}
\pgfpathlineto{\pgfpoint{1.839257\du}{14.112834\du}}
\pgfpathlineto{\pgfpoint{1.825748\du}{14.109183\du}}
\pgfpathlineto{\pgfpoint{1.811874\du}{14.106262\du}}
\pgfpathlineto{\pgfpoint{1.797271\du}{14.103341\du}}
\pgfpathlineto{\pgfpoint{1.783397\du}{14.101151\du}}
\pgfpathlineto{\pgfpoint{1.769158\du}{14.098230\du}}
\pgfpathlineto{\pgfpoint{1.753824\du}{14.095674\du}}
\pgfpathlineto{\pgfpoint{1.738855\du}{14.093484\du}}
\pgfpathlineto{\pgfpoint{1.723886\du}{14.091658\du}}
\pgfpathlineto{\pgfpoint{1.709647\du}{14.089833\du}}
\pgfpathlineto{\pgfpoint{1.693583\du}{14.088007\du}}
\pgfpathlineto{\pgfpoint{1.678614\du}{14.086912\du}}
\pgfpathlineto{\pgfpoint{1.663280\du}{14.085817\du}}
\pgfpathlineto{\pgfpoint{1.647581\du}{14.084721\du}}
\pgfpathlineto{\pgfpoint{1.632247\du}{14.083991\du}}
\pgfpathlineto{\pgfpoint{1.616183\du}{14.083261\du}}
\pgfpathlineto{\pgfpoint{1.600118\du}{14.082896\du}}
\pgfpathlineto{\pgfpoint{1.584054\du}{14.082896\du}}
\pgfpathlineto{\pgfpoint{1.584054\du}{14.103341\du}}
\pgfpathlineto{\pgfpoint{1.600118\du}{14.103341\du}}
\pgfpathlineto{\pgfpoint{1.615452\du}{14.104071\du}}
\pgfpathlineto{\pgfpoint{1.631152\du}{14.104437\du}}
\pgfpathlineto{\pgfpoint{1.646486\du}{14.105167\du}}
\pgfpathlineto{\pgfpoint{1.661455\du}{14.106262\du}}
\pgfpathlineto{\pgfpoint{1.677154\du}{14.107357\du}}
\pgfpathlineto{\pgfpoint{1.691758\du}{14.108453\du}}
\pgfpathlineto{\pgfpoint{1.707092\du}{14.110278\du}}
\pgfpathlineto{\pgfpoint{1.722061\du}{14.112104\du}}
\pgfpathlineto{\pgfpoint{1.736664\du}{14.113929\du}}
\pgfpathlineto{\pgfpoint{1.750903\du}{14.116120\du}}
\pgfpathlineto{\pgfpoint{1.765142\du}{14.118675\du}}
\pgfpathlineto{\pgfpoint{1.779381\du}{14.120866\du}}
\pgfpathlineto{\pgfpoint{1.793254\du}{14.123787\du}}
\pgfpathlineto{\pgfpoint{1.807128\du}{14.126707\du}}
\pgfpathlineto{\pgfpoint{1.821002\du}{14.129628\du}}
\pgfpathlineto{\pgfpoint{1.834510\du}{14.132549\du}}
\pgfpathlineto{\pgfpoint{1.848019\du}{14.136200\du}}
\pgfpathlineto{\pgfpoint{1.861162\du}{14.139486\du}}
\pgfpathlineto{\pgfpoint{1.873941\du}{14.143867\du}}
\pgfpathlineto{\pgfpoint{1.887449\du}{14.147153\du}}
\pgfpathlineto{\pgfpoint{1.900228\du}{14.151169\du}}
\pgfpathlineto{\pgfpoint{1.911911\du}{14.155915\du}}
\pgfpathlineto{\pgfpoint{1.924324\du}{14.159931\du}}
\pgfpathlineto{\pgfpoint{1.936372\du}{14.164678\du}}
\pgfpathlineto{\pgfpoint{1.948056\du}{14.169424\du}}
\pgfpathlineto{\pgfpoint{1.959374\du}{14.174170\du}}
\pgfpathlineto{\pgfpoint{1.970692\du}{14.179281\du}}
\pgfpathlineto{\pgfpoint{1.982375\du}{14.184758\du}}
\pgfpathlineto{\pgfpoint{1.992962\du}{14.189504\du}}
\pgfpathlineto{\pgfpoint{2.003550\du}{14.195346\du}}
\pgfpathlineto{\pgfpoint{2.014138\du}{14.200457\du}}
\pgfpathlineto{\pgfpoint{2.023996\du}{14.206299\du}}
\pgfpathlineto{\pgfpoint{2.033853\du}{14.212140\du}}
\pgfpathlineto{\pgfpoint{2.042981\du}{14.217982\du}}
\pgfpathlineto{\pgfpoint{2.052108\du}{14.223823\du}}
\pgfpathlineto{\pgfpoint{2.061601\du}{14.230395\du}}
\pgfpathlineto{\pgfpoint{2.070363\du}{14.236237\du}}
\pgfpathlineto{\pgfpoint{2.079490\du}{14.242443\du}}
\pgfpathlineto{\pgfpoint{2.087522\du}{14.249015\du}}
\pgfpathlineto{\pgfpoint{2.095920\du}{14.255587\du}}
\pgfpathlineto{\pgfpoint{2.102857\du}{14.262523\du}}
\pgfpathlineto{\pgfpoint{2.110158\du}{14.269460\du}}
\pgfpathlineto{\pgfpoint{2.117095\du}{14.276032\du}}
\pgfpathlineto{\pgfpoint{2.124397\du}{14.282969\du}}
\pgfpathlineto{\pgfpoint{2.130239\du}{14.289906\du}}
\pgfpathlineto{\pgfpoint{2.137176\du}{14.296843\du}}
\pgfpathlineto{\pgfpoint{2.142287\du}{14.304510\du}}
\pgfpathlineto{\pgfpoint{2.147763\du}{14.311812\du}}
\pgfpathlineto{\pgfpoint{2.152510\du}{14.319113\du}}
\pgfpathlineto{\pgfpoint{2.157621\du}{14.326781\du}}
\pgfpathlineto{\pgfpoint{2.161637\du}{14.333717\du}}
\pgfpathlineto{\pgfpoint{2.166383\du}{14.341384\du}}
\pgfpathlineto{\pgfpoint{2.170399\du}{14.349051\du}}
\pgfpathlineto{\pgfpoint{2.173685\du}{14.357449\du}}
\pgfpathlineto{\pgfpoint{2.176241\du}{14.364751\du}}
\pgfpathlineto{\pgfpoint{2.179527\du}{14.372418\du}}
\pgfpathlineto{\pgfpoint{2.181717\du}{14.380085\du}}
\pgfpathlineto{\pgfpoint{2.184273\du}{14.388117\du}}
\pgfpathlineto{\pgfpoint{2.185733\du}{14.396514\du}}
\pgfpathlineto{\pgfpoint{2.187194\du}{14.404181\du}}
\pgfpathlineto{\pgfpoint{2.187924\du}{14.412213\du}}
\pgfpathlineto{\pgfpoint{2.188289\du}{14.420610\du}}
\pgfpathlineto{\pgfpoint{2.189019\du}{14.428643\du}}
\pgfpathlineto{\pgfpoint{2.209100\du}{14.428643\du}}
\pgfusepath{fill}
\pgfsetbuttcap
\pgfsetmiterjoin
\pgfsetdash{}{0pt}
\definecolor{dialinecolor}{rgb}{0.074510, 0.082353, 0.086275}
\pgfsetfillcolor{dialinecolor}
\pgfpathmoveto{\pgfpoint{1.263134\du}{14.522837\du}}
\pgfpathlineto{\pgfpoint{1.490589\du}{14.294652\du}}
\pgfpathlineto{\pgfpoint{1.430713\du}{14.233681\du}}
\pgfpathlineto{\pgfpoint{1.611071\du}{14.233681\du}}
\pgfpathlineto{\pgfpoint{1.611071\du}{14.422071\du}}
\pgfpathlineto{\pgfpoint{1.550830\du}{14.361830\du}}
\pgfpathlineto{\pgfpoint{1.330677\du}{14.583078\du}}
\pgfpathlineto{\pgfpoint{1.263134\du}{14.522837\du}}
\pgfusepath{fill}
\pgfsetbuttcap
\pgfsetmiterjoin
\pgfsetdash{}{0pt}
\definecolor{dialinecolor}{rgb}{0.074510, 0.082353, 0.086275}
\pgfsetfillcolor{dialinecolor}
\pgfpathmoveto{\pgfpoint{1.531115\du}{14.643319\du}}
\pgfpathlineto{\pgfpoint{1.758205\du}{14.415134\du}}
\pgfpathlineto{\pgfpoint{1.697599\du}{14.354893\du}}
\pgfpathlineto{\pgfpoint{1.878687\du}{14.354893\du}}
\pgfpathlineto{\pgfpoint{1.878687\du}{14.542918\du}}
\pgfpathlineto{\pgfpoint{1.818081\du}{14.482677\du}}
\pgfpathlineto{\pgfpoint{1.597563\du}{14.703560\du}}
\pgfpathlineto{\pgfpoint{1.531115\du}{14.643319\du}}
\pgfusepath{fill}
\pgfsetbuttcap
\pgfsetmiterjoin
\pgfsetdash{}{0pt}
\definecolor{dialinecolor}{rgb}{1.000000, 1.000000, 1.000000}
\pgfsetfillcolor{dialinecolor}
\pgfpathmoveto{\pgfpoint{1.249991\du}{14.509329\du}}
\pgfpathlineto{\pgfpoint{1.477081\du}{14.281143\du}}
\pgfpathlineto{\pgfpoint{1.417570\du}{14.220902\du}}
\pgfpathlineto{\pgfpoint{1.597563\du}{14.220902\du}}
\pgfpathlineto{\pgfpoint{1.597563\du}{14.408927\du}}
\pgfpathlineto{\pgfpoint{1.537687\du}{14.347956\du}}
\pgfpathlineto{\pgfpoint{1.317168\du}{14.569570\du}}
\pgfpathlineto{\pgfpoint{1.249991\du}{14.509329\du}}
\pgfusepath{fill}
\pgfsetbuttcap
\pgfsetmiterjoin
\pgfsetdash{}{0pt}
\definecolor{dialinecolor}{rgb}{1.000000, 1.000000, 1.000000}
\pgfsetfillcolor{dialinecolor}
\pgfpathmoveto{\pgfpoint{1.517606\du}{14.629811\du}}
\pgfpathlineto{\pgfpoint{1.744697\du}{14.401625\du}}
\pgfpathlineto{\pgfpoint{1.684456\du}{14.341384\du}}
\pgfpathlineto{\pgfpoint{1.864813\du}{14.341384\du}}
\pgfpathlineto{\pgfpoint{1.864813\du}{14.529409\du}}
\pgfpathlineto{\pgfpoint{1.805303\du}{14.469168\du}}
\pgfpathlineto{\pgfpoint{1.584054\du}{14.690052\du}}
\pgfpathlineto{\pgfpoint{1.517606\du}{14.629811\du}}
\pgfusepath{fill}
\pgfsetlinewidth{0.000000\du}
\pgfsetdash{}{0pt}
\pgfsetdash{}{0pt}
\pgfsetbuttcap
\pgfsetmiterjoin
\pgfsetlinewidth{0.000000\du}
\pgfsetbuttcap
\pgfsetmiterjoin
\pgfsetdash{}{0pt}
\definecolor{dialinecolor}{rgb}{0.788235, 0.788235, 0.713726}
\pgfsetfillcolor{dialinecolor}
\pgfpathmoveto{\pgfpoint{-16.963480\du}{13.162469\du}}
\pgfpathlineto{\pgfpoint{-16.538420\du}{12.763433\du}}
\pgfpathlineto{\pgfpoint{-14.112878\du}{12.763433\du}}
\pgfpathlineto{\pgfpoint{-14.537456\du}{13.162469\du}}
\pgfpathlineto{\pgfpoint{-16.963480\du}{13.162469\du}}
\pgfusepath{fill}
\pgfsetbuttcap
\pgfsetmiterjoin
\pgfsetdash{}{0pt}
\definecolor{dialinecolor}{rgb}{0.286275, 0.286275, 0.211765}
\pgfsetstrokecolor{dialinecolor}
\pgfpathmoveto{\pgfpoint{-16.963480\du}{13.162469\du}}
\pgfpathlineto{\pgfpoint{-16.538420\du}{12.763433\du}}
\pgfpathlineto{\pgfpoint{-14.112878\du}{12.763433\du}}
\pgfpathlineto{\pgfpoint{-14.537456\du}{13.162469\du}}
\pgfpathlineto{\pgfpoint{-16.963480\du}{13.162469\du}}
\pgfusepath{stroke}
\pgfsetbuttcap
\pgfsetmiterjoin
\pgfsetdash{}{0pt}
\definecolor{dialinecolor}{rgb}{0.717647, 0.717647, 0.615686}
\pgfsetfillcolor{dialinecolor}
\pgfpathmoveto{\pgfpoint{-16.963480\du}{13.162469\du}}
\pgfpathlineto{\pgfpoint{-14.537456\du}{13.162469\du}}
\pgfpathlineto{\pgfpoint{-14.537456\du}{14.763433\du}}
\pgfpathlineto{\pgfpoint{-16.963480\du}{14.763433\du}}
\pgfpathlineto{\pgfpoint{-16.963480\du}{13.162469\du}}
\pgfusepath{fill}
\pgfsetbuttcap
\pgfsetmiterjoin
\pgfsetdash{}{0pt}
\definecolor{dialinecolor}{rgb}{0.286275, 0.286275, 0.211765}
\pgfsetstrokecolor{dialinecolor}
\pgfpathmoveto{\pgfpoint{-16.963480\du}{13.162469\du}}
\pgfpathlineto{\pgfpoint{-14.537456\du}{13.162469\du}}
\pgfpathlineto{\pgfpoint{-14.537456\du}{14.763433\du}}
\pgfpathlineto{\pgfpoint{-16.963480\du}{14.763433\du}}
\pgfpathlineto{\pgfpoint{-16.963480\du}{13.162469\du}}
\pgfusepath{stroke}
\pgfsetbuttcap
\pgfsetmiterjoin
\pgfsetdash{}{0pt}
\definecolor{dialinecolor}{rgb}{0.478431, 0.478431, 0.352941}
\pgfsetfillcolor{dialinecolor}
\pgfpathmoveto{\pgfpoint{-14.537456\du}{14.763433\du}}
\pgfpathlineto{\pgfpoint{-14.112878\du}{14.337891\du}}
\pgfpathlineto{\pgfpoint{-14.112878\du}{12.763433\du}}
\pgfpathlineto{\pgfpoint{-14.537456\du}{13.162469\du}}
\pgfpathlineto{\pgfpoint{-14.537456\du}{14.763433\du}}
\pgfusepath{fill}
\pgfsetbuttcap
\pgfsetmiterjoin
\pgfsetdash{}{0pt}
\definecolor{dialinecolor}{rgb}{0.286275, 0.286275, 0.211765}
\pgfsetstrokecolor{dialinecolor}
\pgfpathmoveto{\pgfpoint{-14.537456\du}{14.763433\du}}
\pgfpathlineto{\pgfpoint{-14.112878\du}{14.337891\du}}
\pgfpathlineto{\pgfpoint{-14.112878\du}{12.763433\du}}
\pgfpathlineto{\pgfpoint{-14.537456\du}{13.162469\du}}
\pgfpathlineto{\pgfpoint{-14.537456\du}{14.763433\du}}
\pgfusepath{stroke}
\pgfsetbuttcap
\pgfsetmiterjoin
\pgfsetdash{}{0pt}
\definecolor{dialinecolor}{rgb}{0.478431, 0.478431, 0.352941}
\pgfsetfillcolor{dialinecolor}
\pgfpathmoveto{\pgfpoint{-16.413601\du}{13.462710\du}}
\pgfpathlineto{\pgfpoint{-15.087336\du}{13.462710\du}}
\pgfpathlineto{\pgfpoint{-15.087336\du}{14.487770\du}}
\pgfpathlineto{\pgfpoint{-16.413601\du}{14.487770\du}}
\pgfpathlineto{\pgfpoint{-16.413601\du}{13.462710\du}}
\pgfusepath{fill}
\pgfsetbuttcap
\pgfsetmiterjoin
\pgfsetdash{}{0pt}
\definecolor{dialinecolor}{rgb}{0.286275, 0.286275, 0.211765}
\pgfsetstrokecolor{dialinecolor}
\pgfpathmoveto{\pgfpoint{-16.413601\du}{13.462710\du}}
\pgfpathlineto{\pgfpoint{-15.087336\du}{13.462710\du}}
\pgfpathlineto{\pgfpoint{-15.087336\du}{14.487770\du}}
\pgfpathlineto{\pgfpoint{-16.413601\du}{14.487770\du}}
\pgfpathlineto{\pgfpoint{-16.413601\du}{13.462710\du}}
\pgfusepath{stroke}
\pgfsetbuttcap
\pgfsetmiterjoin
\pgfsetdash{}{0pt}
\definecolor{dialinecolor}{rgb}{0.925490, 0.925490, 0.905882}
\pgfsetstrokecolor{dialinecolor}
\pgfpathmoveto{\pgfpoint{-16.413601\du}{14.487770\du}}
\pgfpathlineto{\pgfpoint{-16.413601\du}{13.462710\du}}
\pgfpathlineto{\pgfpoint{-15.087336\du}{13.462710\du}}
\pgfusepath{stroke}
\pgfsetlinewidth{0.000000\du}
\pgfsetbuttcap
\pgfsetmiterjoin
\pgfsetdash{}{0pt}
\definecolor{dialinecolor}{rgb}{0.858824, 0.858824, 0.807843}
\pgfsetstrokecolor{dialinecolor}
\pgfpathmoveto{\pgfpoint{-15.313360\du}{13.737891\du}}
\pgfpathlineto{\pgfpoint{-16.213119\du}{13.737891\du}}
\pgfusepath{stroke}
\pgfsetbuttcap
\pgfsetmiterjoin
\pgfsetdash{}{0pt}
\definecolor{dialinecolor}{rgb}{0.000000, 0.000000, 0.000000}
\pgfsetstrokecolor{dialinecolor}
\pgfpathmoveto{\pgfpoint{-15.313360\du}{13.987529\du}}
\pgfpathlineto{\pgfpoint{-16.213119\du}{13.987529\du}}
\pgfusepath{stroke}
\pgfsetlinewidth{0.000000\du}
\pgfsetbuttcap
\pgfsetmiterjoin
\pgfsetdash{}{0pt}
\definecolor{dialinecolor}{rgb}{0.000000, 0.000000, 0.000000}
\pgfsetstrokecolor{dialinecolor}
\pgfpathmoveto{\pgfpoint{-15.313360\du}{14.237650\du}}
\pgfpathlineto{\pgfpoint{-16.213119\du}{14.237650\du}}
\pgfusepath{stroke}
\pgfsetbuttcap
\pgfsetmiterjoin
\pgfsetdash{}{0pt}
\definecolor{dialinecolor}{rgb}{0.286275, 0.286275, 0.211765}
\pgfsetstrokecolor{dialinecolor}
\pgfpathmoveto{\pgfpoint{-15.288300\du}{13.762469\du}}
\pgfpathlineto{\pgfpoint{-16.187577\du}{13.762469\du}}
\pgfusepath{stroke}
\pgfsetbuttcap
\pgfsetmiterjoin
\pgfsetdash{}{0pt}
\definecolor{dialinecolor}{rgb}{0.000000, 0.000000, 0.000000}
\pgfsetstrokecolor{dialinecolor}
\pgfpathmoveto{\pgfpoint{-15.288300\du}{14.013553\du}}
\pgfpathlineto{\pgfpoint{-16.187577\du}{14.013553\du}}
\pgfusepath{stroke}
\pgfsetbuttcap
\pgfsetmiterjoin
\pgfsetdash{}{0pt}
\definecolor{dialinecolor}{rgb}{0.000000, 0.000000, 0.000000}
\pgfsetstrokecolor{dialinecolor}
\pgfpathmoveto{\pgfpoint{-15.288300\du}{14.262228\du}}
\pgfpathlineto{\pgfpoint{-16.187577\du}{14.262228\du}}
\pgfusepath{stroke}
\pgfsetlinewidth{0.000000\du}
\pgfsetdash{}{0pt}
\pgfsetdash{}{0pt}
\pgfsetbuttcap
\pgfsetmiterjoin
\pgfsetlinewidth{0.000000\du}
\pgfsetbuttcap
\pgfsetmiterjoin
\pgfsetdash{}{0pt}
\definecolor{dialinecolor}{rgb}{0.717647, 0.717647, 0.615686}
\pgfsetfillcolor{dialinecolor}
\pgfpathmoveto{\pgfpoint{59.829144\du}{12.132258\du}}
\pgfpathlineto{\pgfpoint{59.829144\du}{14.514066\du}}
\pgfpathlineto{\pgfpoint{61.106794\du}{14.514066\du}}
\pgfpathlineto{\pgfpoint{61.106794\du}{12.132258\du}}
\pgfpathlineto{\pgfpoint{59.829144\du}{12.132258\du}}
\pgfusepath{fill}
\pgfsetbuttcap
\pgfsetmiterjoin
\pgfsetdash{}{0pt}
\definecolor{dialinecolor}{rgb}{0.286275, 0.286275, 0.211765}
\pgfsetstrokecolor{dialinecolor}
\pgfpathmoveto{\pgfpoint{59.829144\du}{12.132258\du}}
\pgfpathlineto{\pgfpoint{59.829144\du}{14.514066\du}}
\pgfpathlineto{\pgfpoint{61.106794\du}{14.514066\du}}
\pgfpathlineto{\pgfpoint{61.106794\du}{12.132258\du}}
\pgfpathlineto{\pgfpoint{59.829144\du}{12.132258\du}}
\pgfusepath{stroke}
\pgfsetbuttcap
\pgfsetmiterjoin
\pgfsetdash{}{0pt}
\definecolor{dialinecolor}{rgb}{0.000000, 0.000000, 0.000000}
\pgfsetfillcolor{dialinecolor}
\pgfpathmoveto{\pgfpoint{60.239337\du}{12.400564\du}}
\pgfpathlineto{\pgfpoint{60.239337\du}{12.742504\du}}
\pgfpathlineto{\pgfpoint{60.738965\du}{12.742504\du}}
\pgfpathlineto{\pgfpoint{60.738965\du}{12.400564\du}}
\pgfpathlineto{\pgfpoint{60.239337\du}{12.400564\du}}
\pgfusepath{fill}
\pgfsetbuttcap
\pgfsetmiterjoin
\pgfsetdash{}{0pt}
\definecolor{dialinecolor}{rgb}{0.000000, 0.000000, 0.000000}
\pgfsetfillcolor{dialinecolor}
\pgfpathmoveto{\pgfpoint{60.180498\du}{12.987947\du}}
\pgfpathlineto{\pgfpoint{60.180498\du}{12.983240\du}}
\pgfpathlineto{\pgfpoint{60.180161\du}{12.979541\du}}
\pgfpathlineto{\pgfpoint{60.179825\du}{12.975507\du}}
\pgfpathlineto{\pgfpoint{60.178817\du}{12.971472\du}}
\pgfpathlineto{\pgfpoint{60.178144\du}{12.967774\du}}
\pgfpathlineto{\pgfpoint{60.176799\du}{12.963403\du}}
\pgfpathlineto{\pgfpoint{60.175791\du}{12.959704\du}}
\pgfpathlineto{\pgfpoint{60.173773\du}{12.955670\du}}
\pgfpathlineto{\pgfpoint{60.172092\du}{12.951635\du}}
\pgfpathlineto{\pgfpoint{60.170411\du}{12.948273\du}}
\pgfpathlineto{\pgfpoint{60.168394\du}{12.944910\du}}
\pgfpathlineto{\pgfpoint{60.166376\du}{12.941212\du}}
\pgfpathlineto{\pgfpoint{60.163686\du}{12.938186\du}}
\pgfpathlineto{\pgfpoint{60.161333\du}{12.934824\du}}
\pgfpathlineto{\pgfpoint{60.158307\du}{12.931798\du}}
\pgfpathlineto{\pgfpoint{60.155281\du}{12.928772\du}}
\pgfpathlineto{\pgfpoint{60.152591\du}{12.926418\du}}
\pgfpathlineto{\pgfpoint{60.149565\du}{12.923392\du}}
\pgfpathlineto{\pgfpoint{60.146203\du}{12.921039\du}}
\pgfpathlineto{\pgfpoint{60.142841\du}{12.918685\du}}
\pgfpathlineto{\pgfpoint{60.138806\du}{12.916668\du}}
\pgfpathlineto{\pgfpoint{60.135780\du}{12.914986\du}}
\pgfpathlineto{\pgfpoint{60.131745\du}{12.913305\du}}
\pgfpathlineto{\pgfpoint{60.128047\du}{12.911624\du}}
\pgfpathlineto{\pgfpoint{60.123676\du}{12.910279\du}}
\pgfpathlineto{\pgfpoint{60.119977\du}{12.908934\du}}
\pgfpathlineto{\pgfpoint{60.116279\du}{12.908262\du}}
\pgfpathlineto{\pgfpoint{60.111908\du}{12.907253\du}}
\pgfpathlineto{\pgfpoint{60.107873\du}{12.906917\du}}
\pgfpathlineto{\pgfpoint{60.104175\du}{12.906581\du}}
\pgfpathlineto{\pgfpoint{60.100140\du}{12.906581\du}}
\pgfpathlineto{\pgfpoint{60.100140\du}{12.906581\du}}
\pgfpathlineto{\pgfpoint{60.095769\du}{12.906581\du}}
\pgfpathlineto{\pgfpoint{60.091735\du}{12.906917\du}}
\pgfpathlineto{\pgfpoint{60.087700\du}{12.907253\du}}
\pgfpathlineto{\pgfpoint{60.083329\du}{12.908262\du}}
\pgfpathlineto{\pgfpoint{60.079631\du}{12.908934\du}}
\pgfpathlineto{\pgfpoint{60.075932\du}{12.910279\du}}
\pgfpathlineto{\pgfpoint{60.071561\du}{12.911624\du}}
\pgfpathlineto{\pgfpoint{60.067863\du}{12.913305\du}}
\pgfpathlineto{\pgfpoint{60.064501\du}{12.914986\du}}
\pgfpathlineto{\pgfpoint{60.060802\du}{12.916668\du}}
\pgfpathlineto{\pgfpoint{60.056767\du}{12.918685\du}}
\pgfpathlineto{\pgfpoint{60.053741\du}{12.921039\du}}
\pgfpathlineto{\pgfpoint{60.050043\du}{12.923392\du}}
\pgfpathlineto{\pgfpoint{60.047689\du}{12.926418\du}}
\pgfpathlineto{\pgfpoint{60.044327\du}{12.928772\du}}
\pgfpathlineto{\pgfpoint{60.041301\du}{12.931798\du}}
\pgfpathlineto{\pgfpoint{60.038275\du}{12.934824\du}}
\pgfpathlineto{\pgfpoint{60.035922\du}{12.938186\du}}
\pgfpathlineto{\pgfpoint{60.033232\du}{12.941212\du}}
\pgfpathlineto{\pgfpoint{60.031214\du}{12.944910\du}}
\pgfpathlineto{\pgfpoint{60.029197\du}{12.948273\du}}
\pgfpathlineto{\pgfpoint{60.027516\du}{12.951635\du}}
\pgfpathlineto{\pgfpoint{60.025835\du}{12.955670\du}}
\pgfpathlineto{\pgfpoint{60.024490\du}{12.959704\du}}
\pgfpathlineto{\pgfpoint{60.022809\du}{12.963403\du}}
\pgfpathlineto{\pgfpoint{60.021464\du}{12.967774\du}}
\pgfpathlineto{\pgfpoint{60.020791\du}{12.971472\du}}
\pgfpathlineto{\pgfpoint{60.020119\du}{12.975507\du}}
\pgfpathlineto{\pgfpoint{60.019447\du}{12.979541\du}}
\pgfpathlineto{\pgfpoint{60.019110\du}{12.983240\du}}
\pgfpathlineto{\pgfpoint{60.019110\du}{12.987947\du}}
\pgfpathlineto{\pgfpoint{60.019110\du}{12.987947\du}}
\pgfpathlineto{\pgfpoint{60.019110\du}{12.991645\du}}
\pgfpathlineto{\pgfpoint{60.019447\du}{12.996353\du}}
\pgfpathlineto{\pgfpoint{60.020119\du}{13.000051\du}}
\pgfpathlineto{\pgfpoint{60.020791\du}{13.004422\du}}
\pgfpathlineto{\pgfpoint{60.021464\du}{13.008120\du}}
\pgfpathlineto{\pgfpoint{60.022809\du}{13.012491\du}}
\pgfpathlineto{\pgfpoint{60.024490\du}{13.016190\du}}
\pgfpathlineto{\pgfpoint{60.025835\du}{13.019888\du}}
\pgfpathlineto{\pgfpoint{60.027516\du}{13.023923\du}}
\pgfpathlineto{\pgfpoint{60.029197\du}{13.027285\du}}
\pgfpathlineto{\pgfpoint{60.031214\du}{13.030984\du}}
\pgfpathlineto{\pgfpoint{60.033232\du}{13.034346\du}}
\pgfpathlineto{\pgfpoint{60.035922\du}{13.037708\du}}
\pgfpathlineto{\pgfpoint{60.038275\du}{13.041070\du}}
\pgfpathlineto{\pgfpoint{60.041301\du}{13.044096\du}}
\pgfpathlineto{\pgfpoint{60.044327\du}{13.047122\du}}
\pgfpathlineto{\pgfpoint{60.047689\du}{13.049476\du}}
\pgfpathlineto{\pgfpoint{60.050043\du}{13.052502\du}}
\pgfpathlineto{\pgfpoint{60.053741\du}{13.054519\du}}
\pgfpathlineto{\pgfpoint{60.056767\du}{13.056873\du}}
\pgfpathlineto{\pgfpoint{60.060802\du}{13.058890\du}}
\pgfpathlineto{\pgfpoint{60.064501\du}{13.060907\du}}
\pgfpathlineto{\pgfpoint{60.067863\du}{13.062589\du}}
\pgfpathlineto{\pgfpoint{60.071561\du}{13.064270\du}}
\pgfpathlineto{\pgfpoint{60.075932\du}{13.065278\du}}
\pgfpathlineto{\pgfpoint{60.079631\du}{13.066960\du}}
\pgfpathlineto{\pgfpoint{60.083329\du}{13.067296\du}}
\pgfpathlineto{\pgfpoint{60.087700\du}{13.068641\du}}
\pgfpathlineto{\pgfpoint{60.091735\du}{13.068977\du}}
\pgfpathlineto{\pgfpoint{60.095769\du}{13.069313\du}}
\pgfpathlineto{\pgfpoint{60.100140\du}{13.069313\du}}
\pgfpathlineto{\pgfpoint{60.100140\du}{13.069313\du}}
\pgfpathlineto{\pgfpoint{60.104175\du}{13.069313\du}}
\pgfpathlineto{\pgfpoint{60.107873\du}{13.068977\du}}
\pgfpathlineto{\pgfpoint{60.111908\du}{13.068641\du}}
\pgfpathlineto{\pgfpoint{60.116279\du}{13.067296\du}}
\pgfpathlineto{\pgfpoint{60.119977\du}{13.066960\du}}
\pgfpathlineto{\pgfpoint{60.123676\du}{13.065278\du}}
\pgfpathlineto{\pgfpoint{60.128047\du}{13.064270\du}}
\pgfpathlineto{\pgfpoint{60.131745\du}{13.062589\du}}
\pgfpathlineto{\pgfpoint{60.135780\du}{13.060907\du}}
\pgfpathlineto{\pgfpoint{60.138806\du}{13.058890\du}}
\pgfpathlineto{\pgfpoint{60.142841\du}{13.056873\du}}
\pgfpathlineto{\pgfpoint{60.146203\du}{13.054519\du}}
\pgfpathlineto{\pgfpoint{60.149565\du}{13.052502\du}}
\pgfpathlineto{\pgfpoint{60.152591\du}{13.049476\du}}
\pgfpathlineto{\pgfpoint{60.155281\du}{13.047122\du}}
\pgfpathlineto{\pgfpoint{60.158307\du}{13.044096\du}}
\pgfpathlineto{\pgfpoint{60.161333\du}{13.041070\du}}
\pgfpathlineto{\pgfpoint{60.163686\du}{13.037708\du}}
\pgfpathlineto{\pgfpoint{60.166376\du}{13.034346\du}}
\pgfpathlineto{\pgfpoint{60.168394\du}{13.030984\du}}
\pgfpathlineto{\pgfpoint{60.170411\du}{13.027285\du}}
\pgfpathlineto{\pgfpoint{60.172092\du}{13.023923\du}}
\pgfpathlineto{\pgfpoint{60.173773\du}{13.019888\du}}
\pgfpathlineto{\pgfpoint{60.175791\du}{13.016190\du}}
\pgfpathlineto{\pgfpoint{60.176799\du}{13.012491\du}}
\pgfpathlineto{\pgfpoint{60.178144\du}{13.008120\du}}
\pgfpathlineto{\pgfpoint{60.178817\du}{13.004422\du}}
\pgfpathlineto{\pgfpoint{60.179825\du}{13.000051\du}}
\pgfpathlineto{\pgfpoint{60.180161\du}{12.996353\du}}
\pgfpathlineto{\pgfpoint{60.180498\du}{12.991645\du}}
\pgfpathlineto{\pgfpoint{60.180498\du}{12.987947\du}}
\pgfusepath{fill}
\pgfsetbuttcap
\pgfsetmiterjoin
\pgfsetdash{}{0pt}
\definecolor{dialinecolor}{rgb}{0.000000, 0.000000, 0.000000}
\pgfsetfillcolor{dialinecolor}
\pgfpathmoveto{\pgfpoint{60.914810\du}{12.987947\du}}
\pgfpathlineto{\pgfpoint{60.914810\du}{12.983240\du}}
\pgfpathlineto{\pgfpoint{60.914474\du}{12.979541\du}}
\pgfpathlineto{\pgfpoint{60.913801\du}{12.975507\du}}
\pgfpathlineto{\pgfpoint{60.913465\du}{12.971472\du}}
\pgfpathlineto{\pgfpoint{60.912120\du}{12.967774\du}}
\pgfpathlineto{\pgfpoint{60.911448\du}{12.963403\du}}
\pgfpathlineto{\pgfpoint{60.909767\du}{12.959704\du}}
\pgfpathlineto{\pgfpoint{60.908422\du}{12.955670\du}}
\pgfpathlineto{\pgfpoint{60.906404\du}{12.951635\du}}
\pgfpathlineto{\pgfpoint{60.905059\du}{12.948273\du}}
\pgfpathlineto{\pgfpoint{60.903042\du}{12.944910\du}}
\pgfpathlineto{\pgfpoint{60.900688\du}{12.941212\du}}
\pgfpathlineto{\pgfpoint{60.897999\du}{12.938186\du}}
\pgfpathlineto{\pgfpoint{60.895309\du}{12.934824\du}}
\pgfpathlineto{\pgfpoint{60.892619\du}{12.931798\du}}
\pgfpathlineto{\pgfpoint{60.890266\du}{12.928772\du}}
\pgfpathlineto{\pgfpoint{60.886567\du}{12.926418\du}}
\pgfpathlineto{\pgfpoint{60.883877\du}{12.923392\du}}
\pgfpathlineto{\pgfpoint{60.880179\du}{12.921039\du}}
\pgfpathlineto{\pgfpoint{60.876817\du}{12.918685\du}}
\pgfpathlineto{\pgfpoint{60.873454\du}{12.916668\du}}
\pgfpathlineto{\pgfpoint{60.869756\du}{12.914986\du}}
\pgfpathlineto{\pgfpoint{60.866057\du}{12.913305\du}}
\pgfpathlineto{\pgfpoint{60.862023\du}{12.911624\du}}
\pgfpathlineto{\pgfpoint{60.858661\du}{12.910279\du}}
\pgfpathlineto{\pgfpoint{60.854626\du}{12.908934\du}}
\pgfpathlineto{\pgfpoint{60.850255\du}{12.908262\du}}
\pgfpathlineto{\pgfpoint{60.846557\du}{12.907253\du}}
\pgfpathlineto{\pgfpoint{60.842522\du}{12.906917\du}}
\pgfpathlineto{\pgfpoint{60.838487\du}{12.906581\du}}
\pgfpathlineto{\pgfpoint{60.834452\du}{12.906581\du}}
\pgfpathlineto{\pgfpoint{60.834452\du}{12.906581\du}}
\pgfpathlineto{\pgfpoint{60.830082\du}{12.906581\du}}
\pgfpathlineto{\pgfpoint{60.826383\du}{12.906917\du}}
\pgfpathlineto{\pgfpoint{60.822348\du}{12.907253\du}}
\pgfpathlineto{\pgfpoint{60.818650\du}{12.908262\du}}
\pgfpathlineto{\pgfpoint{60.814615\du}{12.908934\du}}
\pgfpathlineto{\pgfpoint{60.810581\du}{12.910279\du}}
\pgfpathlineto{\pgfpoint{60.806546\du}{12.911624\du}}
\pgfpathlineto{\pgfpoint{60.802847\du}{12.913305\du}}
\pgfpathlineto{\pgfpoint{60.799149\du}{12.914986\du}}
\pgfpathlineto{\pgfpoint{60.795787\du}{12.916668\du}}
\pgfpathlineto{\pgfpoint{60.791752\du}{12.918685\du}}
\pgfpathlineto{\pgfpoint{60.788390\du}{12.921039\du}}
\pgfpathlineto{\pgfpoint{60.785364\du}{12.923392\du}}
\pgfpathlineto{\pgfpoint{60.782338\du}{12.926418\du}}
\pgfpathlineto{\pgfpoint{60.778976\du}{12.928772\du}}
\pgfpathlineto{\pgfpoint{60.776622\du}{12.931798\du}}
\pgfpathlineto{\pgfpoint{60.773260\du}{12.934824\du}}
\pgfpathlineto{\pgfpoint{60.770906\du}{12.938186\du}}
\pgfpathlineto{\pgfpoint{60.768553\du}{12.941212\du}}
\pgfpathlineto{\pgfpoint{60.766199\du}{12.944910\du}}
\pgfpathlineto{\pgfpoint{60.764182\du}{12.948273\du}}
\pgfpathlineto{\pgfpoint{60.762501\du}{12.951635\du}}
\pgfpathlineto{\pgfpoint{60.760483\du}{12.955670\du}}
\pgfpathlineto{\pgfpoint{60.759138\du}{12.959704\du}}
\pgfpathlineto{\pgfpoint{60.757793\du}{12.963403\du}}
\pgfpathlineto{\pgfpoint{60.756449\du}{12.967774\du}}
\pgfpathlineto{\pgfpoint{60.755776\du}{12.971472\du}}
\pgfpathlineto{\pgfpoint{60.754767\du}{12.975507\du}}
\pgfpathlineto{\pgfpoint{60.754431\du}{12.979541\du}}
\pgfpathlineto{\pgfpoint{60.754095\du}{12.983240\du}}
\pgfpathlineto{\pgfpoint{60.754095\du}{12.987947\du}}
\pgfpathlineto{\pgfpoint{60.754095\du}{12.987947\du}}
\pgfpathlineto{\pgfpoint{60.754095\du}{12.991645\du}}
\pgfpathlineto{\pgfpoint{60.754431\du}{12.996353\du}}
\pgfpathlineto{\pgfpoint{60.754767\du}{13.000051\du}}
\pgfpathlineto{\pgfpoint{60.755776\du}{13.004422\du}}
\pgfpathlineto{\pgfpoint{60.756449\du}{13.008120\du}}
\pgfpathlineto{\pgfpoint{60.757793\du}{13.012491\du}}
\pgfpathlineto{\pgfpoint{60.759138\du}{13.016190\du}}
\pgfpathlineto{\pgfpoint{60.760483\du}{13.019888\du}}
\pgfpathlineto{\pgfpoint{60.762501\du}{13.023923\du}}
\pgfpathlineto{\pgfpoint{60.764182\du}{13.027285\du}}
\pgfpathlineto{\pgfpoint{60.766199\du}{13.030984\du}}
\pgfpathlineto{\pgfpoint{60.768553\du}{13.034346\du}}
\pgfpathlineto{\pgfpoint{60.770906\du}{13.037708\du}}
\pgfpathlineto{\pgfpoint{60.773260\du}{13.041070\du}}
\pgfpathlineto{\pgfpoint{60.776622\du}{13.044096\du}}
\pgfpathlineto{\pgfpoint{60.778976\du}{13.047122\du}}
\pgfpathlineto{\pgfpoint{60.782338\du}{13.049476\du}}
\pgfpathlineto{\pgfpoint{60.785364\du}{13.052502\du}}
\pgfpathlineto{\pgfpoint{60.788390\du}{13.054519\du}}
\pgfpathlineto{\pgfpoint{60.791752\du}{13.056873\du}}
\pgfpathlineto{\pgfpoint{60.795787\du}{13.058890\du}}
\pgfpathlineto{\pgfpoint{60.799149\du}{13.060907\du}}
\pgfpathlineto{\pgfpoint{60.802847\du}{13.062589\du}}
\pgfpathlineto{\pgfpoint{60.806546\du}{13.064270\du}}
\pgfpathlineto{\pgfpoint{60.810581\du}{13.065278\du}}
\pgfpathlineto{\pgfpoint{60.814615\du}{13.066960\du}}
\pgfpathlineto{\pgfpoint{60.818650\du}{13.067296\du}}
\pgfpathlineto{\pgfpoint{60.822348\du}{13.068641\du}}
\pgfpathlineto{\pgfpoint{60.826383\du}{13.068977\du}}
\pgfpathlineto{\pgfpoint{60.830082\du}{13.069313\du}}
\pgfpathlineto{\pgfpoint{60.834452\du}{13.069313\du}}
\pgfpathlineto{\pgfpoint{60.834452\du}{13.069313\du}}
\pgfpathlineto{\pgfpoint{60.838487\du}{13.069313\du}}
\pgfpathlineto{\pgfpoint{60.842522\du}{13.068977\du}}
\pgfpathlineto{\pgfpoint{60.846557\du}{13.068641\du}}
\pgfpathlineto{\pgfpoint{60.850255\du}{13.067296\du}}
\pgfpathlineto{\pgfpoint{60.854626\du}{13.066960\du}}
\pgfpathlineto{\pgfpoint{60.858661\du}{13.065278\du}}
\pgfpathlineto{\pgfpoint{60.862023\du}{13.064270\du}}
\pgfpathlineto{\pgfpoint{60.866057\du}{13.062589\du}}
\pgfpathlineto{\pgfpoint{60.869756\du}{13.060907\du}}
\pgfpathlineto{\pgfpoint{60.873454\du}{13.058890\du}}
\pgfpathlineto{\pgfpoint{60.876817\du}{13.056873\du}}
\pgfpathlineto{\pgfpoint{60.880179\du}{13.054519\du}}
\pgfpathlineto{\pgfpoint{60.883877\du}{13.052502\du}}
\pgfpathlineto{\pgfpoint{60.886567\du}{13.049476\du}}
\pgfpathlineto{\pgfpoint{60.890266\du}{13.047122\du}}
\pgfpathlineto{\pgfpoint{60.892619\du}{13.044096\du}}
\pgfpathlineto{\pgfpoint{60.895309\du}{13.041070\du}}
\pgfpathlineto{\pgfpoint{60.897999\du}{13.037708\du}}
\pgfpathlineto{\pgfpoint{60.900688\du}{13.034346\du}}
\pgfpathlineto{\pgfpoint{60.903042\du}{13.030984\du}}
\pgfpathlineto{\pgfpoint{60.905059\du}{13.027285\du}}
\pgfpathlineto{\pgfpoint{60.906404\du}{13.023923\du}}
\pgfpathlineto{\pgfpoint{60.908422\du}{13.019888\du}}
\pgfpathlineto{\pgfpoint{60.909767\du}{13.016190\du}}
\pgfpathlineto{\pgfpoint{60.911448\du}{13.012491\du}}
\pgfpathlineto{\pgfpoint{60.912120\du}{13.008120\du}}
\pgfpathlineto{\pgfpoint{60.913465\du}{13.004422\du}}
\pgfpathlineto{\pgfpoint{60.913801\du}{13.000051\du}}
\pgfpathlineto{\pgfpoint{60.914474\du}{12.996353\du}}
\pgfpathlineto{\pgfpoint{60.914810\du}{12.991645\du}}
\pgfpathlineto{\pgfpoint{60.914810\du}{12.987947\du}}
\pgfusepath{fill}
\pgfsetlinewidth{0.000000\du}
\pgfsetbuttcap
\pgfsetmiterjoin
\pgfsetdash{}{0pt}
\definecolor{dialinecolor}{rgb}{0.000000, 0.000000, 0.000000}
\pgfsetstrokecolor{dialinecolor}
\pgfpathmoveto{\pgfpoint{60.107537\du}{12.981223\du}}
\pgfpathlineto{\pgfpoint{60.811589\du}{12.981223\du}}
\pgfusepath{stroke}
\pgfsetlinewidth{0.000000\du}
\pgfsetbuttcap
\pgfsetmiterjoin
\pgfsetdash{}{0pt}
\definecolor{dialinecolor}{rgb}{1.000000, 1.000000, 1.000000}
\pgfsetfillcolor{dialinecolor}
\pgfpathmoveto{\pgfpoint{60.180498\du}{13.285841\du}}
\pgfpathlineto{\pgfpoint{60.180498\du}{13.281806\du}}
\pgfpathlineto{\pgfpoint{60.180161\du}{13.277435\du}}
\pgfpathlineto{\pgfpoint{60.179825\du}{13.273737\du}}
\pgfpathlineto{\pgfpoint{60.178817\du}{13.269702\du}}
\pgfpathlineto{\pgfpoint{60.178144\du}{13.265668\du}}
\pgfpathlineto{\pgfpoint{60.176799\du}{13.261633\du}}
\pgfpathlineto{\pgfpoint{60.175791\du}{13.257598\du}}
\pgfpathlineto{\pgfpoint{60.173773\du}{13.253900\du}}
\pgfpathlineto{\pgfpoint{60.172092\du}{13.250201\du}}
\pgfpathlineto{\pgfpoint{60.170411\du}{13.246503\du}}
\pgfpathlineto{\pgfpoint{60.168394\du}{13.243141\du}}
\pgfpathlineto{\pgfpoint{60.166376\du}{13.239442\du}}
\pgfpathlineto{\pgfpoint{60.163686\du}{13.236080\du}}
\pgfpathlineto{\pgfpoint{60.161333\du}{13.232718\du}}
\pgfpathlineto{\pgfpoint{60.158307\du}{13.230028\du}}
\pgfpathlineto{\pgfpoint{60.155281\du}{13.227002\du}}
\pgfpathlineto{\pgfpoint{60.152591\du}{13.224312\du}}
\pgfpathlineto{\pgfpoint{60.149565\du}{13.221286\du}}
\pgfpathlineto{\pgfpoint{60.146203\du}{13.219269\du}}
\pgfpathlineto{\pgfpoint{60.142841\du}{13.216579\du}}
\pgfpathlineto{\pgfpoint{60.138806\du}{13.214562\du}}
\pgfpathlineto{\pgfpoint{60.135780\du}{13.212881\du}}
\pgfpathlineto{\pgfpoint{60.131745\du}{13.211199\du}}
\pgfpathlineto{\pgfpoint{60.128047\du}{13.209518\du}}
\pgfpathlineto{\pgfpoint{60.123676\du}{13.208173\du}}
\pgfpathlineto{\pgfpoint{60.119977\du}{13.207165\du}}
\pgfpathlineto{\pgfpoint{60.116279\du}{13.206156\du}}
\pgfpathlineto{\pgfpoint{60.111908\du}{13.205484\du}}
\pgfpathlineto{\pgfpoint{60.107873\du}{13.204811\du}}
\pgfpathlineto{\pgfpoint{60.104175\du}{13.204475\du}}
\pgfpathlineto{\pgfpoint{60.100140\du}{13.204475\du}}
\pgfpathlineto{\pgfpoint{60.100140\du}{13.204475\du}}
\pgfpathlineto{\pgfpoint{60.095769\du}{13.204475\du}}
\pgfpathlineto{\pgfpoint{60.091735\du}{13.204811\du}}
\pgfpathlineto{\pgfpoint{60.087700\du}{13.205484\du}}
\pgfpathlineto{\pgfpoint{60.083329\du}{13.206156\du}}
\pgfpathlineto{\pgfpoint{60.079631\du}{13.207165\du}}
\pgfpathlineto{\pgfpoint{60.075932\du}{13.208173\du}}
\pgfpathlineto{\pgfpoint{60.071561\du}{13.209518\du}}
\pgfpathlineto{\pgfpoint{60.067863\du}{13.211199\du}}
\pgfpathlineto{\pgfpoint{60.064501\du}{13.212881\du}}
\pgfpathlineto{\pgfpoint{60.060802\du}{13.214562\du}}
\pgfpathlineto{\pgfpoint{60.056767\du}{13.216579\du}}
\pgfpathlineto{\pgfpoint{60.053741\du}{13.219269\du}}
\pgfpathlineto{\pgfpoint{60.050043\du}{13.221286\du}}
\pgfpathlineto{\pgfpoint{60.047689\du}{13.224312\du}}
\pgfpathlineto{\pgfpoint{60.044327\du}{13.227002\du}}
\pgfpathlineto{\pgfpoint{60.041301\du}{13.230028\du}}
\pgfpathlineto{\pgfpoint{60.038275\du}{13.232718\du}}
\pgfpathlineto{\pgfpoint{60.035922\du}{13.236080\du}}
\pgfpathlineto{\pgfpoint{60.033232\du}{13.239442\du}}
\pgfpathlineto{\pgfpoint{60.031214\du}{13.243141\du}}
\pgfpathlineto{\pgfpoint{60.029197\du}{13.246503\du}}
\pgfpathlineto{\pgfpoint{60.027516\du}{13.250201\du}}
\pgfpathlineto{\pgfpoint{60.025835\du}{13.253900\du}}
\pgfpathlineto{\pgfpoint{60.024490\du}{13.257598\du}}
\pgfpathlineto{\pgfpoint{60.022809\du}{13.261633\du}}
\pgfpathlineto{\pgfpoint{60.021464\du}{13.265668\du}}
\pgfpathlineto{\pgfpoint{60.020791\du}{13.269702\du}}
\pgfpathlineto{\pgfpoint{60.020119\du}{13.273737\du}}
\pgfpathlineto{\pgfpoint{60.019447\du}{13.277435\du}}
\pgfpathlineto{\pgfpoint{60.019110\du}{13.281806\du}}
\pgfpathlineto{\pgfpoint{60.019110\du}{13.285841\du}}
\pgfpathlineto{\pgfpoint{60.019110\du}{13.285841\du}}
\pgfpathlineto{\pgfpoint{60.019110\du}{13.290212\du}}
\pgfpathlineto{\pgfpoint{60.019447\du}{13.294247\du}}
\pgfpathlineto{\pgfpoint{60.020119\du}{13.298281\du}}
\pgfpathlineto{\pgfpoint{60.020791\du}{13.302652\du}}
\pgfpathlineto{\pgfpoint{60.021464\du}{13.306351\du}}
\pgfpathlineto{\pgfpoint{60.022809\du}{13.310385\du}}
\pgfpathlineto{\pgfpoint{60.024490\du}{13.314420\du}}
\pgfpathlineto{\pgfpoint{60.025835\du}{13.318119\du}}
\pgfpathlineto{\pgfpoint{60.027516\du}{13.321817\du}}
\pgfpathlineto{\pgfpoint{60.029197\du}{13.325852\du}}
\pgfpathlineto{\pgfpoint{60.031214\du}{13.329214\du}}
\pgfpathlineto{\pgfpoint{60.033232\du}{13.332576\du}}
\pgfpathlineto{\pgfpoint{60.035922\du}{13.335938\du}}
\pgfpathlineto{\pgfpoint{60.038275\du}{13.339301\du}}
\pgfpathlineto{\pgfpoint{60.041301\du}{13.342327\du}}
\pgfpathlineto{\pgfpoint{60.044327\du}{13.344680\du}}
\pgfpathlineto{\pgfpoint{60.047689\du}{13.347706\du}}
\pgfpathlineto{\pgfpoint{60.050043\du}{13.350732\du}}
\pgfpathlineto{\pgfpoint{60.053741\du}{13.352750\du}}
\pgfpathlineto{\pgfpoint{60.056767\du}{13.355439\du}}
\pgfpathlineto{\pgfpoint{60.060802\du}{13.357457\du}}
\pgfpathlineto{\pgfpoint{60.064501\du}{13.359138\du}}
\pgfpathlineto{\pgfpoint{60.067863\du}{13.360819\du}}
\pgfpathlineto{\pgfpoint{60.071561\du}{13.362500\du}}
\pgfpathlineto{\pgfpoint{60.075932\du}{13.363845\du}}
\pgfpathlineto{\pgfpoint{60.079631\du}{13.364854\du}}
\pgfpathlineto{\pgfpoint{60.083329\du}{13.365862\du}}
\pgfpathlineto{\pgfpoint{60.087700\du}{13.366871\du}}
\pgfpathlineto{\pgfpoint{60.091735\du}{13.367207\du}}
\pgfpathlineto{\pgfpoint{60.095769\du}{13.367543\du}}
\pgfpathlineto{\pgfpoint{60.100140\du}{13.367543\du}}
\pgfpathlineto{\pgfpoint{60.100140\du}{13.367543\du}}
\pgfpathlineto{\pgfpoint{60.104175\du}{13.367543\du}}
\pgfpathlineto{\pgfpoint{60.107873\du}{13.367207\du}}
\pgfpathlineto{\pgfpoint{60.111908\du}{13.366871\du}}
\pgfpathlineto{\pgfpoint{60.116279\du}{13.365862\du}}
\pgfpathlineto{\pgfpoint{60.119977\du}{13.364854\du}}
\pgfpathlineto{\pgfpoint{60.123676\du}{13.363845\du}}
\pgfpathlineto{\pgfpoint{60.128047\du}{13.362500\du}}
\pgfpathlineto{\pgfpoint{60.131745\du}{13.360819\du}}
\pgfpathlineto{\pgfpoint{60.135780\du}{13.359138\du}}
\pgfpathlineto{\pgfpoint{60.138806\du}{13.357457\du}}
\pgfpathlineto{\pgfpoint{60.142841\du}{13.355439\du}}
\pgfpathlineto{\pgfpoint{60.146203\du}{13.352750\du}}
\pgfpathlineto{\pgfpoint{60.149565\du}{13.350732\du}}
\pgfpathlineto{\pgfpoint{60.152591\du}{13.347706\du}}
\pgfpathlineto{\pgfpoint{60.155281\du}{13.344680\du}}
\pgfpathlineto{\pgfpoint{60.158307\du}{13.342327\du}}
\pgfpathlineto{\pgfpoint{60.161333\du}{13.339301\du}}
\pgfpathlineto{\pgfpoint{60.163686\du}{13.335938\du}}
\pgfpathlineto{\pgfpoint{60.166376\du}{13.332576\du}}
\pgfpathlineto{\pgfpoint{60.168394\du}{13.329214\du}}
\pgfpathlineto{\pgfpoint{60.170411\du}{13.325852\du}}
\pgfpathlineto{\pgfpoint{60.172092\du}{13.321817\du}}
\pgfpathlineto{\pgfpoint{60.173773\du}{13.318119\du}}
\pgfpathlineto{\pgfpoint{60.175791\du}{13.314420\du}}
\pgfpathlineto{\pgfpoint{60.176799\du}{13.310385\du}}
\pgfpathlineto{\pgfpoint{60.178144\du}{13.306351\du}}
\pgfpathlineto{\pgfpoint{60.178817\du}{13.302652\du}}
\pgfpathlineto{\pgfpoint{60.179825\du}{13.298281\du}}
\pgfpathlineto{\pgfpoint{60.180161\du}{13.294247\du}}
\pgfpathlineto{\pgfpoint{60.180498\du}{13.290212\du}}
\pgfpathlineto{\pgfpoint{60.180498\du}{13.285841\du}}
\pgfusepath{fill}
\pgfsetbuttcap
\pgfsetmiterjoin
\pgfsetdash{}{0pt}
\definecolor{dialinecolor}{rgb}{1.000000, 1.000000, 1.000000}
\pgfsetfillcolor{dialinecolor}
\pgfpathmoveto{\pgfpoint{60.914810\du}{13.285841\du}}
\pgfpathlineto{\pgfpoint{60.914810\du}{13.281806\du}}
\pgfpathlineto{\pgfpoint{60.914474\du}{13.277435\du}}
\pgfpathlineto{\pgfpoint{60.913801\du}{13.273737\du}}
\pgfpathlineto{\pgfpoint{60.913465\du}{13.269702\du}}
\pgfpathlineto{\pgfpoint{60.912120\du}{13.265668\du}}
\pgfpathlineto{\pgfpoint{60.911448\du}{13.261633\du}}
\pgfpathlineto{\pgfpoint{60.909767\du}{13.257598\du}}
\pgfpathlineto{\pgfpoint{60.908422\du}{13.253900\du}}
\pgfpathlineto{\pgfpoint{60.906404\du}{13.250201\du}}
\pgfpathlineto{\pgfpoint{60.905059\du}{13.246503\du}}
\pgfpathlineto{\pgfpoint{60.903042\du}{13.243141\du}}
\pgfpathlineto{\pgfpoint{60.900688\du}{13.239442\du}}
\pgfpathlineto{\pgfpoint{60.897999\du}{13.236080\du}}
\pgfpathlineto{\pgfpoint{60.895309\du}{13.232718\du}}
\pgfpathlineto{\pgfpoint{60.892619\du}{13.230028\du}}
\pgfpathlineto{\pgfpoint{60.890266\du}{13.227002\du}}
\pgfpathlineto{\pgfpoint{60.886567\du}{13.224312\du}}
\pgfpathlineto{\pgfpoint{60.883877\du}{13.221286\du}}
\pgfpathlineto{\pgfpoint{60.880179\du}{13.219269\du}}
\pgfpathlineto{\pgfpoint{60.876817\du}{13.216579\du}}
\pgfpathlineto{\pgfpoint{60.873454\du}{13.214562\du}}
\pgfpathlineto{\pgfpoint{60.869756\du}{13.212881\du}}
\pgfpathlineto{\pgfpoint{60.866057\du}{13.211199\du}}
\pgfpathlineto{\pgfpoint{60.862023\du}{13.209518\du}}
\pgfpathlineto{\pgfpoint{60.858661\du}{13.208173\du}}
\pgfpathlineto{\pgfpoint{60.854626\du}{13.207165\du}}
\pgfpathlineto{\pgfpoint{60.850255\du}{13.206156\du}}
\pgfpathlineto{\pgfpoint{60.846557\du}{13.205484\du}}
\pgfpathlineto{\pgfpoint{60.842522\du}{13.204811\du}}
\pgfpathlineto{\pgfpoint{60.838487\du}{13.204475\du}}
\pgfpathlineto{\pgfpoint{60.834452\du}{13.204475\du}}
\pgfpathlineto{\pgfpoint{60.834452\du}{13.204475\du}}
\pgfpathlineto{\pgfpoint{60.830082\du}{13.204475\du}}
\pgfpathlineto{\pgfpoint{60.826383\du}{13.204811\du}}
\pgfpathlineto{\pgfpoint{60.822348\du}{13.205484\du}}
\pgfpathlineto{\pgfpoint{60.818650\du}{13.206156\du}}
\pgfpathlineto{\pgfpoint{60.814615\du}{13.207165\du}}
\pgfpathlineto{\pgfpoint{60.810581\du}{13.208173\du}}
\pgfpathlineto{\pgfpoint{60.806546\du}{13.209518\du}}
\pgfpathlineto{\pgfpoint{60.802847\du}{13.211199\du}}
\pgfpathlineto{\pgfpoint{60.799149\du}{13.212881\du}}
\pgfpathlineto{\pgfpoint{60.795787\du}{13.214562\du}}
\pgfpathlineto{\pgfpoint{60.791752\du}{13.216579\du}}
\pgfpathlineto{\pgfpoint{60.788390\du}{13.219269\du}}
\pgfpathlineto{\pgfpoint{60.785364\du}{13.221286\du}}
\pgfpathlineto{\pgfpoint{60.782338\du}{13.224312\du}}
\pgfpathlineto{\pgfpoint{60.778976\du}{13.227002\du}}
\pgfpathlineto{\pgfpoint{60.776622\du}{13.230028\du}}
\pgfpathlineto{\pgfpoint{60.773260\du}{13.232718\du}}
\pgfpathlineto{\pgfpoint{60.770906\du}{13.236080\du}}
\pgfpathlineto{\pgfpoint{60.768553\du}{13.239442\du}}
\pgfpathlineto{\pgfpoint{60.766199\du}{13.243141\du}}
\pgfpathlineto{\pgfpoint{60.764182\du}{13.246503\du}}
\pgfpathlineto{\pgfpoint{60.762501\du}{13.250201\du}}
\pgfpathlineto{\pgfpoint{60.760483\du}{13.253900\du}}
\pgfpathlineto{\pgfpoint{60.759138\du}{13.257598\du}}
\pgfpathlineto{\pgfpoint{60.757793\du}{13.261633\du}}
\pgfpathlineto{\pgfpoint{60.756449\du}{13.265668\du}}
\pgfpathlineto{\pgfpoint{60.755776\du}{13.269702\du}}
\pgfpathlineto{\pgfpoint{60.754767\du}{13.273737\du}}
\pgfpathlineto{\pgfpoint{60.754431\du}{13.277435\du}}
\pgfpathlineto{\pgfpoint{60.754095\du}{13.281806\du}}
\pgfpathlineto{\pgfpoint{60.754095\du}{13.285841\du}}
\pgfpathlineto{\pgfpoint{60.754095\du}{13.285841\du}}
\pgfpathlineto{\pgfpoint{60.754095\du}{13.290212\du}}
\pgfpathlineto{\pgfpoint{60.754431\du}{13.294247\du}}
\pgfpathlineto{\pgfpoint{60.754767\du}{13.298281\du}}
\pgfpathlineto{\pgfpoint{60.755776\du}{13.302652\du}}
\pgfpathlineto{\pgfpoint{60.756449\du}{13.306351\du}}
\pgfpathlineto{\pgfpoint{60.757793\du}{13.310385\du}}
\pgfpathlineto{\pgfpoint{60.759138\du}{13.314420\du}}
\pgfpathlineto{\pgfpoint{60.760483\du}{13.318119\du}}
\pgfpathlineto{\pgfpoint{60.762501\du}{13.321817\du}}
\pgfpathlineto{\pgfpoint{60.764182\du}{13.325852\du}}
\pgfpathlineto{\pgfpoint{60.766199\du}{13.329214\du}}
\pgfpathlineto{\pgfpoint{60.768553\du}{13.332576\du}}
\pgfpathlineto{\pgfpoint{60.770906\du}{13.335938\du}}
\pgfpathlineto{\pgfpoint{60.773260\du}{13.339301\du}}
\pgfpathlineto{\pgfpoint{60.776622\du}{13.342327\du}}
\pgfpathlineto{\pgfpoint{60.778976\du}{13.344680\du}}
\pgfpathlineto{\pgfpoint{60.782338\du}{13.347706\du}}
\pgfpathlineto{\pgfpoint{60.785364\du}{13.350732\du}}
\pgfpathlineto{\pgfpoint{60.788390\du}{13.352750\du}}
\pgfpathlineto{\pgfpoint{60.791752\du}{13.355439\du}}
\pgfpathlineto{\pgfpoint{60.795787\du}{13.357457\du}}
\pgfpathlineto{\pgfpoint{60.799149\du}{13.359138\du}}
\pgfpathlineto{\pgfpoint{60.802847\du}{13.360819\du}}
\pgfpathlineto{\pgfpoint{60.806546\du}{13.362500\du}}
\pgfpathlineto{\pgfpoint{60.810581\du}{13.363845\du}}
\pgfpathlineto{\pgfpoint{60.814615\du}{13.364854\du}}
\pgfpathlineto{\pgfpoint{60.818650\du}{13.365862\du}}
\pgfpathlineto{\pgfpoint{60.822348\du}{13.366871\du}}
\pgfpathlineto{\pgfpoint{60.826383\du}{13.367207\du}}
\pgfpathlineto{\pgfpoint{60.830082\du}{13.367543\du}}
\pgfpathlineto{\pgfpoint{60.834452\du}{13.367543\du}}
\pgfpathlineto{\pgfpoint{60.834452\du}{13.367543\du}}
\pgfpathlineto{\pgfpoint{60.838487\du}{13.367543\du}}
\pgfpathlineto{\pgfpoint{60.842522\du}{13.367207\du}}
\pgfpathlineto{\pgfpoint{60.846557\du}{13.366871\du}}
\pgfpathlineto{\pgfpoint{60.850255\du}{13.365862\du}}
\pgfpathlineto{\pgfpoint{60.854626\du}{13.364854\du}}
\pgfpathlineto{\pgfpoint{60.858661\du}{13.363845\du}}
\pgfpathlineto{\pgfpoint{60.862023\du}{13.362500\du}}
\pgfpathlineto{\pgfpoint{60.866057\du}{13.360819\du}}
\pgfpathlineto{\pgfpoint{60.869756\du}{13.359138\du}}
\pgfpathlineto{\pgfpoint{60.873454\du}{13.357457\du}}
\pgfpathlineto{\pgfpoint{60.876817\du}{13.355439\du}}
\pgfpathlineto{\pgfpoint{60.880179\du}{13.352750\du}}
\pgfpathlineto{\pgfpoint{60.883877\du}{13.350732\du}}
\pgfpathlineto{\pgfpoint{60.886567\du}{13.347706\du}}
\pgfpathlineto{\pgfpoint{60.890266\du}{13.344680\du}}
\pgfpathlineto{\pgfpoint{60.892619\du}{13.342327\du}}
\pgfpathlineto{\pgfpoint{60.895309\du}{13.339301\du}}
\pgfpathlineto{\pgfpoint{60.897999\du}{13.335938\du}}
\pgfpathlineto{\pgfpoint{60.900688\du}{13.332576\du}}
\pgfpathlineto{\pgfpoint{60.903042\du}{13.329214\du}}
\pgfpathlineto{\pgfpoint{60.905059\du}{13.325852\du}}
\pgfpathlineto{\pgfpoint{60.906404\du}{13.321817\du}}
\pgfpathlineto{\pgfpoint{60.908422\du}{13.318119\du}}
\pgfpathlineto{\pgfpoint{60.909767\du}{13.314420\du}}
\pgfpathlineto{\pgfpoint{60.911448\du}{13.310385\du}}
\pgfpathlineto{\pgfpoint{60.912120\du}{13.306351\du}}
\pgfpathlineto{\pgfpoint{60.913465\du}{13.302652\du}}
\pgfpathlineto{\pgfpoint{60.913801\du}{13.298281\du}}
\pgfpathlineto{\pgfpoint{60.914474\du}{13.294247\du}}
\pgfpathlineto{\pgfpoint{60.914810\du}{13.290212\du}}
\pgfpathlineto{\pgfpoint{60.914810\du}{13.285841\du}}
\pgfusepath{fill}
\pgfsetlinewidth{0.000000\du}
\pgfsetbuttcap
\pgfsetmiterjoin
\pgfsetdash{}{0pt}
\definecolor{dialinecolor}{rgb}{0.000000, 0.000000, 0.000000}
\pgfsetstrokecolor{dialinecolor}
\pgfpathmoveto{\pgfpoint{60.107537\du}{13.278780\du}}
\pgfpathlineto{\pgfpoint{60.811589\du}{13.278780\du}}
\pgfusepath{stroke}
\pgfsetlinewidth{0.000000\du}
\pgfsetbuttcap
\pgfsetmiterjoin
\pgfsetdash{}{0pt}
\definecolor{dialinecolor}{rgb}{1.000000, 1.000000, 1.000000}
\pgfsetfillcolor{dialinecolor}
\pgfpathmoveto{\pgfpoint{60.180498\du}{13.583735\du}}
\pgfpathlineto{\pgfpoint{60.180498\du}{13.579700\du}}
\pgfpathlineto{\pgfpoint{60.180161\du}{13.575330\du}}
\pgfpathlineto{\pgfpoint{60.179825\du}{13.571631\du}}
\pgfpathlineto{\pgfpoint{60.178817\du}{13.567260\du}}
\pgfpathlineto{\pgfpoint{60.178144\du}{13.563225\du}}
\pgfpathlineto{\pgfpoint{60.176799\du}{13.559191\du}}
\pgfpathlineto{\pgfpoint{60.175791\du}{13.555492\du}}
\pgfpathlineto{\pgfpoint{60.173773\du}{13.551794\du}}
\pgfpathlineto{\pgfpoint{60.172092\du}{13.547759\du}}
\pgfpathlineto{\pgfpoint{60.170411\du}{13.544061\du}}
\pgfpathlineto{\pgfpoint{60.168394\du}{13.540698\du}}
\pgfpathlineto{\pgfpoint{60.166376\du}{13.537336\du}}
\pgfpathlineto{\pgfpoint{60.163686\du}{13.533974\du}}
\pgfpathlineto{\pgfpoint{60.161333\du}{13.530612\du}}
\pgfpathlineto{\pgfpoint{60.158307\du}{13.527586\du}}
\pgfpathlineto{\pgfpoint{60.155281\du}{13.524560\du}}
\pgfpathlineto{\pgfpoint{60.152591\du}{13.522206\du}}
\pgfpathlineto{\pgfpoint{60.149565\du}{13.519180\du}}
\pgfpathlineto{\pgfpoint{60.146203\du}{13.517163\du}}
\pgfpathlineto{\pgfpoint{60.142841\du}{13.514473\du}}
\pgfpathlineto{\pgfpoint{60.138806\du}{13.512456\du}}
\pgfpathlineto{\pgfpoint{60.135780\du}{13.510775\du}}
\pgfpathlineto{\pgfpoint{60.131745\du}{13.509093\du}}
\pgfpathlineto{\pgfpoint{60.128047\du}{13.507412\du}}
\pgfpathlineto{\pgfpoint{60.123676\du}{13.506067\du}}
\pgfpathlineto{\pgfpoint{60.119977\du}{13.504386\du}}
\pgfpathlineto{\pgfpoint{60.116279\du}{13.504050\du}}
\pgfpathlineto{\pgfpoint{60.111908\du}{13.503041\du}}
\pgfpathlineto{\pgfpoint{60.107873\du}{13.502369\du}}
\pgfpathlineto{\pgfpoint{60.104175\du}{13.502369\du}}
\pgfpathlineto{\pgfpoint{60.100140\du}{13.502369\du}}
\pgfpathlineto{\pgfpoint{60.100140\du}{13.502369\du}}
\pgfpathlineto{\pgfpoint{60.095769\du}{13.502369\du}}
\pgfpathlineto{\pgfpoint{60.091735\du}{13.502369\du}}
\pgfpathlineto{\pgfpoint{60.087700\du}{13.503041\du}}
\pgfpathlineto{\pgfpoint{60.083329\du}{13.504050\du}}
\pgfpathlineto{\pgfpoint{60.079631\du}{13.504386\du}}
\pgfpathlineto{\pgfpoint{60.075932\du}{13.506067\du}}
\pgfpathlineto{\pgfpoint{60.071561\du}{13.507412\du}}
\pgfpathlineto{\pgfpoint{60.067863\du}{13.509093\du}}
\pgfpathlineto{\pgfpoint{60.064501\du}{13.510775\du}}
\pgfpathlineto{\pgfpoint{60.060802\du}{13.512456\du}}
\pgfpathlineto{\pgfpoint{60.056767\du}{13.514473\du}}
\pgfpathlineto{\pgfpoint{60.053741\du}{13.517163\du}}
\pgfpathlineto{\pgfpoint{60.050043\du}{13.519180\du}}
\pgfpathlineto{\pgfpoint{60.047689\du}{13.522206\du}}
\pgfpathlineto{\pgfpoint{60.044327\du}{13.524560\du}}
\pgfpathlineto{\pgfpoint{60.041301\du}{13.527586\du}}
\pgfpathlineto{\pgfpoint{60.038275\du}{13.530612\du}}
\pgfpathlineto{\pgfpoint{60.035922\du}{13.533974\du}}
\pgfpathlineto{\pgfpoint{60.033232\du}{13.537336\du}}
\pgfpathlineto{\pgfpoint{60.031214\du}{13.540698\du}}
\pgfpathlineto{\pgfpoint{60.029197\du}{13.544061\du}}
\pgfpathlineto{\pgfpoint{60.027516\du}{13.547759\du}}
\pgfpathlineto{\pgfpoint{60.025835\du}{13.551794\du}}
\pgfpathlineto{\pgfpoint{60.024490\du}{13.555492\du}}
\pgfpathlineto{\pgfpoint{60.022809\du}{13.559191\du}}
\pgfpathlineto{\pgfpoint{60.021464\du}{13.563225\du}}
\pgfpathlineto{\pgfpoint{60.020791\du}{13.567260\du}}
\pgfpathlineto{\pgfpoint{60.020119\du}{13.571631\du}}
\pgfpathlineto{\pgfpoint{60.019447\du}{13.575330\du}}
\pgfpathlineto{\pgfpoint{60.019110\du}{13.579700\du}}
\pgfpathlineto{\pgfpoint{60.019110\du}{13.583735\du}}
\pgfpathlineto{\pgfpoint{60.019110\du}{13.583735\du}}
\pgfpathlineto{\pgfpoint{60.019110\du}{13.588106\du}}
\pgfpathlineto{\pgfpoint{60.019447\du}{13.592141\du}}
\pgfpathlineto{\pgfpoint{60.020119\du}{13.596175\du}}
\pgfpathlineto{\pgfpoint{60.020791\du}{13.600210\du}}
\pgfpathlineto{\pgfpoint{60.021464\du}{13.603909\du}}
\pgfpathlineto{\pgfpoint{60.022809\du}{13.608279\du}}
\pgfpathlineto{\pgfpoint{60.024490\du}{13.611642\du}}
\pgfpathlineto{\pgfpoint{60.025835\du}{13.616013\du}}
\pgfpathlineto{\pgfpoint{60.027516\du}{13.619711\du}}
\pgfpathlineto{\pgfpoint{60.029197\du}{13.623409\du}}
\pgfpathlineto{\pgfpoint{60.031214\du}{13.626772\du}}
\pgfpathlineto{\pgfpoint{60.033232\du}{13.630134\du}}
\pgfpathlineto{\pgfpoint{60.035922\du}{13.633496\du}}
\pgfpathlineto{\pgfpoint{60.038275\du}{13.636858\du}}
\pgfpathlineto{\pgfpoint{60.041301\du}{13.639884\du}}
\pgfpathlineto{\pgfpoint{60.044327\du}{13.642910\du}}
\pgfpathlineto{\pgfpoint{60.047689\du}{13.645264\du}}
\pgfpathlineto{\pgfpoint{60.050043\du}{13.648290\du}}
\pgfpathlineto{\pgfpoint{60.053741\du}{13.650644\du}}
\pgfpathlineto{\pgfpoint{60.056767\du}{13.652997\du}}
\pgfpathlineto{\pgfpoint{60.060802\du}{13.655014\du}}
\pgfpathlineto{\pgfpoint{60.064501\du}{13.656696\du}}
\pgfpathlineto{\pgfpoint{60.067863\du}{13.658377\du}}
\pgfpathlineto{\pgfpoint{60.071561\du}{13.660058\du}}
\pgfpathlineto{\pgfpoint{60.075932\du}{13.661403\du}}
\pgfpathlineto{\pgfpoint{60.079631\du}{13.662748\du}}
\pgfpathlineto{\pgfpoint{60.083329\du}{13.663420\du}}
\pgfpathlineto{\pgfpoint{60.087700\du}{13.664093\du}}
\pgfpathlineto{\pgfpoint{60.091735\du}{13.664765\du}}
\pgfpathlineto{\pgfpoint{60.095769\du}{13.665101\du}}
\pgfpathlineto{\pgfpoint{60.100140\du}{13.665101\du}}
\pgfpathlineto{\pgfpoint{60.100140\du}{13.665101\du}}
\pgfpathlineto{\pgfpoint{60.104175\du}{13.665101\du}}
\pgfpathlineto{\pgfpoint{60.107873\du}{13.664765\du}}
\pgfpathlineto{\pgfpoint{60.111908\du}{13.664093\du}}
\pgfpathlineto{\pgfpoint{60.116279\du}{13.663420\du}}
\pgfpathlineto{\pgfpoint{60.119977\du}{13.662748\du}}
\pgfpathlineto{\pgfpoint{60.123676\du}{13.661403\du}}
\pgfpathlineto{\pgfpoint{60.128047\du}{13.660058\du}}
\pgfpathlineto{\pgfpoint{60.131745\du}{13.658377\du}}
\pgfpathlineto{\pgfpoint{60.135780\du}{13.656696\du}}
\pgfpathlineto{\pgfpoint{60.138806\du}{13.655014\du}}
\pgfpathlineto{\pgfpoint{60.142841\du}{13.652997\du}}
\pgfpathlineto{\pgfpoint{60.146203\du}{13.650644\du}}
\pgfpathlineto{\pgfpoint{60.149565\du}{13.648290\du}}
\pgfpathlineto{\pgfpoint{60.152591\du}{13.645264\du}}
\pgfpathlineto{\pgfpoint{60.155281\du}{13.642910\du}}
\pgfpathlineto{\pgfpoint{60.158307\du}{13.639884\du}}
\pgfpathlineto{\pgfpoint{60.161333\du}{13.636858\du}}
\pgfpathlineto{\pgfpoint{60.163686\du}{13.633496\du}}
\pgfpathlineto{\pgfpoint{60.166376\du}{13.630134\du}}
\pgfpathlineto{\pgfpoint{60.168394\du}{13.626772\du}}
\pgfpathlineto{\pgfpoint{60.170411\du}{13.623409\du}}
\pgfpathlineto{\pgfpoint{60.172092\du}{13.619711\du}}
\pgfpathlineto{\pgfpoint{60.173773\du}{13.616013\du}}
\pgfpathlineto{\pgfpoint{60.175791\du}{13.611642\du}}
\pgfpathlineto{\pgfpoint{60.176799\du}{13.608279\du}}
\pgfpathlineto{\pgfpoint{60.178144\du}{13.603909\du}}
\pgfpathlineto{\pgfpoint{60.178817\du}{13.600210\du}}
\pgfpathlineto{\pgfpoint{60.179825\du}{13.596175\du}}
\pgfpathlineto{\pgfpoint{60.180161\du}{13.592141\du}}
\pgfpathlineto{\pgfpoint{60.180498\du}{13.588106\du}}
\pgfpathlineto{\pgfpoint{60.180498\du}{13.583735\du}}
\pgfusepath{fill}
\pgfsetbuttcap
\pgfsetmiterjoin
\pgfsetdash{}{0pt}
\definecolor{dialinecolor}{rgb}{1.000000, 1.000000, 1.000000}
\pgfsetfillcolor{dialinecolor}
\pgfpathmoveto{\pgfpoint{60.914810\du}{13.583735\du}}
\pgfpathlineto{\pgfpoint{60.914810\du}{13.579700\du}}
\pgfpathlineto{\pgfpoint{60.914474\du}{13.575330\du}}
\pgfpathlineto{\pgfpoint{60.913801\du}{13.571631\du}}
\pgfpathlineto{\pgfpoint{60.913465\du}{13.567260\du}}
\pgfpathlineto{\pgfpoint{60.912120\du}{13.563225\du}}
\pgfpathlineto{\pgfpoint{60.911448\du}{13.559191\du}}
\pgfpathlineto{\pgfpoint{60.909767\du}{13.555492\du}}
\pgfpathlineto{\pgfpoint{60.908422\du}{13.551794\du}}
\pgfpathlineto{\pgfpoint{60.906404\du}{13.547759\du}}
\pgfpathlineto{\pgfpoint{60.905059\du}{13.544061\du}}
\pgfpathlineto{\pgfpoint{60.903042\du}{13.540698\du}}
\pgfpathlineto{\pgfpoint{60.900688\du}{13.537336\du}}
\pgfpathlineto{\pgfpoint{60.897999\du}{13.533974\du}}
\pgfpathlineto{\pgfpoint{60.895309\du}{13.530612\du}}
\pgfpathlineto{\pgfpoint{60.892619\du}{13.527586\du}}
\pgfpathlineto{\pgfpoint{60.890266\du}{13.524560\du}}
\pgfpathlineto{\pgfpoint{60.886567\du}{13.522206\du}}
\pgfpathlineto{\pgfpoint{60.883877\du}{13.519180\du}}
\pgfpathlineto{\pgfpoint{60.880179\du}{13.517163\du}}
\pgfpathlineto{\pgfpoint{60.876817\du}{13.514473\du}}
\pgfpathlineto{\pgfpoint{60.873454\du}{13.512456\du}}
\pgfpathlineto{\pgfpoint{60.869756\du}{13.510775\du}}
\pgfpathlineto{\pgfpoint{60.866057\du}{13.509093\du}}
\pgfpathlineto{\pgfpoint{60.862023\du}{13.507412\du}}
\pgfpathlineto{\pgfpoint{60.858661\du}{13.506067\du}}
\pgfpathlineto{\pgfpoint{60.854626\du}{13.504386\du}}
\pgfpathlineto{\pgfpoint{60.850255\du}{13.504050\du}}
\pgfpathlineto{\pgfpoint{60.846557\du}{13.503041\du}}
\pgfpathlineto{\pgfpoint{60.842522\du}{13.502369\du}}
\pgfpathlineto{\pgfpoint{60.838487\du}{13.502369\du}}
\pgfpathlineto{\pgfpoint{60.834452\du}{13.502369\du}}
\pgfpathlineto{\pgfpoint{60.834452\du}{13.502369\du}}
\pgfpathlineto{\pgfpoint{60.830082\du}{13.502369\du}}
\pgfpathlineto{\pgfpoint{60.826383\du}{13.502369\du}}
\pgfpathlineto{\pgfpoint{60.822348\du}{13.503041\du}}
\pgfpathlineto{\pgfpoint{60.818650\du}{13.504050\du}}
\pgfpathlineto{\pgfpoint{60.814615\du}{13.504386\du}}
\pgfpathlineto{\pgfpoint{60.810581\du}{13.506067\du}}
\pgfpathlineto{\pgfpoint{60.806546\du}{13.507412\du}}
\pgfpathlineto{\pgfpoint{60.802847\du}{13.509093\du}}
\pgfpathlineto{\pgfpoint{60.799149\du}{13.510775\du}}
\pgfpathlineto{\pgfpoint{60.795787\du}{13.512456\du}}
\pgfpathlineto{\pgfpoint{60.791752\du}{13.514473\du}}
\pgfpathlineto{\pgfpoint{60.788390\du}{13.517163\du}}
\pgfpathlineto{\pgfpoint{60.785364\du}{13.519180\du}}
\pgfpathlineto{\pgfpoint{60.782338\du}{13.522206\du}}
\pgfpathlineto{\pgfpoint{60.778976\du}{13.524560\du}}
\pgfpathlineto{\pgfpoint{60.776622\du}{13.527586\du}}
\pgfpathlineto{\pgfpoint{60.773260\du}{13.530612\du}}
\pgfpathlineto{\pgfpoint{60.770906\du}{13.533974\du}}
\pgfpathlineto{\pgfpoint{60.768553\du}{13.537336\du}}
\pgfpathlineto{\pgfpoint{60.766199\du}{13.540698\du}}
\pgfpathlineto{\pgfpoint{60.764182\du}{13.544061\du}}
\pgfpathlineto{\pgfpoint{60.762501\du}{13.547759\du}}
\pgfpathlineto{\pgfpoint{60.760483\du}{13.551794\du}}
\pgfpathlineto{\pgfpoint{60.759138\du}{13.555492\du}}
\pgfpathlineto{\pgfpoint{60.757793\du}{13.559191\du}}
\pgfpathlineto{\pgfpoint{60.756449\du}{13.563225\du}}
\pgfpathlineto{\pgfpoint{60.755776\du}{13.567260\du}}
\pgfpathlineto{\pgfpoint{60.754767\du}{13.571631\du}}
\pgfpathlineto{\pgfpoint{60.754431\du}{13.575330\du}}
\pgfpathlineto{\pgfpoint{60.754095\du}{13.579700\du}}
\pgfpathlineto{\pgfpoint{60.754095\du}{13.583735\du}}
\pgfpathlineto{\pgfpoint{60.754095\du}{13.583735\du}}
\pgfpathlineto{\pgfpoint{60.754095\du}{13.588106\du}}
\pgfpathlineto{\pgfpoint{60.754431\du}{13.592141\du}}
\pgfpathlineto{\pgfpoint{60.754767\du}{13.596175\du}}
\pgfpathlineto{\pgfpoint{60.755776\du}{13.600210\du}}
\pgfpathlineto{\pgfpoint{60.756449\du}{13.603909\du}}
\pgfpathlineto{\pgfpoint{60.757793\du}{13.608279\du}}
\pgfpathlineto{\pgfpoint{60.759138\du}{13.611642\du}}
\pgfpathlineto{\pgfpoint{60.760483\du}{13.616013\du}}
\pgfpathlineto{\pgfpoint{60.762501\du}{13.619711\du}}
\pgfpathlineto{\pgfpoint{60.764182\du}{13.623409\du}}
\pgfpathlineto{\pgfpoint{60.766199\du}{13.626772\du}}
\pgfpathlineto{\pgfpoint{60.768553\du}{13.630134\du}}
\pgfpathlineto{\pgfpoint{60.770906\du}{13.633496\du}}
\pgfpathlineto{\pgfpoint{60.773260\du}{13.636858\du}}
\pgfpathlineto{\pgfpoint{60.776622\du}{13.639884\du}}
\pgfpathlineto{\pgfpoint{60.778976\du}{13.642910\du}}
\pgfpathlineto{\pgfpoint{60.782338\du}{13.645264\du}}
\pgfpathlineto{\pgfpoint{60.785364\du}{13.648290\du}}
\pgfpathlineto{\pgfpoint{60.788390\du}{13.650644\du}}
\pgfpathlineto{\pgfpoint{60.791752\du}{13.652997\du}}
\pgfpathlineto{\pgfpoint{60.795787\du}{13.655014\du}}
\pgfpathlineto{\pgfpoint{60.799149\du}{13.656696\du}}
\pgfpathlineto{\pgfpoint{60.802847\du}{13.658377\du}}
\pgfpathlineto{\pgfpoint{60.806546\du}{13.660058\du}}
\pgfpathlineto{\pgfpoint{60.810581\du}{13.661403\du}}
\pgfpathlineto{\pgfpoint{60.814615\du}{13.662748\du}}
\pgfpathlineto{\pgfpoint{60.818650\du}{13.663420\du}}
\pgfpathlineto{\pgfpoint{60.822348\du}{13.664093\du}}
\pgfpathlineto{\pgfpoint{60.826383\du}{13.664765\du}}
\pgfpathlineto{\pgfpoint{60.830082\du}{13.665101\du}}
\pgfpathlineto{\pgfpoint{60.834452\du}{13.665101\du}}
\pgfpathlineto{\pgfpoint{60.834452\du}{13.665101\du}}
\pgfpathlineto{\pgfpoint{60.838487\du}{13.665101\du}}
\pgfpathlineto{\pgfpoint{60.842522\du}{13.664765\du}}
\pgfpathlineto{\pgfpoint{60.846557\du}{13.664093\du}}
\pgfpathlineto{\pgfpoint{60.850255\du}{13.663420\du}}
\pgfpathlineto{\pgfpoint{60.854626\du}{13.662748\du}}
\pgfpathlineto{\pgfpoint{60.858661\du}{13.661403\du}}
\pgfpathlineto{\pgfpoint{60.862023\du}{13.660058\du}}
\pgfpathlineto{\pgfpoint{60.866057\du}{13.658377\du}}
\pgfpathlineto{\pgfpoint{60.869756\du}{13.656696\du}}
\pgfpathlineto{\pgfpoint{60.873454\du}{13.655014\du}}
\pgfpathlineto{\pgfpoint{60.876817\du}{13.652997\du}}
\pgfpathlineto{\pgfpoint{60.880179\du}{13.650644\du}}
\pgfpathlineto{\pgfpoint{60.883877\du}{13.648290\du}}
\pgfpathlineto{\pgfpoint{60.886567\du}{13.645264\du}}
\pgfpathlineto{\pgfpoint{60.890266\du}{13.642910\du}}
\pgfpathlineto{\pgfpoint{60.892619\du}{13.639884\du}}
\pgfpathlineto{\pgfpoint{60.895309\du}{13.636858\du}}
\pgfpathlineto{\pgfpoint{60.897999\du}{13.633496\du}}
\pgfpathlineto{\pgfpoint{60.900688\du}{13.630134\du}}
\pgfpathlineto{\pgfpoint{60.903042\du}{13.626772\du}}
\pgfpathlineto{\pgfpoint{60.905059\du}{13.623409\du}}
\pgfpathlineto{\pgfpoint{60.906404\du}{13.619711\du}}
\pgfpathlineto{\pgfpoint{60.908422\du}{13.616013\du}}
\pgfpathlineto{\pgfpoint{60.909767\du}{13.611642\du}}
\pgfpathlineto{\pgfpoint{60.911448\du}{13.608279\du}}
\pgfpathlineto{\pgfpoint{60.912120\du}{13.603909\du}}
\pgfpathlineto{\pgfpoint{60.913465\du}{13.600210\du}}
\pgfpathlineto{\pgfpoint{60.913801\du}{13.596175\du}}
\pgfpathlineto{\pgfpoint{60.914474\du}{13.592141\du}}
\pgfpathlineto{\pgfpoint{60.914810\du}{13.588106\du}}
\pgfpathlineto{\pgfpoint{60.914810\du}{13.583735\du}}
\pgfusepath{fill}
\pgfsetlinewidth{0.000000\du}
\pgfsetbuttcap
\pgfsetmiterjoin
\pgfsetdash{}{0pt}
\definecolor{dialinecolor}{rgb}{0.000000, 0.000000, 0.000000}
\pgfsetstrokecolor{dialinecolor}
\pgfpathmoveto{\pgfpoint{60.107537\du}{13.576674\du}}
\pgfpathlineto{\pgfpoint{60.811589\du}{13.576674\du}}
\pgfusepath{stroke}
\pgfsetlinewidth{0.000000\du}
\pgfsetbuttcap
\pgfsetmiterjoin
\pgfsetdash{}{0pt}
\definecolor{dialinecolor}{rgb}{1.000000, 1.000000, 1.000000}
\pgfsetfillcolor{dialinecolor}
\pgfpathmoveto{\pgfpoint{60.180498\du}{13.881629\du}}
\pgfpathlineto{\pgfpoint{60.180498\du}{13.877258\du}}
\pgfpathlineto{\pgfpoint{60.180161\du}{13.872887\du}}
\pgfpathlineto{\pgfpoint{60.179825\du}{13.869189\du}}
\pgfpathlineto{\pgfpoint{60.178817\du}{13.865154\du}}
\pgfpathlineto{\pgfpoint{60.178144\du}{13.861120\du}}
\pgfpathlineto{\pgfpoint{60.176799\du}{13.857085\du}}
\pgfpathlineto{\pgfpoint{60.175791\du}{13.853050\du}}
\pgfpathlineto{\pgfpoint{60.173773\du}{13.849015\du}}
\pgfpathlineto{\pgfpoint{60.172092\du}{13.845653\du}}
\pgfpathlineto{\pgfpoint{60.170411\du}{13.841619\du}}
\pgfpathlineto{\pgfpoint{60.168394\du}{13.838256\du}}
\pgfpathlineto{\pgfpoint{60.166376\du}{13.834894\du}}
\pgfpathlineto{\pgfpoint{60.163686\du}{13.831532\du}}
\pgfpathlineto{\pgfpoint{60.161333\du}{13.828170\du}}
\pgfpathlineto{\pgfpoint{60.158307\du}{13.825480\du}}
\pgfpathlineto{\pgfpoint{60.155281\du}{13.822454\du}}
\pgfpathlineto{\pgfpoint{60.152591\du}{13.819764\du}}
\pgfpathlineto{\pgfpoint{60.149565\du}{13.816738\du}}
\pgfpathlineto{\pgfpoint{60.146203\du}{13.814721\du}}
\pgfpathlineto{\pgfpoint{60.142841\du}{13.812367\du}}
\pgfpathlineto{\pgfpoint{60.138806\du}{13.810014\du}}
\pgfpathlineto{\pgfpoint{60.135780\du}{13.808332\du}}
\pgfpathlineto{\pgfpoint{60.131745\du}{13.806651\du}}
\pgfpathlineto{\pgfpoint{60.128047\du}{13.804970\du}}
\pgfpathlineto{\pgfpoint{60.123676\du}{13.803625\du}}
\pgfpathlineto{\pgfpoint{60.119977\du}{13.802617\du}}
\pgfpathlineto{\pgfpoint{60.116279\du}{13.801608\du}}
\pgfpathlineto{\pgfpoint{60.111908\du}{13.800936\du}}
\pgfpathlineto{\pgfpoint{60.107873\du}{13.800263\du}}
\pgfpathlineto{\pgfpoint{60.104175\du}{13.799927\du}}
\pgfpathlineto{\pgfpoint{60.100140\du}{13.799927\du}}
\pgfpathlineto{\pgfpoint{60.100140\du}{13.799927\du}}
\pgfpathlineto{\pgfpoint{60.095769\du}{13.799927\du}}
\pgfpathlineto{\pgfpoint{60.091735\du}{13.800263\du}}
\pgfpathlineto{\pgfpoint{60.087700\du}{13.800936\du}}
\pgfpathlineto{\pgfpoint{60.083329\du}{13.801608\du}}
\pgfpathlineto{\pgfpoint{60.079631\du}{13.802617\du}}
\pgfpathlineto{\pgfpoint{60.075932\du}{13.803625\du}}
\pgfpathlineto{\pgfpoint{60.071561\du}{13.804970\du}}
\pgfpathlineto{\pgfpoint{60.067863\du}{13.806651\du}}
\pgfpathlineto{\pgfpoint{60.064501\du}{13.808332\du}}
\pgfpathlineto{\pgfpoint{60.060802\du}{13.810014\du}}
\pgfpathlineto{\pgfpoint{60.056767\du}{13.812367\du}}
\pgfpathlineto{\pgfpoint{60.053741\du}{13.814721\du}}
\pgfpathlineto{\pgfpoint{60.050043\du}{13.816738\du}}
\pgfpathlineto{\pgfpoint{60.047689\du}{13.819764\du}}
\pgfpathlineto{\pgfpoint{60.044327\du}{13.822454\du}}
\pgfpathlineto{\pgfpoint{60.041301\du}{13.825480\du}}
\pgfpathlineto{\pgfpoint{60.038275\du}{13.828170\du}}
\pgfpathlineto{\pgfpoint{60.035922\du}{13.831532\du}}
\pgfpathlineto{\pgfpoint{60.033232\du}{13.834894\du}}
\pgfpathlineto{\pgfpoint{60.031214\du}{13.838256\du}}
\pgfpathlineto{\pgfpoint{60.029197\du}{13.841619\du}}
\pgfpathlineto{\pgfpoint{60.027516\du}{13.845653\du}}
\pgfpathlineto{\pgfpoint{60.025835\du}{13.849015\du}}
\pgfpathlineto{\pgfpoint{60.024490\du}{13.853050\du}}
\pgfpathlineto{\pgfpoint{60.022809\du}{13.857085\du}}
\pgfpathlineto{\pgfpoint{60.021464\du}{13.861120\du}}
\pgfpathlineto{\pgfpoint{60.020791\du}{13.865154\du}}
\pgfpathlineto{\pgfpoint{60.020119\du}{13.869189\du}}
\pgfpathlineto{\pgfpoint{60.019447\du}{13.872887\du}}
\pgfpathlineto{\pgfpoint{60.019110\du}{13.877258\du}}
\pgfpathlineto{\pgfpoint{60.019110\du}{13.881629\du}}
\pgfpathlineto{\pgfpoint{60.019110\du}{13.881629\du}}
\pgfpathlineto{\pgfpoint{60.019110\du}{13.885664\du}}
\pgfpathlineto{\pgfpoint{60.019447\du}{13.890035\du}}
\pgfpathlineto{\pgfpoint{60.020119\du}{13.893733\du}}
\pgfpathlineto{\pgfpoint{60.020791\du}{13.898104\du}}
\pgfpathlineto{\pgfpoint{60.021464\du}{13.901803\du}}
\pgfpathlineto{\pgfpoint{60.022809\du}{13.905837\du}}
\pgfpathlineto{\pgfpoint{60.024490\du}{13.909872\du}}
\pgfpathlineto{\pgfpoint{60.025835\du}{13.913570\du}}
\pgfpathlineto{\pgfpoint{60.027516\du}{13.917269\du}}
\pgfpathlineto{\pgfpoint{60.029197\du}{13.921304\du}}
\pgfpathlineto{\pgfpoint{60.031214\du}{13.924666\du}}
\pgfpathlineto{\pgfpoint{60.033232\du}{13.928028\du}}
\pgfpathlineto{\pgfpoint{60.035922\du}{13.931054\du}}
\pgfpathlineto{\pgfpoint{60.038275\du}{13.934752\du}}
\pgfpathlineto{\pgfpoint{60.041301\du}{13.937778\du}}
\pgfpathlineto{\pgfpoint{60.044327\du}{13.940468\du}}
\pgfpathlineto{\pgfpoint{60.047689\du}{13.943158\du}}
\pgfpathlineto{\pgfpoint{60.050043\du}{13.946184\du}}
\pgfpathlineto{\pgfpoint{60.053741\du}{13.947865\du}}
\pgfpathlineto{\pgfpoint{60.056767\du}{13.950891\du}}
\pgfpathlineto{\pgfpoint{60.060802\du}{13.952909\du}}
\pgfpathlineto{\pgfpoint{60.064501\du}{13.954590\du}}
\pgfpathlineto{\pgfpoint{60.067863\du}{13.956271\du}}
\pgfpathlineto{\pgfpoint{60.071561\du}{13.957952\du}}
\pgfpathlineto{\pgfpoint{60.075932\du}{13.959297\du}}
\pgfpathlineto{\pgfpoint{60.079631\du}{13.960305\du}}
\pgfpathlineto{\pgfpoint{60.083329\du}{13.961314\du}}
\pgfpathlineto{\pgfpoint{60.087700\du}{13.961987\du}}
\pgfpathlineto{\pgfpoint{60.091735\du}{13.962659\du}}
\pgfpathlineto{\pgfpoint{60.095769\du}{13.962995\du}}
\pgfpathlineto{\pgfpoint{60.100140\du}{13.962995\du}}
\pgfpathlineto{\pgfpoint{60.100140\du}{13.962995\du}}
\pgfpathlineto{\pgfpoint{60.104175\du}{13.962995\du}}
\pgfpathlineto{\pgfpoint{60.107873\du}{13.962659\du}}
\pgfpathlineto{\pgfpoint{60.111908\du}{13.961987\du}}
\pgfpathlineto{\pgfpoint{60.116279\du}{13.961314\du}}
\pgfpathlineto{\pgfpoint{60.119977\du}{13.960305\du}}
\pgfpathlineto{\pgfpoint{60.123676\du}{13.959297\du}}
\pgfpathlineto{\pgfpoint{60.128047\du}{13.957952\du}}
\pgfpathlineto{\pgfpoint{60.131745\du}{13.956271\du}}
\pgfpathlineto{\pgfpoint{60.135780\du}{13.954590\du}}
\pgfpathlineto{\pgfpoint{60.138806\du}{13.952909\du}}
\pgfpathlineto{\pgfpoint{60.142841\du}{13.950891\du}}
\pgfpathlineto{\pgfpoint{60.146203\du}{13.947865\du}}
\pgfpathlineto{\pgfpoint{60.149565\du}{13.946184\du}}
\pgfpathlineto{\pgfpoint{60.152591\du}{13.943158\du}}
\pgfpathlineto{\pgfpoint{60.155281\du}{13.940468\du}}
\pgfpathlineto{\pgfpoint{60.158307\du}{13.937778\du}}
\pgfpathlineto{\pgfpoint{60.161333\du}{13.934752\du}}
\pgfpathlineto{\pgfpoint{60.163686\du}{13.931054\du}}
\pgfpathlineto{\pgfpoint{60.166376\du}{13.928028\du}}
\pgfpathlineto{\pgfpoint{60.168394\du}{13.924666\du}}
\pgfpathlineto{\pgfpoint{60.170411\du}{13.921304\du}}
\pgfpathlineto{\pgfpoint{60.172092\du}{13.917269\du}}
\pgfpathlineto{\pgfpoint{60.173773\du}{13.913570\du}}
\pgfpathlineto{\pgfpoint{60.175791\du}{13.909872\du}}
\pgfpathlineto{\pgfpoint{60.176799\du}{13.905837\du}}
\pgfpathlineto{\pgfpoint{60.178144\du}{13.901803\du}}
\pgfpathlineto{\pgfpoint{60.178817\du}{13.898104\du}}
\pgfpathlineto{\pgfpoint{60.179825\du}{13.893733\du}}
\pgfpathlineto{\pgfpoint{60.180161\du}{13.890035\du}}
\pgfpathlineto{\pgfpoint{60.180498\du}{13.885664\du}}
\pgfpathlineto{\pgfpoint{60.180498\du}{13.881629\du}}
\pgfusepath{fill}
\pgfsetbuttcap
\pgfsetmiterjoin
\pgfsetdash{}{0pt}
\definecolor{dialinecolor}{rgb}{1.000000, 1.000000, 1.000000}
\pgfsetfillcolor{dialinecolor}
\pgfpathmoveto{\pgfpoint{60.914810\du}{13.881629\du}}
\pgfpathlineto{\pgfpoint{60.914810\du}{13.877258\du}}
\pgfpathlineto{\pgfpoint{60.914474\du}{13.872887\du}}
\pgfpathlineto{\pgfpoint{60.913801\du}{13.869189\du}}
\pgfpathlineto{\pgfpoint{60.913465\du}{13.865154\du}}
\pgfpathlineto{\pgfpoint{60.912120\du}{13.861120\du}}
\pgfpathlineto{\pgfpoint{60.911448\du}{13.857085\du}}
\pgfpathlineto{\pgfpoint{60.909767\du}{13.853050\du}}
\pgfpathlineto{\pgfpoint{60.908422\du}{13.849015\du}}
\pgfpathlineto{\pgfpoint{60.906404\du}{13.845653\du}}
\pgfpathlineto{\pgfpoint{60.905059\du}{13.841619\du}}
\pgfpathlineto{\pgfpoint{60.903042\du}{13.838256\du}}
\pgfpathlineto{\pgfpoint{60.900688\du}{13.834894\du}}
\pgfpathlineto{\pgfpoint{60.897999\du}{13.831532\du}}
\pgfpathlineto{\pgfpoint{60.895309\du}{13.828170\du}}
\pgfpathlineto{\pgfpoint{60.892619\du}{13.825480\du}}
\pgfpathlineto{\pgfpoint{60.890266\du}{13.822454\du}}
\pgfpathlineto{\pgfpoint{60.886567\du}{13.819764\du}}
\pgfpathlineto{\pgfpoint{60.883877\du}{13.816738\du}}
\pgfpathlineto{\pgfpoint{60.880179\du}{13.814721\du}}
\pgfpathlineto{\pgfpoint{60.876817\du}{13.812367\du}}
\pgfpathlineto{\pgfpoint{60.873454\du}{13.810014\du}}
\pgfpathlineto{\pgfpoint{60.869756\du}{13.808332\du}}
\pgfpathlineto{\pgfpoint{60.866057\du}{13.806651\du}}
\pgfpathlineto{\pgfpoint{60.862023\du}{13.804970\du}}
\pgfpathlineto{\pgfpoint{60.858661\du}{13.803625\du}}
\pgfpathlineto{\pgfpoint{60.854626\du}{13.802617\du}}
\pgfpathlineto{\pgfpoint{60.850255\du}{13.801608\du}}
\pgfpathlineto{\pgfpoint{60.846557\du}{13.800936\du}}
\pgfpathlineto{\pgfpoint{60.842522\du}{13.800263\du}}
\pgfpathlineto{\pgfpoint{60.838487\du}{13.799927\du}}
\pgfpathlineto{\pgfpoint{60.834452\du}{13.799927\du}}
\pgfpathlineto{\pgfpoint{60.834452\du}{13.799927\du}}
\pgfpathlineto{\pgfpoint{60.830082\du}{13.799927\du}}
\pgfpathlineto{\pgfpoint{60.826383\du}{13.800263\du}}
\pgfpathlineto{\pgfpoint{60.822348\du}{13.800936\du}}
\pgfpathlineto{\pgfpoint{60.818650\du}{13.801608\du}}
\pgfpathlineto{\pgfpoint{60.814615\du}{13.802617\du}}
\pgfpathlineto{\pgfpoint{60.810581\du}{13.803625\du}}
\pgfpathlineto{\pgfpoint{60.806546\du}{13.804970\du}}
\pgfpathlineto{\pgfpoint{60.802847\du}{13.806651\du}}
\pgfpathlineto{\pgfpoint{60.799149\du}{13.808332\du}}
\pgfpathlineto{\pgfpoint{60.795787\du}{13.810014\du}}
\pgfpathlineto{\pgfpoint{60.791752\du}{13.812367\du}}
\pgfpathlineto{\pgfpoint{60.788390\du}{13.814721\du}}
\pgfpathlineto{\pgfpoint{60.785364\du}{13.816738\du}}
\pgfpathlineto{\pgfpoint{60.782338\du}{13.819764\du}}
\pgfpathlineto{\pgfpoint{60.778976\du}{13.822454\du}}
\pgfpathlineto{\pgfpoint{60.776622\du}{13.825480\du}}
\pgfpathlineto{\pgfpoint{60.773260\du}{13.828170\du}}
\pgfpathlineto{\pgfpoint{60.770906\du}{13.831532\du}}
\pgfpathlineto{\pgfpoint{60.768553\du}{13.834894\du}}
\pgfpathlineto{\pgfpoint{60.766199\du}{13.838256\du}}
\pgfpathlineto{\pgfpoint{60.764182\du}{13.841619\du}}
\pgfpathlineto{\pgfpoint{60.762501\du}{13.845653\du}}
\pgfpathlineto{\pgfpoint{60.760483\du}{13.849015\du}}
\pgfpathlineto{\pgfpoint{60.759138\du}{13.853050\du}}
\pgfpathlineto{\pgfpoint{60.757793\du}{13.857085\du}}
\pgfpathlineto{\pgfpoint{60.756449\du}{13.861120\du}}
\pgfpathlineto{\pgfpoint{60.755776\du}{13.865154\du}}
\pgfpathlineto{\pgfpoint{60.754767\du}{13.869189\du}}
\pgfpathlineto{\pgfpoint{60.754431\du}{13.872887\du}}
\pgfpathlineto{\pgfpoint{60.754095\du}{13.877258\du}}
\pgfpathlineto{\pgfpoint{60.754095\du}{13.881629\du}}
\pgfpathlineto{\pgfpoint{60.754095\du}{13.881629\du}}
\pgfpathlineto{\pgfpoint{60.754095\du}{13.885664\du}}
\pgfpathlineto{\pgfpoint{60.754431\du}{13.890035\du}}
\pgfpathlineto{\pgfpoint{60.754767\du}{13.893733\du}}
\pgfpathlineto{\pgfpoint{60.755776\du}{13.898104\du}}
\pgfpathlineto{\pgfpoint{60.756449\du}{13.901803\du}}
\pgfpathlineto{\pgfpoint{60.757793\du}{13.905837\du}}
\pgfpathlineto{\pgfpoint{60.759138\du}{13.909872\du}}
\pgfpathlineto{\pgfpoint{60.760483\du}{13.913570\du}}
\pgfpathlineto{\pgfpoint{60.762501\du}{13.917269\du}}
\pgfpathlineto{\pgfpoint{60.764182\du}{13.921304\du}}
\pgfpathlineto{\pgfpoint{60.766199\du}{13.924666\du}}
\pgfpathlineto{\pgfpoint{60.768553\du}{13.928028\du}}
\pgfpathlineto{\pgfpoint{60.770906\du}{13.931054\du}}
\pgfpathlineto{\pgfpoint{60.773260\du}{13.934752\du}}
\pgfpathlineto{\pgfpoint{60.776622\du}{13.937778\du}}
\pgfpathlineto{\pgfpoint{60.778976\du}{13.940468\du}}
\pgfpathlineto{\pgfpoint{60.782338\du}{13.943158\du}}
\pgfpathlineto{\pgfpoint{60.785364\du}{13.946184\du}}
\pgfpathlineto{\pgfpoint{60.788390\du}{13.947865\du}}
\pgfpathlineto{\pgfpoint{60.791752\du}{13.950891\du}}
\pgfpathlineto{\pgfpoint{60.795787\du}{13.952909\du}}
\pgfpathlineto{\pgfpoint{60.799149\du}{13.954590\du}}
\pgfpathlineto{\pgfpoint{60.802847\du}{13.956271\du}}
\pgfpathlineto{\pgfpoint{60.806546\du}{13.957952\du}}
\pgfpathlineto{\pgfpoint{60.810581\du}{13.959297\du}}
\pgfpathlineto{\pgfpoint{60.814615\du}{13.960305\du}}
\pgfpathlineto{\pgfpoint{60.818650\du}{13.961314\du}}
\pgfpathlineto{\pgfpoint{60.822348\du}{13.961987\du}}
\pgfpathlineto{\pgfpoint{60.826383\du}{13.962659\du}}
\pgfpathlineto{\pgfpoint{60.830082\du}{13.962995\du}}
\pgfpathlineto{\pgfpoint{60.834452\du}{13.962995\du}}
\pgfpathlineto{\pgfpoint{60.834452\du}{13.962995\du}}
\pgfpathlineto{\pgfpoint{60.838487\du}{13.962995\du}}
\pgfpathlineto{\pgfpoint{60.842522\du}{13.962659\du}}
\pgfpathlineto{\pgfpoint{60.846557\du}{13.961987\du}}
\pgfpathlineto{\pgfpoint{60.850255\du}{13.961314\du}}
\pgfpathlineto{\pgfpoint{60.854626\du}{13.960305\du}}
\pgfpathlineto{\pgfpoint{60.858661\du}{13.959297\du}}
\pgfpathlineto{\pgfpoint{60.862023\du}{13.957952\du}}
\pgfpathlineto{\pgfpoint{60.866057\du}{13.956271\du}}
\pgfpathlineto{\pgfpoint{60.869756\du}{13.954590\du}}
\pgfpathlineto{\pgfpoint{60.873454\du}{13.952909\du}}
\pgfpathlineto{\pgfpoint{60.876817\du}{13.950891\du}}
\pgfpathlineto{\pgfpoint{60.880179\du}{13.947865\du}}
\pgfpathlineto{\pgfpoint{60.883877\du}{13.946184\du}}
\pgfpathlineto{\pgfpoint{60.886567\du}{13.943158\du}}
\pgfpathlineto{\pgfpoint{60.890266\du}{13.940468\du}}
\pgfpathlineto{\pgfpoint{60.892619\du}{13.937778\du}}
\pgfpathlineto{\pgfpoint{60.895309\du}{13.934752\du}}
\pgfpathlineto{\pgfpoint{60.897999\du}{13.931054\du}}
\pgfpathlineto{\pgfpoint{60.900688\du}{13.928028\du}}
\pgfpathlineto{\pgfpoint{60.903042\du}{13.924666\du}}
\pgfpathlineto{\pgfpoint{60.905059\du}{13.921304\du}}
\pgfpathlineto{\pgfpoint{60.906404\du}{13.917269\du}}
\pgfpathlineto{\pgfpoint{60.908422\du}{13.913570\du}}
\pgfpathlineto{\pgfpoint{60.909767\du}{13.909872\du}}
\pgfpathlineto{\pgfpoint{60.911448\du}{13.905837\du}}
\pgfpathlineto{\pgfpoint{60.912120\du}{13.901803\du}}
\pgfpathlineto{\pgfpoint{60.913465\du}{13.898104\du}}
\pgfpathlineto{\pgfpoint{60.913801\du}{13.893733\du}}
\pgfpathlineto{\pgfpoint{60.914474\du}{13.890035\du}}
\pgfpathlineto{\pgfpoint{60.914810\du}{13.885664\du}}
\pgfpathlineto{\pgfpoint{60.914810\du}{13.881629\du}}
\pgfusepath{fill}
\pgfsetlinewidth{0.000000\du}
\pgfsetbuttcap
\pgfsetmiterjoin
\pgfsetdash{}{0pt}
\definecolor{dialinecolor}{rgb}{0.000000, 0.000000, 0.000000}
\pgfsetstrokecolor{dialinecolor}
\pgfpathmoveto{\pgfpoint{60.107537\du}{13.873896\du}}
\pgfpathlineto{\pgfpoint{60.811589\du}{13.873896\du}}
\pgfusepath{stroke}
\pgfsetlinewidth{0.000000\du}
\pgfsetbuttcap
\pgfsetmiterjoin
\pgfsetdash{}{0pt}
\definecolor{dialinecolor}{rgb}{1.000000, 1.000000, 1.000000}
\pgfsetfillcolor{dialinecolor}
\pgfpathmoveto{\pgfpoint{60.180498\du}{14.178851\du}}
\pgfpathlineto{\pgfpoint{60.180498\du}{14.174480\du}}
\pgfpathlineto{\pgfpoint{60.180161\du}{14.170445\du}}
\pgfpathlineto{\pgfpoint{60.179825\du}{14.166410\du}}
\pgfpathlineto{\pgfpoint{60.178817\du}{14.162376\du}}
\pgfpathlineto{\pgfpoint{60.178144\du}{14.158005\du}}
\pgfpathlineto{\pgfpoint{60.176799\du}{14.154306\du}}
\pgfpathlineto{\pgfpoint{60.175791\du}{14.150608\du}}
\pgfpathlineto{\pgfpoint{60.173773\du}{14.146237\du}}
\pgfpathlineto{\pgfpoint{60.172092\du}{14.142875\du}}
\pgfpathlineto{\pgfpoint{60.170411\du}{14.139176\du}}
\pgfpathlineto{\pgfpoint{60.168394\du}{14.134805\du}}
\pgfpathlineto{\pgfpoint{60.166376\du}{14.131779\du}}
\pgfpathlineto{\pgfpoint{60.163686\du}{14.128417\du}}
\pgfpathlineto{\pgfpoint{60.161333\du}{14.125391\du}}
\pgfpathlineto{\pgfpoint{60.158307\du}{14.122029\du}}
\pgfpathlineto{\pgfpoint{60.155281\du}{14.119339\du}}
\pgfpathlineto{\pgfpoint{60.152591\du}{14.116649\du}}
\pgfpathlineto{\pgfpoint{60.149565\du}{14.114296\du}}
\pgfpathlineto{\pgfpoint{60.146203\du}{14.111606\du}}
\pgfpathlineto{\pgfpoint{60.142841\du}{14.109252\du}}
\pgfpathlineto{\pgfpoint{60.138806\du}{14.106899\du}}
\pgfpathlineto{\pgfpoint{60.135780\du}{14.105218\du}}
\pgfpathlineto{\pgfpoint{60.131745\du}{14.103537\du}}
\pgfpathlineto{\pgfpoint{60.128047\du}{14.101856\du}}
\pgfpathlineto{\pgfpoint{60.123676\du}{14.100511\du}}
\pgfpathlineto{\pgfpoint{60.119977\du}{14.099166\du}}
\pgfpathlineto{\pgfpoint{60.116279\du}{14.098493\du}}
\pgfpathlineto{\pgfpoint{60.111908\du}{14.097821\du}}
\pgfpathlineto{\pgfpoint{60.107873\du}{14.097148\du}}
\pgfpathlineto{\pgfpoint{60.104175\du}{14.096812\du}}
\pgfpathlineto{\pgfpoint{60.100140\du}{14.096812\du}}
\pgfpathlineto{\pgfpoint{60.100140\du}{14.096812\du}}
\pgfpathlineto{\pgfpoint{60.095769\du}{14.096812\du}}
\pgfpathlineto{\pgfpoint{60.091735\du}{14.097148\du}}
\pgfpathlineto{\pgfpoint{60.087700\du}{14.097821\du}}
\pgfpathlineto{\pgfpoint{60.083329\du}{14.098493\du}}
\pgfpathlineto{\pgfpoint{60.079631\du}{14.099166\du}}
\pgfpathlineto{\pgfpoint{60.075932\du}{14.100511\du}}
\pgfpathlineto{\pgfpoint{60.071561\du}{14.101856\du}}
\pgfpathlineto{\pgfpoint{60.067863\du}{14.103537\du}}
\pgfpathlineto{\pgfpoint{60.064501\du}{14.105218\du}}
\pgfpathlineto{\pgfpoint{60.060802\du}{14.106899\du}}
\pgfpathlineto{\pgfpoint{60.056767\du}{14.109252\du}}
\pgfpathlineto{\pgfpoint{60.053741\du}{14.111606\du}}
\pgfpathlineto{\pgfpoint{60.050043\du}{14.114296\du}}
\pgfpathlineto{\pgfpoint{60.047689\du}{14.116649\du}}
\pgfpathlineto{\pgfpoint{60.044327\du}{14.119339\du}}
\pgfpathlineto{\pgfpoint{60.041301\du}{14.122029\du}}
\pgfpathlineto{\pgfpoint{60.038275\du}{14.125391\du}}
\pgfpathlineto{\pgfpoint{60.035922\du}{14.128417\du}}
\pgfpathlineto{\pgfpoint{60.033232\du}{14.131779\du}}
\pgfpathlineto{\pgfpoint{60.031214\du}{14.134805\du}}
\pgfpathlineto{\pgfpoint{60.029197\du}{14.139176\du}}
\pgfpathlineto{\pgfpoint{60.027516\du}{14.142875\du}}
\pgfpathlineto{\pgfpoint{60.025835\du}{14.146237\du}}
\pgfpathlineto{\pgfpoint{60.024490\du}{14.150608\du}}
\pgfpathlineto{\pgfpoint{60.022809\du}{14.154306\du}}
\pgfpathlineto{\pgfpoint{60.021464\du}{14.158005\du}}
\pgfpathlineto{\pgfpoint{60.020791\du}{14.162376\du}}
\pgfpathlineto{\pgfpoint{60.020119\du}{14.166410\du}}
\pgfpathlineto{\pgfpoint{60.019447\du}{14.170445\du}}
\pgfpathlineto{\pgfpoint{60.019110\du}{14.174480\du}}
\pgfpathlineto{\pgfpoint{60.019110\du}{14.178851\du}}
\pgfpathlineto{\pgfpoint{60.019110\du}{14.178851\du}}
\pgfpathlineto{\pgfpoint{60.019110\du}{14.182885\du}}
\pgfpathlineto{\pgfpoint{60.019447\du}{14.187256\du}}
\pgfpathlineto{\pgfpoint{60.020119\du}{14.190955\du}}
\pgfpathlineto{\pgfpoint{60.020791\du}{14.195326\du}}
\pgfpathlineto{\pgfpoint{60.021464\du}{14.199360\du}}
\pgfpathlineto{\pgfpoint{60.022809\du}{14.203395\du}}
\pgfpathlineto{\pgfpoint{60.024490\du}{14.207094\du}}
\pgfpathlineto{\pgfpoint{60.025835\du}{14.211128\du}}
\pgfpathlineto{\pgfpoint{60.027516\du}{14.214827\du}}
\pgfpathlineto{\pgfpoint{60.029197\du}{14.218525\du}}
\pgfpathlineto{\pgfpoint{60.031214\du}{14.222224\du}}
\pgfpathlineto{\pgfpoint{60.033232\du}{14.225250\du}}
\pgfpathlineto{\pgfpoint{60.035922\du}{14.228948\du}}
\pgfpathlineto{\pgfpoint{60.038275\du}{14.231974\du}}
\pgfpathlineto{\pgfpoint{60.041301\du}{14.235336\du}}
\pgfpathlineto{\pgfpoint{60.044327\du}{14.238362\du}}
\pgfpathlineto{\pgfpoint{60.047689\du}{14.240716\du}}
\pgfpathlineto{\pgfpoint{60.050043\du}{14.243406\du}}
\pgfpathlineto{\pgfpoint{60.053741\du}{14.245759\du}}
\pgfpathlineto{\pgfpoint{60.056767\du}{14.248449\du}}
\pgfpathlineto{\pgfpoint{60.060802\du}{14.250466\du}}
\pgfpathlineto{\pgfpoint{60.064501\du}{14.252147\du}}
\pgfpathlineto{\pgfpoint{60.067863\du}{14.253829\du}}
\pgfpathlineto{\pgfpoint{60.071561\du}{14.255510\du}}
\pgfpathlineto{\pgfpoint{60.075932\du}{14.256855\du}}
\pgfpathlineto{\pgfpoint{60.079631\du}{14.258200\du}}
\pgfpathlineto{\pgfpoint{60.083329\du}{14.258872\du}}
\pgfpathlineto{\pgfpoint{60.087700\du}{14.259881\du}}
\pgfpathlineto{\pgfpoint{60.091735\du}{14.260217\du}}
\pgfpathlineto{\pgfpoint{60.095769\du}{14.260553\du}}
\pgfpathlineto{\pgfpoint{60.100140\du}{14.260553\du}}
\pgfpathlineto{\pgfpoint{60.100140\du}{14.260553\du}}
\pgfpathlineto{\pgfpoint{60.104175\du}{14.260553\du}}
\pgfpathlineto{\pgfpoint{60.107873\du}{14.260217\du}}
\pgfpathlineto{\pgfpoint{60.111908\du}{14.259881\du}}
\pgfpathlineto{\pgfpoint{60.116279\du}{14.258872\du}}
\pgfpathlineto{\pgfpoint{60.119977\du}{14.258200\du}}
\pgfpathlineto{\pgfpoint{60.123676\du}{14.256855\du}}
\pgfpathlineto{\pgfpoint{60.128047\du}{14.255510\du}}
\pgfpathlineto{\pgfpoint{60.131745\du}{14.253829\du}}
\pgfpathlineto{\pgfpoint{60.135780\du}{14.252147\du}}
\pgfpathlineto{\pgfpoint{60.138806\du}{14.250466\du}}
\pgfpathlineto{\pgfpoint{60.142841\du}{14.248449\du}}
\pgfpathlineto{\pgfpoint{60.146203\du}{14.245759\du}}
\pgfpathlineto{\pgfpoint{60.149565\du}{14.243406\du}}
\pgfpathlineto{\pgfpoint{60.152591\du}{14.240716\du}}
\pgfpathlineto{\pgfpoint{60.155281\du}{14.238362\du}}
\pgfpathlineto{\pgfpoint{60.158307\du}{14.235336\du}}
\pgfpathlineto{\pgfpoint{60.161333\du}{14.231974\du}}
\pgfpathlineto{\pgfpoint{60.163686\du}{14.228948\du}}
\pgfpathlineto{\pgfpoint{60.166376\du}{14.225250\du}}
\pgfpathlineto{\pgfpoint{60.168394\du}{14.222224\du}}
\pgfpathlineto{\pgfpoint{60.170411\du}{14.218525\du}}
\pgfpathlineto{\pgfpoint{60.172092\du}{14.214827\du}}
\pgfpathlineto{\pgfpoint{60.173773\du}{14.211128\du}}
\pgfpathlineto{\pgfpoint{60.175791\du}{14.207094\du}}
\pgfpathlineto{\pgfpoint{60.176799\du}{14.203395\du}}
\pgfpathlineto{\pgfpoint{60.178144\du}{14.199360\du}}
\pgfpathlineto{\pgfpoint{60.178817\du}{14.195326\du}}
\pgfpathlineto{\pgfpoint{60.179825\du}{14.190955\du}}
\pgfpathlineto{\pgfpoint{60.180161\du}{14.187256\du}}
\pgfpathlineto{\pgfpoint{60.180498\du}{14.182885\du}}
\pgfpathlineto{\pgfpoint{60.180498\du}{14.178851\du}}
\pgfusepath{fill}
\pgfsetbuttcap
\pgfsetmiterjoin
\pgfsetdash{}{0pt}
\definecolor{dialinecolor}{rgb}{1.000000, 1.000000, 1.000000}
\pgfsetfillcolor{dialinecolor}
\pgfpathmoveto{\pgfpoint{60.914810\du}{14.178851\du}}
\pgfpathlineto{\pgfpoint{60.914810\du}{14.174480\du}}
\pgfpathlineto{\pgfpoint{60.914474\du}{14.170445\du}}
\pgfpathlineto{\pgfpoint{60.913801\du}{14.166410\du}}
\pgfpathlineto{\pgfpoint{60.913465\du}{14.162376\du}}
\pgfpathlineto{\pgfpoint{60.912120\du}{14.158005\du}}
\pgfpathlineto{\pgfpoint{60.911448\du}{14.154306\du}}
\pgfpathlineto{\pgfpoint{60.909767\du}{14.150608\du}}
\pgfpathlineto{\pgfpoint{60.908422\du}{14.146237\du}}
\pgfpathlineto{\pgfpoint{60.906404\du}{14.142875\du}}
\pgfpathlineto{\pgfpoint{60.905059\du}{14.139176\du}}
\pgfpathlineto{\pgfpoint{60.903042\du}{14.134805\du}}
\pgfpathlineto{\pgfpoint{60.900688\du}{14.131779\du}}
\pgfpathlineto{\pgfpoint{60.897999\du}{14.128417\du}}
\pgfpathlineto{\pgfpoint{60.895309\du}{14.125391\du}}
\pgfpathlineto{\pgfpoint{60.892619\du}{14.122029\du}}
\pgfpathlineto{\pgfpoint{60.890266\du}{14.119339\du}}
\pgfpathlineto{\pgfpoint{60.886567\du}{14.116649\du}}
\pgfpathlineto{\pgfpoint{60.883877\du}{14.114296\du}}
\pgfpathlineto{\pgfpoint{60.880179\du}{14.111606\du}}
\pgfpathlineto{\pgfpoint{60.876817\du}{14.109252\du}}
\pgfpathlineto{\pgfpoint{60.873454\du}{14.106899\du}}
\pgfpathlineto{\pgfpoint{60.869756\du}{14.105218\du}}
\pgfpathlineto{\pgfpoint{60.866057\du}{14.103537\du}}
\pgfpathlineto{\pgfpoint{60.862023\du}{14.101856\du}}
\pgfpathlineto{\pgfpoint{60.858661\du}{14.100511\du}}
\pgfpathlineto{\pgfpoint{60.854626\du}{14.099166\du}}
\pgfpathlineto{\pgfpoint{60.850255\du}{14.098493\du}}
\pgfpathlineto{\pgfpoint{60.846557\du}{14.097821\du}}
\pgfpathlineto{\pgfpoint{60.842522\du}{14.097148\du}}
\pgfpathlineto{\pgfpoint{60.838487\du}{14.096812\du}}
\pgfpathlineto{\pgfpoint{60.834452\du}{14.096812\du}}
\pgfpathlineto{\pgfpoint{60.834452\du}{14.096812\du}}
\pgfpathlineto{\pgfpoint{60.830082\du}{14.096812\du}}
\pgfpathlineto{\pgfpoint{60.826383\du}{14.097148\du}}
\pgfpathlineto{\pgfpoint{60.822348\du}{14.097821\du}}
\pgfpathlineto{\pgfpoint{60.818650\du}{14.098493\du}}
\pgfpathlineto{\pgfpoint{60.814615\du}{14.099166\du}}
\pgfpathlineto{\pgfpoint{60.810581\du}{14.100511\du}}
\pgfpathlineto{\pgfpoint{60.806546\du}{14.101856\du}}
\pgfpathlineto{\pgfpoint{60.802847\du}{14.103537\du}}
\pgfpathlineto{\pgfpoint{60.799149\du}{14.105218\du}}
\pgfpathlineto{\pgfpoint{60.795787\du}{14.106899\du}}
\pgfpathlineto{\pgfpoint{60.791752\du}{14.109252\du}}
\pgfpathlineto{\pgfpoint{60.788390\du}{14.111606\du}}
\pgfpathlineto{\pgfpoint{60.785364\du}{14.114296\du}}
\pgfpathlineto{\pgfpoint{60.782338\du}{14.116649\du}}
\pgfpathlineto{\pgfpoint{60.778976\du}{14.119339\du}}
\pgfpathlineto{\pgfpoint{60.776622\du}{14.122029\du}}
\pgfpathlineto{\pgfpoint{60.773260\du}{14.125391\du}}
\pgfpathlineto{\pgfpoint{60.770906\du}{14.128417\du}}
\pgfpathlineto{\pgfpoint{60.768553\du}{14.131779\du}}
\pgfpathlineto{\pgfpoint{60.766199\du}{14.134805\du}}
\pgfpathlineto{\pgfpoint{60.764182\du}{14.139176\du}}
\pgfpathlineto{\pgfpoint{60.762501\du}{14.142875\du}}
\pgfpathlineto{\pgfpoint{60.760483\du}{14.146237\du}}
\pgfpathlineto{\pgfpoint{60.759138\du}{14.150608\du}}
\pgfpathlineto{\pgfpoint{60.757793\du}{14.154306\du}}
\pgfpathlineto{\pgfpoint{60.756449\du}{14.158005\du}}
\pgfpathlineto{\pgfpoint{60.755776\du}{14.162376\du}}
\pgfpathlineto{\pgfpoint{60.754767\du}{14.166410\du}}
\pgfpathlineto{\pgfpoint{60.754431\du}{14.170445\du}}
\pgfpathlineto{\pgfpoint{60.754095\du}{14.174480\du}}
\pgfpathlineto{\pgfpoint{60.754095\du}{14.178851\du}}
\pgfpathlineto{\pgfpoint{60.754095\du}{14.178851\du}}
\pgfpathlineto{\pgfpoint{60.754095\du}{14.182885\du}}
\pgfpathlineto{\pgfpoint{60.754431\du}{14.187256\du}}
\pgfpathlineto{\pgfpoint{60.754767\du}{14.190955\du}}
\pgfpathlineto{\pgfpoint{60.755776\du}{14.195326\du}}
\pgfpathlineto{\pgfpoint{60.756449\du}{14.199360\du}}
\pgfpathlineto{\pgfpoint{60.757793\du}{14.203395\du}}
\pgfpathlineto{\pgfpoint{60.759138\du}{14.207094\du}}
\pgfpathlineto{\pgfpoint{60.760483\du}{14.211128\du}}
\pgfpathlineto{\pgfpoint{60.762501\du}{14.214827\du}}
\pgfpathlineto{\pgfpoint{60.764182\du}{14.218525\du}}
\pgfpathlineto{\pgfpoint{60.766199\du}{14.222224\du}}
\pgfpathlineto{\pgfpoint{60.768553\du}{14.225250\du}}
\pgfpathlineto{\pgfpoint{60.770906\du}{14.228948\du}}
\pgfpathlineto{\pgfpoint{60.773260\du}{14.231974\du}}
\pgfpathlineto{\pgfpoint{60.776622\du}{14.235336\du}}
\pgfpathlineto{\pgfpoint{60.778976\du}{14.238362\du}}
\pgfpathlineto{\pgfpoint{60.782338\du}{14.240716\du}}
\pgfpathlineto{\pgfpoint{60.785364\du}{14.243406\du}}
\pgfpathlineto{\pgfpoint{60.788390\du}{14.245759\du}}
\pgfpathlineto{\pgfpoint{60.791752\du}{14.248449\du}}
\pgfpathlineto{\pgfpoint{60.795787\du}{14.250466\du}}
\pgfpathlineto{\pgfpoint{60.799149\du}{14.252147\du}}
\pgfpathlineto{\pgfpoint{60.802847\du}{14.253829\du}}
\pgfpathlineto{\pgfpoint{60.806546\du}{14.255510\du}}
\pgfpathlineto{\pgfpoint{60.810581\du}{14.256855\du}}
\pgfpathlineto{\pgfpoint{60.814615\du}{14.258200\du}}
\pgfpathlineto{\pgfpoint{60.818650\du}{14.258872\du}}
\pgfpathlineto{\pgfpoint{60.822348\du}{14.259881\du}}
\pgfpathlineto{\pgfpoint{60.826383\du}{14.260217\du}}
\pgfpathlineto{\pgfpoint{60.830082\du}{14.260553\du}}
\pgfpathlineto{\pgfpoint{60.834452\du}{14.260553\du}}
\pgfpathlineto{\pgfpoint{60.834452\du}{14.260553\du}}
\pgfpathlineto{\pgfpoint{60.838487\du}{14.260553\du}}
\pgfpathlineto{\pgfpoint{60.842522\du}{14.260217\du}}
\pgfpathlineto{\pgfpoint{60.846557\du}{14.259881\du}}
\pgfpathlineto{\pgfpoint{60.850255\du}{14.258872\du}}
\pgfpathlineto{\pgfpoint{60.854626\du}{14.258200\du}}
\pgfpathlineto{\pgfpoint{60.858661\du}{14.256855\du}}
\pgfpathlineto{\pgfpoint{60.862023\du}{14.255510\du}}
\pgfpathlineto{\pgfpoint{60.866057\du}{14.253829\du}}
\pgfpathlineto{\pgfpoint{60.869756\du}{14.252147\du}}
\pgfpathlineto{\pgfpoint{60.873454\du}{14.250466\du}}
\pgfpathlineto{\pgfpoint{60.876817\du}{14.248449\du}}
\pgfpathlineto{\pgfpoint{60.880179\du}{14.245759\du}}
\pgfpathlineto{\pgfpoint{60.883877\du}{14.243406\du}}
\pgfpathlineto{\pgfpoint{60.886567\du}{14.240716\du}}
\pgfpathlineto{\pgfpoint{60.890266\du}{14.238362\du}}
\pgfpathlineto{\pgfpoint{60.892619\du}{14.235336\du}}
\pgfpathlineto{\pgfpoint{60.895309\du}{14.231974\du}}
\pgfpathlineto{\pgfpoint{60.897999\du}{14.228948\du}}
\pgfpathlineto{\pgfpoint{60.900688\du}{14.225250\du}}
\pgfpathlineto{\pgfpoint{60.903042\du}{14.222224\du}}
\pgfpathlineto{\pgfpoint{60.905059\du}{14.218525\du}}
\pgfpathlineto{\pgfpoint{60.906404\du}{14.214827\du}}
\pgfpathlineto{\pgfpoint{60.908422\du}{14.211128\du}}
\pgfpathlineto{\pgfpoint{60.909767\du}{14.207094\du}}
\pgfpathlineto{\pgfpoint{60.911448\du}{14.203395\du}}
\pgfpathlineto{\pgfpoint{60.912120\du}{14.199360\du}}
\pgfpathlineto{\pgfpoint{60.913465\du}{14.195326\du}}
\pgfpathlineto{\pgfpoint{60.913801\du}{14.190955\du}}
\pgfpathlineto{\pgfpoint{60.914474\du}{14.187256\du}}
\pgfpathlineto{\pgfpoint{60.914810\du}{14.182885\du}}
\pgfpathlineto{\pgfpoint{60.914810\du}{14.178851\du}}
\pgfusepath{fill}
\pgfsetlinewidth{0.000000\du}
\pgfsetbuttcap
\pgfsetmiterjoin
\pgfsetdash{}{0pt}
\definecolor{dialinecolor}{rgb}{0.000000, 0.000000, 0.000000}
\pgfsetstrokecolor{dialinecolor}
\pgfpathmoveto{\pgfpoint{60.107537\du}{14.171454\du}}
\pgfpathlineto{\pgfpoint{60.811589\du}{14.171454\du}}
\pgfusepath{stroke}
\pgfsetlinewidth{0.000000\du}
\pgfsetbuttcap
\pgfsetmiterjoin
\pgfsetdash{}{0pt}
\definecolor{dialinecolor}{rgb}{1.000000, 1.000000, 1.000000}
\pgfsetfillcolor{dialinecolor}
\pgfpathmoveto{\pgfpoint{60.254803\du}{12.415358\du}}
\pgfpathlineto{\pgfpoint{60.254803\du}{12.757634\du}}
\pgfpathlineto{\pgfpoint{60.754095\du}{12.757634\du}}
\pgfpathlineto{\pgfpoint{60.754095\du}{12.415358\du}}
\pgfpathlineto{\pgfpoint{60.254803\du}{12.415358\du}}
\pgfusepath{fill}
\pgfsetbuttcap
\pgfsetmiterjoin
\pgfsetdash{}{0pt}
\definecolor{dialinecolor}{rgb}{1.000000, 1.000000, 1.000000}
\pgfsetfillcolor{dialinecolor}
\pgfpathmoveto{\pgfpoint{60.195292\du}{13.003077\du}}
\pgfpathlineto{\pgfpoint{60.195292\du}{12.999042\du}}
\pgfpathlineto{\pgfpoint{60.194955\du}{12.994671\du}}
\pgfpathlineto{\pgfpoint{60.194619\du}{12.990973\du}}
\pgfpathlineto{\pgfpoint{60.193610\du}{12.986602\du}}
\pgfpathlineto{\pgfpoint{60.192938\du}{12.982904\du}}
\pgfpathlineto{\pgfpoint{60.191593\du}{12.978533\du}}
\pgfpathlineto{\pgfpoint{60.190584\du}{12.974834\du}}
\pgfpathlineto{\pgfpoint{60.188903\du}{12.971136\du}}
\pgfpathlineto{\pgfpoint{60.186886\du}{12.967101\du}}
\pgfpathlineto{\pgfpoint{60.185205\du}{12.963403\du}}
\pgfpathlineto{\pgfpoint{60.183187\du}{12.960040\du}}
\pgfpathlineto{\pgfpoint{60.181170\du}{12.956678\du}}
\pgfpathlineto{\pgfpoint{60.178480\du}{12.953316\du}}
\pgfpathlineto{\pgfpoint{60.176127\du}{12.949618\du}}
\pgfpathlineto{\pgfpoint{60.173101\du}{12.946928\du}}
\pgfpathlineto{\pgfpoint{60.170411\du}{12.943902\du}}
\pgfpathlineto{\pgfpoint{60.167385\du}{12.941212\du}}
\pgfpathlineto{\pgfpoint{60.164023\du}{12.938522\du}}
\pgfpathlineto{\pgfpoint{60.160997\du}{12.936505\du}}
\pgfpathlineto{\pgfpoint{60.157298\du}{12.933815\du}}
\pgfpathlineto{\pgfpoint{60.153936\du}{12.931798\du}}
\pgfpathlineto{\pgfpoint{60.150574\du}{12.930117\du}}
\pgfpathlineto{\pgfpoint{60.146539\du}{12.928435\du}}
\pgfpathlineto{\pgfpoint{60.142841\du}{12.926418\du}}
\pgfpathlineto{\pgfpoint{60.138806\du}{12.925409\du}}
\pgfpathlineto{\pgfpoint{60.135107\du}{12.924065\du}}
\pgfpathlineto{\pgfpoint{60.131073\du}{12.923392\du}}
\pgfpathlineto{\pgfpoint{60.127374\du}{12.922383\du}}
\pgfpathlineto{\pgfpoint{60.123340\du}{12.922047\du}}
\pgfpathlineto{\pgfpoint{60.119305\du}{12.921711\du}}
\pgfpathlineto{\pgfpoint{60.114934\du}{12.921711\du}}
\pgfpathlineto{\pgfpoint{60.114934\du}{12.921711\du}}
\pgfpathlineto{\pgfpoint{60.110899\du}{12.921711\du}}
\pgfpathlineto{\pgfpoint{60.106865\du}{12.922047\du}}
\pgfpathlineto{\pgfpoint{60.102830\du}{12.922383\du}}
\pgfpathlineto{\pgfpoint{60.099132\du}{12.923392\du}}
\pgfpathlineto{\pgfpoint{60.094761\du}{12.924065\du}}
\pgfpathlineto{\pgfpoint{60.091062\du}{12.925409\du}}
\pgfpathlineto{\pgfpoint{60.087364\du}{12.926418\du}}
\pgfpathlineto{\pgfpoint{60.083329\du}{12.928435\du}}
\pgfpathlineto{\pgfpoint{60.079631\du}{12.930117\du}}
\pgfpathlineto{\pgfpoint{60.075932\du}{12.931798\du}}
\pgfpathlineto{\pgfpoint{60.072906\du}{12.933815\du}}
\pgfpathlineto{\pgfpoint{60.069208\du}{12.936505\du}}
\pgfpathlineto{\pgfpoint{60.065845\du}{12.938522\du}}
\pgfpathlineto{\pgfpoint{60.062819\du}{12.941212\du}}
\pgfpathlineto{\pgfpoint{60.059457\du}{12.943902\du}}
\pgfpathlineto{\pgfpoint{60.056767\du}{12.946928\du}}
\pgfpathlineto{\pgfpoint{60.054078\du}{12.949618\du}}
\pgfpathlineto{\pgfpoint{60.051724\du}{12.953316\du}}
\pgfpathlineto{\pgfpoint{60.049034\du}{12.956678\du}}
\pgfpathlineto{\pgfpoint{60.046681\du}{12.960040\du}}
\pgfpathlineto{\pgfpoint{60.044663\du}{12.963403\du}}
\pgfpathlineto{\pgfpoint{60.043318\du}{12.967101\du}}
\pgfpathlineto{\pgfpoint{60.041301\du}{12.971136\du}}
\pgfpathlineto{\pgfpoint{60.039620\du}{12.974834\du}}
\pgfpathlineto{\pgfpoint{60.038275\du}{12.978533\du}}
\pgfpathlineto{\pgfpoint{60.037266\du}{12.982904\du}}
\pgfpathlineto{\pgfpoint{60.036258\du}{12.986602\du}}
\pgfpathlineto{\pgfpoint{60.035585\du}{12.990973\du}}
\pgfpathlineto{\pgfpoint{60.034913\du}{12.994671\du}}
\pgfpathlineto{\pgfpoint{60.034913\du}{12.999042\du}}
\pgfpathlineto{\pgfpoint{60.034913\du}{13.003077\du}}
\pgfpathlineto{\pgfpoint{60.034913\du}{13.003077\du}}
\pgfpathlineto{\pgfpoint{60.034913\du}{13.007448\du}}
\pgfpathlineto{\pgfpoint{60.034913\du}{13.011483\du}}
\pgfpathlineto{\pgfpoint{60.035585\du}{13.015517\du}}
\pgfpathlineto{\pgfpoint{60.036258\du}{13.019552\du}}
\pgfpathlineto{\pgfpoint{60.037266\du}{13.023250\du}}
\pgfpathlineto{\pgfpoint{60.038275\du}{13.027285\du}}
\pgfpathlineto{\pgfpoint{60.039620\du}{13.031320\du}}
\pgfpathlineto{\pgfpoint{60.041301\du}{13.035355\du}}
\pgfpathlineto{\pgfpoint{60.043318\du}{13.039053\du}}
\pgfpathlineto{\pgfpoint{60.044663\du}{13.042751\du}}
\pgfpathlineto{\pgfpoint{60.046681\du}{13.046114\du}}
\pgfpathlineto{\pgfpoint{60.049034\du}{13.049476\du}}
\pgfpathlineto{\pgfpoint{60.051724\du}{13.052502\du}}
\pgfpathlineto{\pgfpoint{60.054078\du}{13.056200\du}}
\pgfpathlineto{\pgfpoint{60.056767\du}{13.058890\du}}
\pgfpathlineto{\pgfpoint{60.059457\du}{13.062252\du}}
\pgfpathlineto{\pgfpoint{60.062819\du}{13.064606\du}}
\pgfpathlineto{\pgfpoint{60.065845\du}{13.067296\du}}
\pgfpathlineto{\pgfpoint{60.069208\du}{13.069986\du}}
\pgfpathlineto{\pgfpoint{60.072906\du}{13.072339\du}}
\pgfpathlineto{\pgfpoint{60.075932\du}{13.074356\du}}
\pgfpathlineto{\pgfpoint{60.079631\du}{13.075701\du}}
\pgfpathlineto{\pgfpoint{60.083329\du}{13.077719\du}}
\pgfpathlineto{\pgfpoint{60.087364\du}{13.079400\du}}
\pgfpathlineto{\pgfpoint{60.091062\du}{13.080745\du}}
\pgfpathlineto{\pgfpoint{60.094761\du}{13.082090\du}}
\pgfpathlineto{\pgfpoint{60.099132\du}{13.083098\du}}
\pgfpathlineto{\pgfpoint{60.102830\du}{13.083771\du}}
\pgfpathlineto{\pgfpoint{60.106865\du}{13.084107\du}}
\pgfpathlineto{\pgfpoint{60.110899\du}{13.084779\du}}
\pgfpathlineto{\pgfpoint{60.114934\du}{13.084779\du}}
\pgfpathlineto{\pgfpoint{60.114934\du}{13.084779\du}}
\pgfpathlineto{\pgfpoint{60.119305\du}{13.084779\du}}
\pgfpathlineto{\pgfpoint{60.123340\du}{13.084107\du}}
\pgfpathlineto{\pgfpoint{60.127374\du}{13.083771\du}}
\pgfpathlineto{\pgfpoint{60.131073\du}{13.083098\du}}
\pgfpathlineto{\pgfpoint{60.135107\du}{13.082090\du}}
\pgfpathlineto{\pgfpoint{60.138806\du}{13.080745\du}}
\pgfpathlineto{\pgfpoint{60.142841\du}{13.079400\du}}
\pgfpathlineto{\pgfpoint{60.146539\du}{13.077719\du}}
\pgfpathlineto{\pgfpoint{60.150574\du}{13.075701\du}}
\pgfpathlineto{\pgfpoint{60.153936\du}{13.074356\du}}
\pgfpathlineto{\pgfpoint{60.157298\du}{13.072339\du}}
\pgfpathlineto{\pgfpoint{60.160997\du}{13.069986\du}}
\pgfpathlineto{\pgfpoint{60.164023\du}{13.067296\du}}
\pgfpathlineto{\pgfpoint{60.167385\du}{13.064606\du}}
\pgfpathlineto{\pgfpoint{60.170411\du}{13.062252\du}}
\pgfpathlineto{\pgfpoint{60.173101\du}{13.058890\du}}
\pgfpathlineto{\pgfpoint{60.176127\du}{13.056200\du}}
\pgfpathlineto{\pgfpoint{60.178480\du}{13.052502\du}}
\pgfpathlineto{\pgfpoint{60.181170\du}{13.049476\du}}
\pgfpathlineto{\pgfpoint{60.183187\du}{13.046114\du}}
\pgfpathlineto{\pgfpoint{60.185205\du}{13.042751\du}}
\pgfpathlineto{\pgfpoint{60.186886\du}{13.039053\du}}
\pgfpathlineto{\pgfpoint{60.188903\du}{13.035355\du}}
\pgfpathlineto{\pgfpoint{60.190584\du}{13.031320\du}}
\pgfpathlineto{\pgfpoint{60.191593\du}{13.027285\du}}
\pgfpathlineto{\pgfpoint{60.192938\du}{13.023250\du}}
\pgfpathlineto{\pgfpoint{60.193610\du}{13.019552\du}}
\pgfpathlineto{\pgfpoint{60.194619\du}{13.015517\du}}
\pgfpathlineto{\pgfpoint{60.194955\du}{13.011483\du}}
\pgfpathlineto{\pgfpoint{60.195292\du}{13.007448\du}}
\pgfpathlineto{\pgfpoint{60.195292\du}{13.003077\du}}
\pgfusepath{fill}
\pgfsetbuttcap
\pgfsetmiterjoin
\pgfsetdash{}{0pt}
\definecolor{dialinecolor}{rgb}{1.000000, 1.000000, 1.000000}
\pgfsetfillcolor{dialinecolor}
\pgfpathmoveto{\pgfpoint{60.929604\du}{13.003077\du}}
\pgfpathlineto{\pgfpoint{60.929604\du}{12.999042\du}}
\pgfpathlineto{\pgfpoint{60.928931\du}{12.994671\du}}
\pgfpathlineto{\pgfpoint{60.928595\du}{12.990973\du}}
\pgfpathlineto{\pgfpoint{60.928259\du}{12.986602\du}}
\pgfpathlineto{\pgfpoint{60.926914\du}{12.982904\du}}
\pgfpathlineto{\pgfpoint{60.925905\du}{12.978533\du}}
\pgfpathlineto{\pgfpoint{60.924560\du}{12.974834\du}}
\pgfpathlineto{\pgfpoint{60.922879\du}{12.971136\du}}
\pgfpathlineto{\pgfpoint{60.921198\du}{12.967101\du}}
\pgfpathlineto{\pgfpoint{60.919853\du}{12.963403\du}}
\pgfpathlineto{\pgfpoint{60.917500\du}{12.960040\du}}
\pgfpathlineto{\pgfpoint{60.915482\du}{12.956678\du}}
\pgfpathlineto{\pgfpoint{60.912793\du}{12.953316\du}}
\pgfpathlineto{\pgfpoint{60.910103\du}{12.949618\du}}
\pgfpathlineto{\pgfpoint{60.907077\du}{12.946928\du}}
\pgfpathlineto{\pgfpoint{60.904387\du}{12.943902\du}}
\pgfpathlineto{\pgfpoint{60.901361\du}{12.941212\du}}
\pgfpathlineto{\pgfpoint{60.898671\du}{12.938522\du}}
\pgfpathlineto{\pgfpoint{60.894973\du}{12.936505\du}}
\pgfpathlineto{\pgfpoint{60.891610\du}{12.933815\du}}
\pgfpathlineto{\pgfpoint{60.888248\du}{12.931798\du}}
\pgfpathlineto{\pgfpoint{60.884550\du}{12.930117\du}}
\pgfpathlineto{\pgfpoint{60.880851\du}{12.928435\du}}
\pgfpathlineto{\pgfpoint{60.876817\du}{12.926418\du}}
\pgfpathlineto{\pgfpoint{60.872782\du}{12.925409\du}}
\pgfpathlineto{\pgfpoint{60.869083\du}{12.924065\du}}
\pgfpathlineto{\pgfpoint{60.865049\du}{12.923392\du}}
\pgfpathlineto{\pgfpoint{60.861014\du}{12.922383\du}}
\pgfpathlineto{\pgfpoint{60.856643\du}{12.922047\du}}
\pgfpathlineto{\pgfpoint{60.852945\du}{12.921711\du}}
\pgfpathlineto{\pgfpoint{60.848574\du}{12.921711\du}}
\pgfpathlineto{\pgfpoint{60.848574\du}{12.921711\du}}
\pgfpathlineto{\pgfpoint{60.844539\du}{12.921711\du}}
\pgfpathlineto{\pgfpoint{60.840841\du}{12.922047\du}}
\pgfpathlineto{\pgfpoint{60.836470\du}{12.922383\du}}
\pgfpathlineto{\pgfpoint{60.832435\du}{12.923392\du}}
\pgfpathlineto{\pgfpoint{60.828400\du}{12.924065\du}}
\pgfpathlineto{\pgfpoint{60.825038\du}{12.925409\du}}
\pgfpathlineto{\pgfpoint{60.821004\du}{12.926418\du}}
\pgfpathlineto{\pgfpoint{60.816633\du}{12.928435\du}}
\pgfpathlineto{\pgfpoint{60.812934\du}{12.930117\du}}
\pgfpathlineto{\pgfpoint{60.809572\du}{12.931798\du}}
\pgfpathlineto{\pgfpoint{60.806210\du}{12.933815\du}}
\pgfpathlineto{\pgfpoint{60.802511\du}{12.936505\du}}
\pgfpathlineto{\pgfpoint{60.799149\du}{12.938522\du}}
\pgfpathlineto{\pgfpoint{60.796123\du}{12.941212\du}}
\pgfpathlineto{\pgfpoint{60.793433\du}{12.943902\du}}
\pgfpathlineto{\pgfpoint{60.790071\du}{12.946928\du}}
\pgfpathlineto{\pgfpoint{60.787381\du}{12.949618\du}}
\pgfpathlineto{\pgfpoint{60.785028\du}{12.953316\du}}
\pgfpathlineto{\pgfpoint{60.782338\du}{12.956678\du}}
\pgfpathlineto{\pgfpoint{60.779984\du}{12.960040\du}}
\pgfpathlineto{\pgfpoint{60.777967\du}{12.963403\du}}
\pgfpathlineto{\pgfpoint{60.776622\du}{12.967101\du}}
\pgfpathlineto{\pgfpoint{60.774605\du}{12.971136\du}}
\pgfpathlineto{\pgfpoint{60.772924\du}{12.974834\du}}
\pgfpathlineto{\pgfpoint{60.771579\du}{12.978533\du}}
\pgfpathlineto{\pgfpoint{60.770570\du}{12.982904\du}}
\pgfpathlineto{\pgfpoint{60.769561\du}{12.986602\du}}
\pgfpathlineto{\pgfpoint{60.768889\du}{12.990973\du}}
\pgfpathlineto{\pgfpoint{60.768553\du}{12.994671\du}}
\pgfpathlineto{\pgfpoint{60.768216\du}{12.999042\du}}
\pgfpathlineto{\pgfpoint{60.768216\du}{13.003077\du}}
\pgfpathlineto{\pgfpoint{60.768216\du}{13.003077\du}}
\pgfpathlineto{\pgfpoint{60.768216\du}{13.007448\du}}
\pgfpathlineto{\pgfpoint{60.768553\du}{13.011483\du}}
\pgfpathlineto{\pgfpoint{60.768889\du}{13.015517\du}}
\pgfpathlineto{\pgfpoint{60.769561\du}{13.019552\du}}
\pgfpathlineto{\pgfpoint{60.770570\du}{13.023250\du}}
\pgfpathlineto{\pgfpoint{60.771579\du}{13.027285\du}}
\pgfpathlineto{\pgfpoint{60.772924\du}{13.031320\du}}
\pgfpathlineto{\pgfpoint{60.774605\du}{13.035355\du}}
\pgfpathlineto{\pgfpoint{60.776622\du}{13.039053\du}}
\pgfpathlineto{\pgfpoint{60.777967\du}{13.042751\du}}
\pgfpathlineto{\pgfpoint{60.779984\du}{13.046114\du}}
\pgfpathlineto{\pgfpoint{60.782338\du}{13.049476\du}}
\pgfpathlineto{\pgfpoint{60.785028\du}{13.052502\du}}
\pgfpathlineto{\pgfpoint{60.787381\du}{13.056200\du}}
\pgfpathlineto{\pgfpoint{60.790071\du}{13.058890\du}}
\pgfpathlineto{\pgfpoint{60.793433\du}{13.062252\du}}
\pgfpathlineto{\pgfpoint{60.796123\du}{13.064606\du}}
\pgfpathlineto{\pgfpoint{60.799149\du}{13.067296\du}}
\pgfpathlineto{\pgfpoint{60.802511\du}{13.069986\du}}
\pgfpathlineto{\pgfpoint{60.806210\du}{13.072339\du}}
\pgfpathlineto{\pgfpoint{60.809572\du}{13.074356\du}}
\pgfpathlineto{\pgfpoint{60.812934\du}{13.075701\du}}
\pgfpathlineto{\pgfpoint{60.816633\du}{13.077719\du}}
\pgfpathlineto{\pgfpoint{60.821004\du}{13.079400\du}}
\pgfpathlineto{\pgfpoint{60.825038\du}{13.080745\du}}
\pgfpathlineto{\pgfpoint{60.828400\du}{13.082090\du}}
\pgfpathlineto{\pgfpoint{60.832435\du}{13.083098\du}}
\pgfpathlineto{\pgfpoint{60.836470\du}{13.083771\du}}
\pgfpathlineto{\pgfpoint{60.840841\du}{13.084107\du}}
\pgfpathlineto{\pgfpoint{60.844539\du}{13.084779\du}}
\pgfpathlineto{\pgfpoint{60.848574\du}{13.084779\du}}
\pgfpathlineto{\pgfpoint{60.848574\du}{13.084779\du}}
\pgfpathlineto{\pgfpoint{60.852945\du}{13.084779\du}}
\pgfpathlineto{\pgfpoint{60.856643\du}{13.084107\du}}
\pgfpathlineto{\pgfpoint{60.861014\du}{13.083771\du}}
\pgfpathlineto{\pgfpoint{60.865049\du}{13.083098\du}}
\pgfpathlineto{\pgfpoint{60.869083\du}{13.082090\du}}
\pgfpathlineto{\pgfpoint{60.872782\du}{13.080745\du}}
\pgfpathlineto{\pgfpoint{60.876817\du}{13.079400\du}}
\pgfpathlineto{\pgfpoint{60.880851\du}{13.077719\du}}
\pgfpathlineto{\pgfpoint{60.884550\du}{13.075701\du}}
\pgfpathlineto{\pgfpoint{60.888248\du}{13.074356\du}}
\pgfpathlineto{\pgfpoint{60.891610\du}{13.072339\du}}
\pgfpathlineto{\pgfpoint{60.894973\du}{13.069986\du}}
\pgfpathlineto{\pgfpoint{60.898671\du}{13.067296\du}}
\pgfpathlineto{\pgfpoint{60.901361\du}{13.064606\du}}
\pgfpathlineto{\pgfpoint{60.904387\du}{13.062252\du}}
\pgfpathlineto{\pgfpoint{60.907077\du}{13.058890\du}}
\pgfpathlineto{\pgfpoint{60.910103\du}{13.056200\du}}
\pgfpathlineto{\pgfpoint{60.912793\du}{13.052502\du}}
\pgfpathlineto{\pgfpoint{60.915482\du}{13.049476\du}}
\pgfpathlineto{\pgfpoint{60.917500\du}{13.046114\du}}
\pgfpathlineto{\pgfpoint{60.919853\du}{13.042751\du}}
\pgfpathlineto{\pgfpoint{60.921198\du}{13.039053\du}}
\pgfpathlineto{\pgfpoint{60.922879\du}{13.035355\du}}
\pgfpathlineto{\pgfpoint{60.924560\du}{13.031320\du}}
\pgfpathlineto{\pgfpoint{60.925905\du}{13.027285\du}}
\pgfpathlineto{\pgfpoint{60.926914\du}{13.023250\du}}
\pgfpathlineto{\pgfpoint{60.928259\du}{13.019552\du}}
\pgfpathlineto{\pgfpoint{60.928595\du}{13.015517\du}}
\pgfpathlineto{\pgfpoint{60.928931\du}{13.011483\du}}
\pgfpathlineto{\pgfpoint{60.929604\du}{13.007448\du}}
\pgfpathlineto{\pgfpoint{60.929604\du}{13.003077\du}}
\pgfusepath{fill}
\pgfsetlinewidth{0.000000\du}
\pgfsetbuttcap
\pgfsetmiterjoin
\pgfsetdash{}{0pt}
\definecolor{dialinecolor}{rgb}{1.000000, 1.000000, 1.000000}
\pgfsetstrokecolor{dialinecolor}
\pgfpathmoveto{\pgfpoint{60.122667\du}{12.996016\du}}
\pgfpathlineto{\pgfpoint{60.827056\du}{12.996016\du}}
\pgfusepath{stroke}
\pgfsetlinewidth{0.000000\du}
\pgfsetbuttcap
\pgfsetmiterjoin
\pgfsetdash{}{0pt}
\definecolor{dialinecolor}{rgb}{1.000000, 1.000000, 1.000000}
\pgfsetfillcolor{dialinecolor}
\pgfpathmoveto{\pgfpoint{60.195292\du}{13.300635\du}}
\pgfpathlineto{\pgfpoint{60.195292\du}{13.296264\du}}
\pgfpathlineto{\pgfpoint{60.194955\du}{13.292229\du}}
\pgfpathlineto{\pgfpoint{60.194619\du}{13.288531\du}}
\pgfpathlineto{\pgfpoint{60.193610\du}{13.284496\du}}
\pgfpathlineto{\pgfpoint{60.192938\du}{13.280461\du}}
\pgfpathlineto{\pgfpoint{60.191593\du}{13.276427\du}}
\pgfpathlineto{\pgfpoint{60.190584\du}{13.272392\du}}
\pgfpathlineto{\pgfpoint{60.188903\du}{13.268694\du}}
\pgfpathlineto{\pgfpoint{60.186886\du}{13.264995\du}}
\pgfpathlineto{\pgfpoint{60.185205\du}{13.261297\du}}
\pgfpathlineto{\pgfpoint{60.183187\du}{13.257598\du}}
\pgfpathlineto{\pgfpoint{60.181170\du}{13.254236\du}}
\pgfpathlineto{\pgfpoint{60.178480\du}{13.250874\du}}
\pgfpathlineto{\pgfpoint{60.176127\du}{13.247512\du}}
\pgfpathlineto{\pgfpoint{60.173101\du}{13.244822\du}}
\pgfpathlineto{\pgfpoint{60.170411\du}{13.241796\du}}
\pgfpathlineto{\pgfpoint{60.167385\du}{13.239106\du}}
\pgfpathlineto{\pgfpoint{60.164023\du}{13.236080\du}}
\pgfpathlineto{\pgfpoint{60.160997\du}{13.234063\du}}
\pgfpathlineto{\pgfpoint{60.157298\du}{13.231709\du}}
\pgfpathlineto{\pgfpoint{60.153936\du}{13.229019\du}}
\pgfpathlineto{\pgfpoint{60.150574\du}{13.227674\du}}
\pgfpathlineto{\pgfpoint{60.146539\du}{13.225993\du}}
\pgfpathlineto{\pgfpoint{60.142841\du}{13.224312\du}}
\pgfpathlineto{\pgfpoint{60.138806\du}{13.222967\du}}
\pgfpathlineto{\pgfpoint{60.135107\du}{13.221959\du}}
\pgfpathlineto{\pgfpoint{60.131073\du}{13.220614\du}}
\pgfpathlineto{\pgfpoint{60.127374\du}{13.220277\du}}
\pgfpathlineto{\pgfpoint{60.123340\du}{13.219605\du}}
\pgfpathlineto{\pgfpoint{60.119305\du}{13.219269\du}}
\pgfpathlineto{\pgfpoint{60.114934\du}{13.219269\du}}
\pgfpathlineto{\pgfpoint{60.114934\du}{13.219269\du}}
\pgfpathlineto{\pgfpoint{60.110899\du}{13.219269\du}}
\pgfpathlineto{\pgfpoint{60.106865\du}{13.219605\du}}
\pgfpathlineto{\pgfpoint{60.102830\du}{13.220277\du}}
\pgfpathlineto{\pgfpoint{60.099132\du}{13.220614\du}}
\pgfpathlineto{\pgfpoint{60.094761\du}{13.221959\du}}
\pgfpathlineto{\pgfpoint{60.091062\du}{13.222967\du}}
\pgfpathlineto{\pgfpoint{60.087364\du}{13.224312\du}}
\pgfpathlineto{\pgfpoint{60.083329\du}{13.225993\du}}
\pgfpathlineto{\pgfpoint{60.079631\du}{13.227674\du}}
\pgfpathlineto{\pgfpoint{60.075932\du}{13.229019\du}}
\pgfpathlineto{\pgfpoint{60.072906\du}{13.231709\du}}
\pgfpathlineto{\pgfpoint{60.069208\du}{13.234063\du}}
\pgfpathlineto{\pgfpoint{60.065845\du}{13.236080\du}}
\pgfpathlineto{\pgfpoint{60.062819\du}{13.239106\du}}
\pgfpathlineto{\pgfpoint{60.059457\du}{13.241796\du}}
\pgfpathlineto{\pgfpoint{60.056767\du}{13.244822\du}}
\pgfpathlineto{\pgfpoint{60.054078\du}{13.247512\du}}
\pgfpathlineto{\pgfpoint{60.051724\du}{13.250874\du}}
\pgfpathlineto{\pgfpoint{60.049034\du}{13.254236\du}}
\pgfpathlineto{\pgfpoint{60.046681\du}{13.257598\du}}
\pgfpathlineto{\pgfpoint{60.044663\du}{13.261297\du}}
\pgfpathlineto{\pgfpoint{60.043318\du}{13.264995\du}}
\pgfpathlineto{\pgfpoint{60.041301\du}{13.268694\du}}
\pgfpathlineto{\pgfpoint{60.039620\du}{13.272392\du}}
\pgfpathlineto{\pgfpoint{60.038275\du}{13.276427\du}}
\pgfpathlineto{\pgfpoint{60.037266\du}{13.280461\du}}
\pgfpathlineto{\pgfpoint{60.036258\du}{13.284496\du}}
\pgfpathlineto{\pgfpoint{60.035585\du}{13.288531\du}}
\pgfpathlineto{\pgfpoint{60.034913\du}{13.292229\du}}
\pgfpathlineto{\pgfpoint{60.034913\du}{13.296264\du}}
\pgfpathlineto{\pgfpoint{60.034913\du}{13.300635\du}}
\pgfpathlineto{\pgfpoint{60.034913\du}{13.300635\du}}
\pgfpathlineto{\pgfpoint{60.034913\du}{13.304670\du}}
\pgfpathlineto{\pgfpoint{60.034913\du}{13.309040\du}}
\pgfpathlineto{\pgfpoint{60.035585\du}{13.313075\du}}
\pgfpathlineto{\pgfpoint{60.036258\du}{13.317446\du}}
\pgfpathlineto{\pgfpoint{60.037266\du}{13.321145\du}}
\pgfpathlineto{\pgfpoint{60.038275\du}{13.325179\du}}
\pgfpathlineto{\pgfpoint{60.039620\du}{13.329214\du}}
\pgfpathlineto{\pgfpoint{60.041301\du}{13.332912\du}}
\pgfpathlineto{\pgfpoint{60.043318\du}{13.336275\du}}
\pgfpathlineto{\pgfpoint{60.044663\du}{13.340645\du}}
\pgfpathlineto{\pgfpoint{60.046681\du}{13.344008\du}}
\pgfpathlineto{\pgfpoint{60.049034\du}{13.347370\du}}
\pgfpathlineto{\pgfpoint{60.051724\du}{13.350732\du}}
\pgfpathlineto{\pgfpoint{60.054078\du}{13.354094\du}}
\pgfpathlineto{\pgfpoint{60.056767\du}{13.357120\du}}
\pgfpathlineto{\pgfpoint{60.059457\du}{13.359474\du}}
\pgfpathlineto{\pgfpoint{60.062819\du}{13.362500\du}}
\pgfpathlineto{\pgfpoint{60.065845\du}{13.365526\du}}
\pgfpathlineto{\pgfpoint{60.069208\du}{13.367543\du}}
\pgfpathlineto{\pgfpoint{60.072906\du}{13.369897\du}}
\pgfpathlineto{\pgfpoint{60.075932\du}{13.372250\du}}
\pgfpathlineto{\pgfpoint{60.079631\du}{13.373932\du}}
\pgfpathlineto{\pgfpoint{60.083329\du}{13.375613\du}}
\pgfpathlineto{\pgfpoint{60.087364\du}{13.377294\du}}
\pgfpathlineto{\pgfpoint{60.091062\du}{13.378303\du}}
\pgfpathlineto{\pgfpoint{60.094761\du}{13.379647\du}}
\pgfpathlineto{\pgfpoint{60.099132\du}{13.380656\du}}
\pgfpathlineto{\pgfpoint{60.102830\du}{13.381329\du}}
\pgfpathlineto{\pgfpoint{60.106865\du}{13.382001\du}}
\pgfpathlineto{\pgfpoint{60.110899\du}{13.382337\du}}
\pgfpathlineto{\pgfpoint{60.114934\du}{13.382337\du}}
\pgfpathlineto{\pgfpoint{60.114934\du}{13.382337\du}}
\pgfpathlineto{\pgfpoint{60.119305\du}{13.382337\du}}
\pgfpathlineto{\pgfpoint{60.123340\du}{13.382001\du}}
\pgfpathlineto{\pgfpoint{60.127374\du}{13.381329\du}}
\pgfpathlineto{\pgfpoint{60.131073\du}{13.380656\du}}
\pgfpathlineto{\pgfpoint{60.135107\du}{13.379647\du}}
\pgfpathlineto{\pgfpoint{60.138806\du}{13.378303\du}}
\pgfpathlineto{\pgfpoint{60.142841\du}{13.377294\du}}
\pgfpathlineto{\pgfpoint{60.146539\du}{13.375613\du}}
\pgfpathlineto{\pgfpoint{60.150574\du}{13.373932\du}}
\pgfpathlineto{\pgfpoint{60.153936\du}{13.372250\du}}
\pgfpathlineto{\pgfpoint{60.157298\du}{13.369897\du}}
\pgfpathlineto{\pgfpoint{60.160997\du}{13.367543\du}}
\pgfpathlineto{\pgfpoint{60.164023\du}{13.365526\du}}
\pgfpathlineto{\pgfpoint{60.167385\du}{13.362500\du}}
\pgfpathlineto{\pgfpoint{60.170411\du}{13.359474\du}}
\pgfpathlineto{\pgfpoint{60.173101\du}{13.357120\du}}
\pgfpathlineto{\pgfpoint{60.176127\du}{13.354094\du}}
\pgfpathlineto{\pgfpoint{60.178480\du}{13.350732\du}}
\pgfpathlineto{\pgfpoint{60.181170\du}{13.347370\du}}
\pgfpathlineto{\pgfpoint{60.183187\du}{13.344008\du}}
\pgfpathlineto{\pgfpoint{60.185205\du}{13.340645\du}}
\pgfpathlineto{\pgfpoint{60.186886\du}{13.336275\du}}
\pgfpathlineto{\pgfpoint{60.188903\du}{13.332912\du}}
\pgfpathlineto{\pgfpoint{60.190584\du}{13.329214\du}}
\pgfpathlineto{\pgfpoint{60.191593\du}{13.325179\du}}
\pgfpathlineto{\pgfpoint{60.192938\du}{13.321145\du}}
\pgfpathlineto{\pgfpoint{60.193610\du}{13.317446\du}}
\pgfpathlineto{\pgfpoint{60.194619\du}{13.313075\du}}
\pgfpathlineto{\pgfpoint{60.194955\du}{13.309040\du}}
\pgfpathlineto{\pgfpoint{60.195292\du}{13.304670\du}}
\pgfpathlineto{\pgfpoint{60.195292\du}{13.300635\du}}
\pgfusepath{fill}
\pgfsetbuttcap
\pgfsetmiterjoin
\pgfsetdash{}{0pt}
\definecolor{dialinecolor}{rgb}{1.000000, 1.000000, 1.000000}
\pgfsetfillcolor{dialinecolor}
\pgfpathmoveto{\pgfpoint{60.929604\du}{13.300635\du}}
\pgfpathlineto{\pgfpoint{60.929604\du}{13.296264\du}}
\pgfpathlineto{\pgfpoint{60.928931\du}{13.292229\du}}
\pgfpathlineto{\pgfpoint{60.928595\du}{13.288531\du}}
\pgfpathlineto{\pgfpoint{60.928259\du}{13.284496\du}}
\pgfpathlineto{\pgfpoint{60.926914\du}{13.280461\du}}
\pgfpathlineto{\pgfpoint{60.925905\du}{13.276427\du}}
\pgfpathlineto{\pgfpoint{60.924560\du}{13.272392\du}}
\pgfpathlineto{\pgfpoint{60.922879\du}{13.268694\du}}
\pgfpathlineto{\pgfpoint{60.921198\du}{13.264995\du}}
\pgfpathlineto{\pgfpoint{60.919853\du}{13.261297\du}}
\pgfpathlineto{\pgfpoint{60.917500\du}{13.257598\du}}
\pgfpathlineto{\pgfpoint{60.915482\du}{13.254236\du}}
\pgfpathlineto{\pgfpoint{60.912793\du}{13.250874\du}}
\pgfpathlineto{\pgfpoint{60.910103\du}{13.247512\du}}
\pgfpathlineto{\pgfpoint{60.907077\du}{13.244822\du}}
\pgfpathlineto{\pgfpoint{60.904387\du}{13.241796\du}}
\pgfpathlineto{\pgfpoint{60.901361\du}{13.239106\du}}
\pgfpathlineto{\pgfpoint{60.898671\du}{13.236080\du}}
\pgfpathlineto{\pgfpoint{60.894973\du}{13.234063\du}}
\pgfpathlineto{\pgfpoint{60.891610\du}{13.231709\du}}
\pgfpathlineto{\pgfpoint{60.888248\du}{13.229019\du}}
\pgfpathlineto{\pgfpoint{60.884550\du}{13.227674\du}}
\pgfpathlineto{\pgfpoint{60.880851\du}{13.225993\du}}
\pgfpathlineto{\pgfpoint{60.876817\du}{13.224312\du}}
\pgfpathlineto{\pgfpoint{60.872782\du}{13.222967\du}}
\pgfpathlineto{\pgfpoint{60.869083\du}{13.221959\du}}
\pgfpathlineto{\pgfpoint{60.865049\du}{13.220614\du}}
\pgfpathlineto{\pgfpoint{60.861014\du}{13.220277\du}}
\pgfpathlineto{\pgfpoint{60.856643\du}{13.219605\du}}
\pgfpathlineto{\pgfpoint{60.852945\du}{13.219269\du}}
\pgfpathlineto{\pgfpoint{60.848574\du}{13.219269\du}}
\pgfpathlineto{\pgfpoint{60.848574\du}{13.219269\du}}
\pgfpathlineto{\pgfpoint{60.844539\du}{13.219269\du}}
\pgfpathlineto{\pgfpoint{60.840841\du}{13.219605\du}}
\pgfpathlineto{\pgfpoint{60.836470\du}{13.220277\du}}
\pgfpathlineto{\pgfpoint{60.832435\du}{13.220614\du}}
\pgfpathlineto{\pgfpoint{60.828400\du}{13.221959\du}}
\pgfpathlineto{\pgfpoint{60.825038\du}{13.222967\du}}
\pgfpathlineto{\pgfpoint{60.821004\du}{13.224312\du}}
\pgfpathlineto{\pgfpoint{60.816633\du}{13.225993\du}}
\pgfpathlineto{\pgfpoint{60.812934\du}{13.227674\du}}
\pgfpathlineto{\pgfpoint{60.809572\du}{13.229019\du}}
\pgfpathlineto{\pgfpoint{60.806210\du}{13.231709\du}}
\pgfpathlineto{\pgfpoint{60.802511\du}{13.234063\du}}
\pgfpathlineto{\pgfpoint{60.799149\du}{13.236080\du}}
\pgfpathlineto{\pgfpoint{60.796123\du}{13.239106\du}}
\pgfpathlineto{\pgfpoint{60.793433\du}{13.241796\du}}
\pgfpathlineto{\pgfpoint{60.790071\du}{13.244822\du}}
\pgfpathlineto{\pgfpoint{60.787381\du}{13.247512\du}}
\pgfpathlineto{\pgfpoint{60.785028\du}{13.250874\du}}
\pgfpathlineto{\pgfpoint{60.782338\du}{13.254236\du}}
\pgfpathlineto{\pgfpoint{60.779984\du}{13.257598\du}}
\pgfpathlineto{\pgfpoint{60.777967\du}{13.261297\du}}
\pgfpathlineto{\pgfpoint{60.776622\du}{13.264995\du}}
\pgfpathlineto{\pgfpoint{60.774605\du}{13.268694\du}}
\pgfpathlineto{\pgfpoint{60.772924\du}{13.272392\du}}
\pgfpathlineto{\pgfpoint{60.771579\du}{13.276427\du}}
\pgfpathlineto{\pgfpoint{60.770570\du}{13.280461\du}}
\pgfpathlineto{\pgfpoint{60.769561\du}{13.284496\du}}
\pgfpathlineto{\pgfpoint{60.768889\du}{13.288531\du}}
\pgfpathlineto{\pgfpoint{60.768553\du}{13.292229\du}}
\pgfpathlineto{\pgfpoint{60.768216\du}{13.296264\du}}
\pgfpathlineto{\pgfpoint{60.768216\du}{13.300635\du}}
\pgfpathlineto{\pgfpoint{60.768216\du}{13.300635\du}}
\pgfpathlineto{\pgfpoint{60.768216\du}{13.304670\du}}
\pgfpathlineto{\pgfpoint{60.768553\du}{13.309040\du}}
\pgfpathlineto{\pgfpoint{60.768889\du}{13.313075\du}}
\pgfpathlineto{\pgfpoint{60.769561\du}{13.317446\du}}
\pgfpathlineto{\pgfpoint{60.770570\du}{13.321145\du}}
\pgfpathlineto{\pgfpoint{60.771579\du}{13.325179\du}}
\pgfpathlineto{\pgfpoint{60.772924\du}{13.329214\du}}
\pgfpathlineto{\pgfpoint{60.774605\du}{13.332912\du}}
\pgfpathlineto{\pgfpoint{60.776622\du}{13.336275\du}}
\pgfpathlineto{\pgfpoint{60.777967\du}{13.340645\du}}
\pgfpathlineto{\pgfpoint{60.779984\du}{13.344008\du}}
\pgfpathlineto{\pgfpoint{60.782338\du}{13.347370\du}}
\pgfpathlineto{\pgfpoint{60.785028\du}{13.350732\du}}
\pgfpathlineto{\pgfpoint{60.787381\du}{13.354094\du}}
\pgfpathlineto{\pgfpoint{60.790071\du}{13.357120\du}}
\pgfpathlineto{\pgfpoint{60.793433\du}{13.359474\du}}
\pgfpathlineto{\pgfpoint{60.796123\du}{13.362500\du}}
\pgfpathlineto{\pgfpoint{60.799149\du}{13.365526\du}}
\pgfpathlineto{\pgfpoint{60.802511\du}{13.367543\du}}
\pgfpathlineto{\pgfpoint{60.806210\du}{13.369897\du}}
\pgfpathlineto{\pgfpoint{60.809572\du}{13.372250\du}}
\pgfpathlineto{\pgfpoint{60.812934\du}{13.373932\du}}
\pgfpathlineto{\pgfpoint{60.816633\du}{13.375613\du}}
\pgfpathlineto{\pgfpoint{60.821004\du}{13.377294\du}}
\pgfpathlineto{\pgfpoint{60.825038\du}{13.378303\du}}
\pgfpathlineto{\pgfpoint{60.828400\du}{13.379647\du}}
\pgfpathlineto{\pgfpoint{60.832435\du}{13.380656\du}}
\pgfpathlineto{\pgfpoint{60.836470\du}{13.381329\du}}
\pgfpathlineto{\pgfpoint{60.840841\du}{13.382001\du}}
\pgfpathlineto{\pgfpoint{60.844539\du}{13.382337\du}}
\pgfpathlineto{\pgfpoint{60.848574\du}{13.382337\du}}
\pgfpathlineto{\pgfpoint{60.848574\du}{13.382337\du}}
\pgfpathlineto{\pgfpoint{60.852945\du}{13.382337\du}}
\pgfpathlineto{\pgfpoint{60.856643\du}{13.382001\du}}
\pgfpathlineto{\pgfpoint{60.861014\du}{13.381329\du}}
\pgfpathlineto{\pgfpoint{60.865049\du}{13.380656\du}}
\pgfpathlineto{\pgfpoint{60.869083\du}{13.379647\du}}
\pgfpathlineto{\pgfpoint{60.872782\du}{13.378303\du}}
\pgfpathlineto{\pgfpoint{60.876817\du}{13.377294\du}}
\pgfpathlineto{\pgfpoint{60.880851\du}{13.375613\du}}
\pgfpathlineto{\pgfpoint{60.884550\du}{13.373932\du}}
\pgfpathlineto{\pgfpoint{60.888248\du}{13.372250\du}}
\pgfpathlineto{\pgfpoint{60.891610\du}{13.369897\du}}
\pgfpathlineto{\pgfpoint{60.894973\du}{13.367543\du}}
\pgfpathlineto{\pgfpoint{60.898671\du}{13.365526\du}}
\pgfpathlineto{\pgfpoint{60.901361\du}{13.362500\du}}
\pgfpathlineto{\pgfpoint{60.904387\du}{13.359474\du}}
\pgfpathlineto{\pgfpoint{60.907077\du}{13.357120\du}}
\pgfpathlineto{\pgfpoint{60.910103\du}{13.354094\du}}
\pgfpathlineto{\pgfpoint{60.912793\du}{13.350732\du}}
\pgfpathlineto{\pgfpoint{60.915482\du}{13.347370\du}}
\pgfpathlineto{\pgfpoint{60.917500\du}{13.344008\du}}
\pgfpathlineto{\pgfpoint{60.919853\du}{13.340645\du}}
\pgfpathlineto{\pgfpoint{60.921198\du}{13.336275\du}}
\pgfpathlineto{\pgfpoint{60.922879\du}{13.332912\du}}
\pgfpathlineto{\pgfpoint{60.924560\du}{13.329214\du}}
\pgfpathlineto{\pgfpoint{60.925905\du}{13.325179\du}}
\pgfpathlineto{\pgfpoint{60.926914\du}{13.321145\du}}
\pgfpathlineto{\pgfpoint{60.928259\du}{13.317446\du}}
\pgfpathlineto{\pgfpoint{60.928595\du}{13.313075\du}}
\pgfpathlineto{\pgfpoint{60.928931\du}{13.309040\du}}
\pgfpathlineto{\pgfpoint{60.929604\du}{13.304670\du}}
\pgfpathlineto{\pgfpoint{60.929604\du}{13.300635\du}}
\pgfusepath{fill}
\pgfsetlinewidth{0.000000\du}
\pgfsetbuttcap
\pgfsetmiterjoin
\pgfsetdash{}{0pt}
\definecolor{dialinecolor}{rgb}{1.000000, 1.000000, 1.000000}
\pgfsetstrokecolor{dialinecolor}
\pgfpathmoveto{\pgfpoint{60.122667\du}{13.293574\du}}
\pgfpathlineto{\pgfpoint{60.827056\du}{13.293574\du}}
\pgfusepath{stroke}
\pgfsetlinewidth{0.000000\du}
\pgfsetbuttcap
\pgfsetmiterjoin
\pgfsetdash{}{0pt}
\definecolor{dialinecolor}{rgb}{1.000000, 1.000000, 1.000000}
\pgfsetfillcolor{dialinecolor}
\pgfpathmoveto{\pgfpoint{60.195292\du}{13.598193\du}}
\pgfpathlineto{\pgfpoint{60.195292\du}{13.593822\du}}
\pgfpathlineto{\pgfpoint{60.194955\du}{13.589787\du}}
\pgfpathlineto{\pgfpoint{60.194619\du}{13.585752\du}}
\pgfpathlineto{\pgfpoint{60.193610\du}{13.581718\du}}
\pgfpathlineto{\pgfpoint{60.192938\du}{13.577347\du}}
\pgfpathlineto{\pgfpoint{60.191593\du}{13.573648\du}}
\pgfpathlineto{\pgfpoint{60.190584\du}{13.569614\du}}
\pgfpathlineto{\pgfpoint{60.188903\du}{13.565579\du}}
\pgfpathlineto{\pgfpoint{60.186886\du}{13.562217\du}}
\pgfpathlineto{\pgfpoint{60.185205\du}{13.558518\du}}
\pgfpathlineto{\pgfpoint{60.183187\du}{13.554484\du}}
\pgfpathlineto{\pgfpoint{60.181170\du}{13.551121\du}}
\pgfpathlineto{\pgfpoint{60.178480\du}{13.547759\du}}
\pgfpathlineto{\pgfpoint{60.176127\du}{13.545069\du}}
\pgfpathlineto{\pgfpoint{60.173101\du}{13.541371\du}}
\pgfpathlineto{\pgfpoint{60.170411\du}{13.538681\du}}
\pgfpathlineto{\pgfpoint{60.167385\du}{13.535991\du}}
\pgfpathlineto{\pgfpoint{60.164023\du}{13.533638\du}}
\pgfpathlineto{\pgfpoint{60.160997\du}{13.530948\du}}
\pgfpathlineto{\pgfpoint{60.157298\du}{13.528594\du}}
\pgfpathlineto{\pgfpoint{60.153936\du}{13.526241\du}}
\pgfpathlineto{\pgfpoint{60.150574\du}{13.524560\du}}
\pgfpathlineto{\pgfpoint{60.146539\du}{13.522879\du}}
\pgfpathlineto{\pgfpoint{60.142841\du}{13.521198\du}}
\pgfpathlineto{\pgfpoint{60.138806\du}{13.519853\du}}
\pgfpathlineto{\pgfpoint{60.135107\du}{13.518844\du}}
\pgfpathlineto{\pgfpoint{60.131073\du}{13.517835\du}}
\pgfpathlineto{\pgfpoint{60.127374\du}{13.517163\du}}
\pgfpathlineto{\pgfpoint{60.123340\du}{13.516490\du}}
\pgfpathlineto{\pgfpoint{60.119305\du}{13.516154\du}}
\pgfpathlineto{\pgfpoint{60.114934\du}{13.516154\du}}
\pgfpathlineto{\pgfpoint{60.114934\du}{13.516154\du}}
\pgfpathlineto{\pgfpoint{60.110899\du}{13.516154\du}}
\pgfpathlineto{\pgfpoint{60.106865\du}{13.516490\du}}
\pgfpathlineto{\pgfpoint{60.102830\du}{13.517163\du}}
\pgfpathlineto{\pgfpoint{60.099132\du}{13.517835\du}}
\pgfpathlineto{\pgfpoint{60.094761\du}{13.518844\du}}
\pgfpathlineto{\pgfpoint{60.091062\du}{13.519853\du}}
\pgfpathlineto{\pgfpoint{60.087364\du}{13.521198\du}}
\pgfpathlineto{\pgfpoint{60.083329\du}{13.522879\du}}
\pgfpathlineto{\pgfpoint{60.079631\du}{13.524560\du}}
\pgfpathlineto{\pgfpoint{60.075932\du}{13.526241\du}}
\pgfpathlineto{\pgfpoint{60.072906\du}{13.528594\du}}
\pgfpathlineto{\pgfpoint{60.069208\du}{13.530948\du}}
\pgfpathlineto{\pgfpoint{60.065845\du}{13.533638\du}}
\pgfpathlineto{\pgfpoint{60.062819\du}{13.535991\du}}
\pgfpathlineto{\pgfpoint{60.059457\du}{13.538681\du}}
\pgfpathlineto{\pgfpoint{60.056767\du}{13.541371\du}}
\pgfpathlineto{\pgfpoint{60.054078\du}{13.545069\du}}
\pgfpathlineto{\pgfpoint{60.051724\du}{13.547759\du}}
\pgfpathlineto{\pgfpoint{60.049034\du}{13.551121\du}}
\pgfpathlineto{\pgfpoint{60.046681\du}{13.554484\du}}
\pgfpathlineto{\pgfpoint{60.044663\du}{13.558518\du}}
\pgfpathlineto{\pgfpoint{60.043318\du}{13.562217\du}}
\pgfpathlineto{\pgfpoint{60.041301\du}{13.565579\du}}
\pgfpathlineto{\pgfpoint{60.039620\du}{13.569614\du}}
\pgfpathlineto{\pgfpoint{60.038275\du}{13.573648\du}}
\pgfpathlineto{\pgfpoint{60.037266\du}{13.577347\du}}
\pgfpathlineto{\pgfpoint{60.036258\du}{13.581718\du}}
\pgfpathlineto{\pgfpoint{60.035585\du}{13.585752\du}}
\pgfpathlineto{\pgfpoint{60.034913\du}{13.589787\du}}
\pgfpathlineto{\pgfpoint{60.034913\du}{13.593822\du}}
\pgfpathlineto{\pgfpoint{60.034913\du}{13.598193\du}}
\pgfpathlineto{\pgfpoint{60.034913\du}{13.598193\du}}
\pgfpathlineto{\pgfpoint{60.034913\du}{13.602227\du}}
\pgfpathlineto{\pgfpoint{60.034913\du}{13.606598\du}}
\pgfpathlineto{\pgfpoint{60.035585\du}{13.610297\du}}
\pgfpathlineto{\pgfpoint{60.036258\du}{13.614668\du}}
\pgfpathlineto{\pgfpoint{60.037266\du}{13.618702\du}}
\pgfpathlineto{\pgfpoint{60.038275\du}{13.622737\du}}
\pgfpathlineto{\pgfpoint{60.039620\du}{13.626435\du}}
\pgfpathlineto{\pgfpoint{60.041301\du}{13.630470\du}}
\pgfpathlineto{\pgfpoint{60.043318\du}{13.634169\du}}
\pgfpathlineto{\pgfpoint{60.044663\du}{13.637867\du}}
\pgfpathlineto{\pgfpoint{60.046681\du}{13.641566\du}}
\pgfpathlineto{\pgfpoint{60.049034\du}{13.644928\du}}
\pgfpathlineto{\pgfpoint{60.051724\du}{13.648290\du}}
\pgfpathlineto{\pgfpoint{60.054078\du}{13.651316\du}}
\pgfpathlineto{\pgfpoint{60.056767\du}{13.654678\du}}
\pgfpathlineto{\pgfpoint{60.059457\du}{13.657704\du}}
\pgfpathlineto{\pgfpoint{60.062819\du}{13.660058\du}}
\pgfpathlineto{\pgfpoint{60.065845\du}{13.662748\du}}
\pgfpathlineto{\pgfpoint{60.069208\du}{13.665101\du}}
\pgfpathlineto{\pgfpoint{60.072906\du}{13.667791\du}}
\pgfpathlineto{\pgfpoint{60.075932\du}{13.669808\du}}
\pgfpathlineto{\pgfpoint{60.079631\du}{13.671489\du}}
\pgfpathlineto{\pgfpoint{60.083329\du}{13.673171\du}}
\pgfpathlineto{\pgfpoint{60.087364\du}{13.674852\du}}
\pgfpathlineto{\pgfpoint{60.091062\du}{13.676197\du}}
\pgfpathlineto{\pgfpoint{60.094761\du}{13.677541\du}}
\pgfpathlineto{\pgfpoint{60.099132\du}{13.678214\du}}
\pgfpathlineto{\pgfpoint{60.102830\du}{13.678886\du}}
\pgfpathlineto{\pgfpoint{60.106865\du}{13.679559\du}}
\pgfpathlineto{\pgfpoint{60.110899\du}{13.679895\du}}
\pgfpathlineto{\pgfpoint{60.114934\du}{13.679895\du}}
\pgfpathlineto{\pgfpoint{60.114934\du}{13.679895\du}}
\pgfpathlineto{\pgfpoint{60.119305\du}{13.679895\du}}
\pgfpathlineto{\pgfpoint{60.123340\du}{13.679559\du}}
\pgfpathlineto{\pgfpoint{60.127374\du}{13.678886\du}}
\pgfpathlineto{\pgfpoint{60.131073\du}{13.678214\du}}
\pgfpathlineto{\pgfpoint{60.135107\du}{13.677541\du}}
\pgfpathlineto{\pgfpoint{60.138806\du}{13.676197\du}}
\pgfpathlineto{\pgfpoint{60.142841\du}{13.674852\du}}
\pgfpathlineto{\pgfpoint{60.146539\du}{13.673171\du}}
\pgfpathlineto{\pgfpoint{60.150574\du}{13.671489\du}}
\pgfpathlineto{\pgfpoint{60.153936\du}{13.669808\du}}
\pgfpathlineto{\pgfpoint{60.157298\du}{13.667791\du}}
\pgfpathlineto{\pgfpoint{60.160997\du}{13.665101\du}}
\pgfpathlineto{\pgfpoint{60.164023\du}{13.662748\du}}
\pgfpathlineto{\pgfpoint{60.167385\du}{13.660058\du}}
\pgfpathlineto{\pgfpoint{60.170411\du}{13.657704\du}}
\pgfpathlineto{\pgfpoint{60.173101\du}{13.654678\du}}
\pgfpathlineto{\pgfpoint{60.176127\du}{13.651316\du}}
\pgfpathlineto{\pgfpoint{60.178480\du}{13.648290\du}}
\pgfpathlineto{\pgfpoint{60.181170\du}{13.644928\du}}
\pgfpathlineto{\pgfpoint{60.183187\du}{13.641566\du}}
\pgfpathlineto{\pgfpoint{60.185205\du}{13.637867\du}}
\pgfpathlineto{\pgfpoint{60.186886\du}{13.634169\du}}
\pgfpathlineto{\pgfpoint{60.188903\du}{13.630470\du}}
\pgfpathlineto{\pgfpoint{60.190584\du}{13.626435\du}}
\pgfpathlineto{\pgfpoint{60.191593\du}{13.622737\du}}
\pgfpathlineto{\pgfpoint{60.192938\du}{13.618702\du}}
\pgfpathlineto{\pgfpoint{60.193610\du}{13.614668\du}}
\pgfpathlineto{\pgfpoint{60.194619\du}{13.610297\du}}
\pgfpathlineto{\pgfpoint{60.194955\du}{13.606598\du}}
\pgfpathlineto{\pgfpoint{60.195292\du}{13.602227\du}}
\pgfpathlineto{\pgfpoint{60.195292\du}{13.598193\du}}
\pgfusepath{fill}
\pgfsetbuttcap
\pgfsetmiterjoin
\pgfsetdash{}{0pt}
\definecolor{dialinecolor}{rgb}{1.000000, 1.000000, 1.000000}
\pgfsetfillcolor{dialinecolor}
\pgfpathmoveto{\pgfpoint{60.929604\du}{13.598193\du}}
\pgfpathlineto{\pgfpoint{60.929604\du}{13.593822\du}}
\pgfpathlineto{\pgfpoint{60.928931\du}{13.589787\du}}
\pgfpathlineto{\pgfpoint{60.928595\du}{13.585752\du}}
\pgfpathlineto{\pgfpoint{60.928259\du}{13.581718\du}}
\pgfpathlineto{\pgfpoint{60.926914\du}{13.577347\du}}
\pgfpathlineto{\pgfpoint{60.925905\du}{13.573648\du}}
\pgfpathlineto{\pgfpoint{60.924560\du}{13.569614\du}}
\pgfpathlineto{\pgfpoint{60.922879\du}{13.565579\du}}
\pgfpathlineto{\pgfpoint{60.921198\du}{13.562217\du}}
\pgfpathlineto{\pgfpoint{60.919853\du}{13.558518\du}}
\pgfpathlineto{\pgfpoint{60.917500\du}{13.554484\du}}
\pgfpathlineto{\pgfpoint{60.915482\du}{13.551121\du}}
\pgfpathlineto{\pgfpoint{60.912793\du}{13.547759\du}}
\pgfpathlineto{\pgfpoint{60.910103\du}{13.545069\du}}
\pgfpathlineto{\pgfpoint{60.907077\du}{13.541371\du}}
\pgfpathlineto{\pgfpoint{60.904387\du}{13.538681\du}}
\pgfpathlineto{\pgfpoint{60.901361\du}{13.535991\du}}
\pgfpathlineto{\pgfpoint{60.898671\du}{13.533638\du}}
\pgfpathlineto{\pgfpoint{60.894973\du}{13.530948\du}}
\pgfpathlineto{\pgfpoint{60.891610\du}{13.528594\du}}
\pgfpathlineto{\pgfpoint{60.888248\du}{13.526241\du}}
\pgfpathlineto{\pgfpoint{60.884550\du}{13.524560\du}}
\pgfpathlineto{\pgfpoint{60.880851\du}{13.522879\du}}
\pgfpathlineto{\pgfpoint{60.876817\du}{13.521198\du}}
\pgfpathlineto{\pgfpoint{60.872782\du}{13.519853\du}}
\pgfpathlineto{\pgfpoint{60.869083\du}{13.518844\du}}
\pgfpathlineto{\pgfpoint{60.865049\du}{13.517835\du}}
\pgfpathlineto{\pgfpoint{60.861014\du}{13.517163\du}}
\pgfpathlineto{\pgfpoint{60.856643\du}{13.516490\du}}
\pgfpathlineto{\pgfpoint{60.852945\du}{13.516154\du}}
\pgfpathlineto{\pgfpoint{60.848574\du}{13.516154\du}}
\pgfpathlineto{\pgfpoint{60.848574\du}{13.516154\du}}
\pgfpathlineto{\pgfpoint{60.844539\du}{13.516154\du}}
\pgfpathlineto{\pgfpoint{60.840841\du}{13.516490\du}}
\pgfpathlineto{\pgfpoint{60.836470\du}{13.517163\du}}
\pgfpathlineto{\pgfpoint{60.832435\du}{13.517835\du}}
\pgfpathlineto{\pgfpoint{60.828400\du}{13.518844\du}}
\pgfpathlineto{\pgfpoint{60.825038\du}{13.519853\du}}
\pgfpathlineto{\pgfpoint{60.821004\du}{13.521198\du}}
\pgfpathlineto{\pgfpoint{60.816633\du}{13.522879\du}}
\pgfpathlineto{\pgfpoint{60.812934\du}{13.524560\du}}
\pgfpathlineto{\pgfpoint{60.809572\du}{13.526241\du}}
\pgfpathlineto{\pgfpoint{60.806210\du}{13.528594\du}}
\pgfpathlineto{\pgfpoint{60.802511\du}{13.530948\du}}
\pgfpathlineto{\pgfpoint{60.799149\du}{13.533638\du}}
\pgfpathlineto{\pgfpoint{60.796123\du}{13.535991\du}}
\pgfpathlineto{\pgfpoint{60.793433\du}{13.538681\du}}
\pgfpathlineto{\pgfpoint{60.790071\du}{13.541371\du}}
\pgfpathlineto{\pgfpoint{60.787381\du}{13.545069\du}}
\pgfpathlineto{\pgfpoint{60.785028\du}{13.547759\du}}
\pgfpathlineto{\pgfpoint{60.782338\du}{13.551121\du}}
\pgfpathlineto{\pgfpoint{60.779984\du}{13.554484\du}}
\pgfpathlineto{\pgfpoint{60.777967\du}{13.558518\du}}
\pgfpathlineto{\pgfpoint{60.776622\du}{13.562217\du}}
\pgfpathlineto{\pgfpoint{60.774605\du}{13.565579\du}}
\pgfpathlineto{\pgfpoint{60.772924\du}{13.569614\du}}
\pgfpathlineto{\pgfpoint{60.771579\du}{13.573648\du}}
\pgfpathlineto{\pgfpoint{60.770570\du}{13.577347\du}}
\pgfpathlineto{\pgfpoint{60.769561\du}{13.581718\du}}
\pgfpathlineto{\pgfpoint{60.768889\du}{13.585752\du}}
\pgfpathlineto{\pgfpoint{60.768553\du}{13.589787\du}}
\pgfpathlineto{\pgfpoint{60.768216\du}{13.593822\du}}
\pgfpathlineto{\pgfpoint{60.768216\du}{13.598193\du}}
\pgfpathlineto{\pgfpoint{60.768216\du}{13.598193\du}}
\pgfpathlineto{\pgfpoint{60.768216\du}{13.602227\du}}
\pgfpathlineto{\pgfpoint{60.768553\du}{13.606598\du}}
\pgfpathlineto{\pgfpoint{60.768889\du}{13.610297\du}}
\pgfpathlineto{\pgfpoint{60.769561\du}{13.614668\du}}
\pgfpathlineto{\pgfpoint{60.770570\du}{13.618702\du}}
\pgfpathlineto{\pgfpoint{60.771579\du}{13.622737\du}}
\pgfpathlineto{\pgfpoint{60.772924\du}{13.626435\du}}
\pgfpathlineto{\pgfpoint{60.774605\du}{13.630470\du}}
\pgfpathlineto{\pgfpoint{60.776622\du}{13.634169\du}}
\pgfpathlineto{\pgfpoint{60.777967\du}{13.637867\du}}
\pgfpathlineto{\pgfpoint{60.779984\du}{13.641566\du}}
\pgfpathlineto{\pgfpoint{60.782338\du}{13.644928\du}}
\pgfpathlineto{\pgfpoint{60.785028\du}{13.648290\du}}
\pgfpathlineto{\pgfpoint{60.787381\du}{13.651316\du}}
\pgfpathlineto{\pgfpoint{60.790071\du}{13.654678\du}}
\pgfpathlineto{\pgfpoint{60.793433\du}{13.657704\du}}
\pgfpathlineto{\pgfpoint{60.796123\du}{13.660058\du}}
\pgfpathlineto{\pgfpoint{60.799149\du}{13.662748\du}}
\pgfpathlineto{\pgfpoint{60.802511\du}{13.665101\du}}
\pgfpathlineto{\pgfpoint{60.806210\du}{13.667791\du}}
\pgfpathlineto{\pgfpoint{60.809572\du}{13.669808\du}}
\pgfpathlineto{\pgfpoint{60.812934\du}{13.671489\du}}
\pgfpathlineto{\pgfpoint{60.816633\du}{13.673171\du}}
\pgfpathlineto{\pgfpoint{60.821004\du}{13.674852\du}}
\pgfpathlineto{\pgfpoint{60.825038\du}{13.676197\du}}
\pgfpathlineto{\pgfpoint{60.828400\du}{13.677541\du}}
\pgfpathlineto{\pgfpoint{60.832435\du}{13.678214\du}}
\pgfpathlineto{\pgfpoint{60.836470\du}{13.678886\du}}
\pgfpathlineto{\pgfpoint{60.840841\du}{13.679559\du}}
\pgfpathlineto{\pgfpoint{60.844539\du}{13.679895\du}}
\pgfpathlineto{\pgfpoint{60.848574\du}{13.679895\du}}
\pgfpathlineto{\pgfpoint{60.848574\du}{13.679895\du}}
\pgfpathlineto{\pgfpoint{60.852945\du}{13.679895\du}}
\pgfpathlineto{\pgfpoint{60.856643\du}{13.679559\du}}
\pgfpathlineto{\pgfpoint{60.861014\du}{13.678886\du}}
\pgfpathlineto{\pgfpoint{60.865049\du}{13.678214\du}}
\pgfpathlineto{\pgfpoint{60.869083\du}{13.677541\du}}
\pgfpathlineto{\pgfpoint{60.872782\du}{13.676197\du}}
\pgfpathlineto{\pgfpoint{60.876817\du}{13.674852\du}}
\pgfpathlineto{\pgfpoint{60.880851\du}{13.673171\du}}
\pgfpathlineto{\pgfpoint{60.884550\du}{13.671489\du}}
\pgfpathlineto{\pgfpoint{60.888248\du}{13.669808\du}}
\pgfpathlineto{\pgfpoint{60.891610\du}{13.667791\du}}
\pgfpathlineto{\pgfpoint{60.894973\du}{13.665101\du}}
\pgfpathlineto{\pgfpoint{60.898671\du}{13.662748\du}}
\pgfpathlineto{\pgfpoint{60.901361\du}{13.660058\du}}
\pgfpathlineto{\pgfpoint{60.904387\du}{13.657704\du}}
\pgfpathlineto{\pgfpoint{60.907077\du}{13.654678\du}}
\pgfpathlineto{\pgfpoint{60.910103\du}{13.651316\du}}
\pgfpathlineto{\pgfpoint{60.912793\du}{13.648290\du}}
\pgfpathlineto{\pgfpoint{60.915482\du}{13.644928\du}}
\pgfpathlineto{\pgfpoint{60.917500\du}{13.641566\du}}
\pgfpathlineto{\pgfpoint{60.919853\du}{13.637867\du}}
\pgfpathlineto{\pgfpoint{60.921198\du}{13.634169\du}}
\pgfpathlineto{\pgfpoint{60.922879\du}{13.630470\du}}
\pgfpathlineto{\pgfpoint{60.924560\du}{13.626435\du}}
\pgfpathlineto{\pgfpoint{60.925905\du}{13.622737\du}}
\pgfpathlineto{\pgfpoint{60.926914\du}{13.618702\du}}
\pgfpathlineto{\pgfpoint{60.928259\du}{13.614668\du}}
\pgfpathlineto{\pgfpoint{60.928595\du}{13.610297\du}}
\pgfpathlineto{\pgfpoint{60.928931\du}{13.606598\du}}
\pgfpathlineto{\pgfpoint{60.929604\du}{13.602227\du}}
\pgfpathlineto{\pgfpoint{60.929604\du}{13.598193\du}}
\pgfusepath{fill}
\pgfsetlinewidth{0.000000\du}
\pgfsetbuttcap
\pgfsetmiterjoin
\pgfsetdash{}{0pt}
\definecolor{dialinecolor}{rgb}{1.000000, 1.000000, 1.000000}
\pgfsetstrokecolor{dialinecolor}
\pgfpathmoveto{\pgfpoint{60.122667\du}{13.591804\du}}
\pgfpathlineto{\pgfpoint{60.827056\du}{13.591804\du}}
\pgfusepath{stroke}
\pgfsetlinewidth{0.000000\du}
\pgfsetbuttcap
\pgfsetmiterjoin
\pgfsetdash{}{0pt}
\definecolor{dialinecolor}{rgb}{1.000000, 1.000000, 1.000000}
\pgfsetfillcolor{dialinecolor}
\pgfpathmoveto{\pgfpoint{60.195292\du}{13.895751\du}}
\pgfpathlineto{\pgfpoint{60.195292\du}{13.891716\du}}
\pgfpathlineto{\pgfpoint{60.194955\du}{13.887345\du}}
\pgfpathlineto{\pgfpoint{60.194619\du}{13.883646\du}}
\pgfpathlineto{\pgfpoint{60.193610\du}{13.879276\du}}
\pgfpathlineto{\pgfpoint{60.192938\du}{13.875577\du}}
\pgfpathlineto{\pgfpoint{60.191593\du}{13.871206\du}}
\pgfpathlineto{\pgfpoint{60.190584\du}{13.867508\du}}
\pgfpathlineto{\pgfpoint{60.188903\du}{13.863809\du}}
\pgfpathlineto{\pgfpoint{60.186886\du}{13.859775\du}}
\pgfpathlineto{\pgfpoint{60.185205\du}{13.856076\du}}
\pgfpathlineto{\pgfpoint{60.183187\du}{13.852714\du}}
\pgfpathlineto{\pgfpoint{60.181170\du}{13.849015\du}}
\pgfpathlineto{\pgfpoint{60.178480\du}{13.845989\du}}
\pgfpathlineto{\pgfpoint{60.176127\du}{13.842627\du}}
\pgfpathlineto{\pgfpoint{60.173101\du}{13.839601\du}}
\pgfpathlineto{\pgfpoint{60.170411\du}{13.836575\du}}
\pgfpathlineto{\pgfpoint{60.167385\du}{13.834222\du}}
\pgfpathlineto{\pgfpoint{60.164023\du}{13.831196\du}}
\pgfpathlineto{\pgfpoint{60.160997\du}{13.829178\du}}
\pgfpathlineto{\pgfpoint{60.157298\du}{13.826488\du}}
\pgfpathlineto{\pgfpoint{60.153936\du}{13.824471\du}}
\pgfpathlineto{\pgfpoint{60.150574\du}{13.822790\du}}
\pgfpathlineto{\pgfpoint{60.146539\du}{13.821109\du}}
\pgfpathlineto{\pgfpoint{60.142841\du}{13.819428\du}}
\pgfpathlineto{\pgfpoint{60.138806\du}{13.818083\du}}
\pgfpathlineto{\pgfpoint{60.135107\du}{13.816738\du}}
\pgfpathlineto{\pgfpoint{60.131073\du}{13.816066\du}}
\pgfpathlineto{\pgfpoint{60.127374\du}{13.815057\du}}
\pgfpathlineto{\pgfpoint{60.123340\du}{13.814721\du}}
\pgfpathlineto{\pgfpoint{60.119305\du}{13.814384\du}}
\pgfpathlineto{\pgfpoint{60.114934\du}{13.814384\du}}
\pgfpathlineto{\pgfpoint{60.114934\du}{13.814384\du}}
\pgfpathlineto{\pgfpoint{60.110899\du}{13.814384\du}}
\pgfpathlineto{\pgfpoint{60.106865\du}{13.814721\du}}
\pgfpathlineto{\pgfpoint{60.102830\du}{13.815057\du}}
\pgfpathlineto{\pgfpoint{60.099132\du}{13.816066\du}}
\pgfpathlineto{\pgfpoint{60.094761\du}{13.816738\du}}
\pgfpathlineto{\pgfpoint{60.091062\du}{13.818083\du}}
\pgfpathlineto{\pgfpoint{60.087364\du}{13.819428\du}}
\pgfpathlineto{\pgfpoint{60.083329\du}{13.821109\du}}
\pgfpathlineto{\pgfpoint{60.079631\du}{13.822790\du}}
\pgfpathlineto{\pgfpoint{60.075932\du}{13.824471\du}}
\pgfpathlineto{\pgfpoint{60.072906\du}{13.826488\du}}
\pgfpathlineto{\pgfpoint{60.069208\du}{13.829178\du}}
\pgfpathlineto{\pgfpoint{60.065845\du}{13.831196\du}}
\pgfpathlineto{\pgfpoint{60.062819\du}{13.834222\du}}
\pgfpathlineto{\pgfpoint{60.059457\du}{13.836575\du}}
\pgfpathlineto{\pgfpoint{60.056767\du}{13.839601\du}}
\pgfpathlineto{\pgfpoint{60.054078\du}{13.842627\du}}
\pgfpathlineto{\pgfpoint{60.051724\du}{13.845989\du}}
\pgfpathlineto{\pgfpoint{60.049034\du}{13.849015\du}}
\pgfpathlineto{\pgfpoint{60.046681\du}{13.852714\du}}
\pgfpathlineto{\pgfpoint{60.044663\du}{13.856076\du}}
\pgfpathlineto{\pgfpoint{60.043318\du}{13.859775\du}}
\pgfpathlineto{\pgfpoint{60.041301\du}{13.863809\du}}
\pgfpathlineto{\pgfpoint{60.039620\du}{13.867508\du}}
\pgfpathlineto{\pgfpoint{60.038275\du}{13.871206\du}}
\pgfpathlineto{\pgfpoint{60.037266\du}{13.875577\du}}
\pgfpathlineto{\pgfpoint{60.036258\du}{13.879276\du}}
\pgfpathlineto{\pgfpoint{60.035585\du}{13.883646\du}}
\pgfpathlineto{\pgfpoint{60.034913\du}{13.887345\du}}
\pgfpathlineto{\pgfpoint{60.034913\du}{13.891716\du}}
\pgfpathlineto{\pgfpoint{60.034913\du}{13.895751\du}}
\pgfpathlineto{\pgfpoint{60.034913\du}{13.895751\du}}
\pgfpathlineto{\pgfpoint{60.034913\du}{13.900121\du}}
\pgfpathlineto{\pgfpoint{60.034913\du}{13.904156\du}}
\pgfpathlineto{\pgfpoint{60.035585\du}{13.907855\du}}
\pgfpathlineto{\pgfpoint{60.036258\du}{13.912225\du}}
\pgfpathlineto{\pgfpoint{60.037266\du}{13.916260\du}}
\pgfpathlineto{\pgfpoint{60.038275\du}{13.920295\du}}
\pgfpathlineto{\pgfpoint{60.039620\du}{13.923993\du}}
\pgfpathlineto{\pgfpoint{60.041301\du}{13.928028\du}}
\pgfpathlineto{\pgfpoint{60.043318\du}{13.931726\du}}
\pgfpathlineto{\pgfpoint{60.044663\du}{13.935425\du}}
\pgfpathlineto{\pgfpoint{60.046681\du}{13.938787\du}}
\pgfpathlineto{\pgfpoint{60.049034\du}{13.942149\du}}
\pgfpathlineto{\pgfpoint{60.051724\du}{13.945512\du}}
\pgfpathlineto{\pgfpoint{60.054078\du}{13.949210\du}}
\pgfpathlineto{\pgfpoint{60.056767\du}{13.951900\du}}
\pgfpathlineto{\pgfpoint{60.059457\du}{13.954926\du}}
\pgfpathlineto{\pgfpoint{60.062819\du}{13.957616\du}}
\pgfpathlineto{\pgfpoint{60.065845\du}{13.960305\du}}
\pgfpathlineto{\pgfpoint{60.069208\du}{13.962659\du}}
\pgfpathlineto{\pgfpoint{60.072906\du}{13.964676\du}}
\pgfpathlineto{\pgfpoint{60.075932\du}{13.967366\du}}
\pgfpathlineto{\pgfpoint{60.079631\du}{13.969047\du}}
\pgfpathlineto{\pgfpoint{60.083329\du}{13.970728\du}}
\pgfpathlineto{\pgfpoint{60.087364\du}{13.972410\du}}
\pgfpathlineto{\pgfpoint{60.091062\du}{13.973082\du}}
\pgfpathlineto{\pgfpoint{60.094761\du}{13.974763\du}}
\pgfpathlineto{\pgfpoint{60.099132\du}{13.975772\du}}
\pgfpathlineto{\pgfpoint{60.102830\du}{13.976444\du}}
\pgfpathlineto{\pgfpoint{60.106865\du}{13.976780\du}}
\pgfpathlineto{\pgfpoint{60.110899\du}{13.977453\du}}
\pgfpathlineto{\pgfpoint{60.114934\du}{13.977453\du}}
\pgfpathlineto{\pgfpoint{60.114934\du}{13.977453\du}}
\pgfpathlineto{\pgfpoint{60.119305\du}{13.977453\du}}
\pgfpathlineto{\pgfpoint{60.123340\du}{13.976780\du}}
\pgfpathlineto{\pgfpoint{60.127374\du}{13.976444\du}}
\pgfpathlineto{\pgfpoint{60.131073\du}{13.975772\du}}
\pgfpathlineto{\pgfpoint{60.135107\du}{13.974763\du}}
\pgfpathlineto{\pgfpoint{60.138806\du}{13.973082\du}}
\pgfpathlineto{\pgfpoint{60.142841\du}{13.972410\du}}
\pgfpathlineto{\pgfpoint{60.146539\du}{13.970728\du}}
\pgfpathlineto{\pgfpoint{60.150574\du}{13.969047\du}}
\pgfpathlineto{\pgfpoint{60.153936\du}{13.967366\du}}
\pgfpathlineto{\pgfpoint{60.157298\du}{13.964676\du}}
\pgfpathlineto{\pgfpoint{60.160997\du}{13.962659\du}}
\pgfpathlineto{\pgfpoint{60.164023\du}{13.960305\du}}
\pgfpathlineto{\pgfpoint{60.167385\du}{13.957616\du}}
\pgfpathlineto{\pgfpoint{60.170411\du}{13.954926\du}}
\pgfpathlineto{\pgfpoint{60.173101\du}{13.951900\du}}
\pgfpathlineto{\pgfpoint{60.176127\du}{13.949210\du}}
\pgfpathlineto{\pgfpoint{60.178480\du}{13.945512\du}}
\pgfpathlineto{\pgfpoint{60.181170\du}{13.942149\du}}
\pgfpathlineto{\pgfpoint{60.183187\du}{13.938787\du}}
\pgfpathlineto{\pgfpoint{60.185205\du}{13.935425\du}}
\pgfpathlineto{\pgfpoint{60.186886\du}{13.931726\du}}
\pgfpathlineto{\pgfpoint{60.188903\du}{13.928028\du}}
\pgfpathlineto{\pgfpoint{60.190584\du}{13.923993\du}}
\pgfpathlineto{\pgfpoint{60.191593\du}{13.920295\du}}
\pgfpathlineto{\pgfpoint{60.192938\du}{13.916260\du}}
\pgfpathlineto{\pgfpoint{60.193610\du}{13.912225\du}}
\pgfpathlineto{\pgfpoint{60.194619\du}{13.907855\du}}
\pgfpathlineto{\pgfpoint{60.194955\du}{13.904156\du}}
\pgfpathlineto{\pgfpoint{60.195292\du}{13.900121\du}}
\pgfpathlineto{\pgfpoint{60.195292\du}{13.895751\du}}
\pgfusepath{fill}
\pgfsetbuttcap
\pgfsetmiterjoin
\pgfsetdash{}{0pt}
\definecolor{dialinecolor}{rgb}{1.000000, 1.000000, 1.000000}
\pgfsetfillcolor{dialinecolor}
\pgfpathmoveto{\pgfpoint{60.929604\du}{13.895751\du}}
\pgfpathlineto{\pgfpoint{60.929604\du}{13.891716\du}}
\pgfpathlineto{\pgfpoint{60.928931\du}{13.887345\du}}
\pgfpathlineto{\pgfpoint{60.928595\du}{13.883646\du}}
\pgfpathlineto{\pgfpoint{60.928259\du}{13.879276\du}}
\pgfpathlineto{\pgfpoint{60.926914\du}{13.875577\du}}
\pgfpathlineto{\pgfpoint{60.925905\du}{13.871206\du}}
\pgfpathlineto{\pgfpoint{60.924560\du}{13.867508\du}}
\pgfpathlineto{\pgfpoint{60.922879\du}{13.863809\du}}
\pgfpathlineto{\pgfpoint{60.921198\du}{13.859775\du}}
\pgfpathlineto{\pgfpoint{60.919853\du}{13.856076\du}}
\pgfpathlineto{\pgfpoint{60.917500\du}{13.852714\du}}
\pgfpathlineto{\pgfpoint{60.915482\du}{13.849015\du}}
\pgfpathlineto{\pgfpoint{60.912793\du}{13.845989\du}}
\pgfpathlineto{\pgfpoint{60.910103\du}{13.842627\du}}
\pgfpathlineto{\pgfpoint{60.907077\du}{13.839601\du}}
\pgfpathlineto{\pgfpoint{60.904387\du}{13.836575\du}}
\pgfpathlineto{\pgfpoint{60.901361\du}{13.834222\du}}
\pgfpathlineto{\pgfpoint{60.898671\du}{13.831196\du}}
\pgfpathlineto{\pgfpoint{60.894973\du}{13.829178\du}}
\pgfpathlineto{\pgfpoint{60.891610\du}{13.826488\du}}
\pgfpathlineto{\pgfpoint{60.888248\du}{13.824471\du}}
\pgfpathlineto{\pgfpoint{60.884550\du}{13.822790\du}}
\pgfpathlineto{\pgfpoint{60.880851\du}{13.821109\du}}
\pgfpathlineto{\pgfpoint{60.876817\du}{13.819428\du}}
\pgfpathlineto{\pgfpoint{60.872782\du}{13.818083\du}}
\pgfpathlineto{\pgfpoint{60.869083\du}{13.816738\du}}
\pgfpathlineto{\pgfpoint{60.865049\du}{13.816066\du}}
\pgfpathlineto{\pgfpoint{60.861014\du}{13.815057\du}}
\pgfpathlineto{\pgfpoint{60.856643\du}{13.814721\du}}
\pgfpathlineto{\pgfpoint{60.852945\du}{13.814384\du}}
\pgfpathlineto{\pgfpoint{60.848574\du}{13.814384\du}}
\pgfpathlineto{\pgfpoint{60.848574\du}{13.814384\du}}
\pgfpathlineto{\pgfpoint{60.844539\du}{13.814384\du}}
\pgfpathlineto{\pgfpoint{60.840841\du}{13.814721\du}}
\pgfpathlineto{\pgfpoint{60.836470\du}{13.815057\du}}
\pgfpathlineto{\pgfpoint{60.832435\du}{13.816066\du}}
\pgfpathlineto{\pgfpoint{60.828400\du}{13.816738\du}}
\pgfpathlineto{\pgfpoint{60.825038\du}{13.818083\du}}
\pgfpathlineto{\pgfpoint{60.821004\du}{13.819428\du}}
\pgfpathlineto{\pgfpoint{60.816633\du}{13.821109\du}}
\pgfpathlineto{\pgfpoint{60.812934\du}{13.822790\du}}
\pgfpathlineto{\pgfpoint{60.809572\du}{13.824471\du}}
\pgfpathlineto{\pgfpoint{60.806210\du}{13.826488\du}}
\pgfpathlineto{\pgfpoint{60.802511\du}{13.829178\du}}
\pgfpathlineto{\pgfpoint{60.799149\du}{13.831196\du}}
\pgfpathlineto{\pgfpoint{60.796123\du}{13.834222\du}}
\pgfpathlineto{\pgfpoint{60.793433\du}{13.836575\du}}
\pgfpathlineto{\pgfpoint{60.790071\du}{13.839601\du}}
\pgfpathlineto{\pgfpoint{60.787381\du}{13.842627\du}}
\pgfpathlineto{\pgfpoint{60.785028\du}{13.845989\du}}
\pgfpathlineto{\pgfpoint{60.782338\du}{13.849015\du}}
\pgfpathlineto{\pgfpoint{60.779984\du}{13.852714\du}}
\pgfpathlineto{\pgfpoint{60.777967\du}{13.856076\du}}
\pgfpathlineto{\pgfpoint{60.776622\du}{13.859775\du}}
\pgfpathlineto{\pgfpoint{60.774605\du}{13.863809\du}}
\pgfpathlineto{\pgfpoint{60.772924\du}{13.867508\du}}
\pgfpathlineto{\pgfpoint{60.771579\du}{13.871206\du}}
\pgfpathlineto{\pgfpoint{60.770570\du}{13.875577\du}}
\pgfpathlineto{\pgfpoint{60.769561\du}{13.879276\du}}
\pgfpathlineto{\pgfpoint{60.768889\du}{13.883646\du}}
\pgfpathlineto{\pgfpoint{60.768553\du}{13.887345\du}}
\pgfpathlineto{\pgfpoint{60.768216\du}{13.891716\du}}
\pgfpathlineto{\pgfpoint{60.768216\du}{13.895751\du}}
\pgfpathlineto{\pgfpoint{60.768216\du}{13.895751\du}}
\pgfpathlineto{\pgfpoint{60.768216\du}{13.900121\du}}
\pgfpathlineto{\pgfpoint{60.768553\du}{13.904156\du}}
\pgfpathlineto{\pgfpoint{60.768889\du}{13.907855\du}}
\pgfpathlineto{\pgfpoint{60.769561\du}{13.912225\du}}
\pgfpathlineto{\pgfpoint{60.770570\du}{13.916260\du}}
\pgfpathlineto{\pgfpoint{60.771579\du}{13.920295\du}}
\pgfpathlineto{\pgfpoint{60.772924\du}{13.923993\du}}
\pgfpathlineto{\pgfpoint{60.774605\du}{13.928028\du}}
\pgfpathlineto{\pgfpoint{60.776622\du}{13.931726\du}}
\pgfpathlineto{\pgfpoint{60.777967\du}{13.935425\du}}
\pgfpathlineto{\pgfpoint{60.779984\du}{13.938787\du}}
\pgfpathlineto{\pgfpoint{60.782338\du}{13.942149\du}}
\pgfpathlineto{\pgfpoint{60.785028\du}{13.945512\du}}
\pgfpathlineto{\pgfpoint{60.787381\du}{13.949210\du}}
\pgfpathlineto{\pgfpoint{60.790071\du}{13.951900\du}}
\pgfpathlineto{\pgfpoint{60.793433\du}{13.954926\du}}
\pgfpathlineto{\pgfpoint{60.796123\du}{13.957616\du}}
\pgfpathlineto{\pgfpoint{60.799149\du}{13.960305\du}}
\pgfpathlineto{\pgfpoint{60.802511\du}{13.962659\du}}
\pgfpathlineto{\pgfpoint{60.806210\du}{13.964676\du}}
\pgfpathlineto{\pgfpoint{60.809572\du}{13.967366\du}}
\pgfpathlineto{\pgfpoint{60.812934\du}{13.969047\du}}
\pgfpathlineto{\pgfpoint{60.816633\du}{13.970728\du}}
\pgfpathlineto{\pgfpoint{60.821004\du}{13.972410\du}}
\pgfpathlineto{\pgfpoint{60.825038\du}{13.973082\du}}
\pgfpathlineto{\pgfpoint{60.828400\du}{13.974763\du}}
\pgfpathlineto{\pgfpoint{60.832435\du}{13.975772\du}}
\pgfpathlineto{\pgfpoint{60.836470\du}{13.976444\du}}
\pgfpathlineto{\pgfpoint{60.840841\du}{13.976780\du}}
\pgfpathlineto{\pgfpoint{60.844539\du}{13.977453\du}}
\pgfpathlineto{\pgfpoint{60.848574\du}{13.977453\du}}
\pgfpathlineto{\pgfpoint{60.848574\du}{13.977453\du}}
\pgfpathlineto{\pgfpoint{60.852945\du}{13.977453\du}}
\pgfpathlineto{\pgfpoint{60.856643\du}{13.976780\du}}
\pgfpathlineto{\pgfpoint{60.861014\du}{13.976444\du}}
\pgfpathlineto{\pgfpoint{60.865049\du}{13.975772\du}}
\pgfpathlineto{\pgfpoint{60.869083\du}{13.974763\du}}
\pgfpathlineto{\pgfpoint{60.872782\du}{13.973082\du}}
\pgfpathlineto{\pgfpoint{60.876817\du}{13.972410\du}}
\pgfpathlineto{\pgfpoint{60.880851\du}{13.970728\du}}
\pgfpathlineto{\pgfpoint{60.884550\du}{13.969047\du}}
\pgfpathlineto{\pgfpoint{60.888248\du}{13.967366\du}}
\pgfpathlineto{\pgfpoint{60.891610\du}{13.964676\du}}
\pgfpathlineto{\pgfpoint{60.894973\du}{13.962659\du}}
\pgfpathlineto{\pgfpoint{60.898671\du}{13.960305\du}}
\pgfpathlineto{\pgfpoint{60.901361\du}{13.957616\du}}
\pgfpathlineto{\pgfpoint{60.904387\du}{13.954926\du}}
\pgfpathlineto{\pgfpoint{60.907077\du}{13.951900\du}}
\pgfpathlineto{\pgfpoint{60.910103\du}{13.949210\du}}
\pgfpathlineto{\pgfpoint{60.912793\du}{13.945512\du}}
\pgfpathlineto{\pgfpoint{60.915482\du}{13.942149\du}}
\pgfpathlineto{\pgfpoint{60.917500\du}{13.938787\du}}
\pgfpathlineto{\pgfpoint{60.919853\du}{13.935425\du}}
\pgfpathlineto{\pgfpoint{60.921198\du}{13.931726\du}}
\pgfpathlineto{\pgfpoint{60.922879\du}{13.928028\du}}
\pgfpathlineto{\pgfpoint{60.924560\du}{13.923993\du}}
\pgfpathlineto{\pgfpoint{60.925905\du}{13.920295\du}}
\pgfpathlineto{\pgfpoint{60.926914\du}{13.916260\du}}
\pgfpathlineto{\pgfpoint{60.928259\du}{13.912225\du}}
\pgfpathlineto{\pgfpoint{60.928595\du}{13.907855\du}}
\pgfpathlineto{\pgfpoint{60.928931\du}{13.904156\du}}
\pgfpathlineto{\pgfpoint{60.929604\du}{13.900121\du}}
\pgfpathlineto{\pgfpoint{60.929604\du}{13.895751\du}}
\pgfusepath{fill}
\pgfsetlinewidth{0.000000\du}
\pgfsetbuttcap
\pgfsetmiterjoin
\pgfsetdash{}{0pt}
\definecolor{dialinecolor}{rgb}{1.000000, 1.000000, 1.000000}
\pgfsetstrokecolor{dialinecolor}
\pgfpathmoveto{\pgfpoint{60.122667\du}{13.889026\du}}
\pgfpathlineto{\pgfpoint{60.827056\du}{13.889026\du}}
\pgfusepath{stroke}
\pgfsetlinewidth{0.000000\du}
\pgfsetbuttcap
\pgfsetmiterjoin
\pgfsetdash{}{0pt}
\definecolor{dialinecolor}{rgb}{1.000000, 1.000000, 1.000000}
\pgfsetfillcolor{dialinecolor}
\pgfpathmoveto{\pgfpoint{60.195292\du}{14.193645\du}}
\pgfpathlineto{\pgfpoint{60.195292\du}{14.189610\du}}
\pgfpathlineto{\pgfpoint{60.194955\du}{14.185239\du}}
\pgfpathlineto{\pgfpoint{60.194619\du}{14.181877\du}}
\pgfpathlineto{\pgfpoint{60.193610\du}{14.177506\du}}
\pgfpathlineto{\pgfpoint{60.192938\du}{14.173135\du}}
\pgfpathlineto{\pgfpoint{60.191593\du}{14.169437\du}}
\pgfpathlineto{\pgfpoint{60.190584\du}{14.165738\du}}
\pgfpathlineto{\pgfpoint{60.188903\du}{14.161367\du}}
\pgfpathlineto{\pgfpoint{60.186886\du}{14.158005\du}}
\pgfpathlineto{\pgfpoint{60.185205\du}{14.154306\du}}
\pgfpathlineto{\pgfpoint{60.183187\du}{14.150608\du}}
\pgfpathlineto{\pgfpoint{60.181170\du}{14.147246\du}}
\pgfpathlineto{\pgfpoint{60.178480\du}{14.143884\du}}
\pgfpathlineto{\pgfpoint{60.176127\du}{14.140858\du}}
\pgfpathlineto{\pgfpoint{60.173101\du}{14.137495\du}}
\pgfpathlineto{\pgfpoint{60.170411\du}{14.134469\du}}
\pgfpathlineto{\pgfpoint{60.167385\du}{14.131779\du}}
\pgfpathlineto{\pgfpoint{60.164023\du}{14.129426\du}}
\pgfpathlineto{\pgfpoint{60.160997\du}{14.126400\du}}
\pgfpathlineto{\pgfpoint{60.157298\du}{14.124383\du}}
\pgfpathlineto{\pgfpoint{60.153936\du}{14.122029\du}}
\pgfpathlineto{\pgfpoint{60.150574\du}{14.120348\du}}
\pgfpathlineto{\pgfpoint{60.146539\du}{14.118667\du}}
\pgfpathlineto{\pgfpoint{60.142841\du}{14.116986\du}}
\pgfpathlineto{\pgfpoint{60.138806\du}{14.115977\du}}
\pgfpathlineto{\pgfpoint{60.135107\du}{14.114632\du}}
\pgfpathlineto{\pgfpoint{60.131073\du}{14.113623\du}}
\pgfpathlineto{\pgfpoint{60.127374\du}{14.112951\du}}
\pgfpathlineto{\pgfpoint{60.123340\du}{14.112615\du}}
\pgfpathlineto{\pgfpoint{60.119305\du}{14.111942\du}}
\pgfpathlineto{\pgfpoint{60.114934\du}{14.111942\du}}
\pgfpathlineto{\pgfpoint{60.114934\du}{14.111942\du}}
\pgfpathlineto{\pgfpoint{60.110899\du}{14.111942\du}}
\pgfpathlineto{\pgfpoint{60.106865\du}{14.112615\du}}
\pgfpathlineto{\pgfpoint{60.102830\du}{14.112951\du}}
\pgfpathlineto{\pgfpoint{60.099132\du}{14.113623\du}}
\pgfpathlineto{\pgfpoint{60.094761\du}{14.114632\du}}
\pgfpathlineto{\pgfpoint{60.091062\du}{14.115977\du}}
\pgfpathlineto{\pgfpoint{60.087364\du}{14.116986\du}}
\pgfpathlineto{\pgfpoint{60.083329\du}{14.118667\du}}
\pgfpathlineto{\pgfpoint{60.079631\du}{14.120348\du}}
\pgfpathlineto{\pgfpoint{60.075932\du}{14.122029\du}}
\pgfpathlineto{\pgfpoint{60.072906\du}{14.124383\du}}
\pgfpathlineto{\pgfpoint{60.069208\du}{14.126400\du}}
\pgfpathlineto{\pgfpoint{60.065845\du}{14.129426\du}}
\pgfpathlineto{\pgfpoint{60.062819\du}{14.131779\du}}
\pgfpathlineto{\pgfpoint{60.059457\du}{14.134469\du}}
\pgfpathlineto{\pgfpoint{60.056767\du}{14.137495\du}}
\pgfpathlineto{\pgfpoint{60.054078\du}{14.140858\du}}
\pgfpathlineto{\pgfpoint{60.051724\du}{14.143884\du}}
\pgfpathlineto{\pgfpoint{60.049034\du}{14.147246\du}}
\pgfpathlineto{\pgfpoint{60.046681\du}{14.150608\du}}
\pgfpathlineto{\pgfpoint{60.044663\du}{14.154306\du}}
\pgfpathlineto{\pgfpoint{60.043318\du}{14.158005\du}}
\pgfpathlineto{\pgfpoint{60.041301\du}{14.161367\du}}
\pgfpathlineto{\pgfpoint{60.039620\du}{14.165738\du}}
\pgfpathlineto{\pgfpoint{60.038275\du}{14.169437\du}}
\pgfpathlineto{\pgfpoint{60.037266\du}{14.173135\du}}
\pgfpathlineto{\pgfpoint{60.036258\du}{14.177506\du}}
\pgfpathlineto{\pgfpoint{60.035585\du}{14.181877\du}}
\pgfpathlineto{\pgfpoint{60.034913\du}{14.185239\du}}
\pgfpathlineto{\pgfpoint{60.034913\du}{14.189610\du}}
\pgfpathlineto{\pgfpoint{60.034913\du}{14.193645\du}}
\pgfpathlineto{\pgfpoint{60.034913\du}{14.193645\du}}
\pgfpathlineto{\pgfpoint{60.034913\du}{14.198352\du}}
\pgfpathlineto{\pgfpoint{60.034913\du}{14.202050\du}}
\pgfpathlineto{\pgfpoint{60.035585\du}{14.206085\du}}
\pgfpathlineto{\pgfpoint{60.036258\du}{14.210456\du}}
\pgfpathlineto{\pgfpoint{60.037266\du}{14.214827\du}}
\pgfpathlineto{\pgfpoint{60.038275\du}{14.218525\du}}
\pgfpathlineto{\pgfpoint{60.039620\du}{14.222224\du}}
\pgfpathlineto{\pgfpoint{60.041301\du}{14.226595\du}}
\pgfpathlineto{\pgfpoint{60.043318\du}{14.229957\du}}
\pgfpathlineto{\pgfpoint{60.044663\du}{14.233655\du}}
\pgfpathlineto{\pgfpoint{60.046681\du}{14.237354\du}}
\pgfpathlineto{\pgfpoint{60.049034\du}{14.240716\du}}
\pgfpathlineto{\pgfpoint{60.051724\du}{14.244078\du}}
\pgfpathlineto{\pgfpoint{60.054078\du}{14.247104\du}}
\pgfpathlineto{\pgfpoint{60.056767\du}{14.250466\du}}
\pgfpathlineto{\pgfpoint{60.059457\du}{14.253492\du}}
\pgfpathlineto{\pgfpoint{60.062819\du}{14.256182\du}}
\pgfpathlineto{\pgfpoint{60.065845\du}{14.258536\du}}
\pgfpathlineto{\pgfpoint{60.069208\du}{14.260889\du}}
\pgfpathlineto{\pgfpoint{60.072906\du}{14.263579\du}}
\pgfpathlineto{\pgfpoint{60.075932\du}{14.265596\du}}
\pgfpathlineto{\pgfpoint{60.079631\du}{14.267278\du}}
\pgfpathlineto{\pgfpoint{60.083329\du}{14.269295\du}}
\pgfpathlineto{\pgfpoint{60.087364\du}{14.270976\du}}
\pgfpathlineto{\pgfpoint{60.091062\du}{14.271985\du}}
\pgfpathlineto{\pgfpoint{60.094761\du}{14.273330\du}}
\pgfpathlineto{\pgfpoint{60.099132\du}{14.274338\du}}
\pgfpathlineto{\pgfpoint{60.102830\du}{14.275011\du}}
\pgfpathlineto{\pgfpoint{60.106865\du}{14.275347\du}}
\pgfpathlineto{\pgfpoint{60.110899\du}{14.275683\du}}
\pgfpathlineto{\pgfpoint{60.114934\du}{14.275683\du}}
\pgfpathlineto{\pgfpoint{60.114934\du}{14.275683\du}}
\pgfpathlineto{\pgfpoint{60.119305\du}{14.275683\du}}
\pgfpathlineto{\pgfpoint{60.123340\du}{14.275347\du}}
\pgfpathlineto{\pgfpoint{60.127374\du}{14.275011\du}}
\pgfpathlineto{\pgfpoint{60.131073\du}{14.274338\du}}
\pgfpathlineto{\pgfpoint{60.135107\du}{14.273330\du}}
\pgfpathlineto{\pgfpoint{60.138806\du}{14.271985\du}}
\pgfpathlineto{\pgfpoint{60.142841\du}{14.270976\du}}
\pgfpathlineto{\pgfpoint{60.146539\du}{14.269295\du}}
\pgfpathlineto{\pgfpoint{60.150574\du}{14.267278\du}}
\pgfpathlineto{\pgfpoint{60.153936\du}{14.265596\du}}
\pgfpathlineto{\pgfpoint{60.157298\du}{14.263579\du}}
\pgfpathlineto{\pgfpoint{60.160997\du}{14.260889\du}}
\pgfpathlineto{\pgfpoint{60.164023\du}{14.258536\du}}
\pgfpathlineto{\pgfpoint{60.167385\du}{14.256182\du}}
\pgfpathlineto{\pgfpoint{60.170411\du}{14.253492\du}}
\pgfpathlineto{\pgfpoint{60.173101\du}{14.250466\du}}
\pgfpathlineto{\pgfpoint{60.176127\du}{14.247104\du}}
\pgfpathlineto{\pgfpoint{60.178480\du}{14.244078\du}}
\pgfpathlineto{\pgfpoint{60.181170\du}{14.240716\du}}
\pgfpathlineto{\pgfpoint{60.183187\du}{14.237354\du}}
\pgfpathlineto{\pgfpoint{60.185205\du}{14.233655\du}}
\pgfpathlineto{\pgfpoint{60.186886\du}{14.229957\du}}
\pgfpathlineto{\pgfpoint{60.188903\du}{14.226595\du}}
\pgfpathlineto{\pgfpoint{60.190584\du}{14.222224\du}}
\pgfpathlineto{\pgfpoint{60.191593\du}{14.218525\du}}
\pgfpathlineto{\pgfpoint{60.192938\du}{14.214827\du}}
\pgfpathlineto{\pgfpoint{60.193610\du}{14.210456\du}}
\pgfpathlineto{\pgfpoint{60.194619\du}{14.206085\du}}
\pgfpathlineto{\pgfpoint{60.194955\du}{14.202050\du}}
\pgfpathlineto{\pgfpoint{60.195292\du}{14.198352\du}}
\pgfpathlineto{\pgfpoint{60.195292\du}{14.193645\du}}
\pgfusepath{fill}
\pgfsetbuttcap
\pgfsetmiterjoin
\pgfsetdash{}{0pt}
\definecolor{dialinecolor}{rgb}{1.000000, 1.000000, 1.000000}
\pgfsetfillcolor{dialinecolor}
\pgfpathmoveto{\pgfpoint{60.929604\du}{14.193645\du}}
\pgfpathlineto{\pgfpoint{60.929604\du}{14.189610\du}}
\pgfpathlineto{\pgfpoint{60.928931\du}{14.185239\du}}
\pgfpathlineto{\pgfpoint{60.928595\du}{14.181877\du}}
\pgfpathlineto{\pgfpoint{60.928259\du}{14.177506\du}}
\pgfpathlineto{\pgfpoint{60.926914\du}{14.173135\du}}
\pgfpathlineto{\pgfpoint{60.925905\du}{14.169437\du}}
\pgfpathlineto{\pgfpoint{60.924560\du}{14.165738\du}}
\pgfpathlineto{\pgfpoint{60.922879\du}{14.161367\du}}
\pgfpathlineto{\pgfpoint{60.921198\du}{14.158005\du}}
\pgfpathlineto{\pgfpoint{60.919853\du}{14.154306\du}}
\pgfpathlineto{\pgfpoint{60.917500\du}{14.150608\du}}
\pgfpathlineto{\pgfpoint{60.915482\du}{14.147246\du}}
\pgfpathlineto{\pgfpoint{60.912793\du}{14.143884\du}}
\pgfpathlineto{\pgfpoint{60.910103\du}{14.140858\du}}
\pgfpathlineto{\pgfpoint{60.907077\du}{14.137495\du}}
\pgfpathlineto{\pgfpoint{60.904387\du}{14.134469\du}}
\pgfpathlineto{\pgfpoint{60.901361\du}{14.131779\du}}
\pgfpathlineto{\pgfpoint{60.898671\du}{14.129426\du}}
\pgfpathlineto{\pgfpoint{60.894973\du}{14.126400\du}}
\pgfpathlineto{\pgfpoint{60.891610\du}{14.124383\du}}
\pgfpathlineto{\pgfpoint{60.888248\du}{14.122029\du}}
\pgfpathlineto{\pgfpoint{60.884550\du}{14.120348\du}}
\pgfpathlineto{\pgfpoint{60.880851\du}{14.118667\du}}
\pgfpathlineto{\pgfpoint{60.876817\du}{14.116986\du}}
\pgfpathlineto{\pgfpoint{60.872782\du}{14.115977\du}}
\pgfpathlineto{\pgfpoint{60.869083\du}{14.114632\du}}
\pgfpathlineto{\pgfpoint{60.865049\du}{14.113623\du}}
\pgfpathlineto{\pgfpoint{60.861014\du}{14.112951\du}}
\pgfpathlineto{\pgfpoint{60.856643\du}{14.112615\du}}
\pgfpathlineto{\pgfpoint{60.852945\du}{14.111942\du}}
\pgfpathlineto{\pgfpoint{60.848574\du}{14.111942\du}}
\pgfpathlineto{\pgfpoint{60.848574\du}{14.111942\du}}
\pgfpathlineto{\pgfpoint{60.844539\du}{14.111942\du}}
\pgfpathlineto{\pgfpoint{60.840841\du}{14.112615\du}}
\pgfpathlineto{\pgfpoint{60.836470\du}{14.112951\du}}
\pgfpathlineto{\pgfpoint{60.832435\du}{14.113623\du}}
\pgfpathlineto{\pgfpoint{60.828400\du}{14.114632\du}}
\pgfpathlineto{\pgfpoint{60.825038\du}{14.115977\du}}
\pgfpathlineto{\pgfpoint{60.821004\du}{14.116986\du}}
\pgfpathlineto{\pgfpoint{60.816633\du}{14.118667\du}}
\pgfpathlineto{\pgfpoint{60.812934\du}{14.120348\du}}
\pgfpathlineto{\pgfpoint{60.809572\du}{14.122029\du}}
\pgfpathlineto{\pgfpoint{60.806210\du}{14.124383\du}}
\pgfpathlineto{\pgfpoint{60.802511\du}{14.126400\du}}
\pgfpathlineto{\pgfpoint{60.799149\du}{14.129426\du}}
\pgfpathlineto{\pgfpoint{60.796123\du}{14.131779\du}}
\pgfpathlineto{\pgfpoint{60.793433\du}{14.134469\du}}
\pgfpathlineto{\pgfpoint{60.790071\du}{14.137495\du}}
\pgfpathlineto{\pgfpoint{60.787381\du}{14.140858\du}}
\pgfpathlineto{\pgfpoint{60.785028\du}{14.143884\du}}
\pgfpathlineto{\pgfpoint{60.782338\du}{14.147246\du}}
\pgfpathlineto{\pgfpoint{60.779984\du}{14.150608\du}}
\pgfpathlineto{\pgfpoint{60.777967\du}{14.154306\du}}
\pgfpathlineto{\pgfpoint{60.776622\du}{14.158005\du}}
\pgfpathlineto{\pgfpoint{60.774605\du}{14.161367\du}}
\pgfpathlineto{\pgfpoint{60.772924\du}{14.165738\du}}
\pgfpathlineto{\pgfpoint{60.771579\du}{14.169437\du}}
\pgfpathlineto{\pgfpoint{60.770570\du}{14.173135\du}}
\pgfpathlineto{\pgfpoint{60.769561\du}{14.177506\du}}
\pgfpathlineto{\pgfpoint{60.768889\du}{14.181877\du}}
\pgfpathlineto{\pgfpoint{60.768553\du}{14.185239\du}}
\pgfpathlineto{\pgfpoint{60.768216\du}{14.189610\du}}
\pgfpathlineto{\pgfpoint{60.768216\du}{14.193645\du}}
\pgfpathlineto{\pgfpoint{60.768216\du}{14.193645\du}}
\pgfpathlineto{\pgfpoint{60.768216\du}{14.198352\du}}
\pgfpathlineto{\pgfpoint{60.768553\du}{14.202050\du}}
\pgfpathlineto{\pgfpoint{60.768889\du}{14.206085\du}}
\pgfpathlineto{\pgfpoint{60.769561\du}{14.210456\du}}
\pgfpathlineto{\pgfpoint{60.770570\du}{14.214827\du}}
\pgfpathlineto{\pgfpoint{60.771579\du}{14.218525\du}}
\pgfpathlineto{\pgfpoint{60.772924\du}{14.222224\du}}
\pgfpathlineto{\pgfpoint{60.774605\du}{14.226595\du}}
\pgfpathlineto{\pgfpoint{60.776622\du}{14.229957\du}}
\pgfpathlineto{\pgfpoint{60.777967\du}{14.233655\du}}
\pgfpathlineto{\pgfpoint{60.779984\du}{14.237354\du}}
\pgfpathlineto{\pgfpoint{60.782338\du}{14.240716\du}}
\pgfpathlineto{\pgfpoint{60.785028\du}{14.244078\du}}
\pgfpathlineto{\pgfpoint{60.787381\du}{14.247104\du}}
\pgfpathlineto{\pgfpoint{60.790071\du}{14.250466\du}}
\pgfpathlineto{\pgfpoint{60.793433\du}{14.253492\du}}
\pgfpathlineto{\pgfpoint{60.796123\du}{14.256182\du}}
\pgfpathlineto{\pgfpoint{60.799149\du}{14.258536\du}}
\pgfpathlineto{\pgfpoint{60.802511\du}{14.260889\du}}
\pgfpathlineto{\pgfpoint{60.806210\du}{14.263579\du}}
\pgfpathlineto{\pgfpoint{60.809572\du}{14.265596\du}}
\pgfpathlineto{\pgfpoint{60.812934\du}{14.267278\du}}
\pgfpathlineto{\pgfpoint{60.816633\du}{14.269295\du}}
\pgfpathlineto{\pgfpoint{60.821004\du}{14.270976\du}}
\pgfpathlineto{\pgfpoint{60.825038\du}{14.271985\du}}
\pgfpathlineto{\pgfpoint{60.828400\du}{14.273330\du}}
\pgfpathlineto{\pgfpoint{60.832435\du}{14.274338\du}}
\pgfpathlineto{\pgfpoint{60.836470\du}{14.275011\du}}
\pgfpathlineto{\pgfpoint{60.840841\du}{14.275347\du}}
\pgfpathlineto{\pgfpoint{60.844539\du}{14.275683\du}}
\pgfpathlineto{\pgfpoint{60.848574\du}{14.275683\du}}
\pgfpathlineto{\pgfpoint{60.848574\du}{14.275683\du}}
\pgfpathlineto{\pgfpoint{60.852945\du}{14.275683\du}}
\pgfpathlineto{\pgfpoint{60.856643\du}{14.275347\du}}
\pgfpathlineto{\pgfpoint{60.861014\du}{14.275011\du}}
\pgfpathlineto{\pgfpoint{60.865049\du}{14.274338\du}}
\pgfpathlineto{\pgfpoint{60.869083\du}{14.273330\du}}
\pgfpathlineto{\pgfpoint{60.872782\du}{14.271985\du}}
\pgfpathlineto{\pgfpoint{60.876817\du}{14.270976\du}}
\pgfpathlineto{\pgfpoint{60.880851\du}{14.269295\du}}
\pgfpathlineto{\pgfpoint{60.884550\du}{14.267278\du}}
\pgfpathlineto{\pgfpoint{60.888248\du}{14.265596\du}}
\pgfpathlineto{\pgfpoint{60.891610\du}{14.263579\du}}
\pgfpathlineto{\pgfpoint{60.894973\du}{14.260889\du}}
\pgfpathlineto{\pgfpoint{60.898671\du}{14.258536\du}}
\pgfpathlineto{\pgfpoint{60.901361\du}{14.256182\du}}
\pgfpathlineto{\pgfpoint{60.904387\du}{14.253492\du}}
\pgfpathlineto{\pgfpoint{60.907077\du}{14.250466\du}}
\pgfpathlineto{\pgfpoint{60.910103\du}{14.247104\du}}
\pgfpathlineto{\pgfpoint{60.912793\du}{14.244078\du}}
\pgfpathlineto{\pgfpoint{60.915482\du}{14.240716\du}}
\pgfpathlineto{\pgfpoint{60.917500\du}{14.237354\du}}
\pgfpathlineto{\pgfpoint{60.919853\du}{14.233655\du}}
\pgfpathlineto{\pgfpoint{60.921198\du}{14.229957\du}}
\pgfpathlineto{\pgfpoint{60.922879\du}{14.226595\du}}
\pgfpathlineto{\pgfpoint{60.924560\du}{14.222224\du}}
\pgfpathlineto{\pgfpoint{60.925905\du}{14.218525\du}}
\pgfpathlineto{\pgfpoint{60.926914\du}{14.214827\du}}
\pgfpathlineto{\pgfpoint{60.928259\du}{14.210456\du}}
\pgfpathlineto{\pgfpoint{60.928595\du}{14.206085\du}}
\pgfpathlineto{\pgfpoint{60.928931\du}{14.202050\du}}
\pgfpathlineto{\pgfpoint{60.929604\du}{14.198352\du}}
\pgfpathlineto{\pgfpoint{60.929604\du}{14.193645\du}}
\pgfusepath{fill}
\pgfsetlinewidth{0.000000\du}
\pgfsetbuttcap
\pgfsetmiterjoin
\pgfsetdash{}{0pt}
\definecolor{dialinecolor}{rgb}{1.000000, 1.000000, 1.000000}
\pgfsetstrokecolor{dialinecolor}
\pgfpathmoveto{\pgfpoint{60.122667\du}{14.186920\du}}
\pgfpathlineto{\pgfpoint{60.827056\du}{14.186920\du}}
\pgfusepath{stroke}
\pgfsetlinewidth{0.000000\du}
\pgfsetbuttcap
\pgfsetmiterjoin
\pgfsetdash{}{0pt}
\definecolor{dialinecolor}{rgb}{0.788235, 0.788235, 0.713726}
\pgfsetfillcolor{dialinecolor}
\pgfpathmoveto{\pgfpoint{59.829144\du}{12.132258\du}}
\pgfpathlineto{\pgfpoint{60.034913\du}{11.938930\du}}
\pgfpathlineto{\pgfpoint{61.311554\du}{11.938930\du}}
\pgfpathlineto{\pgfpoint{61.106794\du}{12.132258\du}}
\pgfpathlineto{\pgfpoint{59.829144\du}{12.132258\du}}
\pgfusepath{fill}
\pgfsetbuttcap
\pgfsetmiterjoin
\pgfsetdash{}{0pt}
\definecolor{dialinecolor}{rgb}{0.286275, 0.286275, 0.211765}
\pgfsetstrokecolor{dialinecolor}
\pgfpathmoveto{\pgfpoint{59.829144\du}{12.132258\du}}
\pgfpathlineto{\pgfpoint{60.034913\du}{11.938930\du}}
\pgfpathlineto{\pgfpoint{61.311554\du}{11.938930\du}}
\pgfpathlineto{\pgfpoint{61.106794\du}{12.132258\du}}
\pgfpathlineto{\pgfpoint{59.829144\du}{12.132258\du}}
\pgfusepath{stroke}
\pgfsetbuttcap
\pgfsetmiterjoin
\pgfsetdash{}{0pt}
\definecolor{dialinecolor}{rgb}{0.478431, 0.478431, 0.352941}
\pgfsetfillcolor{dialinecolor}
\pgfpathmoveto{\pgfpoint{61.106794\du}{14.514066\du}}
\pgfpathlineto{\pgfpoint{61.311554\du}{14.305943\du}}
\pgfpathlineto{\pgfpoint{61.311554\du}{11.938930\du}}
\pgfpathlineto{\pgfpoint{61.106794\du}{12.132258\du}}
\pgfpathlineto{\pgfpoint{61.106794\du}{14.514066\du}}
\pgfusepath{fill}
\pgfsetbuttcap
\pgfsetmiterjoin
\pgfsetdash{}{0pt}
\definecolor{dialinecolor}{rgb}{0.286275, 0.286275, 0.211765}
\pgfsetstrokecolor{dialinecolor}
\pgfpathmoveto{\pgfpoint{61.106794\du}{14.514066\du}}
\pgfpathlineto{\pgfpoint{61.311554\du}{14.305943\du}}
\pgfpathlineto{\pgfpoint{61.311554\du}{11.938930\du}}
\pgfpathlineto{\pgfpoint{61.106794\du}{12.132258\du}}
\pgfpathlineto{\pgfpoint{61.106794\du}{14.514066\du}}
\pgfusepath{stroke}
% setfont left to latex
\definecolor{dialinecolor}{rgb}{0.000000, 0.000000, 0.000000}
\pgfsetstrokecolor{dialinecolor}
\node[anchor=west] at (58.835891\du,10.934776\du){192.5.5.241};
% setfont left to latex
\definecolor{dialinecolor}{rgb}{0.000000, 0.000000, 0.000000}
\pgfsetstrokecolor{dialinecolor}
\node[anchor=west] at (-17.696758\du,12.232672\du){89.105.202.4};
% setfont left to latex
\definecolor{dialinecolor}{rgb}{0.000000, 0.000000, 0.000000}
\pgfsetstrokecolor{dialinecolor}
\node[anchor=west] at (-11.207418\du,9.651301\du){Hop : 1};
% setfont left to latex
\definecolor{dialinecolor}{rgb}{0.000000, 0.000000, 0.000000}
\pgfsetstrokecolor{dialinecolor}
\node[anchor=west] at (-11.248839\du,10.804854\du){IP : 89.105.200.57};
% setfont left to latex
\definecolor{dialinecolor}{rgb}{0.000000, 0.000000, 0.000000}
\pgfsetstrokecolor{dialinecolor}
\node[anchor=west] at (-11.178129\du,11.887697\du){RTTs : 1.955, 1.7, 1.709};
% setfont left to latex
\definecolor{dialinecolor}{rgb}{0.000000, 0.000000, 0.000000}
\pgfsetstrokecolor{dialinecolor}
\node[anchor=west] at (-2.193814\du,9.622012\du){Hop : 2};
% setfont left to latex
\definecolor{dialinecolor}{rgb}{0.000000, 0.000000, 0.000000}
\pgfsetstrokecolor{dialinecolor}
\node[anchor=west] at (-2.197367\du,10.713433\du){IP : 185.147.12.31};
% setfont left to latex
\definecolor{dialinecolor}{rgb}{0.000000, 0.000000, 0.000000}
\pgfsetstrokecolor{dialinecolor}
\node[anchor=west] at (-2.180210\du,11.949829\du){RTTs: 4.347, 2.876, 3.143};
% setfont left to latex
\definecolor{dialinecolor}{rgb}{0.000000, 0.000000, 0.000000}
\pgfsetstrokecolor{dialinecolor}
\node[anchor=west] at (19.841064\du,12.096194\du){RTTs : 3.655, 3.678, 15.568};
\pgfsetlinewidth{0.000000\du}
\pgfsetdash{}{0pt}
\pgfsetdash{}{0pt}
\pgfsetbuttcap
\pgfsetmiterjoin
\pgfsetlinewidth{0.000000\du}
\pgfsetbuttcap
\pgfsetmiterjoin
\pgfsetdash{}{0pt}
\definecolor{dialinecolor}{rgb}{0.027451, 0.486275, 0.682353}
\pgfsetfillcolor{dialinecolor}
\pgfpathmoveto{\pgfpoint{23.991474\du}{14.406012\du}}
\pgfpathlineto{\pgfpoint{23.990013\du}{14.435219\du}}
\pgfpathlineto{\pgfpoint{23.982711\du}{14.465157\du}}
\pgfpathlineto{\pgfpoint{23.972489\du}{14.493635\du}}
\pgfpathlineto{\pgfpoint{23.957885\du}{14.521747\du}}
\pgfpathlineto{\pgfpoint{23.938535\du}{14.549860\du}}
\pgfpathlineto{\pgfpoint{23.916629\du}{14.577242\du}}
\pgfpathlineto{\pgfpoint{23.889977\du}{14.604259\du}}
\pgfpathlineto{\pgfpoint{23.859309\du}{14.630546\du}}
\pgfpathlineto{\pgfpoint{23.826450\du}{14.655738\du}}
\pgfpathlineto{\pgfpoint{23.789210\du}{14.680930\du}}
\pgfpathlineto{\pgfpoint{23.748319\du}{14.705026\du}}
\pgfpathlineto{\pgfpoint{23.704508\du}{14.728392\du}}
\pgfpathlineto{\pgfpoint{23.658140\du}{14.751028\du}}
\pgfpathlineto{\pgfpoint{23.607757\du}{14.772934\du}}
\pgfpathlineto{\pgfpoint{23.554818\du}{14.793744\du}}
\pgfpathlineto{\pgfpoint{23.499323\du}{14.813825\du}}
\pgfpathlineto{\pgfpoint{23.441273\du}{14.833175\du}}
\pgfpathlineto{\pgfpoint{23.380667\du}{14.851065\du}}
\pgfpathlineto{\pgfpoint{23.316775\du}{14.868224\du}}
\pgfpathlineto{\pgfpoint{23.251788\du}{14.884654\du}}
\pgfpathlineto{\pgfpoint{23.183149\du}{14.899623\du}}
\pgfpathlineto{\pgfpoint{23.112321\du}{14.913131\du}}
\pgfpathlineto{\pgfpoint{23.040397\du}{14.925909\du}}
\pgfpathlineto{\pgfpoint{22.965552\du}{14.937958\du}}
\pgfpathlineto{\pgfpoint{22.889612\du}{14.947815\du}}
\pgfpathlineto{\pgfpoint{22.811116\du}{14.956943\du}}
\pgfpathlineto{\pgfpoint{22.731525\du}{14.964610\du}}
\pgfpathlineto{\pgfpoint{22.650473\du}{14.971181\du}}
\pgfpathlineto{\pgfpoint{22.567231\du}{14.976293\du}}
\pgfpathlineto{\pgfpoint{22.483624\du}{14.979944\du}}
\pgfpathlineto{\pgfpoint{22.398191\du}{14.982134\du}}
\pgfpathlineto{\pgfpoint{22.311664\du}{14.982865\du}}
\pgfpathlineto{\pgfpoint{22.225501\du}{14.982134\du}}
\pgfpathlineto{\pgfpoint{22.139703\du}{14.979944\du}}
\pgfpathlineto{\pgfpoint{22.056096\du}{14.976293\du}}
\pgfpathlineto{\pgfpoint{21.973219\du}{14.971181\du}}
\pgfpathlineto{\pgfpoint{21.891802\du}{14.964610\du}}
\pgfpathlineto{\pgfpoint{21.812211\du}{14.956943\du}}
\pgfpathlineto{\pgfpoint{21.734446\du}{14.947815\du}}
\pgfpathlineto{\pgfpoint{21.657775\du}{14.937958\du}}
\pgfpathlineto{\pgfpoint{21.583661\du}{14.925909\du}}
\pgfpathlineto{\pgfpoint{21.511006\du}{14.913131\du}}
\pgfpathlineto{\pgfpoint{21.440543\du}{14.899623\du}}
\pgfpathlineto{\pgfpoint{21.371904\du}{14.884654\du}}
\pgfpathlineto{\pgfpoint{21.306187\du}{14.868224\du}}
\pgfpathlineto{\pgfpoint{21.242660\du}{14.851065\du}}
\pgfpathlineto{\pgfpoint{21.181689\du}{14.833175\du}}
\pgfpathlineto{\pgfpoint{21.123274\du}{14.813825\du}}
\pgfpathlineto{\pgfpoint{21.068144\du}{14.793744\du}}
\pgfpathlineto{\pgfpoint{21.015205\du}{14.772934\du}}
\pgfpathlineto{\pgfpoint{20.965187\du}{14.751028\du}}
\pgfpathlineto{\pgfpoint{20.918089\du}{14.728392\du}}
\pgfpathlineto{\pgfpoint{20.874643\du}{14.705026\du}}
\pgfpathlineto{\pgfpoint{20.833752\du}{14.680930\du}}
\pgfpathlineto{\pgfpoint{20.796512\du}{14.655738\du}}
\pgfpathlineto{\pgfpoint{20.763288\du}{14.630546\du}}
\pgfpathlineto{\pgfpoint{20.732985\du}{14.604259\du}}
\pgfpathlineto{\pgfpoint{20.706333\du}{14.577242\du}}
\pgfpathlineto{\pgfpoint{20.684427\du}{14.549860\du}}
\pgfpathlineto{\pgfpoint{20.665077\du}{14.521747\du}}
\pgfpathlineto{\pgfpoint{20.650473\du}{14.493635\du}}
\pgfpathlineto{\pgfpoint{20.639885\du}{14.465157\du}}
\pgfpathlineto{\pgfpoint{20.632949\du}{14.435219\du}}
\pgfpathlineto{\pgfpoint{20.631123\du}{14.406012\du}}
\pgfpathlineto{\pgfpoint{20.632949\du}{14.376074\du}}
\pgfpathlineto{\pgfpoint{20.639885\du}{14.346866\du}}
\pgfpathlineto{\pgfpoint{20.650473\du}{14.317658\du}}
\pgfpathlineto{\pgfpoint{20.665077\du}{14.289546\du}}
\pgfpathlineto{\pgfpoint{20.684427\du}{14.261433\du}}
\pgfpathlineto{\pgfpoint{20.706333\du}{14.234051\du}}
\pgfpathlineto{\pgfpoint{20.732985\du}{14.207399\du}}
\pgfpathlineto{\pgfpoint{20.763288\du}{14.181112\du}}
\pgfpathlineto{\pgfpoint{20.796512\du}{14.155555\du}}
\pgfpathlineto{\pgfpoint{20.833752\du}{14.130729\du}}
\pgfpathlineto{\pgfpoint{20.874643\du}{14.106267\du}}
\pgfpathlineto{\pgfpoint{20.918089\du}{14.082901\du}}
\pgfpathlineto{\pgfpoint{20.965187\du}{14.060630\du}}
\pgfpathlineto{\pgfpoint{21.015205\du}{14.038359\du}}
\pgfpathlineto{\pgfpoint{21.068144\du}{14.017549\du}}
\pgfpathlineto{\pgfpoint{21.123274\du}{13.997468\du}}
\pgfpathlineto{\pgfpoint{21.181689\du}{13.978849\du}}
\pgfpathlineto{\pgfpoint{21.242660\du}{13.959863\du}}
\pgfpathlineto{\pgfpoint{21.306187\du}{13.943069\du}}
\pgfpathlineto{\pgfpoint{21.371904\du}{13.927370\du}}
\pgfpathlineto{\pgfpoint{21.440543\du}{13.912036\du}}
\pgfpathlineto{\pgfpoint{21.511006\du}{13.898162\du}}
\pgfpathlineto{\pgfpoint{21.583661\du}{13.885019\du}}
\pgfpathlineto{\pgfpoint{21.657775\du}{13.874066\du}}
\pgfpathlineto{\pgfpoint{21.734446\du}{13.863478\du}}
\pgfpathlineto{\pgfpoint{21.812211\du}{13.853985\du}}
\pgfpathlineto{\pgfpoint{21.891802\du}{13.846683\du}}
\pgfpathlineto{\pgfpoint{21.973219\du}{13.840112\du}}
\pgfpathlineto{\pgfpoint{22.056096\du}{13.835000\du}}
\pgfpathlineto{\pgfpoint{22.139703\du}{13.831349\du}}
\pgfpathlineto{\pgfpoint{22.225501\du}{13.829524\du}}
\pgfpathlineto{\pgfpoint{22.311664\du}{13.828429\du}}
\pgfpathlineto{\pgfpoint{22.398191\du}{13.829524\du}}
\pgfpathlineto{\pgfpoint{22.483624\du}{13.831349\du}}
\pgfpathlineto{\pgfpoint{22.567231\du}{13.835000\du}}
\pgfpathlineto{\pgfpoint{22.650473\du}{13.840112\du}}
\pgfpathlineto{\pgfpoint{22.731525\du}{13.846683\du}}
\pgfpathlineto{\pgfpoint{22.811116\du}{13.853985\du}}
\pgfpathlineto{\pgfpoint{22.889612\du}{13.863478\du}}
\pgfpathlineto{\pgfpoint{22.965552\du}{13.874066\du}}
\pgfpathlineto{\pgfpoint{23.040397\du}{13.885019\du}}
\pgfpathlineto{\pgfpoint{23.112321\du}{13.898162\du}}
\pgfpathlineto{\pgfpoint{23.183149\du}{13.912036\du}}
\pgfpathlineto{\pgfpoint{23.251788\du}{13.927370\du}}
\pgfpathlineto{\pgfpoint{23.316775\du}{13.943069\du}}
\pgfpathlineto{\pgfpoint{23.380667\du}{13.959863\du}}
\pgfpathlineto{\pgfpoint{23.441273\du}{13.978849\du}}
\pgfpathlineto{\pgfpoint{23.499323\du}{13.997468\du}}
\pgfpathlineto{\pgfpoint{23.554818\du}{14.017549\du}}
\pgfpathlineto{\pgfpoint{23.607757\du}{14.038359\du}}
\pgfpathlineto{\pgfpoint{23.658140\du}{14.060630\du}}
\pgfpathlineto{\pgfpoint{23.704508\du}{14.082901\du}}
\pgfpathlineto{\pgfpoint{23.748319\du}{14.106267\du}}
\pgfpathlineto{\pgfpoint{23.789210\du}{14.130729\du}}
\pgfpathlineto{\pgfpoint{23.826450\du}{14.155555\du}}
\pgfpathlineto{\pgfpoint{23.859309\du}{14.181112\du}}
\pgfpathlineto{\pgfpoint{23.889977\du}{14.207399\du}}
\pgfpathlineto{\pgfpoint{23.916629\du}{14.234051\du}}
\pgfpathlineto{\pgfpoint{23.938535\du}{14.261433\du}}
\pgfpathlineto{\pgfpoint{23.957885\du}{14.289546\du}}
\pgfpathlineto{\pgfpoint{23.972489\du}{14.317658\du}}
\pgfpathlineto{\pgfpoint{23.982711\du}{14.346866\du}}
\pgfpathlineto{\pgfpoint{23.990013\du}{14.376074\du}}
\pgfpathlineto{\pgfpoint{23.991474\du}{14.406012\du}}
\pgfusepath{fill}
\pgfsetlinewidth{0.000000\du}
\pgfsetbuttcap
\pgfsetmiterjoin
\pgfsetdash{}{0pt}
\definecolor{dialinecolor}{rgb}{0.678431, 0.839216, 0.905882}
\pgfsetfillcolor{dialinecolor}
\pgfpathmoveto{\pgfpoint{22.311664\du}{14.993452\du}}
\pgfpathlineto{\pgfpoint{22.311664\du}{14.993452\du}}
\pgfpathlineto{\pgfpoint{22.355110\du}{14.993452\du}}
\pgfpathlineto{\pgfpoint{22.398557\du}{14.992722\du}}
\pgfpathlineto{\pgfpoint{22.441638\du}{14.991627\du}}
\pgfpathlineto{\pgfpoint{22.483624\du}{14.990532\du}}
\pgfpathlineto{\pgfpoint{22.526340\du}{14.988706\du}}
\pgfpathlineto{\pgfpoint{22.567961\du}{14.986516\du}}
\pgfpathlineto{\pgfpoint{22.609582\du}{14.983960\du}}
\pgfpathlineto{\pgfpoint{22.651203\du}{14.981769\du}}
\pgfpathlineto{\pgfpoint{22.691729\du}{14.978849\du}}
\pgfpathlineto{\pgfpoint{22.732620\du}{14.975198\du}}
\pgfpathlineto{\pgfpoint{22.772416\du}{14.971181\du}}
\pgfpathlineto{\pgfpoint{22.812576\du}{14.967165\du}}
\pgfpathlineto{\pgfpoint{22.851277\du}{14.962784\du}}
\pgfpathlineto{\pgfpoint{22.890707\du}{14.958403\du}}
\pgfpathlineto{\pgfpoint{22.928677\du}{14.952927\du}}
\pgfpathlineto{\pgfpoint{22.967742\du}{14.947815\du}}
\pgfpathlineto{\pgfpoint{23.004982\du}{14.942339\du}}
\pgfpathlineto{\pgfpoint{23.041857\du}{14.936132\du}}
\pgfpathlineto{\pgfpoint{23.078367\du}{14.930291\du}}
\pgfpathlineto{\pgfpoint{23.114876\du}{14.923719\du}}
\pgfpathlineto{\pgfpoint{23.150291\du}{14.916782\du}}
\pgfpathlineto{\pgfpoint{23.184975\du}{14.909845\du}}
\pgfpathlineto{\pgfpoint{23.219659\du}{14.902178\du}}
\pgfpathlineto{\pgfpoint{23.253613\du}{14.894511\du}}
\pgfpathlineto{\pgfpoint{23.287202\du}{14.886114\du}}
\pgfpathlineto{\pgfpoint{23.320061\du}{14.878082\du}}
\pgfpathlineto{\pgfpoint{23.351824\du}{14.870050\du}}
\pgfpathlineto{\pgfpoint{23.383222\du}{14.861287\du}}
\pgfpathlineto{\pgfpoint{23.398557\du}{14.856541\du}}
\pgfpathlineto{\pgfpoint{23.413891\du}{14.852525\du}}
\pgfpathlineto{\pgfpoint{23.429955\du}{14.847779\du}}
\pgfpathlineto{\pgfpoint{23.444559\du}{14.843033\du}}
\pgfpathlineto{\pgfpoint{23.458798\du}{14.838286\du}}
\pgfpathlineto{\pgfpoint{23.473766\du}{14.833175\du}}
\pgfpathlineto{\pgfpoint{23.488735\du}{14.828429\du}}
\pgfpathlineto{\pgfpoint{23.502609\du}{14.823682\du}}
\pgfpathlineto{\pgfpoint{23.516848\du}{14.818571\du}}
\pgfpathlineto{\pgfpoint{23.530722\du}{14.813825\du}}
\pgfpathlineto{\pgfpoint{23.544960\du}{14.808348\du}}
\pgfpathlineto{\pgfpoint{23.558104\du}{14.803967\du}}
\pgfpathlineto{\pgfpoint{23.572343\du}{14.798491\du}}
\pgfpathlineto{\pgfpoint{23.585486\du}{14.793379\du}}
\pgfpathlineto{\pgfpoint{23.598630\du}{14.787903\du}}
\pgfpathlineto{\pgfpoint{23.612138\du}{14.782061\du}}
\pgfpathlineto{\pgfpoint{23.624917\du}{14.776950\du}}
\pgfpathlineto{\pgfpoint{23.636965\du}{14.771474\du}}
\pgfpathlineto{\pgfpoint{23.649743\du}{14.765632\du}}
\pgfpathlineto{\pgfpoint{23.662156\du}{14.760521\du}}
\pgfpathlineto{\pgfpoint{23.674570\du}{14.754679\du}}
\pgfpathlineto{\pgfpoint{23.685888\du}{14.749203\du}}
\pgfpathlineto{\pgfpoint{23.697936\du}{14.743361\du}}
\pgfpathlineto{\pgfpoint{23.708889\du}{14.737520\du}}
\pgfpathlineto{\pgfpoint{23.720572\du}{14.731678\du}}
\pgfpathlineto{\pgfpoint{23.731890\du}{14.725836\du}}
\pgfpathlineto{\pgfpoint{23.743208\du}{14.719995\du}}
\pgfpathlineto{\pgfpoint{23.753796\du}{14.713788\du}}
\pgfpathlineto{\pgfpoint{23.763653\du}{14.707947\du}}
\pgfpathlineto{\pgfpoint{23.774241\du}{14.702105\du}}
\pgfpathlineto{\pgfpoint{23.784464\du}{14.695533\du}}
\pgfpathlineto{\pgfpoint{23.794321\du}{14.689692\du}}
\pgfpathlineto{\pgfpoint{23.804179\du}{14.683120\du}}
\pgfpathlineto{\pgfpoint{23.813306\du}{14.676913\du}}
\pgfpathlineto{\pgfpoint{23.822434\du}{14.670342\du}}
\pgfpathlineto{\pgfpoint{23.831926\du}{14.664500\du}}
\pgfpathlineto{\pgfpoint{23.840689\du}{14.657928\du}}
\pgfpathlineto{\pgfpoint{23.849816\du}{14.651722\du}}
\pgfpathlineto{\pgfpoint{23.857848\du}{14.645150\du}}
\pgfpathlineto{\pgfpoint{23.866611\du}{14.638213\du}}
\pgfpathlineto{\pgfpoint{23.873913\du}{14.631641\du}}
\pgfpathlineto{\pgfpoint{23.881945\du}{14.625435\du}}
\pgfpathlineto{\pgfpoint{23.889977\du}{14.618133\du}}
\pgfpathlineto{\pgfpoint{23.896549\du}{14.611926\du}}
\pgfpathlineto{\pgfpoint{23.903850\du}{14.604989\du}}
\pgfpathlineto{\pgfpoint{23.910422\du}{14.597688\du}}
\pgfpathlineto{\pgfpoint{23.917359\du}{14.591481\du}}
\pgfpathlineto{\pgfpoint{23.923931\du}{14.584544\du}}
\pgfpathlineto{\pgfpoint{23.930137\du}{14.577242\du}}
\pgfpathlineto{\pgfpoint{23.935979\du}{14.570305\du}}
\pgfpathlineto{\pgfpoint{23.941455\du}{14.563368\du}}
\pgfpathlineto{\pgfpoint{23.947297\du}{14.556432\du}}
\pgfpathlineto{\pgfpoint{23.952043\du}{14.549130\du}}
\pgfpathlineto{\pgfpoint{23.957520\du}{14.541828\du}}
\pgfpathlineto{\pgfpoint{23.962266\du}{14.534526\du}}
\pgfpathlineto{\pgfpoint{23.966647\du}{14.527589\du}}
\pgfpathlineto{\pgfpoint{23.970663\du}{14.519922\du}}
\pgfpathlineto{\pgfpoint{23.974679\du}{14.512985\du}}
\pgfpathlineto{\pgfpoint{23.977965\du}{14.505318\du}}
\pgfpathlineto{\pgfpoint{23.981981\du}{14.497651\du}}
\pgfpathlineto{\pgfpoint{23.985267\du}{14.489984\du}}
\pgfpathlineto{\pgfpoint{23.987823\du}{14.482682\du}}
\pgfpathlineto{\pgfpoint{23.990743\du}{14.475380\du}}
\pgfpathlineto{\pgfpoint{23.992569\du}{14.468078\du}}
\pgfpathlineto{\pgfpoint{23.995490\du}{14.460411\du}}
\pgfpathlineto{\pgfpoint{23.996585\du}{14.452014\du}}
\pgfpathlineto{\pgfpoint{23.998776\du}{14.444347\du}}
\pgfpathlineto{\pgfpoint{23.999871\du}{14.437045\du}}
\pgfpathlineto{\pgfpoint{24.000601\du}{14.429378\du}}
\pgfpathlineto{\pgfpoint{24.001331\du}{14.420981\du}}
\pgfpathlineto{\pgfpoint{24.002061\du}{14.413679\du}}
\pgfpathlineto{\pgfpoint{24.002061\du}{14.406012\du}}
\pgfpathlineto{\pgfpoint{23.981981\du}{14.406012\du}}
\pgfpathlineto{\pgfpoint{23.981251\du}{14.412949\du}}
\pgfpathlineto{\pgfpoint{23.981251\du}{14.419885\du}}
\pgfpathlineto{\pgfpoint{23.980886\du}{14.426822\du}}
\pgfpathlineto{\pgfpoint{23.979060\du}{14.434124\du}}
\pgfpathlineto{\pgfpoint{23.977965\du}{14.441061\du}}
\pgfpathlineto{\pgfpoint{23.977235\du}{14.447998\du}}
\pgfpathlineto{\pgfpoint{23.975044\du}{14.454935\du}}
\pgfpathlineto{\pgfpoint{23.973584\du}{14.462237\du}}
\pgfpathlineto{\pgfpoint{23.971393\du}{14.468443\du}}
\pgfpathlineto{\pgfpoint{23.968838\du}{14.475380\du}}
\pgfpathlineto{\pgfpoint{23.965917\du}{14.482682\du}}
\pgfpathlineto{\pgfpoint{23.963361\du}{14.489619\du}}
\pgfpathlineto{\pgfpoint{23.959345\du}{14.496556\du}}
\pgfpathlineto{\pgfpoint{23.956424\du}{14.503127\du}}
\pgfpathlineto{\pgfpoint{23.952408\du}{14.510064\du}}
\pgfpathlineto{\pgfpoint{23.949488\du}{14.516636\du}}
\pgfpathlineto{\pgfpoint{23.945106\du}{14.523573\du}}
\pgfpathlineto{\pgfpoint{23.940725\du}{14.530510\du}}
\pgfpathlineto{\pgfpoint{23.935979\du}{14.537081\du}}
\pgfpathlineto{\pgfpoint{23.930868\du}{14.543288\du}}
\pgfpathlineto{\pgfpoint{23.926121\du}{14.550590\du}}
\pgfpathlineto{\pgfpoint{23.919915\du}{14.557527\du}}
\pgfpathlineto{\pgfpoint{23.914438\du}{14.563734\du}}
\pgfpathlineto{\pgfpoint{23.909327\du}{14.570305\du}}
\pgfpathlineto{\pgfpoint{23.902755\du}{14.576877\du}}
\pgfpathlineto{\pgfpoint{23.895818\du}{14.583814\du}}
\pgfpathlineto{\pgfpoint{23.889977\du}{14.590386\du}}
\pgfpathlineto{\pgfpoint{23.882675\du}{14.596592\du}}
\pgfpathlineto{\pgfpoint{23.876103\du}{14.603164\du}}
\pgfpathlineto{\pgfpoint{23.868436\du}{14.609371\du}}
\pgfpathlineto{\pgfpoint{23.861134\du}{14.615942\du}}
\pgfpathlineto{\pgfpoint{23.853102\du}{14.622514\du}}
\pgfpathlineto{\pgfpoint{23.845435\du}{14.628721\du}}
\pgfpathlineto{\pgfpoint{23.836673\du}{14.635292\du}}
\pgfpathlineto{\pgfpoint{23.828275\du}{14.641134\du}}
\pgfpathlineto{\pgfpoint{23.819513\du}{14.647706\du}}
\pgfpathlineto{\pgfpoint{23.811481\du}{14.653912\du}}
\pgfpathlineto{\pgfpoint{23.802719\du}{14.659754\du}}
\pgfpathlineto{\pgfpoint{23.793226\du}{14.666326\du}}
\pgfpathlineto{\pgfpoint{23.783003\du}{14.672167\du}}
\pgfpathlineto{\pgfpoint{23.773876\du}{14.678739\du}}
\pgfpathlineto{\pgfpoint{23.763653\du}{14.684581\du}}
\pgfpathlineto{\pgfpoint{23.753796\du}{14.690422\du}}
\pgfpathlineto{\pgfpoint{23.743938\du}{14.696264\du}}
\pgfpathlineto{\pgfpoint{23.732985\du}{14.702105\du}}
\pgfpathlineto{\pgfpoint{23.722032\du}{14.707947\du}}
\pgfpathlineto{\pgfpoint{23.711810\du}{14.713788\du}}
\pgfpathlineto{\pgfpoint{23.699761\du}{14.719630\du}}
\pgfpathlineto{\pgfpoint{23.689174\du}{14.724741\du}}
\pgfpathlineto{\pgfpoint{23.677125\du}{14.730583\du}}
\pgfpathlineto{\pgfpoint{23.665807\du}{14.736424\du}}
\pgfpathlineto{\pgfpoint{23.653394\du}{14.741901\du}}
\pgfpathlineto{\pgfpoint{23.641346\du}{14.747012\du}}
\pgfpathlineto{\pgfpoint{23.628933\du}{14.752854\du}}
\pgfpathlineto{\pgfpoint{23.616884\du}{14.758330\du}}
\pgfpathlineto{\pgfpoint{23.603741\du}{14.763441\du}}
\pgfpathlineto{\pgfpoint{23.590963\du}{14.768553\du}}
\pgfpathlineto{\pgfpoint{23.578184\du}{14.774029\du}}
\pgfpathlineto{\pgfpoint{23.565041\du}{14.779141\du}}
\pgfpathlineto{\pgfpoint{23.551532\du}{14.784617\du}}
\pgfpathlineto{\pgfpoint{23.538389\du}{14.788998\du}}
\pgfpathlineto{\pgfpoint{23.524150\du}{14.794475\du}}
\pgfpathlineto{\pgfpoint{23.510276\du}{14.799221\du}}
\pgfpathlineto{\pgfpoint{23.496037\du}{14.804332\du}}
\pgfpathlineto{\pgfpoint{23.481434\du}{14.809079\du}}
\pgfpathlineto{\pgfpoint{23.467560\du}{14.813825\du}}
\pgfpathlineto{\pgfpoint{23.452956\du}{14.818571\du}}
\pgfpathlineto{\pgfpoint{23.438717\du}{14.822952\du}}
\pgfpathlineto{\pgfpoint{23.423018\du}{14.827698\du}}
\pgfpathlineto{\pgfpoint{23.408779\du}{14.832445\du}}
\pgfpathlineto{\pgfpoint{23.393080\du}{14.837191\du}}
\pgfpathlineto{\pgfpoint{23.378111\du}{14.841207\du}}
\pgfpathlineto{\pgfpoint{23.346713\du}{14.849969\du}}
\pgfpathlineto{\pgfpoint{23.314584\du}{14.858367\du}}
\pgfpathlineto{\pgfpoint{23.282456\du}{14.866399\du}}
\pgfpathlineto{\pgfpoint{23.249232\du}{14.874431\du}}
\pgfpathlineto{\pgfpoint{23.214913\du}{14.882098\du}}
\pgfpathlineto{\pgfpoint{23.180959\du}{14.889400\du}}
\pgfpathlineto{\pgfpoint{23.146275\du}{14.896337\du}}
\pgfpathlineto{\pgfpoint{23.110495\du}{14.903273\du}}
\pgfpathlineto{\pgfpoint{23.075081\du}{14.909845\du}}
\pgfpathlineto{\pgfpoint{23.038206\du}{14.916052\du}}
\pgfpathlineto{\pgfpoint{23.001696\du}{14.921893\du}}
\pgfpathlineto{\pgfpoint{22.964457\du}{14.927735\du}}
\pgfpathlineto{\pgfpoint{22.926852\du}{14.933211\du}}
\pgfpathlineto{\pgfpoint{22.888151\du}{14.937958\du}}
\pgfpathlineto{\pgfpoint{22.849451\du}{14.942339\du}}
\pgfpathlineto{\pgfpoint{22.810021\du}{14.947085\du}}
\pgfpathlineto{\pgfpoint{22.770955\du}{14.950736\du}}
\pgfpathlineto{\pgfpoint{22.730795\du}{14.954752\du}}
\pgfpathlineto{\pgfpoint{22.690634\du}{14.957673\du}}
\pgfpathlineto{\pgfpoint{22.649378\du}{14.961324\du}}
\pgfpathlineto{\pgfpoint{22.608487\du}{14.964245\du}}
\pgfpathlineto{\pgfpoint{22.567231\du}{14.966435\du}}
\pgfpathlineto{\pgfpoint{22.525610\du}{14.968261\du}}
\pgfpathlineto{\pgfpoint{22.482894\du}{14.970086\du}}
\pgfpathlineto{\pgfpoint{22.440178\du}{14.971181\du}}
\pgfpathlineto{\pgfpoint{22.398191\du}{14.972277\du}}
\pgfpathlineto{\pgfpoint{22.354745\du}{14.972277\du}}
\pgfpathlineto{\pgfpoint{22.311664\du}{14.973007\du}}
\pgfpathlineto{\pgfpoint{22.311664\du}{14.973007\du}}
\pgfpathlineto{\pgfpoint{22.311664\du}{14.973007\du}}
\pgfpathlineto{\pgfpoint{22.310933\du}{14.973007\du}}
\pgfpathlineto{\pgfpoint{22.309108\du}{14.973007\du}}
\pgfpathlineto{\pgfpoint{22.308013\du}{14.973372\du}}
\pgfpathlineto{\pgfpoint{22.307282\du}{14.973372\du}}
\pgfpathlineto{\pgfpoint{22.306917\du}{14.974102\du}}
\pgfpathlineto{\pgfpoint{22.305457\du}{14.974467\du}}
\pgfpathlineto{\pgfpoint{22.304727\du}{14.975198\du}}
\pgfpathlineto{\pgfpoint{22.303996\du}{14.975928\du}}
\pgfpathlineto{\pgfpoint{22.302901\du}{14.977753\du}}
\pgfpathlineto{\pgfpoint{22.302171\du}{14.979214\du}}
\pgfpathlineto{\pgfpoint{22.302171\du}{14.981039\du}}
\pgfpathlineto{\pgfpoint{22.301441\du}{14.982865\du}}
\pgfpathlineto{\pgfpoint{22.302171\du}{14.985055\du}}
\pgfpathlineto{\pgfpoint{22.302171\du}{14.986881\du}}
\pgfpathlineto{\pgfpoint{22.302901\du}{14.988706\du}}
\pgfpathlineto{\pgfpoint{22.303996\du}{14.990532\du}}
\pgfpathlineto{\pgfpoint{22.304727\du}{14.990897\du}}
\pgfpathlineto{\pgfpoint{22.305457\du}{14.991627\du}}
\pgfpathlineto{\pgfpoint{22.306917\du}{14.992357\du}}
\pgfpathlineto{\pgfpoint{22.307282\du}{14.992722\du}}
\pgfpathlineto{\pgfpoint{22.308013\du}{14.992722\du}}
\pgfpathlineto{\pgfpoint{22.309108\du}{14.993452\du}}
\pgfpathlineto{\pgfpoint{22.310933\du}{14.993452\du}}
\pgfpathlineto{\pgfpoint{22.311664\du}{14.993452\du}}
\pgfusepath{fill}
\pgfsetbuttcap
\pgfsetmiterjoin
\pgfsetdash{}{0pt}
\definecolor{dialinecolor}{rgb}{0.678431, 0.839216, 0.905882}
\pgfsetfillcolor{dialinecolor}
\pgfpathmoveto{\pgfpoint{20.620900\du}{14.406012\du}}
\pgfpathlineto{\pgfpoint{20.620900\du}{14.406012\du}}
\pgfpathlineto{\pgfpoint{20.620900\du}{14.413679\du}}
\pgfpathlineto{\pgfpoint{20.621266\du}{14.420981\du}}
\pgfpathlineto{\pgfpoint{20.621996\du}{14.429378\du}}
\pgfpathlineto{\pgfpoint{20.623091\du}{14.437045\du}}
\pgfpathlineto{\pgfpoint{20.624186\du}{14.444347\du}}
\pgfpathlineto{\pgfpoint{20.626012\du}{14.452014\du}}
\pgfpathlineto{\pgfpoint{20.627837\du}{14.460411\du}}
\pgfpathlineto{\pgfpoint{20.630028\du}{14.468078\du}}
\pgfpathlineto{\pgfpoint{20.632218\du}{14.475380\du}}
\pgfpathlineto{\pgfpoint{20.634774\du}{14.482682\du}}
\pgfpathlineto{\pgfpoint{20.637695\du}{14.489984\du}}
\pgfpathlineto{\pgfpoint{20.641346\du}{14.497651\du}}
\pgfpathlineto{\pgfpoint{20.644632\du}{14.505318\du}}
\pgfpathlineto{\pgfpoint{20.648283\du}{14.512985\du}}
\pgfpathlineto{\pgfpoint{20.652664\du}{14.519922\du}}
\pgfpathlineto{\pgfpoint{20.656315\du}{14.527589\du}}
\pgfpathlineto{\pgfpoint{20.661426\du}{14.534526\du}}
\pgfpathlineto{\pgfpoint{20.665442\du}{14.541828\du}}
\pgfpathlineto{\pgfpoint{20.670919\du}{14.549130\du}}
\pgfpathlineto{\pgfpoint{20.675665\du}{14.556432\du}}
\pgfpathlineto{\pgfpoint{20.681141\du}{14.563368\du}}
\pgfpathlineto{\pgfpoint{20.686983\du}{14.570305\du}}
\pgfpathlineto{\pgfpoint{20.692825\du}{14.577242\du}}
\pgfpathlineto{\pgfpoint{20.698666\du}{14.584544\du}}
\pgfpathlineto{\pgfpoint{20.705603\du}{14.591481\du}}
\pgfpathlineto{\pgfpoint{20.712175\du}{14.597688\du}}
\pgfpathlineto{\pgfpoint{20.719111\du}{14.604989\du}}
\pgfpathlineto{\pgfpoint{20.726048\du}{14.611926\du}}
\pgfpathlineto{\pgfpoint{20.732985\du}{14.618133\du}}
\pgfpathlineto{\pgfpoint{20.741747\du}{14.625435\du}}
\pgfpathlineto{\pgfpoint{20.748684\du}{14.631641\du}}
\pgfpathlineto{\pgfpoint{20.756351\du}{14.638213\du}}
\pgfpathlineto{\pgfpoint{20.765114\du}{14.645150\du}}
\pgfpathlineto{\pgfpoint{20.773511\du}{14.651722\du}}
\pgfpathlineto{\pgfpoint{20.782273\du}{14.657928\du}}
\pgfpathlineto{\pgfpoint{20.790670\du}{14.664500\du}}
\pgfpathlineto{\pgfpoint{20.800528\du}{14.670342\du}}
\pgfpathlineto{\pgfpoint{20.809655\du}{14.676913\du}}
\pgfpathlineto{\pgfpoint{20.819148\du}{14.683120\du}}
\pgfpathlineto{\pgfpoint{20.828641\du}{14.689692\du}}
\pgfpathlineto{\pgfpoint{20.838498\du}{14.695533\du}}
\pgfpathlineto{\pgfpoint{20.848721\du}{14.702105\du}}
\pgfpathlineto{\pgfpoint{20.858944\du}{14.707947\du}}
\pgfpathlineto{\pgfpoint{20.869531\du}{14.713788\du}}
\pgfpathlineto{\pgfpoint{20.879754\du}{14.719995\du}}
\pgfpathlineto{\pgfpoint{20.890707\du}{14.725836\du}}
\pgfpathlineto{\pgfpoint{20.902390\du}{14.731678\du}}
\pgfpathlineto{\pgfpoint{20.913708\du}{14.737520\du}}
\pgfpathlineto{\pgfpoint{20.925391\du}{14.743361\du}}
\pgfpathlineto{\pgfpoint{20.936709\du}{14.749203\du}}
\pgfpathlineto{\pgfpoint{20.948392\du}{14.754679\du}}
\pgfpathlineto{\pgfpoint{20.960806\du}{14.760521\du}}
\pgfpathlineto{\pgfpoint{20.972854\du}{14.765632\du}}
\pgfpathlineto{\pgfpoint{20.985997\du}{14.771474\du}}
\pgfpathlineto{\pgfpoint{20.998045\du}{14.776950\du}}
\pgfpathlineto{\pgfpoint{21.010824\du}{14.782061\du}}
\pgfpathlineto{\pgfpoint{21.024697\du}{14.787903\du}}
\pgfpathlineto{\pgfpoint{21.037111\du}{14.793379\du}}
\pgfpathlineto{\pgfpoint{21.050254\du}{14.798491\du}}
\pgfpathlineto{\pgfpoint{21.064493\du}{14.803967\du}}
\pgfpathlineto{\pgfpoint{21.077637\du}{14.808348\du}}
\pgfpathlineto{\pgfpoint{21.091875\du}{14.813825\du}}
\pgfpathlineto{\pgfpoint{21.105749\du}{14.818571\du}}
\pgfpathlineto{\pgfpoint{21.120353\du}{14.823682\du}}
\pgfpathlineto{\pgfpoint{21.134227\du}{14.828429\du}}
\pgfpathlineto{\pgfpoint{21.149926\du}{14.833175\du}}
\pgfpathlineto{\pgfpoint{21.163799\du}{14.838286\du}}
\pgfpathlineto{\pgfpoint{21.178403\du}{14.843033\du}}
\pgfpathlineto{\pgfpoint{21.193737\du}{14.847779\du}}
\pgfpathlineto{\pgfpoint{21.209436\du}{14.852525\du}}
\pgfpathlineto{\pgfpoint{21.224405\du}{14.856541\du}}
\pgfpathlineto{\pgfpoint{21.240105\du}{14.861287\du}}
\pgfpathlineto{\pgfpoint{21.271868\du}{14.870050\du}}
\pgfpathlineto{\pgfpoint{21.303631\du}{14.878082\du}}
\pgfpathlineto{\pgfpoint{21.336855\du}{14.886114\du}}
\pgfpathlineto{\pgfpoint{21.369349\du}{14.894511\du}}
\pgfpathlineto{\pgfpoint{21.404033\du}{14.902178\du}}
\pgfpathlineto{\pgfpoint{21.438352\du}{14.909845\du}}
\pgfpathlineto{\pgfpoint{21.473036\du}{14.916782\du}}
\pgfpathlineto{\pgfpoint{21.508816\du}{14.923719\du}}
\pgfpathlineto{\pgfpoint{21.544960\du}{14.930291\du}}
\pgfpathlineto{\pgfpoint{21.581470\du}{14.936132\du}}
\pgfpathlineto{\pgfpoint{21.618710\du}{14.942339\du}}
\pgfpathlineto{\pgfpoint{21.656315\du}{14.947815\du}}
\pgfpathlineto{\pgfpoint{21.694285\du}{14.952927\du}}
\pgfpathlineto{\pgfpoint{21.732985\du}{14.958403\du}}
\pgfpathlineto{\pgfpoint{21.771685\du}{14.962784\du}}
\pgfpathlineto{\pgfpoint{21.810751\du}{14.967165\du}}
\pgfpathlineto{\pgfpoint{21.850911\du}{14.971181\du}}
\pgfpathlineto{\pgfpoint{21.890707\du}{14.975198\du}}
\pgfpathlineto{\pgfpoint{21.931598\du}{14.978849\du}}
\pgfpathlineto{\pgfpoint{21.972489\du}{14.981769\du}}
\pgfpathlineto{\pgfpoint{22.013745\du}{14.983960\du}}
\pgfpathlineto{\pgfpoint{22.055731\du}{14.986516\du}}
\pgfpathlineto{\pgfpoint{22.097352\du}{14.988706\du}}
\pgfpathlineto{\pgfpoint{22.139703\du}{14.990532\du}}
\pgfpathlineto{\pgfpoint{22.181689\du}{14.991627\du}}
\pgfpathlineto{\pgfpoint{22.225136\du}{14.992722\du}}
\pgfpathlineto{\pgfpoint{22.267852\du}{14.993452\du}}
\pgfpathlineto{\pgfpoint{22.311664\du}{14.993452\du}}
\pgfpathlineto{\pgfpoint{22.311664\du}{14.973007\du}}
\pgfpathlineto{\pgfpoint{22.268947\du}{14.972277\du}}
\pgfpathlineto{\pgfpoint{22.225501\du}{14.972277\du}}
\pgfpathlineto{\pgfpoint{22.183149\du}{14.971181\du}}
\pgfpathlineto{\pgfpoint{22.140433\du}{14.970086\du}}
\pgfpathlineto{\pgfpoint{22.098082\du}{14.968261\du}}
\pgfpathlineto{\pgfpoint{22.056096\du}{14.966435\du}}
\pgfpathlineto{\pgfpoint{22.015205\du}{14.964245\du}}
\pgfpathlineto{\pgfpoint{21.973949\du}{14.961324\du}}
\pgfpathlineto{\pgfpoint{21.933058\du}{14.957673\du}}
\pgfpathlineto{\pgfpoint{21.892898\du}{14.954752\du}}
\pgfpathlineto{\pgfpoint{21.853102\du}{14.950736\du}}
\pgfpathlineto{\pgfpoint{21.813672\du}{14.947085\du}}
\pgfpathlineto{\pgfpoint{21.774241\du}{14.942339\du}}
\pgfpathlineto{\pgfpoint{21.735176\du}{14.937958\du}}
\pgfpathlineto{\pgfpoint{21.696841\du}{14.933211\du}}
\pgfpathlineto{\pgfpoint{21.659236\du}{14.927735\du}}
\pgfpathlineto{\pgfpoint{21.621996\du}{14.921893\du}}
\pgfpathlineto{\pgfpoint{21.585486\du}{14.916052\du}}
\pgfpathlineto{\pgfpoint{21.548246\du}{14.909845\du}}
\pgfpathlineto{\pgfpoint{21.513197\du}{14.903273\du}}
\pgfpathlineto{\pgfpoint{21.477052\du}{14.896337\du}}
\pgfpathlineto{\pgfpoint{21.442368\du}{14.889400\du}}
\pgfpathlineto{\pgfpoint{21.408049\du}{14.882098\du}}
\pgfpathlineto{\pgfpoint{21.374095\du}{14.874431\du}}
\pgfpathlineto{\pgfpoint{21.340871\du}{14.866399\du}}
\pgfpathlineto{\pgfpoint{21.309473\du}{14.858367\du}}
\pgfpathlineto{\pgfpoint{21.276979\du}{14.849969\du}}
\pgfpathlineto{\pgfpoint{21.245946\du}{14.841207\du}}
\pgfpathlineto{\pgfpoint{21.230247\du}{14.837191\du}}
\pgfpathlineto{\pgfpoint{21.214548\du}{14.832445\du}}
\pgfpathlineto{\pgfpoint{21.199944\du}{14.827698\du}}
\pgfpathlineto{\pgfpoint{21.184975\du}{14.822952\du}}
\pgfpathlineto{\pgfpoint{21.169641\du}{14.818571\du}}
\pgfpathlineto{\pgfpoint{21.155402\du}{14.813825\du}}
\pgfpathlineto{\pgfpoint{21.141163\du}{14.809079\du}}
\pgfpathlineto{\pgfpoint{21.126925\du}{14.804332\du}}
\pgfpathlineto{\pgfpoint{21.112686\du}{14.799221\du}}
\pgfpathlineto{\pgfpoint{21.098812\du}{14.794475\du}}
\pgfpathlineto{\pgfpoint{21.084938\du}{14.788998\du}}
\pgfpathlineto{\pgfpoint{21.071430\du}{14.784617\du}}
\pgfpathlineto{\pgfpoint{21.057921\du}{14.779141\du}}
\pgfpathlineto{\pgfpoint{21.045143\du}{14.774029\du}}
\pgfpathlineto{\pgfpoint{21.031634\du}{14.768553\du}}
\pgfpathlineto{\pgfpoint{21.018856\du}{14.763441\du}}
\pgfpathlineto{\pgfpoint{21.006443\du}{14.758330\du}}
\pgfpathlineto{\pgfpoint{20.993299\du}{14.752854\du}}
\pgfpathlineto{\pgfpoint{20.981251\du}{14.747012\du}}
\pgfpathlineto{\pgfpoint{20.969933\du}{14.741901\du}}
\pgfpathlineto{\pgfpoint{20.957155\du}{14.736424\du}}
\pgfpathlineto{\pgfpoint{20.945471\du}{14.730583\du}}
\pgfpathlineto{\pgfpoint{20.933788\du}{14.724741\du}}
\pgfpathlineto{\pgfpoint{20.922835\du}{14.719630\du}}
\pgfpathlineto{\pgfpoint{20.911152\du}{14.713788\du}}
\pgfpathlineto{\pgfpoint{20.901295\du}{14.707947\du}}
\pgfpathlineto{\pgfpoint{20.889977\du}{14.702105\du}}
\pgfpathlineto{\pgfpoint{20.879389\du}{14.696264\du}}
\pgfpathlineto{\pgfpoint{20.869531\du}{14.690422\du}}
\pgfpathlineto{\pgfpoint{20.858944\du}{14.684581\du}}
\pgfpathlineto{\pgfpoint{20.849086\du}{14.678739\du}}
\pgfpathlineto{\pgfpoint{20.839593\du}{14.672167\du}}
\pgfpathlineto{\pgfpoint{20.829736\du}{14.666326\du}}
\pgfpathlineto{\pgfpoint{20.820608\du}{14.659754\du}}
\pgfpathlineto{\pgfpoint{20.812211\du}{14.653912\du}}
\pgfpathlineto{\pgfpoint{20.803449\du}{14.647706\du}}
\pgfpathlineto{\pgfpoint{20.794321\du}{14.641134\du}}
\pgfpathlineto{\pgfpoint{20.785559\du}{14.635292\du}}
\pgfpathlineto{\pgfpoint{20.777162\du}{14.628721\du}}
\pgfpathlineto{\pgfpoint{20.769860\du}{14.622514\du}}
\pgfpathlineto{\pgfpoint{20.761828\du}{14.615942\du}}
\pgfpathlineto{\pgfpoint{20.754161\du}{14.609371\du}}
\pgfpathlineto{\pgfpoint{20.747224\du}{14.603164\du}}
\pgfpathlineto{\pgfpoint{20.739922\du}{14.596592\du}}
\pgfpathlineto{\pgfpoint{20.732985\du}{14.590386\du}}
\pgfpathlineto{\pgfpoint{20.726779\du}{14.583814\du}}
\pgfpathlineto{\pgfpoint{20.720207\du}{14.576877\du}}
\pgfpathlineto{\pgfpoint{20.714365\du}{14.570305\du}}
\pgfpathlineto{\pgfpoint{20.708159\du}{14.563734\du}}
\pgfpathlineto{\pgfpoint{20.703047\du}{14.557527\du}}
\pgfpathlineto{\pgfpoint{20.696841\du}{14.550590\du}}
\pgfpathlineto{\pgfpoint{20.692459\du}{14.544018\du}}
\pgfpathlineto{\pgfpoint{20.686983\du}{14.537081\du}}
\pgfpathlineto{\pgfpoint{20.682602\du}{14.530510\du}}
\pgfpathlineto{\pgfpoint{20.678221\du}{14.523573\du}}
\pgfpathlineto{\pgfpoint{20.673839\du}{14.516636\du}}
\pgfpathlineto{\pgfpoint{20.670554\du}{14.510064\du}}
\pgfpathlineto{\pgfpoint{20.666538\du}{14.503127\du}}
\pgfpathlineto{\pgfpoint{20.662521\du}{14.496556\du}}
\pgfpathlineto{\pgfpoint{20.659601\du}{14.489619\du}}
\pgfpathlineto{\pgfpoint{20.657045\du}{14.482682\du}}
\pgfpathlineto{\pgfpoint{20.653759\du}{14.475380\du}}
\pgfpathlineto{\pgfpoint{20.651569\du}{14.468443\du}}
\pgfpathlineto{\pgfpoint{20.649378\du}{14.462237\du}}
\pgfpathlineto{\pgfpoint{20.647918\du}{14.454935\du}}
\pgfpathlineto{\pgfpoint{20.645727\du}{14.447998\du}}
\pgfpathlineto{\pgfpoint{20.644632\du}{14.441061\du}}
\pgfpathlineto{\pgfpoint{20.643536\du}{14.434124\du}}
\pgfpathlineto{\pgfpoint{20.642076\du}{14.426822\du}}
\pgfpathlineto{\pgfpoint{20.641711\du}{14.419885\du}}
\pgfpathlineto{\pgfpoint{20.641711\du}{14.412949\du}}
\pgfpathlineto{\pgfpoint{20.641346\du}{14.406012\du}}
\pgfpathlineto{\pgfpoint{20.641346\du}{14.406012\du}}
\pgfpathlineto{\pgfpoint{20.641346\du}{14.406012\du}}
\pgfpathlineto{\pgfpoint{20.641346\du}{14.404186\du}}
\pgfpathlineto{\pgfpoint{20.641346\du}{14.403456\du}}
\pgfpathlineto{\pgfpoint{20.640981\du}{14.402361\du}}
\pgfpathlineto{\pgfpoint{20.640981\du}{14.401265\du}}
\pgfpathlineto{\pgfpoint{20.639885\du}{14.400535\du}}
\pgfpathlineto{\pgfpoint{20.639520\du}{14.399440\du}}
\pgfpathlineto{\pgfpoint{20.639155\du}{14.398710\du}}
\pgfpathlineto{\pgfpoint{20.638060\du}{14.398345\du}}
\pgfpathlineto{\pgfpoint{20.636600\du}{14.397249\du}}
\pgfpathlineto{\pgfpoint{20.634774\du}{14.395789\du}}
\pgfpathlineto{\pgfpoint{20.632949\du}{14.395424\du}}
\pgfpathlineto{\pgfpoint{20.631123\du}{14.395424\du}}
\pgfpathlineto{\pgfpoint{20.628933\du}{14.395424\du}}
\pgfpathlineto{\pgfpoint{20.627472\du}{14.395789\du}}
\pgfpathlineto{\pgfpoint{20.625282\du}{14.397249\du}}
\pgfpathlineto{\pgfpoint{20.623456\du}{14.398345\du}}
\pgfpathlineto{\pgfpoint{20.623091\du}{14.398710\du}}
\pgfpathlineto{\pgfpoint{20.622726\du}{14.399440\du}}
\pgfpathlineto{\pgfpoint{20.621996\du}{14.400535\du}}
\pgfpathlineto{\pgfpoint{20.621266\du}{14.401265\du}}
\pgfpathlineto{\pgfpoint{20.621266\du}{14.402361\du}}
\pgfpathlineto{\pgfpoint{20.620900\du}{14.403456\du}}
\pgfpathlineto{\pgfpoint{20.620900\du}{14.404186\du}}
\pgfpathlineto{\pgfpoint{20.620900\du}{14.406012\du}}
\pgfusepath{fill}
\pgfsetbuttcap
\pgfsetmiterjoin
\pgfsetdash{}{0pt}
\definecolor{dialinecolor}{rgb}{0.678431, 0.839216, 0.905882}
\pgfsetfillcolor{dialinecolor}
\pgfpathmoveto{\pgfpoint{22.311664\du}{13.818571\du}}
\pgfpathlineto{\pgfpoint{22.311664\du}{13.818571\du}}
\pgfpathlineto{\pgfpoint{22.267852\du}{13.818571\du}}
\pgfpathlineto{\pgfpoint{22.225136\du}{13.818936\du}}
\pgfpathlineto{\pgfpoint{22.181689\du}{13.820031\du}}
\pgfpathlineto{\pgfpoint{22.139703\du}{13.821492\du}}
\pgfpathlineto{\pgfpoint{22.097352\du}{13.822952\du}}
\pgfpathlineto{\pgfpoint{22.055731\du}{13.824778\du}}
\pgfpathlineto{\pgfpoint{22.013745\du}{13.827333\du}}
\pgfpathlineto{\pgfpoint{21.972489\du}{13.830254\du}}
\pgfpathlineto{\pgfpoint{21.931598\du}{13.833175\du}}
\pgfpathlineto{\pgfpoint{21.890707\du}{13.836096\du}}
\pgfpathlineto{\pgfpoint{21.850911\du}{13.840112\du}}
\pgfpathlineto{\pgfpoint{21.810751\du}{13.844128\du}}
\pgfpathlineto{\pgfpoint{21.771685\du}{13.848874\du}}
\pgfpathlineto{\pgfpoint{21.732985\du}{13.853620\du}}
\pgfpathlineto{\pgfpoint{21.694285\du}{13.858367\du}}
\pgfpathlineto{\pgfpoint{21.656315\du}{13.863478\du}}
\pgfpathlineto{\pgfpoint{21.618710\du}{13.869319\du}}
\pgfpathlineto{\pgfpoint{21.581470\du}{13.875161\du}}
\pgfpathlineto{\pgfpoint{21.544960\du}{13.881733\du}}
\pgfpathlineto{\pgfpoint{21.508816\du}{13.887939\du}}
\pgfpathlineto{\pgfpoint{21.473036\du}{13.895241\du}}
\pgfpathlineto{\pgfpoint{21.438352\du}{13.902178\du}}
\pgfpathlineto{\pgfpoint{21.404033\du}{13.909115\du}}
\pgfpathlineto{\pgfpoint{21.369349\du}{13.917147\du}}
\pgfpathlineto{\pgfpoint{21.336855\du}{13.924814\du}}
\pgfpathlineto{\pgfpoint{21.303631\du}{13.933211\du}}
\pgfpathlineto{\pgfpoint{21.271868\du}{13.941974\du}}
\pgfpathlineto{\pgfpoint{21.240105\du}{13.950736\du}}
\pgfpathlineto{\pgfpoint{21.209436\du}{13.959498\du}}
\pgfpathlineto{\pgfpoint{21.178403\du}{13.968991\du}}
\pgfpathlineto{\pgfpoint{21.163799\du}{13.973372\du}}
\pgfpathlineto{\pgfpoint{21.149926\du}{13.978118\du}}
\pgfpathlineto{\pgfpoint{21.134227\du}{13.982865\du}}
\pgfpathlineto{\pgfpoint{21.120353\du}{13.987976\du}}
\pgfpathlineto{\pgfpoint{21.105749\du}{13.992722\du}}
\pgfpathlineto{\pgfpoint{21.091875\du}{13.998199\du}}
\pgfpathlineto{\pgfpoint{21.077637\du}{14.002580\du}}
\pgfpathlineto{\pgfpoint{21.064493\du}{14.008056\du}}
\pgfpathlineto{\pgfpoint{21.050254\du}{14.013168\du}}
\pgfpathlineto{\pgfpoint{21.037111\du}{14.018644\du}}
\pgfpathlineto{\pgfpoint{21.024697\du}{14.023755\du}}
\pgfpathlineto{\pgfpoint{21.010824\du}{14.029232\du}}
\pgfpathlineto{\pgfpoint{20.998045\du}{14.034343\du}}
\pgfpathlineto{\pgfpoint{20.985997\du}{14.039455\du}}
\pgfpathlineto{\pgfpoint{20.972854\du}{14.045296\du}}
\pgfpathlineto{\pgfpoint{20.960806\du}{14.051503\du}}
\pgfpathlineto{\pgfpoint{20.948392\du}{14.056614\du}}
\pgfpathlineto{\pgfpoint{20.936709\du}{14.062456\du}}
\pgfpathlineto{\pgfpoint{20.925391\du}{14.068297\du}}
\pgfpathlineto{\pgfpoint{20.913708\du}{14.074139\du}}
\pgfpathlineto{\pgfpoint{20.902390\du}{14.079980\du}}
\pgfpathlineto{\pgfpoint{20.890707\du}{14.085092\du}}
\pgfpathlineto{\pgfpoint{20.879754\du}{14.091663\du}}
\pgfpathlineto{\pgfpoint{20.869531\du}{14.097505\du}}
\pgfpathlineto{\pgfpoint{20.858944\du}{14.103346\du}}
\pgfpathlineto{\pgfpoint{20.848721\du}{14.109188\du}}
\pgfpathlineto{\pgfpoint{20.838498\du}{14.115760\du}}
\pgfpathlineto{\pgfpoint{20.828641\du}{14.121966\du}}
\pgfpathlineto{\pgfpoint{20.819148\du}{14.127808\du}}
\pgfpathlineto{\pgfpoint{20.809655\du}{14.134380\du}}
\pgfpathlineto{\pgfpoint{20.800528\du}{14.140951\du}}
\pgfpathlineto{\pgfpoint{20.790670\du}{14.147158\du}}
\pgfpathlineto{\pgfpoint{20.782273\du}{14.153730\du}}
\pgfpathlineto{\pgfpoint{20.773511\du}{14.160302\du}}
\pgfpathlineto{\pgfpoint{20.765114\du}{14.166508\du}}
\pgfpathlineto{\pgfpoint{20.756351\du}{14.173080\du}}
\pgfpathlineto{\pgfpoint{20.748684\du}{14.180017\du}}
\pgfpathlineto{\pgfpoint{20.741747\du}{14.186589\du}}
\pgfpathlineto{\pgfpoint{20.732985\du}{14.192795\du}}
\pgfpathlineto{\pgfpoint{20.726048\du}{14.200097\du}}
\pgfpathlineto{\pgfpoint{20.719111\du}{14.206304\du}}
\pgfpathlineto{\pgfpoint{20.712175\du}{14.213241\du}}
\pgfpathlineto{\pgfpoint{20.705603\du}{14.220543\du}}
\pgfpathlineto{\pgfpoint{20.698666\du}{14.226749\du}}
\pgfpathlineto{\pgfpoint{20.692825\du}{14.234051\du}}
\pgfpathlineto{\pgfpoint{20.686983\du}{14.240988\du}}
\pgfpathlineto{\pgfpoint{20.681141\du}{14.248655\du}}
\pgfpathlineto{\pgfpoint{20.675665\du}{14.255592\du}}
\pgfpathlineto{\pgfpoint{20.670919\du}{14.262529\du}}
\pgfpathlineto{\pgfpoint{20.665442\du}{14.269466\du}}
\pgfpathlineto{\pgfpoint{20.661426\du}{14.276767\du}}
\pgfpathlineto{\pgfpoint{20.656315\du}{14.284069\du}}
\pgfpathlineto{\pgfpoint{20.652664\du}{14.291371\du}}
\pgfpathlineto{\pgfpoint{20.648283\du}{14.298673\du}}
\pgfpathlineto{\pgfpoint{20.644632\du}{14.306340\du}}
\pgfpathlineto{\pgfpoint{20.641346\du}{14.314007\du}}
\pgfpathlineto{\pgfpoint{20.637695\du}{14.320944\du}}
\pgfpathlineto{\pgfpoint{20.634774\du}{14.328611\du}}
\pgfpathlineto{\pgfpoint{20.632218\du}{14.336278\du}}
\pgfpathlineto{\pgfpoint{20.630028\du}{14.343945\du}}
\pgfpathlineto{\pgfpoint{20.627837\du}{14.351612\du}}
\pgfpathlineto{\pgfpoint{20.626012\du}{14.358914\du}}
\pgfpathlineto{\pgfpoint{20.624186\du}{14.366581\du}}
\pgfpathlineto{\pgfpoint{20.623091\du}{14.374248\du}}
\pgfpathlineto{\pgfpoint{20.621996\du}{14.382646\du}}
\pgfpathlineto{\pgfpoint{20.621266\du}{14.389947\du}}
\pgfpathlineto{\pgfpoint{20.620900\du}{14.397614\du}}
\pgfpathlineto{\pgfpoint{20.620900\du}{14.406012\du}}
\pgfpathlineto{\pgfpoint{20.641346\du}{14.406012\du}}
\pgfpathlineto{\pgfpoint{20.641711\du}{14.398710\du}}
\pgfpathlineto{\pgfpoint{20.641711\du}{14.391773\du}}
\pgfpathlineto{\pgfpoint{20.642076\du}{14.384836\du}}
\pgfpathlineto{\pgfpoint{20.643536\du}{14.377169\du}}
\pgfpathlineto{\pgfpoint{20.644632\du}{14.370962\du}}
\pgfpathlineto{\pgfpoint{20.645727\du}{14.363660\du}}
\pgfpathlineto{\pgfpoint{20.647918\du}{14.356724\du}}
\pgfpathlineto{\pgfpoint{20.649378\du}{14.349787\du}}
\pgfpathlineto{\pgfpoint{20.651569\du}{14.342850\du}}
\pgfpathlineto{\pgfpoint{20.653759\du}{14.335548\du}}
\pgfpathlineto{\pgfpoint{20.657045\du}{14.329341\du}}
\pgfpathlineto{\pgfpoint{20.659601\du}{14.322405\du}}
\pgfpathlineto{\pgfpoint{20.662521\du}{14.315103\du}}
\pgfpathlineto{\pgfpoint{20.666538\du}{14.308166\du}}
\pgfpathlineto{\pgfpoint{20.670554\du}{14.301229\du}}
\pgfpathlineto{\pgfpoint{20.673839\du}{14.294657\du}}
\pgfpathlineto{\pgfpoint{20.677856\du}{14.287720\du}}
\pgfpathlineto{\pgfpoint{20.682602\du}{14.281149\du}}
\pgfpathlineto{\pgfpoint{20.686983\du}{14.274212\du}}
\pgfpathlineto{\pgfpoint{20.692459\du}{14.267640\du}}
\pgfpathlineto{\pgfpoint{20.696841\du}{14.261433\du}}
\pgfpathlineto{\pgfpoint{20.703047\du}{14.254497\du}}
\pgfpathlineto{\pgfpoint{20.708159\du}{14.247925\du}}
\pgfpathlineto{\pgfpoint{20.714365\du}{14.240988\du}}
\pgfpathlineto{\pgfpoint{20.720207\du}{14.234416\du}}
\pgfpathlineto{\pgfpoint{20.726779\du}{14.228210\du}}
\pgfpathlineto{\pgfpoint{20.732985\du}{14.221638\du}}
\pgfpathlineto{\pgfpoint{20.739922\du}{14.214701\du}}
\pgfpathlineto{\pgfpoint{20.747224\du}{14.208859\du}}
\pgfpathlineto{\pgfpoint{20.754161\du}{14.201558\du}}
\pgfpathlineto{\pgfpoint{20.761828\du}{14.195351\du}}
\pgfpathlineto{\pgfpoint{20.769860\du}{14.189509\du}}
\pgfpathlineto{\pgfpoint{20.777162\du}{14.182938\du}}
\pgfpathlineto{\pgfpoint{20.785559\du}{14.176366\du}}
\pgfpathlineto{\pgfpoint{20.794321\du}{14.170159\du}}
\pgfpathlineto{\pgfpoint{20.803449\du}{14.163587\du}}
\pgfpathlineto{\pgfpoint{20.812211\du}{14.157746\du}}
\pgfpathlineto{\pgfpoint{20.820608\du}{14.151539\du}}
\pgfpathlineto{\pgfpoint{20.829736\du}{14.145698\du}}
\pgfpathlineto{\pgfpoint{20.839593\du}{14.139126\du}}
\pgfpathlineto{\pgfpoint{20.849086\du}{14.133284\du}}
\pgfpathlineto{\pgfpoint{20.858944\du}{14.127443\du}}
\pgfpathlineto{\pgfpoint{20.869531\du}{14.121601\du}}
\pgfpathlineto{\pgfpoint{20.879389\du}{14.115760\du}}
\pgfpathlineto{\pgfpoint{20.889977\du}{14.109188\du}}
\pgfpathlineto{\pgfpoint{20.901295\du}{14.104077\du}}
\pgfpathlineto{\pgfpoint{20.911152\du}{14.098235\du}}
\pgfpathlineto{\pgfpoint{20.922835\du}{14.092394\du}}
\pgfpathlineto{\pgfpoint{20.933788\du}{14.086552\du}}
\pgfpathlineto{\pgfpoint{20.945471\du}{14.080711\du}}
\pgfpathlineto{\pgfpoint{20.957155\du}{14.075234\du}}
\pgfpathlineto{\pgfpoint{20.969933\du}{14.070123\du}}
\pgfpathlineto{\pgfpoint{20.981251\du}{14.064281\du}}
\pgfpathlineto{\pgfpoint{20.993299\du}{14.058805\du}}
\pgfpathlineto{\pgfpoint{21.006443\du}{14.053693\du}}
\pgfpathlineto{\pgfpoint{21.018856\du}{14.048217\du}}
\pgfpathlineto{\pgfpoint{21.031634\du}{14.043106\du}}
\pgfpathlineto{\pgfpoint{21.045143\du}{14.037264\du}}
\pgfpathlineto{\pgfpoint{21.057921\du}{14.032518\du}}
\pgfpathlineto{\pgfpoint{21.071430\du}{14.027406\du}}
\pgfpathlineto{\pgfpoint{21.084938\du}{14.021930\du}}
\pgfpathlineto{\pgfpoint{21.098812\du}{14.017549\du}}
\pgfpathlineto{\pgfpoint{21.112686\du}{14.012072\du}}
\pgfpathlineto{\pgfpoint{21.126925\du}{14.007326\du}}
\pgfpathlineto{\pgfpoint{21.141163\du}{14.002580\du}}
\pgfpathlineto{\pgfpoint{21.155402\du}{13.997468\du}}
\pgfpathlineto{\pgfpoint{21.169641\du}{13.992722\du}}
\pgfpathlineto{\pgfpoint{21.184975\du}{13.987976\du}}
\pgfpathlineto{\pgfpoint{21.214548\du}{13.979214\du}}
\pgfpathlineto{\pgfpoint{21.245946\du}{13.970086\du}}
\pgfpathlineto{\pgfpoint{21.276979\du}{13.961689\du}}
\pgfpathlineto{\pgfpoint{21.309473\du}{13.952927\du}}
\pgfpathlineto{\pgfpoint{21.340871\du}{13.944895\du}}
\pgfpathlineto{\pgfpoint{21.374095\du}{13.937227\du}}
\pgfpathlineto{\pgfpoint{21.408049\du}{13.929560\du}}
\pgfpathlineto{\pgfpoint{21.442368\du}{13.921893\du}}
\pgfpathlineto{\pgfpoint{21.477052\du}{13.914957\du}}
\pgfpathlineto{\pgfpoint{21.513197\du}{13.908020\du}}
\pgfpathlineto{\pgfpoint{21.548246\du}{13.902178\du}}
\pgfpathlineto{\pgfpoint{21.585486\du}{13.895606\du}}
\pgfpathlineto{\pgfpoint{21.621996\du}{13.889765\du}}
\pgfpathlineto{\pgfpoint{21.659236\du}{13.883923\du}}
\pgfpathlineto{\pgfpoint{21.696841\du}{13.878812\du}}
\pgfpathlineto{\pgfpoint{21.735176\du}{13.873336\du}}
\pgfpathlineto{\pgfpoint{21.774241\du}{13.868589\du}}
\pgfpathlineto{\pgfpoint{21.813672\du}{13.864573\du}}
\pgfpathlineto{\pgfpoint{21.853102\du}{13.860557\du}}
\pgfpathlineto{\pgfpoint{21.892898\du}{13.856906\du}}
\pgfpathlineto{\pgfpoint{21.933058\du}{13.853620\du}}
\pgfpathlineto{\pgfpoint{21.973949\du}{13.850700\du}}
\pgfpathlineto{\pgfpoint{22.015205\du}{13.847779\du}}
\pgfpathlineto{\pgfpoint{22.056096\du}{13.845223\du}}
\pgfpathlineto{\pgfpoint{22.098082\du}{13.843033\du}}
\pgfpathlineto{\pgfpoint{22.140433\du}{13.841937\du}}
\pgfpathlineto{\pgfpoint{22.183149\du}{13.840112\du}}
\pgfpathlineto{\pgfpoint{22.225501\du}{13.839382\du}}
\pgfpathlineto{\pgfpoint{22.268947\du}{13.839016\du}}
\pgfpathlineto{\pgfpoint{22.311664\du}{13.839016\du}}
\pgfpathlineto{\pgfpoint{22.311664\du}{13.839016\du}}
\pgfpathlineto{\pgfpoint{22.311664\du}{13.839016\du}}
\pgfpathlineto{\pgfpoint{22.312759\du}{13.838286\du}}
\pgfpathlineto{\pgfpoint{22.313854\du}{13.838286\du}}
\pgfpathlineto{\pgfpoint{22.315314\du}{13.838286\du}}
\pgfpathlineto{\pgfpoint{22.316410\du}{13.837921\du}}
\pgfpathlineto{\pgfpoint{22.316775\du}{13.837191\du}}
\pgfpathlineto{\pgfpoint{22.317870\du}{13.837191\du}}
\pgfpathlineto{\pgfpoint{22.318600\du}{13.836096\du}}
\pgfpathlineto{\pgfpoint{22.319696\du}{13.835365\du}}
\pgfpathlineto{\pgfpoint{22.320791\du}{13.834270\du}}
\pgfpathlineto{\pgfpoint{22.321521\du}{13.832445\du}}
\pgfpathlineto{\pgfpoint{22.321521\du}{13.830254\du}}
\pgfpathlineto{\pgfpoint{22.322251\du}{13.828429\du}}
\pgfpathlineto{\pgfpoint{22.321521\du}{13.826603\du}}
\pgfpathlineto{\pgfpoint{22.321521\du}{13.824778\du}}
\pgfpathlineto{\pgfpoint{22.320791\du}{13.822952\du}}
\pgfpathlineto{\pgfpoint{22.319696\du}{13.821492\du}}
\pgfpathlineto{\pgfpoint{22.318600\du}{13.820762\du}}
\pgfpathlineto{\pgfpoint{22.317870\du}{13.820031\du}}
\pgfpathlineto{\pgfpoint{22.316775\du}{13.819666\du}}
\pgfpathlineto{\pgfpoint{22.316410\du}{13.818936\du}}
\pgfpathlineto{\pgfpoint{22.315314\du}{13.818571\du}}
\pgfpathlineto{\pgfpoint{22.313854\du}{13.818571\du}}
\pgfpathlineto{\pgfpoint{22.312759\du}{13.818571\du}}
\pgfpathlineto{\pgfpoint{22.311664\du}{13.818571\du}}
\pgfusepath{fill}
\pgfsetbuttcap
\pgfsetmiterjoin
\pgfsetdash{}{0pt}
\definecolor{dialinecolor}{rgb}{0.678431, 0.839216, 0.905882}
\pgfsetfillcolor{dialinecolor}
\pgfpathmoveto{\pgfpoint{24.002061\du}{14.406012\du}}
\pgfpathlineto{\pgfpoint{24.002061\du}{14.397614\du}}
\pgfpathlineto{\pgfpoint{24.001331\du}{14.389947\du}}
\pgfpathlineto{\pgfpoint{24.000601\du}{14.382646\du}}
\pgfpathlineto{\pgfpoint{23.999871\du}{14.374248\du}}
\pgfpathlineto{\pgfpoint{23.998776\du}{14.366581\du}}
\pgfpathlineto{\pgfpoint{23.996585\du}{14.358914\du}}
\pgfpathlineto{\pgfpoint{23.995490\du}{14.351612\du}}
\pgfpathlineto{\pgfpoint{23.992569\du}{14.343945\du}}
\pgfpathlineto{\pgfpoint{23.990743\du}{14.336278\du}}
\pgfpathlineto{\pgfpoint{23.987823\du}{14.328611\du}}
\pgfpathlineto{\pgfpoint{23.985267\du}{14.320944\du}}
\pgfpathlineto{\pgfpoint{23.981981\du}{14.314007\du}}
\pgfpathlineto{\pgfpoint{23.977965\du}{14.306340\du}}
\pgfpathlineto{\pgfpoint{23.974679\du}{14.298673\du}}
\pgfpathlineto{\pgfpoint{23.970663\du}{14.291371\du}}
\pgfpathlineto{\pgfpoint{23.966647\du}{14.284069\du}}
\pgfpathlineto{\pgfpoint{23.962266\du}{14.276767\du}}
\pgfpathlineto{\pgfpoint{23.957520\du}{14.269466\du}}
\pgfpathlineto{\pgfpoint{23.952043\du}{14.262529\du}}
\pgfpathlineto{\pgfpoint{23.947297\du}{14.255592\du}}
\pgfpathlineto{\pgfpoint{23.941455\du}{14.247925\du}}
\pgfpathlineto{\pgfpoint{23.935979\du}{14.240988\du}}
\pgfpathlineto{\pgfpoint{23.930137\du}{14.234051\du}}
\pgfpathlineto{\pgfpoint{23.923931\du}{14.226749\du}}
\pgfpathlineto{\pgfpoint{23.917359\du}{14.220543\du}}
\pgfpathlineto{\pgfpoint{23.910422\du}{14.213241\du}}
\pgfpathlineto{\pgfpoint{23.903850\du}{14.206304\du}}
\pgfpathlineto{\pgfpoint{23.896549\du}{14.200097\du}}
\pgfpathlineto{\pgfpoint{23.889977\du}{14.192795\du}}
\pgfpathlineto{\pgfpoint{23.881945\du}{14.186589\du}}
\pgfpathlineto{\pgfpoint{23.873913\du}{14.180017\du}}
\pgfpathlineto{\pgfpoint{23.866611\du}{14.173080\du}}
\pgfpathlineto{\pgfpoint{23.857848\du}{14.166508\du}}
\pgfpathlineto{\pgfpoint{23.849816\du}{14.160302\du}}
\pgfpathlineto{\pgfpoint{23.840689\du}{14.153730\du}}
\pgfpathlineto{\pgfpoint{23.831926\du}{14.147158\du}}
\pgfpathlineto{\pgfpoint{23.822434\du}{14.140951\du}}
\pgfpathlineto{\pgfpoint{23.813306\du}{14.134380\du}}
\pgfpathlineto{\pgfpoint{23.804179\du}{14.127808\du}}
\pgfpathlineto{\pgfpoint{23.794321\du}{14.121966\du}}
\pgfpathlineto{\pgfpoint{23.784464\du}{14.115760\du}}
\pgfpathlineto{\pgfpoint{23.774241\du}{14.109188\du}}
\pgfpathlineto{\pgfpoint{23.763653\du}{14.103346\du}}
\pgfpathlineto{\pgfpoint{23.753796\du}{14.097505\du}}
\pgfpathlineto{\pgfpoint{23.743208\du}{14.091663\du}}
\pgfpathlineto{\pgfpoint{23.731890\du}{14.085092\du}}
\pgfpathlineto{\pgfpoint{23.720572\du}{14.079980\du}}
\pgfpathlineto{\pgfpoint{23.708889\du}{14.074139\du}}
\pgfpathlineto{\pgfpoint{23.697936\du}{14.068297\du}}
\pgfpathlineto{\pgfpoint{23.685888\du}{14.062456\du}}
\pgfpathlineto{\pgfpoint{23.674570\du}{14.056614\du}}
\pgfpathlineto{\pgfpoint{23.662156\du}{14.051503\du}}
\pgfpathlineto{\pgfpoint{23.649743\du}{14.045296\du}}
\pgfpathlineto{\pgfpoint{23.636965\du}{14.039455\du}}
\pgfpathlineto{\pgfpoint{23.624917\du}{14.034343\du}}
\pgfpathlineto{\pgfpoint{23.612138\du}{14.029232\du}}
\pgfpathlineto{\pgfpoint{23.598630\du}{14.023755\du}}
\pgfpathlineto{\pgfpoint{23.585486\du}{14.018644\du}}
\pgfpathlineto{\pgfpoint{23.572343\du}{14.013168\du}}
\pgfpathlineto{\pgfpoint{23.558104\du}{14.008056\du}}
\pgfpathlineto{\pgfpoint{23.544960\du}{14.002580\du}}
\pgfpathlineto{\pgfpoint{23.530722\du}{13.998199\du}}
\pgfpathlineto{\pgfpoint{23.516848\du}{13.992722\du}}
\pgfpathlineto{\pgfpoint{23.502609\du}{13.987976\du}}
\pgfpathlineto{\pgfpoint{23.488735\du}{13.982865\du}}
\pgfpathlineto{\pgfpoint{23.473766\du}{13.978118\du}}
\pgfpathlineto{\pgfpoint{23.458798\du}{13.973372\du}}
\pgfpathlineto{\pgfpoint{23.444559\du}{13.968991\du}}
\pgfpathlineto{\pgfpoint{23.413891\du}{13.959498\du}}
\pgfpathlineto{\pgfpoint{23.383222\du}{13.950736\du}}
\pgfpathlineto{\pgfpoint{23.351824\du}{13.941974\du}}
\pgfpathlineto{\pgfpoint{23.320061\du}{13.933211\du}}
\pgfpathlineto{\pgfpoint{23.287202\du}{13.924814\du}}
\pgfpathlineto{\pgfpoint{23.253613\du}{13.917147\du}}
\pgfpathlineto{\pgfpoint{23.219659\du}{13.909115\du}}
\pgfpathlineto{\pgfpoint{23.184975\du}{13.902178\du}}
\pgfpathlineto{\pgfpoint{23.150291\du}{13.895241\du}}
\pgfpathlineto{\pgfpoint{23.114876\du}{13.887939\du}}
\pgfpathlineto{\pgfpoint{23.078367\du}{13.881733\du}}
\pgfpathlineto{\pgfpoint{23.041857\du}{13.875161\du}}
\pgfpathlineto{\pgfpoint{23.004982\du}{13.869319\du}}
\pgfpathlineto{\pgfpoint{22.967742\du}{13.863478\du}}
\pgfpathlineto{\pgfpoint{22.928677\du}{13.858367\du}}
\pgfpathlineto{\pgfpoint{22.890707\du}{13.853620\du}}
\pgfpathlineto{\pgfpoint{22.851277\du}{13.848874\du}}
\pgfpathlineto{\pgfpoint{22.812576\du}{13.844128\du}}
\pgfpathlineto{\pgfpoint{22.772416\du}{13.840112\du}}
\pgfpathlineto{\pgfpoint{22.732620\du}{13.836096\du}}
\pgfpathlineto{\pgfpoint{22.691729\du}{13.833175\du}}
\pgfpathlineto{\pgfpoint{22.651203\du}{13.830254\du}}
\pgfpathlineto{\pgfpoint{22.609582\du}{13.827333\du}}
\pgfpathlineto{\pgfpoint{22.567961\du}{13.824778\du}}
\pgfpathlineto{\pgfpoint{22.526340\du}{13.822952\du}}
\pgfpathlineto{\pgfpoint{22.483624\du}{13.821492\du}}
\pgfpathlineto{\pgfpoint{22.441638\du}{13.820031\du}}
\pgfpathlineto{\pgfpoint{22.398557\du}{13.818936\du}}
\pgfpathlineto{\pgfpoint{22.355110\du}{13.818571\du}}
\pgfpathlineto{\pgfpoint{22.311664\du}{13.818571\du}}
\pgfpathlineto{\pgfpoint{22.311664\du}{13.839016\du}}
\pgfpathlineto{\pgfpoint{22.354745\du}{13.839016\du}}
\pgfpathlineto{\pgfpoint{22.398191\du}{13.839382\du}}
\pgfpathlineto{\pgfpoint{22.440178\du}{13.840112\du}}
\pgfpathlineto{\pgfpoint{22.482894\du}{13.841937\du}}
\pgfpathlineto{\pgfpoint{22.525610\du}{13.843033\du}}
\pgfpathlineto{\pgfpoint{22.567231\du}{13.845223\du}}
\pgfpathlineto{\pgfpoint{22.608487\du}{13.847779\du}}
\pgfpathlineto{\pgfpoint{22.649378\du}{13.850700\du}}
\pgfpathlineto{\pgfpoint{22.690634\du}{13.853620\du}}
\pgfpathlineto{\pgfpoint{22.730795\du}{13.856906\du}}
\pgfpathlineto{\pgfpoint{22.770955\du}{13.860557\du}}
\pgfpathlineto{\pgfpoint{22.810021\du}{13.864573\du}}
\pgfpathlineto{\pgfpoint{22.849451\du}{13.868589\du}}
\pgfpathlineto{\pgfpoint{22.888151\du}{13.873336\du}}
\pgfpathlineto{\pgfpoint{22.926852\du}{13.878812\du}}
\pgfpathlineto{\pgfpoint{22.964457\du}{13.883923\du}}
\pgfpathlineto{\pgfpoint{23.001696\du}{13.889765\du}}
\pgfpathlineto{\pgfpoint{23.038206\du}{13.895606\du}}
\pgfpathlineto{\pgfpoint{23.075081\du}{13.902178\du}}
\pgfpathlineto{\pgfpoint{23.110495\du}{13.908020\du}}
\pgfpathlineto{\pgfpoint{23.146275\du}{13.914957\du}}
\pgfpathlineto{\pgfpoint{23.180959\du}{13.921893\du}}
\pgfpathlineto{\pgfpoint{23.214913\du}{13.929560\du}}
\pgfpathlineto{\pgfpoint{23.249232\du}{13.937227\du}}
\pgfpathlineto{\pgfpoint{23.282456\du}{13.944895\du}}
\pgfpathlineto{\pgfpoint{23.314584\du}{13.952927\du}}
\pgfpathlineto{\pgfpoint{23.346713\du}{13.961689\du}}
\pgfpathlineto{\pgfpoint{23.378111\du}{13.970086\du}}
\pgfpathlineto{\pgfpoint{23.408779\du}{13.979214\du}}
\pgfpathlineto{\pgfpoint{23.438717\du}{13.987976\du}}
\pgfpathlineto{\pgfpoint{23.452956\du}{13.992722\du}}
\pgfpathlineto{\pgfpoint{23.467560\du}{13.997468\du}}
\pgfpathlineto{\pgfpoint{23.481434\du}{14.002580\du}}
\pgfpathlineto{\pgfpoint{23.496037\du}{14.007326\du}}
\pgfpathlineto{\pgfpoint{23.510276\du}{14.012072\du}}
\pgfpathlineto{\pgfpoint{23.524150\du}{14.017549\du}}
\pgfpathlineto{\pgfpoint{23.538389\du}{14.021930\du}}
\pgfpathlineto{\pgfpoint{23.551532\du}{14.027406\du}}
\pgfpathlineto{\pgfpoint{23.565041\du}{14.032518\du}}
\pgfpathlineto{\pgfpoint{23.578184\du}{14.037264\du}}
\pgfpathlineto{\pgfpoint{23.590963\du}{14.043106\du}}
\pgfpathlineto{\pgfpoint{23.603741\du}{14.048217\du}}
\pgfpathlineto{\pgfpoint{23.616884\du}{14.053693\du}}
\pgfpathlineto{\pgfpoint{23.628933\du}{14.058805\du}}
\pgfpathlineto{\pgfpoint{23.641346\du}{14.064281\du}}
\pgfpathlineto{\pgfpoint{23.653394\du}{14.070123\du}}
\pgfpathlineto{\pgfpoint{23.665807\du}{14.075234\du}}
\pgfpathlineto{\pgfpoint{23.677125\du}{14.080711\du}}
\pgfpathlineto{\pgfpoint{23.689174\du}{14.086552\du}}
\pgfpathlineto{\pgfpoint{23.699761\du}{14.092394\du}}
\pgfpathlineto{\pgfpoint{23.711810\du}{14.098235\du}}
\pgfpathlineto{\pgfpoint{23.722032\du}{14.104077\du}}
\pgfpathlineto{\pgfpoint{23.732985\du}{14.109188\du}}
\pgfpathlineto{\pgfpoint{23.743938\du}{14.115760\du}}
\pgfpathlineto{\pgfpoint{23.753796\du}{14.121601\du}}
\pgfpathlineto{\pgfpoint{23.763653\du}{14.127443\du}}
\pgfpathlineto{\pgfpoint{23.773876\du}{14.133284\du}}
\pgfpathlineto{\pgfpoint{23.783003\du}{14.139126\du}}
\pgfpathlineto{\pgfpoint{23.793226\du}{14.145698\du}}
\pgfpathlineto{\pgfpoint{23.802719\du}{14.151539\du}}
\pgfpathlineto{\pgfpoint{23.811481\du}{14.157746\du}}
\pgfpathlineto{\pgfpoint{23.819513\du}{14.163587\du}}
\pgfpathlineto{\pgfpoint{23.828275\du}{14.170159\du}}
\pgfpathlineto{\pgfpoint{23.836673\du}{14.176366\du}}
\pgfpathlineto{\pgfpoint{23.845435\du}{14.182938\du}}
\pgfpathlineto{\pgfpoint{23.853102\du}{14.189509\du}}
\pgfpathlineto{\pgfpoint{23.861134\du}{14.195351\du}}
\pgfpathlineto{\pgfpoint{23.868436\du}{14.201558\du}}
\pgfpathlineto{\pgfpoint{23.876103\du}{14.208859\du}}
\pgfpathlineto{\pgfpoint{23.882675\du}{14.214701\du}}
\pgfpathlineto{\pgfpoint{23.889977\du}{14.221638\du}}
\pgfpathlineto{\pgfpoint{23.895818\du}{14.228210\du}}
\pgfpathlineto{\pgfpoint{23.902755\du}{14.234416\du}}
\pgfpathlineto{\pgfpoint{23.909327\du}{14.240988\du}}
\pgfpathlineto{\pgfpoint{23.914438\du}{14.247925\du}}
\pgfpathlineto{\pgfpoint{23.919915\du}{14.254497\du}}
\pgfpathlineto{\pgfpoint{23.926121\du}{14.261433\du}}
\pgfpathlineto{\pgfpoint{23.930868\du}{14.267640\du}}
\pgfpathlineto{\pgfpoint{23.935979\du}{14.274212\du}}
\pgfpathlineto{\pgfpoint{23.940725\du}{14.281149\du}}
\pgfpathlineto{\pgfpoint{23.945106\du}{14.287720\du}}
\pgfpathlineto{\pgfpoint{23.949488\du}{14.294657\du}}
\pgfpathlineto{\pgfpoint{23.952408\du}{14.301229\du}}
\pgfpathlineto{\pgfpoint{23.956424\du}{14.308166\du}}
\pgfpathlineto{\pgfpoint{23.959345\du}{14.315103\du}}
\pgfpathlineto{\pgfpoint{23.963361\du}{14.322405\du}}
\pgfpathlineto{\pgfpoint{23.965917\du}{14.328611\du}}
\pgfpathlineto{\pgfpoint{23.968838\du}{14.335548\du}}
\pgfpathlineto{\pgfpoint{23.971393\du}{14.342850\du}}
\pgfpathlineto{\pgfpoint{23.973584\du}{14.349787\du}}
\pgfpathlineto{\pgfpoint{23.975044\du}{14.356724\du}}
\pgfpathlineto{\pgfpoint{23.977235\du}{14.363660\du}}
\pgfpathlineto{\pgfpoint{23.977965\du}{14.370962\du}}
\pgfpathlineto{\pgfpoint{23.979060\du}{14.377169\du}}
\pgfpathlineto{\pgfpoint{23.980886\du}{14.384836\du}}
\pgfpathlineto{\pgfpoint{23.981251\du}{14.391773\du}}
\pgfpathlineto{\pgfpoint{23.981251\du}{14.398710\du}}
\pgfpathlineto{\pgfpoint{23.981981\du}{14.406012\du}}
\pgfpathlineto{\pgfpoint{24.002061\du}{14.406012\du}}
\pgfusepath{fill}
\pgfsetbuttcap
\pgfsetmiterjoin
\pgfsetdash{}{0pt}
\definecolor{dialinecolor}{rgb}{0.027451, 0.486275, 0.682353}
\pgfsetfillcolor{dialinecolor}
\pgfpathmoveto{\pgfpoint{20.626012\du}{13.596592\du}}
\pgfpathlineto{\pgfpoint{20.626012\du}{14.420981\du}}
\pgfpathlineto{\pgfpoint{23.991474\du}{14.420981\du}}
\pgfpathlineto{\pgfpoint{23.992204\du}{13.597322\du}}
\pgfpathlineto{\pgfpoint{20.626012\du}{13.596592\du}}
\pgfusepath{fill}
\pgfsetbuttcap
\pgfsetmiterjoin
\pgfsetdash{}{0pt}
\definecolor{dialinecolor}{rgb}{0.235294, 0.686275, 0.894118}
\pgfsetfillcolor{dialinecolor}
\pgfpathmoveto{\pgfpoint{23.991474\du}{13.580893\du}}
\pgfpathlineto{\pgfpoint{23.990013\du}{13.610831\du}}
\pgfpathlineto{\pgfpoint{23.982711\du}{13.640039\du}}
\pgfpathlineto{\pgfpoint{23.972489\du}{13.669246\du}}
\pgfpathlineto{\pgfpoint{23.957885\du}{13.697359\du}}
\pgfpathlineto{\pgfpoint{23.938535\du}{13.725471\du}}
\pgfpathlineto{\pgfpoint{23.916629\du}{13.752854\du}}
\pgfpathlineto{\pgfpoint{23.889977\du}{13.779141\du}}
\pgfpathlineto{\pgfpoint{23.859309\du}{13.805428\du}}
\pgfpathlineto{\pgfpoint{23.826450\du}{13.831349\du}}
\pgfpathlineto{\pgfpoint{23.789210\du}{13.856541\du}}
\pgfpathlineto{\pgfpoint{23.748319\du}{13.880637\du}}
\pgfpathlineto{\pgfpoint{23.704508\du}{13.904004\du}}
\pgfpathlineto{\pgfpoint{23.658140\du}{13.926275\du}}
\pgfpathlineto{\pgfpoint{23.607757\du}{13.948180\du}}
\pgfpathlineto{\pgfpoint{23.554818\du}{13.969356\du}}
\pgfpathlineto{\pgfpoint{23.499323\du}{13.989436\du}}
\pgfpathlineto{\pgfpoint{23.441273\du}{14.008056\du}}
\pgfpathlineto{\pgfpoint{23.380667\du}{14.026676\du}}
\pgfpathlineto{\pgfpoint{23.316775\du}{14.043836\du}}
\pgfpathlineto{\pgfpoint{23.251788\du}{14.059535\du}}
\pgfpathlineto{\pgfpoint{23.183149\du}{14.074869\du}}
\pgfpathlineto{\pgfpoint{23.112321\du}{14.088743\du}}
\pgfpathlineto{\pgfpoint{23.040397\du}{14.101521\du}}
\pgfpathlineto{\pgfpoint{22.965552\du}{14.112839\du}}
\pgfpathlineto{\pgfpoint{22.889612\du}{14.123427\du}}
\pgfpathlineto{\pgfpoint{22.811116\du}{14.132554\du}}
\pgfpathlineto{\pgfpoint{22.731525\du}{14.140221\du}}
\pgfpathlineto{\pgfpoint{22.650473\du}{14.146793\du}}
\pgfpathlineto{\pgfpoint{22.567231\du}{14.151904\du}}
\pgfpathlineto{\pgfpoint{22.483624\du}{14.155555\du}}
\pgfpathlineto{\pgfpoint{22.398191\du}{14.157381\du}}
\pgfpathlineto{\pgfpoint{22.311664\du}{14.158476\du}}
\pgfpathlineto{\pgfpoint{22.225501\du}{14.157381\du}}
\pgfpathlineto{\pgfpoint{22.139703\du}{14.155555\du}}
\pgfpathlineto{\pgfpoint{22.056096\du}{14.151904\du}}
\pgfpathlineto{\pgfpoint{21.973219\du}{14.146793\du}}
\pgfpathlineto{\pgfpoint{21.891802\du}{14.140221\du}}
\pgfpathlineto{\pgfpoint{21.812211\du}{14.132554\du}}
\pgfpathlineto{\pgfpoint{21.734446\du}{14.123427\du}}
\pgfpathlineto{\pgfpoint{21.657775\du}{14.112839\du}}
\pgfpathlineto{\pgfpoint{21.583661\du}{14.101521\du}}
\pgfpathlineto{\pgfpoint{21.511006\du}{14.088743\du}}
\pgfpathlineto{\pgfpoint{21.440543\du}{14.074869\du}}
\pgfpathlineto{\pgfpoint{21.371904\du}{14.059535\du}}
\pgfpathlineto{\pgfpoint{21.306187\du}{14.043836\du}}
\pgfpathlineto{\pgfpoint{21.242660\du}{14.026676\du}}
\pgfpathlineto{\pgfpoint{21.181689\du}{14.008056\du}}
\pgfpathlineto{\pgfpoint{21.123274\du}{13.989436\du}}
\pgfpathlineto{\pgfpoint{21.068144\du}{13.969356\du}}
\pgfpathlineto{\pgfpoint{21.015205\du}{13.948180\du}}
\pgfpathlineto{\pgfpoint{20.965187\du}{13.926275\du}}
\pgfpathlineto{\pgfpoint{20.918089\du}{13.904004\du}}
\pgfpathlineto{\pgfpoint{20.874643\du}{13.880637\du}}
\pgfpathlineto{\pgfpoint{20.833752\du}{13.856541\du}}
\pgfpathlineto{\pgfpoint{20.796512\du}{13.831349\du}}
\pgfpathlineto{\pgfpoint{20.763288\du}{13.805428\du}}
\pgfpathlineto{\pgfpoint{20.732985\du}{13.779141\du}}
\pgfpathlineto{\pgfpoint{20.706333\du}{13.752854\du}}
\pgfpathlineto{\pgfpoint{20.684427\du}{13.725471\du}}
\pgfpathlineto{\pgfpoint{20.665077\du}{13.697359\du}}
\pgfpathlineto{\pgfpoint{20.650473\du}{13.669246\du}}
\pgfpathlineto{\pgfpoint{20.639885\du}{13.640039\du}}
\pgfpathlineto{\pgfpoint{20.632949\du}{13.610831\du}}
\pgfpathlineto{\pgfpoint{20.631123\du}{13.580893\du}}
\pgfpathlineto{\pgfpoint{20.632949\du}{13.551685\du}}
\pgfpathlineto{\pgfpoint{20.639885\du}{13.521747\du}}
\pgfpathlineto{\pgfpoint{20.650473\du}{13.493270\du}}
\pgfpathlineto{\pgfpoint{20.665077\du}{13.465157\du}}
\pgfpathlineto{\pgfpoint{20.684427\du}{13.437045\du}}
\pgfpathlineto{\pgfpoint{20.706333\du}{13.409298\du}}
\pgfpathlineto{\pgfpoint{20.732985\du}{13.382646\du}}
\pgfpathlineto{\pgfpoint{20.763288\du}{13.356359\du}}
\pgfpathlineto{\pgfpoint{20.796512\du}{13.331167\du}}
\pgfpathlineto{\pgfpoint{20.833752\du}{13.305975\du}}
\pgfpathlineto{\pgfpoint{20.874643\du}{13.281879\du}}
\pgfpathlineto{\pgfpoint{20.918089\du}{13.258513\du}}
\pgfpathlineto{\pgfpoint{20.965187\du}{13.235512\du}}
\pgfpathlineto{\pgfpoint{21.015205\du}{13.213971\du}}
\pgfpathlineto{\pgfpoint{21.068144\du}{13.192795\du}}
\pgfpathlineto{\pgfpoint{21.123274\du}{13.173080\du}}
\pgfpathlineto{\pgfpoint{21.181689\du}{13.153730\du}}
\pgfpathlineto{\pgfpoint{21.242660\du}{13.135475\du}}
\pgfpathlineto{\pgfpoint{21.306187\du}{13.118681\du}}
\pgfpathlineto{\pgfpoint{21.371904\du}{13.102251\du}}
\pgfpathlineto{\pgfpoint{21.440543\du}{13.086917\du}}
\pgfpathlineto{\pgfpoint{21.511006\du}{13.073409\du}}
\pgfpathlineto{\pgfpoint{21.583661\du}{13.060630\du}}
\pgfpathlineto{\pgfpoint{21.657775\du}{13.048947\du}}
\pgfpathlineto{\pgfpoint{21.734446\du}{13.039089\du}}
\pgfpathlineto{\pgfpoint{21.812211\du}{13.029597\du}}
\pgfpathlineto{\pgfpoint{21.891802\du}{13.021930\du}}
\pgfpathlineto{\pgfpoint{21.973219\du}{13.015358\du}}
\pgfpathlineto{\pgfpoint{22.056096\du}{13.010247\du}}
\pgfpathlineto{\pgfpoint{22.139703\du}{13.006961\du}}
\pgfpathlineto{\pgfpoint{22.225501\du}{13.004405\du}}
\pgfpathlineto{\pgfpoint{22.311664\du}{13.004040\du}}
\pgfpathlineto{\pgfpoint{22.398191\du}{13.004405\du}}
\pgfpathlineto{\pgfpoint{22.483624\du}{13.006961\du}}
\pgfpathlineto{\pgfpoint{22.567231\du}{13.010247\du}}
\pgfpathlineto{\pgfpoint{22.650473\du}{13.015358\du}}
\pgfpathlineto{\pgfpoint{22.731525\du}{13.021930\du}}
\pgfpathlineto{\pgfpoint{22.811116\du}{13.029597\du}}
\pgfpathlineto{\pgfpoint{22.889612\du}{13.039089\du}}
\pgfpathlineto{\pgfpoint{22.965552\du}{13.048947\du}}
\pgfpathlineto{\pgfpoint{23.040397\du}{13.060630\du}}
\pgfpathlineto{\pgfpoint{23.112321\du}{13.073409\du}}
\pgfpathlineto{\pgfpoint{23.183149\du}{13.086917\du}}
\pgfpathlineto{\pgfpoint{23.251788\du}{13.102251\du}}
\pgfpathlineto{\pgfpoint{23.316775\du}{13.118681\du}}
\pgfpathlineto{\pgfpoint{23.380667\du}{13.135475\du}}
\pgfpathlineto{\pgfpoint{23.441273\du}{13.153730\du}}
\pgfpathlineto{\pgfpoint{23.499323\du}{13.173080\du}}
\pgfpathlineto{\pgfpoint{23.554818\du}{13.192795\du}}
\pgfpathlineto{\pgfpoint{23.607757\du}{13.213971\du}}
\pgfpathlineto{\pgfpoint{23.658140\du}{13.235512\du}}
\pgfpathlineto{\pgfpoint{23.704508\du}{13.258513\du}}
\pgfpathlineto{\pgfpoint{23.748319\du}{13.281879\du}}
\pgfpathlineto{\pgfpoint{23.789210\du}{13.305975\du}}
\pgfpathlineto{\pgfpoint{23.826450\du}{13.331167\du}}
\pgfpathlineto{\pgfpoint{23.859309\du}{13.356359\du}}
\pgfpathlineto{\pgfpoint{23.889977\du}{13.382646\du}}
\pgfpathlineto{\pgfpoint{23.916629\du}{13.409298\du}}
\pgfpathlineto{\pgfpoint{23.938535\du}{13.437045\du}}
\pgfpathlineto{\pgfpoint{23.957885\du}{13.465157\du}}
\pgfpathlineto{\pgfpoint{23.972489\du}{13.493270\du}}
\pgfpathlineto{\pgfpoint{23.982711\du}{13.521747\du}}
\pgfpathlineto{\pgfpoint{23.990013\du}{13.551685\du}}
\pgfpathlineto{\pgfpoint{23.991474\du}{13.580893\du}}
\pgfusepath{fill}
\pgfsetbuttcap
\pgfsetmiterjoin
\pgfsetdash{}{0pt}
\definecolor{dialinecolor}{rgb}{0.678431, 0.839216, 0.905882}
\pgfsetfillcolor{dialinecolor}
\pgfpathmoveto{\pgfpoint{22.311664\du}{14.168334\du}}
\pgfpathlineto{\pgfpoint{22.311664\du}{14.168334\du}}
\pgfpathlineto{\pgfpoint{22.355110\du}{14.168334\du}}
\pgfpathlineto{\pgfpoint{22.398557\du}{14.167604\du}}
\pgfpathlineto{\pgfpoint{22.441638\du}{14.166508\du}}
\pgfpathlineto{\pgfpoint{22.483624\du}{14.165413\du}}
\pgfpathlineto{\pgfpoint{22.526340\du}{14.163587\du}}
\pgfpathlineto{\pgfpoint{22.567961\du}{14.161762\du}}
\pgfpathlineto{\pgfpoint{22.609582\du}{14.159571\du}}
\pgfpathlineto{\pgfpoint{22.651203\du}{14.156651\du}}
\pgfpathlineto{\pgfpoint{22.691729\du}{14.153730\du}}
\pgfpathlineto{\pgfpoint{22.732620\du}{14.150809\du}}
\pgfpathlineto{\pgfpoint{22.772416\du}{14.146793\du}}
\pgfpathlineto{\pgfpoint{22.812576\du}{14.142777\du}}
\pgfpathlineto{\pgfpoint{22.851277\du}{14.138031\du}}
\pgfpathlineto{\pgfpoint{22.890707\du}{14.133284\du}}
\pgfpathlineto{\pgfpoint{22.928677\du}{14.128538\du}}
\pgfpathlineto{\pgfpoint{22.967742\du}{14.123427\du}}
\pgfpathlineto{\pgfpoint{23.004982\du}{14.117585\du}}
\pgfpathlineto{\pgfpoint{23.041857\du}{14.111744\du}}
\pgfpathlineto{\pgfpoint{23.078367\du}{14.105172\du}}
\pgfpathlineto{\pgfpoint{23.114876\du}{14.098600\du}}
\pgfpathlineto{\pgfpoint{23.150291\du}{14.091663\du}}
\pgfpathlineto{\pgfpoint{23.184975\du}{14.084727\du}}
\pgfpathlineto{\pgfpoint{23.219659\du}{14.077790\du}}
\pgfpathlineto{\pgfpoint{23.253613\du}{14.070123\du}}
\pgfpathlineto{\pgfpoint{23.287202\du}{14.061725\du}}
\pgfpathlineto{\pgfpoint{23.320061\du}{14.053693\du}}
\pgfpathlineto{\pgfpoint{23.351824\du}{14.045296\du}}
\pgfpathlineto{\pgfpoint{23.383222\du}{14.036169\du}}
\pgfpathlineto{\pgfpoint{23.398557\du}{14.032153\du}}
\pgfpathlineto{\pgfpoint{23.413891\du}{14.027406\du}}
\pgfpathlineto{\pgfpoint{23.429955\du}{14.022660\du}}
\pgfpathlineto{\pgfpoint{23.444559\du}{14.017914\du}}
\pgfpathlineto{\pgfpoint{23.458798\du}{14.013168\du}}
\pgfpathlineto{\pgfpoint{23.473766\du}{14.008786\du}}
\pgfpathlineto{\pgfpoint{23.488735\du}{14.004040\du}}
\pgfpathlineto{\pgfpoint{23.502609\du}{13.998564\du}}
\pgfpathlineto{\pgfpoint{23.516848\du}{13.993817\du}}
\pgfpathlineto{\pgfpoint{23.530722\du}{13.988706\du}}
\pgfpathlineto{\pgfpoint{23.544960\du}{13.983960\du}}
\pgfpathlineto{\pgfpoint{23.558104\du}{13.978849\du}}
\pgfpathlineto{\pgfpoint{23.572343\du}{13.973372\du}}
\pgfpathlineto{\pgfpoint{23.585486\du}{13.968261\du}}
\pgfpathlineto{\pgfpoint{23.598630\du}{13.963149\du}}
\pgfpathlineto{\pgfpoint{23.612138\du}{13.957673\du}}
\pgfpathlineto{\pgfpoint{23.624917\du}{13.952562\du}}
\pgfpathlineto{\pgfpoint{23.636965\du}{13.947085\du}}
\pgfpathlineto{\pgfpoint{23.649743\du}{13.941244\du}}
\pgfpathlineto{\pgfpoint{23.662156\du}{13.935402\du}}
\pgfpathlineto{\pgfpoint{23.674570\du}{13.930291\du}}
\pgfpathlineto{\pgfpoint{23.685888\du}{13.924449\du}}
\pgfpathlineto{\pgfpoint{23.697936\du}{13.918608\du}}
\pgfpathlineto{\pgfpoint{23.708889\du}{13.912766\du}}
\pgfpathlineto{\pgfpoint{23.720572\du}{13.907290\du}}
\pgfpathlineto{\pgfpoint{23.731890\du}{13.901448\du}}
\pgfpathlineto{\pgfpoint{23.743208\du}{13.895241\du}}
\pgfpathlineto{\pgfpoint{23.753796\du}{13.889400\du}}
\pgfpathlineto{\pgfpoint{23.763653\du}{13.883558\du}}
\pgfpathlineto{\pgfpoint{23.774241\du}{13.877717\du}}
\pgfpathlineto{\pgfpoint{23.784464\du}{13.871145\du}}
\pgfpathlineto{\pgfpoint{23.794321\du}{13.864573\du}}
\pgfpathlineto{\pgfpoint{23.804179\du}{13.858732\du}}
\pgfpathlineto{\pgfpoint{23.813306\du}{13.852525\du}}
\pgfpathlineto{\pgfpoint{23.822434\du}{13.845953\du}}
\pgfpathlineto{\pgfpoint{23.831926\du}{13.839382\du}}
\pgfpathlineto{\pgfpoint{23.840689\du}{13.833175\du}}
\pgfpathlineto{\pgfpoint{23.849816\du}{13.826603\du}}
\pgfpathlineto{\pgfpoint{23.857848\du}{13.820031\du}}
\pgfpathlineto{\pgfpoint{23.866611\du}{13.813825\du}}
\pgfpathlineto{\pgfpoint{23.873913\du}{13.807253\du}}
\pgfpathlineto{\pgfpoint{23.881945\du}{13.800316\du}}
\pgfpathlineto{\pgfpoint{23.889977\du}{13.793744\du}}
\pgfpathlineto{\pgfpoint{23.896549\du}{13.786808\du}}
\pgfpathlineto{\pgfpoint{23.903850\du}{13.780236\du}}
\pgfpathlineto{\pgfpoint{23.910422\du}{13.773299\du}}
\pgfpathlineto{\pgfpoint{23.917359\du}{13.766362\du}}
\pgfpathlineto{\pgfpoint{23.923931\du}{13.759790\du}}
\pgfpathlineto{\pgfpoint{23.930137\du}{13.752854\du}}
\pgfpathlineto{\pgfpoint{23.935979\du}{13.745917\du}}
\pgfpathlineto{\pgfpoint{23.941455\du}{13.738980\du}}
\pgfpathlineto{\pgfpoint{23.947297\du}{13.731313\du}}
\pgfpathlineto{\pgfpoint{23.952043\du}{13.724376\du}}
\pgfpathlineto{\pgfpoint{23.957520\du}{13.717074\du}}
\pgfpathlineto{\pgfpoint{23.962266\du}{13.710137\du}}
\pgfpathlineto{\pgfpoint{23.966647\du}{13.702470\du}}
\pgfpathlineto{\pgfpoint{23.970663\du}{13.695533\du}}
\pgfpathlineto{\pgfpoint{23.974679\du}{13.687866\du}}
\pgfpathlineto{\pgfpoint{23.977965\du}{13.680199\du}}
\pgfpathlineto{\pgfpoint{23.981981\du}{13.672897\du}}
\pgfpathlineto{\pgfpoint{23.985267\du}{13.665595\du}}
\pgfpathlineto{\pgfpoint{23.987823\du}{13.658294\du}}
\pgfpathlineto{\pgfpoint{23.990743\du}{13.650627\du}}
\pgfpathlineto{\pgfpoint{23.992569\du}{13.642959\du}}
\pgfpathlineto{\pgfpoint{23.995490\du}{13.635292\du}}
\pgfpathlineto{\pgfpoint{23.996585\du}{13.627625\du}}
\pgfpathlineto{\pgfpoint{23.998776\du}{13.619958\du}}
\pgfpathlineto{\pgfpoint{23.999871\du}{13.612656\du}}
\pgfpathlineto{\pgfpoint{24.000601\du}{13.604259\du}}
\pgfpathlineto{\pgfpoint{24.001331\du}{13.596592\du}}
\pgfpathlineto{\pgfpoint{24.002061\du}{13.588925\du}}
\pgfpathlineto{\pgfpoint{24.002061\du}{13.580893\du}}
\pgfpathlineto{\pgfpoint{23.981981\du}{13.580893\du}}
\pgfpathlineto{\pgfpoint{23.981251\du}{13.587830\du}}
\pgfpathlineto{\pgfpoint{23.981251\du}{13.595497\du}}
\pgfpathlineto{\pgfpoint{23.980886\du}{13.602434\du}}
\pgfpathlineto{\pgfpoint{23.979060\du}{13.609371\du}}
\pgfpathlineto{\pgfpoint{23.977965\du}{13.616673\du}}
\pgfpathlineto{\pgfpoint{23.977235\du}{13.622879\du}}
\pgfpathlineto{\pgfpoint{23.975044\du}{13.630181\du}}
\pgfpathlineto{\pgfpoint{23.973584\du}{13.637118\du}}
\pgfpathlineto{\pgfpoint{23.971393\du}{13.644055\du}}
\pgfpathlineto{\pgfpoint{23.968838\du}{13.650992\du}}
\pgfpathlineto{\pgfpoint{23.965917\du}{13.658294\du}}
\pgfpathlineto{\pgfpoint{23.963361\du}{13.664500\du}}
\pgfpathlineto{\pgfpoint{23.959345\du}{13.671437\du}}
\pgfpathlineto{\pgfpoint{23.956424\du}{13.678739\du}}
\pgfpathlineto{\pgfpoint{23.952408\du}{13.685676\du}}
\pgfpathlineto{\pgfpoint{23.949488\du}{13.691882\du}}
\pgfpathlineto{\pgfpoint{23.945106\du}{13.699184\du}}
\pgfpathlineto{\pgfpoint{23.940725\du}{13.705391\du}}
\pgfpathlineto{\pgfpoint{23.935979\du}{13.712693\du}}
\pgfpathlineto{\pgfpoint{23.930868\du}{13.718900\du}}
\pgfpathlineto{\pgfpoint{23.926121\du}{13.725836\du}}
\pgfpathlineto{\pgfpoint{23.919915\du}{13.732408\du}}
\pgfpathlineto{\pgfpoint{23.914438\du}{13.738980\du}}
\pgfpathlineto{\pgfpoint{23.909327\du}{13.745917\du}}
\pgfpathlineto{\pgfpoint{23.902755\du}{13.752489\du}}
\pgfpathlineto{\pgfpoint{23.895818\du}{13.758695\du}}
\pgfpathlineto{\pgfpoint{23.889977\du}{13.765267\du}}
\pgfpathlineto{\pgfpoint{23.882675\du}{13.772204\du}}
\pgfpathlineto{\pgfpoint{23.876103\du}{13.778775\du}}
\pgfpathlineto{\pgfpoint{23.868436\du}{13.784982\du}}
\pgfpathlineto{\pgfpoint{23.861134\du}{13.791554\du}}
\pgfpathlineto{\pgfpoint{23.853102\du}{13.797395\du}}
\pgfpathlineto{\pgfpoint{23.845435\du}{13.803967\du}}
\pgfpathlineto{\pgfpoint{23.836673\du}{13.810174\du}}
\pgfpathlineto{\pgfpoint{23.828275\du}{13.816746\du}}
\pgfpathlineto{\pgfpoint{23.819513\du}{13.822952\du}}
\pgfpathlineto{\pgfpoint{23.811481\du}{13.828794\du}}
\pgfpathlineto{\pgfpoint{23.802719\du}{13.835365\du}}
\pgfpathlineto{\pgfpoint{23.793226\du}{13.841207\du}}
\pgfpathlineto{\pgfpoint{23.783003\du}{13.847779\du}}
\pgfpathlineto{\pgfpoint{23.773876\du}{13.853620\du}}
\pgfpathlineto{\pgfpoint{23.763653\du}{13.859462\du}}
\pgfpathlineto{\pgfpoint{23.753796\du}{13.865303\du}}
\pgfpathlineto{\pgfpoint{23.743938\du}{13.871145\du}}
\pgfpathlineto{\pgfpoint{23.732985\du}{13.877717\du}}
\pgfpathlineto{\pgfpoint{23.722032\du}{13.882828\du}}
\pgfpathlineto{\pgfpoint{23.711810\du}{13.888670\du}}
\pgfpathlineto{\pgfpoint{23.699761\du}{13.894511\du}}
\pgfpathlineto{\pgfpoint{23.689174\du}{13.900353\du}}
\pgfpathlineto{\pgfpoint{23.677125\du}{13.906194\du}}
\pgfpathlineto{\pgfpoint{23.665807\du}{13.911306\du}}
\pgfpathlineto{\pgfpoint{23.653394\du}{13.916782\du}}
\pgfpathlineto{\pgfpoint{23.641346\du}{13.922624\du}}
\pgfpathlineto{\pgfpoint{23.628933\du}{13.927735\du}}
\pgfpathlineto{\pgfpoint{23.616884\du}{13.933211\du}}
\pgfpathlineto{\pgfpoint{23.603741\du}{13.938323\du}}
\pgfpathlineto{\pgfpoint{23.590963\du}{13.943799\du}}
\pgfpathlineto{\pgfpoint{23.578184\du}{13.949641\du}}
\pgfpathlineto{\pgfpoint{23.565041\du}{13.954022\du}}
\pgfpathlineto{\pgfpoint{23.551532\du}{13.959498\du}}
\pgfpathlineto{\pgfpoint{23.538389\du}{13.964610\du}}
\pgfpathlineto{\pgfpoint{23.524150\du}{13.969356\du}}
\pgfpathlineto{\pgfpoint{23.510276\du}{13.974832\du}}
\pgfpathlineto{\pgfpoint{23.496037\du}{13.979214\du}}
\pgfpathlineto{\pgfpoint{23.481434\du}{13.983960\du}}
\pgfpathlineto{\pgfpoint{23.467560\du}{13.989436\du}}
\pgfpathlineto{\pgfpoint{23.452956\du}{13.993817\du}}
\pgfpathlineto{\pgfpoint{23.438717\du}{13.998564\du}}
\pgfpathlineto{\pgfpoint{23.423018\du}{14.003310\du}}
\pgfpathlineto{\pgfpoint{23.408779\du}{14.007326\du}}
\pgfpathlineto{\pgfpoint{23.393080\du}{14.012072\du}}
\pgfpathlineto{\pgfpoint{23.378111\du}{14.016819\du}}
\pgfpathlineto{\pgfpoint{23.346713\du}{14.024851\du}}
\pgfpathlineto{\pgfpoint{23.314584\du}{14.033613\du}}
\pgfpathlineto{\pgfpoint{23.282456\du}{14.042010\du}}
\pgfpathlineto{\pgfpoint{23.249232\du}{14.049677\du}}
\pgfpathlineto{\pgfpoint{23.214913\du}{14.057344\du}}
\pgfpathlineto{\pgfpoint{23.180959\du}{14.064646\du}}
\pgfpathlineto{\pgfpoint{23.146275\du}{14.071948\du}}
\pgfpathlineto{\pgfpoint{23.110495\du}{14.078885\du}}
\pgfpathlineto{\pgfpoint{23.075081\du}{14.085092\du}}
\pgfpathlineto{\pgfpoint{23.038206\du}{14.090933\du}}
\pgfpathlineto{\pgfpoint{23.001696\du}{14.097140\du}}
\pgfpathlineto{\pgfpoint{22.964457\du}{14.102981\du}}
\pgfpathlineto{\pgfpoint{22.926852\du}{14.108093\du}}
\pgfpathlineto{\pgfpoint{22.888151\du}{14.113204\du}}
\pgfpathlineto{\pgfpoint{22.849451\du}{14.117950\du}}
\pgfpathlineto{\pgfpoint{22.810021\du}{14.121966\du}}
\pgfpathlineto{\pgfpoint{22.770955\du}{14.126348\du}}
\pgfpathlineto{\pgfpoint{22.730795\du}{14.129633\du}}
\pgfpathlineto{\pgfpoint{22.690634\du}{14.133284\du}}
\pgfpathlineto{\pgfpoint{22.649378\du}{14.136205\du}}
\pgfpathlineto{\pgfpoint{22.608487\du}{14.139126\du}}
\pgfpathlineto{\pgfpoint{22.567231\du}{14.141317\du}}
\pgfpathlineto{\pgfpoint{22.525610\du}{14.143872\du}}
\pgfpathlineto{\pgfpoint{22.482894\du}{14.144968\du}}
\pgfpathlineto{\pgfpoint{22.440178\du}{14.146793\du}}
\pgfpathlineto{\pgfpoint{22.398191\du}{14.147158\du}}
\pgfpathlineto{\pgfpoint{22.354745\du}{14.147888\du}}
\pgfpathlineto{\pgfpoint{22.311664\du}{14.147888\du}}
\pgfpathlineto{\pgfpoint{22.311664\du}{14.147888\du}}
\pgfpathlineto{\pgfpoint{22.311664\du}{14.147888\du}}
\pgfpathlineto{\pgfpoint{22.310933\du}{14.148618\du}}
\pgfpathlineto{\pgfpoint{22.309108\du}{14.148618\du}}
\pgfpathlineto{\pgfpoint{22.308013\du}{14.148618\du}}
\pgfpathlineto{\pgfpoint{22.307282\du}{14.148984\du}}
\pgfpathlineto{\pgfpoint{22.306917\du}{14.149714\du}}
\pgfpathlineto{\pgfpoint{22.305457\du}{14.149714\du}}
\pgfpathlineto{\pgfpoint{22.304727\du}{14.150809\du}}
\pgfpathlineto{\pgfpoint{22.303996\du}{14.151539\du}}
\pgfpathlineto{\pgfpoint{22.302901\du}{14.152635\du}}
\pgfpathlineto{\pgfpoint{22.302171\du}{14.154460\du}}
\pgfpathlineto{\pgfpoint{22.302171\du}{14.156651\du}}
\pgfpathlineto{\pgfpoint{22.301441\du}{14.158476\du}}
\pgfpathlineto{\pgfpoint{22.302171\du}{14.160302\du}}
\pgfpathlineto{\pgfpoint{22.302171\du}{14.161762\du}}
\pgfpathlineto{\pgfpoint{22.302901\du}{14.163587\du}}
\pgfpathlineto{\pgfpoint{22.303996\du}{14.165413\du}}
\pgfpathlineto{\pgfpoint{22.304727\du}{14.166143\du}}
\pgfpathlineto{\pgfpoint{22.305457\du}{14.166508\du}}
\pgfpathlineto{\pgfpoint{22.306917\du}{14.167238\du}}
\pgfpathlineto{\pgfpoint{22.307282\du}{14.167604\du}}
\pgfpathlineto{\pgfpoint{22.308013\du}{14.168334\du}}
\pgfpathlineto{\pgfpoint{22.309108\du}{14.168334\du}}
\pgfpathlineto{\pgfpoint{22.310933\du}{14.168334\du}}
\pgfpathlineto{\pgfpoint{22.311664\du}{14.168334\du}}
\pgfusepath{fill}
\pgfsetbuttcap
\pgfsetmiterjoin
\pgfsetdash{}{0pt}
\definecolor{dialinecolor}{rgb}{0.678431, 0.839216, 0.905882}
\pgfsetfillcolor{dialinecolor}
\pgfpathmoveto{\pgfpoint{20.620900\du}{13.580893\du}}
\pgfpathlineto{\pgfpoint{20.620900\du}{13.580893\du}}
\pgfpathlineto{\pgfpoint{20.620900\du}{13.588925\du}}
\pgfpathlineto{\pgfpoint{20.621266\du}{13.596592\du}}
\pgfpathlineto{\pgfpoint{20.621996\du}{13.604259\du}}
\pgfpathlineto{\pgfpoint{20.623091\du}{13.612656\du}}
\pgfpathlineto{\pgfpoint{20.624186\du}{13.619958\du}}
\pgfpathlineto{\pgfpoint{20.626012\du}{13.627625\du}}
\pgfpathlineto{\pgfpoint{20.627837\du}{13.635292\du}}
\pgfpathlineto{\pgfpoint{20.630028\du}{13.642959\du}}
\pgfpathlineto{\pgfpoint{20.632218\du}{13.650627\du}}
\pgfpathlineto{\pgfpoint{20.634774\du}{13.658294\du}}
\pgfpathlineto{\pgfpoint{20.637695\du}{13.665595\du}}
\pgfpathlineto{\pgfpoint{20.641346\du}{13.672897\du}}
\pgfpathlineto{\pgfpoint{20.644632\du}{13.680199\du}}
\pgfpathlineto{\pgfpoint{20.648283\du}{13.687866\du}}
\pgfpathlineto{\pgfpoint{20.652664\du}{13.695533\du}}
\pgfpathlineto{\pgfpoint{20.656315\du}{13.702470\du}}
\pgfpathlineto{\pgfpoint{20.661426\du}{13.710137\du}}
\pgfpathlineto{\pgfpoint{20.665442\du}{13.717074\du}}
\pgfpathlineto{\pgfpoint{20.670919\du}{13.724376\du}}
\pgfpathlineto{\pgfpoint{20.675665\du}{13.731313\du}}
\pgfpathlineto{\pgfpoint{20.681141\du}{13.738980\du}}
\pgfpathlineto{\pgfpoint{20.686983\du}{13.745917\du}}
\pgfpathlineto{\pgfpoint{20.692825\du}{13.752854\du}}
\pgfpathlineto{\pgfpoint{20.698666\du}{13.759790\du}}
\pgfpathlineto{\pgfpoint{20.705603\du}{13.766362\du}}
\pgfpathlineto{\pgfpoint{20.712175\du}{13.773299\du}}
\pgfpathlineto{\pgfpoint{20.719111\du}{13.780236\du}}
\pgfpathlineto{\pgfpoint{20.726048\du}{13.786808\du}}
\pgfpathlineto{\pgfpoint{20.732985\du}{13.793744\du}}
\pgfpathlineto{\pgfpoint{20.741747\du}{13.800316\du}}
\pgfpathlineto{\pgfpoint{20.748684\du}{13.807253\du}}
\pgfpathlineto{\pgfpoint{20.756351\du}{13.813825\du}}
\pgfpathlineto{\pgfpoint{20.765114\du}{13.820031\du}}
\pgfpathlineto{\pgfpoint{20.773511\du}{13.826603\du}}
\pgfpathlineto{\pgfpoint{20.782273\du}{13.833175\du}}
\pgfpathlineto{\pgfpoint{20.790670\du}{13.839382\du}}
\pgfpathlineto{\pgfpoint{20.800528\du}{13.845953\du}}
\pgfpathlineto{\pgfpoint{20.809655\du}{13.852525\du}}
\pgfpathlineto{\pgfpoint{20.819148\du}{13.858732\du}}
\pgfpathlineto{\pgfpoint{20.828641\du}{13.864573\du}}
\pgfpathlineto{\pgfpoint{20.838498\du}{13.871145\du}}
\pgfpathlineto{\pgfpoint{20.848721\du}{13.877717\du}}
\pgfpathlineto{\pgfpoint{20.858944\du}{13.883558\du}}
\pgfpathlineto{\pgfpoint{20.869531\du}{13.889400\du}}
\pgfpathlineto{\pgfpoint{20.879754\du}{13.895241\du}}
\pgfpathlineto{\pgfpoint{20.890707\du}{13.901448\du}}
\pgfpathlineto{\pgfpoint{20.902390\du}{13.907290\du}}
\pgfpathlineto{\pgfpoint{20.913708\du}{13.912766\du}}
\pgfpathlineto{\pgfpoint{20.925391\du}{13.918608\du}}
\pgfpathlineto{\pgfpoint{20.936709\du}{13.924449\du}}
\pgfpathlineto{\pgfpoint{20.948392\du}{13.930291\du}}
\pgfpathlineto{\pgfpoint{20.960806\du}{13.935402\du}}
\pgfpathlineto{\pgfpoint{20.972854\du}{13.941244\du}}
\pgfpathlineto{\pgfpoint{20.985997\du}{13.947085\du}}
\pgfpathlineto{\pgfpoint{20.998045\du}{13.952562\du}}
\pgfpathlineto{\pgfpoint{21.010824\du}{13.957673\du}}
\pgfpathlineto{\pgfpoint{21.024697\du}{13.963149\du}}
\pgfpathlineto{\pgfpoint{21.037111\du}{13.968261\du}}
\pgfpathlineto{\pgfpoint{21.050254\du}{13.973372\du}}
\pgfpathlineto{\pgfpoint{21.064493\du}{13.978849\du}}
\pgfpathlineto{\pgfpoint{21.077637\du}{13.983960\du}}
\pgfpathlineto{\pgfpoint{21.091875\du}{13.988706\du}}
\pgfpathlineto{\pgfpoint{21.105749\du}{13.993817\du}}
\pgfpathlineto{\pgfpoint{21.120353\du}{13.998564\du}}
\pgfpathlineto{\pgfpoint{21.134227\du}{14.004040\du}}
\pgfpathlineto{\pgfpoint{21.149926\du}{14.008786\du}}
\pgfpathlineto{\pgfpoint{21.163799\du}{14.013168\du}}
\pgfpathlineto{\pgfpoint{21.178403\du}{14.017914\du}}
\pgfpathlineto{\pgfpoint{21.193737\du}{14.022660\du}}
\pgfpathlineto{\pgfpoint{21.209436\du}{14.027406\du}}
\pgfpathlineto{\pgfpoint{21.224405\du}{14.032153\du}}
\pgfpathlineto{\pgfpoint{21.240105\du}{14.036169\du}}
\pgfpathlineto{\pgfpoint{21.271868\du}{14.045296\du}}
\pgfpathlineto{\pgfpoint{21.303631\du}{14.053693\du}}
\pgfpathlineto{\pgfpoint{21.336855\du}{14.061725\du}}
\pgfpathlineto{\pgfpoint{21.369349\du}{14.070123\du}}
\pgfpathlineto{\pgfpoint{21.404033\du}{14.077790\du}}
\pgfpathlineto{\pgfpoint{21.438352\du}{14.084727\du}}
\pgfpathlineto{\pgfpoint{21.473036\du}{14.091663\du}}
\pgfpathlineto{\pgfpoint{21.508816\du}{14.098600\du}}
\pgfpathlineto{\pgfpoint{21.544960\du}{14.105172\du}}
\pgfpathlineto{\pgfpoint{21.581470\du}{14.111744\du}}
\pgfpathlineto{\pgfpoint{21.618710\du}{14.117585\du}}
\pgfpathlineto{\pgfpoint{21.656315\du}{14.123427\du}}
\pgfpathlineto{\pgfpoint{21.694285\du}{14.128538\du}}
\pgfpathlineto{\pgfpoint{21.732985\du}{14.133284\du}}
\pgfpathlineto{\pgfpoint{21.771685\du}{14.138031\du}}
\pgfpathlineto{\pgfpoint{21.810751\du}{14.142777\du}}
\pgfpathlineto{\pgfpoint{21.850911\du}{14.146793\du}}
\pgfpathlineto{\pgfpoint{21.890707\du}{14.150809\du}}
\pgfpathlineto{\pgfpoint{21.931598\du}{14.153730\du}}
\pgfpathlineto{\pgfpoint{21.972489\du}{14.156651\du}}
\pgfpathlineto{\pgfpoint{22.013745\du}{14.159571\du}}
\pgfpathlineto{\pgfpoint{22.055731\du}{14.161762\du}}
\pgfpathlineto{\pgfpoint{22.097352\du}{14.163587\du}}
\pgfpathlineto{\pgfpoint{22.139703\du}{14.165413\du}}
\pgfpathlineto{\pgfpoint{22.181689\du}{14.166508\du}}
\pgfpathlineto{\pgfpoint{22.225136\du}{14.167604\du}}
\pgfpathlineto{\pgfpoint{22.267852\du}{14.168334\du}}
\pgfpathlineto{\pgfpoint{22.311664\du}{14.168334\du}}
\pgfpathlineto{\pgfpoint{22.311664\du}{14.147888\du}}
\pgfpathlineto{\pgfpoint{22.268947\du}{14.147888\du}}
\pgfpathlineto{\pgfpoint{22.225501\du}{14.147158\du}}
\pgfpathlineto{\pgfpoint{22.183149\du}{14.146793\du}}
\pgfpathlineto{\pgfpoint{22.140433\du}{14.144968\du}}
\pgfpathlineto{\pgfpoint{22.098082\du}{14.143872\du}}
\pgfpathlineto{\pgfpoint{22.056096\du}{14.141317\du}}
\pgfpathlineto{\pgfpoint{22.015205\du}{14.139126\du}}
\pgfpathlineto{\pgfpoint{21.973949\du}{14.136205\du}}
\pgfpathlineto{\pgfpoint{21.933058\du}{14.133284\du}}
\pgfpathlineto{\pgfpoint{21.892898\du}{14.129633\du}}
\pgfpathlineto{\pgfpoint{21.853102\du}{14.126348\du}}
\pgfpathlineto{\pgfpoint{21.813672\du}{14.121966\du}}
\pgfpathlineto{\pgfpoint{21.774241\du}{14.117950\du}}
\pgfpathlineto{\pgfpoint{21.735176\du}{14.113204\du}}
\pgfpathlineto{\pgfpoint{21.696841\du}{14.108093\du}}
\pgfpathlineto{\pgfpoint{21.659236\du}{14.102981\du}}
\pgfpathlineto{\pgfpoint{21.621996\du}{14.097140\du}}
\pgfpathlineto{\pgfpoint{21.585486\du}{14.090933\du}}
\pgfpathlineto{\pgfpoint{21.548246\du}{14.085092\du}}
\pgfpathlineto{\pgfpoint{21.513197\du}{14.078885\du}}
\pgfpathlineto{\pgfpoint{21.477052\du}{14.071948\du}}
\pgfpathlineto{\pgfpoint{21.442368\du}{14.064646\du}}
\pgfpathlineto{\pgfpoint{21.408049\du}{14.057344\du}}
\pgfpathlineto{\pgfpoint{21.374095\du}{14.049677\du}}
\pgfpathlineto{\pgfpoint{21.340871\du}{14.042010\du}}
\pgfpathlineto{\pgfpoint{21.309473\du}{14.033613\du}}
\pgfpathlineto{\pgfpoint{21.276979\du}{14.024851\du}}
\pgfpathlineto{\pgfpoint{21.245946\du}{14.016819\du}}
\pgfpathlineto{\pgfpoint{21.230247\du}{14.012072\du}}
\pgfpathlineto{\pgfpoint{21.214548\du}{14.007326\du}}
\pgfpathlineto{\pgfpoint{21.199944\du}{14.003310\du}}
\pgfpathlineto{\pgfpoint{21.184975\du}{13.998564\du}}
\pgfpathlineto{\pgfpoint{21.169641\du}{13.993817\du}}
\pgfpathlineto{\pgfpoint{21.155402\du}{13.989436\du}}
\pgfpathlineto{\pgfpoint{21.141163\du}{13.983960\du}}
\pgfpathlineto{\pgfpoint{21.126925\du}{13.979214\du}}
\pgfpathlineto{\pgfpoint{21.112686\du}{13.974832\du}}
\pgfpathlineto{\pgfpoint{21.098812\du}{13.969356\du}}
\pgfpathlineto{\pgfpoint{21.084938\du}{13.964610\du}}
\pgfpathlineto{\pgfpoint{21.071430\du}{13.959498\du}}
\pgfpathlineto{\pgfpoint{21.057921\du}{13.954022\du}}
\pgfpathlineto{\pgfpoint{21.045143\du}{13.949641\du}}
\pgfpathlineto{\pgfpoint{21.031634\du}{13.943799\du}}
\pgfpathlineto{\pgfpoint{21.018856\du}{13.938323\du}}
\pgfpathlineto{\pgfpoint{21.006443\du}{13.933211\du}}
\pgfpathlineto{\pgfpoint{20.993299\du}{13.927735\du}}
\pgfpathlineto{\pgfpoint{20.981251\du}{13.922624\du}}
\pgfpathlineto{\pgfpoint{20.969933\du}{13.916782\du}}
\pgfpathlineto{\pgfpoint{20.957155\du}{13.911306\du}}
\pgfpathlineto{\pgfpoint{20.945471\du}{13.906194\du}}
\pgfpathlineto{\pgfpoint{20.933788\du}{13.900353\du}}
\pgfpathlineto{\pgfpoint{20.922835\du}{13.894511\du}}
\pgfpathlineto{\pgfpoint{20.911152\du}{13.888670\du}}
\pgfpathlineto{\pgfpoint{20.901295\du}{13.882828\du}}
\pgfpathlineto{\pgfpoint{20.889977\du}{13.877717\du}}
\pgfpathlineto{\pgfpoint{20.879389\du}{13.871145\du}}
\pgfpathlineto{\pgfpoint{20.869531\du}{13.865303\du}}
\pgfpathlineto{\pgfpoint{20.858944\du}{13.859462\du}}
\pgfpathlineto{\pgfpoint{20.849086\du}{13.853620\du}}
\pgfpathlineto{\pgfpoint{20.839593\du}{13.847779\du}}
\pgfpathlineto{\pgfpoint{20.829736\du}{13.841207\du}}
\pgfpathlineto{\pgfpoint{20.820608\du}{13.835365\du}}
\pgfpathlineto{\pgfpoint{20.812211\du}{13.828794\du}}
\pgfpathlineto{\pgfpoint{20.803449\du}{13.822952\du}}
\pgfpathlineto{\pgfpoint{20.794321\du}{13.816746\du}}
\pgfpathlineto{\pgfpoint{20.785559\du}{13.810174\du}}
\pgfpathlineto{\pgfpoint{20.777162\du}{13.803967\du}}
\pgfpathlineto{\pgfpoint{20.769860\du}{13.797395\du}}
\pgfpathlineto{\pgfpoint{20.761828\du}{13.791554\du}}
\pgfpathlineto{\pgfpoint{20.754161\du}{13.784982\du}}
\pgfpathlineto{\pgfpoint{20.747224\du}{13.778775\du}}
\pgfpathlineto{\pgfpoint{20.739922\du}{13.772204\du}}
\pgfpathlineto{\pgfpoint{20.732985\du}{13.765267\du}}
\pgfpathlineto{\pgfpoint{20.726779\du}{13.758695\du}}
\pgfpathlineto{\pgfpoint{20.720207\du}{13.752489\du}}
\pgfpathlineto{\pgfpoint{20.714365\du}{13.745917\du}}
\pgfpathlineto{\pgfpoint{20.708159\du}{13.738980\du}}
\pgfpathlineto{\pgfpoint{20.703047\du}{13.732408\du}}
\pgfpathlineto{\pgfpoint{20.696841\du}{13.725836\du}}
\pgfpathlineto{\pgfpoint{20.692459\du}{13.718900\du}}
\pgfpathlineto{\pgfpoint{20.686983\du}{13.712693\du}}
\pgfpathlineto{\pgfpoint{20.682602\du}{13.705391\du}}
\pgfpathlineto{\pgfpoint{20.678221\du}{13.699184\du}}
\pgfpathlineto{\pgfpoint{20.673839\du}{13.691882\du}}
\pgfpathlineto{\pgfpoint{20.670554\du}{13.685676\du}}
\pgfpathlineto{\pgfpoint{20.666538\du}{13.678739\du}}
\pgfpathlineto{\pgfpoint{20.662521\du}{13.671437\du}}
\pgfpathlineto{\pgfpoint{20.659601\du}{13.664500\du}}
\pgfpathlineto{\pgfpoint{20.657045\du}{13.658294\du}}
\pgfpathlineto{\pgfpoint{20.653759\du}{13.650992\du}}
\pgfpathlineto{\pgfpoint{20.651569\du}{13.644055\du}}
\pgfpathlineto{\pgfpoint{20.649378\du}{13.637118\du}}
\pgfpathlineto{\pgfpoint{20.647918\du}{13.630181\du}}
\pgfpathlineto{\pgfpoint{20.645727\du}{13.622879\du}}
\pgfpathlineto{\pgfpoint{20.644632\du}{13.616673\du}}
\pgfpathlineto{\pgfpoint{20.643536\du}{13.609371\du}}
\pgfpathlineto{\pgfpoint{20.642076\du}{13.602434\du}}
\pgfpathlineto{\pgfpoint{20.641711\du}{13.595497\du}}
\pgfpathlineto{\pgfpoint{20.641711\du}{13.587830\du}}
\pgfpathlineto{\pgfpoint{20.641346\du}{13.580893\du}}
\pgfpathlineto{\pgfpoint{20.641346\du}{13.580893\du}}
\pgfpathlineto{\pgfpoint{20.641346\du}{13.580893\du}}
\pgfpathlineto{\pgfpoint{20.641346\du}{13.579798\du}}
\pgfpathlineto{\pgfpoint{20.641346\du}{13.578702\du}}
\pgfpathlineto{\pgfpoint{20.640981\du}{13.577242\du}}
\pgfpathlineto{\pgfpoint{20.640981\du}{13.576877\du}}
\pgfpathlineto{\pgfpoint{20.639885\du}{13.575782\du}}
\pgfpathlineto{\pgfpoint{20.639520\du}{13.574321\du}}
\pgfpathlineto{\pgfpoint{20.639155\du}{13.573956\du}}
\pgfpathlineto{\pgfpoint{20.638060\du}{13.573226\du}}
\pgfpathlineto{\pgfpoint{20.636600\du}{13.572131\du}}
\pgfpathlineto{\pgfpoint{20.634774\du}{13.571401\du}}
\pgfpathlineto{\pgfpoint{20.632949\du}{13.571035\du}}
\pgfpathlineto{\pgfpoint{20.631123\du}{13.571035\du}}
\pgfpathlineto{\pgfpoint{20.628933\du}{13.571035\du}}
\pgfpathlineto{\pgfpoint{20.627472\du}{13.571401\du}}
\pgfpathlineto{\pgfpoint{20.625282\du}{13.572131\du}}
\pgfpathlineto{\pgfpoint{20.623456\du}{13.573226\du}}
\pgfpathlineto{\pgfpoint{20.623091\du}{13.573956\du}}
\pgfpathlineto{\pgfpoint{20.622726\du}{13.574321\du}}
\pgfpathlineto{\pgfpoint{20.621996\du}{13.575782\du}}
\pgfpathlineto{\pgfpoint{20.621266\du}{13.576877\du}}
\pgfpathlineto{\pgfpoint{20.621266\du}{13.577242\du}}
\pgfpathlineto{\pgfpoint{20.620900\du}{13.578702\du}}
\pgfpathlineto{\pgfpoint{20.620900\du}{13.579798\du}}
\pgfpathlineto{\pgfpoint{20.620900\du}{13.580893\du}}
\pgfusepath{fill}
\pgfsetbuttcap
\pgfsetmiterjoin
\pgfsetdash{}{0pt}
\definecolor{dialinecolor}{rgb}{0.678431, 0.839216, 0.905882}
\pgfsetfillcolor{dialinecolor}
\pgfpathmoveto{\pgfpoint{22.311664\du}{12.993452\du}}
\pgfpathlineto{\pgfpoint{22.311664\du}{12.993452\du}}
\pgfpathlineto{\pgfpoint{22.267852\du}{12.993452\du}}
\pgfpathlineto{\pgfpoint{22.225136\du}{12.994183\du}}
\pgfpathlineto{\pgfpoint{22.181689\du}{12.995278\du}}
\pgfpathlineto{\pgfpoint{22.139703\du}{12.996373\du}}
\pgfpathlineto{\pgfpoint{22.097352\du}{12.998199\du}}
\pgfpathlineto{\pgfpoint{22.055731\du}{13.000389\du}}
\pgfpathlineto{\pgfpoint{22.013745\du}{13.002945\du}}
\pgfpathlineto{\pgfpoint{21.972489\du}{13.005135\du}}
\pgfpathlineto{\pgfpoint{21.931598\du}{13.008056\du}}
\pgfpathlineto{\pgfpoint{21.890707\du}{13.011707\du}}
\pgfpathlineto{\pgfpoint{21.850911\du}{13.015358\du}}
\pgfpathlineto{\pgfpoint{21.810751\du}{13.019374\du}}
\pgfpathlineto{\pgfpoint{21.771685\du}{13.023755\du}}
\pgfpathlineto{\pgfpoint{21.732985\du}{13.028502\du}}
\pgfpathlineto{\pgfpoint{21.694285\du}{13.033613\du}}
\pgfpathlineto{\pgfpoint{21.656315\du}{13.039089\du}}
\pgfpathlineto{\pgfpoint{21.618710\du}{13.044201\du}}
\pgfpathlineto{\pgfpoint{21.581470\du}{13.050773\du}}
\pgfpathlineto{\pgfpoint{21.544960\du}{13.056614\du}}
\pgfpathlineto{\pgfpoint{21.508816\du}{13.063186\du}}
\pgfpathlineto{\pgfpoint{21.473036\du}{13.070123\du}}
\pgfpathlineto{\pgfpoint{21.438352\du}{13.077060\du}}
\pgfpathlineto{\pgfpoint{21.404033\du}{13.084727\du}}
\pgfpathlineto{\pgfpoint{21.369349\du}{13.092394\du}}
\pgfpathlineto{\pgfpoint{21.336855\du}{13.100426\du}}
\pgfpathlineto{\pgfpoint{21.303631\du}{13.108823\du}}
\pgfpathlineto{\pgfpoint{21.271868\du}{13.116855\du}}
\pgfpathlineto{\pgfpoint{21.240105\du}{13.125617\du}}
\pgfpathlineto{\pgfpoint{21.209436\du}{13.135110\du}}
\pgfpathlineto{\pgfpoint{21.178403\du}{13.143872\du}}
\pgfpathlineto{\pgfpoint{21.163799\du}{13.148618\du}}
\pgfpathlineto{\pgfpoint{21.149926\du}{13.153730\du}}
\pgfpathlineto{\pgfpoint{21.134227\du}{13.158476\du}}
\pgfpathlineto{\pgfpoint{21.120353\du}{13.163222\du}}
\pgfpathlineto{\pgfpoint{21.105749\du}{13.168334\du}}
\pgfpathlineto{\pgfpoint{21.091875\du}{13.173080\du}}
\pgfpathlineto{\pgfpoint{21.077637\du}{13.178191\du}}
\pgfpathlineto{\pgfpoint{21.064493\du}{13.182938\du}}
\pgfpathlineto{\pgfpoint{21.050254\du}{13.188414\du}}
\pgfpathlineto{\pgfpoint{21.037111\du}{13.193525\du}}
\pgfpathlineto{\pgfpoint{21.024697\du}{13.198637\du}}
\pgfpathlineto{\pgfpoint{21.010824\du}{13.204478\du}}
\pgfpathlineto{\pgfpoint{20.998045\du}{13.209955\du}}
\pgfpathlineto{\pgfpoint{20.985997\du}{13.215066\du}}
\pgfpathlineto{\pgfpoint{20.972854\du}{13.220908\du}}
\pgfpathlineto{\pgfpoint{20.960806\du}{13.226384\du}}
\pgfpathlineto{\pgfpoint{20.948392\du}{13.232226\du}}
\pgfpathlineto{\pgfpoint{20.936709\du}{13.237337\du}}
\pgfpathlineto{\pgfpoint{20.925391\du}{13.243179\du}}
\pgfpathlineto{\pgfpoint{20.913708\du}{13.249020\du}}
\pgfpathlineto{\pgfpoint{20.902390\du}{13.254862\du}}
\pgfpathlineto{\pgfpoint{20.890707\du}{13.260703\du}}
\pgfpathlineto{\pgfpoint{20.879754\du}{13.266545\du}}
\pgfpathlineto{\pgfpoint{20.869531\du}{13.273116\du}}
\pgfpathlineto{\pgfpoint{20.858944\du}{13.278958\du}}
\pgfpathlineto{\pgfpoint{20.848721\du}{13.284800\du}}
\pgfpathlineto{\pgfpoint{20.838498\du}{13.291371\du}}
\pgfpathlineto{\pgfpoint{20.828641\du}{13.297213\du}}
\pgfpathlineto{\pgfpoint{20.819148\du}{13.303420\du}}
\pgfpathlineto{\pgfpoint{20.809655\du}{13.309991\du}}
\pgfpathlineto{\pgfpoint{20.800528\du}{13.316563\du}}
\pgfpathlineto{\pgfpoint{20.790670\du}{13.322405\du}}
\pgfpathlineto{\pgfpoint{20.782273\du}{13.328611\du}}
\pgfpathlineto{\pgfpoint{20.773511\du}{13.335183\du}}
\pgfpathlineto{\pgfpoint{20.765114\du}{13.341390\du}}
\pgfpathlineto{\pgfpoint{20.756351\du}{13.348692\du}}
\pgfpathlineto{\pgfpoint{20.748684\du}{13.354898\du}}
\pgfpathlineto{\pgfpoint{20.741747\du}{13.361470\du}}
\pgfpathlineto{\pgfpoint{20.732985\du}{13.368407\du}}
\pgfpathlineto{\pgfpoint{20.726048\du}{13.374978\du}}
\pgfpathlineto{\pgfpoint{20.719111\du}{13.381915\du}}
\pgfpathlineto{\pgfpoint{20.712175\du}{13.388852\du}}
\pgfpathlineto{\pgfpoint{20.705603\du}{13.395424\du}}
\pgfpathlineto{\pgfpoint{20.698666\du}{13.402361\du}}
\pgfpathlineto{\pgfpoint{20.692825\du}{13.409298\du}}
\pgfpathlineto{\pgfpoint{20.686983\du}{13.416600\du}}
\pgfpathlineto{\pgfpoint{20.681141\du}{13.423536\du}}
\pgfpathlineto{\pgfpoint{20.675665\du}{13.430473\du}}
\pgfpathlineto{\pgfpoint{20.670919\du}{13.437410\du}}
\pgfpathlineto{\pgfpoint{20.665442\du}{13.445077\du}}
\pgfpathlineto{\pgfpoint{20.661426\du}{13.452014\du}}
\pgfpathlineto{\pgfpoint{20.656315\du}{13.459316\du}}
\pgfpathlineto{\pgfpoint{20.652664\du}{13.466618\du}}
\pgfpathlineto{\pgfpoint{20.648283\du}{13.473920\du}}
\pgfpathlineto{\pgfpoint{20.644632\du}{13.481587\du}}
\pgfpathlineto{\pgfpoint{20.641346\du}{13.488889\du}}
\pgfpathlineto{\pgfpoint{20.637695\du}{13.496556\du}}
\pgfpathlineto{\pgfpoint{20.634774\du}{13.504223\du}}
\pgfpathlineto{\pgfpoint{20.632218\du}{13.511160\du}}
\pgfpathlineto{\pgfpoint{20.630028\du}{13.518827\du}}
\pgfpathlineto{\pgfpoint{20.627837\du}{13.526494\du}}
\pgfpathlineto{\pgfpoint{20.626012\du}{13.534526\du}}
\pgfpathlineto{\pgfpoint{20.624186\du}{13.542193\du}}
\pgfpathlineto{\pgfpoint{20.623091\du}{13.549860\du}}
\pgfpathlineto{\pgfpoint{20.621996\du}{13.557527\du}}
\pgfpathlineto{\pgfpoint{20.621266\du}{13.565559\du}}
\pgfpathlineto{\pgfpoint{20.620900\du}{13.573226\du}}
\pgfpathlineto{\pgfpoint{20.620900\du}{13.580893\du}}
\pgfpathlineto{\pgfpoint{20.641346\du}{13.580893\du}}
\pgfpathlineto{\pgfpoint{20.641711\du}{13.573956\du}}
\pgfpathlineto{\pgfpoint{20.641711\du}{13.567019\du}}
\pgfpathlineto{\pgfpoint{20.642076\du}{13.559717\du}}
\pgfpathlineto{\pgfpoint{20.643536\du}{13.552781\du}}
\pgfpathlineto{\pgfpoint{20.644632\du}{13.545844\du}}
\pgfpathlineto{\pgfpoint{20.645727\du}{13.538907\du}}
\pgfpathlineto{\pgfpoint{20.647918\du}{13.531605\du}}
\pgfpathlineto{\pgfpoint{20.649378\du}{13.525398\du}}
\pgfpathlineto{\pgfpoint{20.651569\du}{13.518096\du}}
\pgfpathlineto{\pgfpoint{20.653759\du}{13.511160\du}}
\pgfpathlineto{\pgfpoint{20.657045\du}{13.504223\du}}
\pgfpathlineto{\pgfpoint{20.659601\du}{13.497286\du}}
\pgfpathlineto{\pgfpoint{20.662521\du}{13.490349\du}}
\pgfpathlineto{\pgfpoint{20.666538\du}{13.483777\du}}
\pgfpathlineto{\pgfpoint{20.670554\du}{13.476840\du}}
\pgfpathlineto{\pgfpoint{20.673839\du}{13.470269\du}}
\pgfpathlineto{\pgfpoint{20.677856\du}{13.463332\du}}
\pgfpathlineto{\pgfpoint{20.682602\du}{13.456395\du}}
\pgfpathlineto{\pgfpoint{20.686983\du}{13.449823\du}}
\pgfpathlineto{\pgfpoint{20.692459\du}{13.443252\du}}
\pgfpathlineto{\pgfpoint{20.696841\du}{13.436315\du}}
\pgfpathlineto{\pgfpoint{20.703047\du}{13.429743\du}}
\pgfpathlineto{\pgfpoint{20.708159\du}{13.422806\du}}
\pgfpathlineto{\pgfpoint{20.714365\du}{13.416600\du}}
\pgfpathlineto{\pgfpoint{20.720207\du}{13.410028\du}}
\pgfpathlineto{\pgfpoint{20.726779\du}{13.403091\du}}
\pgfpathlineto{\pgfpoint{20.732985\du}{13.396519\du}}
\pgfpathlineto{\pgfpoint{20.739922\du}{13.390313\du}}
\pgfpathlineto{\pgfpoint{20.747224\du}{13.383741\du}}
\pgfpathlineto{\pgfpoint{20.754161\du}{13.377169\du}}
\pgfpathlineto{\pgfpoint{20.761828\du}{13.370962\du}}
\pgfpathlineto{\pgfpoint{20.769860\du}{13.364391\du}}
\pgfpathlineto{\pgfpoint{20.777162\du}{13.357819\du}}
\pgfpathlineto{\pgfpoint{20.785559\du}{13.351612\du}}
\pgfpathlineto{\pgfpoint{20.794321\du}{13.345771\du}}
\pgfpathlineto{\pgfpoint{20.803449\du}{13.338834\du}}
\pgfpathlineto{\pgfpoint{20.812211\du}{13.333357\du}}
\pgfpathlineto{\pgfpoint{20.820608\du}{13.326786\du}}
\pgfpathlineto{\pgfpoint{20.829736\du}{13.320579\du}}
\pgfpathlineto{\pgfpoint{20.839593\du}{13.314738\du}}
\pgfpathlineto{\pgfpoint{20.849086\du}{13.308896\du}}
\pgfpathlineto{\pgfpoint{20.858944\du}{13.302324\du}}
\pgfpathlineto{\pgfpoint{20.869531\du}{13.296483\du}}
\pgfpathlineto{\pgfpoint{20.879389\du}{13.290641\du}}
\pgfpathlineto{\pgfpoint{20.889977\du}{13.284800\du}}
\pgfpathlineto{\pgfpoint{20.901295\du}{13.278958\du}}
\pgfpathlineto{\pgfpoint{20.911152\du}{13.273116\du}}
\pgfpathlineto{\pgfpoint{20.922835\du}{13.267275\du}}
\pgfpathlineto{\pgfpoint{20.933788\du}{13.262164\du}}
\pgfpathlineto{\pgfpoint{20.945471\du}{13.255957\du}}
\pgfpathlineto{\pgfpoint{20.957155\du}{13.250115\du}}
\pgfpathlineto{\pgfpoint{20.969933\du}{13.245004\du}}
\pgfpathlineto{\pgfpoint{20.981251\du}{13.239893\du}}
\pgfpathlineto{\pgfpoint{20.993299\du}{13.234051\du}}
\pgfpathlineto{\pgfpoint{21.006443\du}{13.228575\du}}
\pgfpathlineto{\pgfpoint{21.018856\du}{13.223463\du}}
\pgfpathlineto{\pgfpoint{21.031634\du}{13.217987\du}}
\pgfpathlineto{\pgfpoint{21.045143\du}{13.212876\du}}
\pgfpathlineto{\pgfpoint{21.057921\du}{13.207399\du}}
\pgfpathlineto{\pgfpoint{21.071430\du}{13.201923\du}}
\pgfpathlineto{\pgfpoint{21.084938\du}{13.197541\du}}
\pgfpathlineto{\pgfpoint{21.098812\du}{13.192430\du}}
\pgfpathlineto{\pgfpoint{21.112686\du}{13.187684\du}}
\pgfpathlineto{\pgfpoint{21.126925\du}{13.182572\du}}
\pgfpathlineto{\pgfpoint{21.141163\du}{13.177826\du}}
\pgfpathlineto{\pgfpoint{21.155402\du}{13.173080\du}}
\pgfpathlineto{\pgfpoint{21.169641\du}{13.168334\du}}
\pgfpathlineto{\pgfpoint{21.184975\du}{13.163587\du}}
\pgfpathlineto{\pgfpoint{21.214548\du}{13.154460\du}}
\pgfpathlineto{\pgfpoint{21.245946\du}{13.145698\du}}
\pgfpathlineto{\pgfpoint{21.276979\du}{13.136935\du}}
\pgfpathlineto{\pgfpoint{21.309473\du}{13.128538\du}}
\pgfpathlineto{\pgfpoint{21.340871\du}{13.120506\du}}
\pgfpathlineto{\pgfpoint{21.374095\du}{13.112109\du}}
\pgfpathlineto{\pgfpoint{21.408049\du}{13.104442\du}}
\pgfpathlineto{\pgfpoint{21.442368\du}{13.097505\du}}
\pgfpathlineto{\pgfpoint{21.477052\du}{13.090568\du}}
\pgfpathlineto{\pgfpoint{21.513197\du}{13.083266\du}}
\pgfpathlineto{\pgfpoint{21.548246\du}{13.077060\du}}
\pgfpathlineto{\pgfpoint{21.585486\du}{13.070123\du}}
\pgfpathlineto{\pgfpoint{21.621996\du}{13.064646\du}}
\pgfpathlineto{\pgfpoint{21.659236\du}{13.058805\du}}
\pgfpathlineto{\pgfpoint{21.696841\du}{13.053693\du}}
\pgfpathlineto{\pgfpoint{21.735176\du}{13.048947\du}}
\pgfpathlineto{\pgfpoint{21.774241\du}{13.044201\du}}
\pgfpathlineto{\pgfpoint{21.813672\du}{13.039820\du}}
\pgfpathlineto{\pgfpoint{21.853102\du}{13.036169\du}}
\pgfpathlineto{\pgfpoint{21.892898\du}{13.032153\du}}
\pgfpathlineto{\pgfpoint{21.933058\du}{13.029232\du}}
\pgfpathlineto{\pgfpoint{21.973949\du}{13.025581\du}}
\pgfpathlineto{\pgfpoint{22.015205\du}{13.022660\du}}
\pgfpathlineto{\pgfpoint{22.056096\du}{13.020470\du}}
\pgfpathlineto{\pgfpoint{22.098082\du}{13.018644\du}}
\pgfpathlineto{\pgfpoint{22.140433\du}{13.016819\du}}
\pgfpathlineto{\pgfpoint{22.183149\du}{13.015358\du}}
\pgfpathlineto{\pgfpoint{22.225501\du}{13.014993\du}}
\pgfpathlineto{\pgfpoint{22.268947\du}{13.014628\du}}
\pgfpathlineto{\pgfpoint{22.311664\du}{13.013898\du}}
\pgfpathlineto{\pgfpoint{22.311664\du}{13.013898\du}}
\pgfpathlineto{\pgfpoint{22.311664\du}{13.013898\du}}
\pgfpathlineto{\pgfpoint{22.312759\du}{13.013898\du}}
\pgfpathlineto{\pgfpoint{22.313854\du}{13.013898\du}}
\pgfpathlineto{\pgfpoint{22.315314\du}{13.013168\du}}
\pgfpathlineto{\pgfpoint{22.316410\du}{13.013168\du}}
\pgfpathlineto{\pgfpoint{22.316775\du}{13.012803\du}}
\pgfpathlineto{\pgfpoint{22.317870\du}{13.012072\du}}
\pgfpathlineto{\pgfpoint{22.318600\du}{13.011707\du}}
\pgfpathlineto{\pgfpoint{22.319696\du}{13.010977\du}}
\pgfpathlineto{\pgfpoint{22.320791\du}{13.009152\du}}
\pgfpathlineto{\pgfpoint{22.321521\du}{13.007326\du}}
\pgfpathlineto{\pgfpoint{22.321521\du}{13.005866\du}}
\pgfpathlineto{\pgfpoint{22.322251\du}{13.004040\du}}
\pgfpathlineto{\pgfpoint{22.321521\du}{13.002215\du}}
\pgfpathlineto{\pgfpoint{22.321521\du}{13.000024\du}}
\pgfpathlineto{\pgfpoint{22.320791\du}{12.998199\du}}
\pgfpathlineto{\pgfpoint{22.319696\du}{12.996373\du}}
\pgfpathlineto{\pgfpoint{22.318600\du}{12.995643\du}}
\pgfpathlineto{\pgfpoint{22.317870\du}{12.995278\du}}
\pgfpathlineto{\pgfpoint{22.316775\du}{12.994548\du}}
\pgfpathlineto{\pgfpoint{22.316410\du}{12.994183\du}}
\pgfpathlineto{\pgfpoint{22.315314\du}{12.994183\du}}
\pgfpathlineto{\pgfpoint{22.313854\du}{12.993452\du}}
\pgfpathlineto{\pgfpoint{22.312759\du}{12.993452\du}}
\pgfpathlineto{\pgfpoint{22.311664\du}{12.993452\du}}
\pgfusepath{fill}
\pgfsetbuttcap
\pgfsetmiterjoin
\pgfsetdash{}{0pt}
\definecolor{dialinecolor}{rgb}{0.678431, 0.839216, 0.905882}
\pgfsetfillcolor{dialinecolor}
\pgfpathmoveto{\pgfpoint{24.002061\du}{13.580893\du}}
\pgfpathlineto{\pgfpoint{24.002061\du}{13.573226\du}}
\pgfpathlineto{\pgfpoint{24.001331\du}{13.565559\du}}
\pgfpathlineto{\pgfpoint{24.000601\du}{13.557527\du}}
\pgfpathlineto{\pgfpoint{23.999871\du}{13.549860\du}}
\pgfpathlineto{\pgfpoint{23.998776\du}{13.542193\du}}
\pgfpathlineto{\pgfpoint{23.996585\du}{13.534526\du}}
\pgfpathlineto{\pgfpoint{23.995490\du}{13.526494\du}}
\pgfpathlineto{\pgfpoint{23.992569\du}{13.518827\du}}
\pgfpathlineto{\pgfpoint{23.990743\du}{13.511160\du}}
\pgfpathlineto{\pgfpoint{23.987823\du}{13.504223\du}}
\pgfpathlineto{\pgfpoint{23.985267\du}{13.496556\du}}
\pgfpathlineto{\pgfpoint{23.981981\du}{13.488889\du}}
\pgfpathlineto{\pgfpoint{23.977965\du}{13.481587\du}}
\pgfpathlineto{\pgfpoint{23.974679\du}{13.473920\du}}
\pgfpathlineto{\pgfpoint{23.970663\du}{13.466618\du}}
\pgfpathlineto{\pgfpoint{23.966647\du}{13.459316\du}}
\pgfpathlineto{\pgfpoint{23.962266\du}{13.452014\du}}
\pgfpathlineto{\pgfpoint{23.957520\du}{13.445077\du}}
\pgfpathlineto{\pgfpoint{23.952043\du}{13.437410\du}}
\pgfpathlineto{\pgfpoint{23.947297\du}{13.430473\du}}
\pgfpathlineto{\pgfpoint{23.941455\du}{13.423536\du}}
\pgfpathlineto{\pgfpoint{23.935979\du}{13.416600\du}}
\pgfpathlineto{\pgfpoint{23.930137\du}{13.409298\du}}
\pgfpathlineto{\pgfpoint{23.923931\du}{13.402361\du}}
\pgfpathlineto{\pgfpoint{23.917359\du}{13.395424\du}}
\pgfpathlineto{\pgfpoint{23.910422\du}{13.388852\du}}
\pgfpathlineto{\pgfpoint{23.903850\du}{13.381915\du}}
\pgfpathlineto{\pgfpoint{23.896549\du}{13.374978\du}}
\pgfpathlineto{\pgfpoint{23.889977\du}{13.368407\du}}
\pgfpathlineto{\pgfpoint{23.881945\du}{13.361470\du}}
\pgfpathlineto{\pgfpoint{23.873913\du}{13.354898\du}}
\pgfpathlineto{\pgfpoint{23.866611\du}{13.348692\du}}
\pgfpathlineto{\pgfpoint{23.857848\du}{13.341390\du}}
\pgfpathlineto{\pgfpoint{23.849816\du}{13.335183\du}}
\pgfpathlineto{\pgfpoint{23.840689\du}{13.328611\du}}
\pgfpathlineto{\pgfpoint{23.831926\du}{13.322405\du}}
\pgfpathlineto{\pgfpoint{23.822434\du}{13.316563\du}}
\pgfpathlineto{\pgfpoint{23.813306\du}{13.309991\du}}
\pgfpathlineto{\pgfpoint{23.804179\du}{13.303420\du}}
\pgfpathlineto{\pgfpoint{23.794321\du}{13.297213\du}}
\pgfpathlineto{\pgfpoint{23.784464\du}{13.291371\du}}
\pgfpathlineto{\pgfpoint{23.774241\du}{13.284800\du}}
\pgfpathlineto{\pgfpoint{23.763653\du}{13.278958\du}}
\pgfpathlineto{\pgfpoint{23.753796\du}{13.273116\du}}
\pgfpathlineto{\pgfpoint{23.743208\du}{13.266545\du}}
\pgfpathlineto{\pgfpoint{23.731890\du}{13.260703\du}}
\pgfpathlineto{\pgfpoint{23.720572\du}{13.254862\du}}
\pgfpathlineto{\pgfpoint{23.708889\du}{13.249020\du}}
\pgfpathlineto{\pgfpoint{23.697936\du}{13.243179\du}}
\pgfpathlineto{\pgfpoint{23.685888\du}{13.237337\du}}
\pgfpathlineto{\pgfpoint{23.674570\du}{13.232226\du}}
\pgfpathlineto{\pgfpoint{23.662156\du}{13.226384\du}}
\pgfpathlineto{\pgfpoint{23.649743\du}{13.220908\du}}
\pgfpathlineto{\pgfpoint{23.636965\du}{13.215066\du}}
\pgfpathlineto{\pgfpoint{23.624917\du}{13.209955\du}}
\pgfpathlineto{\pgfpoint{23.612138\du}{13.204478\du}}
\pgfpathlineto{\pgfpoint{23.598630\du}{13.198637\du}}
\pgfpathlineto{\pgfpoint{23.585486\du}{13.193525\du}}
\pgfpathlineto{\pgfpoint{23.572343\du}{13.188414\du}}
\pgfpathlineto{\pgfpoint{23.558104\du}{13.182938\du}}
\pgfpathlineto{\pgfpoint{23.544960\du}{13.178191\du}}
\pgfpathlineto{\pgfpoint{23.530722\du}{13.173080\du}}
\pgfpathlineto{\pgfpoint{23.516848\du}{13.168334\du}}
\pgfpathlineto{\pgfpoint{23.502609\du}{13.163222\du}}
\pgfpathlineto{\pgfpoint{23.488735\du}{13.158476\du}}
\pgfpathlineto{\pgfpoint{23.473766\du}{13.153730\du}}
\pgfpathlineto{\pgfpoint{23.458798\du}{13.148618\du}}
\pgfpathlineto{\pgfpoint{23.444559\du}{13.143872\du}}
\pgfpathlineto{\pgfpoint{23.413891\du}{13.135110\du}}
\pgfpathlineto{\pgfpoint{23.383222\du}{13.125617\du}}
\pgfpathlineto{\pgfpoint{23.351824\du}{13.116855\du}}
\pgfpathlineto{\pgfpoint{23.320061\du}{13.108823\du}}
\pgfpathlineto{\pgfpoint{23.287202\du}{13.100426\du}}
\pgfpathlineto{\pgfpoint{23.253613\du}{13.092394\du}}
\pgfpathlineto{\pgfpoint{23.219659\du}{13.084727\du}}
\pgfpathlineto{\pgfpoint{23.184975\du}{13.077060\du}}
\pgfpathlineto{\pgfpoint{23.150291\du}{13.070123\du}}
\pgfpathlineto{\pgfpoint{23.114876\du}{13.063186\du}}
\pgfpathlineto{\pgfpoint{23.078367\du}{13.056614\du}}
\pgfpathlineto{\pgfpoint{23.041857\du}{13.050773\du}}
\pgfpathlineto{\pgfpoint{23.004982\du}{13.044201\du}}
\pgfpathlineto{\pgfpoint{22.967742\du}{13.039089\du}}
\pgfpathlineto{\pgfpoint{22.928677\du}{13.033613\du}}
\pgfpathlineto{\pgfpoint{22.890707\du}{13.028502\du}}
\pgfpathlineto{\pgfpoint{22.851277\du}{13.023755\du}}
\pgfpathlineto{\pgfpoint{22.812576\du}{13.019374\du}}
\pgfpathlineto{\pgfpoint{22.772416\du}{13.015358\du}}
\pgfpathlineto{\pgfpoint{22.732620\du}{13.011707\du}}
\pgfpathlineto{\pgfpoint{22.691729\du}{13.008056\du}}
\pgfpathlineto{\pgfpoint{22.651203\du}{13.005135\du}}
\pgfpathlineto{\pgfpoint{22.609582\du}{13.002945\du}}
\pgfpathlineto{\pgfpoint{22.567961\du}{13.000389\du}}
\pgfpathlineto{\pgfpoint{22.526340\du}{12.998199\du}}
\pgfpathlineto{\pgfpoint{22.483624\du}{12.996373\du}}
\pgfpathlineto{\pgfpoint{22.441638\du}{12.995278\du}}
\pgfpathlineto{\pgfpoint{22.398557\du}{12.994183\du}}
\pgfpathlineto{\pgfpoint{22.355110\du}{12.993452\du}}
\pgfpathlineto{\pgfpoint{22.311664\du}{12.993452\du}}
\pgfpathlineto{\pgfpoint{22.311664\du}{13.013898\du}}
\pgfpathlineto{\pgfpoint{22.354745\du}{13.014628\du}}
\pgfpathlineto{\pgfpoint{22.398191\du}{13.014993\du}}
\pgfpathlineto{\pgfpoint{22.440178\du}{13.015358\du}}
\pgfpathlineto{\pgfpoint{22.482894\du}{13.016819\du}}
\pgfpathlineto{\pgfpoint{22.525610\du}{13.018644\du}}
\pgfpathlineto{\pgfpoint{22.567231\du}{13.020470\du}}
\pgfpathlineto{\pgfpoint{22.608487\du}{13.022660\du}}
\pgfpathlineto{\pgfpoint{22.649378\du}{13.025581\du}}
\pgfpathlineto{\pgfpoint{22.690634\du}{13.029232\du}}
\pgfpathlineto{\pgfpoint{22.730795\du}{13.032153\du}}
\pgfpathlineto{\pgfpoint{22.770955\du}{13.036169\du}}
\pgfpathlineto{\pgfpoint{22.810021\du}{13.039820\du}}
\pgfpathlineto{\pgfpoint{22.849451\du}{13.044201\du}}
\pgfpathlineto{\pgfpoint{22.888151\du}{13.048947\du}}
\pgfpathlineto{\pgfpoint{22.926852\du}{13.053693\du}}
\pgfpathlineto{\pgfpoint{22.964457\du}{13.058805\du}}
\pgfpathlineto{\pgfpoint{23.001696\du}{13.064646\du}}
\pgfpathlineto{\pgfpoint{23.038206\du}{13.070123\du}}
\pgfpathlineto{\pgfpoint{23.075081\du}{13.077060\du}}
\pgfpathlineto{\pgfpoint{23.110495\du}{13.083266\du}}
\pgfpathlineto{\pgfpoint{23.146275\du}{13.090568\du}}
\pgfpathlineto{\pgfpoint{23.180959\du}{13.097505\du}}
\pgfpathlineto{\pgfpoint{23.214913\du}{13.104442\du}}
\pgfpathlineto{\pgfpoint{23.249232\du}{13.112109\du}}
\pgfpathlineto{\pgfpoint{23.282456\du}{13.120506\du}}
\pgfpathlineto{\pgfpoint{23.314584\du}{13.128538\du}}
\pgfpathlineto{\pgfpoint{23.346713\du}{13.136935\du}}
\pgfpathlineto{\pgfpoint{23.378111\du}{13.145698\du}}
\pgfpathlineto{\pgfpoint{23.408779\du}{13.154460\du}}
\pgfpathlineto{\pgfpoint{23.438717\du}{13.163587\du}}
\pgfpathlineto{\pgfpoint{23.452956\du}{13.168334\du}}
\pgfpathlineto{\pgfpoint{23.467560\du}{13.173080\du}}
\pgfpathlineto{\pgfpoint{23.481434\du}{13.177826\du}}
\pgfpathlineto{\pgfpoint{23.496037\du}{13.182572\du}}
\pgfpathlineto{\pgfpoint{23.510276\du}{13.187684\du}}
\pgfpathlineto{\pgfpoint{23.524150\du}{13.192430\du}}
\pgfpathlineto{\pgfpoint{23.538389\du}{13.197541\du}}
\pgfpathlineto{\pgfpoint{23.551532\du}{13.201923\du}}
\pgfpathlineto{\pgfpoint{23.565041\du}{13.207399\du}}
\pgfpathlineto{\pgfpoint{23.578184\du}{13.212876\du}}
\pgfpathlineto{\pgfpoint{23.590963\du}{13.217987\du}}
\pgfpathlineto{\pgfpoint{23.603741\du}{13.223463\du}}
\pgfpathlineto{\pgfpoint{23.616884\du}{13.228575\du}}
\pgfpathlineto{\pgfpoint{23.628933\du}{13.234051\du}}
\pgfpathlineto{\pgfpoint{23.641346\du}{13.239893\du}}
\pgfpathlineto{\pgfpoint{23.653394\du}{13.245004\du}}
\pgfpathlineto{\pgfpoint{23.665807\du}{13.250115\du}}
\pgfpathlineto{\pgfpoint{23.677125\du}{13.255957\du}}
\pgfpathlineto{\pgfpoint{23.689174\du}{13.262164\du}}
\pgfpathlineto{\pgfpoint{23.699761\du}{13.267275\du}}
\pgfpathlineto{\pgfpoint{23.711810\du}{13.273116\du}}
\pgfpathlineto{\pgfpoint{23.722032\du}{13.278958\du}}
\pgfpathlineto{\pgfpoint{23.732985\du}{13.284800\du}}
\pgfpathlineto{\pgfpoint{23.743938\du}{13.290641\du}}
\pgfpathlineto{\pgfpoint{23.753796\du}{13.296483\du}}
\pgfpathlineto{\pgfpoint{23.763653\du}{13.302324\du}}
\pgfpathlineto{\pgfpoint{23.773876\du}{13.308896\du}}
\pgfpathlineto{\pgfpoint{23.783003\du}{13.314738\du}}
\pgfpathlineto{\pgfpoint{23.793226\du}{13.320579\du}}
\pgfpathlineto{\pgfpoint{23.802719\du}{13.326786\du}}
\pgfpathlineto{\pgfpoint{23.811481\du}{13.333357\du}}
\pgfpathlineto{\pgfpoint{23.819513\du}{13.338834\du}}
\pgfpathlineto{\pgfpoint{23.828275\du}{13.345771\du}}
\pgfpathlineto{\pgfpoint{23.836673\du}{13.351612\du}}
\pgfpathlineto{\pgfpoint{23.845435\du}{13.357819\du}}
\pgfpathlineto{\pgfpoint{23.853102\du}{13.364391\du}}
\pgfpathlineto{\pgfpoint{23.861134\du}{13.370962\du}}
\pgfpathlineto{\pgfpoint{23.868436\du}{13.377169\du}}
\pgfpathlineto{\pgfpoint{23.876103\du}{13.383741\du}}
\pgfpathlineto{\pgfpoint{23.882675\du}{13.390313\du}}
\pgfpathlineto{\pgfpoint{23.889977\du}{13.396519\du}}
\pgfpathlineto{\pgfpoint{23.895818\du}{13.403091\du}}
\pgfpathlineto{\pgfpoint{23.902755\du}{13.410028\du}}
\pgfpathlineto{\pgfpoint{23.909327\du}{13.416600\du}}
\pgfpathlineto{\pgfpoint{23.914438\du}{13.422806\du}}
\pgfpathlineto{\pgfpoint{23.919915\du}{13.429743\du}}
\pgfpathlineto{\pgfpoint{23.926121\du}{13.436315\du}}
\pgfpathlineto{\pgfpoint{23.930868\du}{13.443252\du}}
\pgfpathlineto{\pgfpoint{23.935979\du}{13.449823\du}}
\pgfpathlineto{\pgfpoint{23.940725\du}{13.456395\du}}
\pgfpathlineto{\pgfpoint{23.945106\du}{13.463332\du}}
\pgfpathlineto{\pgfpoint{23.949488\du}{13.470269\du}}
\pgfpathlineto{\pgfpoint{23.952408\du}{13.476840\du}}
\pgfpathlineto{\pgfpoint{23.956424\du}{13.483777\du}}
\pgfpathlineto{\pgfpoint{23.959345\du}{13.490349\du}}
\pgfpathlineto{\pgfpoint{23.963361\du}{13.497286\du}}
\pgfpathlineto{\pgfpoint{23.965917\du}{13.504223\du}}
\pgfpathlineto{\pgfpoint{23.968838\du}{13.511160\du}}
\pgfpathlineto{\pgfpoint{23.971393\du}{13.518096\du}}
\pgfpathlineto{\pgfpoint{23.973584\du}{13.525398\du}}
\pgfpathlineto{\pgfpoint{23.975044\du}{13.531605\du}}
\pgfpathlineto{\pgfpoint{23.977235\du}{13.538907\du}}
\pgfpathlineto{\pgfpoint{23.977965\du}{13.545844\du}}
\pgfpathlineto{\pgfpoint{23.979060\du}{13.552781\du}}
\pgfpathlineto{\pgfpoint{23.980886\du}{13.559717\du}}
\pgfpathlineto{\pgfpoint{23.981251\du}{13.567019\du}}
\pgfpathlineto{\pgfpoint{23.981251\du}{13.573956\du}}
\pgfpathlineto{\pgfpoint{23.981981\du}{13.580893\du}}
\pgfpathlineto{\pgfpoint{24.002061\du}{13.580893\du}}
\pgfusepath{fill}
\pgfsetbuttcap
\pgfsetmiterjoin
\pgfsetdash{}{0pt}
\definecolor{dialinecolor}{rgb}{0.074510, 0.082353, 0.086275}
\pgfsetfillcolor{dialinecolor}
\pgfpathmoveto{\pgfpoint{22.354745\du}{13.453474\du}}
\pgfpathlineto{\pgfpoint{22.602646\du}{13.535986\du}}
\pgfpathlineto{\pgfpoint{23.187896\du}{13.301594\du}}
\pgfpathlineto{\pgfpoint{23.460623\du}{13.369137\du}}
\pgfpathlineto{\pgfpoint{23.316775\du}{13.160667\du}}
\pgfpathlineto{\pgfpoint{22.612503\du}{13.160667\du}}
\pgfpathlineto{\pgfpoint{22.906771\du}{13.233321\du}}
\pgfpathlineto{\pgfpoint{22.354745\du}{13.453474\du}}
\pgfusepath{fill}
\pgfsetbuttcap
\pgfsetmiterjoin
\pgfsetdash{}{0pt}
\definecolor{dialinecolor}{rgb}{0.074510, 0.082353, 0.086275}
\pgfsetfillcolor{dialinecolor}
\pgfpathmoveto{\pgfpoint{22.252883\du}{13.691517\du}}
\pgfpathlineto{\pgfpoint{22.004982\du}{13.609371\du}}
\pgfpathlineto{\pgfpoint{21.419732\du}{13.843033\du}}
\pgfpathlineto{\pgfpoint{21.146640\du}{13.776220\du}}
\pgfpathlineto{\pgfpoint{21.290488\du}{13.983960\du}}
\pgfpathlineto{\pgfpoint{21.995855\du}{13.983960\du}}
\pgfpathlineto{\pgfpoint{21.700857\du}{13.912036\du}}
\pgfpathlineto{\pgfpoint{22.252883\du}{13.691517\du}}
\pgfusepath{fill}
\pgfsetbuttcap
\pgfsetmiterjoin
\pgfsetdash{}{0pt}
\definecolor{dialinecolor}{rgb}{0.074510, 0.082353, 0.086275}
\pgfsetfillcolor{dialinecolor}
\pgfpathmoveto{\pgfpoint{21.206881\du}{13.232591\du}}
\pgfpathlineto{\pgfpoint{21.454416\du}{13.150809\du}}
\pgfpathlineto{\pgfpoint{22.039666\du}{13.384106\du}}
\pgfpathlineto{\pgfpoint{22.312759\du}{13.317658\du}}
\pgfpathlineto{\pgfpoint{22.168911\du}{13.525398\du}}
\pgfpathlineto{\pgfpoint{21.463909\du}{13.525398\du}}
\pgfpathlineto{\pgfpoint{21.758907\du}{13.453474\du}}
\pgfpathlineto{\pgfpoint{21.206881\du}{13.232591\du}}
\pgfusepath{fill}
\pgfsetbuttcap
\pgfsetmiterjoin
\pgfsetdash{}{0pt}
\definecolor{dialinecolor}{rgb}{0.074510, 0.082353, 0.086275}
\pgfsetfillcolor{dialinecolor}
\pgfpathmoveto{\pgfpoint{23.425209\du}{13.927735\du}}
\pgfpathlineto{\pgfpoint{23.177673\du}{14.009882\du}}
\pgfpathlineto{\pgfpoint{22.592423\du}{13.776220\du}}
\pgfpathlineto{\pgfpoint{22.318600\du}{13.843033\du}}
\pgfpathlineto{\pgfpoint{22.463179\du}{13.635292\du}}
\pgfpathlineto{\pgfpoint{23.168546\du}{13.635292\du}}
\pgfpathlineto{\pgfpoint{22.873182\du}{13.707217\du}}
\pgfpathlineto{\pgfpoint{23.425209\du}{13.927735\du}}
\pgfusepath{fill}
\pgfsetbuttcap
\pgfsetmiterjoin
\pgfsetdash{}{0pt}
\definecolor{dialinecolor}{rgb}{1.000000, 1.000000, 1.000000}
\pgfsetfillcolor{dialinecolor}
\pgfpathmoveto{\pgfpoint{22.375555\du}{13.473920\du}}
\pgfpathlineto{\pgfpoint{22.623091\du}{13.556432\du}}
\pgfpathlineto{\pgfpoint{23.208341\du}{13.322405\du}}
\pgfpathlineto{\pgfpoint{23.480703\du}{13.389582\du}}
\pgfpathlineto{\pgfpoint{23.337950\du}{13.181112\du}}
\pgfpathlineto{\pgfpoint{22.632584\du}{13.181112\du}}
\pgfpathlineto{\pgfpoint{22.927582\du}{13.253766\du}}
\pgfpathlineto{\pgfpoint{22.375555\du}{13.473920\du}}
\pgfusepath{fill}
\pgfsetbuttcap
\pgfsetmiterjoin
\pgfsetdash{}{0pt}
\definecolor{dialinecolor}{rgb}{1.000000, 1.000000, 1.000000}
\pgfsetfillcolor{dialinecolor}
\pgfpathmoveto{\pgfpoint{22.273693\du}{13.712693\du}}
\pgfpathlineto{\pgfpoint{22.025063\du}{13.630181\du}}
\pgfpathlineto{\pgfpoint{21.440178\du}{13.864208\du}}
\pgfpathlineto{\pgfpoint{21.166720\du}{13.796665\du}}
\pgfpathlineto{\pgfpoint{21.311664\du}{14.005135\du}}
\pgfpathlineto{\pgfpoint{22.016300\du}{14.005135\du}}
\pgfpathlineto{\pgfpoint{21.721667\du}{13.932481\du}}
\pgfpathlineto{\pgfpoint{22.273693\du}{13.712693\du}}
\pgfusepath{fill}
\pgfsetbuttcap
\pgfsetmiterjoin
\pgfsetdash{}{0pt}
\definecolor{dialinecolor}{rgb}{1.000000, 1.000000, 1.000000}
\pgfsetfillcolor{dialinecolor}
\pgfpathmoveto{\pgfpoint{21.227326\du}{13.253036\du}}
\pgfpathlineto{\pgfpoint{21.474862\du}{13.171254\du}}
\pgfpathlineto{\pgfpoint{22.060477\du}{13.405282\du}}
\pgfpathlineto{\pgfpoint{22.333569\du}{13.338469\du}}
\pgfpathlineto{\pgfpoint{22.188626\du}{13.545844\du}}
\pgfpathlineto{\pgfpoint{21.484354\du}{13.545844\du}}
\pgfpathlineto{\pgfpoint{21.778987\du}{13.473920\du}}
\pgfpathlineto{\pgfpoint{21.227326\du}{13.253036\du}}
\pgfusepath{fill}
\pgfsetbuttcap
\pgfsetmiterjoin
\pgfsetdash{}{0pt}
\definecolor{dialinecolor}{rgb}{1.000000, 1.000000, 1.000000}
\pgfsetfillcolor{dialinecolor}
\pgfpathmoveto{\pgfpoint{23.445289\du}{13.948180\du}}
\pgfpathlineto{\pgfpoint{23.197753\du}{14.030327\du}}
\pgfpathlineto{\pgfpoint{22.612868\du}{13.796665\du}}
\pgfpathlineto{\pgfpoint{22.339411\du}{13.863478\du}}
\pgfpathlineto{\pgfpoint{22.483624\du}{13.655738\du}}
\pgfpathlineto{\pgfpoint{23.188626\du}{13.655738\du}}
\pgfpathlineto{\pgfpoint{22.894358\du}{13.727662\du}}
\pgfpathlineto{\pgfpoint{23.445289\du}{13.948180\du}}
\pgfusepath{fill}
\pgfsetbuttcap
\pgfsetmiterjoin
\pgfsetdash{}{0pt}
\definecolor{dialinecolor}{rgb}{0.678431, 0.839216, 0.905882}
\pgfsetfillcolor{dialinecolor}
\pgfpathmoveto{\pgfpoint{20.641346\du}{13.591481\du}}
\pgfpathlineto{\pgfpoint{20.641346\du}{13.580893\du}}
\pgfpathlineto{\pgfpoint{20.620900\du}{13.580893\du}}
\pgfpathlineto{\pgfpoint{20.620900\du}{13.591481\du}}
\pgfpathlineto{\pgfpoint{20.641346\du}{13.591481\du}}
\pgfusepath{fill}
\pgfsetbuttcap
\pgfsetmiterjoin
\pgfsetdash{}{0pt}
\definecolor{dialinecolor}{rgb}{0.678431, 0.839216, 0.905882}
\pgfsetfillcolor{dialinecolor}
\pgfpathmoveto{\pgfpoint{20.641346\du}{14.420981\du}}
\pgfpathlineto{\pgfpoint{20.641346\du}{13.591481\du}}
\pgfpathlineto{\pgfpoint{20.620900\du}{13.591481\du}}
\pgfpathlineto{\pgfpoint{20.620900\du}{14.420981\du}}
\pgfpathlineto{\pgfpoint{20.641346\du}{14.420981\du}}
\pgfusepath{fill}
\pgfsetbuttcap
\pgfsetmiterjoin
\pgfsetdash{}{0pt}
\definecolor{dialinecolor}{rgb}{0.678431, 0.839216, 0.905882}
\pgfsetfillcolor{dialinecolor}
\pgfpathmoveto{\pgfpoint{20.620900\du}{14.420981\du}}
\pgfpathlineto{\pgfpoint{20.620900\du}{14.431568\du}}
\pgfpathlineto{\pgfpoint{20.641346\du}{14.431568\du}}
\pgfpathlineto{\pgfpoint{20.641346\du}{14.420981\du}}
\pgfpathlineto{\pgfpoint{20.620900\du}{14.420981\du}}
\pgfusepath{fill}
\pgfsetbuttcap
\pgfsetmiterjoin
\pgfsetdash{}{0pt}
\definecolor{dialinecolor}{rgb}{0.678431, 0.839216, 0.905882}
\pgfsetfillcolor{dialinecolor}
\pgfpathmoveto{\pgfpoint{24.002061\du}{13.591481\du}}
\pgfpathlineto{\pgfpoint{24.002061\du}{13.580893\du}}
\pgfpathlineto{\pgfpoint{23.981981\du}{13.580893\du}}
\pgfpathlineto{\pgfpoint{23.981981\du}{13.591481\du}}
\pgfpathlineto{\pgfpoint{24.002061\du}{13.591481\du}}
\pgfusepath{fill}
\pgfsetbuttcap
\pgfsetmiterjoin
\pgfsetdash{}{0pt}
\definecolor{dialinecolor}{rgb}{0.678431, 0.839216, 0.905882}
\pgfsetfillcolor{dialinecolor}
\pgfpathmoveto{\pgfpoint{24.002061\du}{14.420981\du}}
\pgfpathlineto{\pgfpoint{24.002061\du}{13.591481\du}}
\pgfpathlineto{\pgfpoint{23.981981\du}{13.591481\du}}
\pgfpathlineto{\pgfpoint{23.981981\du}{14.420981\du}}
\pgfpathlineto{\pgfpoint{24.002061\du}{14.420981\du}}
\pgfusepath{fill}
\pgfsetbuttcap
\pgfsetmiterjoin
\pgfsetdash{}{0pt}
\definecolor{dialinecolor}{rgb}{0.678431, 0.839216, 0.905882}
\pgfsetfillcolor{dialinecolor}
\pgfpathmoveto{\pgfpoint{23.981981\du}{14.420981\du}}
\pgfpathlineto{\pgfpoint{23.981981\du}{14.431568\du}}
\pgfpathlineto{\pgfpoint{24.002061\du}{14.431568\du}}
\pgfpathlineto{\pgfpoint{24.002061\du}{14.420981\du}}
\pgfpathlineto{\pgfpoint{23.981981\du}{14.420981\du}}
\pgfusepath{fill}
\pgfsetbuttcap
\pgfsetmiterjoin
\pgfsetdash{}{0pt}
\definecolor{dialinecolor}{rgb}{0.027451, 0.372549, 0.529412}
\pgfsetfillcolor{dialinecolor}
\pgfpathmoveto{\pgfpoint{22.928312\du}{14.570305\du}}
\pgfpathlineto{\pgfpoint{22.927947\du}{14.587830\du}}
\pgfpathlineto{\pgfpoint{22.925391\du}{14.604989\du}}
\pgfpathlineto{\pgfpoint{22.921740\du}{14.621054\du}}
\pgfpathlineto{\pgfpoint{22.915899\du}{14.638213\du}}
\pgfpathlineto{\pgfpoint{22.909327\du}{14.653912\du}}
\pgfpathlineto{\pgfpoint{22.900930\du}{14.669977\du}}
\pgfpathlineto{\pgfpoint{22.891437\du}{14.685676\du}}
\pgfpathlineto{\pgfpoint{22.879754\du}{14.700645\du}}
\pgfpathlineto{\pgfpoint{22.868071\du}{14.715614\du}}
\pgfpathlineto{\pgfpoint{22.854562\du}{14.730218\du}}
\pgfpathlineto{\pgfpoint{22.839593\du}{14.744091\du}}
\pgfpathlineto{\pgfpoint{22.823529\du}{14.758330\du}}
\pgfpathlineto{\pgfpoint{22.806005\du}{14.771108\du}}
\pgfpathlineto{\pgfpoint{22.787750\du}{14.783887\du}}
\pgfpathlineto{\pgfpoint{22.768765\du}{14.796300\du}}
\pgfpathlineto{\pgfpoint{22.748684\du}{14.807983\du}}
\pgfpathlineto{\pgfpoint{22.727144\du}{14.818571\du}}
\pgfpathlineto{\pgfpoint{22.704508\du}{14.829524\du}}
\pgfpathlineto{\pgfpoint{22.681507\du}{14.839382\du}}
\pgfpathlineto{\pgfpoint{22.657775\du}{14.848874\du}}
\pgfpathlineto{\pgfpoint{22.632584\du}{14.856906\du}}
\pgfpathlineto{\pgfpoint{22.607027\du}{14.865303\du}}
\pgfpathlineto{\pgfpoint{22.580375\du}{14.872970\du}}
\pgfpathlineto{\pgfpoint{22.553723\du}{14.879907\du}}
\pgfpathlineto{\pgfpoint{22.525610\du}{14.885749\du}}
\pgfpathlineto{\pgfpoint{22.497133\du}{14.890860\du}}
\pgfpathlineto{\pgfpoint{22.467560\du}{14.895241\du}}
\pgfpathlineto{\pgfpoint{22.437257\du}{14.899257\du}}
\pgfpathlineto{\pgfpoint{22.407319\du}{14.902178\du}}
\pgfpathlineto{\pgfpoint{22.376651\du}{14.904369\du}}
\pgfpathlineto{\pgfpoint{22.345252\du}{14.905464\du}}
\pgfpathlineto{\pgfpoint{22.313854\du}{14.906194\du}}
\pgfpathlineto{\pgfpoint{22.282821\du}{14.905464\du}}
\pgfpathlineto{\pgfpoint{22.251423\du}{14.904369\du}}
\pgfpathlineto{\pgfpoint{22.220389\du}{14.902178\du}}
\pgfpathlineto{\pgfpoint{22.190086\du}{14.899257\du}}
\pgfpathlineto{\pgfpoint{22.160513\du}{14.895241\du}}
\pgfpathlineto{\pgfpoint{22.130941\du}{14.890860\du}}
\pgfpathlineto{\pgfpoint{22.102463\du}{14.885749\du}}
\pgfpathlineto{\pgfpoint{22.074716\du}{14.879907\du}}
\pgfpathlineto{\pgfpoint{22.047333\du}{14.872970\du}}
\pgfpathlineto{\pgfpoint{22.021047\du}{14.865303\du}}
\pgfpathlineto{\pgfpoint{21.995855\du}{14.856906\du}}
\pgfpathlineto{\pgfpoint{21.970298\du}{14.848874\du}}
\pgfpathlineto{\pgfpoint{21.946202\du}{14.839382\du}}
\pgfpathlineto{\pgfpoint{21.923566\du}{14.829524\du}}
\pgfpathlineto{\pgfpoint{21.900930\du}{14.818571\du}}
\pgfpathlineto{\pgfpoint{21.880119\du}{14.807983\du}}
\pgfpathlineto{\pgfpoint{21.859674\du}{14.796300\du}}
\pgfpathlineto{\pgfpoint{21.839228\du}{14.783887\du}}
\pgfpathlineto{\pgfpoint{21.821339\du}{14.771108\du}}
\pgfpathlineto{\pgfpoint{21.804909\du}{14.758330\du}}
\pgfpathlineto{\pgfpoint{21.788115\du}{14.744091\du}}
\pgfpathlineto{\pgfpoint{21.773511\du}{14.730218\du}}
\pgfpathlineto{\pgfpoint{21.760367\du}{14.715614\du}}
\pgfpathlineto{\pgfpoint{21.747954\du}{14.700645\du}}
\pgfpathlineto{\pgfpoint{21.737001\du}{14.685676\du}}
\pgfpathlineto{\pgfpoint{21.727509\du}{14.669977\du}}
\pgfpathlineto{\pgfpoint{21.718746\du}{14.653912\du}}
\pgfpathlineto{\pgfpoint{21.711810\du}{14.638213\du}}
\pgfpathlineto{\pgfpoint{21.707063\du}{14.621054\du}}
\pgfpathlineto{\pgfpoint{21.702682\du}{14.604989\du}}
\pgfpathlineto{\pgfpoint{21.700126\du}{14.587830\du}}
\pgfpathlineto{\pgfpoint{21.699031\du}{14.570305\du}}
\pgfpathlineto{\pgfpoint{21.700126\du}{14.552781\du}}
\pgfpathlineto{\pgfpoint{21.702682\du}{14.535621\du}}
\pgfpathlineto{\pgfpoint{21.707063\du}{14.519557\du}}
\pgfpathlineto{\pgfpoint{21.711810\du}{14.502397\du}}
\pgfpathlineto{\pgfpoint{21.718746\du}{14.486698\du}}
\pgfpathlineto{\pgfpoint{21.727509\du}{14.470999\du}}
\pgfpathlineto{\pgfpoint{21.737001\du}{14.454935\du}}
\pgfpathlineto{\pgfpoint{21.747954\du}{14.439966\du}}
\pgfpathlineto{\pgfpoint{21.760367\du}{14.425362\du}}
\pgfpathlineto{\pgfpoint{21.773511\du}{14.410758\du}}
\pgfpathlineto{\pgfpoint{21.788115\du}{14.396519\du}}
\pgfpathlineto{\pgfpoint{21.804909\du}{14.382646\du}}
\pgfpathlineto{\pgfpoint{21.821339\du}{14.369502\du}}
\pgfpathlineto{\pgfpoint{21.839228\du}{14.357454\du}}
\pgfpathlineto{\pgfpoint{21.859674\du}{14.345041\du}}
\pgfpathlineto{\pgfpoint{21.880119\du}{14.333357\du}}
\pgfpathlineto{\pgfpoint{21.900930\du}{14.322405\du}}
\pgfpathlineto{\pgfpoint{21.923566\du}{14.311817\du}}
\pgfpathlineto{\pgfpoint{21.946202\du}{14.301229\du}}
\pgfpathlineto{\pgfpoint{21.970298\du}{14.292467\du}}
\pgfpathlineto{\pgfpoint{21.995855\du}{14.283704\du}}
\pgfpathlineto{\pgfpoint{22.021047\du}{14.275307\du}}
\pgfpathlineto{\pgfpoint{22.047333\du}{14.267640\du}}
\pgfpathlineto{\pgfpoint{22.074716\du}{14.261433\du}}
\pgfpathlineto{\pgfpoint{22.102463\du}{14.254862\du}}
\pgfpathlineto{\pgfpoint{22.130941\du}{14.249750\du}}
\pgfpathlineto{\pgfpoint{22.160513\du}{14.245734\du}}
\pgfpathlineto{\pgfpoint{22.190086\du}{14.241353\du}}
\pgfpathlineto{\pgfpoint{22.220389\du}{14.238432\du}}
\pgfpathlineto{\pgfpoint{22.251423\du}{14.236972\du}}
\pgfpathlineto{\pgfpoint{22.282821\du}{14.235146\du}}
\pgfpathlineto{\pgfpoint{22.313854\du}{14.235146\du}}
\pgfpathlineto{\pgfpoint{22.345252\du}{14.235146\du}}
\pgfpathlineto{\pgfpoint{22.376651\du}{14.236972\du}}
\pgfpathlineto{\pgfpoint{22.407319\du}{14.238432\du}}
\pgfpathlineto{\pgfpoint{22.437257\du}{14.241353\du}}
\pgfpathlineto{\pgfpoint{22.467560\du}{14.245734\du}}
\pgfpathlineto{\pgfpoint{22.497133\du}{14.249750\du}}
\pgfpathlineto{\pgfpoint{22.525610\du}{14.254862\du}}
\pgfpathlineto{\pgfpoint{22.553723\du}{14.261433\du}}
\pgfpathlineto{\pgfpoint{22.580375\du}{14.267640\du}}
\pgfpathlineto{\pgfpoint{22.607027\du}{14.275307\du}}
\pgfpathlineto{\pgfpoint{22.632584\du}{14.283704\du}}
\pgfpathlineto{\pgfpoint{22.657775\du}{14.292467\du}}
\pgfpathlineto{\pgfpoint{22.681507\du}{14.301229\du}}
\pgfpathlineto{\pgfpoint{22.704508\du}{14.311817\du}}
\pgfpathlineto{\pgfpoint{22.727144\du}{14.322405\du}}
\pgfpathlineto{\pgfpoint{22.748684\du}{14.333357\du}}
\pgfpathlineto{\pgfpoint{22.768765\du}{14.345041\du}}
\pgfpathlineto{\pgfpoint{22.787750\du}{14.357454\du}}
\pgfpathlineto{\pgfpoint{22.806005\du}{14.369502\du}}
\pgfpathlineto{\pgfpoint{22.823529\du}{14.382646\du}}
\pgfpathlineto{\pgfpoint{22.839593\du}{14.396519\du}}
\pgfpathlineto{\pgfpoint{22.854562\du}{14.410758\du}}
\pgfpathlineto{\pgfpoint{22.868071\du}{14.425362\du}}
\pgfpathlineto{\pgfpoint{22.879754\du}{14.439966\du}}
\pgfpathlineto{\pgfpoint{22.891437\du}{14.454935\du}}
\pgfpathlineto{\pgfpoint{22.900930\du}{14.470999\du}}
\pgfpathlineto{\pgfpoint{22.909327\du}{14.486698\du}}
\pgfpathlineto{\pgfpoint{22.915899\du}{14.502397\du}}
\pgfpathlineto{\pgfpoint{22.921740\du}{14.519557\du}}
\pgfpathlineto{\pgfpoint{22.925391\du}{14.535621\du}}
\pgfpathlineto{\pgfpoint{22.927947\du}{14.552781\du}}
\pgfpathlineto{\pgfpoint{22.928312\du}{14.570305\du}}
\pgfusepath{fill}
\pgfsetbuttcap
\pgfsetmiterjoin
\pgfsetdash{}{0pt}
\definecolor{dialinecolor}{rgb}{0.678431, 0.839216, 0.905882}
\pgfsetfillcolor{dialinecolor}
\pgfpathmoveto{\pgfpoint{22.313854\du}{14.916052\du}}
\pgfpathlineto{\pgfpoint{22.313854\du}{14.916052\du}}
\pgfpathlineto{\pgfpoint{22.329918\du}{14.916052\du}}
\pgfpathlineto{\pgfpoint{22.345983\du}{14.915687\du}}
\pgfpathlineto{\pgfpoint{22.362047\du}{14.914957\du}}
\pgfpathlineto{\pgfpoint{22.377381\du}{14.914226\du}}
\pgfpathlineto{\pgfpoint{22.393080\du}{14.913131\du}}
\pgfpathlineto{\pgfpoint{22.408414\du}{14.912036\du}}
\pgfpathlineto{\pgfpoint{22.423383\du}{14.910941\du}}
\pgfpathlineto{\pgfpoint{22.439447\du}{14.909115\du}}
\pgfpathlineto{\pgfpoint{22.453686\du}{14.907290\du}}
\pgfpathlineto{\pgfpoint{22.468655\du}{14.905464\du}}
\pgfpathlineto{\pgfpoint{22.483624\du}{14.903273\du}}
\pgfpathlineto{\pgfpoint{22.498958\du}{14.901083\du}}
\pgfpathlineto{\pgfpoint{22.513197\du}{14.898162\du}}
\pgfpathlineto{\pgfpoint{22.527071\du}{14.895606\du}}
\pgfpathlineto{\pgfpoint{22.541674\du}{14.892686\du}}
\pgfpathlineto{\pgfpoint{22.555548\du}{14.889765\du}}
\pgfpathlineto{\pgfpoint{22.569057\du}{14.886114\du}}
\pgfpathlineto{\pgfpoint{22.582930\du}{14.882828\du}}
\pgfpathlineto{\pgfpoint{22.596439\du}{14.879177\du}}
\pgfpathlineto{\pgfpoint{22.609582\du}{14.875161\du}}
\pgfpathlineto{\pgfpoint{22.622726\du}{14.871145\du}}
\pgfpathlineto{\pgfpoint{22.635869\du}{14.867129\du}}
\pgfpathlineto{\pgfpoint{22.648648\du}{14.862383\du}}
\pgfpathlineto{\pgfpoint{22.661426\du}{14.858367\du}}
\pgfpathlineto{\pgfpoint{22.673109\du}{14.853620\du}}
\pgfpathlineto{\pgfpoint{22.685888\du}{14.848874\du}}
\pgfpathlineto{\pgfpoint{22.697206\du}{14.843398\du}}
\pgfpathlineto{\pgfpoint{22.708889\du}{14.838286\du}}
\pgfpathlineto{\pgfpoint{22.720207\du}{14.833175\du}}
\pgfpathlineto{\pgfpoint{22.731525\du}{14.827698\du}}
\pgfpathlineto{\pgfpoint{22.742478\du}{14.822587\du}}
\pgfpathlineto{\pgfpoint{22.753796\du}{14.816746\du}}
\pgfpathlineto{\pgfpoint{22.763653\du}{14.810904\du}}
\pgfpathlineto{\pgfpoint{22.773876\du}{14.804332\du}}
\pgfpathlineto{\pgfpoint{22.783734\du}{14.798491\du}}
\pgfpathlineto{\pgfpoint{22.793956\du}{14.791919\du}}
\pgfpathlineto{\pgfpoint{22.803449\du}{14.785712\du}}
\pgfpathlineto{\pgfpoint{22.812576\du}{14.779141\du}}
\pgfpathlineto{\pgfpoint{22.821339\du}{14.772934\du}}
\pgfpathlineto{\pgfpoint{22.829736\du}{14.765632\du}}
\pgfpathlineto{\pgfpoint{22.838133\du}{14.758695\du}}
\pgfpathlineto{\pgfpoint{22.846165\du}{14.751758\du}}
\pgfpathlineto{\pgfpoint{22.854197\du}{14.744821\du}}
\pgfpathlineto{\pgfpoint{22.861134\du}{14.737520\du}}
\pgfpathlineto{\pgfpoint{22.868801\du}{14.730218\du}}
\pgfpathlineto{\pgfpoint{22.875373\du}{14.722551\du}}
\pgfpathlineto{\pgfpoint{22.881945\du}{14.714884\du}}
\pgfpathlineto{\pgfpoint{22.888151\du}{14.707217\du}}
\pgfpathlineto{\pgfpoint{22.894358\du}{14.699184\du}}
\pgfpathlineto{\pgfpoint{22.900199\du}{14.691517\du}}
\pgfpathlineto{\pgfpoint{22.904946\du}{14.683120\du}}
\pgfpathlineto{\pgfpoint{22.909692\du}{14.675088\du}}
\pgfpathlineto{\pgfpoint{22.914438\du}{14.666691\du}}
\pgfpathlineto{\pgfpoint{22.918819\du}{14.658659\du}}
\pgfpathlineto{\pgfpoint{22.922105\du}{14.649896\du}}
\pgfpathlineto{\pgfpoint{22.926121\du}{14.641864\du}}
\pgfpathlineto{\pgfpoint{22.928677\du}{14.633102\du}}
\pgfpathlineto{\pgfpoint{22.931598\du}{14.624340\du}}
\pgfpathlineto{\pgfpoint{22.933423\du}{14.615212\du}}
\pgfpathlineto{\pgfpoint{22.935614\du}{14.606450\du}}
\pgfpathlineto{\pgfpoint{22.937074\du}{14.597688\du}}
\pgfpathlineto{\pgfpoint{22.938170\du}{14.588560\du}}
\pgfpathlineto{\pgfpoint{22.938900\du}{14.579798\du}}
\pgfpathlineto{\pgfpoint{22.938900\du}{14.570305\du}}
\pgfpathlineto{\pgfpoint{22.918819\du}{14.570305\du}}
\pgfpathlineto{\pgfpoint{22.918089\du}{14.578337\du}}
\pgfpathlineto{\pgfpoint{22.917724\du}{14.586735\du}}
\pgfpathlineto{\pgfpoint{22.916994\du}{14.594767\du}}
\pgfpathlineto{\pgfpoint{22.915534\du}{14.602434\du}}
\pgfpathlineto{\pgfpoint{22.914073\du}{14.610831\du}}
\pgfpathlineto{\pgfpoint{22.911517\du}{14.618863\du}}
\pgfpathlineto{\pgfpoint{22.909327\du}{14.626530\du}}
\pgfpathlineto{\pgfpoint{22.906041\du}{14.634197\du}}
\pgfpathlineto{\pgfpoint{22.903485\du}{14.641864\du}}
\pgfpathlineto{\pgfpoint{22.900199\du}{14.649896\du}}
\pgfpathlineto{\pgfpoint{22.896183\du}{14.657563\du}}
\pgfpathlineto{\pgfpoint{22.891437\du}{14.665230\du}}
\pgfpathlineto{\pgfpoint{22.887421\du}{14.672897\du}}
\pgfpathlineto{\pgfpoint{22.882310\du}{14.679834\du}}
\pgfpathlineto{\pgfpoint{22.877563\du}{14.687501\du}}
\pgfpathlineto{\pgfpoint{22.872087\du}{14.694438\du}}
\pgfpathlineto{\pgfpoint{22.866976\du}{14.702105\du}}
\pgfpathlineto{\pgfpoint{22.860039\du}{14.709042\du}}
\pgfpathlineto{\pgfpoint{22.854197\du}{14.715979\du}}
\pgfpathlineto{\pgfpoint{22.846895\du}{14.722916\du}}
\pgfpathlineto{\pgfpoint{22.839959\du}{14.730218\du}}
\pgfpathlineto{\pgfpoint{22.832657\du}{14.736424\du}}
\pgfpathlineto{\pgfpoint{22.825720\du}{14.743361\du}}
\pgfpathlineto{\pgfpoint{22.817323\du}{14.749933\du}}
\pgfpathlineto{\pgfpoint{22.809290\du}{14.756505\du}}
\pgfpathlineto{\pgfpoint{22.800163\du}{14.762711\du}}
\pgfpathlineto{\pgfpoint{22.791401\du}{14.768553\du}}
\pgfpathlineto{\pgfpoint{22.781908\du}{14.775125\du}}
\pgfpathlineto{\pgfpoint{22.772781\du}{14.781696\du}}
\pgfpathlineto{\pgfpoint{22.763653\du}{14.786808\du}}
\pgfpathlineto{\pgfpoint{22.753796\du}{14.792649\du}}
\pgfpathlineto{\pgfpoint{22.743938\du}{14.798491\du}}
\pgfpathlineto{\pgfpoint{22.733350\du}{14.803967\du}}
\pgfpathlineto{\pgfpoint{22.722762\du}{14.809809\du}}
\pgfpathlineto{\pgfpoint{22.712175\du}{14.814190\du}}
\pgfpathlineto{\pgfpoint{22.700492\du}{14.820031\du}}
\pgfpathlineto{\pgfpoint{22.689174\du}{14.824778\du}}
\pgfpathlineto{\pgfpoint{22.677856\du}{14.829524\du}}
\pgfpathlineto{\pgfpoint{22.666172\du}{14.834270\du}}
\pgfpathlineto{\pgfpoint{22.654124\du}{14.839016\du}}
\pgfpathlineto{\pgfpoint{22.641711\du}{14.843033\du}}
\pgfpathlineto{\pgfpoint{22.630028\du}{14.847779\du}}
\pgfpathlineto{\pgfpoint{22.617250\du}{14.851795\du}}
\pgfpathlineto{\pgfpoint{22.603741\du}{14.855446\du}}
\pgfpathlineto{\pgfpoint{22.590963\du}{14.859462\du}}
\pgfpathlineto{\pgfpoint{22.577819\du}{14.862748\du}}
\pgfpathlineto{\pgfpoint{22.564310\du}{14.866399\du}}
\pgfpathlineto{\pgfpoint{22.550802\du}{14.869319\du}}
\pgfpathlineto{\pgfpoint{22.536928\du}{14.872970\du}}
\pgfpathlineto{\pgfpoint{22.523055\du}{14.875161\du}}
\pgfpathlineto{\pgfpoint{22.509181\du}{14.878082\du}}
\pgfpathlineto{\pgfpoint{22.494942\du}{14.880272\du}}
\pgfpathlineto{\pgfpoint{22.480703\du}{14.882828\du}}
\pgfpathlineto{\pgfpoint{22.466465\du}{14.885019\du}}
\pgfpathlineto{\pgfpoint{22.451861\du}{14.886844\du}}
\pgfpathlineto{\pgfpoint{22.436892\du}{14.888670\du}}
\pgfpathlineto{\pgfpoint{22.421558\du}{14.890495\du}}
\pgfpathlineto{\pgfpoint{22.406954\du}{14.891590\du}}
\pgfpathlineto{\pgfpoint{22.391255\du}{14.892686\du}}
\pgfpathlineto{\pgfpoint{22.376286\du}{14.893781\du}}
\pgfpathlineto{\pgfpoint{22.360952\du}{14.894511\du}}
\pgfpathlineto{\pgfpoint{22.345252\du}{14.895241\du}}
\pgfpathlineto{\pgfpoint{22.329918\du}{14.895606\du}}
\pgfpathlineto{\pgfpoint{22.313854\du}{14.895606\du}}
\pgfpathlineto{\pgfpoint{22.313854\du}{14.895606\du}}
\pgfpathlineto{\pgfpoint{22.313854\du}{14.895606\du}}
\pgfpathlineto{\pgfpoint{22.312759\du}{14.895606\du}}
\pgfpathlineto{\pgfpoint{22.311664\du}{14.895606\du}}
\pgfpathlineto{\pgfpoint{22.310933\du}{14.896337\du}}
\pgfpathlineto{\pgfpoint{22.309108\du}{14.896337\du}}
\pgfpathlineto{\pgfpoint{22.308743\du}{14.896702\du}}
\pgfpathlineto{\pgfpoint{22.307647\du}{14.897432\du}}
\pgfpathlineto{\pgfpoint{22.307282\du}{14.898162\du}}
\pgfpathlineto{\pgfpoint{22.306917\du}{14.898527\du}}
\pgfpathlineto{\pgfpoint{22.305457\du}{14.900353\du}}
\pgfpathlineto{\pgfpoint{22.303996\du}{14.902178\du}}
\pgfpathlineto{\pgfpoint{22.303996\du}{14.904004\du}}
\pgfpathlineto{\pgfpoint{22.303631\du}{14.906194\du}}
\pgfpathlineto{\pgfpoint{22.303996\du}{14.908020\du}}
\pgfpathlineto{\pgfpoint{22.303996\du}{14.909845\du}}
\pgfpathlineto{\pgfpoint{22.305457\du}{14.911306\du}}
\pgfpathlineto{\pgfpoint{22.306917\du}{14.912766\du}}
\pgfpathlineto{\pgfpoint{22.307282\du}{14.913861\du}}
\pgfpathlineto{\pgfpoint{22.307647\du}{14.914226\du}}
\pgfpathlineto{\pgfpoint{22.308743\du}{14.914957\du}}
\pgfpathlineto{\pgfpoint{22.309108\du}{14.914957\du}}
\pgfpathlineto{\pgfpoint{22.310933\du}{14.915687\du}}
\pgfpathlineto{\pgfpoint{22.311664\du}{14.916052\du}}
\pgfpathlineto{\pgfpoint{22.312759\du}{14.916052\du}}
\pgfpathlineto{\pgfpoint{22.313854\du}{14.916052\du}}
\pgfusepath{fill}
\pgfsetbuttcap
\pgfsetmiterjoin
\pgfsetdash{}{0pt}
\definecolor{dialinecolor}{rgb}{0.678431, 0.839216, 0.905882}
\pgfsetfillcolor{dialinecolor}
\pgfpathmoveto{\pgfpoint{21.689174\du}{14.570305\du}}
\pgfpathlineto{\pgfpoint{21.689174\du}{14.570305\du}}
\pgfpathlineto{\pgfpoint{21.689174\du}{14.579068\du}}
\pgfpathlineto{\pgfpoint{21.689904\du}{14.588560\du}}
\pgfpathlineto{\pgfpoint{21.690999\du}{14.597688\du}}
\pgfpathlineto{\pgfpoint{21.693190\du}{14.606450\du}}
\pgfpathlineto{\pgfpoint{21.694285\du}{14.615212\du}}
\pgfpathlineto{\pgfpoint{21.696841\du}{14.624340\du}}
\pgfpathlineto{\pgfpoint{21.699031\du}{14.633102\du}}
\pgfpathlineto{\pgfpoint{21.702317\du}{14.641864\du}}
\pgfpathlineto{\pgfpoint{21.705603\du}{14.649896\du}}
\pgfpathlineto{\pgfpoint{21.709619\du}{14.658659\du}}
\pgfpathlineto{\pgfpoint{21.714000\du}{14.666691\du}}
\pgfpathlineto{\pgfpoint{21.718381\du}{14.675088\du}}
\pgfpathlineto{\pgfpoint{21.723128\du}{14.683120\du}}
\pgfpathlineto{\pgfpoint{21.728239\du}{14.691517\du}}
\pgfpathlineto{\pgfpoint{21.734446\du}{14.699184\du}}
\pgfpathlineto{\pgfpoint{21.739192\du}{14.707217\du}}
\pgfpathlineto{\pgfpoint{21.745764\du}{14.714884\du}}
\pgfpathlineto{\pgfpoint{21.752700\du}{14.722551\du}}
\pgfpathlineto{\pgfpoint{21.759272\du}{14.730218\du}}
\pgfpathlineto{\pgfpoint{21.766209\du}{14.737520\du}}
\pgfpathlineto{\pgfpoint{21.773511\du}{14.744821\du}}
\pgfpathlineto{\pgfpoint{21.782273\du}{14.751758\du}}
\pgfpathlineto{\pgfpoint{21.789210\du}{14.758695\du}}
\pgfpathlineto{\pgfpoint{21.798337\du}{14.765632\du}}
\pgfpathlineto{\pgfpoint{21.806370\du}{14.772934\du}}
\pgfpathlineto{\pgfpoint{21.815497\du}{14.779141\du}}
\pgfpathlineto{\pgfpoint{21.825355\du}{14.785712\du}}
\pgfpathlineto{\pgfpoint{21.834482\du}{14.791919\du}}
\pgfpathlineto{\pgfpoint{21.843975\du}{14.798491\du}}
\pgfpathlineto{\pgfpoint{21.853832\du}{14.804332\du}}
\pgfpathlineto{\pgfpoint{21.864055\du}{14.810904\du}}
\pgfpathlineto{\pgfpoint{21.874278\du}{14.816746\du}}
\pgfpathlineto{\pgfpoint{21.885231\du}{14.822587\du}}
\pgfpathlineto{\pgfpoint{21.896549\du}{14.827698\du}}
\pgfpathlineto{\pgfpoint{21.907867\du}{14.833175\du}}
\pgfpathlineto{\pgfpoint{21.919185\du}{14.838286\du}}
\pgfpathlineto{\pgfpoint{21.931233\du}{14.843398\du}}
\pgfpathlineto{\pgfpoint{21.942551\du}{14.848874\du}}
\pgfpathlineto{\pgfpoint{21.954964\du}{14.853620\du}}
\pgfpathlineto{\pgfpoint{21.967012\du}{14.858367\du}}
\pgfpathlineto{\pgfpoint{21.979060\du}{14.862383\du}}
\pgfpathlineto{\pgfpoint{21.992204\du}{14.867129\du}}
\pgfpathlineto{\pgfpoint{22.004982\du}{14.871145\du}}
\pgfpathlineto{\pgfpoint{22.018491\du}{14.875161\du}}
\pgfpathlineto{\pgfpoint{22.031999\du}{14.879177\du}}
\pgfpathlineto{\pgfpoint{22.044778\du}{14.882828\du}}
\pgfpathlineto{\pgfpoint{22.058286\du}{14.886114\du}}
\pgfpathlineto{\pgfpoint{22.072160\du}{14.889765\du}}
\pgfpathlineto{\pgfpoint{22.086764\du}{14.892686\du}}
\pgfpathlineto{\pgfpoint{22.101368\du}{14.895606\du}}
\pgfpathlineto{\pgfpoint{22.115241\du}{14.898162\du}}
\pgfpathlineto{\pgfpoint{22.129480\du}{14.901083\du}}
\pgfpathlineto{\pgfpoint{22.144449\du}{14.903273\du}}
\pgfpathlineto{\pgfpoint{22.159418\du}{14.905464\du}}
\pgfpathlineto{\pgfpoint{22.174022\du}{14.907290\du}}
\pgfpathlineto{\pgfpoint{22.188626\du}{14.909115\du}}
\pgfpathlineto{\pgfpoint{22.204325\du}{14.910941\du}}
\pgfpathlineto{\pgfpoint{22.220024\du}{14.912036\du}}
\pgfpathlineto{\pgfpoint{22.234628\du}{14.913131\du}}
\pgfpathlineto{\pgfpoint{22.250692\du}{14.914226\du}}
\pgfpathlineto{\pgfpoint{22.266026\du}{14.914957\du}}
\pgfpathlineto{\pgfpoint{22.282091\du}{14.915687\du}}
\pgfpathlineto{\pgfpoint{22.298155\du}{14.916052\du}}
\pgfpathlineto{\pgfpoint{22.313854\du}{14.916052\du}}
\pgfpathlineto{\pgfpoint{22.313854\du}{14.895606\du}}
\pgfpathlineto{\pgfpoint{22.298155\du}{14.895606\du}}
\pgfpathlineto{\pgfpoint{22.282821\du}{14.895241\du}}
\pgfpathlineto{\pgfpoint{22.266757\du}{14.894511\du}}
\pgfpathlineto{\pgfpoint{22.252153\du}{14.893781\du}}
\pgfpathlineto{\pgfpoint{22.236454\du}{14.892686\du}}
\pgfpathlineto{\pgfpoint{22.221485\du}{14.891590\du}}
\pgfpathlineto{\pgfpoint{22.206516\du}{14.890495\du}}
\pgfpathlineto{\pgfpoint{22.191547\du}{14.888670\du}}
\pgfpathlineto{\pgfpoint{22.176578\du}{14.886844\du}}
\pgfpathlineto{\pgfpoint{22.161974\du}{14.885019\du}}
\pgfpathlineto{\pgfpoint{22.147370\du}{14.882828\du}}
\pgfpathlineto{\pgfpoint{22.133131\du}{14.880272\du}}
\pgfpathlineto{\pgfpoint{22.119258\du}{14.878082\du}}
\pgfpathlineto{\pgfpoint{22.105384\du}{14.875161\du}}
\pgfpathlineto{\pgfpoint{22.091145\du}{14.872970\du}}
\pgfpathlineto{\pgfpoint{22.077271\du}{14.869319\du}}
\pgfpathlineto{\pgfpoint{22.063763\du}{14.866399\du}}
\pgfpathlineto{\pgfpoint{22.050619\du}{14.862748\du}}
\pgfpathlineto{\pgfpoint{22.037111\du}{14.859462\du}}
\pgfpathlineto{\pgfpoint{22.024332\du}{14.855446\du}}
\pgfpathlineto{\pgfpoint{22.011189\du}{14.851795\du}}
\pgfpathlineto{\pgfpoint{21.998411\du}{14.847779\du}}
\pgfpathlineto{\pgfpoint{21.986362\du}{14.843033\du}}
\pgfpathlineto{\pgfpoint{21.973949\du}{14.839016\du}}
\pgfpathlineto{\pgfpoint{21.961536\du}{14.834270\du}}
\pgfpathlineto{\pgfpoint{21.950218\du}{14.829524\du}}
\pgfpathlineto{\pgfpoint{21.938535\du}{14.824778\du}}
\pgfpathlineto{\pgfpoint{21.927582\du}{14.820031\du}}
\pgfpathlineto{\pgfpoint{21.916264\du}{14.814190\du}}
\pgfpathlineto{\pgfpoint{21.905676\du}{14.809809\du}}
\pgfpathlineto{\pgfpoint{21.894723\du}{14.803967\du}}
\pgfpathlineto{\pgfpoint{21.884500\du}{14.798491\du}}
\pgfpathlineto{\pgfpoint{21.874278\du}{14.792649\du}}
\pgfpathlineto{\pgfpoint{21.864420\du}{14.786808\du}}
\pgfpathlineto{\pgfpoint{21.854927\du}{14.781696\du}}
\pgfpathlineto{\pgfpoint{21.845070\du}{14.775125\du}}
\pgfpathlineto{\pgfpoint{21.836673\du}{14.768553\du}}
\pgfpathlineto{\pgfpoint{21.827545\du}{14.762711\du}}
\pgfpathlineto{\pgfpoint{21.819148\du}{14.756505\du}}
\pgfpathlineto{\pgfpoint{21.810751\du}{14.749933\du}}
\pgfpathlineto{\pgfpoint{21.802719\du}{14.743361\du}}
\pgfpathlineto{\pgfpoint{21.795417\du}{14.736424\du}}
\pgfpathlineto{\pgfpoint{21.787750\du}{14.730218\du}}
\pgfpathlineto{\pgfpoint{21.781178\du}{14.722916\du}}
\pgfpathlineto{\pgfpoint{21.774241\du}{14.715979\du}}
\pgfpathlineto{\pgfpoint{21.768034\du}{14.709042\du}}
\pgfpathlineto{\pgfpoint{21.761828\du}{14.702105\du}}
\pgfpathlineto{\pgfpoint{21.755621\du}{14.694438\du}}
\pgfpathlineto{\pgfpoint{21.750510\du}{14.687501\du}}
\pgfpathlineto{\pgfpoint{21.745398\du}{14.679834\du}}
\pgfpathlineto{\pgfpoint{21.740652\du}{14.672897\du}}
\pgfpathlineto{\pgfpoint{21.736271\du}{14.665230\du}}
\pgfpathlineto{\pgfpoint{21.731890\du}{14.657563\du}}
\pgfpathlineto{\pgfpoint{21.728239\du}{14.649896\du}}
\pgfpathlineto{\pgfpoint{21.724953\du}{14.641864\du}}
\pgfpathlineto{\pgfpoint{21.721667\du}{14.634197\du}}
\pgfpathlineto{\pgfpoint{21.718746\du}{14.626530\du}}
\pgfpathlineto{\pgfpoint{21.716191\du}{14.618863\du}}
\pgfpathlineto{\pgfpoint{21.714365\du}{14.610831\du}}
\pgfpathlineto{\pgfpoint{21.712540\du}{14.602434\du}}
\pgfpathlineto{\pgfpoint{21.711444\du}{14.594767\du}}
\pgfpathlineto{\pgfpoint{21.709984\du}{14.586735\du}}
\pgfpathlineto{\pgfpoint{21.709619\du}{14.578337\du}}
\pgfpathlineto{\pgfpoint{21.709619\du}{14.570305\du}}
\pgfpathlineto{\pgfpoint{21.709619\du}{14.570305\du}}
\pgfpathlineto{\pgfpoint{21.709619\du}{14.570305\du}}
\pgfpathlineto{\pgfpoint{21.709619\du}{14.569210\du}}
\pgfpathlineto{\pgfpoint{21.709619\du}{14.568115\du}}
\pgfpathlineto{\pgfpoint{21.709254\du}{14.566654\du}}
\pgfpathlineto{\pgfpoint{21.709254\du}{14.565559\du}}
\pgfpathlineto{\pgfpoint{21.708889\du}{14.565194\du}}
\pgfpathlineto{\pgfpoint{21.707793\du}{14.563734\du}}
\pgfpathlineto{\pgfpoint{21.707063\du}{14.563368\du}}
\pgfpathlineto{\pgfpoint{21.707063\du}{14.562638\du}}
\pgfpathlineto{\pgfpoint{21.704873\du}{14.561543\du}}
\pgfpathlineto{\pgfpoint{21.703047\du}{14.560813\du}}
\pgfpathlineto{\pgfpoint{21.701222\du}{14.560448\du}}
\pgfpathlineto{\pgfpoint{21.699031\du}{14.559717\du}}
\pgfpathlineto{\pgfpoint{21.697571\du}{14.560448\du}}
\pgfpathlineto{\pgfpoint{21.695745\du}{14.560813\du}}
\pgfpathlineto{\pgfpoint{21.693920\du}{14.561543\du}}
\pgfpathlineto{\pgfpoint{21.692094\du}{14.562638\du}}
\pgfpathlineto{\pgfpoint{21.691364\du}{14.563368\du}}
\pgfpathlineto{\pgfpoint{21.690999\du}{14.563734\du}}
\pgfpathlineto{\pgfpoint{21.690634\du}{14.565194\du}}
\pgfpathlineto{\pgfpoint{21.689904\du}{14.565559\du}}
\pgfpathlineto{\pgfpoint{21.689904\du}{14.566654\du}}
\pgfpathlineto{\pgfpoint{21.689174\du}{14.568115\du}}
\pgfpathlineto{\pgfpoint{21.689174\du}{14.569210\du}}
\pgfpathlineto{\pgfpoint{21.689174\du}{14.570305\du}}
\pgfusepath{fill}
\pgfsetbuttcap
\pgfsetmiterjoin
\pgfsetdash{}{0pt}
\definecolor{dialinecolor}{rgb}{0.678431, 0.839216, 0.905882}
\pgfsetfillcolor{dialinecolor}
\pgfpathmoveto{\pgfpoint{22.313854\du}{14.224559\du}}
\pgfpathlineto{\pgfpoint{22.313854\du}{14.224559\du}}
\pgfpathlineto{\pgfpoint{22.298155\du}{14.224559\du}}
\pgfpathlineto{\pgfpoint{22.282091\du}{14.224924\du}}
\pgfpathlineto{\pgfpoint{22.266026\du}{14.225654\du}}
\pgfpathlineto{\pgfpoint{22.250692\du}{14.226384\du}}
\pgfpathlineto{\pgfpoint{22.234628\du}{14.227479\du}}
\pgfpathlineto{\pgfpoint{22.220024\du}{14.228575\du}}
\pgfpathlineto{\pgfpoint{22.204325\du}{14.229670\du}}
\pgfpathlineto{\pgfpoint{22.188626\du}{14.231495\du}}
\pgfpathlineto{\pgfpoint{22.174022\du}{14.233321\du}}
\pgfpathlineto{\pgfpoint{22.159418\du}{14.235146\du}}
\pgfpathlineto{\pgfpoint{22.144449\du}{14.237337\du}}
\pgfpathlineto{\pgfpoint{22.129480\du}{14.239893\du}}
\pgfpathlineto{\pgfpoint{22.115241\du}{14.242813\du}}
\pgfpathlineto{\pgfpoint{22.101368\du}{14.245004\du}}
\pgfpathlineto{\pgfpoint{22.086764\du}{14.247925\du}}
\pgfpathlineto{\pgfpoint{22.072160\du}{14.250846\du}}
\pgfpathlineto{\pgfpoint{22.058286\du}{14.254497\du}}
\pgfpathlineto{\pgfpoint{22.044778\du}{14.257782\du}}
\pgfpathlineto{\pgfpoint{22.031999\du}{14.261798\du}}
\pgfpathlineto{\pgfpoint{22.018491\du}{14.265449\du}}
\pgfpathlineto{\pgfpoint{22.004982\du}{14.269466\du}}
\pgfpathlineto{\pgfpoint{21.992204\du}{14.273847\du}}
\pgfpathlineto{\pgfpoint{21.979060\du}{14.278228\du}}
\pgfpathlineto{\pgfpoint{21.967012\du}{14.282609\du}}
\pgfpathlineto{\pgfpoint{21.954964\du}{14.286990\du}}
\pgfpathlineto{\pgfpoint{21.942551\du}{14.292467\du}}
\pgfpathlineto{\pgfpoint{21.931233\du}{14.297213\du}}
\pgfpathlineto{\pgfpoint{21.919185\du}{14.302324\du}}
\pgfpathlineto{\pgfpoint{21.907867\du}{14.307436\du}}
\pgfpathlineto{\pgfpoint{21.896549\du}{14.312912\du}}
\pgfpathlineto{\pgfpoint{21.885231\du}{14.318754\du}}
\pgfpathlineto{\pgfpoint{21.874278\du}{14.323865\du}}
\pgfpathlineto{\pgfpoint{21.864055\du}{14.329706\du}}
\pgfpathlineto{\pgfpoint{21.853832\du}{14.336278\du}}
\pgfpathlineto{\pgfpoint{21.843975\du}{14.342120\du}}
\pgfpathlineto{\pgfpoint{21.834482\du}{14.348692\du}}
\pgfpathlineto{\pgfpoint{21.825355\du}{14.354898\du}}
\pgfpathlineto{\pgfpoint{21.815497\du}{14.361470\du}}
\pgfpathlineto{\pgfpoint{21.806370\du}{14.368042\du}}
\pgfpathlineto{\pgfpoint{21.798337\du}{14.374978\du}}
\pgfpathlineto{\pgfpoint{21.789210\du}{14.381915\du}}
\pgfpathlineto{\pgfpoint{21.782273\du}{14.388852\du}}
\pgfpathlineto{\pgfpoint{21.773511\du}{14.395789\du}}
\pgfpathlineto{\pgfpoint{21.766209\du}{14.403091\du}}
\pgfpathlineto{\pgfpoint{21.759272\du}{14.410758\du}}
\pgfpathlineto{\pgfpoint{21.752700\du}{14.418060\du}}
\pgfpathlineto{\pgfpoint{21.745764\du}{14.425727\du}}
\pgfpathlineto{\pgfpoint{21.739192\du}{14.433394\du}}
\pgfpathlineto{\pgfpoint{21.734446\du}{14.441426\du}}
\pgfpathlineto{\pgfpoint{21.728239\du}{14.449093\du}}
\pgfpathlineto{\pgfpoint{21.723128\du}{14.457490\du}}
\pgfpathlineto{\pgfpoint{21.718381\du}{14.465522\du}}
\pgfpathlineto{\pgfpoint{21.714000\du}{14.473920\du}}
\pgfpathlineto{\pgfpoint{21.709619\du}{14.481952\du}}
\pgfpathlineto{\pgfpoint{21.705603\du}{14.490714\du}}
\pgfpathlineto{\pgfpoint{21.702317\du}{14.499111\du}}
\pgfpathlineto{\pgfpoint{21.699031\du}{14.507874\du}}
\pgfpathlineto{\pgfpoint{21.696841\du}{14.516636\du}}
\pgfpathlineto{\pgfpoint{21.694285\du}{14.525398\du}}
\pgfpathlineto{\pgfpoint{21.693190\du}{14.534161\du}}
\pgfpathlineto{\pgfpoint{21.690999\du}{14.542923\du}}
\pgfpathlineto{\pgfpoint{21.689904\du}{14.552050\du}}
\pgfpathlineto{\pgfpoint{21.689174\du}{14.561543\du}}
\pgfpathlineto{\pgfpoint{21.689174\du}{14.570305\du}}
\pgfpathlineto{\pgfpoint{21.709619\du}{14.570305\du}}
\pgfpathlineto{\pgfpoint{21.709619\du}{14.562273\du}}
\pgfpathlineto{\pgfpoint{21.709984\du}{14.553876\du}}
\pgfpathlineto{\pgfpoint{21.711444\du}{14.545844\du}}
\pgfpathlineto{\pgfpoint{21.712540\du}{14.538177\du}}
\pgfpathlineto{\pgfpoint{21.714365\du}{14.529780\du}}
\pgfpathlineto{\pgfpoint{21.716191\du}{14.521747\du}}
\pgfpathlineto{\pgfpoint{21.718746\du}{14.514080\du}}
\pgfpathlineto{\pgfpoint{21.721667\du}{14.506413\du}}
\pgfpathlineto{\pgfpoint{21.724953\du}{14.499111\du}}
\pgfpathlineto{\pgfpoint{21.728239\du}{14.490714\du}}
\pgfpathlineto{\pgfpoint{21.731890\du}{14.483047\du}}
\pgfpathlineto{\pgfpoint{21.736271\du}{14.475380\du}}
\pgfpathlineto{\pgfpoint{21.740652\du}{14.468443\du}}
\pgfpathlineto{\pgfpoint{21.745398\du}{14.460776\du}}
\pgfpathlineto{\pgfpoint{21.750510\du}{14.453474\du}}
\pgfpathlineto{\pgfpoint{21.755621\du}{14.446172\du}}
\pgfpathlineto{\pgfpoint{21.761828\du}{14.438505\du}}
\pgfpathlineto{\pgfpoint{21.768034\du}{14.431568\du}}
\pgfpathlineto{\pgfpoint{21.774241\du}{14.424632\du}}
\pgfpathlineto{\pgfpoint{21.781178\du}{14.417695\du}}
\pgfpathlineto{\pgfpoint{21.787750\du}{14.411123\du}}
\pgfpathlineto{\pgfpoint{21.795417\du}{14.404186\du}}
\pgfpathlineto{\pgfpoint{21.802719\du}{14.397249\du}}
\pgfpathlineto{\pgfpoint{21.810751\du}{14.390678\du}}
\pgfpathlineto{\pgfpoint{21.819148\du}{14.384106\du}}
\pgfpathlineto{\pgfpoint{21.827545\du}{14.377899\du}}
\pgfpathlineto{\pgfpoint{21.836673\du}{14.372058\du}}
\pgfpathlineto{\pgfpoint{21.845070\du}{14.365486\du}}
\pgfpathlineto{\pgfpoint{21.854927\du}{14.359644\du}}
\pgfpathlineto{\pgfpoint{21.864420\du}{14.353803\du}}
\pgfpathlineto{\pgfpoint{21.874278\du}{14.347961\du}}
\pgfpathlineto{\pgfpoint{21.884500\du}{14.342120\du}}
\pgfpathlineto{\pgfpoint{21.894723\du}{14.337008\du}}
\pgfpathlineto{\pgfpoint{21.905676\du}{14.331167\du}}
\pgfpathlineto{\pgfpoint{21.916264\du}{14.326421\du}}
\pgfpathlineto{\pgfpoint{21.927582\du}{14.320944\du}}
\pgfpathlineto{\pgfpoint{21.938535\du}{14.315833\du}}
\pgfpathlineto{\pgfpoint{21.950218\du}{14.311087\du}}
\pgfpathlineto{\pgfpoint{21.961536\du}{14.306340\du}}
\pgfpathlineto{\pgfpoint{21.973949\du}{14.301594\du}}
\pgfpathlineto{\pgfpoint{21.986362\du}{14.297578\du}}
\pgfpathlineto{\pgfpoint{21.998411\du}{14.292832\du}}
\pgfpathlineto{\pgfpoint{22.011189\du}{14.288816\du}}
\pgfpathlineto{\pgfpoint{22.024332\du}{14.285530\du}}
\pgfpathlineto{\pgfpoint{22.037111\du}{14.281149\du}}
\pgfpathlineto{\pgfpoint{22.050619\du}{14.277863\du}}
\pgfpathlineto{\pgfpoint{22.063763\du}{14.274212\du}}
\pgfpathlineto{\pgfpoint{22.077271\du}{14.271291\du}}
\pgfpathlineto{\pgfpoint{22.091145\du}{14.268370\du}}
\pgfpathlineto{\pgfpoint{22.105384\du}{14.265449\du}}
\pgfpathlineto{\pgfpoint{22.119258\du}{14.262529\du}}
\pgfpathlineto{\pgfpoint{22.133131\du}{14.260338\du}}
\pgfpathlineto{\pgfpoint{22.147370\du}{14.257782\du}}
\pgfpathlineto{\pgfpoint{22.161974\du}{14.255592\du}}
\pgfpathlineto{\pgfpoint{22.176578\du}{14.253766\du}}
\pgfpathlineto{\pgfpoint{22.191547\du}{14.251941\du}}
\pgfpathlineto{\pgfpoint{22.206516\du}{14.250115\du}}
\pgfpathlineto{\pgfpoint{22.221485\du}{14.249020\du}}
\pgfpathlineto{\pgfpoint{22.236454\du}{14.247925\du}}
\pgfpathlineto{\pgfpoint{22.252153\du}{14.246830\du}}
\pgfpathlineto{\pgfpoint{22.266757\du}{14.246099\du}}
\pgfpathlineto{\pgfpoint{22.282821\du}{14.245734\du}}
\pgfpathlineto{\pgfpoint{22.298155\du}{14.245004\du}}
\pgfpathlineto{\pgfpoint{22.313854\du}{14.245004\du}}
\pgfpathlineto{\pgfpoint{22.313854\du}{14.245004\du}}
\pgfpathlineto{\pgfpoint{22.313854\du}{14.245004\du}}
\pgfpathlineto{\pgfpoint{22.315314\du}{14.245004\du}}
\pgfpathlineto{\pgfpoint{22.316410\du}{14.245004\du}}
\pgfpathlineto{\pgfpoint{22.317505\du}{14.245004\du}}
\pgfpathlineto{\pgfpoint{22.318600\du}{14.244274\du}}
\pgfpathlineto{\pgfpoint{22.319696\du}{14.243909\du}}
\pgfpathlineto{\pgfpoint{22.320791\du}{14.243179\du}}
\pgfpathlineto{\pgfpoint{22.320791\du}{14.242813\du}}
\pgfpathlineto{\pgfpoint{22.321521\du}{14.242083\du}}
\pgfpathlineto{\pgfpoint{22.322616\du}{14.240258\du}}
\pgfpathlineto{\pgfpoint{22.323712\du}{14.238432\du}}
\pgfpathlineto{\pgfpoint{22.324077\du}{14.236972\du}}
\pgfpathlineto{\pgfpoint{22.324077\du}{14.235146\du}}
\pgfpathlineto{\pgfpoint{22.324077\du}{14.232591\du}}
\pgfpathlineto{\pgfpoint{22.323712\du}{14.231130\du}}
\pgfpathlineto{\pgfpoint{22.322616\du}{14.229305\du}}
\pgfpathlineto{\pgfpoint{22.321521\du}{14.228210\du}}
\pgfpathlineto{\pgfpoint{22.320791\du}{14.226749\du}}
\pgfpathlineto{\pgfpoint{22.320791\du}{14.226384\du}}
\pgfpathlineto{\pgfpoint{22.319696\du}{14.225654\du}}
\pgfpathlineto{\pgfpoint{22.318600\du}{14.225654\du}}
\pgfpathlineto{\pgfpoint{22.317505\du}{14.224924\du}}
\pgfpathlineto{\pgfpoint{22.316410\du}{14.224924\du}}
\pgfpathlineto{\pgfpoint{22.315314\du}{14.224559\du}}
\pgfpathlineto{\pgfpoint{22.313854\du}{14.224559\du}}
\pgfusepath{fill}
\pgfsetbuttcap
\pgfsetmiterjoin
\pgfsetdash{}{0pt}
\definecolor{dialinecolor}{rgb}{0.678431, 0.839216, 0.905882}
\pgfsetfillcolor{dialinecolor}
\pgfpathmoveto{\pgfpoint{22.938900\du}{14.570305\du}}
\pgfpathlineto{\pgfpoint{22.938900\du}{14.560813\du}}
\pgfpathlineto{\pgfpoint{22.938170\du}{14.552050\du}}
\pgfpathlineto{\pgfpoint{22.937074\du}{14.542923\du}}
\pgfpathlineto{\pgfpoint{22.935614\du}{14.534161\du}}
\pgfpathlineto{\pgfpoint{22.933423\du}{14.525398\du}}
\pgfpathlineto{\pgfpoint{22.931598\du}{14.516636\du}}
\pgfpathlineto{\pgfpoint{22.928677\du}{14.507874\du}}
\pgfpathlineto{\pgfpoint{22.926121\du}{14.499111\du}}
\pgfpathlineto{\pgfpoint{22.922105\du}{14.490714\du}}
\pgfpathlineto{\pgfpoint{22.918819\du}{14.481952\du}}
\pgfpathlineto{\pgfpoint{22.914438\du}{14.473920\du}}
\pgfpathlineto{\pgfpoint{22.909692\du}{14.465522\du}}
\pgfpathlineto{\pgfpoint{22.904946\du}{14.457490\du}}
\pgfpathlineto{\pgfpoint{22.900199\du}{14.449093\du}}
\pgfpathlineto{\pgfpoint{22.894358\du}{14.441426\du}}
\pgfpathlineto{\pgfpoint{22.888151\du}{14.433394\du}}
\pgfpathlineto{\pgfpoint{22.881945\du}{14.425727\du}}
\pgfpathlineto{\pgfpoint{22.875373\du}{14.418060\du}}
\pgfpathlineto{\pgfpoint{22.868801\du}{14.410758\du}}
\pgfpathlineto{\pgfpoint{22.861134\du}{14.403091\du}}
\pgfpathlineto{\pgfpoint{22.854197\du}{14.395789\du}}
\pgfpathlineto{\pgfpoint{22.846165\du}{14.388852\du}}
\pgfpathlineto{\pgfpoint{22.838133\du}{14.381915\du}}
\pgfpathlineto{\pgfpoint{22.829736\du}{14.374978\du}}
\pgfpathlineto{\pgfpoint{22.821339\du}{14.368042\du}}
\pgfpathlineto{\pgfpoint{22.812576\du}{14.361470\du}}
\pgfpathlineto{\pgfpoint{22.803449\du}{14.354898\du}}
\pgfpathlineto{\pgfpoint{22.793956\du}{14.348692\du}}
\pgfpathlineto{\pgfpoint{22.783734\du}{14.342120\du}}
\pgfpathlineto{\pgfpoint{22.773876\du}{14.336278\du}}
\pgfpathlineto{\pgfpoint{22.763653\du}{14.329706\du}}
\pgfpathlineto{\pgfpoint{22.753796\du}{14.323865\du}}
\pgfpathlineto{\pgfpoint{22.742478\du}{14.318754\du}}
\pgfpathlineto{\pgfpoint{22.731525\du}{14.312912\du}}
\pgfpathlineto{\pgfpoint{22.720207\du}{14.307436\du}}
\pgfpathlineto{\pgfpoint{22.708889\du}{14.302324\du}}
\pgfpathlineto{\pgfpoint{22.697206\du}{14.297213\du}}
\pgfpathlineto{\pgfpoint{22.685888\du}{14.292467\du}}
\pgfpathlineto{\pgfpoint{22.673109\du}{14.286990\du}}
\pgfpathlineto{\pgfpoint{22.661426\du}{14.282609\du}}
\pgfpathlineto{\pgfpoint{22.648648\du}{14.278228\du}}
\pgfpathlineto{\pgfpoint{22.635869\du}{14.273847\du}}
\pgfpathlineto{\pgfpoint{22.622726\du}{14.269466\du}}
\pgfpathlineto{\pgfpoint{22.609582\du}{14.265449\du}}
\pgfpathlineto{\pgfpoint{22.596439\du}{14.261798\du}}
\pgfpathlineto{\pgfpoint{22.582930\du}{14.257782\du}}
\pgfpathlineto{\pgfpoint{22.569057\du}{14.254497\du}}
\pgfpathlineto{\pgfpoint{22.555548\du}{14.250846\du}}
\pgfpathlineto{\pgfpoint{22.541674\du}{14.247925\du}}
\pgfpathlineto{\pgfpoint{22.527071\du}{14.245004\du}}
\pgfpathlineto{\pgfpoint{22.513197\du}{14.242813\du}}
\pgfpathlineto{\pgfpoint{22.498958\du}{14.239893\du}}
\pgfpathlineto{\pgfpoint{22.483624\du}{14.237337\du}}
\pgfpathlineto{\pgfpoint{22.468655\du}{14.235146\du}}
\pgfpathlineto{\pgfpoint{22.453686\du}{14.233321\du}}
\pgfpathlineto{\pgfpoint{22.439447\du}{14.231495\du}}
\pgfpathlineto{\pgfpoint{22.423383\du}{14.229670\du}}
\pgfpathlineto{\pgfpoint{22.408414\du}{14.228575\du}}
\pgfpathlineto{\pgfpoint{22.393080\du}{14.227479\du}}
\pgfpathlineto{\pgfpoint{22.377381\du}{14.226384\du}}
\pgfpathlineto{\pgfpoint{22.362047\du}{14.225654\du}}
\pgfpathlineto{\pgfpoint{22.345983\du}{14.224924\du}}
\pgfpathlineto{\pgfpoint{22.329918\du}{14.224559\du}}
\pgfpathlineto{\pgfpoint{22.313854\du}{14.224559\du}}
\pgfpathlineto{\pgfpoint{22.313854\du}{14.245004\du}}
\pgfpathlineto{\pgfpoint{22.329918\du}{14.245004\du}}
\pgfpathlineto{\pgfpoint{22.345252\du}{14.245734\du}}
\pgfpathlineto{\pgfpoint{22.360952\du}{14.246099\du}}
\pgfpathlineto{\pgfpoint{22.376286\du}{14.246830\du}}
\pgfpathlineto{\pgfpoint{22.391255\du}{14.247925\du}}
\pgfpathlineto{\pgfpoint{22.406954\du}{14.249020\du}}
\pgfpathlineto{\pgfpoint{22.421558\du}{14.250115\du}}
\pgfpathlineto{\pgfpoint{22.436892\du}{14.251941\du}}
\pgfpathlineto{\pgfpoint{22.451861\du}{14.253766\du}}
\pgfpathlineto{\pgfpoint{22.466465\du}{14.255592\du}}
\pgfpathlineto{\pgfpoint{22.480703\du}{14.257782\du}}
\pgfpathlineto{\pgfpoint{22.494942\du}{14.260338\du}}
\pgfpathlineto{\pgfpoint{22.509181\du}{14.262529\du}}
\pgfpathlineto{\pgfpoint{22.523055\du}{14.265449\du}}
\pgfpathlineto{\pgfpoint{22.536928\du}{14.268370\du}}
\pgfpathlineto{\pgfpoint{22.550802\du}{14.271291\du}}
\pgfpathlineto{\pgfpoint{22.564310\du}{14.274212\du}}
\pgfpathlineto{\pgfpoint{22.577819\du}{14.277863\du}}
\pgfpathlineto{\pgfpoint{22.590963\du}{14.281149\du}}
\pgfpathlineto{\pgfpoint{22.603741\du}{14.285530\du}}
\pgfpathlineto{\pgfpoint{22.617250\du}{14.288816\du}}
\pgfpathlineto{\pgfpoint{22.630028\du}{14.292832\du}}
\pgfpathlineto{\pgfpoint{22.641711\du}{14.297578\du}}
\pgfpathlineto{\pgfpoint{22.654124\du}{14.301594\du}}
\pgfpathlineto{\pgfpoint{22.666172\du}{14.306340\du}}
\pgfpathlineto{\pgfpoint{22.677856\du}{14.311087\du}}
\pgfpathlineto{\pgfpoint{22.689174\du}{14.315833\du}}
\pgfpathlineto{\pgfpoint{22.700492\du}{14.320944\du}}
\pgfpathlineto{\pgfpoint{22.712175\du}{14.326421\du}}
\pgfpathlineto{\pgfpoint{22.722762\du}{14.331167\du}}
\pgfpathlineto{\pgfpoint{22.733350\du}{14.337008\du}}
\pgfpathlineto{\pgfpoint{22.743938\du}{14.342120\du}}
\pgfpathlineto{\pgfpoint{22.753796\du}{14.347961\du}}
\pgfpathlineto{\pgfpoint{22.763653\du}{14.353803\du}}
\pgfpathlineto{\pgfpoint{22.772781\du}{14.359644\du}}
\pgfpathlineto{\pgfpoint{22.781908\du}{14.365486\du}}
\pgfpathlineto{\pgfpoint{22.791401\du}{14.372058\du}}
\pgfpathlineto{\pgfpoint{22.800163\du}{14.377899\du}}
\pgfpathlineto{\pgfpoint{22.809290\du}{14.384106\du}}
\pgfpathlineto{\pgfpoint{22.817323\du}{14.390678\du}}
\pgfpathlineto{\pgfpoint{22.825720\du}{14.397249\du}}
\pgfpathlineto{\pgfpoint{22.832657\du}{14.404186\du}}
\pgfpathlineto{\pgfpoint{22.839959\du}{14.411123\du}}
\pgfpathlineto{\pgfpoint{22.846895\du}{14.417695\du}}
\pgfpathlineto{\pgfpoint{22.854197\du}{14.424632\du}}
\pgfpathlineto{\pgfpoint{22.860039\du}{14.431568\du}}
\pgfpathlineto{\pgfpoint{22.866976\du}{14.438505\du}}
\pgfpathlineto{\pgfpoint{22.872087\du}{14.446172\du}}
\pgfpathlineto{\pgfpoint{22.877563\du}{14.453474\du}}
\pgfpathlineto{\pgfpoint{22.882310\du}{14.460776\du}}
\pgfpathlineto{\pgfpoint{22.887421\du}{14.468443\du}}
\pgfpathlineto{\pgfpoint{22.891437\du}{14.475380\du}}
\pgfpathlineto{\pgfpoint{22.896183\du}{14.483047\du}}
\pgfpathlineto{\pgfpoint{22.900199\du}{14.490714\du}}
\pgfpathlineto{\pgfpoint{22.903485\du}{14.499111\du}}
\pgfpathlineto{\pgfpoint{22.906041\du}{14.506413\du}}
\pgfpathlineto{\pgfpoint{22.909327\du}{14.514080\du}}
\pgfpathlineto{\pgfpoint{22.911517\du}{14.521747\du}}
\pgfpathlineto{\pgfpoint{22.914073\du}{14.529780\du}}
\pgfpathlineto{\pgfpoint{22.915534\du}{14.538177\du}}
\pgfpathlineto{\pgfpoint{22.916994\du}{14.545844\du}}
\pgfpathlineto{\pgfpoint{22.917724\du}{14.553876\du}}
\pgfpathlineto{\pgfpoint{22.918089\du}{14.562273\du}}
\pgfpathlineto{\pgfpoint{22.918819\du}{14.570305\du}}
\pgfpathlineto{\pgfpoint{22.938900\du}{14.570305\du}}
\pgfusepath{fill}
\pgfsetbuttcap
\pgfsetmiterjoin
\pgfsetdash{}{0pt}
\definecolor{dialinecolor}{rgb}{0.074510, 0.082353, 0.086275}
\pgfsetfillcolor{dialinecolor}
\pgfpathmoveto{\pgfpoint{21.992934\du}{14.664500\du}}
\pgfpathlineto{\pgfpoint{22.220389\du}{14.436315\du}}
\pgfpathlineto{\pgfpoint{22.160513\du}{14.375344\du}}
\pgfpathlineto{\pgfpoint{22.340871\du}{14.375344\du}}
\pgfpathlineto{\pgfpoint{22.340871\du}{14.563734\du}}
\pgfpathlineto{\pgfpoint{22.280630\du}{14.503493\du}}
\pgfpathlineto{\pgfpoint{22.060477\du}{14.724741\du}}
\pgfpathlineto{\pgfpoint{21.992934\du}{14.664500\du}}
\pgfusepath{fill}
\pgfsetbuttcap
\pgfsetmiterjoin
\pgfsetdash{}{0pt}
\definecolor{dialinecolor}{rgb}{0.074510, 0.082353, 0.086275}
\pgfsetfillcolor{dialinecolor}
\pgfpathmoveto{\pgfpoint{22.260915\du}{14.784982\du}}
\pgfpathlineto{\pgfpoint{22.488005\du}{14.556797\du}}
\pgfpathlineto{\pgfpoint{22.427399\du}{14.496556\du}}
\pgfpathlineto{\pgfpoint{22.608487\du}{14.496556\du}}
\pgfpathlineto{\pgfpoint{22.608487\du}{14.684581\du}}
\pgfpathlineto{\pgfpoint{22.547881\du}{14.624340\du}}
\pgfpathlineto{\pgfpoint{22.327363\du}{14.845223\du}}
\pgfpathlineto{\pgfpoint{22.260915\du}{14.784982\du}}
\pgfusepath{fill}
\pgfsetbuttcap
\pgfsetmiterjoin
\pgfsetdash{}{0pt}
\definecolor{dialinecolor}{rgb}{1.000000, 1.000000, 1.000000}
\pgfsetfillcolor{dialinecolor}
\pgfpathmoveto{\pgfpoint{21.979791\du}{14.650992\du}}
\pgfpathlineto{\pgfpoint{22.206881\du}{14.422806\du}}
\pgfpathlineto{\pgfpoint{22.147370\du}{14.362565\du}}
\pgfpathlineto{\pgfpoint{22.327363\du}{14.362565\du}}
\pgfpathlineto{\pgfpoint{22.327363\du}{14.550590\du}}
\pgfpathlineto{\pgfpoint{22.267487\du}{14.489619\du}}
\pgfpathlineto{\pgfpoint{22.046968\du}{14.711233\du}}
\pgfpathlineto{\pgfpoint{21.979791\du}{14.650992\du}}
\pgfusepath{fill}
\pgfsetbuttcap
\pgfsetmiterjoin
\pgfsetdash{}{0pt}
\definecolor{dialinecolor}{rgb}{1.000000, 1.000000, 1.000000}
\pgfsetfillcolor{dialinecolor}
\pgfpathmoveto{\pgfpoint{22.247406\du}{14.771474\du}}
\pgfpathlineto{\pgfpoint{22.474497\du}{14.543288\du}}
\pgfpathlineto{\pgfpoint{22.414256\du}{14.483047\du}}
\pgfpathlineto{\pgfpoint{22.594614\du}{14.483047\du}}
\pgfpathlineto{\pgfpoint{22.594614\du}{14.671072\du}}
\pgfpathlineto{\pgfpoint{22.535103\du}{14.610831\du}}
\pgfpathlineto{\pgfpoint{22.313854\du}{14.831715\du}}
\pgfpathlineto{\pgfpoint{22.247406\du}{14.771474\du}}
\pgfusepath{fill}
% setfont left to latex
\definecolor{dialinecolor}{rgb}{0.000000, 0.000000, 0.000000}
\pgfsetstrokecolor{dialinecolor}
\node[anchor=west] at (20.049643\du,9.602082\du){Hop : 4};
% setfont left to latex
\definecolor{dialinecolor}{rgb}{0.000000, 0.000000, 0.000000}
\pgfsetstrokecolor{dialinecolor}
\node[anchor=west] at (19.911775\du,10.742641\du){IP : 160.242.100.88};
% setfont left to latex
\definecolor{dialinecolor}{rgb}{0.000000, 0.000000, 0.000000}
\pgfsetstrokecolor{dialinecolor}
\node[anchor=west] at (29.951681\du,10.739592\du){IP : 196.216.48.144};
% setfont left to latex
\definecolor{dialinecolor}{rgb}{0.000000, 0.000000, 0.000000}
\pgfsetstrokecolor{dialinecolor}
\node[anchor=west] at (29.830970\du,11.981014\du){RTTs : 4.263, 6.082, 11.834};
% setfont left to latex
\definecolor{dialinecolor}{rgb}{0.000000, 0.000000, 0.000000}
\pgfsetstrokecolor{dialinecolor}
\node[anchor=west] at (29.968838\du,9.441926\du){Hop : 5};
% setfont left to latex
\definecolor{dialinecolor}{rgb}{0.000000, 0.000000, 0.000000}
\pgfsetstrokecolor{dialinecolor}
\node[anchor=west] at (40.121782\du,9.611775\du){Hop : 6};
% setfont left to latex
\definecolor{dialinecolor}{rgb}{0.000000, 0.000000, 0.000000}
\pgfsetstrokecolor{dialinecolor}
\node[anchor=west] at (39.980361\du,10.803196\du){IP : 193.239.116.112};
% setfont left to latex
\definecolor{dialinecolor}{rgb}{0.000000, 0.000000, 0.000000}
\pgfsetstrokecolor{dialinecolor}
\node[anchor=west] at (40.018229\du,12.015328\du){RTTs : 7.802, 7.691, 7.711};
\pgfsetlinewidth{0.000000\du}
\pgfsetdash{}{0pt}
\pgfsetdash{}{0pt}
\pgfsetbuttcap
\pgfsetmiterjoin
\pgfsetlinewidth{0.000000\du}
\pgfsetbuttcap
\pgfsetmiterjoin
\pgfsetdash{}{0pt}
\definecolor{dialinecolor}{rgb}{0.027451, 0.486275, 0.682353}
\pgfsetfillcolor{dialinecolor}
\pgfpathmoveto{\pgfpoint{34.626493\du}{14.355045\du}}
\pgfpathlineto{\pgfpoint{34.625033\du}{14.384253\du}}
\pgfpathlineto{\pgfpoint{34.617731\du}{14.414191\du}}
\pgfpathlineto{\pgfpoint{34.607508\du}{14.442668\du}}
\pgfpathlineto{\pgfpoint{34.592904\du}{14.470781\du}}
\pgfpathlineto{\pgfpoint{34.573554\du}{14.498893\du}}
\pgfpathlineto{\pgfpoint{34.551648\du}{14.526275\du}}
\pgfpathlineto{\pgfpoint{34.524996\du}{14.553292\du}}
\pgfpathlineto{\pgfpoint{34.494328\du}{14.579579\du}}
\pgfpathlineto{\pgfpoint{34.461469\du}{14.604771\du}}
\pgfpathlineto{\pgfpoint{34.424229\du}{14.629963\du}}
\pgfpathlineto{\pgfpoint{34.383339\du}{14.654059\du}}
\pgfpathlineto{\pgfpoint{34.339527\du}{14.677425\du}}
\pgfpathlineto{\pgfpoint{34.293160\du}{14.700061\du}}
\pgfpathlineto{\pgfpoint{34.242776\du}{14.721967\du}}
\pgfpathlineto{\pgfpoint{34.189837\du}{14.742778\du}}
\pgfpathlineto{\pgfpoint{34.134343\du}{14.762858\du}}
\pgfpathlineto{\pgfpoint{34.076292\du}{14.782208\du}}
\pgfpathlineto{\pgfpoint{34.015686\du}{14.800098\du}}
\pgfpathlineto{\pgfpoint{33.951794\du}{14.817257\du}}
\pgfpathlineto{\pgfpoint{33.886807\du}{14.833687\du}}
\pgfpathlineto{\pgfpoint{33.818169\du}{14.848656\du}}
\pgfpathlineto{\pgfpoint{33.747340\du}{14.862164\du}}
\pgfpathlineto{\pgfpoint{33.675416\du}{14.874943\du}}
\pgfpathlineto{\pgfpoint{33.600571\du}{14.886991\du}}
\pgfpathlineto{\pgfpoint{33.524631\du}{14.896848\du}}
\pgfpathlineto{\pgfpoint{33.446135\du}{14.905976\du}}
\pgfpathlineto{\pgfpoint{33.366544\du}{14.913643\du}}
\pgfpathlineto{\pgfpoint{33.285493\du}{14.920215\du}}
\pgfpathlineto{\pgfpoint{33.202251\du}{14.925326\du}}
\pgfpathlineto{\pgfpoint{33.118644\du}{14.928977\du}}
\pgfpathlineto{\pgfpoint{33.033211\du}{14.931168\du}}
\pgfpathlineto{\pgfpoint{32.946683\du}{14.931898\du}}
\pgfpathlineto{\pgfpoint{32.860520\du}{14.931168\du}}
\pgfpathlineto{\pgfpoint{32.774722\du}{14.928977\du}}
\pgfpathlineto{\pgfpoint{32.691115\du}{14.925326\du}}
\pgfpathlineto{\pgfpoint{32.608238\du}{14.920215\du}}
\pgfpathlineto{\pgfpoint{32.526822\du}{14.913643\du}}
\pgfpathlineto{\pgfpoint{32.447231\du}{14.905976\du}}
\pgfpathlineto{\pgfpoint{32.369465\du}{14.896848\du}}
\pgfpathlineto{\pgfpoint{32.292795\du}{14.886991\du}}
\pgfpathlineto{\pgfpoint{32.218680\du}{14.874943\du}}
\pgfpathlineto{\pgfpoint{32.146026\du}{14.862164\du}}
\pgfpathlineto{\pgfpoint{32.075562\du}{14.848656\du}}
\pgfpathlineto{\pgfpoint{32.006924\du}{14.833687\du}}
\pgfpathlineto{\pgfpoint{31.941206\du}{14.817257\du}}
\pgfpathlineto{\pgfpoint{31.877680\du}{14.800098\du}}
\pgfpathlineto{\pgfpoint{31.816708\du}{14.782208\du}}
\pgfpathlineto{\pgfpoint{31.758293\du}{14.762858\du}}
\pgfpathlineto{\pgfpoint{31.703163\du}{14.742778\du}}
\pgfpathlineto{\pgfpoint{31.650224\du}{14.721967\du}}
\pgfpathlineto{\pgfpoint{31.600206\du}{14.700061\du}}
\pgfpathlineto{\pgfpoint{31.553109\du}{14.677425\du}}
\pgfpathlineto{\pgfpoint{31.509662\du}{14.654059\du}}
\pgfpathlineto{\pgfpoint{31.468771\du}{14.629963\du}}
\pgfpathlineto{\pgfpoint{31.431531\du}{14.604771\du}}
\pgfpathlineto{\pgfpoint{31.398308\du}{14.579579\du}}
\pgfpathlineto{\pgfpoint{31.368005\du}{14.553292\du}}
\pgfpathlineto{\pgfpoint{31.341353\du}{14.526275\du}}
\pgfpathlineto{\pgfpoint{31.319447\du}{14.498893\du}}
\pgfpathlineto{\pgfpoint{31.300097\du}{14.470781\du}}
\pgfpathlineto{\pgfpoint{31.285493\du}{14.442668\du}}
\pgfpathlineto{\pgfpoint{31.274905\du}{14.414191\du}}
\pgfpathlineto{\pgfpoint{31.267968\du}{14.384253\du}}
\pgfpathlineto{\pgfpoint{31.266143\du}{14.355045\du}}
\pgfpathlineto{\pgfpoint{31.267968\du}{14.325107\du}}
\pgfpathlineto{\pgfpoint{31.274905\du}{14.295899\du}}
\pgfpathlineto{\pgfpoint{31.285493\du}{14.266691\du}}
\pgfpathlineto{\pgfpoint{31.300097\du}{14.238579\du}}
\pgfpathlineto{\pgfpoint{31.319447\du}{14.210467\du}}
\pgfpathlineto{\pgfpoint{31.341353\du}{14.183084\du}}
\pgfpathlineto{\pgfpoint{31.368005\du}{14.156432\du}}
\pgfpathlineto{\pgfpoint{31.398308\du}{14.130145\du}}
\pgfpathlineto{\pgfpoint{31.431531\du}{14.104589\du}}
\pgfpathlineto{\pgfpoint{31.468771\du}{14.079762\du}}
\pgfpathlineto{\pgfpoint{31.509662\du}{14.055300\du}}
\pgfpathlineto{\pgfpoint{31.553109\du}{14.031934\du}}
\pgfpathlineto{\pgfpoint{31.600206\du}{14.009663\du}}
\pgfpathlineto{\pgfpoint{31.650224\du}{13.987392\du}}
\pgfpathlineto{\pgfpoint{31.703163\du}{13.966582\du}}
\pgfpathlineto{\pgfpoint{31.758293\du}{13.946502\du}}
\pgfpathlineto{\pgfpoint{31.816708\du}{13.927882\du}}
\pgfpathlineto{\pgfpoint{31.877680\du}{13.908897\du}}
\pgfpathlineto{\pgfpoint{31.941206\du}{13.892102\du}}
\pgfpathlineto{\pgfpoint{32.006924\du}{13.876403\du}}
\pgfpathlineto{\pgfpoint{32.075562\du}{13.861069\du}}
\pgfpathlineto{\pgfpoint{32.146026\du}{13.847195\du}}
\pgfpathlineto{\pgfpoint{32.218680\du}{13.834052\du}}
\pgfpathlineto{\pgfpoint{32.292795\du}{13.823099\du}}
\pgfpathlineto{\pgfpoint{32.369465\du}{13.812511\du}}
\pgfpathlineto{\pgfpoint{32.447231\du}{13.803019\du}}
\pgfpathlineto{\pgfpoint{32.526822\du}{13.795717\du}}
\pgfpathlineto{\pgfpoint{32.608238\du}{13.789145\du}}
\pgfpathlineto{\pgfpoint{32.691115\du}{13.784034\du}}
\pgfpathlineto{\pgfpoint{32.774722\du}{13.780383\du}}
\pgfpathlineto{\pgfpoint{32.860520\du}{13.778557\du}}
\pgfpathlineto{\pgfpoint{32.946683\du}{13.777462\du}}
\pgfpathlineto{\pgfpoint{33.033211\du}{13.778557\du}}
\pgfpathlineto{\pgfpoint{33.118644\du}{13.780383\du}}
\pgfpathlineto{\pgfpoint{33.202251\du}{13.784034\du}}
\pgfpathlineto{\pgfpoint{33.285493\du}{13.789145\du}}
\pgfpathlineto{\pgfpoint{33.366544\du}{13.795717\du}}
\pgfpathlineto{\pgfpoint{33.446135\du}{13.803019\du}}
\pgfpathlineto{\pgfpoint{33.524631\du}{13.812511\du}}
\pgfpathlineto{\pgfpoint{33.600571\du}{13.823099\du}}
\pgfpathlineto{\pgfpoint{33.675416\du}{13.834052\du}}
\pgfpathlineto{\pgfpoint{33.747340\du}{13.847195\du}}
\pgfpathlineto{\pgfpoint{33.818169\du}{13.861069\du}}
\pgfpathlineto{\pgfpoint{33.886807\du}{13.876403\du}}
\pgfpathlineto{\pgfpoint{33.951794\du}{13.892102\du}}
\pgfpathlineto{\pgfpoint{34.015686\du}{13.908897\du}}
\pgfpathlineto{\pgfpoint{34.076292\du}{13.927882\du}}
\pgfpathlineto{\pgfpoint{34.134343\du}{13.946502\du}}
\pgfpathlineto{\pgfpoint{34.189837\du}{13.966582\du}}
\pgfpathlineto{\pgfpoint{34.242776\du}{13.987392\du}}
\pgfpathlineto{\pgfpoint{34.293160\du}{14.009663\du}}
\pgfpathlineto{\pgfpoint{34.339527\du}{14.031934\du}}
\pgfpathlineto{\pgfpoint{34.383339\du}{14.055300\du}}
\pgfpathlineto{\pgfpoint{34.424229\du}{14.079762\du}}
\pgfpathlineto{\pgfpoint{34.461469\du}{14.104589\du}}
\pgfpathlineto{\pgfpoint{34.494328\du}{14.130145\du}}
\pgfpathlineto{\pgfpoint{34.524996\du}{14.156432\du}}
\pgfpathlineto{\pgfpoint{34.551648\du}{14.183084\du}}
\pgfpathlineto{\pgfpoint{34.573554\du}{14.210467\du}}
\pgfpathlineto{\pgfpoint{34.592904\du}{14.238579\du}}
\pgfpathlineto{\pgfpoint{34.607508\du}{14.266691\du}}
\pgfpathlineto{\pgfpoint{34.617731\du}{14.295899\du}}
\pgfpathlineto{\pgfpoint{34.625033\du}{14.325107\du}}
\pgfpathlineto{\pgfpoint{34.626493\du}{14.355045\du}}
\pgfusepath{fill}
\pgfsetlinewidth{0.000000\du}
\pgfsetbuttcap
\pgfsetmiterjoin
\pgfsetdash{}{0pt}
\definecolor{dialinecolor}{rgb}{0.678431, 0.839216, 0.905882}
\pgfsetfillcolor{dialinecolor}
\pgfpathmoveto{\pgfpoint{32.946683\du}{14.942486\du}}
\pgfpathlineto{\pgfpoint{32.946683\du}{14.942486\du}}
\pgfpathlineto{\pgfpoint{32.990129\du}{14.942486\du}}
\pgfpathlineto{\pgfpoint{33.033576\du}{14.941755\du}}
\pgfpathlineto{\pgfpoint{33.076657\du}{14.940660\du}}
\pgfpathlineto{\pgfpoint{33.118644\du}{14.939565\du}}
\pgfpathlineto{\pgfpoint{33.161360\du}{14.937739\du}}
\pgfpathlineto{\pgfpoint{33.202981\du}{14.935549\du}}
\pgfpathlineto{\pgfpoint{33.244602\du}{14.932993\du}}
\pgfpathlineto{\pgfpoint{33.286223\du}{14.930802\du}}
\pgfpathlineto{\pgfpoint{33.326749\du}{14.927882\du}}
\pgfpathlineto{\pgfpoint{33.367639\du}{14.924231\du}}
\pgfpathlineto{\pgfpoint{33.407435\du}{14.920215\du}}
\pgfpathlineto{\pgfpoint{33.447596\du}{14.916199\du}}
\pgfpathlineto{\pgfpoint{33.486296\du}{14.911817\du}}
\pgfpathlineto{\pgfpoint{33.525726\du}{14.907436\du}}
\pgfpathlineto{\pgfpoint{33.563696\du}{14.901960\du}}
\pgfpathlineto{\pgfpoint{33.602762\du}{14.896848\du}}
\pgfpathlineto{\pgfpoint{33.640002\du}{14.891372\du}}
\pgfpathlineto{\pgfpoint{33.676876\du}{14.885165\du}}
\pgfpathlineto{\pgfpoint{33.713386\du}{14.879324\du}}
\pgfpathlineto{\pgfpoint{33.749896\du}{14.872752\du}}
\pgfpathlineto{\pgfpoint{33.785310\du}{14.865815\du}}
\pgfpathlineto{\pgfpoint{33.819994\du}{14.858878\du}}
\pgfpathlineto{\pgfpoint{33.854679\du}{14.851211\du}}
\pgfpathlineto{\pgfpoint{33.888633\du}{14.843544\du}}
\pgfpathlineto{\pgfpoint{33.922221\du}{14.835147\du}}
\pgfpathlineto{\pgfpoint{33.955080\du}{14.827115\du}}
\pgfpathlineto{\pgfpoint{33.986844\du}{14.819083\du}}
\pgfpathlineto{\pgfpoint{34.018242\du}{14.810321\du}}
\pgfpathlineto{\pgfpoint{34.033576\du}{14.805574\du}}
\pgfpathlineto{\pgfpoint{34.048910\du}{14.801558\du}}
\pgfpathlineto{\pgfpoint{34.064974\du}{14.796812\du}}
\pgfpathlineto{\pgfpoint{34.079578\du}{14.792066\du}}
\pgfpathlineto{\pgfpoint{34.093817\du}{14.787319\du}}
\pgfpathlineto{\pgfpoint{34.108786\du}{14.782208\du}}
\pgfpathlineto{\pgfpoint{34.123755\du}{14.777462\du}}
\pgfpathlineto{\pgfpoint{34.137629\du}{14.772716\du}}
\pgfpathlineto{\pgfpoint{34.151867\du}{14.767604\du}}
\pgfpathlineto{\pgfpoint{34.165741\du}{14.762858\du}}
\pgfpathlineto{\pgfpoint{34.179980\du}{14.757382\du}}
\pgfpathlineto{\pgfpoint{34.193123\du}{14.753000\du}}
\pgfpathlineto{\pgfpoint{34.207362\du}{14.747524\du}}
\pgfpathlineto{\pgfpoint{34.220505\du}{14.742413\du}}
\pgfpathlineto{\pgfpoint{34.233649\du}{14.736936\du}}
\pgfpathlineto{\pgfpoint{34.247158\du}{14.731095\du}}
\pgfpathlineto{\pgfpoint{34.259936\du}{14.725983\du}}
\pgfpathlineto{\pgfpoint{34.271984\du}{14.720507\du}}
\pgfpathlineto{\pgfpoint{34.284763\du}{14.714665\du}}
\pgfpathlineto{\pgfpoint{34.297176\du}{14.709554\du}}
\pgfpathlineto{\pgfpoint{34.309589\du}{14.703712\du}}
\pgfpathlineto{\pgfpoint{34.320907\du}{14.698236\du}}
\pgfpathlineto{\pgfpoint{34.332955\du}{14.692394\du}}
\pgfpathlineto{\pgfpoint{34.343908\du}{14.686553\du}}
\pgfpathlineto{\pgfpoint{34.355591\du}{14.680711\du}}
\pgfpathlineto{\pgfpoint{34.366909\du}{14.674870\du}}
\pgfpathlineto{\pgfpoint{34.378227\du}{14.669028\du}}
\pgfpathlineto{\pgfpoint{34.388815\du}{14.662821\du}}
\pgfpathlineto{\pgfpoint{34.398673\du}{14.656980\du}}
\pgfpathlineto{\pgfpoint{34.409261\du}{14.651138\du}}
\pgfpathlineto{\pgfpoint{34.419483\du}{14.644567\du}}
\pgfpathlineto{\pgfpoint{34.429341\du}{14.638725\du}}
\pgfpathlineto{\pgfpoint{34.439198\du}{14.632153\du}}
\pgfpathlineto{\pgfpoint{34.448326\du}{14.625947\du}}
\pgfpathlineto{\pgfpoint{34.457453\du}{14.619375\du}}
\pgfpathlineto{\pgfpoint{34.466946\du}{14.613533\du}}
\pgfpathlineto{\pgfpoint{34.475708\du}{14.606962\du}}
\pgfpathlineto{\pgfpoint{34.484836\du}{14.600755\du}}
\pgfpathlineto{\pgfpoint{34.492868\du}{14.594183\du}}
\pgfpathlineto{\pgfpoint{34.501630\du}{14.587246\du}}
\pgfpathlineto{\pgfpoint{34.508932\du}{14.580675\du}}
\pgfpathlineto{\pgfpoint{34.516964\du}{14.574468\du}}
\pgfpathlineto{\pgfpoint{34.524996\du}{14.567166\du}}
\pgfpathlineto{\pgfpoint{34.531568\du}{14.560959\du}}
\pgfpathlineto{\pgfpoint{34.538870\du}{14.554023\du}}
\pgfpathlineto{\pgfpoint{34.545442\du}{14.546721\du}}
\pgfpathlineto{\pgfpoint{34.552378\du}{14.540514\du}}
\pgfpathlineto{\pgfpoint{34.558950\du}{14.533577\du}}
\pgfpathlineto{\pgfpoint{34.565157\du}{14.526275\du}}
\pgfpathlineto{\pgfpoint{34.570998\du}{14.519338\du}}
\pgfpathlineto{\pgfpoint{34.576475\du}{14.512402\du}}
\pgfpathlineto{\pgfpoint{34.582316\du}{14.505465\du}}
\pgfpathlineto{\pgfpoint{34.587063\du}{14.498163\du}}
\pgfpathlineto{\pgfpoint{34.592539\du}{14.490861\du}}
\pgfpathlineto{\pgfpoint{34.597285\du}{14.483559\du}}
\pgfpathlineto{\pgfpoint{34.601667\du}{14.476622\du}}
\pgfpathlineto{\pgfpoint{34.605683\du}{14.468955\du}}
\pgfpathlineto{\pgfpoint{34.609699\du}{14.462018\du}}
\pgfpathlineto{\pgfpoint{34.612985\du}{14.454351\du}}
\pgfpathlineto{\pgfpoint{34.617001\du}{14.446684\du}}
\pgfpathlineto{\pgfpoint{34.620286\du}{14.439017\du}}
\pgfpathlineto{\pgfpoint{34.622842\du}{14.431715\du}}
\pgfpathlineto{\pgfpoint{34.625763\du}{14.424413\du}}
\pgfpathlineto{\pgfpoint{34.627588\du}{14.417111\du}}
\pgfpathlineto{\pgfpoint{34.630509\du}{14.409444\du}}
\pgfpathlineto{\pgfpoint{34.631604\du}{14.401047\du}}
\pgfpathlineto{\pgfpoint{34.633795\du}{14.393380\du}}
\pgfpathlineto{\pgfpoint{34.634890\du}{14.386078\du}}
\pgfpathlineto{\pgfpoint{34.635621\du}{14.378411\du}}
\pgfpathlineto{\pgfpoint{34.636351\du}{14.370014\du}}
\pgfpathlineto{\pgfpoint{34.637081\du}{14.362712\du}}
\pgfpathlineto{\pgfpoint{34.637081\du}{14.355045\du}}
\pgfpathlineto{\pgfpoint{34.617001\du}{14.355045\du}}
\pgfpathlineto{\pgfpoint{34.616270\du}{14.361982\du}}
\pgfpathlineto{\pgfpoint{34.616270\du}{14.368919\du}}
\pgfpathlineto{\pgfpoint{34.615905\du}{14.375855\du}}
\pgfpathlineto{\pgfpoint{34.614080\du}{14.383157\du}}
\pgfpathlineto{\pgfpoint{34.612985\du}{14.390094\du}}
\pgfpathlineto{\pgfpoint{34.612254\du}{14.397031\du}}
\pgfpathlineto{\pgfpoint{34.610064\du}{14.403968\du}}
\pgfpathlineto{\pgfpoint{34.608603\du}{14.411270\du}}
\pgfpathlineto{\pgfpoint{34.606413\du}{14.417476\du}}
\pgfpathlineto{\pgfpoint{34.603857\du}{14.424413\du}}
\pgfpathlineto{\pgfpoint{34.600936\du}{14.431715\du}}
\pgfpathlineto{\pgfpoint{34.598381\du}{14.438652\du}}
\pgfpathlineto{\pgfpoint{34.594365\du}{14.445589\du}}
\pgfpathlineto{\pgfpoint{34.591444\du}{14.452161\du}}
\pgfpathlineto{\pgfpoint{34.587428\du}{14.459097\du}}
\pgfpathlineto{\pgfpoint{34.584507\du}{14.465669\du}}
\pgfpathlineto{\pgfpoint{34.580126\du}{14.472606\du}}
\pgfpathlineto{\pgfpoint{34.575745\du}{14.479543\du}}
\pgfpathlineto{\pgfpoint{34.570998\du}{14.486115\du}}
\pgfpathlineto{\pgfpoint{34.565887\du}{14.492321\du}}
\pgfpathlineto{\pgfpoint{34.561141\du}{14.499623\du}}
\pgfpathlineto{\pgfpoint{34.554934\du}{14.506560\du}}
\pgfpathlineto{\pgfpoint{34.549458\du}{14.512767\du}}
\pgfpathlineto{\pgfpoint{34.544346\du}{14.519338\du}}
\pgfpathlineto{\pgfpoint{34.537775\du}{14.525910\du}}
\pgfpathlineto{\pgfpoint{34.530838\du}{14.532847\du}}
\pgfpathlineto{\pgfpoint{34.524996\du}{14.539419\du}}
\pgfpathlineto{\pgfpoint{34.517694\du}{14.545625\du}}
\pgfpathlineto{\pgfpoint{34.511123\du}{14.552197\du}}
\pgfpathlineto{\pgfpoint{34.503455\du}{14.558404\du}}
\pgfpathlineto{\pgfpoint{34.496154\du}{14.564976\du}}
\pgfpathlineto{\pgfpoint{34.488121\du}{14.571547\du}}
\pgfpathlineto{\pgfpoint{34.480454\du}{14.577754\du}}
\pgfpathlineto{\pgfpoint{34.471692\du}{14.584326\du}}
\pgfpathlineto{\pgfpoint{34.463295\du}{14.590167\du}}
\pgfpathlineto{\pgfpoint{34.454533\du}{14.596739\du}}
\pgfpathlineto{\pgfpoint{34.446500\du}{14.602946\du}}
\pgfpathlineto{\pgfpoint{34.437738\du}{14.608787\du}}
\pgfpathlineto{\pgfpoint{34.428246\du}{14.615359\du}}
\pgfpathlineto{\pgfpoint{34.418023\du}{14.621200\du}}
\pgfpathlineto{\pgfpoint{34.408895\du}{14.627772\du}}
\pgfpathlineto{\pgfpoint{34.398673\du}{14.633614\du}}
\pgfpathlineto{\pgfpoint{34.388815\du}{14.639455\du}}
\pgfpathlineto{\pgfpoint{34.378957\du}{14.645297\du}}
\pgfpathlineto{\pgfpoint{34.368005\du}{14.651138\du}}
\pgfpathlineto{\pgfpoint{34.357052\du}{14.656980\du}}
\pgfpathlineto{\pgfpoint{34.346829\du}{14.662821\du}}
\pgfpathlineto{\pgfpoint{34.334781\du}{14.668663\du}}
\pgfpathlineto{\pgfpoint{34.324193\du}{14.673774\du}}
\pgfpathlineto{\pgfpoint{34.312145\du}{14.679616\du}}
\pgfpathlineto{\pgfpoint{34.300827\du}{14.685457\du}}
\pgfpathlineto{\pgfpoint{34.288413\du}{14.690934\du}}
\pgfpathlineto{\pgfpoint{34.276365\du}{14.696045\du}}
\pgfpathlineto{\pgfpoint{34.263952\du}{14.701887\du}}
\pgfpathlineto{\pgfpoint{34.251904\du}{14.707363\du}}
\pgfpathlineto{\pgfpoint{34.238760\du}{14.712475\du}}
\pgfpathlineto{\pgfpoint{34.225982\du}{14.717586\du}}
\pgfpathlineto{\pgfpoint{34.213204\du}{14.723062\du}}
\pgfpathlineto{\pgfpoint{34.200060\du}{14.728174\du}}
\pgfpathlineto{\pgfpoint{34.186551\du}{14.733650\du}}
\pgfpathlineto{\pgfpoint{34.173408\du}{14.738031\du}}
\pgfpathlineto{\pgfpoint{34.159169\du}{14.743508\du}}
\pgfpathlineto{\pgfpoint{34.145296\du}{14.748254\du}}
\pgfpathlineto{\pgfpoint{34.131057\du}{14.753365\du}}
\pgfpathlineto{\pgfpoint{34.116453\du}{14.758112\du}}
\pgfpathlineto{\pgfpoint{34.102579\du}{14.762858\du}}
\pgfpathlineto{\pgfpoint{34.087975\du}{14.767604\du}}
\pgfpathlineto{\pgfpoint{34.073737\du}{14.771985\du}}
\pgfpathlineto{\pgfpoint{34.058037\du}{14.776732\du}}
\pgfpathlineto{\pgfpoint{34.043799\du}{14.781478\du}}
\pgfpathlineto{\pgfpoint{34.028100\du}{14.786224\du}}
\pgfpathlineto{\pgfpoint{34.013131\du}{14.790240\du}}
\pgfpathlineto{\pgfpoint{33.981732\du}{14.799003\du}}
\pgfpathlineto{\pgfpoint{33.949604\du}{14.807400\du}}
\pgfpathlineto{\pgfpoint{33.917475\du}{14.815432\du}}
\pgfpathlineto{\pgfpoint{33.884251\du}{14.823464\du}}
\pgfpathlineto{\pgfpoint{33.849932\du}{14.831131\du}}
\pgfpathlineto{\pgfpoint{33.815978\du}{14.838433\du}}
\pgfpathlineto{\pgfpoint{33.781294\du}{14.845370\du}}
\pgfpathlineto{\pgfpoint{33.745515\du}{14.852307\du}}
\pgfpathlineto{\pgfpoint{33.710100\du}{14.858878\du}}
\pgfpathlineto{\pgfpoint{33.673225\du}{14.865085\du}}
\pgfpathlineto{\pgfpoint{33.636716\du}{14.870927\du}}
\pgfpathlineto{\pgfpoint{33.599476\du}{14.876768\du}}
\pgfpathlineto{\pgfpoint{33.561871\du}{14.882245\du}}
\pgfpathlineto{\pgfpoint{33.523171\du}{14.886991\du}}
\pgfpathlineto{\pgfpoint{33.484470\du}{14.891372\du}}
\pgfpathlineto{\pgfpoint{33.445040\du}{14.896118\du}}
\pgfpathlineto{\pgfpoint{33.405975\du}{14.899769\du}}
\pgfpathlineto{\pgfpoint{33.365814\du}{14.903785\du}}
\pgfpathlineto{\pgfpoint{33.325653\du}{14.906706\du}}
\pgfpathlineto{\pgfpoint{33.284397\du}{14.910357\du}}
\pgfpathlineto{\pgfpoint{33.243507\du}{14.913278\du}}
\pgfpathlineto{\pgfpoint{33.202251\du}{14.915468\du}}
\pgfpathlineto{\pgfpoint{33.160630\du}{14.917294\du}}
\pgfpathlineto{\pgfpoint{33.117913\du}{14.919119\du}}
\pgfpathlineto{\pgfpoint{33.075197\du}{14.920215\du}}
\pgfpathlineto{\pgfpoint{33.033211\du}{14.921310\du}}
\pgfpathlineto{\pgfpoint{32.989764\du}{14.921310\du}}
\pgfpathlineto{\pgfpoint{32.946683\du}{14.922040\du}}
\pgfpathlineto{\pgfpoint{32.946683\du}{14.922040\du}}
\pgfpathlineto{\pgfpoint{32.946683\du}{14.922040\du}}
\pgfpathlineto{\pgfpoint{32.945953\du}{14.922040\du}}
\pgfpathlineto{\pgfpoint{32.944127\du}{14.922040\du}}
\pgfpathlineto{\pgfpoint{32.943032\du}{14.922405\du}}
\pgfpathlineto{\pgfpoint{32.942302\du}{14.922405\du}}
\pgfpathlineto{\pgfpoint{32.941937\du}{14.923135\du}}
\pgfpathlineto{\pgfpoint{32.940476\du}{14.923501\du}}
\pgfpathlineto{\pgfpoint{32.939746\du}{14.924231\du}}
\pgfpathlineto{\pgfpoint{32.939016\du}{14.924961\du}}
\pgfpathlineto{\pgfpoint{32.937921\du}{14.926786\du}}
\pgfpathlineto{\pgfpoint{32.937190\du}{14.928247\du}}
\pgfpathlineto{\pgfpoint{32.937190\du}{14.930072\du}}
\pgfpathlineto{\pgfpoint{32.936460\du}{14.931898\du}}
\pgfpathlineto{\pgfpoint{32.937190\du}{14.934088\du}}
\pgfpathlineto{\pgfpoint{32.937190\du}{14.935914\du}}
\pgfpathlineto{\pgfpoint{32.937921\du}{14.937739\du}}
\pgfpathlineto{\pgfpoint{32.939016\du}{14.939565\du}}
\pgfpathlineto{\pgfpoint{32.939746\du}{14.939930\du}}
\pgfpathlineto{\pgfpoint{32.940476\du}{14.940660\du}}
\pgfpathlineto{\pgfpoint{32.941937\du}{14.941390\du}}
\pgfpathlineto{\pgfpoint{32.942302\du}{14.941755\du}}
\pgfpathlineto{\pgfpoint{32.943032\du}{14.941755\du}}
\pgfpathlineto{\pgfpoint{32.944127\du}{14.942486\du}}
\pgfpathlineto{\pgfpoint{32.945953\du}{14.942486\du}}
\pgfpathlineto{\pgfpoint{32.946683\du}{14.942486\du}}
\pgfusepath{fill}
\pgfsetbuttcap
\pgfsetmiterjoin
\pgfsetdash{}{0pt}
\definecolor{dialinecolor}{rgb}{0.678431, 0.839216, 0.905882}
\pgfsetfillcolor{dialinecolor}
\pgfpathmoveto{\pgfpoint{31.255920\du}{14.355045\du}}
\pgfpathlineto{\pgfpoint{31.255920\du}{14.355045\du}}
\pgfpathlineto{\pgfpoint{31.255920\du}{14.362712\du}}
\pgfpathlineto{\pgfpoint{31.256285\du}{14.370014\du}}
\pgfpathlineto{\pgfpoint{31.257015\du}{14.378411\du}}
\pgfpathlineto{\pgfpoint{31.258110\du}{14.386078\du}}
\pgfpathlineto{\pgfpoint{31.259206\du}{14.393380\du}}
\pgfpathlineto{\pgfpoint{31.261031\du}{14.401047\du}}
\pgfpathlineto{\pgfpoint{31.262857\du}{14.409444\du}}
\pgfpathlineto{\pgfpoint{31.265047\du}{14.417111\du}}
\pgfpathlineto{\pgfpoint{31.267238\du}{14.424413\du}}
\pgfpathlineto{\pgfpoint{31.269794\du}{14.431715\du}}
\pgfpathlineto{\pgfpoint{31.272714\du}{14.439017\du}}
\pgfpathlineto{\pgfpoint{31.276365\du}{14.446684\du}}
\pgfpathlineto{\pgfpoint{31.279651\du}{14.454351\du}}
\pgfpathlineto{\pgfpoint{31.283302\du}{14.462018\du}}
\pgfpathlineto{\pgfpoint{31.287683\du}{14.468955\du}}
\pgfpathlineto{\pgfpoint{31.291334\du}{14.476622\du}}
\pgfpathlineto{\pgfpoint{31.296446\du}{14.483559\du}}
\pgfpathlineto{\pgfpoint{31.300462\du}{14.490861\du}}
\pgfpathlineto{\pgfpoint{31.305938\du}{14.498163\du}}
\pgfpathlineto{\pgfpoint{31.310684\du}{14.505465\du}}
\pgfpathlineto{\pgfpoint{31.316161\du}{14.512402\du}}
\pgfpathlineto{\pgfpoint{31.322002\du}{14.519338\du}}
\pgfpathlineto{\pgfpoint{31.327844\du}{14.526275\du}}
\pgfpathlineto{\pgfpoint{31.333685\du}{14.533577\du}}
\pgfpathlineto{\pgfpoint{31.340622\du}{14.540514\du}}
\pgfpathlineto{\pgfpoint{31.347194\du}{14.546721\du}}
\pgfpathlineto{\pgfpoint{31.354131\du}{14.554023\du}}
\pgfpathlineto{\pgfpoint{31.361068\du}{14.560959\du}}
\pgfpathlineto{\pgfpoint{31.368005\du}{14.567166\du}}
\pgfpathlineto{\pgfpoint{31.376767\du}{14.574468\du}}
\pgfpathlineto{\pgfpoint{31.383704\du}{14.580675\du}}
\pgfpathlineto{\pgfpoint{31.391371\du}{14.587246\du}}
\pgfpathlineto{\pgfpoint{31.400133\du}{14.594183\du}}
\pgfpathlineto{\pgfpoint{31.408530\du}{14.600755\du}}
\pgfpathlineto{\pgfpoint{31.417293\du}{14.606962\du}}
\pgfpathlineto{\pgfpoint{31.425690\du}{14.613533\du}}
\pgfpathlineto{\pgfpoint{31.435547\du}{14.619375\du}}
\pgfpathlineto{\pgfpoint{31.444675\du}{14.625947\du}}
\pgfpathlineto{\pgfpoint{31.454167\du}{14.632153\du}}
\pgfpathlineto{\pgfpoint{31.463660\du}{14.638725\du}}
\pgfpathlineto{\pgfpoint{31.473518\du}{14.644567\du}}
\pgfpathlineto{\pgfpoint{31.483740\du}{14.651138\du}}
\pgfpathlineto{\pgfpoint{31.493963\du}{14.656980\du}}
\pgfpathlineto{\pgfpoint{31.504551\du}{14.662821\du}}
\pgfpathlineto{\pgfpoint{31.514773\du}{14.669028\du}}
\pgfpathlineto{\pgfpoint{31.525726\du}{14.674870\du}}
\pgfpathlineto{\pgfpoint{31.537409\du}{14.680711\du}}
\pgfpathlineto{\pgfpoint{31.548727\du}{14.686553\du}}
\pgfpathlineto{\pgfpoint{31.560411\du}{14.692394\du}}
\pgfpathlineto{\pgfpoint{31.571729\du}{14.698236\du}}
\pgfpathlineto{\pgfpoint{31.583412\du}{14.703712\du}}
\pgfpathlineto{\pgfpoint{31.595825\du}{14.709554\du}}
\pgfpathlineto{\pgfpoint{31.607873\du}{14.714665\du}}
\pgfpathlineto{\pgfpoint{31.621017\du}{14.720507\du}}
\pgfpathlineto{\pgfpoint{31.633065\du}{14.725983\du}}
\pgfpathlineto{\pgfpoint{31.645843\du}{14.731095\du}}
\pgfpathlineto{\pgfpoint{31.659717\du}{14.736936\du}}
\pgfpathlineto{\pgfpoint{31.672130\du}{14.742413\du}}
\pgfpathlineto{\pgfpoint{31.685274\du}{14.747524\du}}
\pgfpathlineto{\pgfpoint{31.699512\du}{14.753000\du}}
\pgfpathlineto{\pgfpoint{31.712656\du}{14.757382\du}}
\pgfpathlineto{\pgfpoint{31.726895\du}{14.762858\du}}
\pgfpathlineto{\pgfpoint{31.740768\du}{14.767604\du}}
\pgfpathlineto{\pgfpoint{31.755372\du}{14.772716\du}}
\pgfpathlineto{\pgfpoint{31.769246\du}{14.777462\du}}
\pgfpathlineto{\pgfpoint{31.784945\du}{14.782208\du}}
\pgfpathlineto{\pgfpoint{31.798819\du}{14.787319\du}}
\pgfpathlineto{\pgfpoint{31.813423\du}{14.792066\du}}
\pgfpathlineto{\pgfpoint{31.828757\du}{14.796812\du}}
\pgfpathlineto{\pgfpoint{31.844456\du}{14.801558\du}}
\pgfpathlineto{\pgfpoint{31.859425\du}{14.805574\du}}
\pgfpathlineto{\pgfpoint{31.875124\du}{14.810321\du}}
\pgfpathlineto{\pgfpoint{31.906887\du}{14.819083\du}}
\pgfpathlineto{\pgfpoint{31.938651\du}{14.827115\du}}
\pgfpathlineto{\pgfpoint{31.971875\du}{14.835147\du}}
\pgfpathlineto{\pgfpoint{32.004368\du}{14.843544\du}}
\pgfpathlineto{\pgfpoint{32.039052\du}{14.851211\du}}
\pgfpathlineto{\pgfpoint{32.073372\du}{14.858878\du}}
\pgfpathlineto{\pgfpoint{32.108056\du}{14.865815\du}}
\pgfpathlineto{\pgfpoint{32.143835\du}{14.872752\du}}
\pgfpathlineto{\pgfpoint{32.179980\du}{14.879324\du}}
\pgfpathlineto{\pgfpoint{32.216489\du}{14.885165\du}}
\pgfpathlineto{\pgfpoint{32.253729\du}{14.891372\du}}
\pgfpathlineto{\pgfpoint{32.291334\du}{14.896848\du}}
\pgfpathlineto{\pgfpoint{32.329304\du}{14.901960\du}}
\pgfpathlineto{\pgfpoint{32.368005\du}{14.907436\du}}
\pgfpathlineto{\pgfpoint{32.406705\du}{14.911817\du}}
\pgfpathlineto{\pgfpoint{32.445770\du}{14.916199\du}}
\pgfpathlineto{\pgfpoint{32.485931\du}{14.920215\du}}
\pgfpathlineto{\pgfpoint{32.525726\du}{14.924231\du}}
\pgfpathlineto{\pgfpoint{32.566617\du}{14.927882\du}}
\pgfpathlineto{\pgfpoint{32.607508\du}{14.930802\du}}
\pgfpathlineto{\pgfpoint{32.648764\du}{14.932993\du}}
\pgfpathlineto{\pgfpoint{32.690750\du}{14.935549\du}}
\pgfpathlineto{\pgfpoint{32.732371\du}{14.937739\du}}
\pgfpathlineto{\pgfpoint{32.774722\du}{14.939565\du}}
\pgfpathlineto{\pgfpoint{32.816708\du}{14.940660\du}}
\pgfpathlineto{\pgfpoint{32.860155\du}{14.941755\du}}
\pgfpathlineto{\pgfpoint{32.902871\du}{14.942486\du}}
\pgfpathlineto{\pgfpoint{32.946683\du}{14.942486\du}}
\pgfpathlineto{\pgfpoint{32.946683\du}{14.922040\du}}
\pgfpathlineto{\pgfpoint{32.903967\du}{14.921310\du}}
\pgfpathlineto{\pgfpoint{32.860520\du}{14.921310\du}}
\pgfpathlineto{\pgfpoint{32.818169\du}{14.920215\du}}
\pgfpathlineto{\pgfpoint{32.775453\du}{14.919119\du}}
\pgfpathlineto{\pgfpoint{32.733101\du}{14.917294\du}}
\pgfpathlineto{\pgfpoint{32.691115\du}{14.915468\du}}
\pgfpathlineto{\pgfpoint{32.650224\du}{14.913278\du}}
\pgfpathlineto{\pgfpoint{32.608968\du}{14.910357\du}}
\pgfpathlineto{\pgfpoint{32.568078\du}{14.906706\du}}
\pgfpathlineto{\pgfpoint{32.527917\du}{14.903785\du}}
\pgfpathlineto{\pgfpoint{32.488121\du}{14.899769\du}}
\pgfpathlineto{\pgfpoint{32.448691\du}{14.896118\du}}
\pgfpathlineto{\pgfpoint{32.409261\du}{14.891372\du}}
\pgfpathlineto{\pgfpoint{32.370195\du}{14.886991\du}}
\pgfpathlineto{\pgfpoint{32.331860\du}{14.882245\du}}
\pgfpathlineto{\pgfpoint{32.294255\du}{14.876768\du}}
\pgfpathlineto{\pgfpoint{32.257015\du}{14.870927\du}}
\pgfpathlineto{\pgfpoint{32.220505\du}{14.865085\du}}
\pgfpathlineto{\pgfpoint{32.183266\du}{14.858878\du}}
\pgfpathlineto{\pgfpoint{32.148216\du}{14.852307\du}}
\pgfpathlineto{\pgfpoint{32.112072\du}{14.845370\du}}
\pgfpathlineto{\pgfpoint{32.077388\du}{14.838433\du}}
\pgfpathlineto{\pgfpoint{32.043068\du}{14.831131\du}}
\pgfpathlineto{\pgfpoint{32.009114\du}{14.823464\du}}
\pgfpathlineto{\pgfpoint{31.975891\du}{14.815432\du}}
\pgfpathlineto{\pgfpoint{31.944492\du}{14.807400\du}}
\pgfpathlineto{\pgfpoint{31.911999\du}{14.799003\du}}
\pgfpathlineto{\pgfpoint{31.880966\du}{14.790240\du}}
\pgfpathlineto{\pgfpoint{31.865266\du}{14.786224\du}}
\pgfpathlineto{\pgfpoint{31.849567\du}{14.781478\du}}
\pgfpathlineto{\pgfpoint{31.834963\du}{14.776732\du}}
\pgfpathlineto{\pgfpoint{31.819994\du}{14.771985\du}}
\pgfpathlineto{\pgfpoint{31.804660\du}{14.767604\du}}
\pgfpathlineto{\pgfpoint{31.790422\du}{14.762858\du}}
\pgfpathlineto{\pgfpoint{31.776183\du}{14.758112\du}}
\pgfpathlineto{\pgfpoint{31.761944\du}{14.753365\du}}
\pgfpathlineto{\pgfpoint{31.747705\du}{14.748254\du}}
\pgfpathlineto{\pgfpoint{31.733832\du}{14.743508\du}}
\pgfpathlineto{\pgfpoint{31.719958\du}{14.738031\du}}
\pgfpathlineto{\pgfpoint{31.706449\du}{14.733650\du}}
\pgfpathlineto{\pgfpoint{31.692941\du}{14.728174\du}}
\pgfpathlineto{\pgfpoint{31.680162\du}{14.723062\du}}
\pgfpathlineto{\pgfpoint{31.666654\du}{14.717586\du}}
\pgfpathlineto{\pgfpoint{31.653875\du}{14.712475\du}}
\pgfpathlineto{\pgfpoint{31.641462\du}{14.707363\du}}
\pgfpathlineto{\pgfpoint{31.628319\du}{14.701887\du}}
\pgfpathlineto{\pgfpoint{31.616270\du}{14.696045\du}}
\pgfpathlineto{\pgfpoint{31.604952\du}{14.690934\du}}
\pgfpathlineto{\pgfpoint{31.592174\du}{14.685457\du}}
\pgfpathlineto{\pgfpoint{31.580491\du}{14.679616\du}}
\pgfpathlineto{\pgfpoint{31.568808\du}{14.673774\du}}
\pgfpathlineto{\pgfpoint{31.557855\du}{14.668663\du}}
\pgfpathlineto{\pgfpoint{31.546172\du}{14.662821\du}}
\pgfpathlineto{\pgfpoint{31.536314\du}{14.656980\du}}
\pgfpathlineto{\pgfpoint{31.524996\du}{14.651138\du}}
\pgfpathlineto{\pgfpoint{31.514408\du}{14.645297\du}}
\pgfpathlineto{\pgfpoint{31.504551\du}{14.639455\du}}
\pgfpathlineto{\pgfpoint{31.493963\du}{14.633614\du}}
\pgfpathlineto{\pgfpoint{31.484105\du}{14.627772\du}}
\pgfpathlineto{\pgfpoint{31.474613\du}{14.621200\du}}
\pgfpathlineto{\pgfpoint{31.464755\du}{14.615359\du}}
\pgfpathlineto{\pgfpoint{31.455628\du}{14.608787\du}}
\pgfpathlineto{\pgfpoint{31.447231\du}{14.602946\du}}
\pgfpathlineto{\pgfpoint{31.438468\du}{14.596739\du}}
\pgfpathlineto{\pgfpoint{31.429341\du}{14.590167\du}}
\pgfpathlineto{\pgfpoint{31.420579\du}{14.584326\du}}
\pgfpathlineto{\pgfpoint{31.412181\du}{14.577754\du}}
\pgfpathlineto{\pgfpoint{31.404879\du}{14.571547\du}}
\pgfpathlineto{\pgfpoint{31.396847\du}{14.564976\du}}
\pgfpathlineto{\pgfpoint{31.389180\du}{14.558404\du}}
\pgfpathlineto{\pgfpoint{31.382243\du}{14.552197\du}}
\pgfpathlineto{\pgfpoint{31.374941\du}{14.545625\du}}
\pgfpathlineto{\pgfpoint{31.368005\du}{14.539419\du}}
\pgfpathlineto{\pgfpoint{31.361798\du}{14.532847\du}}
\pgfpathlineto{\pgfpoint{31.355226\du}{14.525910\du}}
\pgfpathlineto{\pgfpoint{31.349385\du}{14.519338\du}}
\pgfpathlineto{\pgfpoint{31.343178\du}{14.512767\du}}
\pgfpathlineto{\pgfpoint{31.338067\du}{14.506560\du}}
\pgfpathlineto{\pgfpoint{31.331860\du}{14.499623\du}}
\pgfpathlineto{\pgfpoint{31.327479\du}{14.493051\du}}
\pgfpathlineto{\pgfpoint{31.322002\du}{14.486115\du}}
\pgfpathlineto{\pgfpoint{31.317621\du}{14.479543\du}}
\pgfpathlineto{\pgfpoint{31.313240\du}{14.472606\du}}
\pgfpathlineto{\pgfpoint{31.308859\du}{14.465669\du}}
\pgfpathlineto{\pgfpoint{31.305573\du}{14.459097\du}}
\pgfpathlineto{\pgfpoint{31.301557\du}{14.452161\du}}
\pgfpathlineto{\pgfpoint{31.297541\du}{14.445589\du}}
\pgfpathlineto{\pgfpoint{31.294620\du}{14.438652\du}}
\pgfpathlineto{\pgfpoint{31.292064\du}{14.431715\du}}
\pgfpathlineto{\pgfpoint{31.288779\du}{14.424413\du}}
\pgfpathlineto{\pgfpoint{31.286588\du}{14.417476\du}}
\pgfpathlineto{\pgfpoint{31.284397\du}{14.411270\du}}
\pgfpathlineto{\pgfpoint{31.282937\du}{14.403968\du}}
\pgfpathlineto{\pgfpoint{31.280746\du}{14.397031\du}}
\pgfpathlineto{\pgfpoint{31.279651\du}{14.390094\du}}
\pgfpathlineto{\pgfpoint{31.278556\du}{14.383157\du}}
\pgfpathlineto{\pgfpoint{31.277095\du}{14.375855\du}}
\pgfpathlineto{\pgfpoint{31.276730\du}{14.368919\du}}
\pgfpathlineto{\pgfpoint{31.276730\du}{14.361982\du}}
\pgfpathlineto{\pgfpoint{31.276365\du}{14.355045\du}}
\pgfpathlineto{\pgfpoint{31.276365\du}{14.355045\du}}
\pgfpathlineto{\pgfpoint{31.276365\du}{14.355045\du}}
\pgfpathlineto{\pgfpoint{31.276365\du}{14.353219\du}}
\pgfpathlineto{\pgfpoint{31.276365\du}{14.352489\du}}
\pgfpathlineto{\pgfpoint{31.276000\du}{14.351394\du}}
\pgfpathlineto{\pgfpoint{31.276000\du}{14.350299\du}}
\pgfpathlineto{\pgfpoint{31.274905\du}{14.349568\du}}
\pgfpathlineto{\pgfpoint{31.274540\du}{14.348473\du}}
\pgfpathlineto{\pgfpoint{31.274175\du}{14.347743\du}}
\pgfpathlineto{\pgfpoint{31.273079\du}{14.347378\du}}
\pgfpathlineto{\pgfpoint{31.271619\du}{14.346283\du}}
\pgfpathlineto{\pgfpoint{31.269794\du}{14.344822\du}}
\pgfpathlineto{\pgfpoint{31.267968\du}{14.344457\du}}
\pgfpathlineto{\pgfpoint{31.266143\du}{14.344457\du}}
\pgfpathlineto{\pgfpoint{31.263952\du}{14.344457\du}}
\pgfpathlineto{\pgfpoint{31.262492\du}{14.344822\du}}
\pgfpathlineto{\pgfpoint{31.260301\du}{14.346283\du}}
\pgfpathlineto{\pgfpoint{31.258476\du}{14.347378\du}}
\pgfpathlineto{\pgfpoint{31.258110\du}{14.347743\du}}
\pgfpathlineto{\pgfpoint{31.257745\du}{14.348473\du}}
\pgfpathlineto{\pgfpoint{31.257015\du}{14.349568\du}}
\pgfpathlineto{\pgfpoint{31.256285\du}{14.350299\du}}
\pgfpathlineto{\pgfpoint{31.256285\du}{14.351394\du}}
\pgfpathlineto{\pgfpoint{31.255920\du}{14.352489\du}}
\pgfpathlineto{\pgfpoint{31.255920\du}{14.353219\du}}
\pgfpathlineto{\pgfpoint{31.255920\du}{14.355045\du}}
\pgfusepath{fill}
\pgfsetbuttcap
\pgfsetmiterjoin
\pgfsetdash{}{0pt}
\definecolor{dialinecolor}{rgb}{0.678431, 0.839216, 0.905882}
\pgfsetfillcolor{dialinecolor}
\pgfpathmoveto{\pgfpoint{32.946683\du}{13.767604\du}}
\pgfpathlineto{\pgfpoint{32.946683\du}{13.767604\du}}
\pgfpathlineto{\pgfpoint{32.902871\du}{13.767604\du}}
\pgfpathlineto{\pgfpoint{32.860155\du}{13.767969\du}}
\pgfpathlineto{\pgfpoint{32.816708\du}{13.769065\du}}
\pgfpathlineto{\pgfpoint{32.774722\du}{13.770525\du}}
\pgfpathlineto{\pgfpoint{32.732371\du}{13.771985\du}}
\pgfpathlineto{\pgfpoint{32.690750\du}{13.773811\du}}
\pgfpathlineto{\pgfpoint{32.648764\du}{13.776367\du}}
\pgfpathlineto{\pgfpoint{32.607508\du}{13.779287\du}}
\pgfpathlineto{\pgfpoint{32.566617\du}{13.782208\du}}
\pgfpathlineto{\pgfpoint{32.525726\du}{13.785129\du}}
\pgfpathlineto{\pgfpoint{32.485931\du}{13.789145\du}}
\pgfpathlineto{\pgfpoint{32.445770\du}{13.793161\du}}
\pgfpathlineto{\pgfpoint{32.406705\du}{13.797907\du}}
\pgfpathlineto{\pgfpoint{32.368005\du}{13.802654\du}}
\pgfpathlineto{\pgfpoint{32.329304\du}{13.807400\du}}
\pgfpathlineto{\pgfpoint{32.291334\du}{13.812511\du}}
\pgfpathlineto{\pgfpoint{32.253729\du}{13.818353\du}}
\pgfpathlineto{\pgfpoint{32.216489\du}{13.824194\du}}
\pgfpathlineto{\pgfpoint{32.179980\du}{13.830766\du}}
\pgfpathlineto{\pgfpoint{32.143835\du}{13.836973\du}}
\pgfpathlineto{\pgfpoint{32.108056\du}{13.844275\du}}
\pgfpathlineto{\pgfpoint{32.073372\du}{13.851211\du}}
\pgfpathlineto{\pgfpoint{32.039052\du}{13.858148\du}}
\pgfpathlineto{\pgfpoint{32.004368\du}{13.866180\du}}
\pgfpathlineto{\pgfpoint{31.971875\du}{13.873847\du}}
\pgfpathlineto{\pgfpoint{31.938651\du}{13.882245\du}}
\pgfpathlineto{\pgfpoint{31.906887\du}{13.891007\du}}
\pgfpathlineto{\pgfpoint{31.875124\du}{13.899769\du}}
\pgfpathlineto{\pgfpoint{31.844456\du}{13.908532\du}}
\pgfpathlineto{\pgfpoint{31.813423\du}{13.918024\du}}
\pgfpathlineto{\pgfpoint{31.798819\du}{13.922405\du}}
\pgfpathlineto{\pgfpoint{31.784945\du}{13.927152\du}}
\pgfpathlineto{\pgfpoint{31.769246\du}{13.931898\du}}
\pgfpathlineto{\pgfpoint{31.755372\du}{13.937009\du}}
\pgfpathlineto{\pgfpoint{31.740768\du}{13.941755\du}}
\pgfpathlineto{\pgfpoint{31.726895\du}{13.947232\du}}
\pgfpathlineto{\pgfpoint{31.712656\du}{13.951613\du}}
\pgfpathlineto{\pgfpoint{31.699512\du}{13.957089\du}}
\pgfpathlineto{\pgfpoint{31.685274\du}{13.962201\du}}
\pgfpathlineto{\pgfpoint{31.672130\du}{13.967677\du}}
\pgfpathlineto{\pgfpoint{31.659717\du}{13.972789\du}}
\pgfpathlineto{\pgfpoint{31.645843\du}{13.978265\du}}
\pgfpathlineto{\pgfpoint{31.633065\du}{13.983376\du}}
\pgfpathlineto{\pgfpoint{31.621017\du}{13.988488\du}}
\pgfpathlineto{\pgfpoint{31.607873\du}{13.994329\du}}
\pgfpathlineto{\pgfpoint{31.595825\du}{14.000536\du}}
\pgfpathlineto{\pgfpoint{31.583412\du}{14.005647\du}}
\pgfpathlineto{\pgfpoint{31.571729\du}{14.011489\du}}
\pgfpathlineto{\pgfpoint{31.560411\du}{14.017330\du}}
\pgfpathlineto{\pgfpoint{31.548727\du}{14.023172\du}}
\pgfpathlineto{\pgfpoint{31.537409\du}{14.029014\du}}
\pgfpathlineto{\pgfpoint{31.525726\du}{14.034125\du}}
\pgfpathlineto{\pgfpoint{31.514773\du}{14.040697\du}}
\pgfpathlineto{\pgfpoint{31.504551\du}{14.046538\du}}
\pgfpathlineto{\pgfpoint{31.493963\du}{14.052380\du}}
\pgfpathlineto{\pgfpoint{31.483740\du}{14.058221\du}}
\pgfpathlineto{\pgfpoint{31.473518\du}{14.064793\du}}
\pgfpathlineto{\pgfpoint{31.463660\du}{14.071000\du}}
\pgfpathlineto{\pgfpoint{31.454167\du}{14.076841\du}}
\pgfpathlineto{\pgfpoint{31.444675\du}{14.083413\du}}
\pgfpathlineto{\pgfpoint{31.435547\du}{14.089985\du}}
\pgfpathlineto{\pgfpoint{31.425690\du}{14.096191\du}}
\pgfpathlineto{\pgfpoint{31.417293\du}{14.102763\du}}
\pgfpathlineto{\pgfpoint{31.408530\du}{14.109335\du}}
\pgfpathlineto{\pgfpoint{31.400133\du}{14.115541\du}}
\pgfpathlineto{\pgfpoint{31.391371\du}{14.122113\du}}
\pgfpathlineto{\pgfpoint{31.383704\du}{14.129050\du}}
\pgfpathlineto{\pgfpoint{31.376767\du}{14.135622\du}}
\pgfpathlineto{\pgfpoint{31.368005\du}{14.141828\du}}
\pgfpathlineto{\pgfpoint{31.361068\du}{14.149130\du}}
\pgfpathlineto{\pgfpoint{31.354131\du}{14.155337\du}}
\pgfpathlineto{\pgfpoint{31.347194\du}{14.162274\du}}
\pgfpathlineto{\pgfpoint{31.340622\du}{14.169576\du}}
\pgfpathlineto{\pgfpoint{31.333685\du}{14.175782\du}}
\pgfpathlineto{\pgfpoint{31.327844\du}{14.183084\du}}
\pgfpathlineto{\pgfpoint{31.322002\du}{14.190021\du}}
\pgfpathlineto{\pgfpoint{31.316161\du}{14.197688\du}}
\pgfpathlineto{\pgfpoint{31.310684\du}{14.204625\du}}
\pgfpathlineto{\pgfpoint{31.305938\du}{14.211562\du}}
\pgfpathlineto{\pgfpoint{31.300462\du}{14.218499\du}}
\pgfpathlineto{\pgfpoint{31.296446\du}{14.225801\du}}
\pgfpathlineto{\pgfpoint{31.291334\du}{14.233103\du}}
\pgfpathlineto{\pgfpoint{31.287683\du}{14.240405\du}}
\pgfpathlineto{\pgfpoint{31.283302\du}{14.247706\du}}
\pgfpathlineto{\pgfpoint{31.279651\du}{14.255373\du}}
\pgfpathlineto{\pgfpoint{31.276365\du}{14.263041\du}}
\pgfpathlineto{\pgfpoint{31.272714\du}{14.269977\du}}
\pgfpathlineto{\pgfpoint{31.269794\du}{14.277644\du}}
\pgfpathlineto{\pgfpoint{31.267238\du}{14.285311\du}}
\pgfpathlineto{\pgfpoint{31.265047\du}{14.292978\du}}
\pgfpathlineto{\pgfpoint{31.262857\du}{14.300645\du}}
\pgfpathlineto{\pgfpoint{31.261031\du}{14.307947\du}}
\pgfpathlineto{\pgfpoint{31.259206\du}{14.315614\du}}
\pgfpathlineto{\pgfpoint{31.258110\du}{14.323281\du}}
\pgfpathlineto{\pgfpoint{31.257015\du}{14.331679\du}}
\pgfpathlineto{\pgfpoint{31.256285\du}{14.338981\du}}
\pgfpathlineto{\pgfpoint{31.255920\du}{14.346648\du}}
\pgfpathlineto{\pgfpoint{31.255920\du}{14.355045\du}}
\pgfpathlineto{\pgfpoint{31.276365\du}{14.355045\du}}
\pgfpathlineto{\pgfpoint{31.276730\du}{14.347743\du}}
\pgfpathlineto{\pgfpoint{31.276730\du}{14.340806\du}}
\pgfpathlineto{\pgfpoint{31.277095\du}{14.333869\du}}
\pgfpathlineto{\pgfpoint{31.278556\du}{14.326202\du}}
\pgfpathlineto{\pgfpoint{31.279651\du}{14.319996\du}}
\pgfpathlineto{\pgfpoint{31.280746\du}{14.312694\du}}
\pgfpathlineto{\pgfpoint{31.282937\du}{14.305757\du}}
\pgfpathlineto{\pgfpoint{31.284397\du}{14.298820\du}}
\pgfpathlineto{\pgfpoint{31.286588\du}{14.291883\du}}
\pgfpathlineto{\pgfpoint{31.288779\du}{14.284581\du}}
\pgfpathlineto{\pgfpoint{31.292064\du}{14.278375\du}}
\pgfpathlineto{\pgfpoint{31.294620\du}{14.271438\du}}
\pgfpathlineto{\pgfpoint{31.297541\du}{14.264136\du}}
\pgfpathlineto{\pgfpoint{31.301557\du}{14.257199\du}}
\pgfpathlineto{\pgfpoint{31.305573\du}{14.250262\du}}
\pgfpathlineto{\pgfpoint{31.308859\du}{14.243690\du}}
\pgfpathlineto{\pgfpoint{31.312875\du}{14.236754\du}}
\pgfpathlineto{\pgfpoint{31.317621\du}{14.230182\du}}
\pgfpathlineto{\pgfpoint{31.322002\du}{14.223245\du}}
\pgfpathlineto{\pgfpoint{31.327479\du}{14.216673\du}}
\pgfpathlineto{\pgfpoint{31.331860\du}{14.210467\du}}
\pgfpathlineto{\pgfpoint{31.338067\du}{14.203530\du}}
\pgfpathlineto{\pgfpoint{31.343178\du}{14.196958\du}}
\pgfpathlineto{\pgfpoint{31.349385\du}{14.190021\du}}
\pgfpathlineto{\pgfpoint{31.355226\du}{14.183449\du}}
\pgfpathlineto{\pgfpoint{31.361798\du}{14.177243\du}}
\pgfpathlineto{\pgfpoint{31.368005\du}{14.170671\du}}
\pgfpathlineto{\pgfpoint{31.374941\du}{14.163734\du}}
\pgfpathlineto{\pgfpoint{31.382243\du}{14.157893\du}}
\pgfpathlineto{\pgfpoint{31.389180\du}{14.150591\du}}
\pgfpathlineto{\pgfpoint{31.396847\du}{14.144384\du}}
\pgfpathlineto{\pgfpoint{31.404879\du}{14.138543\du}}
\pgfpathlineto{\pgfpoint{31.412181\du}{14.131971\du}}
\pgfpathlineto{\pgfpoint{31.420579\du}{14.125399\du}}
\pgfpathlineto{\pgfpoint{31.429341\du}{14.119192\du}}
\pgfpathlineto{\pgfpoint{31.438468\du}{14.112621\du}}
\pgfpathlineto{\pgfpoint{31.447231\du}{14.106779\du}}
\pgfpathlineto{\pgfpoint{31.455628\du}{14.100572\du}}
\pgfpathlineto{\pgfpoint{31.464755\du}{14.094731\du}}
\pgfpathlineto{\pgfpoint{31.474613\du}{14.088159\du}}
\pgfpathlineto{\pgfpoint{31.484105\du}{14.082318\du}}
\pgfpathlineto{\pgfpoint{31.493963\du}{14.076476\du}}
\pgfpathlineto{\pgfpoint{31.504551\du}{14.070635\du}}
\pgfpathlineto{\pgfpoint{31.514408\du}{14.064793\du}}
\pgfpathlineto{\pgfpoint{31.524996\du}{14.058221\du}}
\pgfpathlineto{\pgfpoint{31.536314\du}{14.053110\du}}
\pgfpathlineto{\pgfpoint{31.546172\du}{14.047268\du}}
\pgfpathlineto{\pgfpoint{31.557855\du}{14.041427\du}}
\pgfpathlineto{\pgfpoint{31.568808\du}{14.035585\du}}
\pgfpathlineto{\pgfpoint{31.580491\du}{14.029744\du}}
\pgfpathlineto{\pgfpoint{31.592174\du}{14.024267\du}}
\pgfpathlineto{\pgfpoint{31.604952\du}{14.019156\du}}
\pgfpathlineto{\pgfpoint{31.616270\du}{14.013314\du}}
\pgfpathlineto{\pgfpoint{31.628319\du}{14.007838\du}}
\pgfpathlineto{\pgfpoint{31.641462\du}{14.002727\du}}
\pgfpathlineto{\pgfpoint{31.653875\du}{13.997250\du}}
\pgfpathlineto{\pgfpoint{31.666654\du}{13.992139\du}}
\pgfpathlineto{\pgfpoint{31.680162\du}{13.986297\du}}
\pgfpathlineto{\pgfpoint{31.692941\du}{13.981551\du}}
\pgfpathlineto{\pgfpoint{31.706449\du}{13.976440\du}}
\pgfpathlineto{\pgfpoint{31.719958\du}{13.970963\du}}
\pgfpathlineto{\pgfpoint{31.733832\du}{13.966582\du}}
\pgfpathlineto{\pgfpoint{31.747705\du}{13.961106\du}}
\pgfpathlineto{\pgfpoint{31.761944\du}{13.956359\du}}
\pgfpathlineto{\pgfpoint{31.776183\du}{13.951613\du}}
\pgfpathlineto{\pgfpoint{31.790422\du}{13.946502\du}}
\pgfpathlineto{\pgfpoint{31.804660\du}{13.941755\du}}
\pgfpathlineto{\pgfpoint{31.819994\du}{13.937009\du}}
\pgfpathlineto{\pgfpoint{31.849567\du}{13.928247\du}}
\pgfpathlineto{\pgfpoint{31.880966\du}{13.919119\du}}
\pgfpathlineto{\pgfpoint{31.911999\du}{13.910722\du}}
\pgfpathlineto{\pgfpoint{31.944492\du}{13.901960\du}}
\pgfpathlineto{\pgfpoint{31.975891\du}{13.893928\du}}
\pgfpathlineto{\pgfpoint{32.009114\du}{13.886261\du}}
\pgfpathlineto{\pgfpoint{32.043068\du}{13.878594\du}}
\pgfpathlineto{\pgfpoint{32.077388\du}{13.870927\du}}
\pgfpathlineto{\pgfpoint{32.112072\du}{13.863990\du}}
\pgfpathlineto{\pgfpoint{32.148216\du}{13.857053\du}}
\pgfpathlineto{\pgfpoint{32.183266\du}{13.851211\du}}
\pgfpathlineto{\pgfpoint{32.220505\du}{13.844640\du}}
\pgfpathlineto{\pgfpoint{32.257015\du}{13.838798\du}}
\pgfpathlineto{\pgfpoint{32.294255\du}{13.832957\du}}
\pgfpathlineto{\pgfpoint{32.331860\du}{13.827845\du}}
\pgfpathlineto{\pgfpoint{32.370195\du}{13.822369\du}}
\pgfpathlineto{\pgfpoint{32.409261\du}{13.817622\du}}
\pgfpathlineto{\pgfpoint{32.448691\du}{13.813606\du}}
\pgfpathlineto{\pgfpoint{32.488121\du}{13.809590\du}}
\pgfpathlineto{\pgfpoint{32.527917\du}{13.805939\du}}
\pgfpathlineto{\pgfpoint{32.568078\du}{13.802654\du}}
\pgfpathlineto{\pgfpoint{32.608968\du}{13.799733\du}}
\pgfpathlineto{\pgfpoint{32.650224\du}{13.796812\du}}
\pgfpathlineto{\pgfpoint{32.691115\du}{13.794256\du}}
\pgfpathlineto{\pgfpoint{32.733101\du}{13.792066\du}}
\pgfpathlineto{\pgfpoint{32.775453\du}{13.790970\du}}
\pgfpathlineto{\pgfpoint{32.818169\du}{13.789145\du}}
\pgfpathlineto{\pgfpoint{32.860520\du}{13.788415\du}}
\pgfpathlineto{\pgfpoint{32.903967\du}{13.788050\du}}
\pgfpathlineto{\pgfpoint{32.946683\du}{13.788050\du}}
\pgfpathlineto{\pgfpoint{32.946683\du}{13.788050\du}}
\pgfpathlineto{\pgfpoint{32.946683\du}{13.788050\du}}
\pgfpathlineto{\pgfpoint{32.947778\du}{13.787319\du}}
\pgfpathlineto{\pgfpoint{32.948874\du}{13.787319\du}}
\pgfpathlineto{\pgfpoint{32.950334\du}{13.787319\du}}
\pgfpathlineto{\pgfpoint{32.951429\du}{13.786954\du}}
\pgfpathlineto{\pgfpoint{32.951794\du}{13.786224\du}}
\pgfpathlineto{\pgfpoint{32.952890\du}{13.786224\du}}
\pgfpathlineto{\pgfpoint{32.953620\du}{13.785129\du}}
\pgfpathlineto{\pgfpoint{32.954715\du}{13.784399\du}}
\pgfpathlineto{\pgfpoint{32.955810\du}{13.783303\du}}
\pgfpathlineto{\pgfpoint{32.956541\du}{13.781478\du}}
\pgfpathlineto{\pgfpoint{32.956541\du}{13.779287\du}}
\pgfpathlineto{\pgfpoint{32.957271\du}{13.777462\du}}
\pgfpathlineto{\pgfpoint{32.956541\du}{13.775636\du}}
\pgfpathlineto{\pgfpoint{32.956541\du}{13.773811\du}}
\pgfpathlineto{\pgfpoint{32.955810\du}{13.771985\du}}
\pgfpathlineto{\pgfpoint{32.954715\du}{13.770525\du}}
\pgfpathlineto{\pgfpoint{32.953620\du}{13.769795\du}}
\pgfpathlineto{\pgfpoint{32.952890\du}{13.769065\du}}
\pgfpathlineto{\pgfpoint{32.951794\du}{13.768700\du}}
\pgfpathlineto{\pgfpoint{32.951429\du}{13.767969\du}}
\pgfpathlineto{\pgfpoint{32.950334\du}{13.767604\du}}
\pgfpathlineto{\pgfpoint{32.948874\du}{13.767604\du}}
\pgfpathlineto{\pgfpoint{32.947778\du}{13.767604\du}}
\pgfpathlineto{\pgfpoint{32.946683\du}{13.767604\du}}
\pgfusepath{fill}
\pgfsetbuttcap
\pgfsetmiterjoin
\pgfsetdash{}{0pt}
\definecolor{dialinecolor}{rgb}{0.678431, 0.839216, 0.905882}
\pgfsetfillcolor{dialinecolor}
\pgfpathmoveto{\pgfpoint{34.637081\du}{14.355045\du}}
\pgfpathlineto{\pgfpoint{34.637081\du}{14.346648\du}}
\pgfpathlineto{\pgfpoint{34.636351\du}{14.338981\du}}
\pgfpathlineto{\pgfpoint{34.635621\du}{14.331679\du}}
\pgfpathlineto{\pgfpoint{34.634890\du}{14.323281\du}}
\pgfpathlineto{\pgfpoint{34.633795\du}{14.315614\du}}
\pgfpathlineto{\pgfpoint{34.631604\du}{14.307947\du}}
\pgfpathlineto{\pgfpoint{34.630509\du}{14.300645\du}}
\pgfpathlineto{\pgfpoint{34.627588\du}{14.292978\du}}
\pgfpathlineto{\pgfpoint{34.625763\du}{14.285311\du}}
\pgfpathlineto{\pgfpoint{34.622842\du}{14.277644\du}}
\pgfpathlineto{\pgfpoint{34.620286\du}{14.269977\du}}
\pgfpathlineto{\pgfpoint{34.617001\du}{14.263041\du}}
\pgfpathlineto{\pgfpoint{34.612985\du}{14.255373\du}}
\pgfpathlineto{\pgfpoint{34.609699\du}{14.247706\du}}
\pgfpathlineto{\pgfpoint{34.605683\du}{14.240405\du}}
\pgfpathlineto{\pgfpoint{34.601667\du}{14.233103\du}}
\pgfpathlineto{\pgfpoint{34.597285\du}{14.225801\du}}
\pgfpathlineto{\pgfpoint{34.592539\du}{14.218499\du}}
\pgfpathlineto{\pgfpoint{34.587063\du}{14.211562\du}}
\pgfpathlineto{\pgfpoint{34.582316\du}{14.204625\du}}
\pgfpathlineto{\pgfpoint{34.576475\du}{14.196958\du}}
\pgfpathlineto{\pgfpoint{34.570998\du}{14.190021\du}}
\pgfpathlineto{\pgfpoint{34.565157\du}{14.183084\du}}
\pgfpathlineto{\pgfpoint{34.558950\du}{14.175782\du}}
\pgfpathlineto{\pgfpoint{34.552378\du}{14.169576\du}}
\pgfpathlineto{\pgfpoint{34.545442\du}{14.162274\du}}
\pgfpathlineto{\pgfpoint{34.538870\du}{14.155337\du}}
\pgfpathlineto{\pgfpoint{34.531568\du}{14.149130\du}}
\pgfpathlineto{\pgfpoint{34.524996\du}{14.141828\du}}
\pgfpathlineto{\pgfpoint{34.516964\du}{14.135622\du}}
\pgfpathlineto{\pgfpoint{34.508932\du}{14.129050\du}}
\pgfpathlineto{\pgfpoint{34.501630\du}{14.122113\du}}
\pgfpathlineto{\pgfpoint{34.492868\du}{14.115541\du}}
\pgfpathlineto{\pgfpoint{34.484836\du}{14.109335\du}}
\pgfpathlineto{\pgfpoint{34.475708\du}{14.102763\du}}
\pgfpathlineto{\pgfpoint{34.466946\du}{14.096191\du}}
\pgfpathlineto{\pgfpoint{34.457453\du}{14.089985\du}}
\pgfpathlineto{\pgfpoint{34.448326\du}{14.083413\du}}
\pgfpathlineto{\pgfpoint{34.439198\du}{14.076841\du}}
\pgfpathlineto{\pgfpoint{34.429341\du}{14.071000\du}}
\pgfpathlineto{\pgfpoint{34.419483\du}{14.064793\du}}
\pgfpathlineto{\pgfpoint{34.409261\du}{14.058221\du}}
\pgfpathlineto{\pgfpoint{34.398673\du}{14.052380\du}}
\pgfpathlineto{\pgfpoint{34.388815\du}{14.046538\du}}
\pgfpathlineto{\pgfpoint{34.378227\du}{14.040697\du}}
\pgfpathlineto{\pgfpoint{34.366909\du}{14.034125\du}}
\pgfpathlineto{\pgfpoint{34.355591\du}{14.029014\du}}
\pgfpathlineto{\pgfpoint{34.343908\du}{14.023172\du}}
\pgfpathlineto{\pgfpoint{34.332955\du}{14.017330\du}}
\pgfpathlineto{\pgfpoint{34.320907\du}{14.011489\du}}
\pgfpathlineto{\pgfpoint{34.309589\du}{14.005647\du}}
\pgfpathlineto{\pgfpoint{34.297176\du}{14.000536\du}}
\pgfpathlineto{\pgfpoint{34.284763\du}{13.994329\du}}
\pgfpathlineto{\pgfpoint{34.271984\du}{13.988488\du}}
\pgfpathlineto{\pgfpoint{34.259936\du}{13.983376\du}}
\pgfpathlineto{\pgfpoint{34.247158\du}{13.978265\du}}
\pgfpathlineto{\pgfpoint{34.233649\du}{13.972789\du}}
\pgfpathlineto{\pgfpoint{34.220505\du}{13.967677\du}}
\pgfpathlineto{\pgfpoint{34.207362\du}{13.962201\du}}
\pgfpathlineto{\pgfpoint{34.193123\du}{13.957089\du}}
\pgfpathlineto{\pgfpoint{34.179980\du}{13.951613\du}}
\pgfpathlineto{\pgfpoint{34.165741\du}{13.947232\du}}
\pgfpathlineto{\pgfpoint{34.151867\du}{13.941755\du}}
\pgfpathlineto{\pgfpoint{34.137629\du}{13.937009\du}}
\pgfpathlineto{\pgfpoint{34.123755\du}{13.931898\du}}
\pgfpathlineto{\pgfpoint{34.108786\du}{13.927152\du}}
\pgfpathlineto{\pgfpoint{34.093817\du}{13.922405\du}}
\pgfpathlineto{\pgfpoint{34.079578\du}{13.918024\du}}
\pgfpathlineto{\pgfpoint{34.048910\du}{13.908532\du}}
\pgfpathlineto{\pgfpoint{34.018242\du}{13.899769\du}}
\pgfpathlineto{\pgfpoint{33.986844\du}{13.891007\du}}
\pgfpathlineto{\pgfpoint{33.955080\du}{13.882245\du}}
\pgfpathlineto{\pgfpoint{33.922221\du}{13.873847\du}}
\pgfpathlineto{\pgfpoint{33.888633\du}{13.866180\du}}
\pgfpathlineto{\pgfpoint{33.854679\du}{13.858148\du}}
\pgfpathlineto{\pgfpoint{33.819994\du}{13.851211\du}}
\pgfpathlineto{\pgfpoint{33.785310\du}{13.844275\du}}
\pgfpathlineto{\pgfpoint{33.749896\du}{13.836973\du}}
\pgfpathlineto{\pgfpoint{33.713386\du}{13.830766\du}}
\pgfpathlineto{\pgfpoint{33.676876\du}{13.824194\du}}
\pgfpathlineto{\pgfpoint{33.640002\du}{13.818353\du}}
\pgfpathlineto{\pgfpoint{33.602762\du}{13.812511\du}}
\pgfpathlineto{\pgfpoint{33.563696\du}{13.807400\du}}
\pgfpathlineto{\pgfpoint{33.525726\du}{13.802654\du}}
\pgfpathlineto{\pgfpoint{33.486296\du}{13.797907\du}}
\pgfpathlineto{\pgfpoint{33.447596\du}{13.793161\du}}
\pgfpathlineto{\pgfpoint{33.407435\du}{13.789145\du}}
\pgfpathlineto{\pgfpoint{33.367639\du}{13.785129\du}}
\pgfpathlineto{\pgfpoint{33.326749\du}{13.782208\du}}
\pgfpathlineto{\pgfpoint{33.286223\du}{13.779287\du}}
\pgfpathlineto{\pgfpoint{33.244602\du}{13.776367\du}}
\pgfpathlineto{\pgfpoint{33.202981\du}{13.773811\du}}
\pgfpathlineto{\pgfpoint{33.161360\du}{13.771985\du}}
\pgfpathlineto{\pgfpoint{33.118644\du}{13.770525\du}}
\pgfpathlineto{\pgfpoint{33.076657\du}{13.769065\du}}
\pgfpathlineto{\pgfpoint{33.033576\du}{13.767969\du}}
\pgfpathlineto{\pgfpoint{32.990129\du}{13.767604\du}}
\pgfpathlineto{\pgfpoint{32.946683\du}{13.767604\du}}
\pgfpathlineto{\pgfpoint{32.946683\du}{13.788050\du}}
\pgfpathlineto{\pgfpoint{32.989764\du}{13.788050\du}}
\pgfpathlineto{\pgfpoint{33.033211\du}{13.788415\du}}
\pgfpathlineto{\pgfpoint{33.075197\du}{13.789145\du}}
\pgfpathlineto{\pgfpoint{33.117913\du}{13.790970\du}}
\pgfpathlineto{\pgfpoint{33.160630\du}{13.792066\du}}
\pgfpathlineto{\pgfpoint{33.202251\du}{13.794256\du}}
\pgfpathlineto{\pgfpoint{33.243507\du}{13.796812\du}}
\pgfpathlineto{\pgfpoint{33.284397\du}{13.799733\du}}
\pgfpathlineto{\pgfpoint{33.325653\du}{13.802654\du}}
\pgfpathlineto{\pgfpoint{33.365814\du}{13.805939\du}}
\pgfpathlineto{\pgfpoint{33.405975\du}{13.809590\du}}
\pgfpathlineto{\pgfpoint{33.445040\du}{13.813606\du}}
\pgfpathlineto{\pgfpoint{33.484470\du}{13.817622\du}}
\pgfpathlineto{\pgfpoint{33.523171\du}{13.822369\du}}
\pgfpathlineto{\pgfpoint{33.561871\du}{13.827845\du}}
\pgfpathlineto{\pgfpoint{33.599476\du}{13.832957\du}}
\pgfpathlineto{\pgfpoint{33.636716\du}{13.838798\du}}
\pgfpathlineto{\pgfpoint{33.673225\du}{13.844640\du}}
\pgfpathlineto{\pgfpoint{33.710100\du}{13.851211\du}}
\pgfpathlineto{\pgfpoint{33.745515\du}{13.857053\du}}
\pgfpathlineto{\pgfpoint{33.781294\du}{13.863990\du}}
\pgfpathlineto{\pgfpoint{33.815978\du}{13.870927\du}}
\pgfpathlineto{\pgfpoint{33.849932\du}{13.878594\du}}
\pgfpathlineto{\pgfpoint{33.884251\du}{13.886261\du}}
\pgfpathlineto{\pgfpoint{33.917475\du}{13.893928\du}}
\pgfpathlineto{\pgfpoint{33.949604\du}{13.901960\du}}
\pgfpathlineto{\pgfpoint{33.981732\du}{13.910722\du}}
\pgfpathlineto{\pgfpoint{34.013131\du}{13.919119\du}}
\pgfpathlineto{\pgfpoint{34.043799\du}{13.928247\du}}
\pgfpathlineto{\pgfpoint{34.073737\du}{13.937009\du}}
\pgfpathlineto{\pgfpoint{34.087975\du}{13.941755\du}}
\pgfpathlineto{\pgfpoint{34.102579\du}{13.946502\du}}
\pgfpathlineto{\pgfpoint{34.116453\du}{13.951613\du}}
\pgfpathlineto{\pgfpoint{34.131057\du}{13.956359\du}}
\pgfpathlineto{\pgfpoint{34.145296\du}{13.961106\du}}
\pgfpathlineto{\pgfpoint{34.159169\du}{13.966582\du}}
\pgfpathlineto{\pgfpoint{34.173408\du}{13.970963\du}}
\pgfpathlineto{\pgfpoint{34.186551\du}{13.976440\du}}
\pgfpathlineto{\pgfpoint{34.200060\du}{13.981551\du}}
\pgfpathlineto{\pgfpoint{34.213204\du}{13.986297\du}}
\pgfpathlineto{\pgfpoint{34.225982\du}{13.992139\du}}
\pgfpathlineto{\pgfpoint{34.238760\du}{13.997250\du}}
\pgfpathlineto{\pgfpoint{34.251904\du}{14.002727\du}}
\pgfpathlineto{\pgfpoint{34.263952\du}{14.007838\du}}
\pgfpathlineto{\pgfpoint{34.276365\du}{14.013314\du}}
\pgfpathlineto{\pgfpoint{34.288413\du}{14.019156\du}}
\pgfpathlineto{\pgfpoint{34.300827\du}{14.024267\du}}
\pgfpathlineto{\pgfpoint{34.312145\du}{14.029744\du}}
\pgfpathlineto{\pgfpoint{34.324193\du}{14.035585\du}}
\pgfpathlineto{\pgfpoint{34.334781\du}{14.041427\du}}
\pgfpathlineto{\pgfpoint{34.346829\du}{14.047268\du}}
\pgfpathlineto{\pgfpoint{34.357052\du}{14.053110\du}}
\pgfpathlineto{\pgfpoint{34.368005\du}{14.058221\du}}
\pgfpathlineto{\pgfpoint{34.378957\du}{14.064793\du}}
\pgfpathlineto{\pgfpoint{34.388815\du}{14.070635\du}}
\pgfpathlineto{\pgfpoint{34.398673\du}{14.076476\du}}
\pgfpathlineto{\pgfpoint{34.408895\du}{14.082318\du}}
\pgfpathlineto{\pgfpoint{34.418023\du}{14.088159\du}}
\pgfpathlineto{\pgfpoint{34.428246\du}{14.094731\du}}
\pgfpathlineto{\pgfpoint{34.437738\du}{14.100572\du}}
\pgfpathlineto{\pgfpoint{34.446500\du}{14.106779\du}}
\pgfpathlineto{\pgfpoint{34.454533\du}{14.112621\du}}
\pgfpathlineto{\pgfpoint{34.463295\du}{14.119192\du}}
\pgfpathlineto{\pgfpoint{34.471692\du}{14.125399\du}}
\pgfpathlineto{\pgfpoint{34.480454\du}{14.131971\du}}
\pgfpathlineto{\pgfpoint{34.488121\du}{14.138543\du}}
\pgfpathlineto{\pgfpoint{34.496154\du}{14.144384\du}}
\pgfpathlineto{\pgfpoint{34.503455\du}{14.150591\du}}
\pgfpathlineto{\pgfpoint{34.511123\du}{14.157893\du}}
\pgfpathlineto{\pgfpoint{34.517694\du}{14.163734\du}}
\pgfpathlineto{\pgfpoint{34.524996\du}{14.170671\du}}
\pgfpathlineto{\pgfpoint{34.530838\du}{14.177243\du}}
\pgfpathlineto{\pgfpoint{34.537775\du}{14.183449\du}}
\pgfpathlineto{\pgfpoint{34.544346\du}{14.190021\du}}
\pgfpathlineto{\pgfpoint{34.549458\du}{14.196958\du}}
\pgfpathlineto{\pgfpoint{34.554934\du}{14.203530\du}}
\pgfpathlineto{\pgfpoint{34.561141\du}{14.210467\du}}
\pgfpathlineto{\pgfpoint{34.565887\du}{14.216673\du}}
\pgfpathlineto{\pgfpoint{34.570998\du}{14.223245\du}}
\pgfpathlineto{\pgfpoint{34.575745\du}{14.230182\du}}
\pgfpathlineto{\pgfpoint{34.580126\du}{14.236754\du}}
\pgfpathlineto{\pgfpoint{34.584507\du}{14.243690\du}}
\pgfpathlineto{\pgfpoint{34.587428\du}{14.250262\du}}
\pgfpathlineto{\pgfpoint{34.591444\du}{14.257199\du}}
\pgfpathlineto{\pgfpoint{34.594365\du}{14.264136\du}}
\pgfpathlineto{\pgfpoint{34.598381\du}{14.271438\du}}
\pgfpathlineto{\pgfpoint{34.600936\du}{14.277644\du}}
\pgfpathlineto{\pgfpoint{34.603857\du}{14.284581\du}}
\pgfpathlineto{\pgfpoint{34.606413\du}{14.291883\du}}
\pgfpathlineto{\pgfpoint{34.608603\du}{14.298820\du}}
\pgfpathlineto{\pgfpoint{34.610064\du}{14.305757\du}}
\pgfpathlineto{\pgfpoint{34.612254\du}{14.312694\du}}
\pgfpathlineto{\pgfpoint{34.612985\du}{14.319996\du}}
\pgfpathlineto{\pgfpoint{34.614080\du}{14.326202\du}}
\pgfpathlineto{\pgfpoint{34.615905\du}{14.333869\du}}
\pgfpathlineto{\pgfpoint{34.616270\du}{14.340806\du}}
\pgfpathlineto{\pgfpoint{34.616270\du}{14.347743\du}}
\pgfpathlineto{\pgfpoint{34.617001\du}{14.355045\du}}
\pgfpathlineto{\pgfpoint{34.637081\du}{14.355045\du}}
\pgfusepath{fill}
\pgfsetbuttcap
\pgfsetmiterjoin
\pgfsetdash{}{0pt}
\definecolor{dialinecolor}{rgb}{0.027451, 0.486275, 0.682353}
\pgfsetfillcolor{dialinecolor}
\pgfpathmoveto{\pgfpoint{31.261031\du}{13.545625\du}}
\pgfpathlineto{\pgfpoint{31.261031\du}{14.370014\du}}
\pgfpathlineto{\pgfpoint{34.626493\du}{14.370014\du}}
\pgfpathlineto{\pgfpoint{34.627223\du}{13.546356\du}}
\pgfpathlineto{\pgfpoint{31.261031\du}{13.545625\du}}
\pgfusepath{fill}
\pgfsetbuttcap
\pgfsetmiterjoin
\pgfsetdash{}{0pt}
\definecolor{dialinecolor}{rgb}{0.235294, 0.686275, 0.894118}
\pgfsetfillcolor{dialinecolor}
\pgfpathmoveto{\pgfpoint{34.626493\du}{13.529926\du}}
\pgfpathlineto{\pgfpoint{34.625033\du}{13.559864\du}}
\pgfpathlineto{\pgfpoint{34.617731\du}{13.589072\du}}
\pgfpathlineto{\pgfpoint{34.607508\du}{13.618280\du}}
\pgfpathlineto{\pgfpoint{34.592904\du}{13.646392\du}}
\pgfpathlineto{\pgfpoint{34.573554\du}{13.674505\du}}
\pgfpathlineto{\pgfpoint{34.551648\du}{13.701887\du}}
\pgfpathlineto{\pgfpoint{34.524996\du}{13.728174\du}}
\pgfpathlineto{\pgfpoint{34.494328\du}{13.754461\du}}
\pgfpathlineto{\pgfpoint{34.461469\du}{13.780383\du}}
\pgfpathlineto{\pgfpoint{34.424229\du}{13.805574\du}}
\pgfpathlineto{\pgfpoint{34.383339\du}{13.829671\du}}
\pgfpathlineto{\pgfpoint{34.339527\du}{13.853037\du}}
\pgfpathlineto{\pgfpoint{34.293160\du}{13.875308\du}}
\pgfpathlineto{\pgfpoint{34.242776\du}{13.897214\du}}
\pgfpathlineto{\pgfpoint{34.189837\du}{13.918389\du}}
\pgfpathlineto{\pgfpoint{34.134343\du}{13.938470\du}}
\pgfpathlineto{\pgfpoint{34.076292\du}{13.957089\du}}
\pgfpathlineto{\pgfpoint{34.015686\du}{13.975709\du}}
\pgfpathlineto{\pgfpoint{33.951794\du}{13.992869\du}}
\pgfpathlineto{\pgfpoint{33.886807\du}{14.008568\du}}
\pgfpathlineto{\pgfpoint{33.818169\du}{14.023902\du}}
\pgfpathlineto{\pgfpoint{33.747340\du}{14.037776\du}}
\pgfpathlineto{\pgfpoint{33.675416\du}{14.050554\du}}
\pgfpathlineto{\pgfpoint{33.600571\du}{14.061872\du}}
\pgfpathlineto{\pgfpoint{33.524631\du}{14.072460\du}}
\pgfpathlineto{\pgfpoint{33.446135\du}{14.081587\du}}
\pgfpathlineto{\pgfpoint{33.366544\du}{14.089254\du}}
\pgfpathlineto{\pgfpoint{33.285493\du}{14.095826\du}}
\pgfpathlineto{\pgfpoint{33.202251\du}{14.100938\du}}
\pgfpathlineto{\pgfpoint{33.118644\du}{14.104589\du}}
\pgfpathlineto{\pgfpoint{33.033211\du}{14.106414\du}}
\pgfpathlineto{\pgfpoint{32.946683\du}{14.107509\du}}
\pgfpathlineto{\pgfpoint{32.860520\du}{14.106414\du}}
\pgfpathlineto{\pgfpoint{32.774722\du}{14.104589\du}}
\pgfpathlineto{\pgfpoint{32.691115\du}{14.100938\du}}
\pgfpathlineto{\pgfpoint{32.608238\du}{14.095826\du}}
\pgfpathlineto{\pgfpoint{32.526822\du}{14.089254\du}}
\pgfpathlineto{\pgfpoint{32.447231\du}{14.081587\du}}
\pgfpathlineto{\pgfpoint{32.369465\du}{14.072460\du}}
\pgfpathlineto{\pgfpoint{32.292795\du}{14.061872\du}}
\pgfpathlineto{\pgfpoint{32.218680\du}{14.050554\du}}
\pgfpathlineto{\pgfpoint{32.146026\du}{14.037776\du}}
\pgfpathlineto{\pgfpoint{32.075562\du}{14.023902\du}}
\pgfpathlineto{\pgfpoint{32.006924\du}{14.008568\du}}
\pgfpathlineto{\pgfpoint{31.941206\du}{13.992869\du}}
\pgfpathlineto{\pgfpoint{31.877680\du}{13.975709\du}}
\pgfpathlineto{\pgfpoint{31.816708\du}{13.957089\du}}
\pgfpathlineto{\pgfpoint{31.758293\du}{13.938470\du}}
\pgfpathlineto{\pgfpoint{31.703163\du}{13.918389\du}}
\pgfpathlineto{\pgfpoint{31.650224\du}{13.897214\du}}
\pgfpathlineto{\pgfpoint{31.600206\du}{13.875308\du}}
\pgfpathlineto{\pgfpoint{31.553109\du}{13.853037\du}}
\pgfpathlineto{\pgfpoint{31.509662\du}{13.829671\du}}
\pgfpathlineto{\pgfpoint{31.468771\du}{13.805574\du}}
\pgfpathlineto{\pgfpoint{31.431531\du}{13.780383\du}}
\pgfpathlineto{\pgfpoint{31.398308\du}{13.754461\du}}
\pgfpathlineto{\pgfpoint{31.368005\du}{13.728174\du}}
\pgfpathlineto{\pgfpoint{31.341353\du}{13.701887\du}}
\pgfpathlineto{\pgfpoint{31.319447\du}{13.674505\du}}
\pgfpathlineto{\pgfpoint{31.300097\du}{13.646392\du}}
\pgfpathlineto{\pgfpoint{31.285493\du}{13.618280\du}}
\pgfpathlineto{\pgfpoint{31.274905\du}{13.589072\du}}
\pgfpathlineto{\pgfpoint{31.267968\du}{13.559864\du}}
\pgfpathlineto{\pgfpoint{31.266143\du}{13.529926\du}}
\pgfpathlineto{\pgfpoint{31.267968\du}{13.500719\du}}
\pgfpathlineto{\pgfpoint{31.274905\du}{13.470781\du}}
\pgfpathlineto{\pgfpoint{31.285493\du}{13.442303\du}}
\pgfpathlineto{\pgfpoint{31.300097\du}{13.414191\du}}
\pgfpathlineto{\pgfpoint{31.319447\du}{13.386078\du}}
\pgfpathlineto{\pgfpoint{31.341353\du}{13.358331\du}}
\pgfpathlineto{\pgfpoint{31.368005\du}{13.331679\du}}
\pgfpathlineto{\pgfpoint{31.398308\du}{13.305392\du}}
\pgfpathlineto{\pgfpoint{31.431531\du}{13.280200\du}}
\pgfpathlineto{\pgfpoint{31.468771\du}{13.255008\du}}
\pgfpathlineto{\pgfpoint{31.509662\du}{13.230912\du}}
\pgfpathlineto{\pgfpoint{31.553109\du}{13.207546\du}}
\pgfpathlineto{\pgfpoint{31.600206\du}{13.184545\du}}
\pgfpathlineto{\pgfpoint{31.650224\du}{13.163004\du}}
\pgfpathlineto{\pgfpoint{31.703163\du}{13.141828\du}}
\pgfpathlineto{\pgfpoint{31.758293\du}{13.122113\du}}
\pgfpathlineto{\pgfpoint{31.816708\du}{13.102763\du}}
\pgfpathlineto{\pgfpoint{31.877680\du}{13.084508\du}}
\pgfpathlineto{\pgfpoint{31.941206\du}{13.067714\du}}
\pgfpathlineto{\pgfpoint{32.006924\du}{13.051284\du}}
\pgfpathlineto{\pgfpoint{32.075562\du}{13.035950\du}}
\pgfpathlineto{\pgfpoint{32.146026\du}{13.022442\du}}
\pgfpathlineto{\pgfpoint{32.218680\du}{13.009663\du}}
\pgfpathlineto{\pgfpoint{32.292795\du}{12.997980\du}}
\pgfpathlineto{\pgfpoint{32.369465\du}{12.988123\du}}
\pgfpathlineto{\pgfpoint{32.447231\du}{12.978630\du}}
\pgfpathlineto{\pgfpoint{32.526822\du}{12.970963\du}}
\pgfpathlineto{\pgfpoint{32.608238\du}{12.964391\du}}
\pgfpathlineto{\pgfpoint{32.691115\du}{12.959280\du}}
\pgfpathlineto{\pgfpoint{32.774722\du}{12.955994\du}}
\pgfpathlineto{\pgfpoint{32.860520\du}{12.953438\du}}
\pgfpathlineto{\pgfpoint{32.946683\du}{12.953073\du}}
\pgfpathlineto{\pgfpoint{33.033211\du}{12.953438\du}}
\pgfpathlineto{\pgfpoint{33.118644\du}{12.955994\du}}
\pgfpathlineto{\pgfpoint{33.202251\du}{12.959280\du}}
\pgfpathlineto{\pgfpoint{33.285493\du}{12.964391\du}}
\pgfpathlineto{\pgfpoint{33.366544\du}{12.970963\du}}
\pgfpathlineto{\pgfpoint{33.446135\du}{12.978630\du}}
\pgfpathlineto{\pgfpoint{33.524631\du}{12.988123\du}}
\pgfpathlineto{\pgfpoint{33.600571\du}{12.997980\du}}
\pgfpathlineto{\pgfpoint{33.675416\du}{13.009663\du}}
\pgfpathlineto{\pgfpoint{33.747340\du}{13.022442\du}}
\pgfpathlineto{\pgfpoint{33.818169\du}{13.035950\du}}
\pgfpathlineto{\pgfpoint{33.886807\du}{13.051284\du}}
\pgfpathlineto{\pgfpoint{33.951794\du}{13.067714\du}}
\pgfpathlineto{\pgfpoint{34.015686\du}{13.084508\du}}
\pgfpathlineto{\pgfpoint{34.076292\du}{13.102763\du}}
\pgfpathlineto{\pgfpoint{34.134343\du}{13.122113\du}}
\pgfpathlineto{\pgfpoint{34.189837\du}{13.141828\du}}
\pgfpathlineto{\pgfpoint{34.242776\du}{13.163004\du}}
\pgfpathlineto{\pgfpoint{34.293160\du}{13.184545\du}}
\pgfpathlineto{\pgfpoint{34.339527\du}{13.207546\du}}
\pgfpathlineto{\pgfpoint{34.383339\du}{13.230912\du}}
\pgfpathlineto{\pgfpoint{34.424229\du}{13.255008\du}}
\pgfpathlineto{\pgfpoint{34.461469\du}{13.280200\du}}
\pgfpathlineto{\pgfpoint{34.494328\du}{13.305392\du}}
\pgfpathlineto{\pgfpoint{34.524996\du}{13.331679\du}}
\pgfpathlineto{\pgfpoint{34.551648\du}{13.358331\du}}
\pgfpathlineto{\pgfpoint{34.573554\du}{13.386078\du}}
\pgfpathlineto{\pgfpoint{34.592904\du}{13.414191\du}}
\pgfpathlineto{\pgfpoint{34.607508\du}{13.442303\du}}
\pgfpathlineto{\pgfpoint{34.617731\du}{13.470781\du}}
\pgfpathlineto{\pgfpoint{34.625033\du}{13.500719\du}}
\pgfpathlineto{\pgfpoint{34.626493\du}{13.529926\du}}
\pgfusepath{fill}
\pgfsetbuttcap
\pgfsetmiterjoin
\pgfsetdash{}{0pt}
\definecolor{dialinecolor}{rgb}{0.678431, 0.839216, 0.905882}
\pgfsetfillcolor{dialinecolor}
\pgfpathmoveto{\pgfpoint{32.946683\du}{14.117367\du}}
\pgfpathlineto{\pgfpoint{32.946683\du}{14.117367\du}}
\pgfpathlineto{\pgfpoint{32.990129\du}{14.117367\du}}
\pgfpathlineto{\pgfpoint{33.033576\du}{14.116637\du}}
\pgfpathlineto{\pgfpoint{33.076657\du}{14.115541\du}}
\pgfpathlineto{\pgfpoint{33.118644\du}{14.114446\du}}
\pgfpathlineto{\pgfpoint{33.161360\du}{14.112621\du}}
\pgfpathlineto{\pgfpoint{33.202981\du}{14.110795\du}}
\pgfpathlineto{\pgfpoint{33.244602\du}{14.108605\du}}
\pgfpathlineto{\pgfpoint{33.286223\du}{14.105684\du}}
\pgfpathlineto{\pgfpoint{33.326749\du}{14.102763\du}}
\pgfpathlineto{\pgfpoint{33.367639\du}{14.099842\du}}
\pgfpathlineto{\pgfpoint{33.407435\du}{14.095826\du}}
\pgfpathlineto{\pgfpoint{33.447596\du}{14.091810\du}}
\pgfpathlineto{\pgfpoint{33.486296\du}{14.087064\du}}
\pgfpathlineto{\pgfpoint{33.525726\du}{14.082318\du}}
\pgfpathlineto{\pgfpoint{33.563696\du}{14.077571\du}}
\pgfpathlineto{\pgfpoint{33.602762\du}{14.072460\du}}
\pgfpathlineto{\pgfpoint{33.640002\du}{14.066618\du}}
\pgfpathlineto{\pgfpoint{33.676876\du}{14.060777\du}}
\pgfpathlineto{\pgfpoint{33.713386\du}{14.054205\du}}
\pgfpathlineto{\pgfpoint{33.749896\du}{14.047633\du}}
\pgfpathlineto{\pgfpoint{33.785310\du}{14.040697\du}}
\pgfpathlineto{\pgfpoint{33.819994\du}{14.033760\du}}
\pgfpathlineto{\pgfpoint{33.854679\du}{14.026823\du}}
\pgfpathlineto{\pgfpoint{33.888633\du}{14.019156\du}}
\pgfpathlineto{\pgfpoint{33.922221\du}{14.010759\du}}
\pgfpathlineto{\pgfpoint{33.955080\du}{14.002727\du}}
\pgfpathlineto{\pgfpoint{33.986844\du}{13.994329\du}}
\pgfpathlineto{\pgfpoint{34.018242\du}{13.985202\du}}
\pgfpathlineto{\pgfpoint{34.033576\du}{13.981186\du}}
\pgfpathlineto{\pgfpoint{34.048910\du}{13.976440\du}}
\pgfpathlineto{\pgfpoint{34.064974\du}{13.971693\du}}
\pgfpathlineto{\pgfpoint{34.079578\du}{13.966947\du}}
\pgfpathlineto{\pgfpoint{34.093817\du}{13.962201\du}}
\pgfpathlineto{\pgfpoint{34.108786\du}{13.957820\du}}
\pgfpathlineto{\pgfpoint{34.123755\du}{13.953073\du}}
\pgfpathlineto{\pgfpoint{34.137629\du}{13.947597\du}}
\pgfpathlineto{\pgfpoint{34.151867\du}{13.942851\du}}
\pgfpathlineto{\pgfpoint{34.165741\du}{13.937739\du}}
\pgfpathlineto{\pgfpoint{34.179980\du}{13.932993\du}}
\pgfpathlineto{\pgfpoint{34.193123\du}{13.927882\du}}
\pgfpathlineto{\pgfpoint{34.207362\du}{13.922405\du}}
\pgfpathlineto{\pgfpoint{34.220505\du}{13.917294\du}}
\pgfpathlineto{\pgfpoint{34.233649\du}{13.912183\du}}
\pgfpathlineto{\pgfpoint{34.247158\du}{13.906706\du}}
\pgfpathlineto{\pgfpoint{34.259936\du}{13.901595\du}}
\pgfpathlineto{\pgfpoint{34.271984\du}{13.896118\du}}
\pgfpathlineto{\pgfpoint{34.284763\du}{13.890277\du}}
\pgfpathlineto{\pgfpoint{34.297176\du}{13.884435\du}}
\pgfpathlineto{\pgfpoint{34.309589\du}{13.879324\du}}
\pgfpathlineto{\pgfpoint{34.320907\du}{13.873482\du}}
\pgfpathlineto{\pgfpoint{34.332955\du}{13.867641\du}}
\pgfpathlineto{\pgfpoint{34.343908\du}{13.861799\du}}
\pgfpathlineto{\pgfpoint{34.355591\du}{13.856323\du}}
\pgfpathlineto{\pgfpoint{34.366909\du}{13.850481\du}}
\pgfpathlineto{\pgfpoint{34.378227\du}{13.844275\du}}
\pgfpathlineto{\pgfpoint{34.388815\du}{13.838433\du}}
\pgfpathlineto{\pgfpoint{34.398673\du}{13.832591\du}}
\pgfpathlineto{\pgfpoint{34.409261\du}{13.826750\du}}
\pgfpathlineto{\pgfpoint{34.419483\du}{13.820178\du}}
\pgfpathlineto{\pgfpoint{34.429341\du}{13.813606\du}}
\pgfpathlineto{\pgfpoint{34.439198\du}{13.807765\du}}
\pgfpathlineto{\pgfpoint{34.448326\du}{13.801558\du}}
\pgfpathlineto{\pgfpoint{34.457453\du}{13.794986\du}}
\pgfpathlineto{\pgfpoint{34.466946\du}{13.788415\du}}
\pgfpathlineto{\pgfpoint{34.475708\du}{13.782208\du}}
\pgfpathlineto{\pgfpoint{34.484836\du}{13.775636\du}}
\pgfpathlineto{\pgfpoint{34.492868\du}{13.769065\du}}
\pgfpathlineto{\pgfpoint{34.501630\du}{13.762858\du}}
\pgfpathlineto{\pgfpoint{34.508932\du}{13.756286\du}}
\pgfpathlineto{\pgfpoint{34.516964\du}{13.749349\du}}
\pgfpathlineto{\pgfpoint{34.524996\du}{13.742778\du}}
\pgfpathlineto{\pgfpoint{34.531568\du}{13.735841\du}}
\pgfpathlineto{\pgfpoint{34.538870\du}{13.729269\du}}
\pgfpathlineto{\pgfpoint{34.545442\du}{13.722332\du}}
\pgfpathlineto{\pgfpoint{34.552378\du}{13.715395\du}}
\pgfpathlineto{\pgfpoint{34.558950\du}{13.708824\du}}
\pgfpathlineto{\pgfpoint{34.565157\du}{13.701887\du}}
\pgfpathlineto{\pgfpoint{34.570998\du}{13.694950\du}}
\pgfpathlineto{\pgfpoint{34.576475\du}{13.688013\du}}
\pgfpathlineto{\pgfpoint{34.582316\du}{13.680346\du}}
\pgfpathlineto{\pgfpoint{34.587063\du}{13.673409\du}}
\pgfpathlineto{\pgfpoint{34.592539\du}{13.666107\du}}
\pgfpathlineto{\pgfpoint{34.597285\du}{13.659171\du}}
\pgfpathlineto{\pgfpoint{34.601667\du}{13.651503\du}}
\pgfpathlineto{\pgfpoint{34.605683\du}{13.644567\du}}
\pgfpathlineto{\pgfpoint{34.609699\du}{13.636900\du}}
\pgfpathlineto{\pgfpoint{34.612985\du}{13.629233\du}}
\pgfpathlineto{\pgfpoint{34.617001\du}{13.621931\du}}
\pgfpathlineto{\pgfpoint{34.620286\du}{13.614629\du}}
\pgfpathlineto{\pgfpoint{34.622842\du}{13.607327\du}}
\pgfpathlineto{\pgfpoint{34.625763\du}{13.599660\du}}
\pgfpathlineto{\pgfpoint{34.627588\du}{13.591993\du}}
\pgfpathlineto{\pgfpoint{34.630509\du}{13.584326\du}}
\pgfpathlineto{\pgfpoint{34.631604\du}{13.576659\du}}
\pgfpathlineto{\pgfpoint{34.633795\du}{13.568992\du}}
\pgfpathlineto{\pgfpoint{34.634890\du}{13.561690\du}}
\pgfpathlineto{\pgfpoint{34.635621\du}{13.553292\du}}
\pgfpathlineto{\pgfpoint{34.636351\du}{13.545625\du}}
\pgfpathlineto{\pgfpoint{34.637081\du}{13.537958\du}}
\pgfpathlineto{\pgfpoint{34.637081\du}{13.529926\du}}
\pgfpathlineto{\pgfpoint{34.617001\du}{13.529926\du}}
\pgfpathlineto{\pgfpoint{34.616270\du}{13.536863\du}}
\pgfpathlineto{\pgfpoint{34.616270\du}{13.544530\du}}
\pgfpathlineto{\pgfpoint{34.615905\du}{13.551467\du}}
\pgfpathlineto{\pgfpoint{34.614080\du}{13.558404\du}}
\pgfpathlineto{\pgfpoint{34.612985\du}{13.565706\du}}
\pgfpathlineto{\pgfpoint{34.612254\du}{13.571912\du}}
\pgfpathlineto{\pgfpoint{34.610064\du}{13.579214\du}}
\pgfpathlineto{\pgfpoint{34.608603\du}{13.586151\du}}
\pgfpathlineto{\pgfpoint{34.606413\du}{13.593088\du}}
\pgfpathlineto{\pgfpoint{34.603857\du}{13.600025\du}}
\pgfpathlineto{\pgfpoint{34.600936\du}{13.607327\du}}
\pgfpathlineto{\pgfpoint{34.598381\du}{13.613533\du}}
\pgfpathlineto{\pgfpoint{34.594365\du}{13.620470\du}}
\pgfpathlineto{\pgfpoint{34.591444\du}{13.627772\du}}
\pgfpathlineto{\pgfpoint{34.587428\du}{13.634709\du}}
\pgfpathlineto{\pgfpoint{34.584507\du}{13.640916\du}}
\pgfpathlineto{\pgfpoint{34.580126\du}{13.648218\du}}
\pgfpathlineto{\pgfpoint{34.575745\du}{13.654424\du}}
\pgfpathlineto{\pgfpoint{34.570998\du}{13.661726\du}}
\pgfpathlineto{\pgfpoint{34.565887\du}{13.667933\du}}
\pgfpathlineto{\pgfpoint{34.561141\du}{13.674870\du}}
\pgfpathlineto{\pgfpoint{34.554934\du}{13.681441\du}}
\pgfpathlineto{\pgfpoint{34.549458\du}{13.688013\du}}
\pgfpathlineto{\pgfpoint{34.544346\du}{13.694950\du}}
\pgfpathlineto{\pgfpoint{34.537775\du}{13.701522\du}}
\pgfpathlineto{\pgfpoint{34.530838\du}{13.707728\du}}
\pgfpathlineto{\pgfpoint{34.524996\du}{13.714300\du}}
\pgfpathlineto{\pgfpoint{34.517694\du}{13.721237\du}}
\pgfpathlineto{\pgfpoint{34.511123\du}{13.727809\du}}
\pgfpathlineto{\pgfpoint{34.503455\du}{13.734015\du}}
\pgfpathlineto{\pgfpoint{34.496154\du}{13.740587\du}}
\pgfpathlineto{\pgfpoint{34.488121\du}{13.746429\du}}
\pgfpathlineto{\pgfpoint{34.480454\du}{13.753000\du}}
\pgfpathlineto{\pgfpoint{34.471692\du}{13.759207\du}}
\pgfpathlineto{\pgfpoint{34.463295\du}{13.765779\du}}
\pgfpathlineto{\pgfpoint{34.454533\du}{13.771985\du}}
\pgfpathlineto{\pgfpoint{34.446500\du}{13.777827\du}}
\pgfpathlineto{\pgfpoint{34.437738\du}{13.784399\du}}
\pgfpathlineto{\pgfpoint{34.428246\du}{13.790240\du}}
\pgfpathlineto{\pgfpoint{34.418023\du}{13.796812\du}}
\pgfpathlineto{\pgfpoint{34.408895\du}{13.802654\du}}
\pgfpathlineto{\pgfpoint{34.398673\du}{13.808495\du}}
\pgfpathlineto{\pgfpoint{34.388815\du}{13.814337\du}}
\pgfpathlineto{\pgfpoint{34.378957\du}{13.820178\du}}
\pgfpathlineto{\pgfpoint{34.368005\du}{13.826750\du}}
\pgfpathlineto{\pgfpoint{34.357052\du}{13.831861\du}}
\pgfpathlineto{\pgfpoint{34.346829\du}{13.837703\du}}
\pgfpathlineto{\pgfpoint{34.334781\du}{13.843544\du}}
\pgfpathlineto{\pgfpoint{34.324193\du}{13.849386\du}}
\pgfpathlineto{\pgfpoint{34.312145\du}{13.855227\du}}
\pgfpathlineto{\pgfpoint{34.300827\du}{13.860339\du}}
\pgfpathlineto{\pgfpoint{34.288413\du}{13.865815\du}}
\pgfpathlineto{\pgfpoint{34.276365\du}{13.871657\du}}
\pgfpathlineto{\pgfpoint{34.263952\du}{13.876768\du}}
\pgfpathlineto{\pgfpoint{34.251904\du}{13.882245\du}}
\pgfpathlineto{\pgfpoint{34.238760\du}{13.887356\du}}
\pgfpathlineto{\pgfpoint{34.225982\du}{13.892832\du}}
\pgfpathlineto{\pgfpoint{34.213204\du}{13.898674\du}}
\pgfpathlineto{\pgfpoint{34.200060\du}{13.903055\du}}
\pgfpathlineto{\pgfpoint{34.186551\du}{13.908532\du}}
\pgfpathlineto{\pgfpoint{34.173408\du}{13.913643\du}}
\pgfpathlineto{\pgfpoint{34.159169\du}{13.918389\du}}
\pgfpathlineto{\pgfpoint{34.145296\du}{13.923866\du}}
\pgfpathlineto{\pgfpoint{34.131057\du}{13.928247\du}}
\pgfpathlineto{\pgfpoint{34.116453\du}{13.932993\du}}
\pgfpathlineto{\pgfpoint{34.102579\du}{13.938470\du}}
\pgfpathlineto{\pgfpoint{34.087975\du}{13.942851\du}}
\pgfpathlineto{\pgfpoint{34.073737\du}{13.947597\du}}
\pgfpathlineto{\pgfpoint{34.058037\du}{13.952343\du}}
\pgfpathlineto{\pgfpoint{34.043799\du}{13.956359\du}}
\pgfpathlineto{\pgfpoint{34.028100\du}{13.961106\du}}
\pgfpathlineto{\pgfpoint{34.013131\du}{13.965852\du}}
\pgfpathlineto{\pgfpoint{33.981732\du}{13.973884\du}}
\pgfpathlineto{\pgfpoint{33.949604\du}{13.982646\du}}
\pgfpathlineto{\pgfpoint{33.917475\du}{13.991043\du}}
\pgfpathlineto{\pgfpoint{33.884251\du}{13.998710\du}}
\pgfpathlineto{\pgfpoint{33.849932\du}{14.006378\du}}
\pgfpathlineto{\pgfpoint{33.815978\du}{14.013679\du}}
\pgfpathlineto{\pgfpoint{33.781294\du}{14.020981\du}}
\pgfpathlineto{\pgfpoint{33.745515\du}{14.027918\du}}
\pgfpathlineto{\pgfpoint{33.710100\du}{14.034125\du}}
\pgfpathlineto{\pgfpoint{33.673225\du}{14.039966\du}}
\pgfpathlineto{\pgfpoint{33.636716\du}{14.046173\du}}
\pgfpathlineto{\pgfpoint{33.599476\du}{14.052015\du}}
\pgfpathlineto{\pgfpoint{33.561871\du}{14.057126\du}}
\pgfpathlineto{\pgfpoint{33.523171\du}{14.062237\du}}
\pgfpathlineto{\pgfpoint{33.484470\du}{14.066984\du}}
\pgfpathlineto{\pgfpoint{33.445040\du}{14.071000\du}}
\pgfpathlineto{\pgfpoint{33.405975\du}{14.075381\du}}
\pgfpathlineto{\pgfpoint{33.365814\du}{14.078667\du}}
\pgfpathlineto{\pgfpoint{33.325653\du}{14.082318\du}}
\pgfpathlineto{\pgfpoint{33.284397\du}{14.085238\du}}
\pgfpathlineto{\pgfpoint{33.243507\du}{14.088159\du}}
\pgfpathlineto{\pgfpoint{33.202251\du}{14.090350\du}}
\pgfpathlineto{\pgfpoint{33.160630\du}{14.092905\du}}
\pgfpathlineto{\pgfpoint{33.117913\du}{14.094001\du}}
\pgfpathlineto{\pgfpoint{33.075197\du}{14.095826\du}}
\pgfpathlineto{\pgfpoint{33.033211\du}{14.096191\du}}
\pgfpathlineto{\pgfpoint{32.989764\du}{14.096922\du}}
\pgfpathlineto{\pgfpoint{32.946683\du}{14.096922\du}}
\pgfpathlineto{\pgfpoint{32.946683\du}{14.096922\du}}
\pgfpathlineto{\pgfpoint{32.946683\du}{14.096922\du}}
\pgfpathlineto{\pgfpoint{32.945953\du}{14.097652\du}}
\pgfpathlineto{\pgfpoint{32.944127\du}{14.097652\du}}
\pgfpathlineto{\pgfpoint{32.943032\du}{14.097652\du}}
\pgfpathlineto{\pgfpoint{32.942302\du}{14.098017\du}}
\pgfpathlineto{\pgfpoint{32.941937\du}{14.098747\du}}
\pgfpathlineto{\pgfpoint{32.940476\du}{14.098747\du}}
\pgfpathlineto{\pgfpoint{32.939746\du}{14.099842\du}}
\pgfpathlineto{\pgfpoint{32.939016\du}{14.100572\du}}
\pgfpathlineto{\pgfpoint{32.937921\du}{14.101668\du}}
\pgfpathlineto{\pgfpoint{32.937190\du}{14.103493\du}}
\pgfpathlineto{\pgfpoint{32.937190\du}{14.105684\du}}
\pgfpathlineto{\pgfpoint{32.936460\du}{14.107509\du}}
\pgfpathlineto{\pgfpoint{32.937190\du}{14.109335\du}}
\pgfpathlineto{\pgfpoint{32.937190\du}{14.110795\du}}
\pgfpathlineto{\pgfpoint{32.937921\du}{14.112621\du}}
\pgfpathlineto{\pgfpoint{32.939016\du}{14.114446\du}}
\pgfpathlineto{\pgfpoint{32.939746\du}{14.115176\du}}
\pgfpathlineto{\pgfpoint{32.940476\du}{14.115541\du}}
\pgfpathlineto{\pgfpoint{32.941937\du}{14.116272\du}}
\pgfpathlineto{\pgfpoint{32.942302\du}{14.116637\du}}
\pgfpathlineto{\pgfpoint{32.943032\du}{14.117367\du}}
\pgfpathlineto{\pgfpoint{32.944127\du}{14.117367\du}}
\pgfpathlineto{\pgfpoint{32.945953\du}{14.117367\du}}
\pgfpathlineto{\pgfpoint{32.946683\du}{14.117367\du}}
\pgfusepath{fill}
\pgfsetbuttcap
\pgfsetmiterjoin
\pgfsetdash{}{0pt}
\definecolor{dialinecolor}{rgb}{0.678431, 0.839216, 0.905882}
\pgfsetfillcolor{dialinecolor}
\pgfpathmoveto{\pgfpoint{31.255920\du}{13.529926\du}}
\pgfpathlineto{\pgfpoint{31.255920\du}{13.529926\du}}
\pgfpathlineto{\pgfpoint{31.255920\du}{13.537958\du}}
\pgfpathlineto{\pgfpoint{31.256285\du}{13.545625\du}}
\pgfpathlineto{\pgfpoint{31.257015\du}{13.553292\du}}
\pgfpathlineto{\pgfpoint{31.258110\du}{13.561690\du}}
\pgfpathlineto{\pgfpoint{31.259206\du}{13.568992\du}}
\pgfpathlineto{\pgfpoint{31.261031\du}{13.576659\du}}
\pgfpathlineto{\pgfpoint{31.262857\du}{13.584326\du}}
\pgfpathlineto{\pgfpoint{31.265047\du}{13.591993\du}}
\pgfpathlineto{\pgfpoint{31.267238\du}{13.599660\du}}
\pgfpathlineto{\pgfpoint{31.269794\du}{13.607327\du}}
\pgfpathlineto{\pgfpoint{31.272714\du}{13.614629\du}}
\pgfpathlineto{\pgfpoint{31.276365\du}{13.621931\du}}
\pgfpathlineto{\pgfpoint{31.279651\du}{13.629233\du}}
\pgfpathlineto{\pgfpoint{31.283302\du}{13.636900\du}}
\pgfpathlineto{\pgfpoint{31.287683\du}{13.644567\du}}
\pgfpathlineto{\pgfpoint{31.291334\du}{13.651503\du}}
\pgfpathlineto{\pgfpoint{31.296446\du}{13.659171\du}}
\pgfpathlineto{\pgfpoint{31.300462\du}{13.666107\du}}
\pgfpathlineto{\pgfpoint{31.305938\du}{13.673409\du}}
\pgfpathlineto{\pgfpoint{31.310684\du}{13.680346\du}}
\pgfpathlineto{\pgfpoint{31.316161\du}{13.688013\du}}
\pgfpathlineto{\pgfpoint{31.322002\du}{13.694950\du}}
\pgfpathlineto{\pgfpoint{31.327844\du}{13.701887\du}}
\pgfpathlineto{\pgfpoint{31.333685\du}{13.708824\du}}
\pgfpathlineto{\pgfpoint{31.340622\du}{13.715395\du}}
\pgfpathlineto{\pgfpoint{31.347194\du}{13.722332\du}}
\pgfpathlineto{\pgfpoint{31.354131\du}{13.729269\du}}
\pgfpathlineto{\pgfpoint{31.361068\du}{13.735841\du}}
\pgfpathlineto{\pgfpoint{31.368005\du}{13.742778\du}}
\pgfpathlineto{\pgfpoint{31.376767\du}{13.749349\du}}
\pgfpathlineto{\pgfpoint{31.383704\du}{13.756286\du}}
\pgfpathlineto{\pgfpoint{31.391371\du}{13.762858\du}}
\pgfpathlineto{\pgfpoint{31.400133\du}{13.769065\du}}
\pgfpathlineto{\pgfpoint{31.408530\du}{13.775636\du}}
\pgfpathlineto{\pgfpoint{31.417293\du}{13.782208\du}}
\pgfpathlineto{\pgfpoint{31.425690\du}{13.788415\du}}
\pgfpathlineto{\pgfpoint{31.435547\du}{13.794986\du}}
\pgfpathlineto{\pgfpoint{31.444675\du}{13.801558\du}}
\pgfpathlineto{\pgfpoint{31.454167\du}{13.807765\du}}
\pgfpathlineto{\pgfpoint{31.463660\du}{13.813606\du}}
\pgfpathlineto{\pgfpoint{31.473518\du}{13.820178\du}}
\pgfpathlineto{\pgfpoint{31.483740\du}{13.826750\du}}
\pgfpathlineto{\pgfpoint{31.493963\du}{13.832591\du}}
\pgfpathlineto{\pgfpoint{31.504551\du}{13.838433\du}}
\pgfpathlineto{\pgfpoint{31.514773\du}{13.844275\du}}
\pgfpathlineto{\pgfpoint{31.525726\du}{13.850481\du}}
\pgfpathlineto{\pgfpoint{31.537409\du}{13.856323\du}}
\pgfpathlineto{\pgfpoint{31.548727\du}{13.861799\du}}
\pgfpathlineto{\pgfpoint{31.560411\du}{13.867641\du}}
\pgfpathlineto{\pgfpoint{31.571729\du}{13.873482\du}}
\pgfpathlineto{\pgfpoint{31.583412\du}{13.879324\du}}
\pgfpathlineto{\pgfpoint{31.595825\du}{13.884435\du}}
\pgfpathlineto{\pgfpoint{31.607873\du}{13.890277\du}}
\pgfpathlineto{\pgfpoint{31.621017\du}{13.896118\du}}
\pgfpathlineto{\pgfpoint{31.633065\du}{13.901595\du}}
\pgfpathlineto{\pgfpoint{31.645843\du}{13.906706\du}}
\pgfpathlineto{\pgfpoint{31.659717\du}{13.912183\du}}
\pgfpathlineto{\pgfpoint{31.672130\du}{13.917294\du}}
\pgfpathlineto{\pgfpoint{31.685274\du}{13.922405\du}}
\pgfpathlineto{\pgfpoint{31.699512\du}{13.927882\du}}
\pgfpathlineto{\pgfpoint{31.712656\du}{13.932993\du}}
\pgfpathlineto{\pgfpoint{31.726895\du}{13.937739\du}}
\pgfpathlineto{\pgfpoint{31.740768\du}{13.942851\du}}
\pgfpathlineto{\pgfpoint{31.755372\du}{13.947597\du}}
\pgfpathlineto{\pgfpoint{31.769246\du}{13.953073\du}}
\pgfpathlineto{\pgfpoint{31.784945\du}{13.957820\du}}
\pgfpathlineto{\pgfpoint{31.798819\du}{13.962201\du}}
\pgfpathlineto{\pgfpoint{31.813423\du}{13.966947\du}}
\pgfpathlineto{\pgfpoint{31.828757\du}{13.971693\du}}
\pgfpathlineto{\pgfpoint{31.844456\du}{13.976440\du}}
\pgfpathlineto{\pgfpoint{31.859425\du}{13.981186\du}}
\pgfpathlineto{\pgfpoint{31.875124\du}{13.985202\du}}
\pgfpathlineto{\pgfpoint{31.906887\du}{13.994329\du}}
\pgfpathlineto{\pgfpoint{31.938651\du}{14.002727\du}}
\pgfpathlineto{\pgfpoint{31.971875\du}{14.010759\du}}
\pgfpathlineto{\pgfpoint{32.004368\du}{14.019156\du}}
\pgfpathlineto{\pgfpoint{32.039052\du}{14.026823\du}}
\pgfpathlineto{\pgfpoint{32.073372\du}{14.033760\du}}
\pgfpathlineto{\pgfpoint{32.108056\du}{14.040697\du}}
\pgfpathlineto{\pgfpoint{32.143835\du}{14.047633\du}}
\pgfpathlineto{\pgfpoint{32.179980\du}{14.054205\du}}
\pgfpathlineto{\pgfpoint{32.216489\du}{14.060777\du}}
\pgfpathlineto{\pgfpoint{32.253729\du}{14.066618\du}}
\pgfpathlineto{\pgfpoint{32.291334\du}{14.072460\du}}
\pgfpathlineto{\pgfpoint{32.329304\du}{14.077571\du}}
\pgfpathlineto{\pgfpoint{32.368005\du}{14.082318\du}}
\pgfpathlineto{\pgfpoint{32.406705\du}{14.087064\du}}
\pgfpathlineto{\pgfpoint{32.445770\du}{14.091810\du}}
\pgfpathlineto{\pgfpoint{32.485931\du}{14.095826\du}}
\pgfpathlineto{\pgfpoint{32.525726\du}{14.099842\du}}
\pgfpathlineto{\pgfpoint{32.566617\du}{14.102763\du}}
\pgfpathlineto{\pgfpoint{32.607508\du}{14.105684\du}}
\pgfpathlineto{\pgfpoint{32.648764\du}{14.108605\du}}
\pgfpathlineto{\pgfpoint{32.690750\du}{14.110795\du}}
\pgfpathlineto{\pgfpoint{32.732371\du}{14.112621\du}}
\pgfpathlineto{\pgfpoint{32.774722\du}{14.114446\du}}
\pgfpathlineto{\pgfpoint{32.816708\du}{14.115541\du}}
\pgfpathlineto{\pgfpoint{32.860155\du}{14.116637\du}}
\pgfpathlineto{\pgfpoint{32.902871\du}{14.117367\du}}
\pgfpathlineto{\pgfpoint{32.946683\du}{14.117367\du}}
\pgfpathlineto{\pgfpoint{32.946683\du}{14.096922\du}}
\pgfpathlineto{\pgfpoint{32.903967\du}{14.096922\du}}
\pgfpathlineto{\pgfpoint{32.860520\du}{14.096191\du}}
\pgfpathlineto{\pgfpoint{32.818169\du}{14.095826\du}}
\pgfpathlineto{\pgfpoint{32.775453\du}{14.094001\du}}
\pgfpathlineto{\pgfpoint{32.733101\du}{14.092905\du}}
\pgfpathlineto{\pgfpoint{32.691115\du}{14.090350\du}}
\pgfpathlineto{\pgfpoint{32.650224\du}{14.088159\du}}
\pgfpathlineto{\pgfpoint{32.608968\du}{14.085238\du}}
\pgfpathlineto{\pgfpoint{32.568078\du}{14.082318\du}}
\pgfpathlineto{\pgfpoint{32.527917\du}{14.078667\du}}
\pgfpathlineto{\pgfpoint{32.488121\du}{14.075381\du}}
\pgfpathlineto{\pgfpoint{32.448691\du}{14.071000\du}}
\pgfpathlineto{\pgfpoint{32.409261\du}{14.066984\du}}
\pgfpathlineto{\pgfpoint{32.370195\du}{14.062237\du}}
\pgfpathlineto{\pgfpoint{32.331860\du}{14.057126\du}}
\pgfpathlineto{\pgfpoint{32.294255\du}{14.052015\du}}
\pgfpathlineto{\pgfpoint{32.257015\du}{14.046173\du}}
\pgfpathlineto{\pgfpoint{32.220505\du}{14.039966\du}}
\pgfpathlineto{\pgfpoint{32.183266\du}{14.034125\du}}
\pgfpathlineto{\pgfpoint{32.148216\du}{14.027918\du}}
\pgfpathlineto{\pgfpoint{32.112072\du}{14.020981\du}}
\pgfpathlineto{\pgfpoint{32.077388\du}{14.013679\du}}
\pgfpathlineto{\pgfpoint{32.043068\du}{14.006378\du}}
\pgfpathlineto{\pgfpoint{32.009114\du}{13.998710\du}}
\pgfpathlineto{\pgfpoint{31.975891\du}{13.991043\du}}
\pgfpathlineto{\pgfpoint{31.944492\du}{13.982646\du}}
\pgfpathlineto{\pgfpoint{31.911999\du}{13.973884\du}}
\pgfpathlineto{\pgfpoint{31.880966\du}{13.965852\du}}
\pgfpathlineto{\pgfpoint{31.865266\du}{13.961106\du}}
\pgfpathlineto{\pgfpoint{31.849567\du}{13.956359\du}}
\pgfpathlineto{\pgfpoint{31.834963\du}{13.952343\du}}
\pgfpathlineto{\pgfpoint{31.819994\du}{13.947597\du}}
\pgfpathlineto{\pgfpoint{31.804660\du}{13.942851\du}}
\pgfpathlineto{\pgfpoint{31.790422\du}{13.938470\du}}
\pgfpathlineto{\pgfpoint{31.776183\du}{13.932993\du}}
\pgfpathlineto{\pgfpoint{31.761944\du}{13.928247\du}}
\pgfpathlineto{\pgfpoint{31.747705\du}{13.923866\du}}
\pgfpathlineto{\pgfpoint{31.733832\du}{13.918389\du}}
\pgfpathlineto{\pgfpoint{31.719958\du}{13.913643\du}}
\pgfpathlineto{\pgfpoint{31.706449\du}{13.908532\du}}
\pgfpathlineto{\pgfpoint{31.692941\du}{13.903055\du}}
\pgfpathlineto{\pgfpoint{31.680162\du}{13.898674\du}}
\pgfpathlineto{\pgfpoint{31.666654\du}{13.892832\du}}
\pgfpathlineto{\pgfpoint{31.653875\du}{13.887356\du}}
\pgfpathlineto{\pgfpoint{31.641462\du}{13.882245\du}}
\pgfpathlineto{\pgfpoint{31.628319\du}{13.876768\du}}
\pgfpathlineto{\pgfpoint{31.616270\du}{13.871657\du}}
\pgfpathlineto{\pgfpoint{31.604952\du}{13.865815\du}}
\pgfpathlineto{\pgfpoint{31.592174\du}{13.860339\du}}
\pgfpathlineto{\pgfpoint{31.580491\du}{13.855227\du}}
\pgfpathlineto{\pgfpoint{31.568808\du}{13.849386\du}}
\pgfpathlineto{\pgfpoint{31.557855\du}{13.843544\du}}
\pgfpathlineto{\pgfpoint{31.546172\du}{13.837703\du}}
\pgfpathlineto{\pgfpoint{31.536314\du}{13.831861\du}}
\pgfpathlineto{\pgfpoint{31.524996\du}{13.826750\du}}
\pgfpathlineto{\pgfpoint{31.514408\du}{13.820178\du}}
\pgfpathlineto{\pgfpoint{31.504551\du}{13.814337\du}}
\pgfpathlineto{\pgfpoint{31.493963\du}{13.808495\du}}
\pgfpathlineto{\pgfpoint{31.484105\du}{13.802654\du}}
\pgfpathlineto{\pgfpoint{31.474613\du}{13.796812\du}}
\pgfpathlineto{\pgfpoint{31.464755\du}{13.790240\du}}
\pgfpathlineto{\pgfpoint{31.455628\du}{13.784399\du}}
\pgfpathlineto{\pgfpoint{31.447231\du}{13.777827\du}}
\pgfpathlineto{\pgfpoint{31.438468\du}{13.771985\du}}
\pgfpathlineto{\pgfpoint{31.429341\du}{13.765779\du}}
\pgfpathlineto{\pgfpoint{31.420579\du}{13.759207\du}}
\pgfpathlineto{\pgfpoint{31.412181\du}{13.753000\du}}
\pgfpathlineto{\pgfpoint{31.404879\du}{13.746429\du}}
\pgfpathlineto{\pgfpoint{31.396847\du}{13.740587\du}}
\pgfpathlineto{\pgfpoint{31.389180\du}{13.734015\du}}
\pgfpathlineto{\pgfpoint{31.382243\du}{13.727809\du}}
\pgfpathlineto{\pgfpoint{31.374941\du}{13.721237\du}}
\pgfpathlineto{\pgfpoint{31.368005\du}{13.714300\du}}
\pgfpathlineto{\pgfpoint{31.361798\du}{13.707728\du}}
\pgfpathlineto{\pgfpoint{31.355226\du}{13.701522\du}}
\pgfpathlineto{\pgfpoint{31.349385\du}{13.694950\du}}
\pgfpathlineto{\pgfpoint{31.343178\du}{13.688013\du}}
\pgfpathlineto{\pgfpoint{31.338067\du}{13.681441\du}}
\pgfpathlineto{\pgfpoint{31.331860\du}{13.674870\du}}
\pgfpathlineto{\pgfpoint{31.327479\du}{13.667933\du}}
\pgfpathlineto{\pgfpoint{31.322002\du}{13.661726\du}}
\pgfpathlineto{\pgfpoint{31.317621\du}{13.654424\du}}
\pgfpathlineto{\pgfpoint{31.313240\du}{13.648218\du}}
\pgfpathlineto{\pgfpoint{31.308859\du}{13.640916\du}}
\pgfpathlineto{\pgfpoint{31.305573\du}{13.634709\du}}
\pgfpathlineto{\pgfpoint{31.301557\du}{13.627772\du}}
\pgfpathlineto{\pgfpoint{31.297541\du}{13.620470\du}}
\pgfpathlineto{\pgfpoint{31.294620\du}{13.613533\du}}
\pgfpathlineto{\pgfpoint{31.292064\du}{13.607327\du}}
\pgfpathlineto{\pgfpoint{31.288779\du}{13.600025\du}}
\pgfpathlineto{\pgfpoint{31.286588\du}{13.593088\du}}
\pgfpathlineto{\pgfpoint{31.284397\du}{13.586151\du}}
\pgfpathlineto{\pgfpoint{31.282937\du}{13.579214\du}}
\pgfpathlineto{\pgfpoint{31.280746\du}{13.571912\du}}
\pgfpathlineto{\pgfpoint{31.279651\du}{13.565706\du}}
\pgfpathlineto{\pgfpoint{31.278556\du}{13.558404\du}}
\pgfpathlineto{\pgfpoint{31.277095\du}{13.551467\du}}
\pgfpathlineto{\pgfpoint{31.276730\du}{13.544530\du}}
\pgfpathlineto{\pgfpoint{31.276730\du}{13.536863\du}}
\pgfpathlineto{\pgfpoint{31.276365\du}{13.529926\du}}
\pgfpathlineto{\pgfpoint{31.276365\du}{13.529926\du}}
\pgfpathlineto{\pgfpoint{31.276365\du}{13.529926\du}}
\pgfpathlineto{\pgfpoint{31.276365\du}{13.528831\du}}
\pgfpathlineto{\pgfpoint{31.276365\du}{13.527736\du}}
\pgfpathlineto{\pgfpoint{31.276000\du}{13.526275\du}}
\pgfpathlineto{\pgfpoint{31.276000\du}{13.525910\du}}
\pgfpathlineto{\pgfpoint{31.274905\du}{13.524815\du}}
\pgfpathlineto{\pgfpoint{31.274540\du}{13.523355\du}}
\pgfpathlineto{\pgfpoint{31.274175\du}{13.522989\du}}
\pgfpathlineto{\pgfpoint{31.273079\du}{13.522259\du}}
\pgfpathlineto{\pgfpoint{31.271619\du}{13.521164\du}}
\pgfpathlineto{\pgfpoint{31.269794\du}{13.520434\du}}
\pgfpathlineto{\pgfpoint{31.267968\du}{13.520069\du}}
\pgfpathlineto{\pgfpoint{31.266143\du}{13.520069\du}}
\pgfpathlineto{\pgfpoint{31.263952\du}{13.520069\du}}
\pgfpathlineto{\pgfpoint{31.262492\du}{13.520434\du}}
\pgfpathlineto{\pgfpoint{31.260301\du}{13.521164\du}}
\pgfpathlineto{\pgfpoint{31.258476\du}{13.522259\du}}
\pgfpathlineto{\pgfpoint{31.258110\du}{13.522989\du}}
\pgfpathlineto{\pgfpoint{31.257745\du}{13.523355\du}}
\pgfpathlineto{\pgfpoint{31.257015\du}{13.524815\du}}
\pgfpathlineto{\pgfpoint{31.256285\du}{13.525910\du}}
\pgfpathlineto{\pgfpoint{31.256285\du}{13.526275\du}}
\pgfpathlineto{\pgfpoint{31.255920\du}{13.527736\du}}
\pgfpathlineto{\pgfpoint{31.255920\du}{13.528831\du}}
\pgfpathlineto{\pgfpoint{31.255920\du}{13.529926\du}}
\pgfusepath{fill}
\pgfsetbuttcap
\pgfsetmiterjoin
\pgfsetdash{}{0pt}
\definecolor{dialinecolor}{rgb}{0.678431, 0.839216, 0.905882}
\pgfsetfillcolor{dialinecolor}
\pgfpathmoveto{\pgfpoint{32.946683\du}{12.942486\du}}
\pgfpathlineto{\pgfpoint{32.946683\du}{12.942486\du}}
\pgfpathlineto{\pgfpoint{32.902871\du}{12.942486\du}}
\pgfpathlineto{\pgfpoint{32.860155\du}{12.943216\du}}
\pgfpathlineto{\pgfpoint{32.816708\du}{12.944311\du}}
\pgfpathlineto{\pgfpoint{32.774722\du}{12.945406\du}}
\pgfpathlineto{\pgfpoint{32.732371\du}{12.947232\du}}
\pgfpathlineto{\pgfpoint{32.690750\du}{12.949422\du}}
\pgfpathlineto{\pgfpoint{32.648764\du}{12.951978\du}}
\pgfpathlineto{\pgfpoint{32.607508\du}{12.954169\du}}
\pgfpathlineto{\pgfpoint{32.566617\du}{12.957089\du}}
\pgfpathlineto{\pgfpoint{32.525726\du}{12.960740\du}}
\pgfpathlineto{\pgfpoint{32.485931\du}{12.964391\du}}
\pgfpathlineto{\pgfpoint{32.445770\du}{12.968407\du}}
\pgfpathlineto{\pgfpoint{32.406705\du}{12.972789\du}}
\pgfpathlineto{\pgfpoint{32.368005\du}{12.977535\du}}
\pgfpathlineto{\pgfpoint{32.329304\du}{12.982646\du}}
\pgfpathlineto{\pgfpoint{32.291334\du}{12.988123\du}}
\pgfpathlineto{\pgfpoint{32.253729\du}{12.993234\du}}
\pgfpathlineto{\pgfpoint{32.216489\du}{12.999806\du}}
\pgfpathlineto{\pgfpoint{32.179980\du}{13.005647\du}}
\pgfpathlineto{\pgfpoint{32.143835\du}{13.012219\du}}
\pgfpathlineto{\pgfpoint{32.108056\du}{13.019156\du}}
\pgfpathlineto{\pgfpoint{32.073372\du}{13.026093\du}}
\pgfpathlineto{\pgfpoint{32.039052\du}{13.033760\du}}
\pgfpathlineto{\pgfpoint{32.004368\du}{13.041427\du}}
\pgfpathlineto{\pgfpoint{31.971875\du}{13.049459\du}}
\pgfpathlineto{\pgfpoint{31.938651\du}{13.057856\du}}
\pgfpathlineto{\pgfpoint{31.906887\du}{13.065888\du}}
\pgfpathlineto{\pgfpoint{31.875124\du}{13.074651\du}}
\pgfpathlineto{\pgfpoint{31.844456\du}{13.084143\du}}
\pgfpathlineto{\pgfpoint{31.813423\du}{13.092905\du}}
\pgfpathlineto{\pgfpoint{31.798819\du}{13.097652\du}}
\pgfpathlineto{\pgfpoint{31.784945\du}{13.102763\du}}
\pgfpathlineto{\pgfpoint{31.769246\du}{13.107509\du}}
\pgfpathlineto{\pgfpoint{31.755372\du}{13.112256\du}}
\pgfpathlineto{\pgfpoint{31.740768\du}{13.117367\du}}
\pgfpathlineto{\pgfpoint{31.726895\du}{13.122113\du}}
\pgfpathlineto{\pgfpoint{31.712656\du}{13.127225\du}}
\pgfpathlineto{\pgfpoint{31.699512\du}{13.131971\du}}
\pgfpathlineto{\pgfpoint{31.685274\du}{13.137447\du}}
\pgfpathlineto{\pgfpoint{31.672130\du}{13.142559\du}}
\pgfpathlineto{\pgfpoint{31.659717\du}{13.147670\du}}
\pgfpathlineto{\pgfpoint{31.645843\du}{13.153512\du}}
\pgfpathlineto{\pgfpoint{31.633065\du}{13.158988\du}}
\pgfpathlineto{\pgfpoint{31.621017\du}{13.164099\du}}
\pgfpathlineto{\pgfpoint{31.607873\du}{13.169941\du}}
\pgfpathlineto{\pgfpoint{31.595825\du}{13.175417\du}}
\pgfpathlineto{\pgfpoint{31.583412\du}{13.181259\du}}
\pgfpathlineto{\pgfpoint{31.571729\du}{13.186370\du}}
\pgfpathlineto{\pgfpoint{31.560411\du}{13.192212\du}}
\pgfpathlineto{\pgfpoint{31.548727\du}{13.198053\du}}
\pgfpathlineto{\pgfpoint{31.537409\du}{13.203895\du}}
\pgfpathlineto{\pgfpoint{31.525726\du}{13.209736\du}}
\pgfpathlineto{\pgfpoint{31.514773\du}{13.215578\du}}
\pgfpathlineto{\pgfpoint{31.504551\du}{13.222150\du}}
\pgfpathlineto{\pgfpoint{31.493963\du}{13.227991\du}}
\pgfpathlineto{\pgfpoint{31.483740\du}{13.233833\du}}
\pgfpathlineto{\pgfpoint{31.473518\du}{13.240405\du}}
\pgfpathlineto{\pgfpoint{31.463660\du}{13.246246\du}}
\pgfpathlineto{\pgfpoint{31.454167\du}{13.252453\du}}
\pgfpathlineto{\pgfpoint{31.444675\du}{13.259024\du}}
\pgfpathlineto{\pgfpoint{31.435547\du}{13.265596\du}}
\pgfpathlineto{\pgfpoint{31.425690\du}{13.271438\du}}
\pgfpathlineto{\pgfpoint{31.417293\du}{13.277644\du}}
\pgfpathlineto{\pgfpoint{31.408530\du}{13.284216\du}}
\pgfpathlineto{\pgfpoint{31.400133\du}{13.290423\du}}
\pgfpathlineto{\pgfpoint{31.391371\du}{13.297725\du}}
\pgfpathlineto{\pgfpoint{31.383704\du}{13.303931\du}}
\pgfpathlineto{\pgfpoint{31.376767\du}{13.310503\du}}
\pgfpathlineto{\pgfpoint{31.368005\du}{13.317440\du}}
\pgfpathlineto{\pgfpoint{31.361068\du}{13.324012\du}}
\pgfpathlineto{\pgfpoint{31.354131\du}{13.330949\du}}
\pgfpathlineto{\pgfpoint{31.347194\du}{13.337885\du}}
\pgfpathlineto{\pgfpoint{31.340622\du}{13.344457\du}}
\pgfpathlineto{\pgfpoint{31.333685\du}{13.351394\du}}
\pgfpathlineto{\pgfpoint{31.327844\du}{13.358331\du}}
\pgfpathlineto{\pgfpoint{31.322002\du}{13.365633\du}}
\pgfpathlineto{\pgfpoint{31.316161\du}{13.372570\du}}
\pgfpathlineto{\pgfpoint{31.310684\du}{13.379506\du}}
\pgfpathlineto{\pgfpoint{31.305938\du}{13.386443\du}}
\pgfpathlineto{\pgfpoint{31.300462\du}{13.394110\du}}
\pgfpathlineto{\pgfpoint{31.296446\du}{13.401047\du}}
\pgfpathlineto{\pgfpoint{31.291334\du}{13.408349\du}}
\pgfpathlineto{\pgfpoint{31.287683\du}{13.415651\du}}
\pgfpathlineto{\pgfpoint{31.283302\du}{13.422953\du}}
\pgfpathlineto{\pgfpoint{31.279651\du}{13.430620\du}}
\pgfpathlineto{\pgfpoint{31.276365\du}{13.437922\du}}
\pgfpathlineto{\pgfpoint{31.272714\du}{13.445589\du}}
\pgfpathlineto{\pgfpoint{31.269794\du}{13.453256\du}}
\pgfpathlineto{\pgfpoint{31.267238\du}{13.460193\du}}
\pgfpathlineto{\pgfpoint{31.265047\du}{13.467860\du}}
\pgfpathlineto{\pgfpoint{31.262857\du}{13.475527\du}}
\pgfpathlineto{\pgfpoint{31.261031\du}{13.483559\du}}
\pgfpathlineto{\pgfpoint{31.259206\du}{13.491226\du}}
\pgfpathlineto{\pgfpoint{31.258110\du}{13.498893\du}}
\pgfpathlineto{\pgfpoint{31.257015\du}{13.506560\du}}
\pgfpathlineto{\pgfpoint{31.256285\du}{13.514592\du}}
\pgfpathlineto{\pgfpoint{31.255920\du}{13.522259\du}}
\pgfpathlineto{\pgfpoint{31.255920\du}{13.529926\du}}
\pgfpathlineto{\pgfpoint{31.276365\du}{13.529926\du}}
\pgfpathlineto{\pgfpoint{31.276730\du}{13.522989\du}}
\pgfpathlineto{\pgfpoint{31.276730\du}{13.516053\du}}
\pgfpathlineto{\pgfpoint{31.277095\du}{13.508751\du}}
\pgfpathlineto{\pgfpoint{31.278556\du}{13.501814\du}}
\pgfpathlineto{\pgfpoint{31.279651\du}{13.494877\du}}
\pgfpathlineto{\pgfpoint{31.280746\du}{13.487940\du}}
\pgfpathlineto{\pgfpoint{31.282937\du}{13.480638\du}}
\pgfpathlineto{\pgfpoint{31.284397\du}{13.474432\du}}
\pgfpathlineto{\pgfpoint{31.286588\du}{13.467130\du}}
\pgfpathlineto{\pgfpoint{31.288779\du}{13.460193\du}}
\pgfpathlineto{\pgfpoint{31.292064\du}{13.453256\du}}
\pgfpathlineto{\pgfpoint{31.294620\du}{13.446319\du}}
\pgfpathlineto{\pgfpoint{31.297541\du}{13.439382\du}}
\pgfpathlineto{\pgfpoint{31.301557\du}{13.432811\du}}
\pgfpathlineto{\pgfpoint{31.305573\du}{13.425874\du}}
\pgfpathlineto{\pgfpoint{31.308859\du}{13.419302\du}}
\pgfpathlineto{\pgfpoint{31.312875\du}{13.412365\du}}
\pgfpathlineto{\pgfpoint{31.317621\du}{13.405428\du}}
\pgfpathlineto{\pgfpoint{31.322002\du}{13.398857\du}}
\pgfpathlineto{\pgfpoint{31.327479\du}{13.392285\du}}
\pgfpathlineto{\pgfpoint{31.331860\du}{13.385348\du}}
\pgfpathlineto{\pgfpoint{31.338067\du}{13.378776\du}}
\pgfpathlineto{\pgfpoint{31.343178\du}{13.371839\du}}
\pgfpathlineto{\pgfpoint{31.349385\du}{13.365633\du}}
\pgfpathlineto{\pgfpoint{31.355226\du}{13.359061\du}}
\pgfpathlineto{\pgfpoint{31.361798\du}{13.352124\du}}
\pgfpathlineto{\pgfpoint{31.368005\du}{13.345552\du}}
\pgfpathlineto{\pgfpoint{31.374941\du}{13.339346\du}}
\pgfpathlineto{\pgfpoint{31.382243\du}{13.332774\du}}
\pgfpathlineto{\pgfpoint{31.389180\du}{13.326202\du}}
\pgfpathlineto{\pgfpoint{31.396847\du}{13.319996\du}}
\pgfpathlineto{\pgfpoint{31.404879\du}{13.313424\du}}
\pgfpathlineto{\pgfpoint{31.412181\du}{13.306852\du}}
\pgfpathlineto{\pgfpoint{31.420579\du}{13.300645\du}}
\pgfpathlineto{\pgfpoint{31.429341\du}{13.294804\du}}
\pgfpathlineto{\pgfpoint{31.438468\du}{13.287867\du}}
\pgfpathlineto{\pgfpoint{31.447231\du}{13.282391\du}}
\pgfpathlineto{\pgfpoint{31.455628\du}{13.275819\du}}
\pgfpathlineto{\pgfpoint{31.464755\du}{13.269612\du}}
\pgfpathlineto{\pgfpoint{31.474613\du}{13.263771\du}}
\pgfpathlineto{\pgfpoint{31.484105\du}{13.257929\du}}
\pgfpathlineto{\pgfpoint{31.493963\du}{13.251357\du}}
\pgfpathlineto{\pgfpoint{31.504551\du}{13.245516\du}}
\pgfpathlineto{\pgfpoint{31.514408\du}{13.239674\du}}
\pgfpathlineto{\pgfpoint{31.524996\du}{13.233833\du}}
\pgfpathlineto{\pgfpoint{31.536314\du}{13.227991\du}}
\pgfpathlineto{\pgfpoint{31.546172\du}{13.222150\du}}
\pgfpathlineto{\pgfpoint{31.557855\du}{13.216308\du}}
\pgfpathlineto{\pgfpoint{31.568808\du}{13.211197\du}}
\pgfpathlineto{\pgfpoint{31.580491\du}{13.204990\du}}
\pgfpathlineto{\pgfpoint{31.592174\du}{13.199149\du}}
\pgfpathlineto{\pgfpoint{31.604952\du}{13.194037\du}}
\pgfpathlineto{\pgfpoint{31.616270\du}{13.188926\du}}
\pgfpathlineto{\pgfpoint{31.628319\du}{13.183084\du}}
\pgfpathlineto{\pgfpoint{31.641462\du}{13.177608\du}}
\pgfpathlineto{\pgfpoint{31.653875\du}{13.172497\du}}
\pgfpathlineto{\pgfpoint{31.666654\du}{13.167020\du}}
\pgfpathlineto{\pgfpoint{31.680162\du}{13.161909\du}}
\pgfpathlineto{\pgfpoint{31.692941\du}{13.156432\du}}
\pgfpathlineto{\pgfpoint{31.706449\du}{13.150956\du}}
\pgfpathlineto{\pgfpoint{31.719958\du}{13.146575\du}}
\pgfpathlineto{\pgfpoint{31.733832\du}{13.141463\du}}
\pgfpathlineto{\pgfpoint{31.747705\du}{13.136717\du}}
\pgfpathlineto{\pgfpoint{31.761944\du}{13.131606\du}}
\pgfpathlineto{\pgfpoint{31.776183\du}{13.126859\du}}
\pgfpathlineto{\pgfpoint{31.790422\du}{13.122113\du}}
\pgfpathlineto{\pgfpoint{31.804660\du}{13.117367\du}}
\pgfpathlineto{\pgfpoint{31.819994\du}{13.112621\du}}
\pgfpathlineto{\pgfpoint{31.849567\du}{13.103493\du}}
\pgfpathlineto{\pgfpoint{31.880966\du}{13.094731\du}}
\pgfpathlineto{\pgfpoint{31.911999\du}{13.085969\du}}
\pgfpathlineto{\pgfpoint{31.944492\du}{13.077571\du}}
\pgfpathlineto{\pgfpoint{31.975891\du}{13.069539\du}}
\pgfpathlineto{\pgfpoint{32.009114\du}{13.061142\du}}
\pgfpathlineto{\pgfpoint{32.043068\du}{13.053475\du}}
\pgfpathlineto{\pgfpoint{32.077388\du}{13.046538\du}}
\pgfpathlineto{\pgfpoint{32.112072\du}{13.039601\du}}
\pgfpathlineto{\pgfpoint{32.148216\du}{13.032299\du}}
\pgfpathlineto{\pgfpoint{32.183266\du}{13.026093\du}}
\pgfpathlineto{\pgfpoint{32.220505\du}{13.019156\du}}
\pgfpathlineto{\pgfpoint{32.257015\du}{13.013679\du}}
\pgfpathlineto{\pgfpoint{32.294255\du}{13.007838\du}}
\pgfpathlineto{\pgfpoint{32.331860\du}{13.002727\du}}
\pgfpathlineto{\pgfpoint{32.370195\du}{12.997980\du}}
\pgfpathlineto{\pgfpoint{32.409261\du}{12.993234\du}}
\pgfpathlineto{\pgfpoint{32.448691\du}{12.988853\du}}
\pgfpathlineto{\pgfpoint{32.488121\du}{12.985202\du}}
\pgfpathlineto{\pgfpoint{32.527917\du}{12.981186\du}}
\pgfpathlineto{\pgfpoint{32.568078\du}{12.978265\du}}
\pgfpathlineto{\pgfpoint{32.608968\du}{12.974614\du}}
\pgfpathlineto{\pgfpoint{32.650224\du}{12.971693\du}}
\pgfpathlineto{\pgfpoint{32.691115\du}{12.969503\du}}
\pgfpathlineto{\pgfpoint{32.733101\du}{12.967677\du}}
\pgfpathlineto{\pgfpoint{32.775453\du}{12.965852\du}}
\pgfpathlineto{\pgfpoint{32.818169\du}{12.964391\du}}
\pgfpathlineto{\pgfpoint{32.860520\du}{12.964026\du}}
\pgfpathlineto{\pgfpoint{32.903967\du}{12.963661\du}}
\pgfpathlineto{\pgfpoint{32.946683\du}{12.962931\du}}
\pgfpathlineto{\pgfpoint{32.946683\du}{12.962931\du}}
\pgfpathlineto{\pgfpoint{32.946683\du}{12.962931\du}}
\pgfpathlineto{\pgfpoint{32.947778\du}{12.962931\du}}
\pgfpathlineto{\pgfpoint{32.948874\du}{12.962931\du}}
\pgfpathlineto{\pgfpoint{32.950334\du}{12.962201\du}}
\pgfpathlineto{\pgfpoint{32.951429\du}{12.962201\du}}
\pgfpathlineto{\pgfpoint{32.951794\du}{12.961836\du}}
\pgfpathlineto{\pgfpoint{32.952890\du}{12.961106\du}}
\pgfpathlineto{\pgfpoint{32.953620\du}{12.960740\du}}
\pgfpathlineto{\pgfpoint{32.954715\du}{12.960010\du}}
\pgfpathlineto{\pgfpoint{32.955810\du}{12.958185\du}}
\pgfpathlineto{\pgfpoint{32.956541\du}{12.956359\du}}
\pgfpathlineto{\pgfpoint{32.956541\du}{12.954899\du}}
\pgfpathlineto{\pgfpoint{32.957271\du}{12.953073\du}}
\pgfpathlineto{\pgfpoint{32.956541\du}{12.951248\du}}
\pgfpathlineto{\pgfpoint{32.956541\du}{12.949057\du}}
\pgfpathlineto{\pgfpoint{32.955810\du}{12.947232\du}}
\pgfpathlineto{\pgfpoint{32.954715\du}{12.945406\du}}
\pgfpathlineto{\pgfpoint{32.953620\du}{12.944676\du}}
\pgfpathlineto{\pgfpoint{32.952890\du}{12.944311\du}}
\pgfpathlineto{\pgfpoint{32.951794\du}{12.943581\du}}
\pgfpathlineto{\pgfpoint{32.951429\du}{12.943216\du}}
\pgfpathlineto{\pgfpoint{32.950334\du}{12.943216\du}}
\pgfpathlineto{\pgfpoint{32.948874\du}{12.942486\du}}
\pgfpathlineto{\pgfpoint{32.947778\du}{12.942486\du}}
\pgfpathlineto{\pgfpoint{32.946683\du}{12.942486\du}}
\pgfusepath{fill}
\pgfsetbuttcap
\pgfsetmiterjoin
\pgfsetdash{}{0pt}
\definecolor{dialinecolor}{rgb}{0.678431, 0.839216, 0.905882}
\pgfsetfillcolor{dialinecolor}
\pgfpathmoveto{\pgfpoint{34.637081\du}{13.529926\du}}
\pgfpathlineto{\pgfpoint{34.637081\du}{13.522259\du}}
\pgfpathlineto{\pgfpoint{34.636351\du}{13.514592\du}}
\pgfpathlineto{\pgfpoint{34.635621\du}{13.506560\du}}
\pgfpathlineto{\pgfpoint{34.634890\du}{13.498893\du}}
\pgfpathlineto{\pgfpoint{34.633795\du}{13.491226\du}}
\pgfpathlineto{\pgfpoint{34.631604\du}{13.483559\du}}
\pgfpathlineto{\pgfpoint{34.630509\du}{13.475527\du}}
\pgfpathlineto{\pgfpoint{34.627588\du}{13.467860\du}}
\pgfpathlineto{\pgfpoint{34.625763\du}{13.460193\du}}
\pgfpathlineto{\pgfpoint{34.622842\du}{13.453256\du}}
\pgfpathlineto{\pgfpoint{34.620286\du}{13.445589\du}}
\pgfpathlineto{\pgfpoint{34.617001\du}{13.437922\du}}
\pgfpathlineto{\pgfpoint{34.612985\du}{13.430620\du}}
\pgfpathlineto{\pgfpoint{34.609699\du}{13.422953\du}}
\pgfpathlineto{\pgfpoint{34.605683\du}{13.415651\du}}
\pgfpathlineto{\pgfpoint{34.601667\du}{13.408349\du}}
\pgfpathlineto{\pgfpoint{34.597285\du}{13.401047\du}}
\pgfpathlineto{\pgfpoint{34.592539\du}{13.394110\du}}
\pgfpathlineto{\pgfpoint{34.587063\du}{13.386443\du}}
\pgfpathlineto{\pgfpoint{34.582316\du}{13.379506\du}}
\pgfpathlineto{\pgfpoint{34.576475\du}{13.372570\du}}
\pgfpathlineto{\pgfpoint{34.570998\du}{13.365633\du}}
\pgfpathlineto{\pgfpoint{34.565157\du}{13.358331\du}}
\pgfpathlineto{\pgfpoint{34.558950\du}{13.351394\du}}
\pgfpathlineto{\pgfpoint{34.552378\du}{13.344457\du}}
\pgfpathlineto{\pgfpoint{34.545442\du}{13.337885\du}}
\pgfpathlineto{\pgfpoint{34.538870\du}{13.330949\du}}
\pgfpathlineto{\pgfpoint{34.531568\du}{13.324012\du}}
\pgfpathlineto{\pgfpoint{34.524996\du}{13.317440\du}}
\pgfpathlineto{\pgfpoint{34.516964\du}{13.310503\du}}
\pgfpathlineto{\pgfpoint{34.508932\du}{13.303931\du}}
\pgfpathlineto{\pgfpoint{34.501630\du}{13.297725\du}}
\pgfpathlineto{\pgfpoint{34.492868\du}{13.290423\du}}
\pgfpathlineto{\pgfpoint{34.484836\du}{13.284216\du}}
\pgfpathlineto{\pgfpoint{34.475708\du}{13.277644\du}}
\pgfpathlineto{\pgfpoint{34.466946\du}{13.271438\du}}
\pgfpathlineto{\pgfpoint{34.457453\du}{13.265596\du}}
\pgfpathlineto{\pgfpoint{34.448326\du}{13.259024\du}}
\pgfpathlineto{\pgfpoint{34.439198\du}{13.252453\du}}
\pgfpathlineto{\pgfpoint{34.429341\du}{13.246246\du}}
\pgfpathlineto{\pgfpoint{34.419483\du}{13.240405\du}}
\pgfpathlineto{\pgfpoint{34.409261\du}{13.233833\du}}
\pgfpathlineto{\pgfpoint{34.398673\du}{13.227991\du}}
\pgfpathlineto{\pgfpoint{34.388815\du}{13.222150\du}}
\pgfpathlineto{\pgfpoint{34.378227\du}{13.215578\du}}
\pgfpathlineto{\pgfpoint{34.366909\du}{13.209736\du}}
\pgfpathlineto{\pgfpoint{34.355591\du}{13.203895\du}}
\pgfpathlineto{\pgfpoint{34.343908\du}{13.198053\du}}
\pgfpathlineto{\pgfpoint{34.332955\du}{13.192212\du}}
\pgfpathlineto{\pgfpoint{34.320907\du}{13.186370\du}}
\pgfpathlineto{\pgfpoint{34.309589\du}{13.181259\du}}
\pgfpathlineto{\pgfpoint{34.297176\du}{13.175417\du}}
\pgfpathlineto{\pgfpoint{34.284763\du}{13.169941\du}}
\pgfpathlineto{\pgfpoint{34.271984\du}{13.164099\du}}
\pgfpathlineto{\pgfpoint{34.259936\du}{13.158988\du}}
\pgfpathlineto{\pgfpoint{34.247158\du}{13.153512\du}}
\pgfpathlineto{\pgfpoint{34.233649\du}{13.147670\du}}
\pgfpathlineto{\pgfpoint{34.220505\du}{13.142559\du}}
\pgfpathlineto{\pgfpoint{34.207362\du}{13.137447\du}}
\pgfpathlineto{\pgfpoint{34.193123\du}{13.131971\du}}
\pgfpathlineto{\pgfpoint{34.179980\du}{13.127225\du}}
\pgfpathlineto{\pgfpoint{34.165741\du}{13.122113\du}}
\pgfpathlineto{\pgfpoint{34.151867\du}{13.117367\du}}
\pgfpathlineto{\pgfpoint{34.137629\du}{13.112256\du}}
\pgfpathlineto{\pgfpoint{34.123755\du}{13.107509\du}}
\pgfpathlineto{\pgfpoint{34.108786\du}{13.102763\du}}
\pgfpathlineto{\pgfpoint{34.093817\du}{13.097652\du}}
\pgfpathlineto{\pgfpoint{34.079578\du}{13.092905\du}}
\pgfpathlineto{\pgfpoint{34.048910\du}{13.084143\du}}
\pgfpathlineto{\pgfpoint{34.018242\du}{13.074651\du}}
\pgfpathlineto{\pgfpoint{33.986844\du}{13.065888\du}}
\pgfpathlineto{\pgfpoint{33.955080\du}{13.057856\du}}
\pgfpathlineto{\pgfpoint{33.922221\du}{13.049459\du}}
\pgfpathlineto{\pgfpoint{33.888633\du}{13.041427\du}}
\pgfpathlineto{\pgfpoint{33.854679\du}{13.033760\du}}
\pgfpathlineto{\pgfpoint{33.819994\du}{13.026093\du}}
\pgfpathlineto{\pgfpoint{33.785310\du}{13.019156\du}}
\pgfpathlineto{\pgfpoint{33.749896\du}{13.012219\du}}
\pgfpathlineto{\pgfpoint{33.713386\du}{13.005647\du}}
\pgfpathlineto{\pgfpoint{33.676876\du}{12.999806\du}}
\pgfpathlineto{\pgfpoint{33.640002\du}{12.993234\du}}
\pgfpathlineto{\pgfpoint{33.602762\du}{12.988123\du}}
\pgfpathlineto{\pgfpoint{33.563696\du}{12.982646\du}}
\pgfpathlineto{\pgfpoint{33.525726\du}{12.977535\du}}
\pgfpathlineto{\pgfpoint{33.486296\du}{12.972789\du}}
\pgfpathlineto{\pgfpoint{33.447596\du}{12.968407\du}}
\pgfpathlineto{\pgfpoint{33.407435\du}{12.964391\du}}
\pgfpathlineto{\pgfpoint{33.367639\du}{12.960740\du}}
\pgfpathlineto{\pgfpoint{33.326749\du}{12.957089\du}}
\pgfpathlineto{\pgfpoint{33.286223\du}{12.954169\du}}
\pgfpathlineto{\pgfpoint{33.244602\du}{12.951978\du}}
\pgfpathlineto{\pgfpoint{33.202981\du}{12.949422\du}}
\pgfpathlineto{\pgfpoint{33.161360\du}{12.947232\du}}
\pgfpathlineto{\pgfpoint{33.118644\du}{12.945406\du}}
\pgfpathlineto{\pgfpoint{33.076657\du}{12.944311\du}}
\pgfpathlineto{\pgfpoint{33.033576\du}{12.943216\du}}
\pgfpathlineto{\pgfpoint{32.990129\du}{12.942486\du}}
\pgfpathlineto{\pgfpoint{32.946683\du}{12.942486\du}}
\pgfpathlineto{\pgfpoint{32.946683\du}{12.962931\du}}
\pgfpathlineto{\pgfpoint{32.989764\du}{12.963661\du}}
\pgfpathlineto{\pgfpoint{33.033211\du}{12.964026\du}}
\pgfpathlineto{\pgfpoint{33.075197\du}{12.964391\du}}
\pgfpathlineto{\pgfpoint{33.117913\du}{12.965852\du}}
\pgfpathlineto{\pgfpoint{33.160630\du}{12.967677\du}}
\pgfpathlineto{\pgfpoint{33.202251\du}{12.969503\du}}
\pgfpathlineto{\pgfpoint{33.243507\du}{12.971693\du}}
\pgfpathlineto{\pgfpoint{33.284397\du}{12.974614\du}}
\pgfpathlineto{\pgfpoint{33.325653\du}{12.978265\du}}
\pgfpathlineto{\pgfpoint{33.365814\du}{12.981186\du}}
\pgfpathlineto{\pgfpoint{33.405975\du}{12.985202\du}}
\pgfpathlineto{\pgfpoint{33.445040\du}{12.988853\du}}
\pgfpathlineto{\pgfpoint{33.484470\du}{12.993234\du}}
\pgfpathlineto{\pgfpoint{33.523171\du}{12.997980\du}}
\pgfpathlineto{\pgfpoint{33.561871\du}{13.002727\du}}
\pgfpathlineto{\pgfpoint{33.599476\du}{13.007838\du}}
\pgfpathlineto{\pgfpoint{33.636716\du}{13.013679\du}}
\pgfpathlineto{\pgfpoint{33.673225\du}{13.019156\du}}
\pgfpathlineto{\pgfpoint{33.710100\du}{13.026093\du}}
\pgfpathlineto{\pgfpoint{33.745515\du}{13.032299\du}}
\pgfpathlineto{\pgfpoint{33.781294\du}{13.039601\du}}
\pgfpathlineto{\pgfpoint{33.815978\du}{13.046538\du}}
\pgfpathlineto{\pgfpoint{33.849932\du}{13.053475\du}}
\pgfpathlineto{\pgfpoint{33.884251\du}{13.061142\du}}
\pgfpathlineto{\pgfpoint{33.917475\du}{13.069539\du}}
\pgfpathlineto{\pgfpoint{33.949604\du}{13.077571\du}}
\pgfpathlineto{\pgfpoint{33.981732\du}{13.085969\du}}
\pgfpathlineto{\pgfpoint{34.013131\du}{13.094731\du}}
\pgfpathlineto{\pgfpoint{34.043799\du}{13.103493\du}}
\pgfpathlineto{\pgfpoint{34.073737\du}{13.112621\du}}
\pgfpathlineto{\pgfpoint{34.087975\du}{13.117367\du}}
\pgfpathlineto{\pgfpoint{34.102579\du}{13.122113\du}}
\pgfpathlineto{\pgfpoint{34.116453\du}{13.126859\du}}
\pgfpathlineto{\pgfpoint{34.131057\du}{13.131606\du}}
\pgfpathlineto{\pgfpoint{34.145296\du}{13.136717\du}}
\pgfpathlineto{\pgfpoint{34.159169\du}{13.141463\du}}
\pgfpathlineto{\pgfpoint{34.173408\du}{13.146575\du}}
\pgfpathlineto{\pgfpoint{34.186551\du}{13.150956\du}}
\pgfpathlineto{\pgfpoint{34.200060\du}{13.156432\du}}
\pgfpathlineto{\pgfpoint{34.213204\du}{13.161909\du}}
\pgfpathlineto{\pgfpoint{34.225982\du}{13.167020\du}}
\pgfpathlineto{\pgfpoint{34.238760\du}{13.172497\du}}
\pgfpathlineto{\pgfpoint{34.251904\du}{13.177608\du}}
\pgfpathlineto{\pgfpoint{34.263952\du}{13.183084\du}}
\pgfpathlineto{\pgfpoint{34.276365\du}{13.188926\du}}
\pgfpathlineto{\pgfpoint{34.288413\du}{13.194037\du}}
\pgfpathlineto{\pgfpoint{34.300827\du}{13.199149\du}}
\pgfpathlineto{\pgfpoint{34.312145\du}{13.204990\du}}
\pgfpathlineto{\pgfpoint{34.324193\du}{13.211197\du}}
\pgfpathlineto{\pgfpoint{34.334781\du}{13.216308\du}}
\pgfpathlineto{\pgfpoint{34.346829\du}{13.222150\du}}
\pgfpathlineto{\pgfpoint{34.357052\du}{13.227991\du}}
\pgfpathlineto{\pgfpoint{34.368005\du}{13.233833\du}}
\pgfpathlineto{\pgfpoint{34.378957\du}{13.239674\du}}
\pgfpathlineto{\pgfpoint{34.388815\du}{13.245516\du}}
\pgfpathlineto{\pgfpoint{34.398673\du}{13.251357\du}}
\pgfpathlineto{\pgfpoint{34.408895\du}{13.257929\du}}
\pgfpathlineto{\pgfpoint{34.418023\du}{13.263771\du}}
\pgfpathlineto{\pgfpoint{34.428246\du}{13.269612\du}}
\pgfpathlineto{\pgfpoint{34.437738\du}{13.275819\du}}
\pgfpathlineto{\pgfpoint{34.446500\du}{13.282391\du}}
\pgfpathlineto{\pgfpoint{34.454533\du}{13.287867\du}}
\pgfpathlineto{\pgfpoint{34.463295\du}{13.294804\du}}
\pgfpathlineto{\pgfpoint{34.471692\du}{13.300645\du}}
\pgfpathlineto{\pgfpoint{34.480454\du}{13.306852\du}}
\pgfpathlineto{\pgfpoint{34.488121\du}{13.313424\du}}
\pgfpathlineto{\pgfpoint{34.496154\du}{13.319996\du}}
\pgfpathlineto{\pgfpoint{34.503455\du}{13.326202\du}}
\pgfpathlineto{\pgfpoint{34.511123\du}{13.332774\du}}
\pgfpathlineto{\pgfpoint{34.517694\du}{13.339346\du}}
\pgfpathlineto{\pgfpoint{34.524996\du}{13.345552\du}}
\pgfpathlineto{\pgfpoint{34.530838\du}{13.352124\du}}
\pgfpathlineto{\pgfpoint{34.537775\du}{13.359061\du}}
\pgfpathlineto{\pgfpoint{34.544346\du}{13.365633\du}}
\pgfpathlineto{\pgfpoint{34.549458\du}{13.371839\du}}
\pgfpathlineto{\pgfpoint{34.554934\du}{13.378776\du}}
\pgfpathlineto{\pgfpoint{34.561141\du}{13.385348\du}}
\pgfpathlineto{\pgfpoint{34.565887\du}{13.392285\du}}
\pgfpathlineto{\pgfpoint{34.570998\du}{13.398857\du}}
\pgfpathlineto{\pgfpoint{34.575745\du}{13.405428\du}}
\pgfpathlineto{\pgfpoint{34.580126\du}{13.412365\du}}
\pgfpathlineto{\pgfpoint{34.584507\du}{13.419302\du}}
\pgfpathlineto{\pgfpoint{34.587428\du}{13.425874\du}}
\pgfpathlineto{\pgfpoint{34.591444\du}{13.432811\du}}
\pgfpathlineto{\pgfpoint{34.594365\du}{13.439382\du}}
\pgfpathlineto{\pgfpoint{34.598381\du}{13.446319\du}}
\pgfpathlineto{\pgfpoint{34.600936\du}{13.453256\du}}
\pgfpathlineto{\pgfpoint{34.603857\du}{13.460193\du}}
\pgfpathlineto{\pgfpoint{34.606413\du}{13.467130\du}}
\pgfpathlineto{\pgfpoint{34.608603\du}{13.474432\du}}
\pgfpathlineto{\pgfpoint{34.610064\du}{13.480638\du}}
\pgfpathlineto{\pgfpoint{34.612254\du}{13.487940\du}}
\pgfpathlineto{\pgfpoint{34.612985\du}{13.494877\du}}
\pgfpathlineto{\pgfpoint{34.614080\du}{13.501814\du}}
\pgfpathlineto{\pgfpoint{34.615905\du}{13.508751\du}}
\pgfpathlineto{\pgfpoint{34.616270\du}{13.516053\du}}
\pgfpathlineto{\pgfpoint{34.616270\du}{13.522989\du}}
\pgfpathlineto{\pgfpoint{34.617001\du}{13.529926\du}}
\pgfpathlineto{\pgfpoint{34.637081\du}{13.529926\du}}
\pgfusepath{fill}
\pgfsetbuttcap
\pgfsetmiterjoin
\pgfsetdash{}{0pt}
\definecolor{dialinecolor}{rgb}{0.074510, 0.082353, 0.086275}
\pgfsetfillcolor{dialinecolor}
\pgfpathmoveto{\pgfpoint{32.989764\du}{13.402507\du}}
\pgfpathlineto{\pgfpoint{33.237665\du}{13.485019\du}}
\pgfpathlineto{\pgfpoint{33.822915\du}{13.250627\du}}
\pgfpathlineto{\pgfpoint{34.095642\du}{13.318170\du}}
\pgfpathlineto{\pgfpoint{33.951794\du}{13.109700\du}}
\pgfpathlineto{\pgfpoint{33.247523\du}{13.109700\du}}
\pgfpathlineto{\pgfpoint{33.541791\du}{13.182354\du}}
\pgfpathlineto{\pgfpoint{32.989764\du}{13.402507\du}}
\pgfusepath{fill}
\pgfsetbuttcap
\pgfsetmiterjoin
\pgfsetdash{}{0pt}
\definecolor{dialinecolor}{rgb}{0.074510, 0.082353, 0.086275}
\pgfsetfillcolor{dialinecolor}
\pgfpathmoveto{\pgfpoint{32.887902\du}{13.640551\du}}
\pgfpathlineto{\pgfpoint{32.640002\du}{13.558404\du}}
\pgfpathlineto{\pgfpoint{32.054752\du}{13.792066\du}}
\pgfpathlineto{\pgfpoint{31.781659\du}{13.725253\du}}
\pgfpathlineto{\pgfpoint{31.925507\du}{13.932993\du}}
\pgfpathlineto{\pgfpoint{32.630874\du}{13.932993\du}}
\pgfpathlineto{\pgfpoint{32.335876\du}{13.861069\du}}
\pgfpathlineto{\pgfpoint{32.887902\du}{13.640551\du}}
\pgfusepath{fill}
\pgfsetbuttcap
\pgfsetmiterjoin
\pgfsetdash{}{0pt}
\definecolor{dialinecolor}{rgb}{0.074510, 0.082353, 0.086275}
\pgfsetfillcolor{dialinecolor}
\pgfpathmoveto{\pgfpoint{31.841900\du}{13.181624\du}}
\pgfpathlineto{\pgfpoint{32.089436\du}{13.099842\du}}
\pgfpathlineto{\pgfpoint{32.674686\du}{13.333139\du}}
\pgfpathlineto{\pgfpoint{32.947778\du}{13.266691\du}}
\pgfpathlineto{\pgfpoint{32.803930\du}{13.474432\du}}
\pgfpathlineto{\pgfpoint{32.098928\du}{13.474432\du}}
\pgfpathlineto{\pgfpoint{32.393926\du}{13.402507\du}}
\pgfpathlineto{\pgfpoint{31.841900\du}{13.181624\du}}
\pgfusepath{fill}
\pgfsetbuttcap
\pgfsetmiterjoin
\pgfsetdash{}{0pt}
\definecolor{dialinecolor}{rgb}{0.074510, 0.082353, 0.086275}
\pgfsetfillcolor{dialinecolor}
\pgfpathmoveto{\pgfpoint{34.060228\du}{13.876768\du}}
\pgfpathlineto{\pgfpoint{33.812692\du}{13.958915\du}}
\pgfpathlineto{\pgfpoint{33.227442\du}{13.725253\du}}
\pgfpathlineto{\pgfpoint{32.953620\du}{13.792066\du}}
\pgfpathlineto{\pgfpoint{33.098198\du}{13.584326\du}}
\pgfpathlineto{\pgfpoint{33.803565\du}{13.584326\du}}
\pgfpathlineto{\pgfpoint{33.508202\du}{13.656250\du}}
\pgfpathlineto{\pgfpoint{34.060228\du}{13.876768\du}}
\pgfusepath{fill}
\pgfsetbuttcap
\pgfsetmiterjoin
\pgfsetdash{}{0pt}
\definecolor{dialinecolor}{rgb}{1.000000, 1.000000, 1.000000}
\pgfsetfillcolor{dialinecolor}
\pgfpathmoveto{\pgfpoint{33.010575\du}{13.422953\du}}
\pgfpathlineto{\pgfpoint{33.258110\du}{13.505465\du}}
\pgfpathlineto{\pgfpoint{33.843361\du}{13.271438\du}}
\pgfpathlineto{\pgfpoint{34.115723\du}{13.338616\du}}
\pgfpathlineto{\pgfpoint{33.972970\du}{13.130145\du}}
\pgfpathlineto{\pgfpoint{33.267603\du}{13.130145\du}}
\pgfpathlineto{\pgfpoint{33.562601\du}{13.202800\du}}
\pgfpathlineto{\pgfpoint{33.010575\du}{13.422953\du}}
\pgfusepath{fill}
\pgfsetbuttcap
\pgfsetmiterjoin
\pgfsetdash{}{0pt}
\definecolor{dialinecolor}{rgb}{1.000000, 1.000000, 1.000000}
\pgfsetfillcolor{dialinecolor}
\pgfpathmoveto{\pgfpoint{32.908713\du}{13.661726\du}}
\pgfpathlineto{\pgfpoint{32.660082\du}{13.579214\du}}
\pgfpathlineto{\pgfpoint{32.075197\du}{13.813241\du}}
\pgfpathlineto{\pgfpoint{31.801740\du}{13.745698\du}}
\pgfpathlineto{\pgfpoint{31.946683\du}{13.954169\du}}
\pgfpathlineto{\pgfpoint{32.651320\du}{13.954169\du}}
\pgfpathlineto{\pgfpoint{32.356687\du}{13.881514\du}}
\pgfpathlineto{\pgfpoint{32.908713\du}{13.661726\du}}
\pgfusepath{fill}
\pgfsetbuttcap
\pgfsetmiterjoin
\pgfsetdash{}{0pt}
\definecolor{dialinecolor}{rgb}{1.000000, 1.000000, 1.000000}
\pgfsetfillcolor{dialinecolor}
\pgfpathmoveto{\pgfpoint{31.862346\du}{13.202069\du}}
\pgfpathlineto{\pgfpoint{32.109881\du}{13.120288\du}}
\pgfpathlineto{\pgfpoint{32.695496\du}{13.354315\du}}
\pgfpathlineto{\pgfpoint{32.968589\du}{13.287502\du}}
\pgfpathlineto{\pgfpoint{32.823645\du}{13.494877\du}}
\pgfpathlineto{\pgfpoint{32.119374\du}{13.494877\du}}
\pgfpathlineto{\pgfpoint{32.414007\du}{13.422953\du}}
\pgfpathlineto{\pgfpoint{31.862346\du}{13.202069\du}}
\pgfusepath{fill}
\pgfsetbuttcap
\pgfsetmiterjoin
\pgfsetdash{}{0pt}
\definecolor{dialinecolor}{rgb}{1.000000, 1.000000, 1.000000}
\pgfsetfillcolor{dialinecolor}
\pgfpathmoveto{\pgfpoint{34.080308\du}{13.897214\du}}
\pgfpathlineto{\pgfpoint{33.832773\du}{13.979360\du}}
\pgfpathlineto{\pgfpoint{33.247888\du}{13.745698\du}}
\pgfpathlineto{\pgfpoint{32.974430\du}{13.812511\du}}
\pgfpathlineto{\pgfpoint{33.118644\du}{13.604771\du}}
\pgfpathlineto{\pgfpoint{33.823645\du}{13.604771\du}}
\pgfpathlineto{\pgfpoint{33.529377\du}{13.676695\du}}
\pgfpathlineto{\pgfpoint{34.080308\du}{13.897214\du}}
\pgfusepath{fill}
\pgfsetbuttcap
\pgfsetmiterjoin
\pgfsetdash{}{0pt}
\definecolor{dialinecolor}{rgb}{0.678431, 0.839216, 0.905882}
\pgfsetfillcolor{dialinecolor}
\pgfpathmoveto{\pgfpoint{31.276365\du}{13.540514\du}}
\pgfpathlineto{\pgfpoint{31.276365\du}{13.529926\du}}
\pgfpathlineto{\pgfpoint{31.255920\du}{13.529926\du}}
\pgfpathlineto{\pgfpoint{31.255920\du}{13.540514\du}}
\pgfpathlineto{\pgfpoint{31.276365\du}{13.540514\du}}
\pgfusepath{fill}
\pgfsetbuttcap
\pgfsetmiterjoin
\pgfsetdash{}{0pt}
\definecolor{dialinecolor}{rgb}{0.678431, 0.839216, 0.905882}
\pgfsetfillcolor{dialinecolor}
\pgfpathmoveto{\pgfpoint{31.276365\du}{14.370014\du}}
\pgfpathlineto{\pgfpoint{31.276365\du}{13.540514\du}}
\pgfpathlineto{\pgfpoint{31.255920\du}{13.540514\du}}
\pgfpathlineto{\pgfpoint{31.255920\du}{14.370014\du}}
\pgfpathlineto{\pgfpoint{31.276365\du}{14.370014\du}}
\pgfusepath{fill}
\pgfsetbuttcap
\pgfsetmiterjoin
\pgfsetdash{}{0pt}
\definecolor{dialinecolor}{rgb}{0.678431, 0.839216, 0.905882}
\pgfsetfillcolor{dialinecolor}
\pgfpathmoveto{\pgfpoint{31.255920\du}{14.370014\du}}
\pgfpathlineto{\pgfpoint{31.255920\du}{14.380602\du}}
\pgfpathlineto{\pgfpoint{31.276365\du}{14.380602\du}}
\pgfpathlineto{\pgfpoint{31.276365\du}{14.370014\du}}
\pgfpathlineto{\pgfpoint{31.255920\du}{14.370014\du}}
\pgfusepath{fill}
\pgfsetbuttcap
\pgfsetmiterjoin
\pgfsetdash{}{0pt}
\definecolor{dialinecolor}{rgb}{0.678431, 0.839216, 0.905882}
\pgfsetfillcolor{dialinecolor}
\pgfpathmoveto{\pgfpoint{34.637081\du}{13.540514\du}}
\pgfpathlineto{\pgfpoint{34.637081\du}{13.529926\du}}
\pgfpathlineto{\pgfpoint{34.617001\du}{13.529926\du}}
\pgfpathlineto{\pgfpoint{34.617001\du}{13.540514\du}}
\pgfpathlineto{\pgfpoint{34.637081\du}{13.540514\du}}
\pgfusepath{fill}
\pgfsetbuttcap
\pgfsetmiterjoin
\pgfsetdash{}{0pt}
\definecolor{dialinecolor}{rgb}{0.678431, 0.839216, 0.905882}
\pgfsetfillcolor{dialinecolor}
\pgfpathmoveto{\pgfpoint{34.637081\du}{14.370014\du}}
\pgfpathlineto{\pgfpoint{34.637081\du}{13.540514\du}}
\pgfpathlineto{\pgfpoint{34.617001\du}{13.540514\du}}
\pgfpathlineto{\pgfpoint{34.617001\du}{14.370014\du}}
\pgfpathlineto{\pgfpoint{34.637081\du}{14.370014\du}}
\pgfusepath{fill}
\pgfsetbuttcap
\pgfsetmiterjoin
\pgfsetdash{}{0pt}
\definecolor{dialinecolor}{rgb}{0.678431, 0.839216, 0.905882}
\pgfsetfillcolor{dialinecolor}
\pgfpathmoveto{\pgfpoint{34.617001\du}{14.370014\du}}
\pgfpathlineto{\pgfpoint{34.617001\du}{14.380602\du}}
\pgfpathlineto{\pgfpoint{34.637081\du}{14.380602\du}}
\pgfpathlineto{\pgfpoint{34.637081\du}{14.370014\du}}
\pgfpathlineto{\pgfpoint{34.617001\du}{14.370014\du}}
\pgfusepath{fill}
\pgfsetbuttcap
\pgfsetmiterjoin
\pgfsetdash{}{0pt}
\definecolor{dialinecolor}{rgb}{0.027451, 0.372549, 0.529412}
\pgfsetfillcolor{dialinecolor}
\pgfpathmoveto{\pgfpoint{33.563331\du}{14.519338\du}}
\pgfpathlineto{\pgfpoint{33.562966\du}{14.536863\du}}
\pgfpathlineto{\pgfpoint{33.560411\du}{14.554023\du}}
\pgfpathlineto{\pgfpoint{33.556760\du}{14.570087\du}}
\pgfpathlineto{\pgfpoint{33.550918\du}{14.587246\du}}
\pgfpathlineto{\pgfpoint{33.544346\du}{14.602946\du}}
\pgfpathlineto{\pgfpoint{33.535949\du}{14.619010\du}}
\pgfpathlineto{\pgfpoint{33.526457\du}{14.634709\du}}
\pgfpathlineto{\pgfpoint{33.514773\du}{14.649678\du}}
\pgfpathlineto{\pgfpoint{33.503090\du}{14.664647\du}}
\pgfpathlineto{\pgfpoint{33.489582\du}{14.679251\du}}
\pgfpathlineto{\pgfpoint{33.474613\du}{14.693124\du}}
\pgfpathlineto{\pgfpoint{33.458549\du}{14.707363\du}}
\pgfpathlineto{\pgfpoint{33.441024\du}{14.720142\du}}
\pgfpathlineto{\pgfpoint{33.422769\du}{14.732920\du}}
\pgfpathlineto{\pgfpoint{33.403784\du}{14.745333\du}}
\pgfpathlineto{\pgfpoint{33.383704\du}{14.757016\du}}
\pgfpathlineto{\pgfpoint{33.362163\du}{14.767604\du}}
\pgfpathlineto{\pgfpoint{33.339527\du}{14.778557\du}}
\pgfpathlineto{\pgfpoint{33.316526\du}{14.788415\du}}
\pgfpathlineto{\pgfpoint{33.292795\du}{14.797907\du}}
\pgfpathlineto{\pgfpoint{33.267603\du}{14.805939\du}}
\pgfpathlineto{\pgfpoint{33.242046\du}{14.814337\du}}
\pgfpathlineto{\pgfpoint{33.215394\du}{14.822004\du}}
\pgfpathlineto{\pgfpoint{33.188742\du}{14.828940\du}}
\pgfpathlineto{\pgfpoint{33.160630\du}{14.834782\du}}
\pgfpathlineto{\pgfpoint{33.132152\du}{14.839893\du}}
\pgfpathlineto{\pgfpoint{33.102579\du}{14.844275\du}}
\pgfpathlineto{\pgfpoint{33.072276\du}{14.848291\du}}
\pgfpathlineto{\pgfpoint{33.042338\du}{14.851211\du}}
\pgfpathlineto{\pgfpoint{33.011670\du}{14.853402\du}}
\pgfpathlineto{\pgfpoint{32.980272\du}{14.854497\du}}
\pgfpathlineto{\pgfpoint{32.948874\du}{14.855227\du}}
\pgfpathlineto{\pgfpoint{32.917840\du}{14.854497\du}}
\pgfpathlineto{\pgfpoint{32.886442\du}{14.853402\du}}
\pgfpathlineto{\pgfpoint{32.855409\du}{14.851211\du}}
\pgfpathlineto{\pgfpoint{32.825106\du}{14.848291\du}}
\pgfpathlineto{\pgfpoint{32.795533\du}{14.844275\du}}
\pgfpathlineto{\pgfpoint{32.765960\du}{14.839893\du}}
\pgfpathlineto{\pgfpoint{32.737482\du}{14.834782\du}}
\pgfpathlineto{\pgfpoint{32.709735\du}{14.828940\du}}
\pgfpathlineto{\pgfpoint{32.682353\du}{14.822004\du}}
\pgfpathlineto{\pgfpoint{32.656066\du}{14.814337\du}}
\pgfpathlineto{\pgfpoint{32.630874\du}{14.805939\du}}
\pgfpathlineto{\pgfpoint{32.605317\du}{14.797907\du}}
\pgfpathlineto{\pgfpoint{32.581221\du}{14.788415\du}}
\pgfpathlineto{\pgfpoint{32.558585\du}{14.778557\du}}
\pgfpathlineto{\pgfpoint{32.535949\du}{14.767604\du}}
\pgfpathlineto{\pgfpoint{32.515139\du}{14.757016\du}}
\pgfpathlineto{\pgfpoint{32.494693\du}{14.745333\du}}
\pgfpathlineto{\pgfpoint{32.474248\du}{14.732920\du}}
\pgfpathlineto{\pgfpoint{32.456358\du}{14.720142\du}}
\pgfpathlineto{\pgfpoint{32.439929\du}{14.707363\du}}
\pgfpathlineto{\pgfpoint{32.423134\du}{14.693124\du}}
\pgfpathlineto{\pgfpoint{32.408530\du}{14.679251\du}}
\pgfpathlineto{\pgfpoint{32.395387\du}{14.664647\du}}
\pgfpathlineto{\pgfpoint{32.382974\du}{14.649678\du}}
\pgfpathlineto{\pgfpoint{32.372021\du}{14.634709\du}}
\pgfpathlineto{\pgfpoint{32.362528\du}{14.619010\du}}
\pgfpathlineto{\pgfpoint{32.353766\du}{14.602946\du}}
\pgfpathlineto{\pgfpoint{32.346829\du}{14.587246\du}}
\pgfpathlineto{\pgfpoint{32.342083\du}{14.570087\du}}
\pgfpathlineto{\pgfpoint{32.337702\du}{14.554023\du}}
\pgfpathlineto{\pgfpoint{32.335146\du}{14.536863\du}}
\pgfpathlineto{\pgfpoint{32.334051\du}{14.519338\du}}
\pgfpathlineto{\pgfpoint{32.335146\du}{14.501814\du}}
\pgfpathlineto{\pgfpoint{32.337702\du}{14.484654\du}}
\pgfpathlineto{\pgfpoint{32.342083\du}{14.468590\du}}
\pgfpathlineto{\pgfpoint{32.346829\du}{14.451430\du}}
\pgfpathlineto{\pgfpoint{32.353766\du}{14.435731\du}}
\pgfpathlineto{\pgfpoint{32.362528\du}{14.420032\du}}
\pgfpathlineto{\pgfpoint{32.372021\du}{14.403968\du}}
\pgfpathlineto{\pgfpoint{32.382974\du}{14.388999\du}}
\pgfpathlineto{\pgfpoint{32.395387\du}{14.374395\du}}
\pgfpathlineto{\pgfpoint{32.408530\du}{14.359791\du}}
\pgfpathlineto{\pgfpoint{32.423134\du}{14.345552\du}}
\pgfpathlineto{\pgfpoint{32.439929\du}{14.331679\du}}
\pgfpathlineto{\pgfpoint{32.456358\du}{14.318535\du}}
\pgfpathlineto{\pgfpoint{32.474248\du}{14.306487\du}}
\pgfpathlineto{\pgfpoint{32.494693\du}{14.294074\du}}
\pgfpathlineto{\pgfpoint{32.515139\du}{14.282391\du}}
\pgfpathlineto{\pgfpoint{32.535949\du}{14.271438\du}}
\pgfpathlineto{\pgfpoint{32.558585\du}{14.260850\du}}
\pgfpathlineto{\pgfpoint{32.581221\du}{14.250262\du}}
\pgfpathlineto{\pgfpoint{32.605317\du}{14.241500\du}}
\pgfpathlineto{\pgfpoint{32.630874\du}{14.232737\du}}
\pgfpathlineto{\pgfpoint{32.656066\du}{14.224340\du}}
\pgfpathlineto{\pgfpoint{32.682353\du}{14.216673\du}}
\pgfpathlineto{\pgfpoint{32.709735\du}{14.210467\du}}
\pgfpathlineto{\pgfpoint{32.737482\du}{14.203895\du}}
\pgfpathlineto{\pgfpoint{32.765960\du}{14.198783\du}}
\pgfpathlineto{\pgfpoint{32.795533\du}{14.194767\du}}
\pgfpathlineto{\pgfpoint{32.825106\du}{14.190386\du}}
\pgfpathlineto{\pgfpoint{32.855409\du}{14.187465\du}}
\pgfpathlineto{\pgfpoint{32.886442\du}{14.186005\du}}
\pgfpathlineto{\pgfpoint{32.917840\du}{14.184180\du}}
\pgfpathlineto{\pgfpoint{32.948874\du}{14.184180\du}}
\pgfpathlineto{\pgfpoint{32.980272\du}{14.184180\du}}
\pgfpathlineto{\pgfpoint{33.011670\du}{14.186005\du}}
\pgfpathlineto{\pgfpoint{33.042338\du}{14.187465\du}}
\pgfpathlineto{\pgfpoint{33.072276\du}{14.190386\du}}
\pgfpathlineto{\pgfpoint{33.102579\du}{14.194767\du}}
\pgfpathlineto{\pgfpoint{33.132152\du}{14.198783\du}}
\pgfpathlineto{\pgfpoint{33.160630\du}{14.203895\du}}
\pgfpathlineto{\pgfpoint{33.188742\du}{14.210467\du}}
\pgfpathlineto{\pgfpoint{33.215394\du}{14.216673\du}}
\pgfpathlineto{\pgfpoint{33.242046\du}{14.224340\du}}
\pgfpathlineto{\pgfpoint{33.267603\du}{14.232737\du}}
\pgfpathlineto{\pgfpoint{33.292795\du}{14.241500\du}}
\pgfpathlineto{\pgfpoint{33.316526\du}{14.250262\du}}
\pgfpathlineto{\pgfpoint{33.339527\du}{14.260850\du}}
\pgfpathlineto{\pgfpoint{33.362163\du}{14.271438\du}}
\pgfpathlineto{\pgfpoint{33.383704\du}{14.282391\du}}
\pgfpathlineto{\pgfpoint{33.403784\du}{14.294074\du}}
\pgfpathlineto{\pgfpoint{33.422769\du}{14.306487\du}}
\pgfpathlineto{\pgfpoint{33.441024\du}{14.318535\du}}
\pgfpathlineto{\pgfpoint{33.458549\du}{14.331679\du}}
\pgfpathlineto{\pgfpoint{33.474613\du}{14.345552\du}}
\pgfpathlineto{\pgfpoint{33.489582\du}{14.359791\du}}
\pgfpathlineto{\pgfpoint{33.503090\du}{14.374395\du}}
\pgfpathlineto{\pgfpoint{33.514773\du}{14.388999\du}}
\pgfpathlineto{\pgfpoint{33.526457\du}{14.403968\du}}
\pgfpathlineto{\pgfpoint{33.535949\du}{14.420032\du}}
\pgfpathlineto{\pgfpoint{33.544346\du}{14.435731\du}}
\pgfpathlineto{\pgfpoint{33.550918\du}{14.451430\du}}
\pgfpathlineto{\pgfpoint{33.556760\du}{14.468590\du}}
\pgfpathlineto{\pgfpoint{33.560411\du}{14.484654\du}}
\pgfpathlineto{\pgfpoint{33.562966\du}{14.501814\du}}
\pgfpathlineto{\pgfpoint{33.563331\du}{14.519338\du}}
\pgfusepath{fill}
\pgfsetbuttcap
\pgfsetmiterjoin
\pgfsetdash{}{0pt}
\definecolor{dialinecolor}{rgb}{0.678431, 0.839216, 0.905882}
\pgfsetfillcolor{dialinecolor}
\pgfpathmoveto{\pgfpoint{32.948874\du}{14.865085\du}}
\pgfpathlineto{\pgfpoint{32.948874\du}{14.865085\du}}
\pgfpathlineto{\pgfpoint{32.964938\du}{14.865085\du}}
\pgfpathlineto{\pgfpoint{32.981002\du}{14.864720\du}}
\pgfpathlineto{\pgfpoint{32.997066\du}{14.863990\du}}
\pgfpathlineto{\pgfpoint{33.012400\du}{14.863260\du}}
\pgfpathlineto{\pgfpoint{33.028100\du}{14.862164\du}}
\pgfpathlineto{\pgfpoint{33.043434\du}{14.861069\du}}
\pgfpathlineto{\pgfpoint{33.058403\du}{14.859974\du}}
\pgfpathlineto{\pgfpoint{33.074467\du}{14.858148\du}}
\pgfpathlineto{\pgfpoint{33.088706\du}{14.856323\du}}
\pgfpathlineto{\pgfpoint{33.103675\du}{14.854497\du}}
\pgfpathlineto{\pgfpoint{33.118644\du}{14.852307\du}}
\pgfpathlineto{\pgfpoint{33.133978\du}{14.850116\du}}
\pgfpathlineto{\pgfpoint{33.148216\du}{14.847195\du}}
\pgfpathlineto{\pgfpoint{33.162090\du}{14.844640\du}}
\pgfpathlineto{\pgfpoint{33.176694\du}{14.841719\du}}
\pgfpathlineto{\pgfpoint{33.190568\du}{14.838798\du}}
\pgfpathlineto{\pgfpoint{33.204076\du}{14.835147\du}}
\pgfpathlineto{\pgfpoint{33.217950\du}{14.831861\du}}
\pgfpathlineto{\pgfpoint{33.231458\du}{14.828210\du}}
\pgfpathlineto{\pgfpoint{33.244602\du}{14.824194\du}}
\pgfpathlineto{\pgfpoint{33.257745\du}{14.820178\du}}
\pgfpathlineto{\pgfpoint{33.270889\du}{14.816162\du}}
\pgfpathlineto{\pgfpoint{33.283667\du}{14.811416\du}}
\pgfpathlineto{\pgfpoint{33.296446\du}{14.807400\du}}
\pgfpathlineto{\pgfpoint{33.308129\du}{14.802654\du}}
\pgfpathlineto{\pgfpoint{33.320907\du}{14.797907\du}}
\pgfpathlineto{\pgfpoint{33.332225\du}{14.792431\du}}
\pgfpathlineto{\pgfpoint{33.343908\du}{14.787319\du}}
\pgfpathlineto{\pgfpoint{33.355226\du}{14.782208\du}}
\pgfpathlineto{\pgfpoint{33.366544\du}{14.776732\du}}
\pgfpathlineto{\pgfpoint{33.377497\du}{14.771620\du}}
\pgfpathlineto{\pgfpoint{33.388815\du}{14.765779\du}}
\pgfpathlineto{\pgfpoint{33.398673\du}{14.759937\du}}
\pgfpathlineto{\pgfpoint{33.408895\du}{14.753365\du}}
\pgfpathlineto{\pgfpoint{33.418753\du}{14.747524\du}}
\pgfpathlineto{\pgfpoint{33.428976\du}{14.740952\du}}
\pgfpathlineto{\pgfpoint{33.438468\du}{14.734746\du}}
\pgfpathlineto{\pgfpoint{33.447596\du}{14.728174\du}}
\pgfpathlineto{\pgfpoint{33.456358\du}{14.721967\du}}
\pgfpathlineto{\pgfpoint{33.464755\du}{14.714665\du}}
\pgfpathlineto{\pgfpoint{33.473152\du}{14.707728\du}}
\pgfpathlineto{\pgfpoint{33.481185\du}{14.700792\du}}
\pgfpathlineto{\pgfpoint{33.489217\du}{14.693855\du}}
\pgfpathlineto{\pgfpoint{33.496154\du}{14.686553\du}}
\pgfpathlineto{\pgfpoint{33.503821\du}{14.679251\du}}
\pgfpathlineto{\pgfpoint{33.510392\du}{14.671584\du}}
\pgfpathlineto{\pgfpoint{33.516964\du}{14.663917\du}}
\pgfpathlineto{\pgfpoint{33.523171\du}{14.656250\du}}
\pgfpathlineto{\pgfpoint{33.529377\du}{14.648218\du}}
\pgfpathlineto{\pgfpoint{33.535219\du}{14.640551\du}}
\pgfpathlineto{\pgfpoint{33.539965\du}{14.632153\du}}
\pgfpathlineto{\pgfpoint{33.544711\du}{14.624121\du}}
\pgfpathlineto{\pgfpoint{33.549458\du}{14.615724\du}}
\pgfpathlineto{\pgfpoint{33.553839\du}{14.607692\du}}
\pgfpathlineto{\pgfpoint{33.557125\du}{14.598930\du}}
\pgfpathlineto{\pgfpoint{33.561141\du}{14.590897\du}}
\pgfpathlineto{\pgfpoint{33.563696\du}{14.582135\du}}
\pgfpathlineto{\pgfpoint{33.566617\du}{14.573373\du}}
\pgfpathlineto{\pgfpoint{33.568443\du}{14.564245\du}}
\pgfpathlineto{\pgfpoint{33.570633\du}{14.555483\du}}
\pgfpathlineto{\pgfpoint{33.572094\du}{14.546721\du}}
\pgfpathlineto{\pgfpoint{33.573189\du}{14.537593\du}}
\pgfpathlineto{\pgfpoint{33.573919\du}{14.528831\du}}
\pgfpathlineto{\pgfpoint{33.573919\du}{14.519338\du}}
\pgfpathlineto{\pgfpoint{33.553839\du}{14.519338\du}}
\pgfpathlineto{\pgfpoint{33.553109\du}{14.527371\du}}
\pgfpathlineto{\pgfpoint{33.552744\du}{14.535768\du}}
\pgfpathlineto{\pgfpoint{33.552013\du}{14.543800\du}}
\pgfpathlineto{\pgfpoint{33.550553\du}{14.551467\du}}
\pgfpathlineto{\pgfpoint{33.549093\du}{14.559864\du}}
\pgfpathlineto{\pgfpoint{33.546537\du}{14.567896\du}}
\pgfpathlineto{\pgfpoint{33.544346\du}{14.575563\du}}
\pgfpathlineto{\pgfpoint{33.541060\du}{14.583230\du}}
\pgfpathlineto{\pgfpoint{33.538505\du}{14.590897\du}}
\pgfpathlineto{\pgfpoint{33.535219\du}{14.598930\du}}
\pgfpathlineto{\pgfpoint{33.531203\du}{14.606597\du}}
\pgfpathlineto{\pgfpoint{33.526457\du}{14.614264\du}}
\pgfpathlineto{\pgfpoint{33.522441\du}{14.621931\du}}
\pgfpathlineto{\pgfpoint{33.517329\du}{14.628867\du}}
\pgfpathlineto{\pgfpoint{33.512583\du}{14.636535\du}}
\pgfpathlineto{\pgfpoint{33.507106\du}{14.643471\du}}
\pgfpathlineto{\pgfpoint{33.501995\du}{14.651138\du}}
\pgfpathlineto{\pgfpoint{33.495058\du}{14.658075\du}}
\pgfpathlineto{\pgfpoint{33.489217\du}{14.665012\du}}
\pgfpathlineto{\pgfpoint{33.481915\du}{14.671949\du}}
\pgfpathlineto{\pgfpoint{33.474978\du}{14.679251\du}}
\pgfpathlineto{\pgfpoint{33.467676\du}{14.685457\du}}
\pgfpathlineto{\pgfpoint{33.460739\du}{14.692394\du}}
\pgfpathlineto{\pgfpoint{33.452342\du}{14.698966\du}}
\pgfpathlineto{\pgfpoint{33.444310\du}{14.705538\du}}
\pgfpathlineto{\pgfpoint{33.435182\du}{14.711744\du}}
\pgfpathlineto{\pgfpoint{33.426420\du}{14.717586\du}}
\pgfpathlineto{\pgfpoint{33.416928\du}{14.724158\du}}
\pgfpathlineto{\pgfpoint{33.407800\du}{14.730729\du}}
\pgfpathlineto{\pgfpoint{33.398673\du}{14.735841\du}}
\pgfpathlineto{\pgfpoint{33.388815\du}{14.741682\du}}
\pgfpathlineto{\pgfpoint{33.378957\du}{14.747524\du}}
\pgfpathlineto{\pgfpoint{33.368370\du}{14.753000\du}}
\pgfpathlineto{\pgfpoint{33.357782\du}{14.758842\du}}
\pgfpathlineto{\pgfpoint{33.347194\du}{14.763223\du}}
\pgfpathlineto{\pgfpoint{33.335511\du}{14.769065\du}}
\pgfpathlineto{\pgfpoint{33.324193\du}{14.773811\du}}
\pgfpathlineto{\pgfpoint{33.312875\du}{14.778557\du}}
\pgfpathlineto{\pgfpoint{33.301192\du}{14.783303\du}}
\pgfpathlineto{\pgfpoint{33.289144\du}{14.788050\du}}
\pgfpathlineto{\pgfpoint{33.276730\du}{14.792066\du}}
\pgfpathlineto{\pgfpoint{33.265047\du}{14.796812\du}}
\pgfpathlineto{\pgfpoint{33.252269\du}{14.800828\du}}
\pgfpathlineto{\pgfpoint{33.238760\du}{14.804479\du}}
\pgfpathlineto{\pgfpoint{33.225982\du}{14.808495\du}}
\pgfpathlineto{\pgfpoint{33.212838\du}{14.811781\du}}
\pgfpathlineto{\pgfpoint{33.199330\du}{14.815432\du}}
\pgfpathlineto{\pgfpoint{33.185821\du}{14.818353\du}}
\pgfpathlineto{\pgfpoint{33.171948\du}{14.822004\du}}
\pgfpathlineto{\pgfpoint{33.158074\du}{14.824194\du}}
\pgfpathlineto{\pgfpoint{33.144200\du}{14.827115\du}}
\pgfpathlineto{\pgfpoint{33.129962\du}{14.829306\du}}
\pgfpathlineto{\pgfpoint{33.115723\du}{14.831861\du}}
\pgfpathlineto{\pgfpoint{33.101484\du}{14.834052\du}}
\pgfpathlineto{\pgfpoint{33.086880\du}{14.835877\du}}
\pgfpathlineto{\pgfpoint{33.071911\du}{14.837703\du}}
\pgfpathlineto{\pgfpoint{33.056577\du}{14.839528\du}}
\pgfpathlineto{\pgfpoint{33.041973\du}{14.840624\du}}
\pgfpathlineto{\pgfpoint{33.026274\du}{14.841719\du}}
\pgfpathlineto{\pgfpoint{33.011305\du}{14.842814\du}}
\pgfpathlineto{\pgfpoint{32.995971\du}{14.843544\du}}
\pgfpathlineto{\pgfpoint{32.980272\du}{14.844275\du}}
\pgfpathlineto{\pgfpoint{32.964938\du}{14.844640\du}}
\pgfpathlineto{\pgfpoint{32.948874\du}{14.844640\du}}
\pgfpathlineto{\pgfpoint{32.948874\du}{14.844640\du}}
\pgfpathlineto{\pgfpoint{32.948874\du}{14.844640\du}}
\pgfpathlineto{\pgfpoint{32.947778\du}{14.844640\du}}
\pgfpathlineto{\pgfpoint{32.946683\du}{14.844640\du}}
\pgfpathlineto{\pgfpoint{32.945953\du}{14.845370\du}}
\pgfpathlineto{\pgfpoint{32.944127\du}{14.845370\du}}
\pgfpathlineto{\pgfpoint{32.943762\du}{14.845735\du}}
\pgfpathlineto{\pgfpoint{32.942667\du}{14.846465\du}}
\pgfpathlineto{\pgfpoint{32.942302\du}{14.847195\du}}
\pgfpathlineto{\pgfpoint{32.941937\du}{14.847560\du}}
\pgfpathlineto{\pgfpoint{32.940476\du}{14.849386\du}}
\pgfpathlineto{\pgfpoint{32.939016\du}{14.851211\du}}
\pgfpathlineto{\pgfpoint{32.939016\du}{14.853037\du}}
\pgfpathlineto{\pgfpoint{32.938651\du}{14.855227\du}}
\pgfpathlineto{\pgfpoint{32.939016\du}{14.857053\du}}
\pgfpathlineto{\pgfpoint{32.939016\du}{14.858878\du}}
\pgfpathlineto{\pgfpoint{32.940476\du}{14.860339\du}}
\pgfpathlineto{\pgfpoint{32.941937\du}{14.861799\du}}
\pgfpathlineto{\pgfpoint{32.942302\du}{14.862894\du}}
\pgfpathlineto{\pgfpoint{32.942667\du}{14.863260\du}}
\pgfpathlineto{\pgfpoint{32.943762\du}{14.863990\du}}
\pgfpathlineto{\pgfpoint{32.944127\du}{14.863990\du}}
\pgfpathlineto{\pgfpoint{32.945953\du}{14.864720\du}}
\pgfpathlineto{\pgfpoint{32.946683\du}{14.865085\du}}
\pgfpathlineto{\pgfpoint{32.947778\du}{14.865085\du}}
\pgfpathlineto{\pgfpoint{32.948874\du}{14.865085\du}}
\pgfusepath{fill}
\pgfsetbuttcap
\pgfsetmiterjoin
\pgfsetdash{}{0pt}
\definecolor{dialinecolor}{rgb}{0.678431, 0.839216, 0.905882}
\pgfsetfillcolor{dialinecolor}
\pgfpathmoveto{\pgfpoint{32.324193\du}{14.519338\du}}
\pgfpathlineto{\pgfpoint{32.324193\du}{14.519338\du}}
\pgfpathlineto{\pgfpoint{32.324193\du}{14.528101\du}}
\pgfpathlineto{\pgfpoint{32.324923\du}{14.537593\du}}
\pgfpathlineto{\pgfpoint{32.326018\du}{14.546721\du}}
\pgfpathlineto{\pgfpoint{32.328209\du}{14.555483\du}}
\pgfpathlineto{\pgfpoint{32.329304\du}{14.564245\du}}
\pgfpathlineto{\pgfpoint{32.331860\du}{14.573373\du}}
\pgfpathlineto{\pgfpoint{32.334051\du}{14.582135\du}}
\pgfpathlineto{\pgfpoint{32.337336\du}{14.590897\du}}
\pgfpathlineto{\pgfpoint{32.340622\du}{14.598930\du}}
\pgfpathlineto{\pgfpoint{32.344638\du}{14.607692\du}}
\pgfpathlineto{\pgfpoint{32.349020\du}{14.615724\du}}
\pgfpathlineto{\pgfpoint{32.353401\du}{14.624121\du}}
\pgfpathlineto{\pgfpoint{32.358147\du}{14.632153\du}}
\pgfpathlineto{\pgfpoint{32.363258\du}{14.640551\du}}
\pgfpathlineto{\pgfpoint{32.369465\du}{14.648218\du}}
\pgfpathlineto{\pgfpoint{32.374211\du}{14.656250\du}}
\pgfpathlineto{\pgfpoint{32.380783\du}{14.663917\du}}
\pgfpathlineto{\pgfpoint{32.387720\du}{14.671584\du}}
\pgfpathlineto{\pgfpoint{32.394292\du}{14.679251\du}}
\pgfpathlineto{\pgfpoint{32.401228\du}{14.686553\du}}
\pgfpathlineto{\pgfpoint{32.408530\du}{14.693855\du}}
\pgfpathlineto{\pgfpoint{32.417293\du}{14.700792\du}}
\pgfpathlineto{\pgfpoint{32.424229\du}{14.707728\du}}
\pgfpathlineto{\pgfpoint{32.433357\du}{14.714665\du}}
\pgfpathlineto{\pgfpoint{32.441389\du}{14.721967\du}}
\pgfpathlineto{\pgfpoint{32.450516\du}{14.728174\du}}
\pgfpathlineto{\pgfpoint{32.460374\du}{14.734746\du}}
\pgfpathlineto{\pgfpoint{32.469501\du}{14.740952\du}}
\pgfpathlineto{\pgfpoint{32.478994\du}{14.747524\du}}
\pgfpathlineto{\pgfpoint{32.488852\du}{14.753365\du}}
\pgfpathlineto{\pgfpoint{32.499074\du}{14.759937\du}}
\pgfpathlineto{\pgfpoint{32.509297\du}{14.765779\du}}
\pgfpathlineto{\pgfpoint{32.520250\du}{14.771620\du}}
\pgfpathlineto{\pgfpoint{32.531568\du}{14.776732\du}}
\pgfpathlineto{\pgfpoint{32.542886\du}{14.782208\du}}
\pgfpathlineto{\pgfpoint{32.554204\du}{14.787319\du}}
\pgfpathlineto{\pgfpoint{32.566252\du}{14.792431\du}}
\pgfpathlineto{\pgfpoint{32.577570\du}{14.797907\du}}
\pgfpathlineto{\pgfpoint{32.589983\du}{14.802654\du}}
\pgfpathlineto{\pgfpoint{32.602032\du}{14.807400\du}}
\pgfpathlineto{\pgfpoint{32.614080\du}{14.811416\du}}
\pgfpathlineto{\pgfpoint{32.627223\du}{14.816162\du}}
\pgfpathlineto{\pgfpoint{32.640002\du}{14.820178\du}}
\pgfpathlineto{\pgfpoint{32.653510\du}{14.824194\du}}
\pgfpathlineto{\pgfpoint{32.667019\du}{14.828210\du}}
\pgfpathlineto{\pgfpoint{32.679797\du}{14.831861\du}}
\pgfpathlineto{\pgfpoint{32.693306\du}{14.835147\du}}
\pgfpathlineto{\pgfpoint{32.707179\du}{14.838798\du}}
\pgfpathlineto{\pgfpoint{32.721783\du}{14.841719\du}}
\pgfpathlineto{\pgfpoint{32.736387\du}{14.844640\du}}
\pgfpathlineto{\pgfpoint{32.750261\du}{14.847195\du}}
\pgfpathlineto{\pgfpoint{32.764500\du}{14.850116\du}}
\pgfpathlineto{\pgfpoint{32.779469\du}{14.852307\du}}
\pgfpathlineto{\pgfpoint{32.794438\du}{14.854497\du}}
\pgfpathlineto{\pgfpoint{32.809041\du}{14.856323\du}}
\pgfpathlineto{\pgfpoint{32.823645\du}{14.858148\du}}
\pgfpathlineto{\pgfpoint{32.839344\du}{14.859974\du}}
\pgfpathlineto{\pgfpoint{32.855044\du}{14.861069\du}}
\pgfpathlineto{\pgfpoint{32.869648\du}{14.862164\du}}
\pgfpathlineto{\pgfpoint{32.885712\du}{14.863260\du}}
\pgfpathlineto{\pgfpoint{32.901046\du}{14.863990\du}}
\pgfpathlineto{\pgfpoint{32.917110\du}{14.864720\du}}
\pgfpathlineto{\pgfpoint{32.933174\du}{14.865085\du}}
\pgfpathlineto{\pgfpoint{32.948874\du}{14.865085\du}}
\pgfpathlineto{\pgfpoint{32.948874\du}{14.844640\du}}
\pgfpathlineto{\pgfpoint{32.933174\du}{14.844640\du}}
\pgfpathlineto{\pgfpoint{32.917840\du}{14.844275\du}}
\pgfpathlineto{\pgfpoint{32.901776\du}{14.843544\du}}
\pgfpathlineto{\pgfpoint{32.887172\du}{14.842814\du}}
\pgfpathlineto{\pgfpoint{32.871473\du}{14.841719\du}}
\pgfpathlineto{\pgfpoint{32.856504\du}{14.840624\du}}
\pgfpathlineto{\pgfpoint{32.841535\du}{14.839528\du}}
\pgfpathlineto{\pgfpoint{32.826566\du}{14.837703\du}}
\pgfpathlineto{\pgfpoint{32.811597\du}{14.835877\du}}
\pgfpathlineto{\pgfpoint{32.796993\du}{14.834052\du}}
\pgfpathlineto{\pgfpoint{32.782389\du}{14.831861\du}}
\pgfpathlineto{\pgfpoint{32.768151\du}{14.829306\du}}
\pgfpathlineto{\pgfpoint{32.754277\du}{14.827115\du}}
\pgfpathlineto{\pgfpoint{32.740403\du}{14.824194\du}}
\pgfpathlineto{\pgfpoint{32.726164\du}{14.822004\du}}
\pgfpathlineto{\pgfpoint{32.712291\du}{14.818353\du}}
\pgfpathlineto{\pgfpoint{32.698782\du}{14.815432\du}}
\pgfpathlineto{\pgfpoint{32.685639\du}{14.811781\du}}
\pgfpathlineto{\pgfpoint{32.672130\du}{14.808495\du}}
\pgfpathlineto{\pgfpoint{32.659352\du}{14.804479\du}}
\pgfpathlineto{\pgfpoint{32.646208\du}{14.800828\du}}
\pgfpathlineto{\pgfpoint{32.633430\du}{14.796812\du}}
\pgfpathlineto{\pgfpoint{32.621382\du}{14.792066\du}}
\pgfpathlineto{\pgfpoint{32.608968\du}{14.788050\du}}
\pgfpathlineto{\pgfpoint{32.596555\du}{14.783303\du}}
\pgfpathlineto{\pgfpoint{32.585237\du}{14.778557\du}}
\pgfpathlineto{\pgfpoint{32.573554\du}{14.773811\du}}
\pgfpathlineto{\pgfpoint{32.562601\du}{14.769065\du}}
\pgfpathlineto{\pgfpoint{32.551283\du}{14.763223\du}}
\pgfpathlineto{\pgfpoint{32.540695\du}{14.758842\du}}
\pgfpathlineto{\pgfpoint{32.529742\du}{14.753000\du}}
\pgfpathlineto{\pgfpoint{32.519520\du}{14.747524\du}}
\pgfpathlineto{\pgfpoint{32.509297\du}{14.741682\du}}
\pgfpathlineto{\pgfpoint{32.499439\du}{14.735841\du}}
\pgfpathlineto{\pgfpoint{32.489947\du}{14.730729\du}}
\pgfpathlineto{\pgfpoint{32.480089\du}{14.724158\du}}
\pgfpathlineto{\pgfpoint{32.471692\du}{14.717586\du}}
\pgfpathlineto{\pgfpoint{32.462565\du}{14.711744\du}}
\pgfpathlineto{\pgfpoint{32.454167\du}{14.705538\du}}
\pgfpathlineto{\pgfpoint{32.445770\du}{14.698966\du}}
\pgfpathlineto{\pgfpoint{32.437738\du}{14.692394\du}}
\pgfpathlineto{\pgfpoint{32.430436\du}{14.685457\du}}
\pgfpathlineto{\pgfpoint{32.422769\du}{14.679251\du}}
\pgfpathlineto{\pgfpoint{32.416197\du}{14.671949\du}}
\pgfpathlineto{\pgfpoint{32.409261\du}{14.665012\du}}
\pgfpathlineto{\pgfpoint{32.403054\du}{14.658075\du}}
\pgfpathlineto{\pgfpoint{32.396847\du}{14.651138\du}}
\pgfpathlineto{\pgfpoint{32.390641\du}{14.643471\du}}
\pgfpathlineto{\pgfpoint{32.385529\du}{14.636535\du}}
\pgfpathlineto{\pgfpoint{32.380418\du}{14.628867\du}}
\pgfpathlineto{\pgfpoint{32.375672\du}{14.621931\du}}
\pgfpathlineto{\pgfpoint{32.371290\du}{14.614264\du}}
\pgfpathlineto{\pgfpoint{32.366909\du}{14.606597\du}}
\pgfpathlineto{\pgfpoint{32.363258\du}{14.598930\du}}
\pgfpathlineto{\pgfpoint{32.359972\du}{14.590897\du}}
\pgfpathlineto{\pgfpoint{32.356687\du}{14.583230\du}}
\pgfpathlineto{\pgfpoint{32.353766\du}{14.575563\du}}
\pgfpathlineto{\pgfpoint{32.351210\du}{14.567896\du}}
\pgfpathlineto{\pgfpoint{32.349385\du}{14.559864\du}}
\pgfpathlineto{\pgfpoint{32.347559\du}{14.551467\du}}
\pgfpathlineto{\pgfpoint{32.346464\du}{14.543800\du}}
\pgfpathlineto{\pgfpoint{32.345003\du}{14.535768\du}}
\pgfpathlineto{\pgfpoint{32.344638\du}{14.527371\du}}
\pgfpathlineto{\pgfpoint{32.344638\du}{14.519338\du}}
\pgfpathlineto{\pgfpoint{32.344638\du}{14.519338\du}}
\pgfpathlineto{\pgfpoint{32.344638\du}{14.519338\du}}
\pgfpathlineto{\pgfpoint{32.344638\du}{14.518243\du}}
\pgfpathlineto{\pgfpoint{32.344638\du}{14.517148\du}}
\pgfpathlineto{\pgfpoint{32.344273\du}{14.515687\du}}
\pgfpathlineto{\pgfpoint{32.344273\du}{14.514592\du}}
\pgfpathlineto{\pgfpoint{32.343908\du}{14.514227\du}}
\pgfpathlineto{\pgfpoint{32.342813\du}{14.512767\du}}
\pgfpathlineto{\pgfpoint{32.342083\du}{14.512402\du}}
\pgfpathlineto{\pgfpoint{32.342083\du}{14.511671\du}}
\pgfpathlineto{\pgfpoint{32.339892\du}{14.510576\du}}
\pgfpathlineto{\pgfpoint{32.338067\du}{14.509846\du}}
\pgfpathlineto{\pgfpoint{32.336241\du}{14.509481\du}}
\pgfpathlineto{\pgfpoint{32.334051\du}{14.508751\du}}
\pgfpathlineto{\pgfpoint{32.332590\du}{14.509481\du}}
\pgfpathlineto{\pgfpoint{32.330765\du}{14.509846\du}}
\pgfpathlineto{\pgfpoint{32.328939\du}{14.510576\du}}
\pgfpathlineto{\pgfpoint{32.327114\du}{14.511671\du}}
\pgfpathlineto{\pgfpoint{32.326384\du}{14.512402\du}}
\pgfpathlineto{\pgfpoint{32.326018\du}{14.512767\du}}
\pgfpathlineto{\pgfpoint{32.325653\du}{14.514227\du}}
\pgfpathlineto{\pgfpoint{32.324923\du}{14.514592\du}}
\pgfpathlineto{\pgfpoint{32.324923\du}{14.515687\du}}
\pgfpathlineto{\pgfpoint{32.324193\du}{14.517148\du}}
\pgfpathlineto{\pgfpoint{32.324193\du}{14.518243\du}}
\pgfpathlineto{\pgfpoint{32.324193\du}{14.519338\du}}
\pgfusepath{fill}
\pgfsetbuttcap
\pgfsetmiterjoin
\pgfsetdash{}{0pt}
\definecolor{dialinecolor}{rgb}{0.678431, 0.839216, 0.905882}
\pgfsetfillcolor{dialinecolor}
\pgfpathmoveto{\pgfpoint{32.948874\du}{14.173592\du}}
\pgfpathlineto{\pgfpoint{32.948874\du}{14.173592\du}}
\pgfpathlineto{\pgfpoint{32.933174\du}{14.173592\du}}
\pgfpathlineto{\pgfpoint{32.917110\du}{14.173957\du}}
\pgfpathlineto{\pgfpoint{32.901046\du}{14.174687\du}}
\pgfpathlineto{\pgfpoint{32.885712\du}{14.175417\du}}
\pgfpathlineto{\pgfpoint{32.869648\du}{14.176513\du}}
\pgfpathlineto{\pgfpoint{32.855044\du}{14.177608\du}}
\pgfpathlineto{\pgfpoint{32.839344\du}{14.178703\du}}
\pgfpathlineto{\pgfpoint{32.823645\du}{14.180529\du}}
\pgfpathlineto{\pgfpoint{32.809041\du}{14.182354\du}}
\pgfpathlineto{\pgfpoint{32.794438\du}{14.184180\du}}
\pgfpathlineto{\pgfpoint{32.779469\du}{14.186370\du}}
\pgfpathlineto{\pgfpoint{32.764500\du}{14.188926\du}}
\pgfpathlineto{\pgfpoint{32.750261\du}{14.191847\du}}
\pgfpathlineto{\pgfpoint{32.736387\du}{14.194037\du}}
\pgfpathlineto{\pgfpoint{32.721783\du}{14.196958\du}}
\pgfpathlineto{\pgfpoint{32.707179\du}{14.199879\du}}
\pgfpathlineto{\pgfpoint{32.693306\du}{14.203530\du}}
\pgfpathlineto{\pgfpoint{32.679797\du}{14.206816\du}}
\pgfpathlineto{\pgfpoint{32.667019\du}{14.210832\du}}
\pgfpathlineto{\pgfpoint{32.653510\du}{14.214483\du}}
\pgfpathlineto{\pgfpoint{32.640002\du}{14.218499\du}}
\pgfpathlineto{\pgfpoint{32.627223\du}{14.222880\du}}
\pgfpathlineto{\pgfpoint{32.614080\du}{14.227261\du}}
\pgfpathlineto{\pgfpoint{32.602032\du}{14.231642\du}}
\pgfpathlineto{\pgfpoint{32.589983\du}{14.236023\du}}
\pgfpathlineto{\pgfpoint{32.577570\du}{14.241500\du}}
\pgfpathlineto{\pgfpoint{32.566252\du}{14.246246\du}}
\pgfpathlineto{\pgfpoint{32.554204\du}{14.251357\du}}
\pgfpathlineto{\pgfpoint{32.542886\du}{14.256469\du}}
\pgfpathlineto{\pgfpoint{32.531568\du}{14.261945\du}}
\pgfpathlineto{\pgfpoint{32.520250\du}{14.267787\du}}
\pgfpathlineto{\pgfpoint{32.509297\du}{14.272898\du}}
\pgfpathlineto{\pgfpoint{32.499074\du}{14.278740\du}}
\pgfpathlineto{\pgfpoint{32.488852\du}{14.285311\du}}
\pgfpathlineto{\pgfpoint{32.478994\du}{14.291153\du}}
\pgfpathlineto{\pgfpoint{32.469501\du}{14.297725\du}}
\pgfpathlineto{\pgfpoint{32.460374\du}{14.303931\du}}
\pgfpathlineto{\pgfpoint{32.450516\du}{14.310503\du}}
\pgfpathlineto{\pgfpoint{32.441389\du}{14.317075\du}}
\pgfpathlineto{\pgfpoint{32.433357\du}{14.324012\du}}
\pgfpathlineto{\pgfpoint{32.424229\du}{14.330949\du}}
\pgfpathlineto{\pgfpoint{32.417293\du}{14.337885\du}}
\pgfpathlineto{\pgfpoint{32.408530\du}{14.344822\du}}
\pgfpathlineto{\pgfpoint{32.401228\du}{14.352124\du}}
\pgfpathlineto{\pgfpoint{32.394292\du}{14.359791\du}}
\pgfpathlineto{\pgfpoint{32.387720\du}{14.367093\du}}
\pgfpathlineto{\pgfpoint{32.380783\du}{14.374760\du}}
\pgfpathlineto{\pgfpoint{32.374211\du}{14.382427\du}}
\pgfpathlineto{\pgfpoint{32.369465\du}{14.390459\du}}
\pgfpathlineto{\pgfpoint{32.363258\du}{14.398126\du}}
\pgfpathlineto{\pgfpoint{32.358147\du}{14.406524\du}}
\pgfpathlineto{\pgfpoint{32.353401\du}{14.414556\du}}
\pgfpathlineto{\pgfpoint{32.349020\du}{14.422953\du}}
\pgfpathlineto{\pgfpoint{32.344638\du}{14.430985\du}}
\pgfpathlineto{\pgfpoint{32.340622\du}{14.439747\du}}
\pgfpathlineto{\pgfpoint{32.337336\du}{14.448145\du}}
\pgfpathlineto{\pgfpoint{32.334051\du}{14.456907\du}}
\pgfpathlineto{\pgfpoint{32.331860\du}{14.465669\du}}
\pgfpathlineto{\pgfpoint{32.329304\du}{14.474432\du}}
\pgfpathlineto{\pgfpoint{32.328209\du}{14.483194\du}}
\pgfpathlineto{\pgfpoint{32.326018\du}{14.491956\du}}
\pgfpathlineto{\pgfpoint{32.324923\du}{14.501084\du}}
\pgfpathlineto{\pgfpoint{32.324193\du}{14.510576\du}}
\pgfpathlineto{\pgfpoint{32.324193\du}{14.519338\du}}
\pgfpathlineto{\pgfpoint{32.344638\du}{14.519338\du}}
\pgfpathlineto{\pgfpoint{32.344638\du}{14.511306\du}}
\pgfpathlineto{\pgfpoint{32.345003\du}{14.502909\du}}
\pgfpathlineto{\pgfpoint{32.346464\du}{14.494877\du}}
\pgfpathlineto{\pgfpoint{32.347559\du}{14.487210\du}}
\pgfpathlineto{\pgfpoint{32.349385\du}{14.478813\du}}
\pgfpathlineto{\pgfpoint{32.351210\du}{14.470781\du}}
\pgfpathlineto{\pgfpoint{32.353766\du}{14.463114\du}}
\pgfpathlineto{\pgfpoint{32.356687\du}{14.455447\du}}
\pgfpathlineto{\pgfpoint{32.359972\du}{14.448145\du}}
\pgfpathlineto{\pgfpoint{32.363258\du}{14.439747\du}}
\pgfpathlineto{\pgfpoint{32.366909\du}{14.432080\du}}
\pgfpathlineto{\pgfpoint{32.371290\du}{14.424413\du}}
\pgfpathlineto{\pgfpoint{32.375672\du}{14.417476\du}}
\pgfpathlineto{\pgfpoint{32.380418\du}{14.409809\du}}
\pgfpathlineto{\pgfpoint{32.385529\du}{14.402507\du}}
\pgfpathlineto{\pgfpoint{32.390641\du}{14.395206\du}}
\pgfpathlineto{\pgfpoint{32.396847\du}{14.387539\du}}
\pgfpathlineto{\pgfpoint{32.403054\du}{14.380602\du}}
\pgfpathlineto{\pgfpoint{32.409261\du}{14.373665\du}}
\pgfpathlineto{\pgfpoint{32.416197\du}{14.366728\du}}
\pgfpathlineto{\pgfpoint{32.422769\du}{14.360156\du}}
\pgfpathlineto{\pgfpoint{32.430436\du}{14.353219\du}}
\pgfpathlineto{\pgfpoint{32.437738\du}{14.346283\du}}
\pgfpathlineto{\pgfpoint{32.445770\du}{14.339711\du}}
\pgfpathlineto{\pgfpoint{32.454167\du}{14.333139\du}}
\pgfpathlineto{\pgfpoint{32.462565\du}{14.326932\du}}
\pgfpathlineto{\pgfpoint{32.471692\du}{14.321091\du}}
\pgfpathlineto{\pgfpoint{32.480089\du}{14.314519\du}}
\pgfpathlineto{\pgfpoint{32.489947\du}{14.308678\du}}
\pgfpathlineto{\pgfpoint{32.499439\du}{14.302836\du}}
\pgfpathlineto{\pgfpoint{32.509297\du}{14.296995\du}}
\pgfpathlineto{\pgfpoint{32.519520\du}{14.291153\du}}
\pgfpathlineto{\pgfpoint{32.529742\du}{14.286042\du}}
\pgfpathlineto{\pgfpoint{32.540695\du}{14.280200\du}}
\pgfpathlineto{\pgfpoint{32.551283\du}{14.275454\du}}
\pgfpathlineto{\pgfpoint{32.562601\du}{14.269977\du}}
\pgfpathlineto{\pgfpoint{32.573554\du}{14.264866\du}}
\pgfpathlineto{\pgfpoint{32.585237\du}{14.260120\du}}
\pgfpathlineto{\pgfpoint{32.596555\du}{14.255373\du}}
\pgfpathlineto{\pgfpoint{32.608968\du}{14.250627\du}}
\pgfpathlineto{\pgfpoint{32.621382\du}{14.246611\du}}
\pgfpathlineto{\pgfpoint{32.633430\du}{14.241865\du}}
\pgfpathlineto{\pgfpoint{32.646208\du}{14.237849\du}}
\pgfpathlineto{\pgfpoint{32.659352\du}{14.234563\du}}
\pgfpathlineto{\pgfpoint{32.672130\du}{14.230182\du}}
\pgfpathlineto{\pgfpoint{32.685639\du}{14.226896\du}}
\pgfpathlineto{\pgfpoint{32.698782\du}{14.223245\du}}
\pgfpathlineto{\pgfpoint{32.712291\du}{14.220324\du}}
\pgfpathlineto{\pgfpoint{32.726164\du}{14.217403\du}}
\pgfpathlineto{\pgfpoint{32.740403\du}{14.214483\du}}
\pgfpathlineto{\pgfpoint{32.754277\du}{14.211562\du}}
\pgfpathlineto{\pgfpoint{32.768151\du}{14.209371\du}}
\pgfpathlineto{\pgfpoint{32.782389\du}{14.206816\du}}
\pgfpathlineto{\pgfpoint{32.796993\du}{14.204625\du}}
\pgfpathlineto{\pgfpoint{32.811597\du}{14.202800\du}}
\pgfpathlineto{\pgfpoint{32.826566\du}{14.200974\du}}
\pgfpathlineto{\pgfpoint{32.841535\du}{14.199149\du}}
\pgfpathlineto{\pgfpoint{32.856504\du}{14.198053\du}}
\pgfpathlineto{\pgfpoint{32.871473\du}{14.196958\du}}
\pgfpathlineto{\pgfpoint{32.887172\du}{14.195863\du}}
\pgfpathlineto{\pgfpoint{32.901776\du}{14.195133\du}}
\pgfpathlineto{\pgfpoint{32.917840\du}{14.194767\du}}
\pgfpathlineto{\pgfpoint{32.933174\du}{14.194037\du}}
\pgfpathlineto{\pgfpoint{32.948874\du}{14.194037\du}}
\pgfpathlineto{\pgfpoint{32.948874\du}{14.194037\du}}
\pgfpathlineto{\pgfpoint{32.948874\du}{14.194037\du}}
\pgfpathlineto{\pgfpoint{32.950334\du}{14.194037\du}}
\pgfpathlineto{\pgfpoint{32.951429\du}{14.194037\du}}
\pgfpathlineto{\pgfpoint{32.952524\du}{14.194037\du}}
\pgfpathlineto{\pgfpoint{32.953620\du}{14.193307\du}}
\pgfpathlineto{\pgfpoint{32.954715\du}{14.192942\du}}
\pgfpathlineto{\pgfpoint{32.955810\du}{14.192212\du}}
\pgfpathlineto{\pgfpoint{32.955810\du}{14.191847\du}}
\pgfpathlineto{\pgfpoint{32.956541\du}{14.191116\du}}
\pgfpathlineto{\pgfpoint{32.957636\du}{14.189291\du}}
\pgfpathlineto{\pgfpoint{32.958731\du}{14.187465\du}}
\pgfpathlineto{\pgfpoint{32.959096\du}{14.186005\du}}
\pgfpathlineto{\pgfpoint{32.959096\du}{14.184180\du}}
\pgfpathlineto{\pgfpoint{32.959096\du}{14.181624\du}}
\pgfpathlineto{\pgfpoint{32.958731\du}{14.180164\du}}
\pgfpathlineto{\pgfpoint{32.957636\du}{14.178338\du}}
\pgfpathlineto{\pgfpoint{32.956541\du}{14.177243\du}}
\pgfpathlineto{\pgfpoint{32.955810\du}{14.175782\du}}
\pgfpathlineto{\pgfpoint{32.955810\du}{14.175417\du}}
\pgfpathlineto{\pgfpoint{32.954715\du}{14.174687\du}}
\pgfpathlineto{\pgfpoint{32.953620\du}{14.174687\du}}
\pgfpathlineto{\pgfpoint{32.952524\du}{14.173957\du}}
\pgfpathlineto{\pgfpoint{32.951429\du}{14.173957\du}}
\pgfpathlineto{\pgfpoint{32.950334\du}{14.173592\du}}
\pgfpathlineto{\pgfpoint{32.948874\du}{14.173592\du}}
\pgfusepath{fill}
\pgfsetbuttcap
\pgfsetmiterjoin
\pgfsetdash{}{0pt}
\definecolor{dialinecolor}{rgb}{0.678431, 0.839216, 0.905882}
\pgfsetfillcolor{dialinecolor}
\pgfpathmoveto{\pgfpoint{33.573919\du}{14.519338\du}}
\pgfpathlineto{\pgfpoint{33.573919\du}{14.509846\du}}
\pgfpathlineto{\pgfpoint{33.573189\du}{14.501084\du}}
\pgfpathlineto{\pgfpoint{33.572094\du}{14.491956\du}}
\pgfpathlineto{\pgfpoint{33.570633\du}{14.483194\du}}
\pgfpathlineto{\pgfpoint{33.568443\du}{14.474432\du}}
\pgfpathlineto{\pgfpoint{33.566617\du}{14.465669\du}}
\pgfpathlineto{\pgfpoint{33.563696\du}{14.456907\du}}
\pgfpathlineto{\pgfpoint{33.561141\du}{14.448145\du}}
\pgfpathlineto{\pgfpoint{33.557125\du}{14.439747\du}}
\pgfpathlineto{\pgfpoint{33.553839\du}{14.430985\du}}
\pgfpathlineto{\pgfpoint{33.549458\du}{14.422953\du}}
\pgfpathlineto{\pgfpoint{33.544711\du}{14.414556\du}}
\pgfpathlineto{\pgfpoint{33.539965\du}{14.406524\du}}
\pgfpathlineto{\pgfpoint{33.535219\du}{14.398126\du}}
\pgfpathlineto{\pgfpoint{33.529377\du}{14.390459\du}}
\pgfpathlineto{\pgfpoint{33.523171\du}{14.382427\du}}
\pgfpathlineto{\pgfpoint{33.516964\du}{14.374760\du}}
\pgfpathlineto{\pgfpoint{33.510392\du}{14.367093\du}}
\pgfpathlineto{\pgfpoint{33.503821\du}{14.359791\du}}
\pgfpathlineto{\pgfpoint{33.496154\du}{14.352124\du}}
\pgfpathlineto{\pgfpoint{33.489217\du}{14.344822\du}}
\pgfpathlineto{\pgfpoint{33.481185\du}{14.337885\du}}
\pgfpathlineto{\pgfpoint{33.473152\du}{14.330949\du}}
\pgfpathlineto{\pgfpoint{33.464755\du}{14.324012\du}}
\pgfpathlineto{\pgfpoint{33.456358\du}{14.317075\du}}
\pgfpathlineto{\pgfpoint{33.447596\du}{14.310503\du}}
\pgfpathlineto{\pgfpoint{33.438468\du}{14.303931\du}}
\pgfpathlineto{\pgfpoint{33.428976\du}{14.297725\du}}
\pgfpathlineto{\pgfpoint{33.418753\du}{14.291153\du}}
\pgfpathlineto{\pgfpoint{33.408895\du}{14.285311\du}}
\pgfpathlineto{\pgfpoint{33.398673\du}{14.278740\du}}
\pgfpathlineto{\pgfpoint{33.388815\du}{14.272898\du}}
\pgfpathlineto{\pgfpoint{33.377497\du}{14.267787\du}}
\pgfpathlineto{\pgfpoint{33.366544\du}{14.261945\du}}
\pgfpathlineto{\pgfpoint{33.355226\du}{14.256469\du}}
\pgfpathlineto{\pgfpoint{33.343908\du}{14.251357\du}}
\pgfpathlineto{\pgfpoint{33.332225\du}{14.246246\du}}
\pgfpathlineto{\pgfpoint{33.320907\du}{14.241500\du}}
\pgfpathlineto{\pgfpoint{33.308129\du}{14.236023\du}}
\pgfpathlineto{\pgfpoint{33.296446\du}{14.231642\du}}
\pgfpathlineto{\pgfpoint{33.283667\du}{14.227261\du}}
\pgfpathlineto{\pgfpoint{33.270889\du}{14.222880\du}}
\pgfpathlineto{\pgfpoint{33.257745\du}{14.218499\du}}
\pgfpathlineto{\pgfpoint{33.244602\du}{14.214483\du}}
\pgfpathlineto{\pgfpoint{33.231458\du}{14.210832\du}}
\pgfpathlineto{\pgfpoint{33.217950\du}{14.206816\du}}
\pgfpathlineto{\pgfpoint{33.204076\du}{14.203530\du}}
\pgfpathlineto{\pgfpoint{33.190568\du}{14.199879\du}}
\pgfpathlineto{\pgfpoint{33.176694\du}{14.196958\du}}
\pgfpathlineto{\pgfpoint{33.162090\du}{14.194037\du}}
\pgfpathlineto{\pgfpoint{33.148216\du}{14.191847\du}}
\pgfpathlineto{\pgfpoint{33.133978\du}{14.188926\du}}
\pgfpathlineto{\pgfpoint{33.118644\du}{14.186370\du}}
\pgfpathlineto{\pgfpoint{33.103675\du}{14.184180\du}}
\pgfpathlineto{\pgfpoint{33.088706\du}{14.182354\du}}
\pgfpathlineto{\pgfpoint{33.074467\du}{14.180529\du}}
\pgfpathlineto{\pgfpoint{33.058403\du}{14.178703\du}}
\pgfpathlineto{\pgfpoint{33.043434\du}{14.177608\du}}
\pgfpathlineto{\pgfpoint{33.028100\du}{14.176513\du}}
\pgfpathlineto{\pgfpoint{33.012400\du}{14.175417\du}}
\pgfpathlineto{\pgfpoint{32.997066\du}{14.174687\du}}
\pgfpathlineto{\pgfpoint{32.981002\du}{14.173957\du}}
\pgfpathlineto{\pgfpoint{32.964938\du}{14.173592\du}}
\pgfpathlineto{\pgfpoint{32.948874\du}{14.173592\du}}
\pgfpathlineto{\pgfpoint{32.948874\du}{14.194037\du}}
\pgfpathlineto{\pgfpoint{32.964938\du}{14.194037\du}}
\pgfpathlineto{\pgfpoint{32.980272\du}{14.194767\du}}
\pgfpathlineto{\pgfpoint{32.995971\du}{14.195133\du}}
\pgfpathlineto{\pgfpoint{33.011305\du}{14.195863\du}}
\pgfpathlineto{\pgfpoint{33.026274\du}{14.196958\du}}
\pgfpathlineto{\pgfpoint{33.041973\du}{14.198053\du}}
\pgfpathlineto{\pgfpoint{33.056577\du}{14.199149\du}}
\pgfpathlineto{\pgfpoint{33.071911\du}{14.200974\du}}
\pgfpathlineto{\pgfpoint{33.086880\du}{14.202800\du}}
\pgfpathlineto{\pgfpoint{33.101484\du}{14.204625\du}}
\pgfpathlineto{\pgfpoint{33.115723\du}{14.206816\du}}
\pgfpathlineto{\pgfpoint{33.129962\du}{14.209371\du}}
\pgfpathlineto{\pgfpoint{33.144200\du}{14.211562\du}}
\pgfpathlineto{\pgfpoint{33.158074\du}{14.214483\du}}
\pgfpathlineto{\pgfpoint{33.171948\du}{14.217403\du}}
\pgfpathlineto{\pgfpoint{33.185821\du}{14.220324\du}}
\pgfpathlineto{\pgfpoint{33.199330\du}{14.223245\du}}
\pgfpathlineto{\pgfpoint{33.212838\du}{14.226896\du}}
\pgfpathlineto{\pgfpoint{33.225982\du}{14.230182\du}}
\pgfpathlineto{\pgfpoint{33.238760\du}{14.234563\du}}
\pgfpathlineto{\pgfpoint{33.252269\du}{14.237849\du}}
\pgfpathlineto{\pgfpoint{33.265047\du}{14.241865\du}}
\pgfpathlineto{\pgfpoint{33.276730\du}{14.246611\du}}
\pgfpathlineto{\pgfpoint{33.289144\du}{14.250627\du}}
\pgfpathlineto{\pgfpoint{33.301192\du}{14.255373\du}}
\pgfpathlineto{\pgfpoint{33.312875\du}{14.260120\du}}
\pgfpathlineto{\pgfpoint{33.324193\du}{14.264866\du}}
\pgfpathlineto{\pgfpoint{33.335511\du}{14.269977\du}}
\pgfpathlineto{\pgfpoint{33.347194\du}{14.275454\du}}
\pgfpathlineto{\pgfpoint{33.357782\du}{14.280200\du}}
\pgfpathlineto{\pgfpoint{33.368370\du}{14.286042\du}}
\pgfpathlineto{\pgfpoint{33.378957\du}{14.291153\du}}
\pgfpathlineto{\pgfpoint{33.388815\du}{14.296995\du}}
\pgfpathlineto{\pgfpoint{33.398673\du}{14.302836\du}}
\pgfpathlineto{\pgfpoint{33.407800\du}{14.308678\du}}
\pgfpathlineto{\pgfpoint{33.416928\du}{14.314519\du}}
\pgfpathlineto{\pgfpoint{33.426420\du}{14.321091\du}}
\pgfpathlineto{\pgfpoint{33.435182\du}{14.326932\du}}
\pgfpathlineto{\pgfpoint{33.444310\du}{14.333139\du}}
\pgfpathlineto{\pgfpoint{33.452342\du}{14.339711\du}}
\pgfpathlineto{\pgfpoint{33.460739\du}{14.346283\du}}
\pgfpathlineto{\pgfpoint{33.467676\du}{14.353219\du}}
\pgfpathlineto{\pgfpoint{33.474978\du}{14.360156\du}}
\pgfpathlineto{\pgfpoint{33.481915\du}{14.366728\du}}
\pgfpathlineto{\pgfpoint{33.489217\du}{14.373665\du}}
\pgfpathlineto{\pgfpoint{33.495058\du}{14.380602\du}}
\pgfpathlineto{\pgfpoint{33.501995\du}{14.387539\du}}
\pgfpathlineto{\pgfpoint{33.507106\du}{14.395206\du}}
\pgfpathlineto{\pgfpoint{33.512583\du}{14.402507\du}}
\pgfpathlineto{\pgfpoint{33.517329\du}{14.409809\du}}
\pgfpathlineto{\pgfpoint{33.522441\du}{14.417476\du}}
\pgfpathlineto{\pgfpoint{33.526457\du}{14.424413\du}}
\pgfpathlineto{\pgfpoint{33.531203\du}{14.432080\du}}
\pgfpathlineto{\pgfpoint{33.535219\du}{14.439747\du}}
\pgfpathlineto{\pgfpoint{33.538505\du}{14.448145\du}}
\pgfpathlineto{\pgfpoint{33.541060\du}{14.455447\du}}
\pgfpathlineto{\pgfpoint{33.544346\du}{14.463114\du}}
\pgfpathlineto{\pgfpoint{33.546537\du}{14.470781\du}}
\pgfpathlineto{\pgfpoint{33.549093\du}{14.478813\du}}
\pgfpathlineto{\pgfpoint{33.550553\du}{14.487210\du}}
\pgfpathlineto{\pgfpoint{33.552013\du}{14.494877\du}}
\pgfpathlineto{\pgfpoint{33.552744\du}{14.502909\du}}
\pgfpathlineto{\pgfpoint{33.553109\du}{14.511306\du}}
\pgfpathlineto{\pgfpoint{33.553839\du}{14.519338\du}}
\pgfpathlineto{\pgfpoint{33.573919\du}{14.519338\du}}
\pgfusepath{fill}
\pgfsetbuttcap
\pgfsetmiterjoin
\pgfsetdash{}{0pt}
\definecolor{dialinecolor}{rgb}{0.074510, 0.082353, 0.086275}
\pgfsetfillcolor{dialinecolor}
\pgfpathmoveto{\pgfpoint{32.627953\du}{14.613533\du}}
\pgfpathlineto{\pgfpoint{32.855409\du}{14.385348\du}}
\pgfpathlineto{\pgfpoint{32.795533\du}{14.324377\du}}
\pgfpathlineto{\pgfpoint{32.975891\du}{14.324377\du}}
\pgfpathlineto{\pgfpoint{32.975891\du}{14.512767\du}}
\pgfpathlineto{\pgfpoint{32.915650\du}{14.452526\du}}
\pgfpathlineto{\pgfpoint{32.695496\du}{14.673774\du}}
\pgfpathlineto{\pgfpoint{32.627953\du}{14.613533\du}}
\pgfusepath{fill}
\pgfsetbuttcap
\pgfsetmiterjoin
\pgfsetdash{}{0pt}
\definecolor{dialinecolor}{rgb}{0.074510, 0.082353, 0.086275}
\pgfsetfillcolor{dialinecolor}
\pgfpathmoveto{\pgfpoint{32.895934\du}{14.734015\du}}
\pgfpathlineto{\pgfpoint{33.123025\du}{14.505830\du}}
\pgfpathlineto{\pgfpoint{33.062419\du}{14.445589\du}}
\pgfpathlineto{\pgfpoint{33.243507\du}{14.445589\du}}
\pgfpathlineto{\pgfpoint{33.243507\du}{14.633614\du}}
\pgfpathlineto{\pgfpoint{33.182901\du}{14.573373\du}}
\pgfpathlineto{\pgfpoint{32.962382\du}{14.794256\du}}
\pgfpathlineto{\pgfpoint{32.895934\du}{14.734015\du}}
\pgfusepath{fill}
\pgfsetbuttcap
\pgfsetmiterjoin
\pgfsetdash{}{0pt}
\definecolor{dialinecolor}{rgb}{1.000000, 1.000000, 1.000000}
\pgfsetfillcolor{dialinecolor}
\pgfpathmoveto{\pgfpoint{32.614810\du}{14.600025\du}}
\pgfpathlineto{\pgfpoint{32.841900\du}{14.371839\du}}
\pgfpathlineto{\pgfpoint{32.782389\du}{14.311598\du}}
\pgfpathlineto{\pgfpoint{32.962382\du}{14.311598\du}}
\pgfpathlineto{\pgfpoint{32.962382\du}{14.499623\du}}
\pgfpathlineto{\pgfpoint{32.902506\du}{14.438652\du}}
\pgfpathlineto{\pgfpoint{32.681988\du}{14.660266\du}}
\pgfpathlineto{\pgfpoint{32.614810\du}{14.600025\du}}
\pgfusepath{fill}
\pgfsetbuttcap
\pgfsetmiterjoin
\pgfsetdash{}{0pt}
\definecolor{dialinecolor}{rgb}{1.000000, 1.000000, 1.000000}
\pgfsetfillcolor{dialinecolor}
\pgfpathmoveto{\pgfpoint{32.882426\du}{14.720507\du}}
\pgfpathlineto{\pgfpoint{33.109516\du}{14.492321\du}}
\pgfpathlineto{\pgfpoint{33.049275\du}{14.432080\du}}
\pgfpathlineto{\pgfpoint{33.229633\du}{14.432080\du}}
\pgfpathlineto{\pgfpoint{33.229633\du}{14.620105\du}}
\pgfpathlineto{\pgfpoint{33.170122\du}{14.559864\du}}
\pgfpathlineto{\pgfpoint{32.948874\du}{14.780748\du}}
\pgfpathlineto{\pgfpoint{32.882426\du}{14.720507\du}}
\pgfusepath{fill}
\pgfsetlinewidth{0.000000\du}
\pgfsetdash{}{0pt}
\pgfsetdash{}{0pt}
\pgfsetbuttcap
\pgfsetmiterjoin
\pgfsetlinewidth{0.000000\du}
\pgfsetbuttcap
\pgfsetmiterjoin
\pgfsetdash{}{0pt}
\definecolor{dialinecolor}{rgb}{0.027451, 0.486275, 0.682353}
\pgfsetfillcolor{dialinecolor}
\pgfpathmoveto{\pgfpoint{44.565909\du}{14.318598\du}}
\pgfpathlineto{\pgfpoint{44.564448\du}{14.347806\du}}
\pgfpathlineto{\pgfpoint{44.557146\du}{14.377744\du}}
\pgfpathlineto{\pgfpoint{44.546924\du}{14.406222\du}}
\pgfpathlineto{\pgfpoint{44.532320\du}{14.434334\du}}
\pgfpathlineto{\pgfpoint{44.512970\du}{14.462446\du}}
\pgfpathlineto{\pgfpoint{44.491064\du}{14.489829\du}}
\pgfpathlineto{\pgfpoint{44.464412\du}{14.516846\du}}
\pgfpathlineto{\pgfpoint{44.433744\du}{14.543133\du}}
\pgfpathlineto{\pgfpoint{44.400885\du}{14.568324\du}}
\pgfpathlineto{\pgfpoint{44.363645\du}{14.593516\du}}
\pgfpathlineto{\pgfpoint{44.322754\du}{14.617613\du}}
\pgfpathlineto{\pgfpoint{44.278943\du}{14.640979\du}}
\pgfpathlineto{\pgfpoint{44.232575\du}{14.663615\du}}
\pgfpathlineto{\pgfpoint{44.182192\du}{14.685521\du}}
\pgfpathlineto{\pgfpoint{44.129253\du}{14.706331\du}}
\pgfpathlineto{\pgfpoint{44.073758\du}{14.726411\du}}
\pgfpathlineto{\pgfpoint{44.015708\du}{14.745761\du}}
\pgfpathlineto{\pgfpoint{43.955102\du}{14.763651\du}}
\pgfpathlineto{\pgfpoint{43.891210\du}{14.780811\du}}
\pgfpathlineto{\pgfpoint{43.826223\du}{14.797240\du}}
\pgfpathlineto{\pgfpoint{43.757584\du}{14.812209\du}}
\pgfpathlineto{\pgfpoint{43.686756\du}{14.825718\du}}
\pgfpathlineto{\pgfpoint{43.614832\du}{14.838496\du}}
\pgfpathlineto{\pgfpoint{43.539987\du}{14.850544\du}}
\pgfpathlineto{\pgfpoint{43.464047\du}{14.860402\du}}
\pgfpathlineto{\pgfpoint{43.385551\du}{14.869529\du}}
\pgfpathlineto{\pgfpoint{43.305960\du}{14.877196\du}}
\pgfpathlineto{\pgfpoint{43.224908\du}{14.883768\du}}
\pgfpathlineto{\pgfpoint{43.141666\du}{14.888879\du}}
\pgfpathlineto{\pgfpoint{43.058059\du}{14.892530\du}}
\pgfpathlineto{\pgfpoint{42.972626\du}{14.894721\du}}
\pgfpathlineto{\pgfpoint{42.886099\du}{14.895451\du}}
\pgfpathlineto{\pgfpoint{42.799936\du}{14.894721\du}}
\pgfpathlineto{\pgfpoint{42.714138\du}{14.892530\du}}
\pgfpathlineto{\pgfpoint{42.630531\du}{14.888879\du}}
\pgfpathlineto{\pgfpoint{42.547654\du}{14.883768\du}}
\pgfpathlineto{\pgfpoint{42.466237\du}{14.877196\du}}
\pgfpathlineto{\pgfpoint{42.386646\du}{14.869529\du}}
\pgfpathlineto{\pgfpoint{42.308881\du}{14.860402\du}}
\pgfpathlineto{\pgfpoint{42.232210\du}{14.850544\du}}
\pgfpathlineto{\pgfpoint{42.158096\du}{14.838496\du}}
\pgfpathlineto{\pgfpoint{42.085441\du}{14.825718\du}}
\pgfpathlineto{\pgfpoint{42.014978\du}{14.812209\du}}
\pgfpathlineto{\pgfpoint{41.946339\du}{14.797240\du}}
\pgfpathlineto{\pgfpoint{41.880622\du}{14.780811\du}}
\pgfpathlineto{\pgfpoint{41.817095\du}{14.763651\du}}
\pgfpathlineto{\pgfpoint{41.756124\du}{14.745761\du}}
\pgfpathlineto{\pgfpoint{41.697709\du}{14.726411\du}}
\pgfpathlineto{\pgfpoint{41.642579\du}{14.706331\du}}
\pgfpathlineto{\pgfpoint{41.589640\du}{14.685521\du}}
\pgfpathlineto{\pgfpoint{41.539622\du}{14.663615\du}}
\pgfpathlineto{\pgfpoint{41.492524\du}{14.640979\du}}
\pgfpathlineto{\pgfpoint{41.449078\du}{14.617613\du}}
\pgfpathlineto{\pgfpoint{41.408187\du}{14.593516\du}}
\pgfpathlineto{\pgfpoint{41.370947\du}{14.568324\du}}
\pgfpathlineto{\pgfpoint{41.337723\du}{14.543133\du}}
\pgfpathlineto{\pgfpoint{41.307420\du}{14.516846\du}}
\pgfpathlineto{\pgfpoint{41.280768\du}{14.489829\du}}
\pgfpathlineto{\pgfpoint{41.258862\du}{14.462446\du}}
\pgfpathlineto{\pgfpoint{41.239512\du}{14.434334\du}}
\pgfpathlineto{\pgfpoint{41.224908\du}{14.406222\du}}
\pgfpathlineto{\pgfpoint{41.214321\du}{14.377744\du}}
\pgfpathlineto{\pgfpoint{41.207384\du}{14.347806\du}}
\pgfpathlineto{\pgfpoint{41.205558\du}{14.318598\du}}
\pgfpathlineto{\pgfpoint{41.207384\du}{14.288660\du}}
\pgfpathlineto{\pgfpoint{41.214321\du}{14.259453\du}}
\pgfpathlineto{\pgfpoint{41.224908\du}{14.230245\du}}
\pgfpathlineto{\pgfpoint{41.239512\du}{14.202132\du}}
\pgfpathlineto{\pgfpoint{41.258862\du}{14.174020\du}}
\pgfpathlineto{\pgfpoint{41.280768\du}{14.146638\du}}
\pgfpathlineto{\pgfpoint{41.307420\du}{14.119986\du}}
\pgfpathlineto{\pgfpoint{41.337723\du}{14.093699\du}}
\pgfpathlineto{\pgfpoint{41.370947\du}{14.068142\du}}
\pgfpathlineto{\pgfpoint{41.408187\du}{14.043315\du}}
\pgfpathlineto{\pgfpoint{41.449078\du}{14.018854\du}}
\pgfpathlineto{\pgfpoint{41.492524\du}{13.995488\du}}
\pgfpathlineto{\pgfpoint{41.539622\du}{13.973217\du}}
\pgfpathlineto{\pgfpoint{41.589640\du}{13.950946\du}}
\pgfpathlineto{\pgfpoint{41.642579\du}{13.930135\du}}
\pgfpathlineto{\pgfpoint{41.697709\du}{13.910055\du}}
\pgfpathlineto{\pgfpoint{41.756124\du}{13.891435\du}}
\pgfpathlineto{\pgfpoint{41.817095\du}{13.872450\du}}
\pgfpathlineto{\pgfpoint{41.880622\du}{13.855656\du}}
\pgfpathlineto{\pgfpoint{41.946339\du}{13.839956\du}}
\pgfpathlineto{\pgfpoint{42.014978\du}{13.824622\du}}
\pgfpathlineto{\pgfpoint{42.085441\du}{13.810749\du}}
\pgfpathlineto{\pgfpoint{42.158096\du}{13.797605\du}}
\pgfpathlineto{\pgfpoint{42.232210\du}{13.786652\du}}
\pgfpathlineto{\pgfpoint{42.308881\du}{13.776065\du}}
\pgfpathlineto{\pgfpoint{42.386646\du}{13.766572\du}}
\pgfpathlineto{\pgfpoint{42.466237\du}{13.759270\du}}
\pgfpathlineto{\pgfpoint{42.547654\du}{13.752698\du}}
\pgfpathlineto{\pgfpoint{42.630531\du}{13.747587\du}}
\pgfpathlineto{\pgfpoint{42.714138\du}{13.743936\du}}
\pgfpathlineto{\pgfpoint{42.799936\du}{13.742111\du}}
\pgfpathlineto{\pgfpoint{42.886099\du}{13.741015\du}}
\pgfpathlineto{\pgfpoint{42.972626\du}{13.742111\du}}
\pgfpathlineto{\pgfpoint{43.058059\du}{13.743936\du}}
\pgfpathlineto{\pgfpoint{43.141666\du}{13.747587\du}}
\pgfpathlineto{\pgfpoint{43.224908\du}{13.752698\du}}
\pgfpathlineto{\pgfpoint{43.305960\du}{13.759270\du}}
\pgfpathlineto{\pgfpoint{43.385551\du}{13.766572\du}}
\pgfpathlineto{\pgfpoint{43.464047\du}{13.776065\du}}
\pgfpathlineto{\pgfpoint{43.539987\du}{13.786652\du}}
\pgfpathlineto{\pgfpoint{43.614832\du}{13.797605\du}}
\pgfpathlineto{\pgfpoint{43.686756\du}{13.810749\du}}
\pgfpathlineto{\pgfpoint{43.757584\du}{13.824622\du}}
\pgfpathlineto{\pgfpoint{43.826223\du}{13.839956\du}}
\pgfpathlineto{\pgfpoint{43.891210\du}{13.855656\du}}
\pgfpathlineto{\pgfpoint{43.955102\du}{13.872450\du}}
\pgfpathlineto{\pgfpoint{44.015708\du}{13.891435\du}}
\pgfpathlineto{\pgfpoint{44.073758\du}{13.910055\du}}
\pgfpathlineto{\pgfpoint{44.129253\du}{13.930135\du}}
\pgfpathlineto{\pgfpoint{44.182192\du}{13.950946\du}}
\pgfpathlineto{\pgfpoint{44.232575\du}{13.973217\du}}
\pgfpathlineto{\pgfpoint{44.278943\du}{13.995488\du}}
\pgfpathlineto{\pgfpoint{44.322754\du}{14.018854\du}}
\pgfpathlineto{\pgfpoint{44.363645\du}{14.043315\du}}
\pgfpathlineto{\pgfpoint{44.400885\du}{14.068142\du}}
\pgfpathlineto{\pgfpoint{44.433744\du}{14.093699\du}}
\pgfpathlineto{\pgfpoint{44.464412\du}{14.119986\du}}
\pgfpathlineto{\pgfpoint{44.491064\du}{14.146638\du}}
\pgfpathlineto{\pgfpoint{44.512970\du}{14.174020\du}}
\pgfpathlineto{\pgfpoint{44.532320\du}{14.202132\du}}
\pgfpathlineto{\pgfpoint{44.546924\du}{14.230245\du}}
\pgfpathlineto{\pgfpoint{44.557146\du}{14.259453\du}}
\pgfpathlineto{\pgfpoint{44.564448\du}{14.288660\du}}
\pgfpathlineto{\pgfpoint{44.565909\du}{14.318598\du}}
\pgfusepath{fill}
\pgfsetlinewidth{0.000000\du}
\pgfsetbuttcap
\pgfsetmiterjoin
\pgfsetdash{}{0pt}
\definecolor{dialinecolor}{rgb}{0.678431, 0.839216, 0.905882}
\pgfsetfillcolor{dialinecolor}
\pgfpathmoveto{\pgfpoint{42.886099\du}{14.906039\du}}
\pgfpathlineto{\pgfpoint{42.886099\du}{14.906039\du}}
\pgfpathlineto{\pgfpoint{42.929545\du}{14.906039\du}}
\pgfpathlineto{\pgfpoint{42.972992\du}{14.905309\du}}
\pgfpathlineto{\pgfpoint{43.016073\du}{14.904213\du}}
\pgfpathlineto{\pgfpoint{43.058059\du}{14.903118\du}}
\pgfpathlineto{\pgfpoint{43.100775\du}{14.901293\du}}
\pgfpathlineto{\pgfpoint{43.142396\du}{14.899102\du}}
\pgfpathlineto{\pgfpoint{43.184017\du}{14.896546\du}}
\pgfpathlineto{\pgfpoint{43.225639\du}{14.894356\du}}
\pgfpathlineto{\pgfpoint{43.266164\du}{14.891435\du}}
\pgfpathlineto{\pgfpoint{43.307055\du}{14.887784\du}}
\pgfpathlineto{\pgfpoint{43.346851\du}{14.883768\du}}
\pgfpathlineto{\pgfpoint{43.387011\du}{14.879752\du}}
\pgfpathlineto{\pgfpoint{43.425712\du}{14.875371\du}}
\pgfpathlineto{\pgfpoint{43.465142\du}{14.870990\du}}
\pgfpathlineto{\pgfpoint{43.503112\du}{14.865513\du}}
\pgfpathlineto{\pgfpoint{43.542177\du}{14.860402\du}}
\pgfpathlineto{\pgfpoint{43.579417\du}{14.854925\du}}
\pgfpathlineto{\pgfpoint{43.616292\du}{14.848719\du}}
\pgfpathlineto{\pgfpoint{43.652802\du}{14.842877\du}}
\pgfpathlineto{\pgfpoint{43.689311\du}{14.836305\du}}
\pgfpathlineto{\pgfpoint{43.724726\du}{14.829369\du}}
\pgfpathlineto{\pgfpoint{43.759410\du}{14.822432\du}}
\pgfpathlineto{\pgfpoint{43.794094\du}{14.814765\du}}
\pgfpathlineto{\pgfpoint{43.828048\du}{14.807098\du}}
\pgfpathlineto{\pgfpoint{43.861637\du}{14.798701\du}}
\pgfpathlineto{\pgfpoint{43.894496\du}{14.790668\du}}
\pgfpathlineto{\pgfpoint{43.926259\du}{14.782636\du}}
\pgfpathlineto{\pgfpoint{43.957657\du}{14.773874\du}}
\pgfpathlineto{\pgfpoint{43.972992\du}{14.769128\du}}
\pgfpathlineto{\pgfpoint{43.988326\du}{14.765112\du}}
\pgfpathlineto{\pgfpoint{44.004390\du}{14.760365\du}}
\pgfpathlineto{\pgfpoint{44.018994\du}{14.755619\du}}
\pgfpathlineto{\pgfpoint{44.033233\du}{14.750873\du}}
\pgfpathlineto{\pgfpoint{44.048201\du}{14.745761\du}}
\pgfpathlineto{\pgfpoint{44.063170\du}{14.741015\du}}
\pgfpathlineto{\pgfpoint{44.077044\du}{14.736269\du}}
\pgfpathlineto{\pgfpoint{44.091283\du}{14.731158\du}}
\pgfpathlineto{\pgfpoint{44.105157\du}{14.726411\du}}
\pgfpathlineto{\pgfpoint{44.119395\du}{14.720935\du}}
\pgfpathlineto{\pgfpoint{44.132539\du}{14.716554\du}}
\pgfpathlineto{\pgfpoint{44.146778\du}{14.711077\du}}
\pgfpathlineto{\pgfpoint{44.159921\du}{14.705966\du}}
\pgfpathlineto{\pgfpoint{44.173065\du}{14.700489\du}}
\pgfpathlineto{\pgfpoint{44.186573\du}{14.694648\du}}
\pgfpathlineto{\pgfpoint{44.199352\du}{14.689537\du}}
\pgfpathlineto{\pgfpoint{44.211400\du}{14.684060\du}}
\pgfpathlineto{\pgfpoint{44.224178\du}{14.678219\du}}
\pgfpathlineto{\pgfpoint{44.236591\du}{14.673107\du}}
\pgfpathlineto{\pgfpoint{44.249005\du}{14.667266\du}}
\pgfpathlineto{\pgfpoint{44.260323\du}{14.661789\du}}
\pgfpathlineto{\pgfpoint{44.272371\du}{14.655948\du}}
\pgfpathlineto{\pgfpoint{44.283324\du}{14.650106\du}}
\pgfpathlineto{\pgfpoint{44.295007\du}{14.644265\du}}
\pgfpathlineto{\pgfpoint{44.306325\du}{14.638423\du}}
\pgfpathlineto{\pgfpoint{44.317643\du}{14.632582\du}}
\pgfpathlineto{\pgfpoint{44.328231\du}{14.626375\du}}
\pgfpathlineto{\pgfpoint{44.338088\du}{14.620533\du}}
\pgfpathlineto{\pgfpoint{44.348676\du}{14.614692\du}}
\pgfpathlineto{\pgfpoint{44.358899\du}{14.608120\du}}
\pgfpathlineto{\pgfpoint{44.368756\du}{14.602278\du}}
\pgfpathlineto{\pgfpoint{44.378614\du}{14.595707\du}}
\pgfpathlineto{\pgfpoint{44.387741\du}{14.589500\du}}
\pgfpathlineto{\pgfpoint{44.396869\du}{14.582928\du}}
\pgfpathlineto{\pgfpoint{44.406361\du}{14.577087\du}}
\pgfpathlineto{\pgfpoint{44.415124\du}{14.570515\du}}
\pgfpathlineto{\pgfpoint{44.424251\du}{14.564308\du}}
\pgfpathlineto{\pgfpoint{44.432283\du}{14.557737\du}}
\pgfpathlineto{\pgfpoint{44.441046\du}{14.550800\du}}
\pgfpathlineto{\pgfpoint{44.448348\du}{14.544228\du}}
\pgfpathlineto{\pgfpoint{44.456380\du}{14.538021\du}}
\pgfpathlineto{\pgfpoint{44.464412\du}{14.530720\du}}
\pgfpathlineto{\pgfpoint{44.470984\du}{14.524513\du}}
\pgfpathlineto{\pgfpoint{44.478285\du}{14.517576\du}}
\pgfpathlineto{\pgfpoint{44.484857\du}{14.510274\du}}
\pgfpathlineto{\pgfpoint{44.491794\du}{14.504067\du}}
\pgfpathlineto{\pgfpoint{44.498366\du}{14.497131\du}}
\pgfpathlineto{\pgfpoint{44.504572\du}{14.489829\du}}
\pgfpathlineto{\pgfpoint{44.510414\du}{14.482892\du}}
\pgfpathlineto{\pgfpoint{44.515890\du}{14.475955\du}}
\pgfpathlineto{\pgfpoint{44.521732\du}{14.469018\du}}
\pgfpathlineto{\pgfpoint{44.526478\du}{14.461716\du}}
\pgfpathlineto{\pgfpoint{44.531955\du}{14.454414\du}}
\pgfpathlineto{\pgfpoint{44.536701\du}{14.447112\du}}
\pgfpathlineto{\pgfpoint{44.541082\du}{14.440176\du}}
\pgfpathlineto{\pgfpoint{44.545098\du}{14.432508\du}}
\pgfpathlineto{\pgfpoint{44.549114\du}{14.425572\du}}
\pgfpathlineto{\pgfpoint{44.552400\du}{14.417905\du}}
\pgfpathlineto{\pgfpoint{44.556416\du}{14.410238\du}}
\pgfpathlineto{\pgfpoint{44.559702\du}{14.402571\du}}
\pgfpathlineto{\pgfpoint{44.562258\du}{14.395269\du}}
\pgfpathlineto{\pgfpoint{44.565178\du}{14.387967\du}}
\pgfpathlineto{\pgfpoint{44.567004\du}{14.380665\du}}
\pgfpathlineto{\pgfpoint{44.569925\du}{14.372998\du}}
\pgfpathlineto{\pgfpoint{44.571020\du}{14.364600\du}}
\pgfpathlineto{\pgfpoint{44.573211\du}{14.356933\du}}
\pgfpathlineto{\pgfpoint{44.574306\du}{14.349632\du}}
\pgfpathlineto{\pgfpoint{44.575036\du}{14.341964\du}}
\pgfpathlineto{\pgfpoint{44.575766\du}{14.333567\du}}
\pgfpathlineto{\pgfpoint{44.576496\du}{14.326265\du}}
\pgfpathlineto{\pgfpoint{44.576496\du}{14.318598\du}}
\pgfpathlineto{\pgfpoint{44.556416\du}{14.318598\du}}
\pgfpathlineto{\pgfpoint{44.555686\du}{14.325535\du}}
\pgfpathlineto{\pgfpoint{44.555686\du}{14.332472\du}}
\pgfpathlineto{\pgfpoint{44.555321\du}{14.339409\du}}
\pgfpathlineto{\pgfpoint{44.553495\du}{14.346711\du}}
\pgfpathlineto{\pgfpoint{44.552400\du}{14.353648\du}}
\pgfpathlineto{\pgfpoint{44.551670\du}{14.360584\du}}
\pgfpathlineto{\pgfpoint{44.549479\du}{14.367521\du}}
\pgfpathlineto{\pgfpoint{44.548019\du}{14.374823\du}}
\pgfpathlineto{\pgfpoint{44.545828\du}{14.381030\du}}
\pgfpathlineto{\pgfpoint{44.543273\du}{14.387967\du}}
\pgfpathlineto{\pgfpoint{44.540352\du}{14.395269\du}}
\pgfpathlineto{\pgfpoint{44.537796\du}{14.402205\du}}
\pgfpathlineto{\pgfpoint{44.533780\du}{14.409142\du}}
\pgfpathlineto{\pgfpoint{44.530859\du}{14.415714\du}}
\pgfpathlineto{\pgfpoint{44.526843\du}{14.422651\du}}
\pgfpathlineto{\pgfpoint{44.523923\du}{14.429223\du}}
\pgfpathlineto{\pgfpoint{44.519541\du}{14.436159\du}}
\pgfpathlineto{\pgfpoint{44.515160\du}{14.443096\du}}
\pgfpathlineto{\pgfpoint{44.510414\du}{14.449668\du}}
\pgfpathlineto{\pgfpoint{44.505303\du}{14.455875\du}}
\pgfpathlineto{\pgfpoint{44.500556\du}{14.463177\du}}
\pgfpathlineto{\pgfpoint{44.494350\du}{14.470113\du}}
\pgfpathlineto{\pgfpoint{44.488873\du}{14.476320\du}}
\pgfpathlineto{\pgfpoint{44.483762\du}{14.482892\du}}
\pgfpathlineto{\pgfpoint{44.477190\du}{14.489464\du}}
\pgfpathlineto{\pgfpoint{44.470253\du}{14.496400\du}}
\pgfpathlineto{\pgfpoint{44.464412\du}{14.502972\du}}
\pgfpathlineto{\pgfpoint{44.457110\du}{14.509179\du}}
\pgfpathlineto{\pgfpoint{44.450538\du}{14.515751\du}}
\pgfpathlineto{\pgfpoint{44.442871\du}{14.521957\du}}
\pgfpathlineto{\pgfpoint{44.435569\du}{14.528529\du}}
\pgfpathlineto{\pgfpoint{44.427537\du}{14.535101\du}}
\pgfpathlineto{\pgfpoint{44.419870\du}{14.541307\du}}
\pgfpathlineto{\pgfpoint{44.411108\du}{14.547879\du}}
\pgfpathlineto{\pgfpoint{44.402710\du}{14.553721\du}}
\pgfpathlineto{\pgfpoint{44.393948\du}{14.560292\du}}
\pgfpathlineto{\pgfpoint{44.385916\du}{14.566499\du}}
\pgfpathlineto{\pgfpoint{44.377154\du}{14.572341\du}}
\pgfpathlineto{\pgfpoint{44.367661\du}{14.578912\du}}
\pgfpathlineto{\pgfpoint{44.357438\du}{14.584754\du}}
\pgfpathlineto{\pgfpoint{44.348311\du}{14.591326\du}}
\pgfpathlineto{\pgfpoint{44.338088\du}{14.597167\du}}
\pgfpathlineto{\pgfpoint{44.328231\du}{14.603009\du}}
\pgfpathlineto{\pgfpoint{44.318373\du}{14.608850\du}}
\pgfpathlineto{\pgfpoint{44.307420\du}{14.614692\du}}
\pgfpathlineto{\pgfpoint{44.296467\du}{14.620533\du}}
\pgfpathlineto{\pgfpoint{44.286245\du}{14.626375\du}}
\pgfpathlineto{\pgfpoint{44.274196\du}{14.632216\du}}
\pgfpathlineto{\pgfpoint{44.263609\du}{14.637328\du}}
\pgfpathlineto{\pgfpoint{44.251560\du}{14.643169\du}}
\pgfpathlineto{\pgfpoint{44.240242\du}{14.649011\du}}
\pgfpathlineto{\pgfpoint{44.227829\du}{14.654487\du}}
\pgfpathlineto{\pgfpoint{44.215781\du}{14.659599\du}}
\pgfpathlineto{\pgfpoint{44.203368\du}{14.665440\du}}
\pgfpathlineto{\pgfpoint{44.191319\du}{14.670917\du}}
\pgfpathlineto{\pgfpoint{44.178176\du}{14.676028\du}}
\pgfpathlineto{\pgfpoint{44.165398\du}{14.681139\du}}
\pgfpathlineto{\pgfpoint{44.152619\du}{14.686616\du}}
\pgfpathlineto{\pgfpoint{44.139476\du}{14.691727\du}}
\pgfpathlineto{\pgfpoint{44.125967\du}{14.697204\du}}
\pgfpathlineto{\pgfpoint{44.112824\du}{14.701585\du}}
\pgfpathlineto{\pgfpoint{44.098585\du}{14.707061\du}}
\pgfpathlineto{\pgfpoint{44.084711\du}{14.711807\du}}
\pgfpathlineto{\pgfpoint{44.070472\du}{14.716919\du}}
\pgfpathlineto{\pgfpoint{44.055869\du}{14.721665\du}}
\pgfpathlineto{\pgfpoint{44.041995\du}{14.726411\du}}
\pgfpathlineto{\pgfpoint{44.027391\du}{14.731158\du}}
\pgfpathlineto{\pgfpoint{44.013152\du}{14.735539\du}}
\pgfpathlineto{\pgfpoint{43.997453\du}{14.740285\du}}
\pgfpathlineto{\pgfpoint{43.983214\du}{14.745031\du}}
\pgfpathlineto{\pgfpoint{43.967515\du}{14.749778\du}}
\pgfpathlineto{\pgfpoint{43.952546\du}{14.753794\du}}
\pgfpathlineto{\pgfpoint{43.921148\du}{14.762556\du}}
\pgfpathlineto{\pgfpoint{43.889019\du}{14.770953\du}}
\pgfpathlineto{\pgfpoint{43.856891\du}{14.778985\du}}
\pgfpathlineto{\pgfpoint{43.823667\du}{14.787017\du}}
\pgfpathlineto{\pgfpoint{43.789348\du}{14.794684\du}}
\pgfpathlineto{\pgfpoint{43.755394\du}{14.801986\du}}
\pgfpathlineto{\pgfpoint{43.720710\du}{14.808923\du}}
\pgfpathlineto{\pgfpoint{43.684930\du}{14.815860\du}}
\pgfpathlineto{\pgfpoint{43.649516\du}{14.822432\du}}
\pgfpathlineto{\pgfpoint{43.612641\du}{14.828638\du}}
\pgfpathlineto{\pgfpoint{43.576131\du}{14.834480\du}}
\pgfpathlineto{\pgfpoint{43.538892\du}{14.840322\du}}
\pgfpathlineto{\pgfpoint{43.501287\du}{14.845798\du}}
\pgfpathlineto{\pgfpoint{43.462586\du}{14.850544\du}}
\pgfpathlineto{\pgfpoint{43.423886\du}{14.854925\du}}
\pgfpathlineto{\pgfpoint{43.384456\du}{14.859672\du}}
\pgfpathlineto{\pgfpoint{43.345390\du}{14.863323\du}}
\pgfpathlineto{\pgfpoint{43.305230\du}{14.867339\du}}
\pgfpathlineto{\pgfpoint{43.265069\du}{14.870259\du}}
\pgfpathlineto{\pgfpoint{43.223813\du}{14.873910\du}}
\pgfpathlineto{\pgfpoint{43.182922\du}{14.876831\du}}
\pgfpathlineto{\pgfpoint{43.141666\du}{14.879022\du}}
\pgfpathlineto{\pgfpoint{43.100045\du}{14.880847\du}}
\pgfpathlineto{\pgfpoint{43.057329\du}{14.882673\du}}
\pgfpathlineto{\pgfpoint{43.014613\du}{14.883768\du}}
\pgfpathlineto{\pgfpoint{42.972626\du}{14.884863\du}}
\pgfpathlineto{\pgfpoint{42.929180\du}{14.884863\du}}
\pgfpathlineto{\pgfpoint{42.886099\du}{14.885594\du}}
\pgfpathlineto{\pgfpoint{42.886099\du}{14.885594\du}}
\pgfpathlineto{\pgfpoint{42.886099\du}{14.885594\du}}
\pgfpathlineto{\pgfpoint{42.885368\du}{14.885594\du}}
\pgfpathlineto{\pgfpoint{42.883543\du}{14.885594\du}}
\pgfpathlineto{\pgfpoint{42.882448\du}{14.885959\du}}
\pgfpathlineto{\pgfpoint{42.881717\du}{14.885959\du}}
\pgfpathlineto{\pgfpoint{42.881352\du}{14.886689\du}}
\pgfpathlineto{\pgfpoint{42.879892\du}{14.887054\du}}
\pgfpathlineto{\pgfpoint{42.879162\du}{14.887784\du}}
\pgfpathlineto{\pgfpoint{42.878431\du}{14.888514\du}}
\pgfpathlineto{\pgfpoint{42.877336\du}{14.890340\du}}
\pgfpathlineto{\pgfpoint{42.876606\du}{14.891800\du}}
\pgfpathlineto{\pgfpoint{42.876606\du}{14.893626\du}}
\pgfpathlineto{\pgfpoint{42.875876\du}{14.895451\du}}
\pgfpathlineto{\pgfpoint{42.876606\du}{14.897642\du}}
\pgfpathlineto{\pgfpoint{42.876606\du}{14.899467\du}}
\pgfpathlineto{\pgfpoint{42.877336\du}{14.901293\du}}
\pgfpathlineto{\pgfpoint{42.878431\du}{14.903118\du}}
\pgfpathlineto{\pgfpoint{42.879162\du}{14.903483\du}}
\pgfpathlineto{\pgfpoint{42.879892\du}{14.904213\du}}
\pgfpathlineto{\pgfpoint{42.881352\du}{14.904944\du}}
\pgfpathlineto{\pgfpoint{42.881717\du}{14.905309\du}}
\pgfpathlineto{\pgfpoint{42.882448\du}{14.905309\du}}
\pgfpathlineto{\pgfpoint{42.883543\du}{14.906039\du}}
\pgfpathlineto{\pgfpoint{42.885368\du}{14.906039\du}}
\pgfpathlineto{\pgfpoint{42.886099\du}{14.906039\du}}
\pgfusepath{fill}
\pgfsetbuttcap
\pgfsetmiterjoin
\pgfsetdash{}{0pt}
\definecolor{dialinecolor}{rgb}{0.678431, 0.839216, 0.905882}
\pgfsetfillcolor{dialinecolor}
\pgfpathmoveto{\pgfpoint{41.195335\du}{14.318598\du}}
\pgfpathlineto{\pgfpoint{41.195335\du}{14.318598\du}}
\pgfpathlineto{\pgfpoint{41.195335\du}{14.326265\du}}
\pgfpathlineto{\pgfpoint{41.195701\du}{14.333567\du}}
\pgfpathlineto{\pgfpoint{41.196431\du}{14.341964\du}}
\pgfpathlineto{\pgfpoint{41.197526\du}{14.349632\du}}
\pgfpathlineto{\pgfpoint{41.198621\du}{14.356933\du}}
\pgfpathlineto{\pgfpoint{41.200447\du}{14.364600\du}}
\pgfpathlineto{\pgfpoint{41.202272\du}{14.372998\du}}
\pgfpathlineto{\pgfpoint{41.204463\du}{14.380665\du}}
\pgfpathlineto{\pgfpoint{41.206653\du}{14.387967\du}}
\pgfpathlineto{\pgfpoint{41.209209\du}{14.395269\du}}
\pgfpathlineto{\pgfpoint{41.212130\du}{14.402571\du}}
\pgfpathlineto{\pgfpoint{41.215781\du}{14.410238\du}}
\pgfpathlineto{\pgfpoint{41.219067\du}{14.417905\du}}
\pgfpathlineto{\pgfpoint{41.222718\du}{14.425572\du}}
\pgfpathlineto{\pgfpoint{41.227099\du}{14.432508\du}}
\pgfpathlineto{\pgfpoint{41.230750\du}{14.440176\du}}
\pgfpathlineto{\pgfpoint{41.235861\du}{14.447112\du}}
\pgfpathlineto{\pgfpoint{41.239877\du}{14.454414\du}}
\pgfpathlineto{\pgfpoint{41.245354\du}{14.461716\du}}
\pgfpathlineto{\pgfpoint{41.250100\du}{14.469018\du}}
\pgfpathlineto{\pgfpoint{41.255576\du}{14.475955\du}}
\pgfpathlineto{\pgfpoint{41.261418\du}{14.482892\du}}
\pgfpathlineto{\pgfpoint{41.267260\du}{14.489829\du}}
\pgfpathlineto{\pgfpoint{41.273101\du}{14.497131\du}}
\pgfpathlineto{\pgfpoint{41.280038\du}{14.504067\du}}
\pgfpathlineto{\pgfpoint{41.286610\du}{14.510274\du}}
\pgfpathlineto{\pgfpoint{41.293546\du}{14.517576\du}}
\pgfpathlineto{\pgfpoint{41.300483\du}{14.524513\du}}
\pgfpathlineto{\pgfpoint{41.307420\du}{14.530720\du}}
\pgfpathlineto{\pgfpoint{41.316182\du}{14.538021\du}}
\pgfpathlineto{\pgfpoint{41.323119\du}{14.544228\du}}
\pgfpathlineto{\pgfpoint{41.330786\du}{14.550800\du}}
\pgfpathlineto{\pgfpoint{41.339549\du}{14.557737\du}}
\pgfpathlineto{\pgfpoint{41.347946\du}{14.564308\du}}
\pgfpathlineto{\pgfpoint{41.356708\du}{14.570515\du}}
\pgfpathlineto{\pgfpoint{41.365105\du}{14.577087\du}}
\pgfpathlineto{\pgfpoint{41.374963\du}{14.582928\du}}
\pgfpathlineto{\pgfpoint{41.384090\du}{14.589500\du}}
\pgfpathlineto{\pgfpoint{41.393583\du}{14.595707\du}}
\pgfpathlineto{\pgfpoint{41.403076\du}{14.602278\du}}
\pgfpathlineto{\pgfpoint{41.412933\du}{14.608120\du}}
\pgfpathlineto{\pgfpoint{41.423156\du}{14.614692\du}}
\pgfpathlineto{\pgfpoint{41.433379\du}{14.620533\du}}
\pgfpathlineto{\pgfpoint{41.443966\du}{14.626375\du}}
\pgfpathlineto{\pgfpoint{41.454189\du}{14.632582\du}}
\pgfpathlineto{\pgfpoint{41.465142\du}{14.638423\du}}
\pgfpathlineto{\pgfpoint{41.476825\du}{14.644265\du}}
\pgfpathlineto{\pgfpoint{41.488143\du}{14.650106\du}}
\pgfpathlineto{\pgfpoint{41.499826\du}{14.655948\du}}
\pgfpathlineto{\pgfpoint{41.511144\du}{14.661789\du}}
\pgfpathlineto{\pgfpoint{41.522827\du}{14.667266\du}}
\pgfpathlineto{\pgfpoint{41.535241\du}{14.673107\du}}
\pgfpathlineto{\pgfpoint{41.547289\du}{14.678219\du}}
\pgfpathlineto{\pgfpoint{41.560432\du}{14.684060\du}}
\pgfpathlineto{\pgfpoint{41.572480\du}{14.689537\du}}
\pgfpathlineto{\pgfpoint{41.585259\du}{14.694648\du}}
\pgfpathlineto{\pgfpoint{41.599132\du}{14.700489\du}}
\pgfpathlineto{\pgfpoint{41.611546\du}{14.705966\du}}
\pgfpathlineto{\pgfpoint{41.624689\du}{14.711077\du}}
\pgfpathlineto{\pgfpoint{41.638928\du}{14.716554\du}}
\pgfpathlineto{\pgfpoint{41.652072\du}{14.720935\du}}
\pgfpathlineto{\pgfpoint{41.666310\du}{14.726411\du}}
\pgfpathlineto{\pgfpoint{41.680184\du}{14.731158\du}}
\pgfpathlineto{\pgfpoint{41.694788\du}{14.736269\du}}
\pgfpathlineto{\pgfpoint{41.708662\du}{14.741015\du}}
\pgfpathlineto{\pgfpoint{41.724361\du}{14.745761\du}}
\pgfpathlineto{\pgfpoint{41.738234\du}{14.750873\du}}
\pgfpathlineto{\pgfpoint{41.752838\du}{14.755619\du}}
\pgfpathlineto{\pgfpoint{41.768172\du}{14.760365\du}}
\pgfpathlineto{\pgfpoint{41.783871\du}{14.765112\du}}
\pgfpathlineto{\pgfpoint{41.798840\du}{14.769128\du}}
\pgfpathlineto{\pgfpoint{41.814540\du}{14.773874\du}}
\pgfpathlineto{\pgfpoint{41.846303\du}{14.782636\du}}
\pgfpathlineto{\pgfpoint{41.878066\du}{14.790668\du}}
\pgfpathlineto{\pgfpoint{41.911290\du}{14.798701\du}}
\pgfpathlineto{\pgfpoint{41.943784\du}{14.807098\du}}
\pgfpathlineto{\pgfpoint{41.978468\du}{14.814765\du}}
\pgfpathlineto{\pgfpoint{42.012787\du}{14.822432\du}}
\pgfpathlineto{\pgfpoint{42.047471\du}{14.829369\du}}
\pgfpathlineto{\pgfpoint{42.083251\du}{14.836305\du}}
\pgfpathlineto{\pgfpoint{42.119395\du}{14.842877\du}}
\pgfpathlineto{\pgfpoint{42.155905\du}{14.848719\du}}
\pgfpathlineto{\pgfpoint{42.193145\du}{14.854925\du}}
\pgfpathlineto{\pgfpoint{42.230750\du}{14.860402\du}}
\pgfpathlineto{\pgfpoint{42.268720\du}{14.865513\du}}
\pgfpathlineto{\pgfpoint{42.307420\du}{14.870990\du}}
\pgfpathlineto{\pgfpoint{42.346120\du}{14.875371\du}}
\pgfpathlineto{\pgfpoint{42.385186\du}{14.879752\du}}
\pgfpathlineto{\pgfpoint{42.425346\du}{14.883768\du}}
\pgfpathlineto{\pgfpoint{42.465142\du}{14.887784\du}}
\pgfpathlineto{\pgfpoint{42.506033\du}{14.891435\du}}
\pgfpathlineto{\pgfpoint{42.546924\du}{14.894356\du}}
\pgfpathlineto{\pgfpoint{42.588180\du}{14.896546\du}}
\pgfpathlineto{\pgfpoint{42.630166\du}{14.899102\du}}
\pgfpathlineto{\pgfpoint{42.671787\du}{14.901293\du}}
\pgfpathlineto{\pgfpoint{42.714138\du}{14.903118\du}}
\pgfpathlineto{\pgfpoint{42.756124\du}{14.904213\du}}
\pgfpathlineto{\pgfpoint{42.799571\du}{14.905309\du}}
\pgfpathlineto{\pgfpoint{42.842287\du}{14.906039\du}}
\pgfpathlineto{\pgfpoint{42.886099\du}{14.906039\du}}
\pgfpathlineto{\pgfpoint{42.886099\du}{14.885594\du}}
\pgfpathlineto{\pgfpoint{42.843382\du}{14.884863\du}}
\pgfpathlineto{\pgfpoint{42.799936\du}{14.884863\du}}
\pgfpathlineto{\pgfpoint{42.757584\du}{14.883768\du}}
\pgfpathlineto{\pgfpoint{42.714868\du}{14.882673\du}}
\pgfpathlineto{\pgfpoint{42.672517\du}{14.880847\du}}
\pgfpathlineto{\pgfpoint{42.630531\du}{14.879022\du}}
\pgfpathlineto{\pgfpoint{42.589640\du}{14.876831\du}}
\pgfpathlineto{\pgfpoint{42.548384\du}{14.873910\du}}
\pgfpathlineto{\pgfpoint{42.507493\du}{14.870259\du}}
\pgfpathlineto{\pgfpoint{42.467333\du}{14.867339\du}}
\pgfpathlineto{\pgfpoint{42.427537\du}{14.863323\du}}
\pgfpathlineto{\pgfpoint{42.388107\du}{14.859672\du}}
\pgfpathlineto{\pgfpoint{42.348676\du}{14.854925\du}}
\pgfpathlineto{\pgfpoint{42.309611\du}{14.850544\du}}
\pgfpathlineto{\pgfpoint{42.271276\du}{14.845798\du}}
\pgfpathlineto{\pgfpoint{42.233671\du}{14.840322\du}}
\pgfpathlineto{\pgfpoint{42.196431\du}{14.834480\du}}
\pgfpathlineto{\pgfpoint{42.159921\du}{14.828638\du}}
\pgfpathlineto{\pgfpoint{42.122681\du}{14.822432\du}}
\pgfpathlineto{\pgfpoint{42.087632\du}{14.815860\du}}
\pgfpathlineto{\pgfpoint{42.051487\du}{14.808923\du}}
\pgfpathlineto{\pgfpoint{42.016803\du}{14.801986\du}}
\pgfpathlineto{\pgfpoint{41.982484\du}{14.794684\du}}
\pgfpathlineto{\pgfpoint{41.948530\du}{14.787017\du}}
\pgfpathlineto{\pgfpoint{41.915306\du}{14.778985\du}}
\pgfpathlineto{\pgfpoint{41.883908\du}{14.770953\du}}
\pgfpathlineto{\pgfpoint{41.851414\du}{14.762556\du}}
\pgfpathlineto{\pgfpoint{41.820381\du}{14.753794\du}}
\pgfpathlineto{\pgfpoint{41.804682\du}{14.749778\du}}
\pgfpathlineto{\pgfpoint{41.788983\du}{14.745031\du}}
\pgfpathlineto{\pgfpoint{41.774379\du}{14.740285\du}}
\pgfpathlineto{\pgfpoint{41.759410\du}{14.735539\du}}
\pgfpathlineto{\pgfpoint{41.744076\du}{14.731158\du}}
\pgfpathlineto{\pgfpoint{41.729837\du}{14.726411\du}}
\pgfpathlineto{\pgfpoint{41.715598\du}{14.721665\du}}
\pgfpathlineto{\pgfpoint{41.701360\du}{14.716919\du}}
\pgfpathlineto{\pgfpoint{41.687121\du}{14.711807\du}}
\pgfpathlineto{\pgfpoint{41.673247\du}{14.707061\du}}
\pgfpathlineto{\pgfpoint{41.659373\du}{14.701585\du}}
\pgfpathlineto{\pgfpoint{41.645865\du}{14.697204\du}}
\pgfpathlineto{\pgfpoint{41.632356\du}{14.691727\du}}
\pgfpathlineto{\pgfpoint{41.619578\du}{14.686616\du}}
\pgfpathlineto{\pgfpoint{41.606069\du}{14.681139\du}}
\pgfpathlineto{\pgfpoint{41.593291\du}{14.676028\du}}
\pgfpathlineto{\pgfpoint{41.580878\du}{14.670917\du}}
\pgfpathlineto{\pgfpoint{41.567734\du}{14.665440\du}}
\pgfpathlineto{\pgfpoint{41.555686\du}{14.659599\du}}
\pgfpathlineto{\pgfpoint{41.544368\du}{14.654487\du}}
\pgfpathlineto{\pgfpoint{41.531590\du}{14.649011\du}}
\pgfpathlineto{\pgfpoint{41.519906\du}{14.643169\du}}
\pgfpathlineto{\pgfpoint{41.508223\du}{14.637328\du}}
\pgfpathlineto{\pgfpoint{41.497270\du}{14.632216\du}}
\pgfpathlineto{\pgfpoint{41.485587\du}{14.626375\du}}
\pgfpathlineto{\pgfpoint{41.475730\du}{14.620533\du}}
\pgfpathlineto{\pgfpoint{41.464412\du}{14.614692\du}}
\pgfpathlineto{\pgfpoint{41.453824\du}{14.608850\du}}
\pgfpathlineto{\pgfpoint{41.443966\du}{14.603009\du}}
\pgfpathlineto{\pgfpoint{41.433379\du}{14.597167\du}}
\pgfpathlineto{\pgfpoint{41.423521\du}{14.591326\du}}
\pgfpathlineto{\pgfpoint{41.414028\du}{14.584754\du}}
\pgfpathlineto{\pgfpoint{41.404171\du}{14.578912\du}}
\pgfpathlineto{\pgfpoint{41.395043\du}{14.572341\du}}
\pgfpathlineto{\pgfpoint{41.386646\du}{14.566499\du}}
\pgfpathlineto{\pgfpoint{41.377884\du}{14.560292\du}}
\pgfpathlineto{\pgfpoint{41.368756\du}{14.553721\du}}
\pgfpathlineto{\pgfpoint{41.359994\du}{14.547879\du}}
\pgfpathlineto{\pgfpoint{41.351597\du}{14.541307\du}}
\pgfpathlineto{\pgfpoint{41.344295\du}{14.535101\du}}
\pgfpathlineto{\pgfpoint{41.336263\du}{14.528529\du}}
\pgfpathlineto{\pgfpoint{41.328596\du}{14.521957\du}}
\pgfpathlineto{\pgfpoint{41.321659\du}{14.515751\du}}
\pgfpathlineto{\pgfpoint{41.314357\du}{14.509179\du}}
\pgfpathlineto{\pgfpoint{41.307420\du}{14.502972\du}}
\pgfpathlineto{\pgfpoint{41.301214\du}{14.496400\du}}
\pgfpathlineto{\pgfpoint{41.294642\du}{14.489464\du}}
\pgfpathlineto{\pgfpoint{41.288800\du}{14.482892\du}}
\pgfpathlineto{\pgfpoint{41.282594\du}{14.476320\du}}
\pgfpathlineto{\pgfpoint{41.277482\du}{14.470113\du}}
\pgfpathlineto{\pgfpoint{41.271276\du}{14.463177\du}}
\pgfpathlineto{\pgfpoint{41.266894\du}{14.456605\du}}
\pgfpathlineto{\pgfpoint{41.261418\du}{14.449668\du}}
\pgfpathlineto{\pgfpoint{41.257037\du}{14.443096\du}}
\pgfpathlineto{\pgfpoint{41.252656\du}{14.436159\du}}
\pgfpathlineto{\pgfpoint{41.248274\du}{14.429223\du}}
\pgfpathlineto{\pgfpoint{41.244989\du}{14.422651\du}}
\pgfpathlineto{\pgfpoint{41.240973\du}{14.415714\du}}
\pgfpathlineto{\pgfpoint{41.236956\du}{14.409142\du}}
\pgfpathlineto{\pgfpoint{41.234036\du}{14.402205\du}}
\pgfpathlineto{\pgfpoint{41.231480\du}{14.395269\du}}
\pgfpathlineto{\pgfpoint{41.228194\du}{14.387967\du}}
\pgfpathlineto{\pgfpoint{41.226004\du}{14.381030\du}}
\pgfpathlineto{\pgfpoint{41.223813\du}{14.374823\du}}
\pgfpathlineto{\pgfpoint{41.222353\du}{14.367521\du}}
\pgfpathlineto{\pgfpoint{41.220162\du}{14.360584\du}}
\pgfpathlineto{\pgfpoint{41.219067\du}{14.353648\du}}
\pgfpathlineto{\pgfpoint{41.217971\du}{14.346711\du}}
\pgfpathlineto{\pgfpoint{41.216511\du}{14.339409\du}}
\pgfpathlineto{\pgfpoint{41.216146\du}{14.332472\du}}
\pgfpathlineto{\pgfpoint{41.216146\du}{14.325535\du}}
\pgfpathlineto{\pgfpoint{41.215781\du}{14.318598\du}}
\pgfpathlineto{\pgfpoint{41.215781\du}{14.318598\du}}
\pgfpathlineto{\pgfpoint{41.215781\du}{14.318598\du}}
\pgfpathlineto{\pgfpoint{41.215781\du}{14.316773\du}}
\pgfpathlineto{\pgfpoint{41.215781\du}{14.316043\du}}
\pgfpathlineto{\pgfpoint{41.215416\du}{14.314947\du}}
\pgfpathlineto{\pgfpoint{41.215416\du}{14.313852\du}}
\pgfpathlineto{\pgfpoint{41.214321\du}{14.313122\du}}
\pgfpathlineto{\pgfpoint{41.213955\du}{14.312027\du}}
\pgfpathlineto{\pgfpoint{41.213590\du}{14.311296\du}}
\pgfpathlineto{\pgfpoint{41.212495\du}{14.310931\du}}
\pgfpathlineto{\pgfpoint{41.211035\du}{14.309836\du}}
\pgfpathlineto{\pgfpoint{41.209209\du}{14.308376\du}}
\pgfpathlineto{\pgfpoint{41.207384\du}{14.308010\du}}
\pgfpathlineto{\pgfpoint{41.205558\du}{14.308010\du}}
\pgfpathlineto{\pgfpoint{41.203368\du}{14.308010\du}}
\pgfpathlineto{\pgfpoint{41.201907\du}{14.308376\du}}
\pgfpathlineto{\pgfpoint{41.199717\du}{14.309836\du}}
\pgfpathlineto{\pgfpoint{41.197891\du}{14.310931\du}}
\pgfpathlineto{\pgfpoint{41.197526\du}{14.311296\du}}
\pgfpathlineto{\pgfpoint{41.197161\du}{14.312027\du}}
\pgfpathlineto{\pgfpoint{41.196431\du}{14.313122\du}}
\pgfpathlineto{\pgfpoint{41.195701\du}{14.313852\du}}
\pgfpathlineto{\pgfpoint{41.195701\du}{14.314947\du}}
\pgfpathlineto{\pgfpoint{41.195335\du}{14.316043\du}}
\pgfpathlineto{\pgfpoint{41.195335\du}{14.316773\du}}
\pgfpathlineto{\pgfpoint{41.195335\du}{14.318598\du}}
\pgfusepath{fill}
\pgfsetbuttcap
\pgfsetmiterjoin
\pgfsetdash{}{0pt}
\definecolor{dialinecolor}{rgb}{0.678431, 0.839216, 0.905882}
\pgfsetfillcolor{dialinecolor}
\pgfpathmoveto{\pgfpoint{42.886099\du}{13.731158\du}}
\pgfpathlineto{\pgfpoint{42.886099\du}{13.731158\du}}
\pgfpathlineto{\pgfpoint{42.842287\du}{13.731158\du}}
\pgfpathlineto{\pgfpoint{42.799571\du}{13.731523\du}}
\pgfpathlineto{\pgfpoint{42.756124\du}{13.732618\du}}
\pgfpathlineto{\pgfpoint{42.714138\du}{13.734078\du}}
\pgfpathlineto{\pgfpoint{42.671787\du}{13.735539\du}}
\pgfpathlineto{\pgfpoint{42.630166\du}{13.737364\du}}
\pgfpathlineto{\pgfpoint{42.588180\du}{13.739920\du}}
\pgfpathlineto{\pgfpoint{42.546924\du}{13.742841\du}}
\pgfpathlineto{\pgfpoint{42.506033\du}{13.745761\du}}
\pgfpathlineto{\pgfpoint{42.465142\du}{13.748682\du}}
\pgfpathlineto{\pgfpoint{42.425346\du}{13.752698\du}}
\pgfpathlineto{\pgfpoint{42.385186\du}{13.756714\du}}
\pgfpathlineto{\pgfpoint{42.346120\du}{13.761461\du}}
\pgfpathlineto{\pgfpoint{42.307420\du}{13.766207\du}}
\pgfpathlineto{\pgfpoint{42.268720\du}{13.770953\du}}
\pgfpathlineto{\pgfpoint{42.230750\du}{13.776065\du}}
\pgfpathlineto{\pgfpoint{42.193145\du}{13.781906\du}}
\pgfpathlineto{\pgfpoint{42.155905\du}{13.787748\du}}
\pgfpathlineto{\pgfpoint{42.119395\du}{13.794319\du}}
\pgfpathlineto{\pgfpoint{42.083251\du}{13.800526\du}}
\pgfpathlineto{\pgfpoint{42.047471\du}{13.807828\du}}
\pgfpathlineto{\pgfpoint{42.012787\du}{13.814765\du}}
\pgfpathlineto{\pgfpoint{41.978468\du}{13.821702\du}}
\pgfpathlineto{\pgfpoint{41.943784\du}{13.829734\du}}
\pgfpathlineto{\pgfpoint{41.911290\du}{13.837401\du}}
\pgfpathlineto{\pgfpoint{41.878066\du}{13.845798\du}}
\pgfpathlineto{\pgfpoint{41.846303\du}{13.854560\du}}
\pgfpathlineto{\pgfpoint{41.814540\du}{13.863323\du}}
\pgfpathlineto{\pgfpoint{41.783871\du}{13.872085\du}}
\pgfpathlineto{\pgfpoint{41.752838\du}{13.881577\du}}
\pgfpathlineto{\pgfpoint{41.738234\du}{13.885959\du}}
\pgfpathlineto{\pgfpoint{41.724361\du}{13.890705\du}}
\pgfpathlineto{\pgfpoint{41.708662\du}{13.895451\du}}
\pgfpathlineto{\pgfpoint{41.694788\du}{13.900563\du}}
\pgfpathlineto{\pgfpoint{41.680184\du}{13.905309\du}}
\pgfpathlineto{\pgfpoint{41.666310\du}{13.910785\du}}
\pgfpathlineto{\pgfpoint{41.652072\du}{13.915166\du}}
\pgfpathlineto{\pgfpoint{41.638928\du}{13.920643\du}}
\pgfpathlineto{\pgfpoint{41.624689\du}{13.925754\du}}
\pgfpathlineto{\pgfpoint{41.611546\du}{13.931231\du}}
\pgfpathlineto{\pgfpoint{41.599132\du}{13.936342\du}}
\pgfpathlineto{\pgfpoint{41.585259\du}{13.941818\du}}
\pgfpathlineto{\pgfpoint{41.572480\du}{13.946930\du}}
\pgfpathlineto{\pgfpoint{41.560432\du}{13.952041\du}}
\pgfpathlineto{\pgfpoint{41.547289\du}{13.957883\du}}
\pgfpathlineto{\pgfpoint{41.535241\du}{13.964089\du}}
\pgfpathlineto{\pgfpoint{41.522827\du}{13.969201\du}}
\pgfpathlineto{\pgfpoint{41.511144\du}{13.975042\du}}
\pgfpathlineto{\pgfpoint{41.499826\du}{13.980884\du}}
\pgfpathlineto{\pgfpoint{41.488143\du}{13.986725\du}}
\pgfpathlineto{\pgfpoint{41.476825\du}{13.992567\du}}
\pgfpathlineto{\pgfpoint{41.465142\du}{13.997678\du}}
\pgfpathlineto{\pgfpoint{41.454189\du}{14.004250\du}}
\pgfpathlineto{\pgfpoint{41.443966\du}{14.010092\du}}
\pgfpathlineto{\pgfpoint{41.433379\du}{14.015933\du}}
\pgfpathlineto{\pgfpoint{41.423156\du}{14.021775\du}}
\pgfpathlineto{\pgfpoint{41.412933\du}{14.028346\du}}
\pgfpathlineto{\pgfpoint{41.403076\du}{14.034553\du}}
\pgfpathlineto{\pgfpoint{41.393583\du}{14.040395\du}}
\pgfpathlineto{\pgfpoint{41.384090\du}{14.046966\du}}
\pgfpathlineto{\pgfpoint{41.374963\du}{14.053538\du}}
\pgfpathlineto{\pgfpoint{41.365105\du}{14.059745\du}}
\pgfpathlineto{\pgfpoint{41.356708\du}{14.066316\du}}
\pgfpathlineto{\pgfpoint{41.347946\du}{14.072888\du}}
\pgfpathlineto{\pgfpoint{41.339549\du}{14.079095\du}}
\pgfpathlineto{\pgfpoint{41.330786\du}{14.085667\du}}
\pgfpathlineto{\pgfpoint{41.323119\du}{14.092603\du}}
\pgfpathlineto{\pgfpoint{41.316182\du}{14.099175\du}}
\pgfpathlineto{\pgfpoint{41.307420\du}{14.105382\du}}
\pgfpathlineto{\pgfpoint{41.300483\du}{14.112684\du}}
\pgfpathlineto{\pgfpoint{41.293546\du}{14.118890\du}}
\pgfpathlineto{\pgfpoint{41.286610\du}{14.125827\du}}
\pgfpathlineto{\pgfpoint{41.280038\du}{14.133129\du}}
\pgfpathlineto{\pgfpoint{41.273101\du}{14.139336\du}}
\pgfpathlineto{\pgfpoint{41.267260\du}{14.146638\du}}
\pgfpathlineto{\pgfpoint{41.261418\du}{14.153575\du}}
\pgfpathlineto{\pgfpoint{41.255576\du}{14.161242\du}}
\pgfpathlineto{\pgfpoint{41.250100\du}{14.168178\du}}
\pgfpathlineto{\pgfpoint{41.245354\du}{14.175115\du}}
\pgfpathlineto{\pgfpoint{41.239877\du}{14.182052\du}}
\pgfpathlineto{\pgfpoint{41.235861\du}{14.189354\du}}
\pgfpathlineto{\pgfpoint{41.230750\du}{14.196656\du}}
\pgfpathlineto{\pgfpoint{41.227099\du}{14.203958\du}}
\pgfpathlineto{\pgfpoint{41.222718\du}{14.211260\du}}
\pgfpathlineto{\pgfpoint{41.219067\du}{14.218927\du}}
\pgfpathlineto{\pgfpoint{41.215781\du}{14.226594\du}}
\pgfpathlineto{\pgfpoint{41.212130\du}{14.233531\du}}
\pgfpathlineto{\pgfpoint{41.209209\du}{14.241198\du}}
\pgfpathlineto{\pgfpoint{41.206653\du}{14.248865\du}}
\pgfpathlineto{\pgfpoint{41.204463\du}{14.256532\du}}
\pgfpathlineto{\pgfpoint{41.202272\du}{14.264199\du}}
\pgfpathlineto{\pgfpoint{41.200447\du}{14.271501\du}}
\pgfpathlineto{\pgfpoint{41.198621\du}{14.279168\du}}
\pgfpathlineto{\pgfpoint{41.197526\du}{14.286835\du}}
\pgfpathlineto{\pgfpoint{41.196431\du}{14.295232\du}}
\pgfpathlineto{\pgfpoint{41.195701\du}{14.302534\du}}
\pgfpathlineto{\pgfpoint{41.195335\du}{14.310201\du}}
\pgfpathlineto{\pgfpoint{41.195335\du}{14.318598\du}}
\pgfpathlineto{\pgfpoint{41.215781\du}{14.318598\du}}
\pgfpathlineto{\pgfpoint{41.216146\du}{14.311296\du}}
\pgfpathlineto{\pgfpoint{41.216146\du}{14.304360\du}}
\pgfpathlineto{\pgfpoint{41.216511\du}{14.297423\du}}
\pgfpathlineto{\pgfpoint{41.217971\du}{14.289756\du}}
\pgfpathlineto{\pgfpoint{41.219067\du}{14.283549\du}}
\pgfpathlineto{\pgfpoint{41.220162\du}{14.276247\du}}
\pgfpathlineto{\pgfpoint{41.222353\du}{14.269310\du}}
\pgfpathlineto{\pgfpoint{41.223813\du}{14.262373\du}}
\pgfpathlineto{\pgfpoint{41.226004\du}{14.255437\du}}
\pgfpathlineto{\pgfpoint{41.228194\du}{14.248135\du}}
\pgfpathlineto{\pgfpoint{41.231480\du}{14.241928\du}}
\pgfpathlineto{\pgfpoint{41.234036\du}{14.234991\du}}
\pgfpathlineto{\pgfpoint{41.236956\du}{14.227689\du}}
\pgfpathlineto{\pgfpoint{41.240973\du}{14.220752\du}}
\pgfpathlineto{\pgfpoint{41.244989\du}{14.213816\du}}
\pgfpathlineto{\pgfpoint{41.248274\du}{14.207244\du}}
\pgfpathlineto{\pgfpoint{41.252291\du}{14.200307\du}}
\pgfpathlineto{\pgfpoint{41.257037\du}{14.193735\du}}
\pgfpathlineto{\pgfpoint{41.261418\du}{14.186798\du}}
\pgfpathlineto{\pgfpoint{41.266894\du}{14.180227\du}}
\pgfpathlineto{\pgfpoint{41.271276\du}{14.174020\du}}
\pgfpathlineto{\pgfpoint{41.277482\du}{14.167083\du}}
\pgfpathlineto{\pgfpoint{41.282594\du}{14.160511\du}}
\pgfpathlineto{\pgfpoint{41.288800\du}{14.153575\du}}
\pgfpathlineto{\pgfpoint{41.294642\du}{14.147003\du}}
\pgfpathlineto{\pgfpoint{41.301214\du}{14.140796\du}}
\pgfpathlineto{\pgfpoint{41.307420\du}{14.134224\du}}
\pgfpathlineto{\pgfpoint{41.314357\du}{14.127288\du}}
\pgfpathlineto{\pgfpoint{41.321659\du}{14.121446\du}}
\pgfpathlineto{\pgfpoint{41.328596\du}{14.114144\du}}
\pgfpathlineto{\pgfpoint{41.336263\du}{14.107937\du}}
\pgfpathlineto{\pgfpoint{41.344295\du}{14.102096\du}}
\pgfpathlineto{\pgfpoint{41.351597\du}{14.095524\du}}
\pgfpathlineto{\pgfpoint{41.359994\du}{14.088952\du}}
\pgfpathlineto{\pgfpoint{41.368756\du}{14.082746\du}}
\pgfpathlineto{\pgfpoint{41.377884\du}{14.076174\du}}
\pgfpathlineto{\pgfpoint{41.386646\du}{14.070333\du}}
\pgfpathlineto{\pgfpoint{41.395043\du}{14.064126\du}}
\pgfpathlineto{\pgfpoint{41.404171\du}{14.058284\du}}
\pgfpathlineto{\pgfpoint{41.414028\du}{14.051713\du}}
\pgfpathlineto{\pgfpoint{41.423521\du}{14.045871\du}}
\pgfpathlineto{\pgfpoint{41.433379\du}{14.040029\du}}
\pgfpathlineto{\pgfpoint{41.443966\du}{14.034188\du}}
\pgfpathlineto{\pgfpoint{41.453824\du}{14.028346\du}}
\pgfpathlineto{\pgfpoint{41.464412\du}{14.021775\du}}
\pgfpathlineto{\pgfpoint{41.475730\du}{14.016663\du}}
\pgfpathlineto{\pgfpoint{41.485587\du}{14.010822\du}}
\pgfpathlineto{\pgfpoint{41.497270\du}{14.004980\du}}
\pgfpathlineto{\pgfpoint{41.508223\du}{13.999139\du}}
\pgfpathlineto{\pgfpoint{41.519906\du}{13.993297\du}}
\pgfpathlineto{\pgfpoint{41.531590\du}{13.987821\du}}
\pgfpathlineto{\pgfpoint{41.544368\du}{13.982709\du}}
\pgfpathlineto{\pgfpoint{41.555686\du}{13.976868\du}}
\pgfpathlineto{\pgfpoint{41.567734\du}{13.971391\du}}
\pgfpathlineto{\pgfpoint{41.580878\du}{13.966280\du}}
\pgfpathlineto{\pgfpoint{41.593291\du}{13.960803\du}}
\pgfpathlineto{\pgfpoint{41.606069\du}{13.955692\du}}
\pgfpathlineto{\pgfpoint{41.619578\du}{13.949851\du}}
\pgfpathlineto{\pgfpoint{41.632356\du}{13.945104\du}}
\pgfpathlineto{\pgfpoint{41.645865\du}{13.939993\du}}
\pgfpathlineto{\pgfpoint{41.659373\du}{13.934517\du}}
\pgfpathlineto{\pgfpoint{41.673247\du}{13.930135\du}}
\pgfpathlineto{\pgfpoint{41.687121\du}{13.924659\du}}
\pgfpathlineto{\pgfpoint{41.701360\du}{13.919913\du}}
\pgfpathlineto{\pgfpoint{41.715598\du}{13.915166\du}}
\pgfpathlineto{\pgfpoint{41.729837\du}{13.910055\du}}
\pgfpathlineto{\pgfpoint{41.744076\du}{13.905309\du}}
\pgfpathlineto{\pgfpoint{41.759410\du}{13.900563\du}}
\pgfpathlineto{\pgfpoint{41.788983\du}{13.891800\du}}
\pgfpathlineto{\pgfpoint{41.820381\du}{13.882673\du}}
\pgfpathlineto{\pgfpoint{41.851414\du}{13.874276\du}}
\pgfpathlineto{\pgfpoint{41.883908\du}{13.865513\du}}
\pgfpathlineto{\pgfpoint{41.915306\du}{13.857481\du}}
\pgfpathlineto{\pgfpoint{41.948530\du}{13.849814\du}}
\pgfpathlineto{\pgfpoint{41.982484\du}{13.842147\du}}
\pgfpathlineto{\pgfpoint{42.016803\du}{13.834480\du}}
\pgfpathlineto{\pgfpoint{42.051487\du}{13.827543\du}}
\pgfpathlineto{\pgfpoint{42.087632\du}{13.820606\du}}
\pgfpathlineto{\pgfpoint{42.122681\du}{13.814765\du}}
\pgfpathlineto{\pgfpoint{42.159921\du}{13.808193\du}}
\pgfpathlineto{\pgfpoint{42.196431\du}{13.802351\du}}
\pgfpathlineto{\pgfpoint{42.233671\du}{13.796510\du}}
\pgfpathlineto{\pgfpoint{42.271276\du}{13.791399\du}}
\pgfpathlineto{\pgfpoint{42.309611\du}{13.785922\du}}
\pgfpathlineto{\pgfpoint{42.348676\du}{13.781176\du}}
\pgfpathlineto{\pgfpoint{42.388107\du}{13.777160\du}}
\pgfpathlineto{\pgfpoint{42.427537\du}{13.773144\du}}
\pgfpathlineto{\pgfpoint{42.467333\du}{13.769493\du}}
\pgfpathlineto{\pgfpoint{42.507493\du}{13.766207\du}}
\pgfpathlineto{\pgfpoint{42.548384\du}{13.763286\du}}
\pgfpathlineto{\pgfpoint{42.589640\du}{13.760365\du}}
\pgfpathlineto{\pgfpoint{42.630531\du}{13.757810\du}}
\pgfpathlineto{\pgfpoint{42.672517\du}{13.755619\du}}
\pgfpathlineto{\pgfpoint{42.714868\du}{13.754524\du}}
\pgfpathlineto{\pgfpoint{42.757584\du}{13.752698\du}}
\pgfpathlineto{\pgfpoint{42.799936\du}{13.751968\du}}
\pgfpathlineto{\pgfpoint{42.843382\du}{13.751603\du}}
\pgfpathlineto{\pgfpoint{42.886099\du}{13.751603\du}}
\pgfpathlineto{\pgfpoint{42.886099\du}{13.751603\du}}
\pgfpathlineto{\pgfpoint{42.886099\du}{13.751603\du}}
\pgfpathlineto{\pgfpoint{42.887194\du}{13.750873\du}}
\pgfpathlineto{\pgfpoint{42.888289\du}{13.750873\du}}
\pgfpathlineto{\pgfpoint{42.889749\du}{13.750873\du}}
\pgfpathlineto{\pgfpoint{42.890845\du}{13.750508\du}}
\pgfpathlineto{\pgfpoint{42.891210\du}{13.749778\du}}
\pgfpathlineto{\pgfpoint{42.892305\du}{13.749778\du}}
\pgfpathlineto{\pgfpoint{42.893035\du}{13.748682\du}}
\pgfpathlineto{\pgfpoint{42.894131\du}{13.747952\du}}
\pgfpathlineto{\pgfpoint{42.895226\du}{13.746857\du}}
\pgfpathlineto{\pgfpoint{42.895956\du}{13.745031\du}}
\pgfpathlineto{\pgfpoint{42.895956\du}{13.742841\du}}
\pgfpathlineto{\pgfpoint{42.896686\du}{13.741015\du}}
\pgfpathlineto{\pgfpoint{42.895956\du}{13.739190\du}}
\pgfpathlineto{\pgfpoint{42.895956\du}{13.737364\du}}
\pgfpathlineto{\pgfpoint{42.895226\du}{13.735539\du}}
\pgfpathlineto{\pgfpoint{42.894131\du}{13.734078\du}}
\pgfpathlineto{\pgfpoint{42.893035\du}{13.733348\du}}
\pgfpathlineto{\pgfpoint{42.892305\du}{13.732618\du}}
\pgfpathlineto{\pgfpoint{42.891210\du}{13.732253\du}}
\pgfpathlineto{\pgfpoint{42.890845\du}{13.731523\du}}
\pgfpathlineto{\pgfpoint{42.889749\du}{13.731158\du}}
\pgfpathlineto{\pgfpoint{42.888289\du}{13.731158\du}}
\pgfpathlineto{\pgfpoint{42.887194\du}{13.731158\du}}
\pgfpathlineto{\pgfpoint{42.886099\du}{13.731158\du}}
\pgfusepath{fill}
\pgfsetbuttcap
\pgfsetmiterjoin
\pgfsetdash{}{0pt}
\definecolor{dialinecolor}{rgb}{0.678431, 0.839216, 0.905882}
\pgfsetfillcolor{dialinecolor}
\pgfpathmoveto{\pgfpoint{44.576496\du}{14.318598\du}}
\pgfpathlineto{\pgfpoint{44.576496\du}{14.310201\du}}
\pgfpathlineto{\pgfpoint{44.575766\du}{14.302534\du}}
\pgfpathlineto{\pgfpoint{44.575036\du}{14.295232\du}}
\pgfpathlineto{\pgfpoint{44.574306\du}{14.286835\du}}
\pgfpathlineto{\pgfpoint{44.573211\du}{14.279168\du}}
\pgfpathlineto{\pgfpoint{44.571020\du}{14.271501\du}}
\pgfpathlineto{\pgfpoint{44.569925\du}{14.264199\du}}
\pgfpathlineto{\pgfpoint{44.567004\du}{14.256532\du}}
\pgfpathlineto{\pgfpoint{44.565178\du}{14.248865\du}}
\pgfpathlineto{\pgfpoint{44.562258\du}{14.241198\du}}
\pgfpathlineto{\pgfpoint{44.559702\du}{14.233531\du}}
\pgfpathlineto{\pgfpoint{44.556416\du}{14.226594\du}}
\pgfpathlineto{\pgfpoint{44.552400\du}{14.218927\du}}
\pgfpathlineto{\pgfpoint{44.549114\du}{14.211260\du}}
\pgfpathlineto{\pgfpoint{44.545098\du}{14.203958\du}}
\pgfpathlineto{\pgfpoint{44.541082\du}{14.196656\du}}
\pgfpathlineto{\pgfpoint{44.536701\du}{14.189354\du}}
\pgfpathlineto{\pgfpoint{44.531955\du}{14.182052\du}}
\pgfpathlineto{\pgfpoint{44.526478\du}{14.175115\du}}
\pgfpathlineto{\pgfpoint{44.521732\du}{14.168178\du}}
\pgfpathlineto{\pgfpoint{44.515890\du}{14.160511\du}}
\pgfpathlineto{\pgfpoint{44.510414\du}{14.153575\du}}
\pgfpathlineto{\pgfpoint{44.504572\du}{14.146638\du}}
\pgfpathlineto{\pgfpoint{44.498366\du}{14.139336\du}}
\pgfpathlineto{\pgfpoint{44.491794\du}{14.133129\du}}
\pgfpathlineto{\pgfpoint{44.484857\du}{14.125827\du}}
\pgfpathlineto{\pgfpoint{44.478285\du}{14.118890\du}}
\pgfpathlineto{\pgfpoint{44.470984\du}{14.112684\du}}
\pgfpathlineto{\pgfpoint{44.464412\du}{14.105382\du}}
\pgfpathlineto{\pgfpoint{44.456380\du}{14.099175\du}}
\pgfpathlineto{\pgfpoint{44.448348\du}{14.092603\du}}
\pgfpathlineto{\pgfpoint{44.441046\du}{14.085667\du}}
\pgfpathlineto{\pgfpoint{44.432283\du}{14.079095\du}}
\pgfpathlineto{\pgfpoint{44.424251\du}{14.072888\du}}
\pgfpathlineto{\pgfpoint{44.415124\du}{14.066316\du}}
\pgfpathlineto{\pgfpoint{44.406361\du}{14.059745\du}}
\pgfpathlineto{\pgfpoint{44.396869\du}{14.053538\du}}
\pgfpathlineto{\pgfpoint{44.387741\du}{14.046966\du}}
\pgfpathlineto{\pgfpoint{44.378614\du}{14.040395\du}}
\pgfpathlineto{\pgfpoint{44.368756\du}{14.034553\du}}
\pgfpathlineto{\pgfpoint{44.358899\du}{14.028346\du}}
\pgfpathlineto{\pgfpoint{44.348676\du}{14.021775\du}}
\pgfpathlineto{\pgfpoint{44.338088\du}{14.015933\du}}
\pgfpathlineto{\pgfpoint{44.328231\du}{14.010092\du}}
\pgfpathlineto{\pgfpoint{44.317643\du}{14.004250\du}}
\pgfpathlineto{\pgfpoint{44.306325\du}{13.997678\du}}
\pgfpathlineto{\pgfpoint{44.295007\du}{13.992567\du}}
\pgfpathlineto{\pgfpoint{44.283324\du}{13.986725\du}}
\pgfpathlineto{\pgfpoint{44.272371\du}{13.980884\du}}
\pgfpathlineto{\pgfpoint{44.260323\du}{13.975042\du}}
\pgfpathlineto{\pgfpoint{44.249005\du}{13.969201\du}}
\pgfpathlineto{\pgfpoint{44.236591\du}{13.964089\du}}
\pgfpathlineto{\pgfpoint{44.224178\du}{13.957883\du}}
\pgfpathlineto{\pgfpoint{44.211400\du}{13.952041\du}}
\pgfpathlineto{\pgfpoint{44.199352\du}{13.946930\du}}
\pgfpathlineto{\pgfpoint{44.186573\du}{13.941818\du}}
\pgfpathlineto{\pgfpoint{44.173065\du}{13.936342\du}}
\pgfpathlineto{\pgfpoint{44.159921\du}{13.931231\du}}
\pgfpathlineto{\pgfpoint{44.146778\du}{13.925754\du}}
\pgfpathlineto{\pgfpoint{44.132539\du}{13.920643\du}}
\pgfpathlineto{\pgfpoint{44.119395\du}{13.915166\du}}
\pgfpathlineto{\pgfpoint{44.105157\du}{13.910785\du}}
\pgfpathlineto{\pgfpoint{44.091283\du}{13.905309\du}}
\pgfpathlineto{\pgfpoint{44.077044\du}{13.900563\du}}
\pgfpathlineto{\pgfpoint{44.063170\du}{13.895451\du}}
\pgfpathlineto{\pgfpoint{44.048201\du}{13.890705\du}}
\pgfpathlineto{\pgfpoint{44.033233\du}{13.885959\du}}
\pgfpathlineto{\pgfpoint{44.018994\du}{13.881577\du}}
\pgfpathlineto{\pgfpoint{43.988326\du}{13.872085\du}}
\pgfpathlineto{\pgfpoint{43.957657\du}{13.863323\du}}
\pgfpathlineto{\pgfpoint{43.926259\du}{13.854560\du}}
\pgfpathlineto{\pgfpoint{43.894496\du}{13.845798\du}}
\pgfpathlineto{\pgfpoint{43.861637\du}{13.837401\du}}
\pgfpathlineto{\pgfpoint{43.828048\du}{13.829734\du}}
\pgfpathlineto{\pgfpoint{43.794094\du}{13.821702\du}}
\pgfpathlineto{\pgfpoint{43.759410\du}{13.814765\du}}
\pgfpathlineto{\pgfpoint{43.724726\du}{13.807828\du}}
\pgfpathlineto{\pgfpoint{43.689311\du}{13.800526\du}}
\pgfpathlineto{\pgfpoint{43.652802\du}{13.794319\du}}
\pgfpathlineto{\pgfpoint{43.616292\du}{13.787748\du}}
\pgfpathlineto{\pgfpoint{43.579417\du}{13.781906\du}}
\pgfpathlineto{\pgfpoint{43.542177\du}{13.776065\du}}
\pgfpathlineto{\pgfpoint{43.503112\du}{13.770953\du}}
\pgfpathlineto{\pgfpoint{43.465142\du}{13.766207\du}}
\pgfpathlineto{\pgfpoint{43.425712\du}{13.761461\du}}
\pgfpathlineto{\pgfpoint{43.387011\du}{13.756714\du}}
\pgfpathlineto{\pgfpoint{43.346851\du}{13.752698\du}}
\pgfpathlineto{\pgfpoint{43.307055\du}{13.748682\du}}
\pgfpathlineto{\pgfpoint{43.266164\du}{13.745761\du}}
\pgfpathlineto{\pgfpoint{43.225639\du}{13.742841\du}}
\pgfpathlineto{\pgfpoint{43.184017\du}{13.739920\du}}
\pgfpathlineto{\pgfpoint{43.142396\du}{13.737364\du}}
\pgfpathlineto{\pgfpoint{43.100775\du}{13.735539\du}}
\pgfpathlineto{\pgfpoint{43.058059\du}{13.734078\du}}
\pgfpathlineto{\pgfpoint{43.016073\du}{13.732618\du}}
\pgfpathlineto{\pgfpoint{42.972992\du}{13.731523\du}}
\pgfpathlineto{\pgfpoint{42.929545\du}{13.731158\du}}
\pgfpathlineto{\pgfpoint{42.886099\du}{13.731158\du}}
\pgfpathlineto{\pgfpoint{42.886099\du}{13.751603\du}}
\pgfpathlineto{\pgfpoint{42.929180\du}{13.751603\du}}
\pgfpathlineto{\pgfpoint{42.972626\du}{13.751968\du}}
\pgfpathlineto{\pgfpoint{43.014613\du}{13.752698\du}}
\pgfpathlineto{\pgfpoint{43.057329\du}{13.754524\du}}
\pgfpathlineto{\pgfpoint{43.100045\du}{13.755619\du}}
\pgfpathlineto{\pgfpoint{43.141666\du}{13.757810\du}}
\pgfpathlineto{\pgfpoint{43.182922\du}{13.760365\du}}
\pgfpathlineto{\pgfpoint{43.223813\du}{13.763286\du}}
\pgfpathlineto{\pgfpoint{43.265069\du}{13.766207\du}}
\pgfpathlineto{\pgfpoint{43.305230\du}{13.769493\du}}
\pgfpathlineto{\pgfpoint{43.345390\du}{13.773144\du}}
\pgfpathlineto{\pgfpoint{43.384456\du}{13.777160\du}}
\pgfpathlineto{\pgfpoint{43.423886\du}{13.781176\du}}
\pgfpathlineto{\pgfpoint{43.462586\du}{13.785922\du}}
\pgfpathlineto{\pgfpoint{43.501287\du}{13.791399\du}}
\pgfpathlineto{\pgfpoint{43.538892\du}{13.796510\du}}
\pgfpathlineto{\pgfpoint{43.576131\du}{13.802351\du}}
\pgfpathlineto{\pgfpoint{43.612641\du}{13.808193\du}}
\pgfpathlineto{\pgfpoint{43.649516\du}{13.814765\du}}
\pgfpathlineto{\pgfpoint{43.684930\du}{13.820606\du}}
\pgfpathlineto{\pgfpoint{43.720710\du}{13.827543\du}}
\pgfpathlineto{\pgfpoint{43.755394\du}{13.834480\du}}
\pgfpathlineto{\pgfpoint{43.789348\du}{13.842147\du}}
\pgfpathlineto{\pgfpoint{43.823667\du}{13.849814\du}}
\pgfpathlineto{\pgfpoint{43.856891\du}{13.857481\du}}
\pgfpathlineto{\pgfpoint{43.889019\du}{13.865513\du}}
\pgfpathlineto{\pgfpoint{43.921148\du}{13.874276\du}}
\pgfpathlineto{\pgfpoint{43.952546\du}{13.882673\du}}
\pgfpathlineto{\pgfpoint{43.983214\du}{13.891800\du}}
\pgfpathlineto{\pgfpoint{44.013152\du}{13.900563\du}}
\pgfpathlineto{\pgfpoint{44.027391\du}{13.905309\du}}
\pgfpathlineto{\pgfpoint{44.041995\du}{13.910055\du}}
\pgfpathlineto{\pgfpoint{44.055869\du}{13.915166\du}}
\pgfpathlineto{\pgfpoint{44.070472\du}{13.919913\du}}
\pgfpathlineto{\pgfpoint{44.084711\du}{13.924659\du}}
\pgfpathlineto{\pgfpoint{44.098585\du}{13.930135\du}}
\pgfpathlineto{\pgfpoint{44.112824\du}{13.934517\du}}
\pgfpathlineto{\pgfpoint{44.125967\du}{13.939993\du}}
\pgfpathlineto{\pgfpoint{44.139476\du}{13.945104\du}}
\pgfpathlineto{\pgfpoint{44.152619\du}{13.949851\du}}
\pgfpathlineto{\pgfpoint{44.165398\du}{13.955692\du}}
\pgfpathlineto{\pgfpoint{44.178176\du}{13.960803\du}}
\pgfpathlineto{\pgfpoint{44.191319\du}{13.966280\du}}
\pgfpathlineto{\pgfpoint{44.203368\du}{13.971391\du}}
\pgfpathlineto{\pgfpoint{44.215781\du}{13.976868\du}}
\pgfpathlineto{\pgfpoint{44.227829\du}{13.982709\du}}
\pgfpathlineto{\pgfpoint{44.240242\du}{13.987821\du}}
\pgfpathlineto{\pgfpoint{44.251560\du}{13.993297\du}}
\pgfpathlineto{\pgfpoint{44.263609\du}{13.999139\du}}
\pgfpathlineto{\pgfpoint{44.274196\du}{14.004980\du}}
\pgfpathlineto{\pgfpoint{44.286245\du}{14.010822\du}}
\pgfpathlineto{\pgfpoint{44.296467\du}{14.016663\du}}
\pgfpathlineto{\pgfpoint{44.307420\du}{14.021775\du}}
\pgfpathlineto{\pgfpoint{44.318373\du}{14.028346\du}}
\pgfpathlineto{\pgfpoint{44.328231\du}{14.034188\du}}
\pgfpathlineto{\pgfpoint{44.338088\du}{14.040029\du}}
\pgfpathlineto{\pgfpoint{44.348311\du}{14.045871\du}}
\pgfpathlineto{\pgfpoint{44.357438\du}{14.051713\du}}
\pgfpathlineto{\pgfpoint{44.367661\du}{14.058284\du}}
\pgfpathlineto{\pgfpoint{44.377154\du}{14.064126\du}}
\pgfpathlineto{\pgfpoint{44.385916\du}{14.070333\du}}
\pgfpathlineto{\pgfpoint{44.393948\du}{14.076174\du}}
\pgfpathlineto{\pgfpoint{44.402710\du}{14.082746\du}}
\pgfpathlineto{\pgfpoint{44.411108\du}{14.088952\du}}
\pgfpathlineto{\pgfpoint{44.419870\du}{14.095524\du}}
\pgfpathlineto{\pgfpoint{44.427537\du}{14.102096\du}}
\pgfpathlineto{\pgfpoint{44.435569\du}{14.107937\du}}
\pgfpathlineto{\pgfpoint{44.442871\du}{14.114144\du}}
\pgfpathlineto{\pgfpoint{44.450538\du}{14.121446\du}}
\pgfpathlineto{\pgfpoint{44.457110\du}{14.127288\du}}
\pgfpathlineto{\pgfpoint{44.464412\du}{14.134224\du}}
\pgfpathlineto{\pgfpoint{44.470253\du}{14.140796\du}}
\pgfpathlineto{\pgfpoint{44.477190\du}{14.147003\du}}
\pgfpathlineto{\pgfpoint{44.483762\du}{14.153575\du}}
\pgfpathlineto{\pgfpoint{44.488873\du}{14.160511\du}}
\pgfpathlineto{\pgfpoint{44.494350\du}{14.167083\du}}
\pgfpathlineto{\pgfpoint{44.500556\du}{14.174020\du}}
\pgfpathlineto{\pgfpoint{44.505303\du}{14.180227\du}}
\pgfpathlineto{\pgfpoint{44.510414\du}{14.186798\du}}
\pgfpathlineto{\pgfpoint{44.515160\du}{14.193735\du}}
\pgfpathlineto{\pgfpoint{44.519541\du}{14.200307\du}}
\pgfpathlineto{\pgfpoint{44.523923\du}{14.207244\du}}
\pgfpathlineto{\pgfpoint{44.526843\du}{14.213816\du}}
\pgfpathlineto{\pgfpoint{44.530859\du}{14.220752\du}}
\pgfpathlineto{\pgfpoint{44.533780\du}{14.227689\du}}
\pgfpathlineto{\pgfpoint{44.537796\du}{14.234991\du}}
\pgfpathlineto{\pgfpoint{44.540352\du}{14.241198\du}}
\pgfpathlineto{\pgfpoint{44.543273\du}{14.248135\du}}
\pgfpathlineto{\pgfpoint{44.545828\du}{14.255437\du}}
\pgfpathlineto{\pgfpoint{44.548019\du}{14.262373\du}}
\pgfpathlineto{\pgfpoint{44.549479\du}{14.269310\du}}
\pgfpathlineto{\pgfpoint{44.551670\du}{14.276247\du}}
\pgfpathlineto{\pgfpoint{44.552400\du}{14.283549\du}}
\pgfpathlineto{\pgfpoint{44.553495\du}{14.289756\du}}
\pgfpathlineto{\pgfpoint{44.555321\du}{14.297423\du}}
\pgfpathlineto{\pgfpoint{44.555686\du}{14.304360\du}}
\pgfpathlineto{\pgfpoint{44.555686\du}{14.311296\du}}
\pgfpathlineto{\pgfpoint{44.556416\du}{14.318598\du}}
\pgfpathlineto{\pgfpoint{44.576496\du}{14.318598\du}}
\pgfusepath{fill}
\pgfsetbuttcap
\pgfsetmiterjoin
\pgfsetdash{}{0pt}
\definecolor{dialinecolor}{rgb}{0.027451, 0.486275, 0.682353}
\pgfsetfillcolor{dialinecolor}
\pgfpathmoveto{\pgfpoint{41.200447\du}{13.509179\du}}
\pgfpathlineto{\pgfpoint{41.200447\du}{14.333567\du}}
\pgfpathlineto{\pgfpoint{44.565909\du}{14.333567\du}}
\pgfpathlineto{\pgfpoint{44.566639\du}{13.509909\du}}
\pgfpathlineto{\pgfpoint{41.200447\du}{13.509179\du}}
\pgfusepath{fill}
\pgfsetbuttcap
\pgfsetmiterjoin
\pgfsetdash{}{0pt}
\definecolor{dialinecolor}{rgb}{0.235294, 0.686275, 0.894118}
\pgfsetfillcolor{dialinecolor}
\pgfpathmoveto{\pgfpoint{44.565909\du}{13.493480\du}}
\pgfpathlineto{\pgfpoint{44.564448\du}{13.523418\du}}
\pgfpathlineto{\pgfpoint{44.557146\du}{13.552625\du}}
\pgfpathlineto{\pgfpoint{44.546924\du}{13.581833\du}}
\pgfpathlineto{\pgfpoint{44.532320\du}{13.609946\du}}
\pgfpathlineto{\pgfpoint{44.512970\du}{13.638058\du}}
\pgfpathlineto{\pgfpoint{44.491064\du}{13.665440\du}}
\pgfpathlineto{\pgfpoint{44.464412\du}{13.691727\du}}
\pgfpathlineto{\pgfpoint{44.433744\du}{13.718014\du}}
\pgfpathlineto{\pgfpoint{44.400885\du}{13.743936\du}}
\pgfpathlineto{\pgfpoint{44.363645\du}{13.769128\du}}
\pgfpathlineto{\pgfpoint{44.322754\du}{13.793224\du}}
\pgfpathlineto{\pgfpoint{44.278943\du}{13.816590\du}}
\pgfpathlineto{\pgfpoint{44.232575\du}{13.838861\du}}
\pgfpathlineto{\pgfpoint{44.182192\du}{13.860767\du}}
\pgfpathlineto{\pgfpoint{44.129253\du}{13.881943\du}}
\pgfpathlineto{\pgfpoint{44.073758\du}{13.902023\du}}
\pgfpathlineto{\pgfpoint{44.015708\du}{13.920643\du}}
\pgfpathlineto{\pgfpoint{43.955102\du}{13.939263\du}}
\pgfpathlineto{\pgfpoint{43.891210\du}{13.956422\du}}
\pgfpathlineto{\pgfpoint{43.826223\du}{13.972121\du}}
\pgfpathlineto{\pgfpoint{43.757584\du}{13.987456\du}}
\pgfpathlineto{\pgfpoint{43.686756\du}{14.001329\du}}
\pgfpathlineto{\pgfpoint{43.614832\du}{14.014108\du}}
\pgfpathlineto{\pgfpoint{43.539987\du}{14.025426\du}}
\pgfpathlineto{\pgfpoint{43.464047\du}{14.036013\du}}
\pgfpathlineto{\pgfpoint{43.385551\du}{14.045141\du}}
\pgfpathlineto{\pgfpoint{43.305960\du}{14.052808\du}}
\pgfpathlineto{\pgfpoint{43.224908\du}{14.059380\du}}
\pgfpathlineto{\pgfpoint{43.141666\du}{14.064491\du}}
\pgfpathlineto{\pgfpoint{43.058059\du}{14.068142\du}}
\pgfpathlineto{\pgfpoint{42.972626\du}{14.069967\du}}
\pgfpathlineto{\pgfpoint{42.886099\du}{14.071063\du}}
\pgfpathlineto{\pgfpoint{42.799936\du}{14.069967\du}}
\pgfpathlineto{\pgfpoint{42.714138\du}{14.068142\du}}
\pgfpathlineto{\pgfpoint{42.630531\du}{14.064491\du}}
\pgfpathlineto{\pgfpoint{42.547654\du}{14.059380\du}}
\pgfpathlineto{\pgfpoint{42.466237\du}{14.052808\du}}
\pgfpathlineto{\pgfpoint{42.386646\du}{14.045141\du}}
\pgfpathlineto{\pgfpoint{42.308881\du}{14.036013\du}}
\pgfpathlineto{\pgfpoint{42.232210\du}{14.025426\du}}
\pgfpathlineto{\pgfpoint{42.158096\du}{14.014108\du}}
\pgfpathlineto{\pgfpoint{42.085441\du}{14.001329\du}}
\pgfpathlineto{\pgfpoint{42.014978\du}{13.987456\du}}
\pgfpathlineto{\pgfpoint{41.946339\du}{13.972121\du}}
\pgfpathlineto{\pgfpoint{41.880622\du}{13.956422\du}}
\pgfpathlineto{\pgfpoint{41.817095\du}{13.939263\du}}
\pgfpathlineto{\pgfpoint{41.756124\du}{13.920643\du}}
\pgfpathlineto{\pgfpoint{41.697709\du}{13.902023\du}}
\pgfpathlineto{\pgfpoint{41.642579\du}{13.881943\du}}
\pgfpathlineto{\pgfpoint{41.589640\du}{13.860767\du}}
\pgfpathlineto{\pgfpoint{41.539622\du}{13.838861\du}}
\pgfpathlineto{\pgfpoint{41.492524\du}{13.816590\du}}
\pgfpathlineto{\pgfpoint{41.449078\du}{13.793224\du}}
\pgfpathlineto{\pgfpoint{41.408187\du}{13.769128\du}}
\pgfpathlineto{\pgfpoint{41.370947\du}{13.743936\du}}
\pgfpathlineto{\pgfpoint{41.337723\du}{13.718014\du}}
\pgfpathlineto{\pgfpoint{41.307420\du}{13.691727\du}}
\pgfpathlineto{\pgfpoint{41.280768\du}{13.665440\du}}
\pgfpathlineto{\pgfpoint{41.258862\du}{13.638058\du}}
\pgfpathlineto{\pgfpoint{41.239512\du}{13.609946\du}}
\pgfpathlineto{\pgfpoint{41.224908\du}{13.581833\du}}
\pgfpathlineto{\pgfpoint{41.214321\du}{13.552625\du}}
\pgfpathlineto{\pgfpoint{41.207384\du}{13.523418\du}}
\pgfpathlineto{\pgfpoint{41.205558\du}{13.493480\du}}
\pgfpathlineto{\pgfpoint{41.207384\du}{13.464272\du}}
\pgfpathlineto{\pgfpoint{41.214321\du}{13.434334\du}}
\pgfpathlineto{\pgfpoint{41.224908\du}{13.405856\du}}
\pgfpathlineto{\pgfpoint{41.239512\du}{13.377744\du}}
\pgfpathlineto{\pgfpoint{41.258862\du}{13.349632\du}}
\pgfpathlineto{\pgfpoint{41.280768\du}{13.321884\du}}
\pgfpathlineto{\pgfpoint{41.307420\du}{13.295232\du}}
\pgfpathlineto{\pgfpoint{41.337723\du}{13.268945\du}}
\pgfpathlineto{\pgfpoint{41.370947\du}{13.243753\du}}
\pgfpathlineto{\pgfpoint{41.408187\du}{13.218562\du}}
\pgfpathlineto{\pgfpoint{41.449078\du}{13.194465\du}}
\pgfpathlineto{\pgfpoint{41.492524\du}{13.171099\du}}
\pgfpathlineto{\pgfpoint{41.539622\du}{13.148098\du}}
\pgfpathlineto{\pgfpoint{41.589640\du}{13.126557\du}}
\pgfpathlineto{\pgfpoint{41.642579\du}{13.105382\du}}
\pgfpathlineto{\pgfpoint{41.697709\du}{13.085667\du}}
\pgfpathlineto{\pgfpoint{41.756124\du}{13.066316\du}}
\pgfpathlineto{\pgfpoint{41.817095\du}{13.048062\du}}
\pgfpathlineto{\pgfpoint{41.880622\du}{13.031267\du}}
\pgfpathlineto{\pgfpoint{41.946339\du}{13.014838\du}}
\pgfpathlineto{\pgfpoint{42.014978\du}{12.999504\du}}
\pgfpathlineto{\pgfpoint{42.085441\du}{12.985995\du}}
\pgfpathlineto{\pgfpoint{42.158096\du}{12.973217\du}}
\pgfpathlineto{\pgfpoint{42.232210\du}{12.961534\du}}
\pgfpathlineto{\pgfpoint{42.308881\du}{12.951676\du}}
\pgfpathlineto{\pgfpoint{42.386646\du}{12.942184\du}}
\pgfpathlineto{\pgfpoint{42.466237\du}{12.934517\du}}
\pgfpathlineto{\pgfpoint{42.547654\du}{12.927945\du}}
\pgfpathlineto{\pgfpoint{42.630531\du}{12.922833\du}}
\pgfpathlineto{\pgfpoint{42.714138\du}{12.919548\du}}
\pgfpathlineto{\pgfpoint{42.799936\du}{12.916992\du}}
\pgfpathlineto{\pgfpoint{42.886099\du}{12.916627\du}}
\pgfpathlineto{\pgfpoint{42.972626\du}{12.916992\du}}
\pgfpathlineto{\pgfpoint{43.058059\du}{12.919548\du}}
\pgfpathlineto{\pgfpoint{43.141666\du}{12.922833\du}}
\pgfpathlineto{\pgfpoint{43.224908\du}{12.927945\du}}
\pgfpathlineto{\pgfpoint{43.305960\du}{12.934517\du}}
\pgfpathlineto{\pgfpoint{43.385551\du}{12.942184\du}}
\pgfpathlineto{\pgfpoint{43.464047\du}{12.951676\du}}
\pgfpathlineto{\pgfpoint{43.539987\du}{12.961534\du}}
\pgfpathlineto{\pgfpoint{43.614832\du}{12.973217\du}}
\pgfpathlineto{\pgfpoint{43.686756\du}{12.985995\du}}
\pgfpathlineto{\pgfpoint{43.757584\du}{12.999504\du}}
\pgfpathlineto{\pgfpoint{43.826223\du}{13.014838\du}}
\pgfpathlineto{\pgfpoint{43.891210\du}{13.031267\du}}
\pgfpathlineto{\pgfpoint{43.955102\du}{13.048062\du}}
\pgfpathlineto{\pgfpoint{44.015708\du}{13.066316\du}}
\pgfpathlineto{\pgfpoint{44.073758\du}{13.085667\du}}
\pgfpathlineto{\pgfpoint{44.129253\du}{13.105382\du}}
\pgfpathlineto{\pgfpoint{44.182192\du}{13.126557\du}}
\pgfpathlineto{\pgfpoint{44.232575\du}{13.148098\du}}
\pgfpathlineto{\pgfpoint{44.278943\du}{13.171099\du}}
\pgfpathlineto{\pgfpoint{44.322754\du}{13.194465\du}}
\pgfpathlineto{\pgfpoint{44.363645\du}{13.218562\du}}
\pgfpathlineto{\pgfpoint{44.400885\du}{13.243753\du}}
\pgfpathlineto{\pgfpoint{44.433744\du}{13.268945\du}}
\pgfpathlineto{\pgfpoint{44.464412\du}{13.295232\du}}
\pgfpathlineto{\pgfpoint{44.491064\du}{13.321884\du}}
\pgfpathlineto{\pgfpoint{44.512970\du}{13.349632\du}}
\pgfpathlineto{\pgfpoint{44.532320\du}{13.377744\du}}
\pgfpathlineto{\pgfpoint{44.546924\du}{13.405856\du}}
\pgfpathlineto{\pgfpoint{44.557146\du}{13.434334\du}}
\pgfpathlineto{\pgfpoint{44.564448\du}{13.464272\du}}
\pgfpathlineto{\pgfpoint{44.565909\du}{13.493480\du}}
\pgfusepath{fill}
\pgfsetbuttcap
\pgfsetmiterjoin
\pgfsetdash{}{0pt}
\definecolor{dialinecolor}{rgb}{0.678431, 0.839216, 0.905882}
\pgfsetfillcolor{dialinecolor}
\pgfpathmoveto{\pgfpoint{42.886099\du}{14.080920\du}}
\pgfpathlineto{\pgfpoint{42.886099\du}{14.080920\du}}
\pgfpathlineto{\pgfpoint{42.929545\du}{14.080920\du}}
\pgfpathlineto{\pgfpoint{42.972992\du}{14.080190\du}}
\pgfpathlineto{\pgfpoint{43.016073\du}{14.079095\du}}
\pgfpathlineto{\pgfpoint{43.058059\du}{14.078000\du}}
\pgfpathlineto{\pgfpoint{43.100775\du}{14.076174\du}}
\pgfpathlineto{\pgfpoint{43.142396\du}{14.074349\du}}
\pgfpathlineto{\pgfpoint{43.184017\du}{14.072158\du}}
\pgfpathlineto{\pgfpoint{43.225639\du}{14.069237\du}}
\pgfpathlineto{\pgfpoint{43.266164\du}{14.066316\du}}
\pgfpathlineto{\pgfpoint{43.307055\du}{14.063396\du}}
\pgfpathlineto{\pgfpoint{43.346851\du}{14.059380\du}}
\pgfpathlineto{\pgfpoint{43.387011\du}{14.055364\du}}
\pgfpathlineto{\pgfpoint{43.425712\du}{14.050617\du}}
\pgfpathlineto{\pgfpoint{43.465142\du}{14.045871\du}}
\pgfpathlineto{\pgfpoint{43.503112\du}{14.041125\du}}
\pgfpathlineto{\pgfpoint{43.542177\du}{14.036013\du}}
\pgfpathlineto{\pgfpoint{43.579417\du}{14.030172\du}}
\pgfpathlineto{\pgfpoint{43.616292\du}{14.024330\du}}
\pgfpathlineto{\pgfpoint{43.652802\du}{14.017759\du}}
\pgfpathlineto{\pgfpoint{43.689311\du}{14.011187\du}}
\pgfpathlineto{\pgfpoint{43.724726\du}{14.004250\du}}
\pgfpathlineto{\pgfpoint{43.759410\du}{13.997313\du}}
\pgfpathlineto{\pgfpoint{43.794094\du}{13.990376\du}}
\pgfpathlineto{\pgfpoint{43.828048\du}{13.982709\du}}
\pgfpathlineto{\pgfpoint{43.861637\du}{13.974312\du}}
\pgfpathlineto{\pgfpoint{43.894496\du}{13.966280\du}}
\pgfpathlineto{\pgfpoint{43.926259\du}{13.957883\du}}
\pgfpathlineto{\pgfpoint{43.957657\du}{13.948755\du}}
\pgfpathlineto{\pgfpoint{43.972992\du}{13.944739\du}}
\pgfpathlineto{\pgfpoint{43.988326\du}{13.939993\du}}
\pgfpathlineto{\pgfpoint{44.004390\du}{13.935247\du}}
\pgfpathlineto{\pgfpoint{44.018994\du}{13.930500\du}}
\pgfpathlineto{\pgfpoint{44.033233\du}{13.925754\du}}
\pgfpathlineto{\pgfpoint{44.048201\du}{13.921373\du}}
\pgfpathlineto{\pgfpoint{44.063170\du}{13.916627\du}}
\pgfpathlineto{\pgfpoint{44.077044\du}{13.911150\du}}
\pgfpathlineto{\pgfpoint{44.091283\du}{13.906404\du}}
\pgfpathlineto{\pgfpoint{44.105157\du}{13.901293\du}}
\pgfpathlineto{\pgfpoint{44.119395\du}{13.896546\du}}
\pgfpathlineto{\pgfpoint{44.132539\du}{13.891435\du}}
\pgfpathlineto{\pgfpoint{44.146778\du}{13.885959\du}}
\pgfpathlineto{\pgfpoint{44.159921\du}{13.880847\du}}
\pgfpathlineto{\pgfpoint{44.173065\du}{13.875736\du}}
\pgfpathlineto{\pgfpoint{44.186573\du}{13.870259\du}}
\pgfpathlineto{\pgfpoint{44.199352\du}{13.865148\du}}
\pgfpathlineto{\pgfpoint{44.211400\du}{13.859672\du}}
\pgfpathlineto{\pgfpoint{44.224178\du}{13.853830\du}}
\pgfpathlineto{\pgfpoint{44.236591\du}{13.847989\du}}
\pgfpathlineto{\pgfpoint{44.249005\du}{13.842877\du}}
\pgfpathlineto{\pgfpoint{44.260323\du}{13.837036\du}}
\pgfpathlineto{\pgfpoint{44.272371\du}{13.831194\du}}
\pgfpathlineto{\pgfpoint{44.283324\du}{13.825353\du}}
\pgfpathlineto{\pgfpoint{44.295007\du}{13.819876\du}}
\pgfpathlineto{\pgfpoint{44.306325\du}{13.814035\du}}
\pgfpathlineto{\pgfpoint{44.317643\du}{13.807828\du}}
\pgfpathlineto{\pgfpoint{44.328231\du}{13.801986\du}}
\pgfpathlineto{\pgfpoint{44.338088\du}{13.796145\du}}
\pgfpathlineto{\pgfpoint{44.348676\du}{13.790303\du}}
\pgfpathlineto{\pgfpoint{44.358899\du}{13.783732\du}}
\pgfpathlineto{\pgfpoint{44.368756\du}{13.777160\du}}
\pgfpathlineto{\pgfpoint{44.378614\du}{13.771318\du}}
\pgfpathlineto{\pgfpoint{44.387741\du}{13.765112\du}}
\pgfpathlineto{\pgfpoint{44.396869\du}{13.758540\du}}
\pgfpathlineto{\pgfpoint{44.406361\du}{13.751968\du}}
\pgfpathlineto{\pgfpoint{44.415124\du}{13.745761\du}}
\pgfpathlineto{\pgfpoint{44.424251\du}{13.739190\du}}
\pgfpathlineto{\pgfpoint{44.432283\du}{13.732618\du}}
\pgfpathlineto{\pgfpoint{44.441046\du}{13.726411\du}}
\pgfpathlineto{\pgfpoint{44.448348\du}{13.719840\du}}
\pgfpathlineto{\pgfpoint{44.456380\du}{13.712903\du}}
\pgfpathlineto{\pgfpoint{44.464412\du}{13.706331\du}}
\pgfpathlineto{\pgfpoint{44.470984\du}{13.699394\du}}
\pgfpathlineto{\pgfpoint{44.478285\du}{13.692822\du}}
\pgfpathlineto{\pgfpoint{44.484857\du}{13.685886\du}}
\pgfpathlineto{\pgfpoint{44.491794\du}{13.678949\du}}
\pgfpathlineto{\pgfpoint{44.498366\du}{13.672377\du}}
\pgfpathlineto{\pgfpoint{44.504572\du}{13.665440\du}}
\pgfpathlineto{\pgfpoint{44.510414\du}{13.658503\du}}
\pgfpathlineto{\pgfpoint{44.515890\du}{13.651567\du}}
\pgfpathlineto{\pgfpoint{44.521732\du}{13.643900\du}}
\pgfpathlineto{\pgfpoint{44.526478\du}{13.636963\du}}
\pgfpathlineto{\pgfpoint{44.531955\du}{13.629661\du}}
\pgfpathlineto{\pgfpoint{44.536701\du}{13.622724\du}}
\pgfpathlineto{\pgfpoint{44.541082\du}{13.615057\du}}
\pgfpathlineto{\pgfpoint{44.545098\du}{13.608120\du}}
\pgfpathlineto{\pgfpoint{44.549114\du}{13.600453\du}}
\pgfpathlineto{\pgfpoint{44.552400\du}{13.592786\du}}
\pgfpathlineto{\pgfpoint{44.556416\du}{13.585484\du}}
\pgfpathlineto{\pgfpoint{44.559702\du}{13.578182\du}}
\pgfpathlineto{\pgfpoint{44.562258\du}{13.570880\du}}
\pgfpathlineto{\pgfpoint{44.565178\du}{13.563213\du}}
\pgfpathlineto{\pgfpoint{44.567004\du}{13.555546\du}}
\pgfpathlineto{\pgfpoint{44.569925\du}{13.547879\du}}
\pgfpathlineto{\pgfpoint{44.571020\du}{13.540212\du}}
\pgfpathlineto{\pgfpoint{44.573211\du}{13.532545\du}}
\pgfpathlineto{\pgfpoint{44.574306\du}{13.525243\du}}
\pgfpathlineto{\pgfpoint{44.575036\du}{13.516846\du}}
\pgfpathlineto{\pgfpoint{44.575766\du}{13.509179\du}}
\pgfpathlineto{\pgfpoint{44.576496\du}{13.501512\du}}
\pgfpathlineto{\pgfpoint{44.576496\du}{13.493480\du}}
\pgfpathlineto{\pgfpoint{44.556416\du}{13.493480\du}}
\pgfpathlineto{\pgfpoint{44.555686\du}{13.500416\du}}
\pgfpathlineto{\pgfpoint{44.555686\du}{13.508084\du}}
\pgfpathlineto{\pgfpoint{44.555321\du}{13.515020\du}}
\pgfpathlineto{\pgfpoint{44.553495\du}{13.521957\du}}
\pgfpathlineto{\pgfpoint{44.552400\du}{13.529259\du}}
\pgfpathlineto{\pgfpoint{44.551670\du}{13.535466\du}}
\pgfpathlineto{\pgfpoint{44.549479\du}{13.542768\du}}
\pgfpathlineto{\pgfpoint{44.548019\du}{13.549705\du}}
\pgfpathlineto{\pgfpoint{44.545828\du}{13.556641\du}}
\pgfpathlineto{\pgfpoint{44.543273\du}{13.563578\du}}
\pgfpathlineto{\pgfpoint{44.540352\du}{13.570880\du}}
\pgfpathlineto{\pgfpoint{44.537796\du}{13.577087\du}}
\pgfpathlineto{\pgfpoint{44.533780\du}{13.584024\du}}
\pgfpathlineto{\pgfpoint{44.530859\du}{13.591326\du}}
\pgfpathlineto{\pgfpoint{44.526843\du}{13.598262\du}}
\pgfpathlineto{\pgfpoint{44.523923\du}{13.604469\du}}
\pgfpathlineto{\pgfpoint{44.519541\du}{13.611771\du}}
\pgfpathlineto{\pgfpoint{44.515160\du}{13.617978\du}}
\pgfpathlineto{\pgfpoint{44.510414\du}{13.625280\du}}
\pgfpathlineto{\pgfpoint{44.505303\du}{13.631486\du}}
\pgfpathlineto{\pgfpoint{44.500556\du}{13.638423\du}}
\pgfpathlineto{\pgfpoint{44.494350\du}{13.644995\du}}
\pgfpathlineto{\pgfpoint{44.488873\du}{13.651567\du}}
\pgfpathlineto{\pgfpoint{44.483762\du}{13.658503\du}}
\pgfpathlineto{\pgfpoint{44.477190\du}{13.665075\du}}
\pgfpathlineto{\pgfpoint{44.470253\du}{13.671282\du}}
\pgfpathlineto{\pgfpoint{44.464412\du}{13.677854\du}}
\pgfpathlineto{\pgfpoint{44.457110\du}{13.684790\du}}
\pgfpathlineto{\pgfpoint{44.450538\du}{13.691362\du}}
\pgfpathlineto{\pgfpoint{44.442871\du}{13.697569\du}}
\pgfpathlineto{\pgfpoint{44.435569\du}{13.704140\du}}
\pgfpathlineto{\pgfpoint{44.427537\du}{13.709982\du}}
\pgfpathlineto{\pgfpoint{44.419870\du}{13.716554\du}}
\pgfpathlineto{\pgfpoint{44.411108\du}{13.722760\du}}
\pgfpathlineto{\pgfpoint{44.402710\du}{13.729332\du}}
\pgfpathlineto{\pgfpoint{44.393948\du}{13.735539\du}}
\pgfpathlineto{\pgfpoint{44.385916\du}{13.741380\du}}
\pgfpathlineto{\pgfpoint{44.377154\du}{13.747952\du}}
\pgfpathlineto{\pgfpoint{44.367661\du}{13.753794\du}}
\pgfpathlineto{\pgfpoint{44.357438\du}{13.760365\du}}
\pgfpathlineto{\pgfpoint{44.348311\du}{13.766207\du}}
\pgfpathlineto{\pgfpoint{44.338088\du}{13.772048\du}}
\pgfpathlineto{\pgfpoint{44.328231\du}{13.777890\du}}
\pgfpathlineto{\pgfpoint{44.318373\du}{13.783732\du}}
\pgfpathlineto{\pgfpoint{44.307420\du}{13.790303\du}}
\pgfpathlineto{\pgfpoint{44.296467\du}{13.795415\du}}
\pgfpathlineto{\pgfpoint{44.286245\du}{13.801256\du}}
\pgfpathlineto{\pgfpoint{44.274196\du}{13.807098\du}}
\pgfpathlineto{\pgfpoint{44.263609\du}{13.812939\du}}
\pgfpathlineto{\pgfpoint{44.251560\du}{13.818781\du}}
\pgfpathlineto{\pgfpoint{44.240242\du}{13.823892\du}}
\pgfpathlineto{\pgfpoint{44.227829\du}{13.829369\du}}
\pgfpathlineto{\pgfpoint{44.215781\du}{13.835210\du}}
\pgfpathlineto{\pgfpoint{44.203368\du}{13.840322\du}}
\pgfpathlineto{\pgfpoint{44.191319\du}{13.845798\du}}
\pgfpathlineto{\pgfpoint{44.178176\du}{13.850909\du}}
\pgfpathlineto{\pgfpoint{44.165398\du}{13.856386\du}}
\pgfpathlineto{\pgfpoint{44.152619\du}{13.862227\du}}
\pgfpathlineto{\pgfpoint{44.139476\du}{13.866609\du}}
\pgfpathlineto{\pgfpoint{44.125967\du}{13.872085\du}}
\pgfpathlineto{\pgfpoint{44.112824\du}{13.877196\du}}
\pgfpathlineto{\pgfpoint{44.098585\du}{13.881943\du}}
\pgfpathlineto{\pgfpoint{44.084711\du}{13.887419\du}}
\pgfpathlineto{\pgfpoint{44.070472\du}{13.891800\du}}
\pgfpathlineto{\pgfpoint{44.055869\du}{13.896546\du}}
\pgfpathlineto{\pgfpoint{44.041995\du}{13.902023\du}}
\pgfpathlineto{\pgfpoint{44.027391\du}{13.906404\du}}
\pgfpathlineto{\pgfpoint{44.013152\du}{13.911150\du}}
\pgfpathlineto{\pgfpoint{43.997453\du}{13.915897\du}}
\pgfpathlineto{\pgfpoint{43.983214\du}{13.919913\du}}
\pgfpathlineto{\pgfpoint{43.967515\du}{13.924659\du}}
\pgfpathlineto{\pgfpoint{43.952546\du}{13.929405\du}}
\pgfpathlineto{\pgfpoint{43.921148\du}{13.937437\du}}
\pgfpathlineto{\pgfpoint{43.889019\du}{13.946200\du}}
\pgfpathlineto{\pgfpoint{43.856891\du}{13.954597\du}}
\pgfpathlineto{\pgfpoint{43.823667\du}{13.962264\du}}
\pgfpathlineto{\pgfpoint{43.789348\du}{13.969931\du}}
\pgfpathlineto{\pgfpoint{43.755394\du}{13.977233\du}}
\pgfpathlineto{\pgfpoint{43.720710\du}{13.984535\du}}
\pgfpathlineto{\pgfpoint{43.684930\du}{13.991472\du}}
\pgfpathlineto{\pgfpoint{43.649516\du}{13.997678\du}}
\pgfpathlineto{\pgfpoint{43.612641\du}{14.003520\du}}
\pgfpathlineto{\pgfpoint{43.576131\du}{14.009726\du}}
\pgfpathlineto{\pgfpoint{43.538892\du}{14.015568\du}}
\pgfpathlineto{\pgfpoint{43.501287\du}{14.020679\du}}
\pgfpathlineto{\pgfpoint{43.462586\du}{14.025791\du}}
\pgfpathlineto{\pgfpoint{43.423886\du}{14.030537\du}}
\pgfpathlineto{\pgfpoint{43.384456\du}{14.034553\du}}
\pgfpathlineto{\pgfpoint{43.345390\du}{14.038934\du}}
\pgfpathlineto{\pgfpoint{43.305230\du}{14.042220\du}}
\pgfpathlineto{\pgfpoint{43.265069\du}{14.045871\du}}
\pgfpathlineto{\pgfpoint{43.223813\du}{14.048792\du}}
\pgfpathlineto{\pgfpoint{43.182922\du}{14.051713\du}}
\pgfpathlineto{\pgfpoint{43.141666\du}{14.053903\du}}
\pgfpathlineto{\pgfpoint{43.100045\du}{14.056459\du}}
\pgfpathlineto{\pgfpoint{43.057329\du}{14.057554\du}}
\pgfpathlineto{\pgfpoint{43.014613\du}{14.059380\du}}
\pgfpathlineto{\pgfpoint{42.972626\du}{14.059745\du}}
\pgfpathlineto{\pgfpoint{42.929180\du}{14.060475\du}}
\pgfpathlineto{\pgfpoint{42.886099\du}{14.060475\du}}
\pgfpathlineto{\pgfpoint{42.886099\du}{14.060475\du}}
\pgfpathlineto{\pgfpoint{42.886099\du}{14.060475\du}}
\pgfpathlineto{\pgfpoint{42.885368\du}{14.061205\du}}
\pgfpathlineto{\pgfpoint{42.883543\du}{14.061205\du}}
\pgfpathlineto{\pgfpoint{42.882448\du}{14.061205\du}}
\pgfpathlineto{\pgfpoint{42.881717\du}{14.061570\du}}
\pgfpathlineto{\pgfpoint{42.881352\du}{14.062300\du}}
\pgfpathlineto{\pgfpoint{42.879892\du}{14.062300\du}}
\pgfpathlineto{\pgfpoint{42.879162\du}{14.063396\du}}
\pgfpathlineto{\pgfpoint{42.878431\du}{14.064126\du}}
\pgfpathlineto{\pgfpoint{42.877336\du}{14.065221\du}}
\pgfpathlineto{\pgfpoint{42.876606\du}{14.067047\du}}
\pgfpathlineto{\pgfpoint{42.876606\du}{14.069237\du}}
\pgfpathlineto{\pgfpoint{42.875876\du}{14.071063\du}}
\pgfpathlineto{\pgfpoint{42.876606\du}{14.072888\du}}
\pgfpathlineto{\pgfpoint{42.876606\du}{14.074349\du}}
\pgfpathlineto{\pgfpoint{42.877336\du}{14.076174\du}}
\pgfpathlineto{\pgfpoint{42.878431\du}{14.078000\du}}
\pgfpathlineto{\pgfpoint{42.879162\du}{14.078730\du}}
\pgfpathlineto{\pgfpoint{42.879892\du}{14.079095\du}}
\pgfpathlineto{\pgfpoint{42.881352\du}{14.079825\du}}
\pgfpathlineto{\pgfpoint{42.881717\du}{14.080190\du}}
\pgfpathlineto{\pgfpoint{42.882448\du}{14.080920\du}}
\pgfpathlineto{\pgfpoint{42.883543\du}{14.080920\du}}
\pgfpathlineto{\pgfpoint{42.885368\du}{14.080920\du}}
\pgfpathlineto{\pgfpoint{42.886099\du}{14.080920\du}}
\pgfusepath{fill}
\pgfsetbuttcap
\pgfsetmiterjoin
\pgfsetdash{}{0pt}
\definecolor{dialinecolor}{rgb}{0.678431, 0.839216, 0.905882}
\pgfsetfillcolor{dialinecolor}
\pgfpathmoveto{\pgfpoint{41.195335\du}{13.493480\du}}
\pgfpathlineto{\pgfpoint{41.195335\du}{13.493480\du}}
\pgfpathlineto{\pgfpoint{41.195335\du}{13.501512\du}}
\pgfpathlineto{\pgfpoint{41.195701\du}{13.509179\du}}
\pgfpathlineto{\pgfpoint{41.196431\du}{13.516846\du}}
\pgfpathlineto{\pgfpoint{41.197526\du}{13.525243\du}}
\pgfpathlineto{\pgfpoint{41.198621\du}{13.532545\du}}
\pgfpathlineto{\pgfpoint{41.200447\du}{13.540212\du}}
\pgfpathlineto{\pgfpoint{41.202272\du}{13.547879\du}}
\pgfpathlineto{\pgfpoint{41.204463\du}{13.555546\du}}
\pgfpathlineto{\pgfpoint{41.206653\du}{13.563213\du}}
\pgfpathlineto{\pgfpoint{41.209209\du}{13.570880\du}}
\pgfpathlineto{\pgfpoint{41.212130\du}{13.578182\du}}
\pgfpathlineto{\pgfpoint{41.215781\du}{13.585484\du}}
\pgfpathlineto{\pgfpoint{41.219067\du}{13.592786\du}}
\pgfpathlineto{\pgfpoint{41.222718\du}{13.600453\du}}
\pgfpathlineto{\pgfpoint{41.227099\du}{13.608120\du}}
\pgfpathlineto{\pgfpoint{41.230750\du}{13.615057\du}}
\pgfpathlineto{\pgfpoint{41.235861\du}{13.622724\du}}
\pgfpathlineto{\pgfpoint{41.239877\du}{13.629661\du}}
\pgfpathlineto{\pgfpoint{41.245354\du}{13.636963\du}}
\pgfpathlineto{\pgfpoint{41.250100\du}{13.643900\du}}
\pgfpathlineto{\pgfpoint{41.255576\du}{13.651567\du}}
\pgfpathlineto{\pgfpoint{41.261418\du}{13.658503\du}}
\pgfpathlineto{\pgfpoint{41.267260\du}{13.665440\du}}
\pgfpathlineto{\pgfpoint{41.273101\du}{13.672377\du}}
\pgfpathlineto{\pgfpoint{41.280038\du}{13.678949\du}}
\pgfpathlineto{\pgfpoint{41.286610\du}{13.685886\du}}
\pgfpathlineto{\pgfpoint{41.293546\du}{13.692822\du}}
\pgfpathlineto{\pgfpoint{41.300483\du}{13.699394\du}}
\pgfpathlineto{\pgfpoint{41.307420\du}{13.706331\du}}
\pgfpathlineto{\pgfpoint{41.316182\du}{13.712903\du}}
\pgfpathlineto{\pgfpoint{41.323119\du}{13.719840\du}}
\pgfpathlineto{\pgfpoint{41.330786\du}{13.726411\du}}
\pgfpathlineto{\pgfpoint{41.339549\du}{13.732618\du}}
\pgfpathlineto{\pgfpoint{41.347946\du}{13.739190\du}}
\pgfpathlineto{\pgfpoint{41.356708\du}{13.745761\du}}
\pgfpathlineto{\pgfpoint{41.365105\du}{13.751968\du}}
\pgfpathlineto{\pgfpoint{41.374963\du}{13.758540\du}}
\pgfpathlineto{\pgfpoint{41.384090\du}{13.765112\du}}
\pgfpathlineto{\pgfpoint{41.393583\du}{13.771318\du}}
\pgfpathlineto{\pgfpoint{41.403076\du}{13.777160\du}}
\pgfpathlineto{\pgfpoint{41.412933\du}{13.783732\du}}
\pgfpathlineto{\pgfpoint{41.423156\du}{13.790303\du}}
\pgfpathlineto{\pgfpoint{41.433379\du}{13.796145\du}}
\pgfpathlineto{\pgfpoint{41.443966\du}{13.801986\du}}
\pgfpathlineto{\pgfpoint{41.454189\du}{13.807828\du}}
\pgfpathlineto{\pgfpoint{41.465142\du}{13.814035\du}}
\pgfpathlineto{\pgfpoint{41.476825\du}{13.819876\du}}
\pgfpathlineto{\pgfpoint{41.488143\du}{13.825353\du}}
\pgfpathlineto{\pgfpoint{41.499826\du}{13.831194\du}}
\pgfpathlineto{\pgfpoint{41.511144\du}{13.837036\du}}
\pgfpathlineto{\pgfpoint{41.522827\du}{13.842877\du}}
\pgfpathlineto{\pgfpoint{41.535241\du}{13.847989\du}}
\pgfpathlineto{\pgfpoint{41.547289\du}{13.853830\du}}
\pgfpathlineto{\pgfpoint{41.560432\du}{13.859672\du}}
\pgfpathlineto{\pgfpoint{41.572480\du}{13.865148\du}}
\pgfpathlineto{\pgfpoint{41.585259\du}{13.870259\du}}
\pgfpathlineto{\pgfpoint{41.599132\du}{13.875736\du}}
\pgfpathlineto{\pgfpoint{41.611546\du}{13.880847\du}}
\pgfpathlineto{\pgfpoint{41.624689\du}{13.885959\du}}
\pgfpathlineto{\pgfpoint{41.638928\du}{13.891435\du}}
\pgfpathlineto{\pgfpoint{41.652072\du}{13.896546\du}}
\pgfpathlineto{\pgfpoint{41.666310\du}{13.901293\du}}
\pgfpathlineto{\pgfpoint{41.680184\du}{13.906404\du}}
\pgfpathlineto{\pgfpoint{41.694788\du}{13.911150\du}}
\pgfpathlineto{\pgfpoint{41.708662\du}{13.916627\du}}
\pgfpathlineto{\pgfpoint{41.724361\du}{13.921373\du}}
\pgfpathlineto{\pgfpoint{41.738234\du}{13.925754\du}}
\pgfpathlineto{\pgfpoint{41.752838\du}{13.930500\du}}
\pgfpathlineto{\pgfpoint{41.768172\du}{13.935247\du}}
\pgfpathlineto{\pgfpoint{41.783871\du}{13.939993\du}}
\pgfpathlineto{\pgfpoint{41.798840\du}{13.944739\du}}
\pgfpathlineto{\pgfpoint{41.814540\du}{13.948755\du}}
\pgfpathlineto{\pgfpoint{41.846303\du}{13.957883\du}}
\pgfpathlineto{\pgfpoint{41.878066\du}{13.966280\du}}
\pgfpathlineto{\pgfpoint{41.911290\du}{13.974312\du}}
\pgfpathlineto{\pgfpoint{41.943784\du}{13.982709\du}}
\pgfpathlineto{\pgfpoint{41.978468\du}{13.990376\du}}
\pgfpathlineto{\pgfpoint{42.012787\du}{13.997313\du}}
\pgfpathlineto{\pgfpoint{42.047471\du}{14.004250\du}}
\pgfpathlineto{\pgfpoint{42.083251\du}{14.011187\du}}
\pgfpathlineto{\pgfpoint{42.119395\du}{14.017759\du}}
\pgfpathlineto{\pgfpoint{42.155905\du}{14.024330\du}}
\pgfpathlineto{\pgfpoint{42.193145\du}{14.030172\du}}
\pgfpathlineto{\pgfpoint{42.230750\du}{14.036013\du}}
\pgfpathlineto{\pgfpoint{42.268720\du}{14.041125\du}}
\pgfpathlineto{\pgfpoint{42.307420\du}{14.045871\du}}
\pgfpathlineto{\pgfpoint{42.346120\du}{14.050617\du}}
\pgfpathlineto{\pgfpoint{42.385186\du}{14.055364\du}}
\pgfpathlineto{\pgfpoint{42.425346\du}{14.059380\du}}
\pgfpathlineto{\pgfpoint{42.465142\du}{14.063396\du}}
\pgfpathlineto{\pgfpoint{42.506033\du}{14.066316\du}}
\pgfpathlineto{\pgfpoint{42.546924\du}{14.069237\du}}
\pgfpathlineto{\pgfpoint{42.588180\du}{14.072158\du}}
\pgfpathlineto{\pgfpoint{42.630166\du}{14.074349\du}}
\pgfpathlineto{\pgfpoint{42.671787\du}{14.076174\du}}
\pgfpathlineto{\pgfpoint{42.714138\du}{14.078000\du}}
\pgfpathlineto{\pgfpoint{42.756124\du}{14.079095\du}}
\pgfpathlineto{\pgfpoint{42.799571\du}{14.080190\du}}
\pgfpathlineto{\pgfpoint{42.842287\du}{14.080920\du}}
\pgfpathlineto{\pgfpoint{42.886099\du}{14.080920\du}}
\pgfpathlineto{\pgfpoint{42.886099\du}{14.060475\du}}
\pgfpathlineto{\pgfpoint{42.843382\du}{14.060475\du}}
\pgfpathlineto{\pgfpoint{42.799936\du}{14.059745\du}}
\pgfpathlineto{\pgfpoint{42.757584\du}{14.059380\du}}
\pgfpathlineto{\pgfpoint{42.714868\du}{14.057554\du}}
\pgfpathlineto{\pgfpoint{42.672517\du}{14.056459\du}}
\pgfpathlineto{\pgfpoint{42.630531\du}{14.053903\du}}
\pgfpathlineto{\pgfpoint{42.589640\du}{14.051713\du}}
\pgfpathlineto{\pgfpoint{42.548384\du}{14.048792\du}}
\pgfpathlineto{\pgfpoint{42.507493\du}{14.045871\du}}
\pgfpathlineto{\pgfpoint{42.467333\du}{14.042220\du}}
\pgfpathlineto{\pgfpoint{42.427537\du}{14.038934\du}}
\pgfpathlineto{\pgfpoint{42.388107\du}{14.034553\du}}
\pgfpathlineto{\pgfpoint{42.348676\du}{14.030537\du}}
\pgfpathlineto{\pgfpoint{42.309611\du}{14.025791\du}}
\pgfpathlineto{\pgfpoint{42.271276\du}{14.020679\du}}
\pgfpathlineto{\pgfpoint{42.233671\du}{14.015568\du}}
\pgfpathlineto{\pgfpoint{42.196431\du}{14.009726\du}}
\pgfpathlineto{\pgfpoint{42.159921\du}{14.003520\du}}
\pgfpathlineto{\pgfpoint{42.122681\du}{13.997678\du}}
\pgfpathlineto{\pgfpoint{42.087632\du}{13.991472\du}}
\pgfpathlineto{\pgfpoint{42.051487\du}{13.984535\du}}
\pgfpathlineto{\pgfpoint{42.016803\du}{13.977233\du}}
\pgfpathlineto{\pgfpoint{41.982484\du}{13.969931\du}}
\pgfpathlineto{\pgfpoint{41.948530\du}{13.962264\du}}
\pgfpathlineto{\pgfpoint{41.915306\du}{13.954597\du}}
\pgfpathlineto{\pgfpoint{41.883908\du}{13.946200\du}}
\pgfpathlineto{\pgfpoint{41.851414\du}{13.937437\du}}
\pgfpathlineto{\pgfpoint{41.820381\du}{13.929405\du}}
\pgfpathlineto{\pgfpoint{41.804682\du}{13.924659\du}}
\pgfpathlineto{\pgfpoint{41.788983\du}{13.919913\du}}
\pgfpathlineto{\pgfpoint{41.774379\du}{13.915897\du}}
\pgfpathlineto{\pgfpoint{41.759410\du}{13.911150\du}}
\pgfpathlineto{\pgfpoint{41.744076\du}{13.906404\du}}
\pgfpathlineto{\pgfpoint{41.729837\du}{13.902023\du}}
\pgfpathlineto{\pgfpoint{41.715598\du}{13.896546\du}}
\pgfpathlineto{\pgfpoint{41.701360\du}{13.891800\du}}
\pgfpathlineto{\pgfpoint{41.687121\du}{13.887419\du}}
\pgfpathlineto{\pgfpoint{41.673247\du}{13.881943\du}}
\pgfpathlineto{\pgfpoint{41.659373\du}{13.877196\du}}
\pgfpathlineto{\pgfpoint{41.645865\du}{13.872085\du}}
\pgfpathlineto{\pgfpoint{41.632356\du}{13.866609\du}}
\pgfpathlineto{\pgfpoint{41.619578\du}{13.862227\du}}
\pgfpathlineto{\pgfpoint{41.606069\du}{13.856386\du}}
\pgfpathlineto{\pgfpoint{41.593291\du}{13.850909\du}}
\pgfpathlineto{\pgfpoint{41.580878\du}{13.845798\du}}
\pgfpathlineto{\pgfpoint{41.567734\du}{13.840322\du}}
\pgfpathlineto{\pgfpoint{41.555686\du}{13.835210\du}}
\pgfpathlineto{\pgfpoint{41.544368\du}{13.829369\du}}
\pgfpathlineto{\pgfpoint{41.531590\du}{13.823892\du}}
\pgfpathlineto{\pgfpoint{41.519906\du}{13.818781\du}}
\pgfpathlineto{\pgfpoint{41.508223\du}{13.812939\du}}
\pgfpathlineto{\pgfpoint{41.497270\du}{13.807098\du}}
\pgfpathlineto{\pgfpoint{41.485587\du}{13.801256\du}}
\pgfpathlineto{\pgfpoint{41.475730\du}{13.795415\du}}
\pgfpathlineto{\pgfpoint{41.464412\du}{13.790303\du}}
\pgfpathlineto{\pgfpoint{41.453824\du}{13.783732\du}}
\pgfpathlineto{\pgfpoint{41.443966\du}{13.777890\du}}
\pgfpathlineto{\pgfpoint{41.433379\du}{13.772048\du}}
\pgfpathlineto{\pgfpoint{41.423521\du}{13.766207\du}}
\pgfpathlineto{\pgfpoint{41.414028\du}{13.760365\du}}
\pgfpathlineto{\pgfpoint{41.404171\du}{13.753794\du}}
\pgfpathlineto{\pgfpoint{41.395043\du}{13.747952\du}}
\pgfpathlineto{\pgfpoint{41.386646\du}{13.741380\du}}
\pgfpathlineto{\pgfpoint{41.377884\du}{13.735539\du}}
\pgfpathlineto{\pgfpoint{41.368756\du}{13.729332\du}}
\pgfpathlineto{\pgfpoint{41.359994\du}{13.722760\du}}
\pgfpathlineto{\pgfpoint{41.351597\du}{13.716554\du}}
\pgfpathlineto{\pgfpoint{41.344295\du}{13.709982\du}}
\pgfpathlineto{\pgfpoint{41.336263\du}{13.704140\du}}
\pgfpathlineto{\pgfpoint{41.328596\du}{13.697569\du}}
\pgfpathlineto{\pgfpoint{41.321659\du}{13.691362\du}}
\pgfpathlineto{\pgfpoint{41.314357\du}{13.684790\du}}
\pgfpathlineto{\pgfpoint{41.307420\du}{13.677854\du}}
\pgfpathlineto{\pgfpoint{41.301214\du}{13.671282\du}}
\pgfpathlineto{\pgfpoint{41.294642\du}{13.665075\du}}
\pgfpathlineto{\pgfpoint{41.288800\du}{13.658503\du}}
\pgfpathlineto{\pgfpoint{41.282594\du}{13.651567\du}}
\pgfpathlineto{\pgfpoint{41.277482\du}{13.644995\du}}
\pgfpathlineto{\pgfpoint{41.271276\du}{13.638423\du}}
\pgfpathlineto{\pgfpoint{41.266894\du}{13.631486\du}}
\pgfpathlineto{\pgfpoint{41.261418\du}{13.625280\du}}
\pgfpathlineto{\pgfpoint{41.257037\du}{13.617978\du}}
\pgfpathlineto{\pgfpoint{41.252656\du}{13.611771\du}}
\pgfpathlineto{\pgfpoint{41.248274\du}{13.604469\du}}
\pgfpathlineto{\pgfpoint{41.244989\du}{13.598262\du}}
\pgfpathlineto{\pgfpoint{41.240973\du}{13.591326\du}}
\pgfpathlineto{\pgfpoint{41.236956\du}{13.584024\du}}
\pgfpathlineto{\pgfpoint{41.234036\du}{13.577087\du}}
\pgfpathlineto{\pgfpoint{41.231480\du}{13.570880\du}}
\pgfpathlineto{\pgfpoint{41.228194\du}{13.563578\du}}
\pgfpathlineto{\pgfpoint{41.226004\du}{13.556641\du}}
\pgfpathlineto{\pgfpoint{41.223813\du}{13.549705\du}}
\pgfpathlineto{\pgfpoint{41.222353\du}{13.542768\du}}
\pgfpathlineto{\pgfpoint{41.220162\du}{13.535466\du}}
\pgfpathlineto{\pgfpoint{41.219067\du}{13.529259\du}}
\pgfpathlineto{\pgfpoint{41.217971\du}{13.521957\du}}
\pgfpathlineto{\pgfpoint{41.216511\du}{13.515020\du}}
\pgfpathlineto{\pgfpoint{41.216146\du}{13.508084\du}}
\pgfpathlineto{\pgfpoint{41.216146\du}{13.500416\du}}
\pgfpathlineto{\pgfpoint{41.215781\du}{13.493480\du}}
\pgfpathlineto{\pgfpoint{41.215781\du}{13.493480\du}}
\pgfpathlineto{\pgfpoint{41.215781\du}{13.493480\du}}
\pgfpathlineto{\pgfpoint{41.215781\du}{13.492384\du}}
\pgfpathlineto{\pgfpoint{41.215781\du}{13.491289\du}}
\pgfpathlineto{\pgfpoint{41.215416\du}{13.489829\du}}
\pgfpathlineto{\pgfpoint{41.215416\du}{13.489464\du}}
\pgfpathlineto{\pgfpoint{41.214321\du}{13.488368\du}}
\pgfpathlineto{\pgfpoint{41.213955\du}{13.486908\du}}
\pgfpathlineto{\pgfpoint{41.213590\du}{13.486543\du}}
\pgfpathlineto{\pgfpoint{41.212495\du}{13.485813\du}}
\pgfpathlineto{\pgfpoint{41.211035\du}{13.484717\du}}
\pgfpathlineto{\pgfpoint{41.209209\du}{13.483987\du}}
\pgfpathlineto{\pgfpoint{41.207384\du}{13.483622\du}}
\pgfpathlineto{\pgfpoint{41.205558\du}{13.483622\du}}
\pgfpathlineto{\pgfpoint{41.203368\du}{13.483622\du}}
\pgfpathlineto{\pgfpoint{41.201907\du}{13.483987\du}}
\pgfpathlineto{\pgfpoint{41.199717\du}{13.484717\du}}
\pgfpathlineto{\pgfpoint{41.197891\du}{13.485813\du}}
\pgfpathlineto{\pgfpoint{41.197526\du}{13.486543\du}}
\pgfpathlineto{\pgfpoint{41.197161\du}{13.486908\du}}
\pgfpathlineto{\pgfpoint{41.196431\du}{13.488368\du}}
\pgfpathlineto{\pgfpoint{41.195701\du}{13.489464\du}}
\pgfpathlineto{\pgfpoint{41.195701\du}{13.489829\du}}
\pgfpathlineto{\pgfpoint{41.195335\du}{13.491289\du}}
\pgfpathlineto{\pgfpoint{41.195335\du}{13.492384\du}}
\pgfpathlineto{\pgfpoint{41.195335\du}{13.493480\du}}
\pgfusepath{fill}
\pgfsetbuttcap
\pgfsetmiterjoin
\pgfsetdash{}{0pt}
\definecolor{dialinecolor}{rgb}{0.678431, 0.839216, 0.905882}
\pgfsetfillcolor{dialinecolor}
\pgfpathmoveto{\pgfpoint{42.886099\du}{12.906039\du}}
\pgfpathlineto{\pgfpoint{42.886099\du}{12.906039\du}}
\pgfpathlineto{\pgfpoint{42.842287\du}{12.906039\du}}
\pgfpathlineto{\pgfpoint{42.799571\du}{12.906769\du}}
\pgfpathlineto{\pgfpoint{42.756124\du}{12.907864\du}}
\pgfpathlineto{\pgfpoint{42.714138\du}{12.908960\du}}
\pgfpathlineto{\pgfpoint{42.671787\du}{12.910785\du}}
\pgfpathlineto{\pgfpoint{42.630166\du}{12.912976\du}}
\pgfpathlineto{\pgfpoint{42.588180\du}{12.915531\du}}
\pgfpathlineto{\pgfpoint{42.546924\du}{12.917722\du}}
\pgfpathlineto{\pgfpoint{42.506033\du}{12.920643\du}}
\pgfpathlineto{\pgfpoint{42.465142\du}{12.924294\du}}
\pgfpathlineto{\pgfpoint{42.425346\du}{12.927945\du}}
\pgfpathlineto{\pgfpoint{42.385186\du}{12.931961\du}}
\pgfpathlineto{\pgfpoint{42.346120\du}{12.936342\du}}
\pgfpathlineto{\pgfpoint{42.307420\du}{12.941088\du}}
\pgfpathlineto{\pgfpoint{42.268720\du}{12.946200\du}}
\pgfpathlineto{\pgfpoint{42.230750\du}{12.951676\du}}
\pgfpathlineto{\pgfpoint{42.193145\du}{12.956787\du}}
\pgfpathlineto{\pgfpoint{42.155905\du}{12.963359\du}}
\pgfpathlineto{\pgfpoint{42.119395\du}{12.969201\du}}
\pgfpathlineto{\pgfpoint{42.083251\du}{12.975772\du}}
\pgfpathlineto{\pgfpoint{42.047471\du}{12.982709\du}}
\pgfpathlineto{\pgfpoint{42.012787\du}{12.989646\du}}
\pgfpathlineto{\pgfpoint{41.978468\du}{12.997313\du}}
\pgfpathlineto{\pgfpoint{41.943784\du}{13.004980\du}}
\pgfpathlineto{\pgfpoint{41.911290\du}{13.013012\du}}
\pgfpathlineto{\pgfpoint{41.878066\du}{13.021410\du}}
\pgfpathlineto{\pgfpoint{41.846303\du}{13.029442\du}}
\pgfpathlineto{\pgfpoint{41.814540\du}{13.038204\du}}
\pgfpathlineto{\pgfpoint{41.783871\du}{13.047697\du}}
\pgfpathlineto{\pgfpoint{41.752838\du}{13.056459\du}}
\pgfpathlineto{\pgfpoint{41.738234\du}{13.061205\du}}
\pgfpathlineto{\pgfpoint{41.724361\du}{13.066316\du}}
\pgfpathlineto{\pgfpoint{41.708662\du}{13.071063\du}}
\pgfpathlineto{\pgfpoint{41.694788\du}{13.075809\du}}
\pgfpathlineto{\pgfpoint{41.680184\du}{13.080920\du}}
\pgfpathlineto{\pgfpoint{41.666310\du}{13.085667\du}}
\pgfpathlineto{\pgfpoint{41.652072\du}{13.090778\du}}
\pgfpathlineto{\pgfpoint{41.638928\du}{13.095524\du}}
\pgfpathlineto{\pgfpoint{41.624689\du}{13.101001\du}}
\pgfpathlineto{\pgfpoint{41.611546\du}{13.106112\du}}
\pgfpathlineto{\pgfpoint{41.599132\du}{13.111223\du}}
\pgfpathlineto{\pgfpoint{41.585259\du}{13.117065\du}}
\pgfpathlineto{\pgfpoint{41.572480\du}{13.122541\du}}
\pgfpathlineto{\pgfpoint{41.560432\du}{13.127653\du}}
\pgfpathlineto{\pgfpoint{41.547289\du}{13.133494\du}}
\pgfpathlineto{\pgfpoint{41.535241\du}{13.138971\du}}
\pgfpathlineto{\pgfpoint{41.522827\du}{13.144812\du}}
\pgfpathlineto{\pgfpoint{41.511144\du}{13.149924\du}}
\pgfpathlineto{\pgfpoint{41.499826\du}{13.155765\du}}
\pgfpathlineto{\pgfpoint{41.488143\du}{13.161607\du}}
\pgfpathlineto{\pgfpoint{41.476825\du}{13.167448\du}}
\pgfpathlineto{\pgfpoint{41.465142\du}{13.173290\du}}
\pgfpathlineto{\pgfpoint{41.454189\du}{13.179131\du}}
\pgfpathlineto{\pgfpoint{41.443966\du}{13.185703\du}}
\pgfpathlineto{\pgfpoint{41.433379\du}{13.191545\du}}
\pgfpathlineto{\pgfpoint{41.423156\du}{13.197386\du}}
\pgfpathlineto{\pgfpoint{41.412933\du}{13.203958\du}}
\pgfpathlineto{\pgfpoint{41.403076\du}{13.209799\du}}
\pgfpathlineto{\pgfpoint{41.393583\du}{13.216006\du}}
\pgfpathlineto{\pgfpoint{41.384090\du}{13.222578\du}}
\pgfpathlineto{\pgfpoint{41.374963\du}{13.229150\du}}
\pgfpathlineto{\pgfpoint{41.365105\du}{13.234991\du}}
\pgfpathlineto{\pgfpoint{41.356708\du}{13.241198\du}}
\pgfpathlineto{\pgfpoint{41.347946\du}{13.247770\du}}
\pgfpathlineto{\pgfpoint{41.339549\du}{13.253976\du}}
\pgfpathlineto{\pgfpoint{41.330786\du}{13.261278\du}}
\pgfpathlineto{\pgfpoint{41.323119\du}{13.267485\du}}
\pgfpathlineto{\pgfpoint{41.316182\du}{13.274056\du}}
\pgfpathlineto{\pgfpoint{41.307420\du}{13.280993\du}}
\pgfpathlineto{\pgfpoint{41.300483\du}{13.287565\du}}
\pgfpathlineto{\pgfpoint{41.293546\du}{13.294502\du}}
\pgfpathlineto{\pgfpoint{41.286610\du}{13.301439\du}}
\pgfpathlineto{\pgfpoint{41.280038\du}{13.308010\du}}
\pgfpathlineto{\pgfpoint{41.273101\du}{13.314947\du}}
\pgfpathlineto{\pgfpoint{41.267260\du}{13.321884\du}}
\pgfpathlineto{\pgfpoint{41.261418\du}{13.329186\du}}
\pgfpathlineto{\pgfpoint{41.255576\du}{13.336123\du}}
\pgfpathlineto{\pgfpoint{41.250100\du}{13.343060\du}}
\pgfpathlineto{\pgfpoint{41.245354\du}{13.349997\du}}
\pgfpathlineto{\pgfpoint{41.239877\du}{13.357664\du}}
\pgfpathlineto{\pgfpoint{41.235861\du}{13.364600\du}}
\pgfpathlineto{\pgfpoint{41.230750\du}{13.371902\du}}
\pgfpathlineto{\pgfpoint{41.227099\du}{13.379204\du}}
\pgfpathlineto{\pgfpoint{41.222718\du}{13.386506\du}}
\pgfpathlineto{\pgfpoint{41.219067\du}{13.394173\du}}
\pgfpathlineto{\pgfpoint{41.215781\du}{13.401475\du}}
\pgfpathlineto{\pgfpoint{41.212130\du}{13.409142\du}}
\pgfpathlineto{\pgfpoint{41.209209\du}{13.416809\du}}
\pgfpathlineto{\pgfpoint{41.206653\du}{13.423746\du}}
\pgfpathlineto{\pgfpoint{41.204463\du}{13.431413\du}}
\pgfpathlineto{\pgfpoint{41.202272\du}{13.439080\du}}
\pgfpathlineto{\pgfpoint{41.200447\du}{13.447112\du}}
\pgfpathlineto{\pgfpoint{41.198621\du}{13.454779\du}}
\pgfpathlineto{\pgfpoint{41.197526\du}{13.462446\du}}
\pgfpathlineto{\pgfpoint{41.196431\du}{13.470113\du}}
\pgfpathlineto{\pgfpoint{41.195701\du}{13.478146\du}}
\pgfpathlineto{\pgfpoint{41.195335\du}{13.485813\du}}
\pgfpathlineto{\pgfpoint{41.195335\du}{13.493480\du}}
\pgfpathlineto{\pgfpoint{41.215781\du}{13.493480\du}}
\pgfpathlineto{\pgfpoint{41.216146\du}{13.486543\du}}
\pgfpathlineto{\pgfpoint{41.216146\du}{13.479606\du}}
\pgfpathlineto{\pgfpoint{41.216511\du}{13.472304\du}}
\pgfpathlineto{\pgfpoint{41.217971\du}{13.465367\du}}
\pgfpathlineto{\pgfpoint{41.219067\du}{13.458430\du}}
\pgfpathlineto{\pgfpoint{41.220162\du}{13.451494\du}}
\pgfpathlineto{\pgfpoint{41.222353\du}{13.444192\du}}
\pgfpathlineto{\pgfpoint{41.223813\du}{13.437985\du}}
\pgfpathlineto{\pgfpoint{41.226004\du}{13.430683\du}}
\pgfpathlineto{\pgfpoint{41.228194\du}{13.423746\du}}
\pgfpathlineto{\pgfpoint{41.231480\du}{13.416809\du}}
\pgfpathlineto{\pgfpoint{41.234036\du}{13.409872\du}}
\pgfpathlineto{\pgfpoint{41.236956\du}{13.402936\du}}
\pgfpathlineto{\pgfpoint{41.240973\du}{13.396364\du}}
\pgfpathlineto{\pgfpoint{41.244989\du}{13.389427\du}}
\pgfpathlineto{\pgfpoint{41.248274\du}{13.382855\du}}
\pgfpathlineto{\pgfpoint{41.252291\du}{13.375918\du}}
\pgfpathlineto{\pgfpoint{41.257037\du}{13.368982\du}}
\pgfpathlineto{\pgfpoint{41.261418\du}{13.362410\du}}
\pgfpathlineto{\pgfpoint{41.266894\du}{13.355838\du}}
\pgfpathlineto{\pgfpoint{41.271276\du}{13.348901\du}}
\pgfpathlineto{\pgfpoint{41.277482\du}{13.342330\du}}
\pgfpathlineto{\pgfpoint{41.282594\du}{13.335393\du}}
\pgfpathlineto{\pgfpoint{41.288800\du}{13.329186\du}}
\pgfpathlineto{\pgfpoint{41.294642\du}{13.322614\du}}
\pgfpathlineto{\pgfpoint{41.301214\du}{13.315678\du}}
\pgfpathlineto{\pgfpoint{41.307420\du}{13.309106\du}}
\pgfpathlineto{\pgfpoint{41.314357\du}{13.302899\du}}
\pgfpathlineto{\pgfpoint{41.321659\du}{13.296327\du}}
\pgfpathlineto{\pgfpoint{41.328596\du}{13.289756\du}}
\pgfpathlineto{\pgfpoint{41.336263\du}{13.283549\du}}
\pgfpathlineto{\pgfpoint{41.344295\du}{13.276977\du}}
\pgfpathlineto{\pgfpoint{41.351597\du}{13.270406\du}}
\pgfpathlineto{\pgfpoint{41.359994\du}{13.264199\du}}
\pgfpathlineto{\pgfpoint{41.368756\du}{13.258357\du}}
\pgfpathlineto{\pgfpoint{41.377884\du}{13.251420\du}}
\pgfpathlineto{\pgfpoint{41.386646\du}{13.245944\du}}
\pgfpathlineto{\pgfpoint{41.395043\du}{13.239372\du}}
\pgfpathlineto{\pgfpoint{41.404171\du}{13.233166\du}}
\pgfpathlineto{\pgfpoint{41.414028\du}{13.227324\du}}
\pgfpathlineto{\pgfpoint{41.423521\du}{13.221483\du}}
\pgfpathlineto{\pgfpoint{41.433379\du}{13.214911\du}}
\pgfpathlineto{\pgfpoint{41.443966\du}{13.209069\du}}
\pgfpathlineto{\pgfpoint{41.453824\du}{13.203228\du}}
\pgfpathlineto{\pgfpoint{41.464412\du}{13.197386\du}}
\pgfpathlineto{\pgfpoint{41.475730\du}{13.191545\du}}
\pgfpathlineto{\pgfpoint{41.485587\du}{13.185703\du}}
\pgfpathlineto{\pgfpoint{41.497270\du}{13.179862\du}}
\pgfpathlineto{\pgfpoint{41.508223\du}{13.174750\du}}
\pgfpathlineto{\pgfpoint{41.519906\du}{13.168544\du}}
\pgfpathlineto{\pgfpoint{41.531590\du}{13.162702\du}}
\pgfpathlineto{\pgfpoint{41.544368\du}{13.157591\du}}
\pgfpathlineto{\pgfpoint{41.555686\du}{13.152479\du}}
\pgfpathlineto{\pgfpoint{41.567734\du}{13.146638\du}}
\pgfpathlineto{\pgfpoint{41.580878\du}{13.141161\du}}
\pgfpathlineto{\pgfpoint{41.593291\du}{13.136050\du}}
\pgfpathlineto{\pgfpoint{41.606069\du}{13.130573\du}}
\pgfpathlineto{\pgfpoint{41.619578\du}{13.125462\du}}
\pgfpathlineto{\pgfpoint{41.632356\du}{13.119986\du}}
\pgfpathlineto{\pgfpoint{41.645865\du}{13.114509\du}}
\pgfpathlineto{\pgfpoint{41.659373\du}{13.110128\du}}
\pgfpathlineto{\pgfpoint{41.673247\du}{13.105017\du}}
\pgfpathlineto{\pgfpoint{41.687121\du}{13.100270\du}}
\pgfpathlineto{\pgfpoint{41.701360\du}{13.095159\du}}
\pgfpathlineto{\pgfpoint{41.715598\du}{13.090413\du}}
\pgfpathlineto{\pgfpoint{41.729837\du}{13.085667\du}}
\pgfpathlineto{\pgfpoint{41.744076\du}{13.080920\du}}
\pgfpathlineto{\pgfpoint{41.759410\du}{13.076174\du}}
\pgfpathlineto{\pgfpoint{41.788983\du}{13.067047\du}}
\pgfpathlineto{\pgfpoint{41.820381\du}{13.058284\du}}
\pgfpathlineto{\pgfpoint{41.851414\du}{13.049522\du}}
\pgfpathlineto{\pgfpoint{41.883908\du}{13.041125\du}}
\pgfpathlineto{\pgfpoint{41.915306\du}{13.033093\du}}
\pgfpathlineto{\pgfpoint{41.948530\du}{13.024695\du}}
\pgfpathlineto{\pgfpoint{41.982484\du}{13.017028\du}}
\pgfpathlineto{\pgfpoint{42.016803\du}{13.010092\du}}
\pgfpathlineto{\pgfpoint{42.051487\du}{13.003155\du}}
\pgfpathlineto{\pgfpoint{42.087632\du}{12.995853\du}}
\pgfpathlineto{\pgfpoint{42.122681\du}{12.989646\du}}
\pgfpathlineto{\pgfpoint{42.159921\du}{12.982709\du}}
\pgfpathlineto{\pgfpoint{42.196431\du}{12.977233\du}}
\pgfpathlineto{\pgfpoint{42.233671\du}{12.971391\du}}
\pgfpathlineto{\pgfpoint{42.271276\du}{12.966280\du}}
\pgfpathlineto{\pgfpoint{42.309611\du}{12.961534\du}}
\pgfpathlineto{\pgfpoint{42.348676\du}{12.956787\du}}
\pgfpathlineto{\pgfpoint{42.388107\du}{12.952406\du}}
\pgfpathlineto{\pgfpoint{42.427537\du}{12.948755\du}}
\pgfpathlineto{\pgfpoint{42.467333\du}{12.944739\du}}
\pgfpathlineto{\pgfpoint{42.507493\du}{12.941818\du}}
\pgfpathlineto{\pgfpoint{42.548384\du}{12.938167\du}}
\pgfpathlineto{\pgfpoint{42.589640\du}{12.935247\du}}
\pgfpathlineto{\pgfpoint{42.630531\du}{12.933056\du}}
\pgfpathlineto{\pgfpoint{42.672517\du}{12.931231\du}}
\pgfpathlineto{\pgfpoint{42.714868\du}{12.929405\du}}
\pgfpathlineto{\pgfpoint{42.757584\du}{12.927945\du}}
\pgfpathlineto{\pgfpoint{42.799936\du}{12.927580\du}}
\pgfpathlineto{\pgfpoint{42.843382\du}{12.927215\du}}
\pgfpathlineto{\pgfpoint{42.886099\du}{12.926484\du}}
\pgfpathlineto{\pgfpoint{42.886099\du}{12.926484\du}}
\pgfpathlineto{\pgfpoint{42.886099\du}{12.926484\du}}
\pgfpathlineto{\pgfpoint{42.887194\du}{12.926484\du}}
\pgfpathlineto{\pgfpoint{42.888289\du}{12.926484\du}}
\pgfpathlineto{\pgfpoint{42.889749\du}{12.925754\du}}
\pgfpathlineto{\pgfpoint{42.890845\du}{12.925754\du}}
\pgfpathlineto{\pgfpoint{42.891210\du}{12.925389\du}}
\pgfpathlineto{\pgfpoint{42.892305\du}{12.924659\du}}
\pgfpathlineto{\pgfpoint{42.893035\du}{12.924294\du}}
\pgfpathlineto{\pgfpoint{42.894131\du}{12.923564\du}}
\pgfpathlineto{\pgfpoint{42.895226\du}{12.921738\du}}
\pgfpathlineto{\pgfpoint{42.895956\du}{12.919913\du}}
\pgfpathlineto{\pgfpoint{42.895956\du}{12.918452\du}}
\pgfpathlineto{\pgfpoint{42.896686\du}{12.916627\du}}
\pgfpathlineto{\pgfpoint{42.895956\du}{12.914801\du}}
\pgfpathlineto{\pgfpoint{42.895956\du}{12.912611\du}}
\pgfpathlineto{\pgfpoint{42.895226\du}{12.910785\du}}
\pgfpathlineto{\pgfpoint{42.894131\du}{12.908960\du}}
\pgfpathlineto{\pgfpoint{42.893035\du}{12.908230\du}}
\pgfpathlineto{\pgfpoint{42.892305\du}{12.907864\du}}
\pgfpathlineto{\pgfpoint{42.891210\du}{12.907134\du}}
\pgfpathlineto{\pgfpoint{42.890845\du}{12.906769\du}}
\pgfpathlineto{\pgfpoint{42.889749\du}{12.906769\du}}
\pgfpathlineto{\pgfpoint{42.888289\du}{12.906039\du}}
\pgfpathlineto{\pgfpoint{42.887194\du}{12.906039\du}}
\pgfpathlineto{\pgfpoint{42.886099\du}{12.906039\du}}
\pgfusepath{fill}
\pgfsetbuttcap
\pgfsetmiterjoin
\pgfsetdash{}{0pt}
\definecolor{dialinecolor}{rgb}{0.678431, 0.839216, 0.905882}
\pgfsetfillcolor{dialinecolor}
\pgfpathmoveto{\pgfpoint{44.576496\du}{13.493480\du}}
\pgfpathlineto{\pgfpoint{44.576496\du}{13.485813\du}}
\pgfpathlineto{\pgfpoint{44.575766\du}{13.478146\du}}
\pgfpathlineto{\pgfpoint{44.575036\du}{13.470113\du}}
\pgfpathlineto{\pgfpoint{44.574306\du}{13.462446\du}}
\pgfpathlineto{\pgfpoint{44.573211\du}{13.454779\du}}
\pgfpathlineto{\pgfpoint{44.571020\du}{13.447112\du}}
\pgfpathlineto{\pgfpoint{44.569925\du}{13.439080\du}}
\pgfpathlineto{\pgfpoint{44.567004\du}{13.431413\du}}
\pgfpathlineto{\pgfpoint{44.565178\du}{13.423746\du}}
\pgfpathlineto{\pgfpoint{44.562258\du}{13.416809\du}}
\pgfpathlineto{\pgfpoint{44.559702\du}{13.409142\du}}
\pgfpathlineto{\pgfpoint{44.556416\du}{13.401475\du}}
\pgfpathlineto{\pgfpoint{44.552400\du}{13.394173\du}}
\pgfpathlineto{\pgfpoint{44.549114\du}{13.386506\du}}
\pgfpathlineto{\pgfpoint{44.545098\du}{13.379204\du}}
\pgfpathlineto{\pgfpoint{44.541082\du}{13.371902\du}}
\pgfpathlineto{\pgfpoint{44.536701\du}{13.364600\du}}
\pgfpathlineto{\pgfpoint{44.531955\du}{13.357664\du}}
\pgfpathlineto{\pgfpoint{44.526478\du}{13.349997\du}}
\pgfpathlineto{\pgfpoint{44.521732\du}{13.343060\du}}
\pgfpathlineto{\pgfpoint{44.515890\du}{13.336123\du}}
\pgfpathlineto{\pgfpoint{44.510414\du}{13.329186\du}}
\pgfpathlineto{\pgfpoint{44.504572\du}{13.321884\du}}
\pgfpathlineto{\pgfpoint{44.498366\du}{13.314947\du}}
\pgfpathlineto{\pgfpoint{44.491794\du}{13.308010\du}}
\pgfpathlineto{\pgfpoint{44.484857\du}{13.301439\du}}
\pgfpathlineto{\pgfpoint{44.478285\du}{13.294502\du}}
\pgfpathlineto{\pgfpoint{44.470984\du}{13.287565\du}}
\pgfpathlineto{\pgfpoint{44.464412\du}{13.280993\du}}
\pgfpathlineto{\pgfpoint{44.456380\du}{13.274056\du}}
\pgfpathlineto{\pgfpoint{44.448348\du}{13.267485\du}}
\pgfpathlineto{\pgfpoint{44.441046\du}{13.261278\du}}
\pgfpathlineto{\pgfpoint{44.432283\du}{13.253976\du}}
\pgfpathlineto{\pgfpoint{44.424251\du}{13.247770\du}}
\pgfpathlineto{\pgfpoint{44.415124\du}{13.241198\du}}
\pgfpathlineto{\pgfpoint{44.406361\du}{13.234991\du}}
\pgfpathlineto{\pgfpoint{44.396869\du}{13.229150\du}}
\pgfpathlineto{\pgfpoint{44.387741\du}{13.222578\du}}
\pgfpathlineto{\pgfpoint{44.378614\du}{13.216006\du}}
\pgfpathlineto{\pgfpoint{44.368756\du}{13.209799\du}}
\pgfpathlineto{\pgfpoint{44.358899\du}{13.203958\du}}
\pgfpathlineto{\pgfpoint{44.348676\du}{13.197386\du}}
\pgfpathlineto{\pgfpoint{44.338088\du}{13.191545\du}}
\pgfpathlineto{\pgfpoint{44.328231\du}{13.185703\du}}
\pgfpathlineto{\pgfpoint{44.317643\du}{13.179131\du}}
\pgfpathlineto{\pgfpoint{44.306325\du}{13.173290\du}}
\pgfpathlineto{\pgfpoint{44.295007\du}{13.167448\du}}
\pgfpathlineto{\pgfpoint{44.283324\du}{13.161607\du}}
\pgfpathlineto{\pgfpoint{44.272371\du}{13.155765\du}}
\pgfpathlineto{\pgfpoint{44.260323\du}{13.149924\du}}
\pgfpathlineto{\pgfpoint{44.249005\du}{13.144812\du}}
\pgfpathlineto{\pgfpoint{44.236591\du}{13.138971\du}}
\pgfpathlineto{\pgfpoint{44.224178\du}{13.133494\du}}
\pgfpathlineto{\pgfpoint{44.211400\du}{13.127653\du}}
\pgfpathlineto{\pgfpoint{44.199352\du}{13.122541\du}}
\pgfpathlineto{\pgfpoint{44.186573\du}{13.117065\du}}
\pgfpathlineto{\pgfpoint{44.173065\du}{13.111223\du}}
\pgfpathlineto{\pgfpoint{44.159921\du}{13.106112\du}}
\pgfpathlineto{\pgfpoint{44.146778\du}{13.101001\du}}
\pgfpathlineto{\pgfpoint{44.132539\du}{13.095524\du}}
\pgfpathlineto{\pgfpoint{44.119395\du}{13.090778\du}}
\pgfpathlineto{\pgfpoint{44.105157\du}{13.085667\du}}
\pgfpathlineto{\pgfpoint{44.091283\du}{13.080920\du}}
\pgfpathlineto{\pgfpoint{44.077044\du}{13.075809\du}}
\pgfpathlineto{\pgfpoint{44.063170\du}{13.071063\du}}
\pgfpathlineto{\pgfpoint{44.048201\du}{13.066316\du}}
\pgfpathlineto{\pgfpoint{44.033233\du}{13.061205\du}}
\pgfpathlineto{\pgfpoint{44.018994\du}{13.056459\du}}
\pgfpathlineto{\pgfpoint{43.988326\du}{13.047697\du}}
\pgfpathlineto{\pgfpoint{43.957657\du}{13.038204\du}}
\pgfpathlineto{\pgfpoint{43.926259\du}{13.029442\du}}
\pgfpathlineto{\pgfpoint{43.894496\du}{13.021410\du}}
\pgfpathlineto{\pgfpoint{43.861637\du}{13.013012\du}}
\pgfpathlineto{\pgfpoint{43.828048\du}{13.004980\du}}
\pgfpathlineto{\pgfpoint{43.794094\du}{12.997313\du}}
\pgfpathlineto{\pgfpoint{43.759410\du}{12.989646\du}}
\pgfpathlineto{\pgfpoint{43.724726\du}{12.982709\du}}
\pgfpathlineto{\pgfpoint{43.689311\du}{12.975772\du}}
\pgfpathlineto{\pgfpoint{43.652802\du}{12.969201\du}}
\pgfpathlineto{\pgfpoint{43.616292\du}{12.963359\du}}
\pgfpathlineto{\pgfpoint{43.579417\du}{12.956787\du}}
\pgfpathlineto{\pgfpoint{43.542177\du}{12.951676\du}}
\pgfpathlineto{\pgfpoint{43.503112\du}{12.946200\du}}
\pgfpathlineto{\pgfpoint{43.465142\du}{12.941088\du}}
\pgfpathlineto{\pgfpoint{43.425712\du}{12.936342\du}}
\pgfpathlineto{\pgfpoint{43.387011\du}{12.931961\du}}
\pgfpathlineto{\pgfpoint{43.346851\du}{12.927945\du}}
\pgfpathlineto{\pgfpoint{43.307055\du}{12.924294\du}}
\pgfpathlineto{\pgfpoint{43.266164\du}{12.920643\du}}
\pgfpathlineto{\pgfpoint{43.225639\du}{12.917722\du}}
\pgfpathlineto{\pgfpoint{43.184017\du}{12.915531\du}}
\pgfpathlineto{\pgfpoint{43.142396\du}{12.912976\du}}
\pgfpathlineto{\pgfpoint{43.100775\du}{12.910785\du}}
\pgfpathlineto{\pgfpoint{43.058059\du}{12.908960\du}}
\pgfpathlineto{\pgfpoint{43.016073\du}{12.907864\du}}
\pgfpathlineto{\pgfpoint{42.972992\du}{12.906769\du}}
\pgfpathlineto{\pgfpoint{42.929545\du}{12.906039\du}}
\pgfpathlineto{\pgfpoint{42.886099\du}{12.906039\du}}
\pgfpathlineto{\pgfpoint{42.886099\du}{12.926484\du}}
\pgfpathlineto{\pgfpoint{42.929180\du}{12.927215\du}}
\pgfpathlineto{\pgfpoint{42.972626\du}{12.927580\du}}
\pgfpathlineto{\pgfpoint{43.014613\du}{12.927945\du}}
\pgfpathlineto{\pgfpoint{43.057329\du}{12.929405\du}}
\pgfpathlineto{\pgfpoint{43.100045\du}{12.931231\du}}
\pgfpathlineto{\pgfpoint{43.141666\du}{12.933056\du}}
\pgfpathlineto{\pgfpoint{43.182922\du}{12.935247\du}}
\pgfpathlineto{\pgfpoint{43.223813\du}{12.938167\du}}
\pgfpathlineto{\pgfpoint{43.265069\du}{12.941818\du}}
\pgfpathlineto{\pgfpoint{43.305230\du}{12.944739\du}}
\pgfpathlineto{\pgfpoint{43.345390\du}{12.948755\du}}
\pgfpathlineto{\pgfpoint{43.384456\du}{12.952406\du}}
\pgfpathlineto{\pgfpoint{43.423886\du}{12.956787\du}}
\pgfpathlineto{\pgfpoint{43.462586\du}{12.961534\du}}
\pgfpathlineto{\pgfpoint{43.501287\du}{12.966280\du}}
\pgfpathlineto{\pgfpoint{43.538892\du}{12.971391\du}}
\pgfpathlineto{\pgfpoint{43.576131\du}{12.977233\du}}
\pgfpathlineto{\pgfpoint{43.612641\du}{12.982709\du}}
\pgfpathlineto{\pgfpoint{43.649516\du}{12.989646\du}}
\pgfpathlineto{\pgfpoint{43.684930\du}{12.995853\du}}
\pgfpathlineto{\pgfpoint{43.720710\du}{13.003155\du}}
\pgfpathlineto{\pgfpoint{43.755394\du}{13.010092\du}}
\pgfpathlineto{\pgfpoint{43.789348\du}{13.017028\du}}
\pgfpathlineto{\pgfpoint{43.823667\du}{13.024695\du}}
\pgfpathlineto{\pgfpoint{43.856891\du}{13.033093\du}}
\pgfpathlineto{\pgfpoint{43.889019\du}{13.041125\du}}
\pgfpathlineto{\pgfpoint{43.921148\du}{13.049522\du}}
\pgfpathlineto{\pgfpoint{43.952546\du}{13.058284\du}}
\pgfpathlineto{\pgfpoint{43.983214\du}{13.067047\du}}
\pgfpathlineto{\pgfpoint{44.013152\du}{13.076174\du}}
\pgfpathlineto{\pgfpoint{44.027391\du}{13.080920\du}}
\pgfpathlineto{\pgfpoint{44.041995\du}{13.085667\du}}
\pgfpathlineto{\pgfpoint{44.055869\du}{13.090413\du}}
\pgfpathlineto{\pgfpoint{44.070472\du}{13.095159\du}}
\pgfpathlineto{\pgfpoint{44.084711\du}{13.100270\du}}
\pgfpathlineto{\pgfpoint{44.098585\du}{13.105017\du}}
\pgfpathlineto{\pgfpoint{44.112824\du}{13.110128\du}}
\pgfpathlineto{\pgfpoint{44.125967\du}{13.114509\du}}
\pgfpathlineto{\pgfpoint{44.139476\du}{13.119986\du}}
\pgfpathlineto{\pgfpoint{44.152619\du}{13.125462\du}}
\pgfpathlineto{\pgfpoint{44.165398\du}{13.130573\du}}
\pgfpathlineto{\pgfpoint{44.178176\du}{13.136050\du}}
\pgfpathlineto{\pgfpoint{44.191319\du}{13.141161\du}}
\pgfpathlineto{\pgfpoint{44.203368\du}{13.146638\du}}
\pgfpathlineto{\pgfpoint{44.215781\du}{13.152479\du}}
\pgfpathlineto{\pgfpoint{44.227829\du}{13.157591\du}}
\pgfpathlineto{\pgfpoint{44.240242\du}{13.162702\du}}
\pgfpathlineto{\pgfpoint{44.251560\du}{13.168544\du}}
\pgfpathlineto{\pgfpoint{44.263609\du}{13.174750\du}}
\pgfpathlineto{\pgfpoint{44.274196\du}{13.179862\du}}
\pgfpathlineto{\pgfpoint{44.286245\du}{13.185703\du}}
\pgfpathlineto{\pgfpoint{44.296467\du}{13.191545\du}}
\pgfpathlineto{\pgfpoint{44.307420\du}{13.197386\du}}
\pgfpathlineto{\pgfpoint{44.318373\du}{13.203228\du}}
\pgfpathlineto{\pgfpoint{44.328231\du}{13.209069\du}}
\pgfpathlineto{\pgfpoint{44.338088\du}{13.214911\du}}
\pgfpathlineto{\pgfpoint{44.348311\du}{13.221483\du}}
\pgfpathlineto{\pgfpoint{44.357438\du}{13.227324\du}}
\pgfpathlineto{\pgfpoint{44.367661\du}{13.233166\du}}
\pgfpathlineto{\pgfpoint{44.377154\du}{13.239372\du}}
\pgfpathlineto{\pgfpoint{44.385916\du}{13.245944\du}}
\pgfpathlineto{\pgfpoint{44.393948\du}{13.251420\du}}
\pgfpathlineto{\pgfpoint{44.402710\du}{13.258357\du}}
\pgfpathlineto{\pgfpoint{44.411108\du}{13.264199\du}}
\pgfpathlineto{\pgfpoint{44.419870\du}{13.270406\du}}
\pgfpathlineto{\pgfpoint{44.427537\du}{13.276977\du}}
\pgfpathlineto{\pgfpoint{44.435569\du}{13.283549\du}}
\pgfpathlineto{\pgfpoint{44.442871\du}{13.289756\du}}
\pgfpathlineto{\pgfpoint{44.450538\du}{13.296327\du}}
\pgfpathlineto{\pgfpoint{44.457110\du}{13.302899\du}}
\pgfpathlineto{\pgfpoint{44.464412\du}{13.309106\du}}
\pgfpathlineto{\pgfpoint{44.470253\du}{13.315678\du}}
\pgfpathlineto{\pgfpoint{44.477190\du}{13.322614\du}}
\pgfpathlineto{\pgfpoint{44.483762\du}{13.329186\du}}
\pgfpathlineto{\pgfpoint{44.488873\du}{13.335393\du}}
\pgfpathlineto{\pgfpoint{44.494350\du}{13.342330\du}}
\pgfpathlineto{\pgfpoint{44.500556\du}{13.348901\du}}
\pgfpathlineto{\pgfpoint{44.505303\du}{13.355838\du}}
\pgfpathlineto{\pgfpoint{44.510414\du}{13.362410\du}}
\pgfpathlineto{\pgfpoint{44.515160\du}{13.368982\du}}
\pgfpathlineto{\pgfpoint{44.519541\du}{13.375918\du}}
\pgfpathlineto{\pgfpoint{44.523923\du}{13.382855\du}}
\pgfpathlineto{\pgfpoint{44.526843\du}{13.389427\du}}
\pgfpathlineto{\pgfpoint{44.530859\du}{13.396364\du}}
\pgfpathlineto{\pgfpoint{44.533780\du}{13.402936\du}}
\pgfpathlineto{\pgfpoint{44.537796\du}{13.409872\du}}
\pgfpathlineto{\pgfpoint{44.540352\du}{13.416809\du}}
\pgfpathlineto{\pgfpoint{44.543273\du}{13.423746\du}}
\pgfpathlineto{\pgfpoint{44.545828\du}{13.430683\du}}
\pgfpathlineto{\pgfpoint{44.548019\du}{13.437985\du}}
\pgfpathlineto{\pgfpoint{44.549479\du}{13.444192\du}}
\pgfpathlineto{\pgfpoint{44.551670\du}{13.451494\du}}
\pgfpathlineto{\pgfpoint{44.552400\du}{13.458430\du}}
\pgfpathlineto{\pgfpoint{44.553495\du}{13.465367\du}}
\pgfpathlineto{\pgfpoint{44.555321\du}{13.472304\du}}
\pgfpathlineto{\pgfpoint{44.555686\du}{13.479606\du}}
\pgfpathlineto{\pgfpoint{44.555686\du}{13.486543\du}}
\pgfpathlineto{\pgfpoint{44.556416\du}{13.493480\du}}
\pgfpathlineto{\pgfpoint{44.576496\du}{13.493480\du}}
\pgfusepath{fill}
\pgfsetbuttcap
\pgfsetmiterjoin
\pgfsetdash{}{0pt}
\definecolor{dialinecolor}{rgb}{0.074510, 0.082353, 0.086275}
\pgfsetfillcolor{dialinecolor}
\pgfpathmoveto{\pgfpoint{42.929180\du}{13.366061\du}}
\pgfpathlineto{\pgfpoint{43.177081\du}{13.448573\du}}
\pgfpathlineto{\pgfpoint{43.762331\du}{13.214181\du}}
\pgfpathlineto{\pgfpoint{44.035058\du}{13.281724\du}}
\pgfpathlineto{\pgfpoint{43.891210\du}{13.073253\du}}
\pgfpathlineto{\pgfpoint{43.186938\du}{13.073253\du}}
\pgfpathlineto{\pgfpoint{43.481206\du}{13.145908\du}}
\pgfpathlineto{\pgfpoint{42.929180\du}{13.366061\du}}
\pgfusepath{fill}
\pgfsetbuttcap
\pgfsetmiterjoin
\pgfsetdash{}{0pt}
\definecolor{dialinecolor}{rgb}{0.074510, 0.082353, 0.086275}
\pgfsetfillcolor{dialinecolor}
\pgfpathmoveto{\pgfpoint{42.827318\du}{13.604104\du}}
\pgfpathlineto{\pgfpoint{42.579417\du}{13.521957\du}}
\pgfpathlineto{\pgfpoint{41.994167\du}{13.755619\du}}
\pgfpathlineto{\pgfpoint{41.721075\du}{13.688806\du}}
\pgfpathlineto{\pgfpoint{41.864923\du}{13.896546\du}}
\pgfpathlineto{\pgfpoint{42.570290\du}{13.896546\du}}
\pgfpathlineto{\pgfpoint{42.275292\du}{13.824622\du}}
\pgfpathlineto{\pgfpoint{42.827318\du}{13.604104\du}}
\pgfusepath{fill}
\pgfsetbuttcap
\pgfsetmiterjoin
\pgfsetdash{}{0pt}
\definecolor{dialinecolor}{rgb}{0.074510, 0.082353, 0.086275}
\pgfsetfillcolor{dialinecolor}
\pgfpathmoveto{\pgfpoint{41.781316\du}{13.145177\du}}
\pgfpathlineto{\pgfpoint{42.028851\du}{13.063396\du}}
\pgfpathlineto{\pgfpoint{42.614101\du}{13.296692\du}}
\pgfpathlineto{\pgfpoint{42.887194\du}{13.230245\du}}
\pgfpathlineto{\pgfpoint{42.743346\du}{13.437985\du}}
\pgfpathlineto{\pgfpoint{42.038344\du}{13.437985\du}}
\pgfpathlineto{\pgfpoint{42.333342\du}{13.366061\du}}
\pgfpathlineto{\pgfpoint{41.781316\du}{13.145177\du}}
\pgfusepath{fill}
\pgfsetbuttcap
\pgfsetmiterjoin
\pgfsetdash{}{0pt}
\definecolor{dialinecolor}{rgb}{0.074510, 0.082353, 0.086275}
\pgfsetfillcolor{dialinecolor}
\pgfpathmoveto{\pgfpoint{43.999644\du}{13.840322\du}}
\pgfpathlineto{\pgfpoint{43.752108\du}{13.922468\du}}
\pgfpathlineto{\pgfpoint{43.166858\du}{13.688806\du}}
\pgfpathlineto{\pgfpoint{42.893035\du}{13.755619\du}}
\pgfpathlineto{\pgfpoint{43.037614\du}{13.547879\du}}
\pgfpathlineto{\pgfpoint{43.742981\du}{13.547879\du}}
\pgfpathlineto{\pgfpoint{43.447617\du}{13.619803\du}}
\pgfpathlineto{\pgfpoint{43.999644\du}{13.840322\du}}
\pgfusepath{fill}
\pgfsetbuttcap
\pgfsetmiterjoin
\pgfsetdash{}{0pt}
\definecolor{dialinecolor}{rgb}{1.000000, 1.000000, 1.000000}
\pgfsetfillcolor{dialinecolor}
\pgfpathmoveto{\pgfpoint{42.949990\du}{13.386506\du}}
\pgfpathlineto{\pgfpoint{43.197526\du}{13.469018\du}}
\pgfpathlineto{\pgfpoint{43.782776\du}{13.234991\du}}
\pgfpathlineto{\pgfpoint{44.055138\du}{13.302169\du}}
\pgfpathlineto{\pgfpoint{43.912385\du}{13.093699\du}}
\pgfpathlineto{\pgfpoint{43.207019\du}{13.093699\du}}
\pgfpathlineto{\pgfpoint{43.502017\du}{13.166353\du}}
\pgfpathlineto{\pgfpoint{42.949990\du}{13.386506\du}}
\pgfusepath{fill}
\pgfsetbuttcap
\pgfsetmiterjoin
\pgfsetdash{}{0pt}
\definecolor{dialinecolor}{rgb}{1.000000, 1.000000, 1.000000}
\pgfsetfillcolor{dialinecolor}
\pgfpathmoveto{\pgfpoint{42.848128\du}{13.625280\du}}
\pgfpathlineto{\pgfpoint{42.599498\du}{13.542768\du}}
\pgfpathlineto{\pgfpoint{42.014613\du}{13.776795\du}}
\pgfpathlineto{\pgfpoint{41.741155\du}{13.709252\du}}
\pgfpathlineto{\pgfpoint{41.886099\du}{13.917722\du}}
\pgfpathlineto{\pgfpoint{42.590735\du}{13.917722\du}}
\pgfpathlineto{\pgfpoint{42.296102\du}{13.845068\du}}
\pgfpathlineto{\pgfpoint{42.848128\du}{13.625280\du}}
\pgfusepath{fill}
\pgfsetbuttcap
\pgfsetmiterjoin
\pgfsetdash{}{0pt}
\definecolor{dialinecolor}{rgb}{1.000000, 1.000000, 1.000000}
\pgfsetfillcolor{dialinecolor}
\pgfpathmoveto{\pgfpoint{41.801761\du}{13.165623\du}}
\pgfpathlineto{\pgfpoint{42.049297\du}{13.083841\du}}
\pgfpathlineto{\pgfpoint{42.634912\du}{13.317868\du}}
\pgfpathlineto{\pgfpoint{42.908004\du}{13.251055\du}}
\pgfpathlineto{\pgfpoint{42.763061\du}{13.458430\du}}
\pgfpathlineto{\pgfpoint{42.058789\du}{13.458430\du}}
\pgfpathlineto{\pgfpoint{42.353422\du}{13.386506\du}}
\pgfpathlineto{\pgfpoint{41.801761\du}{13.165623\du}}
\pgfusepath{fill}
\pgfsetbuttcap
\pgfsetmiterjoin
\pgfsetdash{}{0pt}
\definecolor{dialinecolor}{rgb}{1.000000, 1.000000, 1.000000}
\pgfsetfillcolor{dialinecolor}
\pgfpathmoveto{\pgfpoint{44.019724\du}{13.860767\du}}
\pgfpathlineto{\pgfpoint{43.772188\du}{13.942914\du}}
\pgfpathlineto{\pgfpoint{43.187303\du}{13.709252\du}}
\pgfpathlineto{\pgfpoint{42.913846\du}{13.776065\du}}
\pgfpathlineto{\pgfpoint{43.058059\du}{13.568324\du}}
\pgfpathlineto{\pgfpoint{43.763061\du}{13.568324\du}}
\pgfpathlineto{\pgfpoint{43.468793\du}{13.640249\du}}
\pgfpathlineto{\pgfpoint{44.019724\du}{13.860767\du}}
\pgfusepath{fill}
\pgfsetbuttcap
\pgfsetmiterjoin
\pgfsetdash{}{0pt}
\definecolor{dialinecolor}{rgb}{0.678431, 0.839216, 0.905882}
\pgfsetfillcolor{dialinecolor}
\pgfpathmoveto{\pgfpoint{41.215781\du}{13.504067\du}}
\pgfpathlineto{\pgfpoint{41.215781\du}{13.493480\du}}
\pgfpathlineto{\pgfpoint{41.195335\du}{13.493480\du}}
\pgfpathlineto{\pgfpoint{41.195335\du}{13.504067\du}}
\pgfpathlineto{\pgfpoint{41.215781\du}{13.504067\du}}
\pgfusepath{fill}
\pgfsetbuttcap
\pgfsetmiterjoin
\pgfsetdash{}{0pt}
\definecolor{dialinecolor}{rgb}{0.678431, 0.839216, 0.905882}
\pgfsetfillcolor{dialinecolor}
\pgfpathmoveto{\pgfpoint{41.215781\du}{14.333567\du}}
\pgfpathlineto{\pgfpoint{41.215781\du}{13.504067\du}}
\pgfpathlineto{\pgfpoint{41.195335\du}{13.504067\du}}
\pgfpathlineto{\pgfpoint{41.195335\du}{14.333567\du}}
\pgfpathlineto{\pgfpoint{41.215781\du}{14.333567\du}}
\pgfusepath{fill}
\pgfsetbuttcap
\pgfsetmiterjoin
\pgfsetdash{}{0pt}
\definecolor{dialinecolor}{rgb}{0.678431, 0.839216, 0.905882}
\pgfsetfillcolor{dialinecolor}
\pgfpathmoveto{\pgfpoint{41.195335\du}{14.333567\du}}
\pgfpathlineto{\pgfpoint{41.195335\du}{14.344155\du}}
\pgfpathlineto{\pgfpoint{41.215781\du}{14.344155\du}}
\pgfpathlineto{\pgfpoint{41.215781\du}{14.333567\du}}
\pgfpathlineto{\pgfpoint{41.195335\du}{14.333567\du}}
\pgfusepath{fill}
\pgfsetbuttcap
\pgfsetmiterjoin
\pgfsetdash{}{0pt}
\definecolor{dialinecolor}{rgb}{0.678431, 0.839216, 0.905882}
\pgfsetfillcolor{dialinecolor}
\pgfpathmoveto{\pgfpoint{44.576496\du}{13.504067\du}}
\pgfpathlineto{\pgfpoint{44.576496\du}{13.493480\du}}
\pgfpathlineto{\pgfpoint{44.556416\du}{13.493480\du}}
\pgfpathlineto{\pgfpoint{44.556416\du}{13.504067\du}}
\pgfpathlineto{\pgfpoint{44.576496\du}{13.504067\du}}
\pgfusepath{fill}
\pgfsetbuttcap
\pgfsetmiterjoin
\pgfsetdash{}{0pt}
\definecolor{dialinecolor}{rgb}{0.678431, 0.839216, 0.905882}
\pgfsetfillcolor{dialinecolor}
\pgfpathmoveto{\pgfpoint{44.576496\du}{14.333567\du}}
\pgfpathlineto{\pgfpoint{44.576496\du}{13.504067\du}}
\pgfpathlineto{\pgfpoint{44.556416\du}{13.504067\du}}
\pgfpathlineto{\pgfpoint{44.556416\du}{14.333567\du}}
\pgfpathlineto{\pgfpoint{44.576496\du}{14.333567\du}}
\pgfusepath{fill}
\pgfsetbuttcap
\pgfsetmiterjoin
\pgfsetdash{}{0pt}
\definecolor{dialinecolor}{rgb}{0.678431, 0.839216, 0.905882}
\pgfsetfillcolor{dialinecolor}
\pgfpathmoveto{\pgfpoint{44.556416\du}{14.333567\du}}
\pgfpathlineto{\pgfpoint{44.556416\du}{14.344155\du}}
\pgfpathlineto{\pgfpoint{44.576496\du}{14.344155\du}}
\pgfpathlineto{\pgfpoint{44.576496\du}{14.333567\du}}
\pgfpathlineto{\pgfpoint{44.556416\du}{14.333567\du}}
\pgfusepath{fill}
\pgfsetbuttcap
\pgfsetmiterjoin
\pgfsetdash{}{0pt}
\definecolor{dialinecolor}{rgb}{0.027451, 0.372549, 0.529412}
\pgfsetfillcolor{dialinecolor}
\pgfpathmoveto{\pgfpoint{43.502747\du}{14.482892\du}}
\pgfpathlineto{\pgfpoint{43.502382\du}{14.500416\du}}
\pgfpathlineto{\pgfpoint{43.499826\du}{14.517576\du}}
\pgfpathlineto{\pgfpoint{43.496175\du}{14.533640\du}}
\pgfpathlineto{\pgfpoint{43.490334\du}{14.550800\du}}
\pgfpathlineto{\pgfpoint{43.483762\du}{14.566499\du}}
\pgfpathlineto{\pgfpoint{43.475365\du}{14.582563\du}}
\pgfpathlineto{\pgfpoint{43.465872\du}{14.598262\du}}
\pgfpathlineto{\pgfpoint{43.454189\du}{14.613231\du}}
\pgfpathlineto{\pgfpoint{43.442506\du}{14.628200\du}}
\pgfpathlineto{\pgfpoint{43.428997\du}{14.642804\du}}
\pgfpathlineto{\pgfpoint{43.414028\du}{14.656678\du}}
\pgfpathlineto{\pgfpoint{43.397964\du}{14.670917\du}}
\pgfpathlineto{\pgfpoint{43.380440\du}{14.683695\du}}
\pgfpathlineto{\pgfpoint{43.362185\du}{14.696473\du}}
\pgfpathlineto{\pgfpoint{43.343200\du}{14.708887\du}}
\pgfpathlineto{\pgfpoint{43.323119\du}{14.720570\du}}
\pgfpathlineto{\pgfpoint{43.301579\du}{14.731158\du}}
\pgfpathlineto{\pgfpoint{43.278943\du}{14.742111\du}}
\pgfpathlineto{\pgfpoint{43.255942\du}{14.751968\du}}
\pgfpathlineto{\pgfpoint{43.232210\du}{14.761461\du}}
\pgfpathlineto{\pgfpoint{43.207019\du}{14.769493\du}}
\pgfpathlineto{\pgfpoint{43.181462\du}{14.777890\du}}
\pgfpathlineto{\pgfpoint{43.154810\du}{14.785557\du}}
\pgfpathlineto{\pgfpoint{43.128158\du}{14.792494\du}}
\pgfpathlineto{\pgfpoint{43.100045\du}{14.798335\du}}
\pgfpathlineto{\pgfpoint{43.071568\du}{14.803447\du}}
\pgfpathlineto{\pgfpoint{43.041995\du}{14.807828\du}}
\pgfpathlineto{\pgfpoint{43.011692\du}{14.811844\du}}
\pgfpathlineto{\pgfpoint{42.981754\du}{14.814765\du}}
\pgfpathlineto{\pgfpoint{42.951086\du}{14.816955\du}}
\pgfpathlineto{\pgfpoint{42.919687\du}{14.818051\du}}
\pgfpathlineto{\pgfpoint{42.888289\du}{14.818781\du}}
\pgfpathlineto{\pgfpoint{42.857256\du}{14.818051\du}}
\pgfpathlineto{\pgfpoint{42.825858\du}{14.816955\du}}
\pgfpathlineto{\pgfpoint{42.794824\du}{14.814765\du}}
\pgfpathlineto{\pgfpoint{42.764521\du}{14.811844\du}}
\pgfpathlineto{\pgfpoint{42.734948\du}{14.807828\du}}
\pgfpathlineto{\pgfpoint{42.705376\du}{14.803447\du}}
\pgfpathlineto{\pgfpoint{42.676898\du}{14.798335\du}}
\pgfpathlineto{\pgfpoint{42.649151\du}{14.792494\du}}
\pgfpathlineto{\pgfpoint{42.621768\du}{14.785557\du}}
\pgfpathlineto{\pgfpoint{42.595482\du}{14.777890\du}}
\pgfpathlineto{\pgfpoint{42.570290\du}{14.769493\du}}
\pgfpathlineto{\pgfpoint{42.544733\du}{14.761461\du}}
\pgfpathlineto{\pgfpoint{42.520637\du}{14.751968\du}}
\pgfpathlineto{\pgfpoint{42.498001\du}{14.742111\du}}
\pgfpathlineto{\pgfpoint{42.475365\du}{14.731158\du}}
\pgfpathlineto{\pgfpoint{42.454554\du}{14.720570\du}}
\pgfpathlineto{\pgfpoint{42.434109\du}{14.708887\du}}
\pgfpathlineto{\pgfpoint{42.413663\du}{14.696473\du}}
\pgfpathlineto{\pgfpoint{42.395774\du}{14.683695\du}}
\pgfpathlineto{\pgfpoint{42.379344\du}{14.670917\du}}
\pgfpathlineto{\pgfpoint{42.362550\du}{14.656678\du}}
\pgfpathlineto{\pgfpoint{42.347946\du}{14.642804\du}}
\pgfpathlineto{\pgfpoint{42.334802\du}{14.628200\du}}
\pgfpathlineto{\pgfpoint{42.322389\du}{14.613231\du}}
\pgfpathlineto{\pgfpoint{42.311436\du}{14.598262\du}}
\pgfpathlineto{\pgfpoint{42.301944\du}{14.582563\du}}
\pgfpathlineto{\pgfpoint{42.293181\du}{14.566499\du}}
\pgfpathlineto{\pgfpoint{42.286245\du}{14.550800\du}}
\pgfpathlineto{\pgfpoint{42.281498\du}{14.533640\du}}
\pgfpathlineto{\pgfpoint{42.277117\du}{14.517576\du}}
\pgfpathlineto{\pgfpoint{42.274561\du}{14.500416\du}}
\pgfpathlineto{\pgfpoint{42.273466\du}{14.482892\du}}
\pgfpathlineto{\pgfpoint{42.274561\du}{14.465367\du}}
\pgfpathlineto{\pgfpoint{42.277117\du}{14.448208\du}}
\pgfpathlineto{\pgfpoint{42.281498\du}{14.432143\du}}
\pgfpathlineto{\pgfpoint{42.286245\du}{14.414984\du}}
\pgfpathlineto{\pgfpoint{42.293181\du}{14.399285\du}}
\pgfpathlineto{\pgfpoint{42.301944\du}{14.383586\du}}
\pgfpathlineto{\pgfpoint{42.311436\du}{14.367521\du}}
\pgfpathlineto{\pgfpoint{42.322389\du}{14.352552\du}}
\pgfpathlineto{\pgfpoint{42.334802\du}{14.337948\du}}
\pgfpathlineto{\pgfpoint{42.347946\du}{14.323345\du}}
\pgfpathlineto{\pgfpoint{42.362550\du}{14.309106\du}}
\pgfpathlineto{\pgfpoint{42.379344\du}{14.295232\du}}
\pgfpathlineto{\pgfpoint{42.395774\du}{14.282089\du}}
\pgfpathlineto{\pgfpoint{42.413663\du}{14.270040\du}}
\pgfpathlineto{\pgfpoint{42.434109\du}{14.257627\du}}
\pgfpathlineto{\pgfpoint{42.454554\du}{14.245944\du}}
\pgfpathlineto{\pgfpoint{42.475365\du}{14.234991\du}}
\pgfpathlineto{\pgfpoint{42.498001\du}{14.224403\du}}
\pgfpathlineto{\pgfpoint{42.520637\du}{14.213816\du}}
\pgfpathlineto{\pgfpoint{42.544733\du}{14.205053\du}}
\pgfpathlineto{\pgfpoint{42.570290\du}{14.196291\du}}
\pgfpathlineto{\pgfpoint{42.595482\du}{14.187894\du}}
\pgfpathlineto{\pgfpoint{42.621768\du}{14.180227\du}}
\pgfpathlineto{\pgfpoint{42.649151\du}{14.174020\du}}
\pgfpathlineto{\pgfpoint{42.676898\du}{14.167448\du}}
\pgfpathlineto{\pgfpoint{42.705376\du}{14.162337\du}}
\pgfpathlineto{\pgfpoint{42.734948\du}{14.158321\du}}
\pgfpathlineto{\pgfpoint{42.764521\du}{14.153940\du}}
\pgfpathlineto{\pgfpoint{42.794824\du}{14.151019\du}}
\pgfpathlineto{\pgfpoint{42.825858\du}{14.149559\du}}
\pgfpathlineto{\pgfpoint{42.857256\du}{14.147733\du}}
\pgfpathlineto{\pgfpoint{42.888289\du}{14.147733\du}}
\pgfpathlineto{\pgfpoint{42.919687\du}{14.147733\du}}
\pgfpathlineto{\pgfpoint{42.951086\du}{14.149559\du}}
\pgfpathlineto{\pgfpoint{42.981754\du}{14.151019\du}}
\pgfpathlineto{\pgfpoint{43.011692\du}{14.153940\du}}
\pgfpathlineto{\pgfpoint{43.041995\du}{14.158321\du}}
\pgfpathlineto{\pgfpoint{43.071568\du}{14.162337\du}}
\pgfpathlineto{\pgfpoint{43.100045\du}{14.167448\du}}
\pgfpathlineto{\pgfpoint{43.128158\du}{14.174020\du}}
\pgfpathlineto{\pgfpoint{43.154810\du}{14.180227\du}}
\pgfpathlineto{\pgfpoint{43.181462\du}{14.187894\du}}
\pgfpathlineto{\pgfpoint{43.207019\du}{14.196291\du}}
\pgfpathlineto{\pgfpoint{43.232210\du}{14.205053\du}}
\pgfpathlineto{\pgfpoint{43.255942\du}{14.213816\du}}
\pgfpathlineto{\pgfpoint{43.278943\du}{14.224403\du}}
\pgfpathlineto{\pgfpoint{43.301579\du}{14.234991\du}}
\pgfpathlineto{\pgfpoint{43.323119\du}{14.245944\du}}
\pgfpathlineto{\pgfpoint{43.343200\du}{14.257627\du}}
\pgfpathlineto{\pgfpoint{43.362185\du}{14.270040\du}}
\pgfpathlineto{\pgfpoint{43.380440\du}{14.282089\du}}
\pgfpathlineto{\pgfpoint{43.397964\du}{14.295232\du}}
\pgfpathlineto{\pgfpoint{43.414028\du}{14.309106\du}}
\pgfpathlineto{\pgfpoint{43.428997\du}{14.323345\du}}
\pgfpathlineto{\pgfpoint{43.442506\du}{14.337948\du}}
\pgfpathlineto{\pgfpoint{43.454189\du}{14.352552\du}}
\pgfpathlineto{\pgfpoint{43.465872\du}{14.367521\du}}
\pgfpathlineto{\pgfpoint{43.475365\du}{14.383586\du}}
\pgfpathlineto{\pgfpoint{43.483762\du}{14.399285\du}}
\pgfpathlineto{\pgfpoint{43.490334\du}{14.414984\du}}
\pgfpathlineto{\pgfpoint{43.496175\du}{14.432143\du}}
\pgfpathlineto{\pgfpoint{43.499826\du}{14.448208\du}}
\pgfpathlineto{\pgfpoint{43.502382\du}{14.465367\du}}
\pgfpathlineto{\pgfpoint{43.502747\du}{14.482892\du}}
\pgfusepath{fill}
\pgfsetbuttcap
\pgfsetmiterjoin
\pgfsetdash{}{0pt}
\definecolor{dialinecolor}{rgb}{0.678431, 0.839216, 0.905882}
\pgfsetfillcolor{dialinecolor}
\pgfpathmoveto{\pgfpoint{42.888289\du}{14.828638\du}}
\pgfpathlineto{\pgfpoint{42.888289\du}{14.828638\du}}
\pgfpathlineto{\pgfpoint{42.904353\du}{14.828638\du}}
\pgfpathlineto{\pgfpoint{42.920418\du}{14.828273\du}}
\pgfpathlineto{\pgfpoint{42.936482\du}{14.827543\du}}
\pgfpathlineto{\pgfpoint{42.951816\du}{14.826813\du}}
\pgfpathlineto{\pgfpoint{42.967515\du}{14.825718\du}}
\pgfpathlineto{\pgfpoint{42.982849\du}{14.824622\du}}
\pgfpathlineto{\pgfpoint{42.997818\du}{14.823527\du}}
\pgfpathlineto{\pgfpoint{43.013882\du}{14.821702\du}}
\pgfpathlineto{\pgfpoint{43.028121\du}{14.819876\du}}
\pgfpathlineto{\pgfpoint{43.043090\du}{14.818051\du}}
\pgfpathlineto{\pgfpoint{43.058059\du}{14.815860\du}}
\pgfpathlineto{\pgfpoint{43.073393\du}{14.813669\du}}
\pgfpathlineto{\pgfpoint{43.087632\du}{14.810749\du}}
\pgfpathlineto{\pgfpoint{43.101506\du}{14.808193\du}}
\pgfpathlineto{\pgfpoint{43.116109\du}{14.805272\du}}
\pgfpathlineto{\pgfpoint{43.129983\du}{14.802351\du}}
\pgfpathlineto{\pgfpoint{43.143492\du}{14.798701\du}}
\pgfpathlineto{\pgfpoint{43.157365\du}{14.795415\du}}
\pgfpathlineto{\pgfpoint{43.170874\du}{14.791764\du}}
\pgfpathlineto{\pgfpoint{43.184017\du}{14.787748\du}}
\pgfpathlineto{\pgfpoint{43.197161\du}{14.783732\du}}
\pgfpathlineto{\pgfpoint{43.210304\du}{14.779715\du}}
\pgfpathlineto{\pgfpoint{43.223083\du}{14.774969\du}}
\pgfpathlineto{\pgfpoint{43.235861\du}{14.770953\du}}
\pgfpathlineto{\pgfpoint{43.247544\du}{14.766207\du}}
\pgfpathlineto{\pgfpoint{43.260323\du}{14.761461\du}}
\pgfpathlineto{\pgfpoint{43.271641\du}{14.755984\du}}
\pgfpathlineto{\pgfpoint{43.283324\du}{14.750873\du}}
\pgfpathlineto{\pgfpoint{43.294642\du}{14.745761\du}}
\pgfpathlineto{\pgfpoint{43.305960\du}{14.740285\du}}
\pgfpathlineto{\pgfpoint{43.316913\du}{14.735174\du}}
\pgfpathlineto{\pgfpoint{43.328231\du}{14.729332\du}}
\pgfpathlineto{\pgfpoint{43.338088\du}{14.723491\du}}
\pgfpathlineto{\pgfpoint{43.348311\du}{14.716919\du}}
\pgfpathlineto{\pgfpoint{43.358169\du}{14.711077\du}}
\pgfpathlineto{\pgfpoint{43.368391\du}{14.704506\du}}
\pgfpathlineto{\pgfpoint{43.377884\du}{14.698299\du}}
\pgfpathlineto{\pgfpoint{43.387011\du}{14.691727\du}}
\pgfpathlineto{\pgfpoint{43.395774\du}{14.685521\du}}
\pgfpathlineto{\pgfpoint{43.404171\du}{14.678219\du}}
\pgfpathlineto{\pgfpoint{43.412568\du}{14.671282\du}}
\pgfpathlineto{\pgfpoint{43.420600\du}{14.664345\du}}
\pgfpathlineto{\pgfpoint{43.428632\du}{14.657408\du}}
\pgfpathlineto{\pgfpoint{43.435569\du}{14.650106\du}}
\pgfpathlineto{\pgfpoint{43.443236\du}{14.642804\du}}
\pgfpathlineto{\pgfpoint{43.449808\du}{14.635137\du}}
\pgfpathlineto{\pgfpoint{43.456380\du}{14.627470\du}}
\pgfpathlineto{\pgfpoint{43.462586\du}{14.619803\du}}
\pgfpathlineto{\pgfpoint{43.468793\du}{14.611771\du}}
\pgfpathlineto{\pgfpoint{43.474634\du}{14.604104\du}}
\pgfpathlineto{\pgfpoint{43.479381\du}{14.595707\du}}
\pgfpathlineto{\pgfpoint{43.484127\du}{14.587675\du}}
\pgfpathlineto{\pgfpoint{43.488873\du}{14.579277\du}}
\pgfpathlineto{\pgfpoint{43.493254\du}{14.571245\du}}
\pgfpathlineto{\pgfpoint{43.496540\du}{14.562483\du}}
\pgfpathlineto{\pgfpoint{43.500556\du}{14.554451\du}}
\pgfpathlineto{\pgfpoint{43.503112\du}{14.545688\du}}
\pgfpathlineto{\pgfpoint{43.506033\du}{14.536926\du}}
\pgfpathlineto{\pgfpoint{43.507858\du}{14.527799\du}}
\pgfpathlineto{\pgfpoint{43.510049\du}{14.519036\du}}
\pgfpathlineto{\pgfpoint{43.511509\du}{14.510274\du}}
\pgfpathlineto{\pgfpoint{43.512605\du}{14.501147\du}}
\pgfpathlineto{\pgfpoint{43.513335\du}{14.492384\du}}
\pgfpathlineto{\pgfpoint{43.513335\du}{14.482892\du}}
\pgfpathlineto{\pgfpoint{43.493254\du}{14.482892\du}}
\pgfpathlineto{\pgfpoint{43.492524\du}{14.490924\du}}
\pgfpathlineto{\pgfpoint{43.492159\du}{14.499321\du}}
\pgfpathlineto{\pgfpoint{43.491429\du}{14.507353\du}}
\pgfpathlineto{\pgfpoint{43.489969\du}{14.515020\du}}
\pgfpathlineto{\pgfpoint{43.488508\du}{14.523418\du}}
\pgfpathlineto{\pgfpoint{43.485952\du}{14.531450\du}}
\pgfpathlineto{\pgfpoint{43.483762\du}{14.539117\du}}
\pgfpathlineto{\pgfpoint{43.480476\du}{14.546784\du}}
\pgfpathlineto{\pgfpoint{43.477920\du}{14.554451\du}}
\pgfpathlineto{\pgfpoint{43.474634\du}{14.562483\du}}
\pgfpathlineto{\pgfpoint{43.470618\du}{14.570150\du}}
\pgfpathlineto{\pgfpoint{43.465872\du}{14.577817\du}}
\pgfpathlineto{\pgfpoint{43.461856\du}{14.585484\du}}
\pgfpathlineto{\pgfpoint{43.456745\du}{14.592421\du}}
\pgfpathlineto{\pgfpoint{43.451998\du}{14.600088\du}}
\pgfpathlineto{\pgfpoint{43.446522\du}{14.607025\du}}
\pgfpathlineto{\pgfpoint{43.441411\du}{14.614692\du}}
\pgfpathlineto{\pgfpoint{43.434474\du}{14.621629\du}}
\pgfpathlineto{\pgfpoint{43.428632\du}{14.628565\du}}
\pgfpathlineto{\pgfpoint{43.421330\du}{14.635502\du}}
\pgfpathlineto{\pgfpoint{43.414394\du}{14.642804\du}}
\pgfpathlineto{\pgfpoint{43.407092\du}{14.649011\du}}
\pgfpathlineto{\pgfpoint{43.400155\du}{14.655948\du}}
\pgfpathlineto{\pgfpoint{43.391758\du}{14.662519\du}}
\pgfpathlineto{\pgfpoint{43.383725\du}{14.669091\du}}
\pgfpathlineto{\pgfpoint{43.374598\du}{14.675298\du}}
\pgfpathlineto{\pgfpoint{43.365836\du}{14.681139\du}}
\pgfpathlineto{\pgfpoint{43.356343\du}{14.687711\du}}
\pgfpathlineto{\pgfpoint{43.347216\du}{14.694283\du}}
\pgfpathlineto{\pgfpoint{43.338088\du}{14.699394\du}}
\pgfpathlineto{\pgfpoint{43.328231\du}{14.705236\du}}
\pgfpathlineto{\pgfpoint{43.318373\du}{14.711077\du}}
\pgfpathlineto{\pgfpoint{43.307785\du}{14.716554\du}}
\pgfpathlineto{\pgfpoint{43.297197\du}{14.722395\du}}
\pgfpathlineto{\pgfpoint{43.286610\du}{14.726776\du}}
\pgfpathlineto{\pgfpoint{43.274927\du}{14.732618\du}}
\pgfpathlineto{\pgfpoint{43.263609\du}{14.737364\du}}
\pgfpathlineto{\pgfpoint{43.252291\du}{14.742111\du}}
\pgfpathlineto{\pgfpoint{43.240607\du}{14.746857\du}}
\pgfpathlineto{\pgfpoint{43.228559\du}{14.751603\du}}
\pgfpathlineto{\pgfpoint{43.216146\du}{14.755619\du}}
\pgfpathlineto{\pgfpoint{43.204463\du}{14.760365\du}}
\pgfpathlineto{\pgfpoint{43.191685\du}{14.764381\du}}
\pgfpathlineto{\pgfpoint{43.178176\du}{14.768032\du}}
\pgfpathlineto{\pgfpoint{43.165398\du}{14.772048\du}}
\pgfpathlineto{\pgfpoint{43.152254\du}{14.775334\du}}
\pgfpathlineto{\pgfpoint{43.138745\du}{14.778985\du}}
\pgfpathlineto{\pgfpoint{43.125237\du}{14.781906\du}}
\pgfpathlineto{\pgfpoint{43.111363\du}{14.785557\du}}
\pgfpathlineto{\pgfpoint{43.097490\du}{14.787748\du}}
\pgfpathlineto{\pgfpoint{43.083616\du}{14.790668\du}}
\pgfpathlineto{\pgfpoint{43.069377\du}{14.792859\du}}
\pgfpathlineto{\pgfpoint{43.055138\du}{14.795415\du}}
\pgfpathlineto{\pgfpoint{43.040900\du}{14.797605\du}}
\pgfpathlineto{\pgfpoint{43.026296\du}{14.799431\du}}
\pgfpathlineto{\pgfpoint{43.011327\du}{14.801256\du}}
\pgfpathlineto{\pgfpoint{42.995993\du}{14.803082\du}}
\pgfpathlineto{\pgfpoint{42.981389\du}{14.804177\du}}
\pgfpathlineto{\pgfpoint{42.965690\du}{14.805272\du}}
\pgfpathlineto{\pgfpoint{42.950721\du}{14.806368\du}}
\pgfpathlineto{\pgfpoint{42.935387\du}{14.807098\du}}
\pgfpathlineto{\pgfpoint{42.919687\du}{14.807828\du}}
\pgfpathlineto{\pgfpoint{42.904353\du}{14.808193\du}}
\pgfpathlineto{\pgfpoint{42.888289\du}{14.808193\du}}
\pgfpathlineto{\pgfpoint{42.888289\du}{14.808193\du}}
\pgfpathlineto{\pgfpoint{42.888289\du}{14.808193\du}}
\pgfpathlineto{\pgfpoint{42.887194\du}{14.808193\du}}
\pgfpathlineto{\pgfpoint{42.886099\du}{14.808193\du}}
\pgfpathlineto{\pgfpoint{42.885368\du}{14.808923\du}}
\pgfpathlineto{\pgfpoint{42.883543\du}{14.808923\du}}
\pgfpathlineto{\pgfpoint{42.883178\du}{14.809288\du}}
\pgfpathlineto{\pgfpoint{42.882082\du}{14.810019\du}}
\pgfpathlineto{\pgfpoint{42.881717\du}{14.810749\du}}
\pgfpathlineto{\pgfpoint{42.881352\du}{14.811114\du}}
\pgfpathlineto{\pgfpoint{42.879892\du}{14.812939\du}}
\pgfpathlineto{\pgfpoint{42.878431\du}{14.814765\du}}
\pgfpathlineto{\pgfpoint{42.878431\du}{14.816590\du}}
\pgfpathlineto{\pgfpoint{42.878066\du}{14.818781\du}}
\pgfpathlineto{\pgfpoint{42.878431\du}{14.820606\du}}
\pgfpathlineto{\pgfpoint{42.878431\du}{14.822432\du}}
\pgfpathlineto{\pgfpoint{42.879892\du}{14.823892\du}}
\pgfpathlineto{\pgfpoint{42.881352\du}{14.825353\du}}
\pgfpathlineto{\pgfpoint{42.881717\du}{14.826448\du}}
\pgfpathlineto{\pgfpoint{42.882082\du}{14.826813\du}}
\pgfpathlineto{\pgfpoint{42.883178\du}{14.827543\du}}
\pgfpathlineto{\pgfpoint{42.883543\du}{14.827543\du}}
\pgfpathlineto{\pgfpoint{42.885368\du}{14.828273\du}}
\pgfpathlineto{\pgfpoint{42.886099\du}{14.828638\du}}
\pgfpathlineto{\pgfpoint{42.887194\du}{14.828638\du}}
\pgfpathlineto{\pgfpoint{42.888289\du}{14.828638\du}}
\pgfusepath{fill}
\pgfsetbuttcap
\pgfsetmiterjoin
\pgfsetdash{}{0pt}
\definecolor{dialinecolor}{rgb}{0.678431, 0.839216, 0.905882}
\pgfsetfillcolor{dialinecolor}
\pgfpathmoveto{\pgfpoint{42.263609\du}{14.482892\du}}
\pgfpathlineto{\pgfpoint{42.263609\du}{14.482892\du}}
\pgfpathlineto{\pgfpoint{42.263609\du}{14.491654\du}}
\pgfpathlineto{\pgfpoint{42.264339\du}{14.501147\du}}
\pgfpathlineto{\pgfpoint{42.265434\du}{14.510274\du}}
\pgfpathlineto{\pgfpoint{42.267625\du}{14.519036\du}}
\pgfpathlineto{\pgfpoint{42.268720\du}{14.527799\du}}
\pgfpathlineto{\pgfpoint{42.271276\du}{14.536926\du}}
\pgfpathlineto{\pgfpoint{42.273466\du}{14.545688\du}}
\pgfpathlineto{\pgfpoint{42.276752\du}{14.554451\du}}
\pgfpathlineto{\pgfpoint{42.280038\du}{14.562483\du}}
\pgfpathlineto{\pgfpoint{42.284054\du}{14.571245\du}}
\pgfpathlineto{\pgfpoint{42.288435\du}{14.579277\du}}
\pgfpathlineto{\pgfpoint{42.292816\du}{14.587675\du}}
\pgfpathlineto{\pgfpoint{42.297563\du}{14.595707\du}}
\pgfpathlineto{\pgfpoint{42.302674\du}{14.604104\du}}
\pgfpathlineto{\pgfpoint{42.308881\du}{14.611771\du}}
\pgfpathlineto{\pgfpoint{42.313627\du}{14.619803\du}}
\pgfpathlineto{\pgfpoint{42.320199\du}{14.627470\du}}
\pgfpathlineto{\pgfpoint{42.327135\du}{14.635137\du}}
\pgfpathlineto{\pgfpoint{42.333707\du}{14.642804\du}}
\pgfpathlineto{\pgfpoint{42.340644\du}{14.650106\du}}
\pgfpathlineto{\pgfpoint{42.347946\du}{14.657408\du}}
\pgfpathlineto{\pgfpoint{42.356708\du}{14.664345\du}}
\pgfpathlineto{\pgfpoint{42.363645\du}{14.671282\du}}
\pgfpathlineto{\pgfpoint{42.372772\du}{14.678219\du}}
\pgfpathlineto{\pgfpoint{42.380805\du}{14.685521\du}}
\pgfpathlineto{\pgfpoint{42.389932\du}{14.691727\du}}
\pgfpathlineto{\pgfpoint{42.399790\du}{14.698299\du}}
\pgfpathlineto{\pgfpoint{42.408917\du}{14.704506\du}}
\pgfpathlineto{\pgfpoint{42.418410\du}{14.711077\du}}
\pgfpathlineto{\pgfpoint{42.428267\du}{14.716919\du}}
\pgfpathlineto{\pgfpoint{42.438490\du}{14.723491\du}}
\pgfpathlineto{\pgfpoint{42.448713\du}{14.729332\du}}
\pgfpathlineto{\pgfpoint{42.459666\du}{14.735174\du}}
\pgfpathlineto{\pgfpoint{42.470984\du}{14.740285\du}}
\pgfpathlineto{\pgfpoint{42.482302\du}{14.745761\du}}
\pgfpathlineto{\pgfpoint{42.493620\du}{14.750873\du}}
\pgfpathlineto{\pgfpoint{42.505668\du}{14.755984\du}}
\pgfpathlineto{\pgfpoint{42.516986\du}{14.761461\du}}
\pgfpathlineto{\pgfpoint{42.529399\du}{14.766207\du}}
\pgfpathlineto{\pgfpoint{42.541447\du}{14.770953\du}}
\pgfpathlineto{\pgfpoint{42.553495\du}{14.774969\du}}
\pgfpathlineto{\pgfpoint{42.566639\du}{14.779715\du}}
\pgfpathlineto{\pgfpoint{42.579417\du}{14.783732\du}}
\pgfpathlineto{\pgfpoint{42.592926\du}{14.787748\du}}
\pgfpathlineto{\pgfpoint{42.606434\du}{14.791764\du}}
\pgfpathlineto{\pgfpoint{42.619213\du}{14.795415\du}}
\pgfpathlineto{\pgfpoint{42.632721\du}{14.798701\du}}
\pgfpathlineto{\pgfpoint{42.646595\du}{14.802351\du}}
\pgfpathlineto{\pgfpoint{42.661199\du}{14.805272\du}}
\pgfpathlineto{\pgfpoint{42.675803\du}{14.808193\du}}
\pgfpathlineto{\pgfpoint{42.689676\du}{14.810749\du}}
\pgfpathlineto{\pgfpoint{42.703915\du}{14.813669\du}}
\pgfpathlineto{\pgfpoint{42.718884\du}{14.815860\du}}
\pgfpathlineto{\pgfpoint{42.733853\du}{14.818051\du}}
\pgfpathlineto{\pgfpoint{42.748457\du}{14.819876\du}}
\pgfpathlineto{\pgfpoint{42.763061\du}{14.821702\du}}
\pgfpathlineto{\pgfpoint{42.778760\du}{14.823527\du}}
\pgfpathlineto{\pgfpoint{42.794459\du}{14.824622\du}}
\pgfpathlineto{\pgfpoint{42.809063\du}{14.825718\du}}
\pgfpathlineto{\pgfpoint{42.825127\du}{14.826813\du}}
\pgfpathlineto{\pgfpoint{42.840461\du}{14.827543\du}}
\pgfpathlineto{\pgfpoint{42.856526\du}{14.828273\du}}
\pgfpathlineto{\pgfpoint{42.872590\du}{14.828638\du}}
\pgfpathlineto{\pgfpoint{42.888289\du}{14.828638\du}}
\pgfpathlineto{\pgfpoint{42.888289\du}{14.808193\du}}
\pgfpathlineto{\pgfpoint{42.872590\du}{14.808193\du}}
\pgfpathlineto{\pgfpoint{42.857256\du}{14.807828\du}}
\pgfpathlineto{\pgfpoint{42.841192\du}{14.807098\du}}
\pgfpathlineto{\pgfpoint{42.826588\du}{14.806368\du}}
\pgfpathlineto{\pgfpoint{42.810889\du}{14.805272\du}}
\pgfpathlineto{\pgfpoint{42.795920\du}{14.804177\du}}
\pgfpathlineto{\pgfpoint{42.780951\du}{14.803082\du}}
\pgfpathlineto{\pgfpoint{42.765982\du}{14.801256\du}}
\pgfpathlineto{\pgfpoint{42.751013\du}{14.799431\du}}
\pgfpathlineto{\pgfpoint{42.736409\du}{14.797605\du}}
\pgfpathlineto{\pgfpoint{42.721805\du}{14.795415\du}}
\pgfpathlineto{\pgfpoint{42.707566\du}{14.792859\du}}
\pgfpathlineto{\pgfpoint{42.693693\du}{14.790668\du}}
\pgfpathlineto{\pgfpoint{42.679819\du}{14.787748\du}}
\pgfpathlineto{\pgfpoint{42.665580\du}{14.785557\du}}
\pgfpathlineto{\pgfpoint{42.651706\du}{14.781906\du}}
\pgfpathlineto{\pgfpoint{42.638198\du}{14.778985\du}}
\pgfpathlineto{\pgfpoint{42.625054\du}{14.775334\du}}
\pgfpathlineto{\pgfpoint{42.611546\du}{14.772048\du}}
\pgfpathlineto{\pgfpoint{42.598767\du}{14.768032\du}}
\pgfpathlineto{\pgfpoint{42.585624\du}{14.764381\du}}
\pgfpathlineto{\pgfpoint{42.572846\du}{14.760365\du}}
\pgfpathlineto{\pgfpoint{42.560797\du}{14.755619\du}}
\pgfpathlineto{\pgfpoint{42.548384\du}{14.751603\du}}
\pgfpathlineto{\pgfpoint{42.535971\du}{14.746857\du}}
\pgfpathlineto{\pgfpoint{42.524653\du}{14.742111\du}}
\pgfpathlineto{\pgfpoint{42.512970\du}{14.737364\du}}
\pgfpathlineto{\pgfpoint{42.502017\du}{14.732618\du}}
\pgfpathlineto{\pgfpoint{42.490699\du}{14.726776\du}}
\pgfpathlineto{\pgfpoint{42.480111\du}{14.722395\du}}
\pgfpathlineto{\pgfpoint{42.469158\du}{14.716554\du}}
\pgfpathlineto{\pgfpoint{42.458935\du}{14.711077\du}}
\pgfpathlineto{\pgfpoint{42.448713\du}{14.705236\du}}
\pgfpathlineto{\pgfpoint{42.438855\du}{14.699394\du}}
\pgfpathlineto{\pgfpoint{42.429362\du}{14.694283\du}}
\pgfpathlineto{\pgfpoint{42.419505\du}{14.687711\du}}
\pgfpathlineto{\pgfpoint{42.411108\du}{14.681139\du}}
\pgfpathlineto{\pgfpoint{42.401980\du}{14.675298\du}}
\pgfpathlineto{\pgfpoint{42.393583\du}{14.669091\du}}
\pgfpathlineto{\pgfpoint{42.385186\du}{14.662519\du}}
\pgfpathlineto{\pgfpoint{42.377154\du}{14.655948\du}}
\pgfpathlineto{\pgfpoint{42.369852\du}{14.649011\du}}
\pgfpathlineto{\pgfpoint{42.362185\du}{14.642804\du}}
\pgfpathlineto{\pgfpoint{42.355613\du}{14.635502\du}}
\pgfpathlineto{\pgfpoint{42.348676\du}{14.628565\du}}
\pgfpathlineto{\pgfpoint{42.342469\du}{14.621629\du}}
\pgfpathlineto{\pgfpoint{42.336263\du}{14.614692\du}}
\pgfpathlineto{\pgfpoint{42.330056\du}{14.607025\du}}
\pgfpathlineto{\pgfpoint{42.324945\du}{14.600088\du}}
\pgfpathlineto{\pgfpoint{42.319833\du}{14.592421\du}}
\pgfpathlineto{\pgfpoint{42.315087\du}{14.585484\du}}
\pgfpathlineto{\pgfpoint{42.310706\du}{14.577817\du}}
\pgfpathlineto{\pgfpoint{42.306325\du}{14.570150\du}}
\pgfpathlineto{\pgfpoint{42.302674\du}{14.562483\du}}
\pgfpathlineto{\pgfpoint{42.299388\du}{14.554451\du}}
\pgfpathlineto{\pgfpoint{42.296102\du}{14.546784\du}}
\pgfpathlineto{\pgfpoint{42.293181\du}{14.539117\du}}
\pgfpathlineto{\pgfpoint{42.290626\du}{14.531450\du}}
\pgfpathlineto{\pgfpoint{42.288800\du}{14.523418\du}}
\pgfpathlineto{\pgfpoint{42.286975\du}{14.515020\du}}
\pgfpathlineto{\pgfpoint{42.285879\du}{14.507353\du}}
\pgfpathlineto{\pgfpoint{42.284419\du}{14.499321\du}}
\pgfpathlineto{\pgfpoint{42.284054\du}{14.490924\du}}
\pgfpathlineto{\pgfpoint{42.284054\du}{14.482892\du}}
\pgfpathlineto{\pgfpoint{42.284054\du}{14.482892\du}}
\pgfpathlineto{\pgfpoint{42.284054\du}{14.482892\du}}
\pgfpathlineto{\pgfpoint{42.284054\du}{14.481797\du}}
\pgfpathlineto{\pgfpoint{42.284054\du}{14.480701\du}}
\pgfpathlineto{\pgfpoint{42.283689\du}{14.479241\du}}
\pgfpathlineto{\pgfpoint{42.283689\du}{14.478146\du}}
\pgfpathlineto{\pgfpoint{42.283324\du}{14.477780\du}}
\pgfpathlineto{\pgfpoint{42.282228\du}{14.476320\du}}
\pgfpathlineto{\pgfpoint{42.281498\du}{14.475955\du}}
\pgfpathlineto{\pgfpoint{42.281498\du}{14.475225\du}}
\pgfpathlineto{\pgfpoint{42.279308\du}{14.474130\du}}
\pgfpathlineto{\pgfpoint{42.277482\du}{14.473399\du}}
\pgfpathlineto{\pgfpoint{42.275657\du}{14.473034\du}}
\pgfpathlineto{\pgfpoint{42.273466\du}{14.472304\du}}
\pgfpathlineto{\pgfpoint{42.272006\du}{14.473034\du}}
\pgfpathlineto{\pgfpoint{42.270180\du}{14.473399\du}}
\pgfpathlineto{\pgfpoint{42.268355\du}{14.474130\du}}
\pgfpathlineto{\pgfpoint{42.266529\du}{14.475225\du}}
\pgfpathlineto{\pgfpoint{42.265799\du}{14.475955\du}}
\pgfpathlineto{\pgfpoint{42.265434\du}{14.476320\du}}
\pgfpathlineto{\pgfpoint{42.265069\du}{14.477780\du}}
\pgfpathlineto{\pgfpoint{42.264339\du}{14.478146\du}}
\pgfpathlineto{\pgfpoint{42.264339\du}{14.479241\du}}
\pgfpathlineto{\pgfpoint{42.263609\du}{14.480701\du}}
\pgfpathlineto{\pgfpoint{42.263609\du}{14.481797\du}}
\pgfpathlineto{\pgfpoint{42.263609\du}{14.482892\du}}
\pgfusepath{fill}
\pgfsetbuttcap
\pgfsetmiterjoin
\pgfsetdash{}{0pt}
\definecolor{dialinecolor}{rgb}{0.678431, 0.839216, 0.905882}
\pgfsetfillcolor{dialinecolor}
\pgfpathmoveto{\pgfpoint{42.888289\du}{14.137145\du}}
\pgfpathlineto{\pgfpoint{42.888289\du}{14.137145\du}}
\pgfpathlineto{\pgfpoint{42.872590\du}{14.137145\du}}
\pgfpathlineto{\pgfpoint{42.856526\du}{14.137510\du}}
\pgfpathlineto{\pgfpoint{42.840461\du}{14.138241\du}}
\pgfpathlineto{\pgfpoint{42.825127\du}{14.138971\du}}
\pgfpathlineto{\pgfpoint{42.809063\du}{14.140066\du}}
\pgfpathlineto{\pgfpoint{42.794459\du}{14.141161\du}}
\pgfpathlineto{\pgfpoint{42.778760\du}{14.142257\du}}
\pgfpathlineto{\pgfpoint{42.763061\du}{14.144082\du}}
\pgfpathlineto{\pgfpoint{42.748457\du}{14.145908\du}}
\pgfpathlineto{\pgfpoint{42.733853\du}{14.147733\du}}
\pgfpathlineto{\pgfpoint{42.718884\du}{14.149924\du}}
\pgfpathlineto{\pgfpoint{42.703915\du}{14.152479\du}}
\pgfpathlineto{\pgfpoint{42.689676\du}{14.155400\du}}
\pgfpathlineto{\pgfpoint{42.675803\du}{14.157591\du}}
\pgfpathlineto{\pgfpoint{42.661199\du}{14.160511\du}}
\pgfpathlineto{\pgfpoint{42.646595\du}{14.163432\du}}
\pgfpathlineto{\pgfpoint{42.632721\du}{14.167083\du}}
\pgfpathlineto{\pgfpoint{42.619213\du}{14.170369\du}}
\pgfpathlineto{\pgfpoint{42.606434\du}{14.174385\du}}
\pgfpathlineto{\pgfpoint{42.592926\du}{14.178036\du}}
\pgfpathlineto{\pgfpoint{42.579417\du}{14.182052\du}}
\pgfpathlineto{\pgfpoint{42.566639\du}{14.186433\du}}
\pgfpathlineto{\pgfpoint{42.553495\du}{14.190814\du}}
\pgfpathlineto{\pgfpoint{42.541447\du}{14.195196\du}}
\pgfpathlineto{\pgfpoint{42.529399\du}{14.199577\du}}
\pgfpathlineto{\pgfpoint{42.516986\du}{14.205053\du}}
\pgfpathlineto{\pgfpoint{42.505668\du}{14.209799\du}}
\pgfpathlineto{\pgfpoint{42.493620\du}{14.214911\du}}
\pgfpathlineto{\pgfpoint{42.482302\du}{14.220022\du}}
\pgfpathlineto{\pgfpoint{42.470984\du}{14.225499\du}}
\pgfpathlineto{\pgfpoint{42.459666\du}{14.231340\du}}
\pgfpathlineto{\pgfpoint{42.448713\du}{14.236452\du}}
\pgfpathlineto{\pgfpoint{42.438490\du}{14.242293\du}}
\pgfpathlineto{\pgfpoint{42.428267\du}{14.248865\du}}
\pgfpathlineto{\pgfpoint{42.418410\du}{14.254706\du}}
\pgfpathlineto{\pgfpoint{42.408917\du}{14.261278\du}}
\pgfpathlineto{\pgfpoint{42.399790\du}{14.267485\du}}
\pgfpathlineto{\pgfpoint{42.389932\du}{14.274056\du}}
\pgfpathlineto{\pgfpoint{42.380805\du}{14.280628\du}}
\pgfpathlineto{\pgfpoint{42.372772\du}{14.287565\du}}
\pgfpathlineto{\pgfpoint{42.363645\du}{14.294502\du}}
\pgfpathlineto{\pgfpoint{42.356708\du}{14.301439\du}}
\pgfpathlineto{\pgfpoint{42.347946\du}{14.308376\du}}
\pgfpathlineto{\pgfpoint{42.340644\du}{14.315678\du}}
\pgfpathlineto{\pgfpoint{42.333707\du}{14.323345\du}}
\pgfpathlineto{\pgfpoint{42.327135\du}{14.330646\du}}
\pgfpathlineto{\pgfpoint{42.320199\du}{14.338314\du}}
\pgfpathlineto{\pgfpoint{42.313627\du}{14.345981\du}}
\pgfpathlineto{\pgfpoint{42.308881\du}{14.354013\du}}
\pgfpathlineto{\pgfpoint{42.302674\du}{14.361680\du}}
\pgfpathlineto{\pgfpoint{42.297563\du}{14.370077\du}}
\pgfpathlineto{\pgfpoint{42.292816\du}{14.378109\du}}
\pgfpathlineto{\pgfpoint{42.288435\du}{14.386506\du}}
\pgfpathlineto{\pgfpoint{42.284054\du}{14.394538\du}}
\pgfpathlineto{\pgfpoint{42.280038\du}{14.403301\du}}
\pgfpathlineto{\pgfpoint{42.276752\du}{14.411698\du}}
\pgfpathlineto{\pgfpoint{42.273466\du}{14.420460\du}}
\pgfpathlineto{\pgfpoint{42.271276\du}{14.429223\du}}
\pgfpathlineto{\pgfpoint{42.268720\du}{14.437985\du}}
\pgfpathlineto{\pgfpoint{42.267625\du}{14.446747\du}}
\pgfpathlineto{\pgfpoint{42.265434\du}{14.455510\du}}
\pgfpathlineto{\pgfpoint{42.264339\du}{14.464637\du}}
\pgfpathlineto{\pgfpoint{42.263609\du}{14.474130\du}}
\pgfpathlineto{\pgfpoint{42.263609\du}{14.482892\du}}
\pgfpathlineto{\pgfpoint{42.284054\du}{14.482892\du}}
\pgfpathlineto{\pgfpoint{42.284054\du}{14.474860\du}}
\pgfpathlineto{\pgfpoint{42.284419\du}{14.466462\du}}
\pgfpathlineto{\pgfpoint{42.285879\du}{14.458430\du}}
\pgfpathlineto{\pgfpoint{42.286975\du}{14.450763\du}}
\pgfpathlineto{\pgfpoint{42.288800\du}{14.442366\du}}
\pgfpathlineto{\pgfpoint{42.290626\du}{14.434334\du}}
\pgfpathlineto{\pgfpoint{42.293181\du}{14.426667\du}}
\pgfpathlineto{\pgfpoint{42.296102\du}{14.419000\du}}
\pgfpathlineto{\pgfpoint{42.299388\du}{14.411698\du}}
\pgfpathlineto{\pgfpoint{42.302674\du}{14.403301\du}}
\pgfpathlineto{\pgfpoint{42.306325\du}{14.395634\du}}
\pgfpathlineto{\pgfpoint{42.310706\du}{14.387967\du}}
\pgfpathlineto{\pgfpoint{42.315087\du}{14.381030\du}}
\pgfpathlineto{\pgfpoint{42.319833\du}{14.373363\du}}
\pgfpathlineto{\pgfpoint{42.324945\du}{14.366061\du}}
\pgfpathlineto{\pgfpoint{42.330056\du}{14.358759\du}}
\pgfpathlineto{\pgfpoint{42.336263\du}{14.351092\du}}
\pgfpathlineto{\pgfpoint{42.342469\du}{14.344155\du}}
\pgfpathlineto{\pgfpoint{42.348676\du}{14.337218\du}}
\pgfpathlineto{\pgfpoint{42.355613\du}{14.330281\du}}
\pgfpathlineto{\pgfpoint{42.362185\du}{14.323710\du}}
\pgfpathlineto{\pgfpoint{42.369852\du}{14.316773\du}}
\pgfpathlineto{\pgfpoint{42.377154\du}{14.309836\du}}
\pgfpathlineto{\pgfpoint{42.385186\du}{14.303264\du}}
\pgfpathlineto{\pgfpoint{42.393583\du}{14.296692\du}}
\pgfpathlineto{\pgfpoint{42.401980\du}{14.290486\du}}
\pgfpathlineto{\pgfpoint{42.411108\du}{14.284644\du}}
\pgfpathlineto{\pgfpoint{42.419505\du}{14.278073\du}}
\pgfpathlineto{\pgfpoint{42.429362\du}{14.272231\du}}
\pgfpathlineto{\pgfpoint{42.438855\du}{14.266389\du}}
\pgfpathlineto{\pgfpoint{42.448713\du}{14.260548\du}}
\pgfpathlineto{\pgfpoint{42.458935\du}{14.254706\du}}
\pgfpathlineto{\pgfpoint{42.469158\du}{14.249595\du}}
\pgfpathlineto{\pgfpoint{42.480111\du}{14.243753\du}}
\pgfpathlineto{\pgfpoint{42.490699\du}{14.239007\du}}
\pgfpathlineto{\pgfpoint{42.502017\du}{14.233531\du}}
\pgfpathlineto{\pgfpoint{42.512970\du}{14.228419\du}}
\pgfpathlineto{\pgfpoint{42.524653\du}{14.223673\du}}
\pgfpathlineto{\pgfpoint{42.535971\du}{14.218927\du}}
\pgfpathlineto{\pgfpoint{42.548384\du}{14.214181\du}}
\pgfpathlineto{\pgfpoint{42.560797\du}{14.210165\du}}
\pgfpathlineto{\pgfpoint{42.572846\du}{14.205418\du}}
\pgfpathlineto{\pgfpoint{42.585624\du}{14.201402\du}}
\pgfpathlineto{\pgfpoint{42.598767\du}{14.198116\du}}
\pgfpathlineto{\pgfpoint{42.611546\du}{14.193735\du}}
\pgfpathlineto{\pgfpoint{42.625054\du}{14.190449\du}}
\pgfpathlineto{\pgfpoint{42.638198\du}{14.186798\du}}
\pgfpathlineto{\pgfpoint{42.651706\du}{14.183878\du}}
\pgfpathlineto{\pgfpoint{42.665580\du}{14.180957\du}}
\pgfpathlineto{\pgfpoint{42.679819\du}{14.178036\du}}
\pgfpathlineto{\pgfpoint{42.693693\du}{14.175115\du}}
\pgfpathlineto{\pgfpoint{42.707566\du}{14.172925\du}}
\pgfpathlineto{\pgfpoint{42.721805\du}{14.170369\du}}
\pgfpathlineto{\pgfpoint{42.736409\du}{14.168178\du}}
\pgfpathlineto{\pgfpoint{42.751013\du}{14.166353\du}}
\pgfpathlineto{\pgfpoint{42.765982\du}{14.164527\du}}
\pgfpathlineto{\pgfpoint{42.780951\du}{14.162702\du}}
\pgfpathlineto{\pgfpoint{42.795920\du}{14.161607\du}}
\pgfpathlineto{\pgfpoint{42.810889\du}{14.160511\du}}
\pgfpathlineto{\pgfpoint{42.826588\du}{14.159416\du}}
\pgfpathlineto{\pgfpoint{42.841192\du}{14.158686\du}}
\pgfpathlineto{\pgfpoint{42.857256\du}{14.158321\du}}
\pgfpathlineto{\pgfpoint{42.872590\du}{14.157591\du}}
\pgfpathlineto{\pgfpoint{42.888289\du}{14.157591\du}}
\pgfpathlineto{\pgfpoint{42.888289\du}{14.157591\du}}
\pgfpathlineto{\pgfpoint{42.888289\du}{14.157591\du}}
\pgfpathlineto{\pgfpoint{42.889749\du}{14.157591\du}}
\pgfpathlineto{\pgfpoint{42.890845\du}{14.157591\du}}
\pgfpathlineto{\pgfpoint{42.891940\du}{14.157591\du}}
\pgfpathlineto{\pgfpoint{42.893035\du}{14.156860\du}}
\pgfpathlineto{\pgfpoint{42.894131\du}{14.156495\du}}
\pgfpathlineto{\pgfpoint{42.895226\du}{14.155765\du}}
\pgfpathlineto{\pgfpoint{42.895226\du}{14.155400\du}}
\pgfpathlineto{\pgfpoint{42.895956\du}{14.154670\du}}
\pgfpathlineto{\pgfpoint{42.897051\du}{14.152844\du}}
\pgfpathlineto{\pgfpoint{42.898147\du}{14.151019\du}}
\pgfpathlineto{\pgfpoint{42.898512\du}{14.149559\du}}
\pgfpathlineto{\pgfpoint{42.898512\du}{14.147733\du}}
\pgfpathlineto{\pgfpoint{42.898512\du}{14.145177\du}}
\pgfpathlineto{\pgfpoint{42.898147\du}{14.143717\du}}
\pgfpathlineto{\pgfpoint{42.897051\du}{14.141891\du}}
\pgfpathlineto{\pgfpoint{42.895956\du}{14.140796\du}}
\pgfpathlineto{\pgfpoint{42.895226\du}{14.139336\du}}
\pgfpathlineto{\pgfpoint{42.895226\du}{14.138971\du}}
\pgfpathlineto{\pgfpoint{42.894131\du}{14.138241\du}}
\pgfpathlineto{\pgfpoint{42.893035\du}{14.138241\du}}
\pgfpathlineto{\pgfpoint{42.891940\du}{14.137510\du}}
\pgfpathlineto{\pgfpoint{42.890845\du}{14.137510\du}}
\pgfpathlineto{\pgfpoint{42.889749\du}{14.137145\du}}
\pgfpathlineto{\pgfpoint{42.888289\du}{14.137145\du}}
\pgfusepath{fill}
\pgfsetbuttcap
\pgfsetmiterjoin
\pgfsetdash{}{0pt}
\definecolor{dialinecolor}{rgb}{0.678431, 0.839216, 0.905882}
\pgfsetfillcolor{dialinecolor}
\pgfpathmoveto{\pgfpoint{43.513335\du}{14.482892\du}}
\pgfpathlineto{\pgfpoint{43.513335\du}{14.473399\du}}
\pgfpathlineto{\pgfpoint{43.512605\du}{14.464637\du}}
\pgfpathlineto{\pgfpoint{43.511509\du}{14.455510\du}}
\pgfpathlineto{\pgfpoint{43.510049\du}{14.446747\du}}
\pgfpathlineto{\pgfpoint{43.507858\du}{14.437985\du}}
\pgfpathlineto{\pgfpoint{43.506033\du}{14.429223\du}}
\pgfpathlineto{\pgfpoint{43.503112\du}{14.420460\du}}
\pgfpathlineto{\pgfpoint{43.500556\du}{14.411698\du}}
\pgfpathlineto{\pgfpoint{43.496540\du}{14.403301\du}}
\pgfpathlineto{\pgfpoint{43.493254\du}{14.394538\du}}
\pgfpathlineto{\pgfpoint{43.488873\du}{14.386506\du}}
\pgfpathlineto{\pgfpoint{43.484127\du}{14.378109\du}}
\pgfpathlineto{\pgfpoint{43.479381\du}{14.370077\du}}
\pgfpathlineto{\pgfpoint{43.474634\du}{14.361680\du}}
\pgfpathlineto{\pgfpoint{43.468793\du}{14.354013\du}}
\pgfpathlineto{\pgfpoint{43.462586\du}{14.345981\du}}
\pgfpathlineto{\pgfpoint{43.456380\du}{14.338314\du}}
\pgfpathlineto{\pgfpoint{43.449808\du}{14.330646\du}}
\pgfpathlineto{\pgfpoint{43.443236\du}{14.323345\du}}
\pgfpathlineto{\pgfpoint{43.435569\du}{14.315678\du}}
\pgfpathlineto{\pgfpoint{43.428632\du}{14.308376\du}}
\pgfpathlineto{\pgfpoint{43.420600\du}{14.301439\du}}
\pgfpathlineto{\pgfpoint{43.412568\du}{14.294502\du}}
\pgfpathlineto{\pgfpoint{43.404171\du}{14.287565\du}}
\pgfpathlineto{\pgfpoint{43.395774\du}{14.280628\du}}
\pgfpathlineto{\pgfpoint{43.387011\du}{14.274056\du}}
\pgfpathlineto{\pgfpoint{43.377884\du}{14.267485\du}}
\pgfpathlineto{\pgfpoint{43.368391\du}{14.261278\du}}
\pgfpathlineto{\pgfpoint{43.358169\du}{14.254706\du}}
\pgfpathlineto{\pgfpoint{43.348311\du}{14.248865\du}}
\pgfpathlineto{\pgfpoint{43.338088\du}{14.242293\du}}
\pgfpathlineto{\pgfpoint{43.328231\du}{14.236452\du}}
\pgfpathlineto{\pgfpoint{43.316913\du}{14.231340\du}}
\pgfpathlineto{\pgfpoint{43.305960\du}{14.225499\du}}
\pgfpathlineto{\pgfpoint{43.294642\du}{14.220022\du}}
\pgfpathlineto{\pgfpoint{43.283324\du}{14.214911\du}}
\pgfpathlineto{\pgfpoint{43.271641\du}{14.209799\du}}
\pgfpathlineto{\pgfpoint{43.260323\du}{14.205053\du}}
\pgfpathlineto{\pgfpoint{43.247544\du}{14.199577\du}}
\pgfpathlineto{\pgfpoint{43.235861\du}{14.195196\du}}
\pgfpathlineto{\pgfpoint{43.223083\du}{14.190814\du}}
\pgfpathlineto{\pgfpoint{43.210304\du}{14.186433\du}}
\pgfpathlineto{\pgfpoint{43.197161\du}{14.182052\du}}
\pgfpathlineto{\pgfpoint{43.184017\du}{14.178036\du}}
\pgfpathlineto{\pgfpoint{43.170874\du}{14.174385\du}}
\pgfpathlineto{\pgfpoint{43.157365\du}{14.170369\du}}
\pgfpathlineto{\pgfpoint{43.143492\du}{14.167083\du}}
\pgfpathlineto{\pgfpoint{43.129983\du}{14.163432\du}}
\pgfpathlineto{\pgfpoint{43.116109\du}{14.160511\du}}
\pgfpathlineto{\pgfpoint{43.101506\du}{14.157591\du}}
\pgfpathlineto{\pgfpoint{43.087632\du}{14.155400\du}}
\pgfpathlineto{\pgfpoint{43.073393\du}{14.152479\du}}
\pgfpathlineto{\pgfpoint{43.058059\du}{14.149924\du}}
\pgfpathlineto{\pgfpoint{43.043090\du}{14.147733\du}}
\pgfpathlineto{\pgfpoint{43.028121\du}{14.145908\du}}
\pgfpathlineto{\pgfpoint{43.013882\du}{14.144082\du}}
\pgfpathlineto{\pgfpoint{42.997818\du}{14.142257\du}}
\pgfpathlineto{\pgfpoint{42.982849\du}{14.141161\du}}
\pgfpathlineto{\pgfpoint{42.967515\du}{14.140066\du}}
\pgfpathlineto{\pgfpoint{42.951816\du}{14.138971\du}}
\pgfpathlineto{\pgfpoint{42.936482\du}{14.138241\du}}
\pgfpathlineto{\pgfpoint{42.920418\du}{14.137510\du}}
\pgfpathlineto{\pgfpoint{42.904353\du}{14.137145\du}}
\pgfpathlineto{\pgfpoint{42.888289\du}{14.137145\du}}
\pgfpathlineto{\pgfpoint{42.888289\du}{14.157591\du}}
\pgfpathlineto{\pgfpoint{42.904353\du}{14.157591\du}}
\pgfpathlineto{\pgfpoint{42.919687\du}{14.158321\du}}
\pgfpathlineto{\pgfpoint{42.935387\du}{14.158686\du}}
\pgfpathlineto{\pgfpoint{42.950721\du}{14.159416\du}}
\pgfpathlineto{\pgfpoint{42.965690\du}{14.160511\du}}
\pgfpathlineto{\pgfpoint{42.981389\du}{14.161607\du}}
\pgfpathlineto{\pgfpoint{42.995993\du}{14.162702\du}}
\pgfpathlineto{\pgfpoint{43.011327\du}{14.164527\du}}
\pgfpathlineto{\pgfpoint{43.026296\du}{14.166353\du}}
\pgfpathlineto{\pgfpoint{43.040900\du}{14.168178\du}}
\pgfpathlineto{\pgfpoint{43.055138\du}{14.170369\du}}
\pgfpathlineto{\pgfpoint{43.069377\du}{14.172925\du}}
\pgfpathlineto{\pgfpoint{43.083616\du}{14.175115\du}}
\pgfpathlineto{\pgfpoint{43.097490\du}{14.178036\du}}
\pgfpathlineto{\pgfpoint{43.111363\du}{14.180957\du}}
\pgfpathlineto{\pgfpoint{43.125237\du}{14.183878\du}}
\pgfpathlineto{\pgfpoint{43.138745\du}{14.186798\du}}
\pgfpathlineto{\pgfpoint{43.152254\du}{14.190449\du}}
\pgfpathlineto{\pgfpoint{43.165398\du}{14.193735\du}}
\pgfpathlineto{\pgfpoint{43.178176\du}{14.198116\du}}
\pgfpathlineto{\pgfpoint{43.191685\du}{14.201402\du}}
\pgfpathlineto{\pgfpoint{43.204463\du}{14.205418\du}}
\pgfpathlineto{\pgfpoint{43.216146\du}{14.210165\du}}
\pgfpathlineto{\pgfpoint{43.228559\du}{14.214181\du}}
\pgfpathlineto{\pgfpoint{43.240607\du}{14.218927\du}}
\pgfpathlineto{\pgfpoint{43.252291\du}{14.223673\du}}
\pgfpathlineto{\pgfpoint{43.263609\du}{14.228419\du}}
\pgfpathlineto{\pgfpoint{43.274927\du}{14.233531\du}}
\pgfpathlineto{\pgfpoint{43.286610\du}{14.239007\du}}
\pgfpathlineto{\pgfpoint{43.297197\du}{14.243753\du}}
\pgfpathlineto{\pgfpoint{43.307785\du}{14.249595\du}}
\pgfpathlineto{\pgfpoint{43.318373\du}{14.254706\du}}
\pgfpathlineto{\pgfpoint{43.328231\du}{14.260548\du}}
\pgfpathlineto{\pgfpoint{43.338088\du}{14.266389\du}}
\pgfpathlineto{\pgfpoint{43.347216\du}{14.272231\du}}
\pgfpathlineto{\pgfpoint{43.356343\du}{14.278073\du}}
\pgfpathlineto{\pgfpoint{43.365836\du}{14.284644\du}}
\pgfpathlineto{\pgfpoint{43.374598\du}{14.290486\du}}
\pgfpathlineto{\pgfpoint{43.383725\du}{14.296692\du}}
\pgfpathlineto{\pgfpoint{43.391758\du}{14.303264\du}}
\pgfpathlineto{\pgfpoint{43.400155\du}{14.309836\du}}
\pgfpathlineto{\pgfpoint{43.407092\du}{14.316773\du}}
\pgfpathlineto{\pgfpoint{43.414394\du}{14.323710\du}}
\pgfpathlineto{\pgfpoint{43.421330\du}{14.330281\du}}
\pgfpathlineto{\pgfpoint{43.428632\du}{14.337218\du}}
\pgfpathlineto{\pgfpoint{43.434474\du}{14.344155\du}}
\pgfpathlineto{\pgfpoint{43.441411\du}{14.351092\du}}
\pgfpathlineto{\pgfpoint{43.446522\du}{14.358759\du}}
\pgfpathlineto{\pgfpoint{43.451998\du}{14.366061\du}}
\pgfpathlineto{\pgfpoint{43.456745\du}{14.373363\du}}
\pgfpathlineto{\pgfpoint{43.461856\du}{14.381030\du}}
\pgfpathlineto{\pgfpoint{43.465872\du}{14.387967\du}}
\pgfpathlineto{\pgfpoint{43.470618\du}{14.395634\du}}
\pgfpathlineto{\pgfpoint{43.474634\du}{14.403301\du}}
\pgfpathlineto{\pgfpoint{43.477920\du}{14.411698\du}}
\pgfpathlineto{\pgfpoint{43.480476\du}{14.419000\du}}
\pgfpathlineto{\pgfpoint{43.483762\du}{14.426667\du}}
\pgfpathlineto{\pgfpoint{43.485952\du}{14.434334\du}}
\pgfpathlineto{\pgfpoint{43.488508\du}{14.442366\du}}
\pgfpathlineto{\pgfpoint{43.489969\du}{14.450763\du}}
\pgfpathlineto{\pgfpoint{43.491429\du}{14.458430\du}}
\pgfpathlineto{\pgfpoint{43.492159\du}{14.466462\du}}
\pgfpathlineto{\pgfpoint{43.492524\du}{14.474860\du}}
\pgfpathlineto{\pgfpoint{43.493254\du}{14.482892\du}}
\pgfpathlineto{\pgfpoint{43.513335\du}{14.482892\du}}
\pgfusepath{fill}
\pgfsetbuttcap
\pgfsetmiterjoin
\pgfsetdash{}{0pt}
\definecolor{dialinecolor}{rgb}{0.074510, 0.082353, 0.086275}
\pgfsetfillcolor{dialinecolor}
\pgfpathmoveto{\pgfpoint{42.567369\du}{14.577087\du}}
\pgfpathlineto{\pgfpoint{42.794824\du}{14.348901\du}}
\pgfpathlineto{\pgfpoint{42.734948\du}{14.287930\du}}
\pgfpathlineto{\pgfpoint{42.915306\du}{14.287930\du}}
\pgfpathlineto{\pgfpoint{42.915306\du}{14.476320\du}}
\pgfpathlineto{\pgfpoint{42.855065\du}{14.416079\du}}
\pgfpathlineto{\pgfpoint{42.634912\du}{14.637328\du}}
\pgfpathlineto{\pgfpoint{42.567369\du}{14.577087\du}}
\pgfusepath{fill}
\pgfsetbuttcap
\pgfsetmiterjoin
\pgfsetdash{}{0pt}
\definecolor{dialinecolor}{rgb}{0.074510, 0.082353, 0.086275}
\pgfsetfillcolor{dialinecolor}
\pgfpathmoveto{\pgfpoint{42.835350\du}{14.697569\du}}
\pgfpathlineto{\pgfpoint{43.062440\du}{14.469383\du}}
\pgfpathlineto{\pgfpoint{43.001834\du}{14.409142\du}}
\pgfpathlineto{\pgfpoint{43.182922\du}{14.409142\du}}
\pgfpathlineto{\pgfpoint{43.182922\du}{14.597167\du}}
\pgfpathlineto{\pgfpoint{43.122316\du}{14.536926\du}}
\pgfpathlineto{\pgfpoint{42.901798\du}{14.757810\du}}
\pgfpathlineto{\pgfpoint{42.835350\du}{14.697569\du}}
\pgfusepath{fill}
\pgfsetbuttcap
\pgfsetmiterjoin
\pgfsetdash{}{0pt}
\definecolor{dialinecolor}{rgb}{1.000000, 1.000000, 1.000000}
\pgfsetfillcolor{dialinecolor}
\pgfpathmoveto{\pgfpoint{42.554226\du}{14.563578\du}}
\pgfpathlineto{\pgfpoint{42.781316\du}{14.335393\du}}
\pgfpathlineto{\pgfpoint{42.721805\du}{14.275152\du}}
\pgfpathlineto{\pgfpoint{42.901798\du}{14.275152\du}}
\pgfpathlineto{\pgfpoint{42.901798\du}{14.463177\du}}
\pgfpathlineto{\pgfpoint{42.841922\du}{14.402205\du}}
\pgfpathlineto{\pgfpoint{42.621403\du}{14.623819\du}}
\pgfpathlineto{\pgfpoint{42.554226\du}{14.563578\du}}
\pgfusepath{fill}
\pgfsetbuttcap
\pgfsetmiterjoin
\pgfsetdash{}{0pt}
\definecolor{dialinecolor}{rgb}{1.000000, 1.000000, 1.000000}
\pgfsetfillcolor{dialinecolor}
\pgfpathmoveto{\pgfpoint{42.821841\du}{14.684060\du}}
\pgfpathlineto{\pgfpoint{43.048932\du}{14.455875\du}}
\pgfpathlineto{\pgfpoint{42.988691\du}{14.395634\du}}
\pgfpathlineto{\pgfpoint{43.169049\du}{14.395634\du}}
\pgfpathlineto{\pgfpoint{43.169049\du}{14.583659\du}}
\pgfpathlineto{\pgfpoint{43.109538\du}{14.523418\du}}
\pgfpathlineto{\pgfpoint{42.888289\du}{14.744301\du}}
\pgfpathlineto{\pgfpoint{42.821841\du}{14.684060\du}}
\pgfusepath{fill}
% setfont left to latex
\definecolor{dialinecolor}{rgb}{0.000000, 0.000000, 0.000000}
\pgfsetstrokecolor{dialinecolor}
\node[anchor=west] at (49.048023\du,10.818986\du){IP : 192.5.5.241};
% setfont left to latex
\definecolor{dialinecolor}{rgb}{0.000000, 0.000000, 0.000000}
\pgfsetstrokecolor{dialinecolor}
\node[anchor=west] at (49.094470\du,9.635534\du){Hop : 7};
% setfont left to latex
\definecolor{dialinecolor}{rgb}{0.000000, 0.000000, 0.000000}
\pgfsetstrokecolor{dialinecolor}
\node[anchor=west] at (49.280866\du,12.025483\du){RTTs : 8.228, 8.026, 8.254};
\pgfsetlinewidth{0.000000\du}
\pgfsetdash{}{0pt}
\pgfsetdash{}{0pt}
\pgfsetbuttcap
\pgfsetmiterjoin
\pgfsetlinewidth{0.000000\du}
\pgfsetbuttcap
\pgfsetmiterjoin
\pgfsetdash{}{0pt}
\definecolor{dialinecolor}{rgb}{0.027451, 0.486275, 0.682353}
\pgfsetfillcolor{dialinecolor}
\pgfpathmoveto{\pgfpoint{54.397099\du}{14.377332\du}}
\pgfpathlineto{\pgfpoint{54.395639\du}{14.406540\du}}
\pgfpathlineto{\pgfpoint{54.388337\du}{14.436478\du}}
\pgfpathlineto{\pgfpoint{54.378114\du}{14.464955\du}}
\pgfpathlineto{\pgfpoint{54.363510\du}{14.493068\du}}
\pgfpathlineto{\pgfpoint{54.344160\du}{14.521180\du}}
\pgfpathlineto{\pgfpoint{54.322254\du}{14.548562\du}}
\pgfpathlineto{\pgfpoint{54.295602\du}{14.575580\du}}
\pgfpathlineto{\pgfpoint{54.264934\du}{14.601867\du}}
\pgfpathlineto{\pgfpoint{54.232075\du}{14.627058\du}}
\pgfpathlineto{\pgfpoint{54.194836\du}{14.652250\du}}
\pgfpathlineto{\pgfpoint{54.153945\du}{14.676346\du}}
\pgfpathlineto{\pgfpoint{54.110133\du}{14.699712\du}}
\pgfpathlineto{\pgfpoint{54.063766\du}{14.722348\du}}
\pgfpathlineto{\pgfpoint{54.013382\du}{14.744254\du}}
\pgfpathlineto{\pgfpoint{53.960443\du}{14.765065\du}}
\pgfpathlineto{\pgfpoint{53.904949\du}{14.785145\du}}
\pgfpathlineto{\pgfpoint{53.846898\du}{14.804495\du}}
\pgfpathlineto{\pgfpoint{53.786292\du}{14.822385\du}}
\pgfpathlineto{\pgfpoint{53.722400\du}{14.839545\du}}
\pgfpathlineto{\pgfpoint{53.657413\du}{14.855974\du}}
\pgfpathlineto{\pgfpoint{53.588775\du}{14.870943\du}}
\pgfpathlineto{\pgfpoint{53.517946\du}{14.884451\du}}
\pgfpathlineto{\pgfpoint{53.446022\du}{14.897230\du}}
\pgfpathlineto{\pgfpoint{53.371177\du}{14.909278\du}}
\pgfpathlineto{\pgfpoint{53.295237\du}{14.919136\du}}
\pgfpathlineto{\pgfpoint{53.216741\du}{14.928263\du}}
\pgfpathlineto{\pgfpoint{53.137150\du}{14.935930\du}}
\pgfpathlineto{\pgfpoint{53.056099\du}{14.942502\du}}
\pgfpathlineto{\pgfpoint{52.972857\du}{14.947613\du}}
\pgfpathlineto{\pgfpoint{52.889250\du}{14.951264\du}}
\pgfpathlineto{\pgfpoint{52.803817\du}{14.953455\du}}
\pgfpathlineto{\pgfpoint{52.717289\du}{14.954185\du}}
\pgfpathlineto{\pgfpoint{52.631126\du}{14.953455\du}}
\pgfpathlineto{\pgfpoint{52.545328\du}{14.951264\du}}
\pgfpathlineto{\pgfpoint{52.461721\du}{14.947613\du}}
\pgfpathlineto{\pgfpoint{52.378844\du}{14.942502\du}}
\pgfpathlineto{\pgfpoint{52.297428\du}{14.935930\du}}
\pgfpathlineto{\pgfpoint{52.217837\du}{14.928263\du}}
\pgfpathlineto{\pgfpoint{52.140071\du}{14.919136\du}}
\pgfpathlineto{\pgfpoint{52.063401\du}{14.909278\du}}
\pgfpathlineto{\pgfpoint{51.989286\du}{14.897230\du}}
\pgfpathlineto{\pgfpoint{51.916632\du}{14.884451\du}}
\pgfpathlineto{\pgfpoint{51.846168\du}{14.870943\du}}
\pgfpathlineto{\pgfpoint{51.777530\du}{14.855974\du}}
\pgfpathlineto{\pgfpoint{51.711813\du}{14.839545\du}}
\pgfpathlineto{\pgfpoint{51.648286\du}{14.822385\du}}
\pgfpathlineto{\pgfpoint{51.587315\du}{14.804495\du}}
\pgfpathlineto{\pgfpoint{51.528899\du}{14.785145\du}}
\pgfpathlineto{\pgfpoint{51.473769\du}{14.765065\du}}
\pgfpathlineto{\pgfpoint{51.420830\du}{14.744254\du}}
\pgfpathlineto{\pgfpoint{51.370812\du}{14.722348\du}}
\pgfpathlineto{\pgfpoint{51.323715\du}{14.699712\du}}
\pgfpathlineto{\pgfpoint{51.280268\du}{14.676346\du}}
\pgfpathlineto{\pgfpoint{51.239377\du}{14.652250\du}}
\pgfpathlineto{\pgfpoint{51.202137\du}{14.627058\du}}
\pgfpathlineto{\pgfpoint{51.168914\du}{14.601867\du}}
\pgfpathlineto{\pgfpoint{51.138611\du}{14.575580\du}}
\pgfpathlineto{\pgfpoint{51.111959\du}{14.548562\du}}
\pgfpathlineto{\pgfpoint{51.090053\du}{14.521180\du}}
\pgfpathlineto{\pgfpoint{51.070703\du}{14.493068\du}}
\pgfpathlineto{\pgfpoint{51.056099\du}{14.464955\du}}
\pgfpathlineto{\pgfpoint{51.045511\du}{14.436478\du}}
\pgfpathlineto{\pgfpoint{51.038574\du}{14.406540\du}}
\pgfpathlineto{\pgfpoint{51.036749\du}{14.377332\du}}
\pgfpathlineto{\pgfpoint{51.038574\du}{14.347394\du}}
\pgfpathlineto{\pgfpoint{51.045511\du}{14.318186\du}}
\pgfpathlineto{\pgfpoint{51.056099\du}{14.288979\du}}
\pgfpathlineto{\pgfpoint{51.070703\du}{14.260866\du}}
\pgfpathlineto{\pgfpoint{51.090053\du}{14.232754\du}}
\pgfpathlineto{\pgfpoint{51.111959\du}{14.205371\du}}
\pgfpathlineto{\pgfpoint{51.138611\du}{14.178719\du}}
\pgfpathlineto{\pgfpoint{51.168914\du}{14.152432\du}}
\pgfpathlineto{\pgfpoint{51.202137\du}{14.126876\du}}
\pgfpathlineto{\pgfpoint{51.239377\du}{14.102049\du}}
\pgfpathlineto{\pgfpoint{51.280268\du}{14.077588\du}}
\pgfpathlineto{\pgfpoint{51.323715\du}{14.054221\du}}
\pgfpathlineto{\pgfpoint{51.370812\du}{14.031951\du}}
\pgfpathlineto{\pgfpoint{51.420830\du}{14.009680\du}}
\pgfpathlineto{\pgfpoint{51.473769\du}{13.988869\du}}
\pgfpathlineto{\pgfpoint{51.528899\du}{13.968789\du}}
\pgfpathlineto{\pgfpoint{51.587315\du}{13.950169\du}}
\pgfpathlineto{\pgfpoint{51.648286\du}{13.931184\du}}
\pgfpathlineto{\pgfpoint{51.711813\du}{13.914389\du}}
\pgfpathlineto{\pgfpoint{51.777530\du}{13.898690\du}}
\pgfpathlineto{\pgfpoint{51.846168\du}{13.883356\du}}
\pgfpathlineto{\pgfpoint{51.916632\du}{13.869482\du}}
\pgfpathlineto{\pgfpoint{51.989286\du}{13.856339\du}}
\pgfpathlineto{\pgfpoint{52.063401\du}{13.845386\du}}
\pgfpathlineto{\pgfpoint{52.140071\du}{13.834798\du}}
\pgfpathlineto{\pgfpoint{52.217837\du}{13.825306\du}}
\pgfpathlineto{\pgfpoint{52.297428\du}{13.818004\du}}
\pgfpathlineto{\pgfpoint{52.378844\du}{13.811432\du}}
\pgfpathlineto{\pgfpoint{52.461721\du}{13.806321\du}}
\pgfpathlineto{\pgfpoint{52.545328\du}{13.802670\du}}
\pgfpathlineto{\pgfpoint{52.631126\du}{13.800844\du}}
\pgfpathlineto{\pgfpoint{52.717289\du}{13.799749\du}}
\pgfpathlineto{\pgfpoint{52.803817\du}{13.800844\du}}
\pgfpathlineto{\pgfpoint{52.889250\du}{13.802670\du}}
\pgfpathlineto{\pgfpoint{52.972857\du}{13.806321\du}}
\pgfpathlineto{\pgfpoint{53.056099\du}{13.811432\du}}
\pgfpathlineto{\pgfpoint{53.137150\du}{13.818004\du}}
\pgfpathlineto{\pgfpoint{53.216741\du}{13.825306\du}}
\pgfpathlineto{\pgfpoint{53.295237\du}{13.834798\du}}
\pgfpathlineto{\pgfpoint{53.371177\du}{13.845386\du}}
\pgfpathlineto{\pgfpoint{53.446022\du}{13.856339\du}}
\pgfpathlineto{\pgfpoint{53.517946\du}{13.869482\du}}
\pgfpathlineto{\pgfpoint{53.588775\du}{13.883356\du}}
\pgfpathlineto{\pgfpoint{53.657413\du}{13.898690\du}}
\pgfpathlineto{\pgfpoint{53.722400\du}{13.914389\du}}
\pgfpathlineto{\pgfpoint{53.786292\du}{13.931184\du}}
\pgfpathlineto{\pgfpoint{53.846898\du}{13.950169\du}}
\pgfpathlineto{\pgfpoint{53.904949\du}{13.968789\du}}
\pgfpathlineto{\pgfpoint{53.960443\du}{13.988869\du}}
\pgfpathlineto{\pgfpoint{54.013382\du}{14.009680\du}}
\pgfpathlineto{\pgfpoint{54.063766\du}{14.031951\du}}
\pgfpathlineto{\pgfpoint{54.110133\du}{14.054221\du}}
\pgfpathlineto{\pgfpoint{54.153945\du}{14.077588\du}}
\pgfpathlineto{\pgfpoint{54.194836\du}{14.102049\du}}
\pgfpathlineto{\pgfpoint{54.232075\du}{14.126876\du}}
\pgfpathlineto{\pgfpoint{54.264934\du}{14.152432\du}}
\pgfpathlineto{\pgfpoint{54.295602\du}{14.178719\du}}
\pgfpathlineto{\pgfpoint{54.322254\du}{14.205371\du}}
\pgfpathlineto{\pgfpoint{54.344160\du}{14.232754\du}}
\pgfpathlineto{\pgfpoint{54.363510\du}{14.260866\du}}
\pgfpathlineto{\pgfpoint{54.378114\du}{14.288979\du}}
\pgfpathlineto{\pgfpoint{54.388337\du}{14.318186\du}}
\pgfpathlineto{\pgfpoint{54.395639\du}{14.347394\du}}
\pgfpathlineto{\pgfpoint{54.397099\du}{14.377332\du}}
\pgfusepath{fill}
\pgfsetlinewidth{0.000000\du}
\pgfsetbuttcap
\pgfsetmiterjoin
\pgfsetdash{}{0pt}
\definecolor{dialinecolor}{rgb}{0.678431, 0.839216, 0.905882}
\pgfsetfillcolor{dialinecolor}
\pgfpathmoveto{\pgfpoint{52.717289\du}{14.964773\du}}
\pgfpathlineto{\pgfpoint{52.717289\du}{14.964773\du}}
\pgfpathlineto{\pgfpoint{52.760736\du}{14.964773\du}}
\pgfpathlineto{\pgfpoint{52.804182\du}{14.964043\du}}
\pgfpathlineto{\pgfpoint{52.847263\du}{14.962947\du}}
\pgfpathlineto{\pgfpoint{52.889250\du}{14.961852\du}}
\pgfpathlineto{\pgfpoint{52.931966\du}{14.960026\du}}
\pgfpathlineto{\pgfpoint{52.973587\du}{14.957836\du}}
\pgfpathlineto{\pgfpoint{53.015208\du}{14.955280\du}}
\pgfpathlineto{\pgfpoint{53.056829\du}{14.953090\du}}
\pgfpathlineto{\pgfpoint{53.097355\du}{14.950169\du}}
\pgfpathlineto{\pgfpoint{53.138246\du}{14.946518\du}}
\pgfpathlineto{\pgfpoint{53.178041\du}{14.942502\du}}
\pgfpathlineto{\pgfpoint{53.218202\du}{14.938486\du}}
\pgfpathlineto{\pgfpoint{53.256902\du}{14.934105\du}}
\pgfpathlineto{\pgfpoint{53.296332\du}{14.929723\du}}
\pgfpathlineto{\pgfpoint{53.334303\du}{14.924247\du}}
\pgfpathlineto{\pgfpoint{53.373368\du}{14.919136\du}}
\pgfpathlineto{\pgfpoint{53.410608\du}{14.913659\du}}
\pgfpathlineto{\pgfpoint{53.447483\du}{14.907453\du}}
\pgfpathlineto{\pgfpoint{53.483992\du}{14.901611\du}}
\pgfpathlineto{\pgfpoint{53.520502\du}{14.895039\du}}
\pgfpathlineto{\pgfpoint{53.555916\du}{14.888102\du}}
\pgfpathlineto{\pgfpoint{53.590600\du}{14.881166\du}}
\pgfpathlineto{\pgfpoint{53.625285\du}{14.873499\du}}
\pgfpathlineto{\pgfpoint{53.659239\du}{14.865832\du}}
\pgfpathlineto{\pgfpoint{53.692828\du}{14.857434\du}}
\pgfpathlineto{\pgfpoint{53.725686\du}{14.849402\du}}
\pgfpathlineto{\pgfpoint{53.757450\du}{14.841370\du}}
\pgfpathlineto{\pgfpoint{53.788848\du}{14.832608\du}}
\pgfpathlineto{\pgfpoint{53.804182\du}{14.827861\du}}
\pgfpathlineto{\pgfpoint{53.819516\du}{14.823845\du}}
\pgfpathlineto{\pgfpoint{53.835580\du}{14.819099\du}}
\pgfpathlineto{\pgfpoint{53.850184\du}{14.814353\du}}
\pgfpathlineto{\pgfpoint{53.864423\du}{14.809607\du}}
\pgfpathlineto{\pgfpoint{53.879392\du}{14.804495\du}}
\pgfpathlineto{\pgfpoint{53.894361\du}{14.799749\du}}
\pgfpathlineto{\pgfpoint{53.908235\du}{14.795003\du}}
\pgfpathlineto{\pgfpoint{53.922473\du}{14.789891\du}}
\pgfpathlineto{\pgfpoint{53.936347\du}{14.785145\du}}
\pgfpathlineto{\pgfpoint{53.950586\du}{14.779669\du}}
\pgfpathlineto{\pgfpoint{53.963729\du}{14.775288\du}}
\pgfpathlineto{\pgfpoint{53.977968\du}{14.769811\du}}
\pgfpathlineto{\pgfpoint{53.991112\du}{14.764700\du}}
\pgfpathlineto{\pgfpoint{54.004255\du}{14.759223\du}}
\pgfpathlineto{\pgfpoint{54.017764\du}{14.753382\du}}
\pgfpathlineto{\pgfpoint{54.030542\du}{14.748270\du}}
\pgfpathlineto{\pgfpoint{54.042590\du}{14.742794\du}}
\pgfpathlineto{\pgfpoint{54.055369\du}{14.736952\du}}
\pgfpathlineto{\pgfpoint{54.067782\du}{14.731841\du}}
\pgfpathlineto{\pgfpoint{54.080195\du}{14.725999\du}}
\pgfpathlineto{\pgfpoint{54.091513\du}{14.720523\du}}
\pgfpathlineto{\pgfpoint{54.103561\du}{14.714681\du}}
\pgfpathlineto{\pgfpoint{54.114514\du}{14.708840\du}}
\pgfpathlineto{\pgfpoint{54.126197\du}{14.702998\du}}
\pgfpathlineto{\pgfpoint{54.137515\du}{14.697157\du}}
\pgfpathlineto{\pgfpoint{54.148833\du}{14.691315\du}}
\pgfpathlineto{\pgfpoint{54.159421\du}{14.685109\du}}
\pgfpathlineto{\pgfpoint{54.169279\du}{14.679267\du}}
\pgfpathlineto{\pgfpoint{54.179867\du}{14.673426\du}}
\pgfpathlineto{\pgfpoint{54.190089\du}{14.666854\du}}
\pgfpathlineto{\pgfpoint{54.199947\du}{14.661012\du}}
\pgfpathlineto{\pgfpoint{54.209805\du}{14.654440\du}}
\pgfpathlineto{\pgfpoint{54.218932\du}{14.648234\du}}
\pgfpathlineto{\pgfpoint{54.228059\du}{14.641662\du}}
\pgfpathlineto{\pgfpoint{54.237552\du}{14.635821\du}}
\pgfpathlineto{\pgfpoint{54.246314\du}{14.629249\du}}
\pgfpathlineto{\pgfpoint{54.255442\du}{14.623042\du}}
\pgfpathlineto{\pgfpoint{54.263474\du}{14.616470\du}}
\pgfpathlineto{\pgfpoint{54.272236\du}{14.609534\du}}
\pgfpathlineto{\pgfpoint{54.279538\du}{14.602962\du}}
\pgfpathlineto{\pgfpoint{54.287570\du}{14.596755\du}}
\pgfpathlineto{\pgfpoint{54.295602\du}{14.589453\du}}
\pgfpathlineto{\pgfpoint{54.302174\du}{14.583247\du}}
\pgfpathlineto{\pgfpoint{54.309476\du}{14.576310\du}}
\pgfpathlineto{\pgfpoint{54.316048\du}{14.569008\du}}
\pgfpathlineto{\pgfpoint{54.322985\du}{14.562801\du}}
\pgfpathlineto{\pgfpoint{54.329556\du}{14.555864\du}}
\pgfpathlineto{\pgfpoint{54.335763\du}{14.548562\du}}
\pgfpathlineto{\pgfpoint{54.341604\du}{14.541626\du}}
\pgfpathlineto{\pgfpoint{54.347081\du}{14.534689\du}}
\pgfpathlineto{\pgfpoint{54.352922\du}{14.527752\du}}
\pgfpathlineto{\pgfpoint{54.357669\du}{14.520450\du}}
\pgfpathlineto{\pgfpoint{54.363145\du}{14.513148\du}}
\pgfpathlineto{\pgfpoint{54.367891\du}{14.505846\du}}
\pgfpathlineto{\pgfpoint{54.372273\du}{14.498909\du}}
\pgfpathlineto{\pgfpoint{54.376289\du}{14.491242\du}}
\pgfpathlineto{\pgfpoint{54.380305\du}{14.484305\du}}
\pgfpathlineto{\pgfpoint{54.383591\du}{14.476638\du}}
\pgfpathlineto{\pgfpoint{54.387607\du}{14.468971\du}}
\pgfpathlineto{\pgfpoint{54.390893\du}{14.461304\du}}
\pgfpathlineto{\pgfpoint{54.393448\du}{14.454002\du}}
\pgfpathlineto{\pgfpoint{54.396369\du}{14.446700\du}}
\pgfpathlineto{\pgfpoint{54.398194\du}{14.439398\du}}
\pgfpathlineto{\pgfpoint{54.401115\du}{14.431731\du}}
\pgfpathlineto{\pgfpoint{54.402211\du}{14.423334\du}}
\pgfpathlineto{\pgfpoint{54.404401\du}{14.415667\du}}
\pgfpathlineto{\pgfpoint{54.405496\du}{14.408365\du}}
\pgfpathlineto{\pgfpoint{54.406227\du}{14.400698\du}}
\pgfpathlineto{\pgfpoint{54.406957\du}{14.392301\du}}
\pgfpathlineto{\pgfpoint{54.407687\du}{14.384999\du}}
\pgfpathlineto{\pgfpoint{54.407687\du}{14.377332\du}}
\pgfpathlineto{\pgfpoint{54.387607\du}{14.377332\du}}
\pgfpathlineto{\pgfpoint{54.386876\du}{14.384269\du}}
\pgfpathlineto{\pgfpoint{54.386876\du}{14.391206\du}}
\pgfpathlineto{\pgfpoint{54.386511\du}{14.398143\du}}
\pgfpathlineto{\pgfpoint{54.384686\du}{14.405444\du}}
\pgfpathlineto{\pgfpoint{54.383591\du}{14.412381\du}}
\pgfpathlineto{\pgfpoint{54.382860\du}{14.419318\du}}
\pgfpathlineto{\pgfpoint{54.380670\du}{14.426255\du}}
\pgfpathlineto{\pgfpoint{54.379209\du}{14.433557\du}}
\pgfpathlineto{\pgfpoint{54.377019\du}{14.439764\du}}
\pgfpathlineto{\pgfpoint{54.374463\du}{14.446700\du}}
\pgfpathlineto{\pgfpoint{54.371542\du}{14.454002\du}}
\pgfpathlineto{\pgfpoint{54.368987\du}{14.460939\du}}
\pgfpathlineto{\pgfpoint{54.364971\du}{14.467876\du}}
\pgfpathlineto{\pgfpoint{54.362050\du}{14.474448\du}}
\pgfpathlineto{\pgfpoint{54.358034\du}{14.481385\du}}
\pgfpathlineto{\pgfpoint{54.355113\du}{14.487956\du}}
\pgfpathlineto{\pgfpoint{54.350732\du}{14.494893\du}}
\pgfpathlineto{\pgfpoint{54.346351\du}{14.501830\du}}
\pgfpathlineto{\pgfpoint{54.341604\du}{14.508402\du}}
\pgfpathlineto{\pgfpoint{54.336493\du}{14.514608\du}}
\pgfpathlineto{\pgfpoint{54.331747\du}{14.521910\du}}
\pgfpathlineto{\pgfpoint{54.325540\du}{14.528847\du}}
\pgfpathlineto{\pgfpoint{54.320064\du}{14.535054\du}}
\pgfpathlineto{\pgfpoint{54.314952\du}{14.541626\du}}
\pgfpathlineto{\pgfpoint{54.308381\du}{14.548197\du}}
\pgfpathlineto{\pgfpoint{54.301444\du}{14.555134\du}}
\pgfpathlineto{\pgfpoint{54.295602\du}{14.561706\du}}
\pgfpathlineto{\pgfpoint{54.288300\du}{14.567913\du}}
\pgfpathlineto{\pgfpoint{54.281729\du}{14.574484\du}}
\pgfpathlineto{\pgfpoint{54.274062\du}{14.580691\du}}
\pgfpathlineto{\pgfpoint{54.266760\du}{14.587263\du}}
\pgfpathlineto{\pgfpoint{54.258727\du}{14.593834\du}}
\pgfpathlineto{\pgfpoint{54.251060\du}{14.600041\du}}
\pgfpathlineto{\pgfpoint{54.242298\du}{14.606613\du}}
\pgfpathlineto{\pgfpoint{54.233901\du}{14.612454\du}}
\pgfpathlineto{\pgfpoint{54.225139\du}{14.619026\du}}
\pgfpathlineto{\pgfpoint{54.217106\du}{14.625233\du}}
\pgfpathlineto{\pgfpoint{54.208344\du}{14.631074\du}}
\pgfpathlineto{\pgfpoint{54.198852\du}{14.637646\du}}
\pgfpathlineto{\pgfpoint{54.188629\du}{14.643488\du}}
\pgfpathlineto{\pgfpoint{54.179501\du}{14.650059\du}}
\pgfpathlineto{\pgfpoint{54.169279\du}{14.655901\du}}
\pgfpathlineto{\pgfpoint{54.159421\du}{14.661742\du}}
\pgfpathlineto{\pgfpoint{54.149564\du}{14.667584\du}}
\pgfpathlineto{\pgfpoint{54.138611\du}{14.673426\du}}
\pgfpathlineto{\pgfpoint{54.127658\du}{14.679267\du}}
\pgfpathlineto{\pgfpoint{54.117435\du}{14.685109\du}}
\pgfpathlineto{\pgfpoint{54.105387\du}{14.690950\du}}
\pgfpathlineto{\pgfpoint{54.094799\du}{14.696062\du}}
\pgfpathlineto{\pgfpoint{54.082751\du}{14.701903\du}}
\pgfpathlineto{\pgfpoint{54.071433\du}{14.707745\du}}
\pgfpathlineto{\pgfpoint{54.059020\du}{14.713221\du}}
\pgfpathlineto{\pgfpoint{54.046971\du}{14.718332\du}}
\pgfpathlineto{\pgfpoint{54.034558\du}{14.724174\du}}
\pgfpathlineto{\pgfpoint{54.022510\du}{14.729650\du}}
\pgfpathlineto{\pgfpoint{54.009366\du}{14.734762\du}}
\pgfpathlineto{\pgfpoint{53.996588\du}{14.739873\du}}
\pgfpathlineto{\pgfpoint{53.983810\du}{14.745350\du}}
\pgfpathlineto{\pgfpoint{53.970666\du}{14.750461\du}}
\pgfpathlineto{\pgfpoint{53.957158\du}{14.755937\du}}
\pgfpathlineto{\pgfpoint{53.944014\du}{14.760319\du}}
\pgfpathlineto{\pgfpoint{53.929775\du}{14.765795\du}}
\pgfpathlineto{\pgfpoint{53.915902\du}{14.770541\du}}
\pgfpathlineto{\pgfpoint{53.901663\du}{14.775653\du}}
\pgfpathlineto{\pgfpoint{53.887059\du}{14.780399\du}}
\pgfpathlineto{\pgfpoint{53.873185\du}{14.785145\du}}
\pgfpathlineto{\pgfpoint{53.858581\du}{14.789891\du}}
\pgfpathlineto{\pgfpoint{53.844343\du}{14.794273\du}}
\pgfpathlineto{\pgfpoint{53.828644\du}{14.799019\du}}
\pgfpathlineto{\pgfpoint{53.814405\du}{14.803765\du}}
\pgfpathlineto{\pgfpoint{53.798706\du}{14.808511\du}}
\pgfpathlineto{\pgfpoint{53.783737\du}{14.812527\du}}
\pgfpathlineto{\pgfpoint{53.752338\du}{14.821290\du}}
\pgfpathlineto{\pgfpoint{53.720210\du}{14.829687\du}}
\pgfpathlineto{\pgfpoint{53.688081\du}{14.837719\du}}
\pgfpathlineto{\pgfpoint{53.654857\du}{14.845751\du}}
\pgfpathlineto{\pgfpoint{53.620538\du}{14.853418\du}}
\pgfpathlineto{\pgfpoint{53.586584\du}{14.860720\du}}
\pgfpathlineto{\pgfpoint{53.551900\du}{14.867657\du}}
\pgfpathlineto{\pgfpoint{53.516121\du}{14.874594\du}}
\pgfpathlineto{\pgfpoint{53.480706\du}{14.881166\du}}
\pgfpathlineto{\pgfpoint{53.443832\du}{14.887372\du}}
\pgfpathlineto{\pgfpoint{53.407322\du}{14.893214\du}}
\pgfpathlineto{\pgfpoint{53.370082\du}{14.899055\du}}
\pgfpathlineto{\pgfpoint{53.332477\du}{14.904532\du}}
\pgfpathlineto{\pgfpoint{53.293777\du}{14.909278\du}}
\pgfpathlineto{\pgfpoint{53.255077\du}{14.913659\du}}
\pgfpathlineto{\pgfpoint{53.215646\du}{14.918405\du}}
\pgfpathlineto{\pgfpoint{53.176581\du}{14.922056\du}}
\pgfpathlineto{\pgfpoint{53.136420\du}{14.926072\du}}
\pgfpathlineto{\pgfpoint{53.096259\du}{14.928993\du}}
\pgfpathlineto{\pgfpoint{53.055004\du}{14.932644\du}}
\pgfpathlineto{\pgfpoint{53.014113\du}{14.935565\du}}
\pgfpathlineto{\pgfpoint{52.972857\du}{14.937756\du}}
\pgfpathlineto{\pgfpoint{52.931236\du}{14.939581\du}}
\pgfpathlineto{\pgfpoint{52.888519\du}{14.941407\du}}
\pgfpathlineto{\pgfpoint{52.845803\du}{14.942502\du}}
\pgfpathlineto{\pgfpoint{52.803817\du}{14.943597\du}}
\pgfpathlineto{\pgfpoint{52.760370\du}{14.943597\du}}
\pgfpathlineto{\pgfpoint{52.717289\du}{14.944327\du}}
\pgfpathlineto{\pgfpoint{52.717289\du}{14.944327\du}}
\pgfpathlineto{\pgfpoint{52.717289\du}{14.944327\du}}
\pgfpathlineto{\pgfpoint{52.716559\du}{14.944327\du}}
\pgfpathlineto{\pgfpoint{52.714733\du}{14.944327\du}}
\pgfpathlineto{\pgfpoint{52.713638\du}{14.944692\du}}
\pgfpathlineto{\pgfpoint{52.712908\du}{14.944692\du}}
\pgfpathlineto{\pgfpoint{52.712543\du}{14.945423\du}}
\pgfpathlineto{\pgfpoint{52.711082\du}{14.945788\du}}
\pgfpathlineto{\pgfpoint{52.710352\du}{14.946518\du}}
\pgfpathlineto{\pgfpoint{52.709622\du}{14.947248\du}}
\pgfpathlineto{\pgfpoint{52.708527\du}{14.949074\du}}
\pgfpathlineto{\pgfpoint{52.707796\du}{14.950534\du}}
\pgfpathlineto{\pgfpoint{52.707796\du}{14.952359\du}}
\pgfpathlineto{\pgfpoint{52.707066\du}{14.954185\du}}
\pgfpathlineto{\pgfpoint{52.707796\du}{14.956375\du}}
\pgfpathlineto{\pgfpoint{52.707796\du}{14.958201\du}}
\pgfpathlineto{\pgfpoint{52.708527\du}{14.960026\du}}
\pgfpathlineto{\pgfpoint{52.709622\du}{14.961852\du}}
\pgfpathlineto{\pgfpoint{52.710352\du}{14.962217\du}}
\pgfpathlineto{\pgfpoint{52.711082\du}{14.962947\du}}
\pgfpathlineto{\pgfpoint{52.712543\du}{14.963677\du}}
\pgfpathlineto{\pgfpoint{52.712908\du}{14.964043\du}}
\pgfpathlineto{\pgfpoint{52.713638\du}{14.964043\du}}
\pgfpathlineto{\pgfpoint{52.714733\du}{14.964773\du}}
\pgfpathlineto{\pgfpoint{52.716559\du}{14.964773\du}}
\pgfpathlineto{\pgfpoint{52.717289\du}{14.964773\du}}
\pgfusepath{fill}
\pgfsetbuttcap
\pgfsetmiterjoin
\pgfsetdash{}{0pt}
\definecolor{dialinecolor}{rgb}{0.678431, 0.839216, 0.905882}
\pgfsetfillcolor{dialinecolor}
\pgfpathmoveto{\pgfpoint{51.026526\du}{14.377332\du}}
\pgfpathlineto{\pgfpoint{51.026526\du}{14.377332\du}}
\pgfpathlineto{\pgfpoint{51.026526\du}{14.384999\du}}
\pgfpathlineto{\pgfpoint{51.026891\du}{14.392301\du}}
\pgfpathlineto{\pgfpoint{51.027621\du}{14.400698\du}}
\pgfpathlineto{\pgfpoint{51.028717\du}{14.408365\du}}
\pgfpathlineto{\pgfpoint{51.029812\du}{14.415667\du}}
\pgfpathlineto{\pgfpoint{51.031637\du}{14.423334\du}}
\pgfpathlineto{\pgfpoint{51.033463\du}{14.431731\du}}
\pgfpathlineto{\pgfpoint{51.035653\du}{14.439398\du}}
\pgfpathlineto{\pgfpoint{51.037844\du}{14.446700\du}}
\pgfpathlineto{\pgfpoint{51.040400\du}{14.454002\du}}
\pgfpathlineto{\pgfpoint{51.043320\du}{14.461304\du}}
\pgfpathlineto{\pgfpoint{51.046971\du}{14.468971\du}}
\pgfpathlineto{\pgfpoint{51.050257\du}{14.476638\du}}
\pgfpathlineto{\pgfpoint{51.053908\du}{14.484305\du}}
\pgfpathlineto{\pgfpoint{51.058289\du}{14.491242\du}}
\pgfpathlineto{\pgfpoint{51.061940\du}{14.498909\du}}
\pgfpathlineto{\pgfpoint{51.067052\du}{14.505846\du}}
\pgfpathlineto{\pgfpoint{51.071068\du}{14.513148\du}}
\pgfpathlineto{\pgfpoint{51.076544\du}{14.520450\du}}
\pgfpathlineto{\pgfpoint{51.081290\du}{14.527752\du}}
\pgfpathlineto{\pgfpoint{51.086767\du}{14.534689\du}}
\pgfpathlineto{\pgfpoint{51.092608\du}{14.541626\du}}
\pgfpathlineto{\pgfpoint{51.098450\du}{14.548562\du}}
\pgfpathlineto{\pgfpoint{51.104292\du}{14.555864\du}}
\pgfpathlineto{\pgfpoint{51.111228\du}{14.562801\du}}
\pgfpathlineto{\pgfpoint{51.117800\du}{14.569008\du}}
\pgfpathlineto{\pgfpoint{51.124737\du}{14.576310\du}}
\pgfpathlineto{\pgfpoint{51.131674\du}{14.583247\du}}
\pgfpathlineto{\pgfpoint{51.138611\du}{14.589453\du}}
\pgfpathlineto{\pgfpoint{51.147373\du}{14.596755\du}}
\pgfpathlineto{\pgfpoint{51.154310\du}{14.602962\du}}
\pgfpathlineto{\pgfpoint{51.161977\du}{14.609534\du}}
\pgfpathlineto{\pgfpoint{51.170739\du}{14.616470\du}}
\pgfpathlineto{\pgfpoint{51.179136\du}{14.623042\du}}
\pgfpathlineto{\pgfpoint{51.187899\du}{14.629249\du}}
\pgfpathlineto{\pgfpoint{51.196296\du}{14.635821\du}}
\pgfpathlineto{\pgfpoint{51.206154\du}{14.641662\du}}
\pgfpathlineto{\pgfpoint{51.215281\du}{14.648234\du}}
\pgfpathlineto{\pgfpoint{51.224773\du}{14.654440\du}}
\pgfpathlineto{\pgfpoint{51.234266\du}{14.661012\du}}
\pgfpathlineto{\pgfpoint{51.244124\du}{14.666854\du}}
\pgfpathlineto{\pgfpoint{51.254346\du}{14.673426\du}}
\pgfpathlineto{\pgfpoint{51.264569\du}{14.679267\du}}
\pgfpathlineto{\pgfpoint{51.275157\du}{14.685109\du}}
\pgfpathlineto{\pgfpoint{51.285380\du}{14.691315\du}}
\pgfpathlineto{\pgfpoint{51.296332\du}{14.697157\du}}
\pgfpathlineto{\pgfpoint{51.308016\du}{14.702998\du}}
\pgfpathlineto{\pgfpoint{51.319334\du}{14.708840\du}}
\pgfpathlineto{\pgfpoint{51.331017\du}{14.714681\du}}
\pgfpathlineto{\pgfpoint{51.342335\du}{14.720523\du}}
\pgfpathlineto{\pgfpoint{51.354018\du}{14.725999\du}}
\pgfpathlineto{\pgfpoint{51.366431\du}{14.731841\du}}
\pgfpathlineto{\pgfpoint{51.378479\du}{14.736952\du}}
\pgfpathlineto{\pgfpoint{51.391623\du}{14.742794\du}}
\pgfpathlineto{\pgfpoint{51.403671\du}{14.748270\du}}
\pgfpathlineto{\pgfpoint{51.416449\du}{14.753382\du}}
\pgfpathlineto{\pgfpoint{51.430323\du}{14.759223\du}}
\pgfpathlineto{\pgfpoint{51.442736\du}{14.764700\du}}
\pgfpathlineto{\pgfpoint{51.455880\du}{14.769811\du}}
\pgfpathlineto{\pgfpoint{51.470119\du}{14.775288\du}}
\pgfpathlineto{\pgfpoint{51.483262\du}{14.779669\du}}
\pgfpathlineto{\pgfpoint{51.497501\du}{14.785145\du}}
\pgfpathlineto{\pgfpoint{51.511374\du}{14.789891\du}}
\pgfpathlineto{\pgfpoint{51.525978\du}{14.795003\du}}
\pgfpathlineto{\pgfpoint{51.539852\du}{14.799749\du}}
\pgfpathlineto{\pgfpoint{51.555551\du}{14.804495\du}}
\pgfpathlineto{\pgfpoint{51.569425\du}{14.809607\du}}
\pgfpathlineto{\pgfpoint{51.584029\du}{14.814353\du}}
\pgfpathlineto{\pgfpoint{51.599363\du}{14.819099\du}}
\pgfpathlineto{\pgfpoint{51.615062\du}{14.823845\du}}
\pgfpathlineto{\pgfpoint{51.630031\du}{14.827861\du}}
\pgfpathlineto{\pgfpoint{51.645730\du}{14.832608\du}}
\pgfpathlineto{\pgfpoint{51.677493\du}{14.841370\du}}
\pgfpathlineto{\pgfpoint{51.709257\du}{14.849402\du}}
\pgfpathlineto{\pgfpoint{51.742481\du}{14.857434\du}}
\pgfpathlineto{\pgfpoint{51.774974\du}{14.865832\du}}
\pgfpathlineto{\pgfpoint{51.809658\du}{14.873499\du}}
\pgfpathlineto{\pgfpoint{51.843978\du}{14.881166\du}}
\pgfpathlineto{\pgfpoint{51.878662\du}{14.888102\du}}
\pgfpathlineto{\pgfpoint{51.914441\du}{14.895039\du}}
\pgfpathlineto{\pgfpoint{51.950586\du}{14.901611\du}}
\pgfpathlineto{\pgfpoint{51.987096\du}{14.907453\du}}
\pgfpathlineto{\pgfpoint{52.024335\du}{14.913659\du}}
\pgfpathlineto{\pgfpoint{52.061940\du}{14.919136\du}}
\pgfpathlineto{\pgfpoint{52.099910\du}{14.924247\du}}
\pgfpathlineto{\pgfpoint{52.138611\du}{14.929723\du}}
\pgfpathlineto{\pgfpoint{52.177311\du}{14.934105\du}}
\pgfpathlineto{\pgfpoint{52.216376\du}{14.938486\du}}
\pgfpathlineto{\pgfpoint{52.256537\du}{14.942502\du}}
\pgfpathlineto{\pgfpoint{52.296332\du}{14.946518\du}}
\pgfpathlineto{\pgfpoint{52.337223\du}{14.950169\du}}
\pgfpathlineto{\pgfpoint{52.378114\du}{14.953090\du}}
\pgfpathlineto{\pgfpoint{52.419370\du}{14.955280\du}}
\pgfpathlineto{\pgfpoint{52.461356\du}{14.957836\du}}
\pgfpathlineto{\pgfpoint{52.502977\du}{14.960026\du}}
\pgfpathlineto{\pgfpoint{52.545328\du}{14.961852\du}}
\pgfpathlineto{\pgfpoint{52.587315\du}{14.962947\du}}
\pgfpathlineto{\pgfpoint{52.630761\du}{14.964043\du}}
\pgfpathlineto{\pgfpoint{52.673477\du}{14.964773\du}}
\pgfpathlineto{\pgfpoint{52.717289\du}{14.964773\du}}
\pgfpathlineto{\pgfpoint{52.717289\du}{14.944327\du}}
\pgfpathlineto{\pgfpoint{52.674573\du}{14.943597\du}}
\pgfpathlineto{\pgfpoint{52.631126\du}{14.943597\du}}
\pgfpathlineto{\pgfpoint{52.588775\du}{14.942502\du}}
\pgfpathlineto{\pgfpoint{52.546059\du}{14.941407\du}}
\pgfpathlineto{\pgfpoint{52.503707\du}{14.939581\du}}
\pgfpathlineto{\pgfpoint{52.461721\du}{14.937756\du}}
\pgfpathlineto{\pgfpoint{52.420830\du}{14.935565\du}}
\pgfpathlineto{\pgfpoint{52.379575\du}{14.932644\du}}
\pgfpathlineto{\pgfpoint{52.338684\du}{14.928993\du}}
\pgfpathlineto{\pgfpoint{52.298523\du}{14.926072\du}}
\pgfpathlineto{\pgfpoint{52.258727\du}{14.922056\du}}
\pgfpathlineto{\pgfpoint{52.219297\du}{14.918405\du}}
\pgfpathlineto{\pgfpoint{52.179867\du}{14.913659\du}}
\pgfpathlineto{\pgfpoint{52.140801\du}{14.909278\du}}
\pgfpathlineto{\pgfpoint{52.102466\du}{14.904532\du}}
\pgfpathlineto{\pgfpoint{52.064861\du}{14.899055\du}}
\pgfpathlineto{\pgfpoint{52.027621\du}{14.893214\du}}
\pgfpathlineto{\pgfpoint{51.991112\du}{14.887372\du}}
\pgfpathlineto{\pgfpoint{51.953872\du}{14.881166\du}}
\pgfpathlineto{\pgfpoint{51.918822\du}{14.874594\du}}
\pgfpathlineto{\pgfpoint{51.882678\du}{14.867657\du}}
\pgfpathlineto{\pgfpoint{51.847994\du}{14.860720\du}}
\pgfpathlineto{\pgfpoint{51.813675\du}{14.853418\du}}
\pgfpathlineto{\pgfpoint{51.779721\du}{14.845751\du}}
\pgfpathlineto{\pgfpoint{51.746497\du}{14.837719\du}}
\pgfpathlineto{\pgfpoint{51.715098\du}{14.829687\du}}
\pgfpathlineto{\pgfpoint{51.682605\du}{14.821290\du}}
\pgfpathlineto{\pgfpoint{51.651572\du}{14.812527\du}}
\pgfpathlineto{\pgfpoint{51.635872\du}{14.808511\du}}
\pgfpathlineto{\pgfpoint{51.620173\du}{14.803765\du}}
\pgfpathlineto{\pgfpoint{51.605569\du}{14.799019\du}}
\pgfpathlineto{\pgfpoint{51.590600\du}{14.794273\du}}
\pgfpathlineto{\pgfpoint{51.575266\du}{14.789891\du}}
\pgfpathlineto{\pgfpoint{51.561028\du}{14.785145\du}}
\pgfpathlineto{\pgfpoint{51.546789\du}{14.780399\du}}
\pgfpathlineto{\pgfpoint{51.532550\du}{14.775653\du}}
\pgfpathlineto{\pgfpoint{51.518311\du}{14.770541\du}}
\pgfpathlineto{\pgfpoint{51.504438\du}{14.765795\du}}
\pgfpathlineto{\pgfpoint{51.490564\du}{14.760319\du}}
\pgfpathlineto{\pgfpoint{51.477055\du}{14.755937\du}}
\pgfpathlineto{\pgfpoint{51.463547\du}{14.750461\du}}
\pgfpathlineto{\pgfpoint{51.450768\du}{14.745350\du}}
\pgfpathlineto{\pgfpoint{51.437260\du}{14.739873\du}}
\pgfpathlineto{\pgfpoint{51.424481\du}{14.734762\du}}
\pgfpathlineto{\pgfpoint{51.412068\du}{14.729650\du}}
\pgfpathlineto{\pgfpoint{51.398925\du}{14.724174\du}}
\pgfpathlineto{\pgfpoint{51.386876\du}{14.718332\du}}
\pgfpathlineto{\pgfpoint{51.375558\du}{14.713221\du}}
\pgfpathlineto{\pgfpoint{51.362780\du}{14.707745\du}}
\pgfpathlineto{\pgfpoint{51.351097\du}{14.701903\du}}
\pgfpathlineto{\pgfpoint{51.339414\du}{14.696062\du}}
\pgfpathlineto{\pgfpoint{51.328461\du}{14.690950\du}}
\pgfpathlineto{\pgfpoint{51.316778\du}{14.685109\du}}
\pgfpathlineto{\pgfpoint{51.306920\du}{14.679267\du}}
\pgfpathlineto{\pgfpoint{51.295602\du}{14.673426\du}}
\pgfpathlineto{\pgfpoint{51.285014\du}{14.667584\du}}
\pgfpathlineto{\pgfpoint{51.275157\du}{14.661742\du}}
\pgfpathlineto{\pgfpoint{51.264569\du}{14.655901\du}}
\pgfpathlineto{\pgfpoint{51.254711\du}{14.650059\du}}
\pgfpathlineto{\pgfpoint{51.245219\du}{14.643488\du}}
\pgfpathlineto{\pgfpoint{51.235361\du}{14.637646\du}}
\pgfpathlineto{\pgfpoint{51.226234\du}{14.631074\du}}
\pgfpathlineto{\pgfpoint{51.217837\du}{14.625233\du}}
\pgfpathlineto{\pgfpoint{51.209074\du}{14.619026\du}}
\pgfpathlineto{\pgfpoint{51.199947\du}{14.612454\du}}
\pgfpathlineto{\pgfpoint{51.191185\du}{14.606613\du}}
\pgfpathlineto{\pgfpoint{51.182787\du}{14.600041\du}}
\pgfpathlineto{\pgfpoint{51.175485\du}{14.593834\du}}
\pgfpathlineto{\pgfpoint{51.167453\du}{14.587263\du}}
\pgfpathlineto{\pgfpoint{51.159786\du}{14.580691\du}}
\pgfpathlineto{\pgfpoint{51.152849\du}{14.574484\du}}
\pgfpathlineto{\pgfpoint{51.145547\du}{14.567913\du}}
\pgfpathlineto{\pgfpoint{51.138611\du}{14.561706\du}}
\pgfpathlineto{\pgfpoint{51.132404\du}{14.555134\du}}
\pgfpathlineto{\pgfpoint{51.125832\du}{14.548197\du}}
\pgfpathlineto{\pgfpoint{51.119991\du}{14.541626\du}}
\pgfpathlineto{\pgfpoint{51.113784\du}{14.535054\du}}
\pgfpathlineto{\pgfpoint{51.108673\du}{14.528847\du}}
\pgfpathlineto{\pgfpoint{51.102466\du}{14.521910\du}}
\pgfpathlineto{\pgfpoint{51.098085\du}{14.515339\du}}
\pgfpathlineto{\pgfpoint{51.092608\du}{14.508402\du}}
\pgfpathlineto{\pgfpoint{51.088227\du}{14.501830\du}}
\pgfpathlineto{\pgfpoint{51.083846\du}{14.494893\du}}
\pgfpathlineto{\pgfpoint{51.079465\du}{14.487956\du}}
\pgfpathlineto{\pgfpoint{51.076179\du}{14.481385\du}}
\pgfpathlineto{\pgfpoint{51.072163\du}{14.474448\du}}
\pgfpathlineto{\pgfpoint{51.068147\du}{14.467876\du}}
\pgfpathlineto{\pgfpoint{51.065226\du}{14.460939\du}}
\pgfpathlineto{\pgfpoint{51.062671\du}{14.454002\du}}
\pgfpathlineto{\pgfpoint{51.059385\du}{14.446700\du}}
\pgfpathlineto{\pgfpoint{51.057194\du}{14.439764\du}}
\pgfpathlineto{\pgfpoint{51.055004\du}{14.433557\du}}
\pgfpathlineto{\pgfpoint{51.053543\du}{14.426255\du}}
\pgfpathlineto{\pgfpoint{51.051353\du}{14.419318\du}}
\pgfpathlineto{\pgfpoint{51.050257\du}{14.412381\du}}
\pgfpathlineto{\pgfpoint{51.049162\du}{14.405444\du}}
\pgfpathlineto{\pgfpoint{51.047702\du}{14.398143\du}}
\pgfpathlineto{\pgfpoint{51.047336\du}{14.391206\du}}
\pgfpathlineto{\pgfpoint{51.047336\du}{14.384269\du}}
\pgfpathlineto{\pgfpoint{51.046971\du}{14.377332\du}}
\pgfpathlineto{\pgfpoint{51.046971\du}{14.377332\du}}
\pgfpathlineto{\pgfpoint{51.046971\du}{14.377332\du}}
\pgfpathlineto{\pgfpoint{51.046971\du}{14.375507\du}}
\pgfpathlineto{\pgfpoint{51.046971\du}{14.374776\du}}
\pgfpathlineto{\pgfpoint{51.046606\du}{14.373681\du}}
\pgfpathlineto{\pgfpoint{51.046606\du}{14.372586\du}}
\pgfpathlineto{\pgfpoint{51.045511\du}{14.371856\du}}
\pgfpathlineto{\pgfpoint{51.045146\du}{14.370760\du}}
\pgfpathlineto{\pgfpoint{51.044781\du}{14.370030\du}}
\pgfpathlineto{\pgfpoint{51.043686\du}{14.369665\du}}
\pgfpathlineto{\pgfpoint{51.042225\du}{14.368570\du}}
\pgfpathlineto{\pgfpoint{51.040400\du}{14.367109\du}}
\pgfpathlineto{\pgfpoint{51.038574\du}{14.366744\du}}
\pgfpathlineto{\pgfpoint{51.036749\du}{14.366744\du}}
\pgfpathlineto{\pgfpoint{51.034558\du}{14.366744\du}}
\pgfpathlineto{\pgfpoint{51.033098\du}{14.367109\du}}
\pgfpathlineto{\pgfpoint{51.030907\du}{14.368570\du}}
\pgfpathlineto{\pgfpoint{51.029082\du}{14.369665\du}}
\pgfpathlineto{\pgfpoint{51.028717\du}{14.370030\du}}
\pgfpathlineto{\pgfpoint{51.028351\du}{14.370760\du}}
\pgfpathlineto{\pgfpoint{51.027621\du}{14.371856\du}}
\pgfpathlineto{\pgfpoint{51.026891\du}{14.372586\du}}
\pgfpathlineto{\pgfpoint{51.026891\du}{14.373681\du}}
\pgfpathlineto{\pgfpoint{51.026526\du}{14.374776\du}}
\pgfpathlineto{\pgfpoint{51.026526\du}{14.375507\du}}
\pgfpathlineto{\pgfpoint{51.026526\du}{14.377332\du}}
\pgfusepath{fill}
\pgfsetbuttcap
\pgfsetmiterjoin
\pgfsetdash{}{0pt}
\definecolor{dialinecolor}{rgb}{0.678431, 0.839216, 0.905882}
\pgfsetfillcolor{dialinecolor}
\pgfpathmoveto{\pgfpoint{52.717289\du}{13.789891\du}}
\pgfpathlineto{\pgfpoint{52.717289\du}{13.789891\du}}
\pgfpathlineto{\pgfpoint{52.673477\du}{13.789891\du}}
\pgfpathlineto{\pgfpoint{52.630761\du}{13.790256\du}}
\pgfpathlineto{\pgfpoint{52.587315\du}{13.791352\du}}
\pgfpathlineto{\pgfpoint{52.545328\du}{13.792812\du}}
\pgfpathlineto{\pgfpoint{52.502977\du}{13.794273\du}}
\pgfpathlineto{\pgfpoint{52.461356\du}{13.796098\du}}
\pgfpathlineto{\pgfpoint{52.419370\du}{13.798654\du}}
\pgfpathlineto{\pgfpoint{52.378114\du}{13.801574\du}}
\pgfpathlineto{\pgfpoint{52.337223\du}{13.804495\du}}
\pgfpathlineto{\pgfpoint{52.296332\du}{13.807416\du}}
\pgfpathlineto{\pgfpoint{52.256537\du}{13.811432\du}}
\pgfpathlineto{\pgfpoint{52.216376\du}{13.815448\du}}
\pgfpathlineto{\pgfpoint{52.177311\du}{13.820194\du}}
\pgfpathlineto{\pgfpoint{52.138611\du}{13.824941\du}}
\pgfpathlineto{\pgfpoint{52.099910\du}{13.829687\du}}
\pgfpathlineto{\pgfpoint{52.061940\du}{13.834798\du}}
\pgfpathlineto{\pgfpoint{52.024335\du}{13.840640\du}}
\pgfpathlineto{\pgfpoint{51.987096\du}{13.846481\du}}
\pgfpathlineto{\pgfpoint{51.950586\du}{13.853053\du}}
\pgfpathlineto{\pgfpoint{51.914441\du}{13.859260\du}}
\pgfpathlineto{\pgfpoint{51.878662\du}{13.866562\du}}
\pgfpathlineto{\pgfpoint{51.843978\du}{13.873499\du}}
\pgfpathlineto{\pgfpoint{51.809658\du}{13.880435\du}}
\pgfpathlineto{\pgfpoint{51.774974\du}{13.888468\du}}
\pgfpathlineto{\pgfpoint{51.742481\du}{13.896135\du}}
\pgfpathlineto{\pgfpoint{51.709257\du}{13.904532\du}}
\pgfpathlineto{\pgfpoint{51.677493\du}{13.913294\du}}
\pgfpathlineto{\pgfpoint{51.645730\du}{13.922056\du}}
\pgfpathlineto{\pgfpoint{51.615062\du}{13.930819\du}}
\pgfpathlineto{\pgfpoint{51.584029\du}{13.940311\du}}
\pgfpathlineto{\pgfpoint{51.569425\du}{13.944692\du}}
\pgfpathlineto{\pgfpoint{51.555551\du}{13.949439\du}}
\pgfpathlineto{\pgfpoint{51.539852\du}{13.954185\du}}
\pgfpathlineto{\pgfpoint{51.525978\du}{13.959296\du}}
\pgfpathlineto{\pgfpoint{51.511374\du}{13.964043\du}}
\pgfpathlineto{\pgfpoint{51.497501\du}{13.969519\du}}
\pgfpathlineto{\pgfpoint{51.483262\du}{13.973900\du}}
\pgfpathlineto{\pgfpoint{51.470119\du}{13.979377\du}}
\pgfpathlineto{\pgfpoint{51.455880\du}{13.984488\du}}
\pgfpathlineto{\pgfpoint{51.442736\du}{13.989964\du}}
\pgfpathlineto{\pgfpoint{51.430323\du}{13.995076\du}}
\pgfpathlineto{\pgfpoint{51.416449\du}{14.000552\du}}
\pgfpathlineto{\pgfpoint{51.403671\du}{14.005664\du}}
\pgfpathlineto{\pgfpoint{51.391623\du}{14.010775\du}}
\pgfpathlineto{\pgfpoint{51.378479\du}{14.016616\du}}
\pgfpathlineto{\pgfpoint{51.366431\du}{14.022823\du}}
\pgfpathlineto{\pgfpoint{51.354018\du}{14.027934\du}}
\pgfpathlineto{\pgfpoint{51.342335\du}{14.033776\du}}
\pgfpathlineto{\pgfpoint{51.331017\du}{14.039618\du}}
\pgfpathlineto{\pgfpoint{51.319334\du}{14.045459\du}}
\pgfpathlineto{\pgfpoint{51.308016\du}{14.051301\du}}
\pgfpathlineto{\pgfpoint{51.296332\du}{14.056412\du}}
\pgfpathlineto{\pgfpoint{51.285380\du}{14.062984\du}}
\pgfpathlineto{\pgfpoint{51.275157\du}{14.068825\du}}
\pgfpathlineto{\pgfpoint{51.264569\du}{14.074667\du}}
\pgfpathlineto{\pgfpoint{51.254346\du}{14.080508\du}}
\pgfpathlineto{\pgfpoint{51.244124\du}{14.087080\du}}
\pgfpathlineto{\pgfpoint{51.234266\du}{14.093287\du}}
\pgfpathlineto{\pgfpoint{51.224773\du}{14.099128\du}}
\pgfpathlineto{\pgfpoint{51.215281\du}{14.105700\du}}
\pgfpathlineto{\pgfpoint{51.206154\du}{14.112272\du}}
\pgfpathlineto{\pgfpoint{51.196296\du}{14.118478\du}}
\pgfpathlineto{\pgfpoint{51.187899\du}{14.125050\du}}
\pgfpathlineto{\pgfpoint{51.179136\du}{14.131622\du}}
\pgfpathlineto{\pgfpoint{51.170739\du}{14.137829\du}}
\pgfpathlineto{\pgfpoint{51.161977\du}{14.144400\du}}
\pgfpathlineto{\pgfpoint{51.154310\du}{14.151337\du}}
\pgfpathlineto{\pgfpoint{51.147373\du}{14.157909\du}}
\pgfpathlineto{\pgfpoint{51.138611\du}{14.164116\du}}
\pgfpathlineto{\pgfpoint{51.131674\du}{14.171417\du}}
\pgfpathlineto{\pgfpoint{51.124737\du}{14.177624\du}}
\pgfpathlineto{\pgfpoint{51.117800\du}{14.184561\du}}
\pgfpathlineto{\pgfpoint{51.111228\du}{14.191863\du}}
\pgfpathlineto{\pgfpoint{51.104292\du}{14.198070\du}}
\pgfpathlineto{\pgfpoint{51.098450\du}{14.205371\du}}
\pgfpathlineto{\pgfpoint{51.092608\du}{14.212308\du}}
\pgfpathlineto{\pgfpoint{51.086767\du}{14.219975\du}}
\pgfpathlineto{\pgfpoint{51.081290\du}{14.226912\du}}
\pgfpathlineto{\pgfpoint{51.076544\du}{14.233849\du}}
\pgfpathlineto{\pgfpoint{51.071068\du}{14.240786\du}}
\pgfpathlineto{\pgfpoint{51.067052\du}{14.248088\du}}
\pgfpathlineto{\pgfpoint{51.061940\du}{14.255390\du}}
\pgfpathlineto{\pgfpoint{51.058289\du}{14.262692\du}}
\pgfpathlineto{\pgfpoint{51.053908\du}{14.269994\du}}
\pgfpathlineto{\pgfpoint{51.050257\du}{14.277661\du}}
\pgfpathlineto{\pgfpoint{51.046971\du}{14.285328\du}}
\pgfpathlineto{\pgfpoint{51.043320\du}{14.292265\du}}
\pgfpathlineto{\pgfpoint{51.040400\du}{14.299932\du}}
\pgfpathlineto{\pgfpoint{51.037844\du}{14.307599\du}}
\pgfpathlineto{\pgfpoint{51.035653\du}{14.315266\du}}
\pgfpathlineto{\pgfpoint{51.033463\du}{14.322933\du}}
\pgfpathlineto{\pgfpoint{51.031637\du}{14.330235\du}}
\pgfpathlineto{\pgfpoint{51.029812\du}{14.337902\du}}
\pgfpathlineto{\pgfpoint{51.028717\du}{14.345569\du}}
\pgfpathlineto{\pgfpoint{51.027621\du}{14.353966\du}}
\pgfpathlineto{\pgfpoint{51.026891\du}{14.361268\du}}
\pgfpathlineto{\pgfpoint{51.026526\du}{14.368935\du}}
\pgfpathlineto{\pgfpoint{51.026526\du}{14.377332\du}}
\pgfpathlineto{\pgfpoint{51.046971\du}{14.377332\du}}
\pgfpathlineto{\pgfpoint{51.047336\du}{14.370030\du}}
\pgfpathlineto{\pgfpoint{51.047336\du}{14.363093\du}}
\pgfpathlineto{\pgfpoint{51.047702\du}{14.356156\du}}
\pgfpathlineto{\pgfpoint{51.049162\du}{14.348489\du}}
\pgfpathlineto{\pgfpoint{51.050257\du}{14.342283\du}}
\pgfpathlineto{\pgfpoint{51.051353\du}{14.334981\du}}
\pgfpathlineto{\pgfpoint{51.053543\du}{14.328044\du}}
\pgfpathlineto{\pgfpoint{51.055004\du}{14.321107\du}}
\pgfpathlineto{\pgfpoint{51.057194\du}{14.314170\du}}
\pgfpathlineto{\pgfpoint{51.059385\du}{14.306868\du}}
\pgfpathlineto{\pgfpoint{51.062671\du}{14.300662\du}}
\pgfpathlineto{\pgfpoint{51.065226\du}{14.293725\du}}
\pgfpathlineto{\pgfpoint{51.068147\du}{14.286423\du}}
\pgfpathlineto{\pgfpoint{51.072163\du}{14.279486\du}}
\pgfpathlineto{\pgfpoint{51.076179\du}{14.272549\du}}
\pgfpathlineto{\pgfpoint{51.079465\du}{14.265978\du}}
\pgfpathlineto{\pgfpoint{51.083481\du}{14.259041\du}}
\pgfpathlineto{\pgfpoint{51.088227\du}{14.252469\du}}
\pgfpathlineto{\pgfpoint{51.092608\du}{14.245532\du}}
\pgfpathlineto{\pgfpoint{51.098085\du}{14.238960\du}}
\pgfpathlineto{\pgfpoint{51.102466\du}{14.232754\du}}
\pgfpathlineto{\pgfpoint{51.108673\du}{14.225817\du}}
\pgfpathlineto{\pgfpoint{51.113784\du}{14.219245\du}}
\pgfpathlineto{\pgfpoint{51.119991\du}{14.212308\du}}
\pgfpathlineto{\pgfpoint{51.125832\du}{14.205737\du}}
\pgfpathlineto{\pgfpoint{51.132404\du}{14.199530\du}}
\pgfpathlineto{\pgfpoint{51.138611\du}{14.192958\du}}
\pgfpathlineto{\pgfpoint{51.145547\du}{14.186021\du}}
\pgfpathlineto{\pgfpoint{51.152849\du}{14.180180\du}}
\pgfpathlineto{\pgfpoint{51.159786\du}{14.172878\du}}
\pgfpathlineto{\pgfpoint{51.167453\du}{14.166671\du}}
\pgfpathlineto{\pgfpoint{51.175485\du}{14.160830\du}}
\pgfpathlineto{\pgfpoint{51.182787\du}{14.154258\du}}
\pgfpathlineto{\pgfpoint{51.191185\du}{14.147686\du}}
\pgfpathlineto{\pgfpoint{51.199947\du}{14.141480\du}}
\pgfpathlineto{\pgfpoint{51.209074\du}{14.134908\du}}
\pgfpathlineto{\pgfpoint{51.217837\du}{14.129066\du}}
\pgfpathlineto{\pgfpoint{51.226234\du}{14.122860\du}}
\pgfpathlineto{\pgfpoint{51.235361\du}{14.117018\du}}
\pgfpathlineto{\pgfpoint{51.245219\du}{14.110446\du}}
\pgfpathlineto{\pgfpoint{51.254711\du}{14.104605\du}}
\pgfpathlineto{\pgfpoint{51.264569\du}{14.098763\du}}
\pgfpathlineto{\pgfpoint{51.275157\du}{14.092922\du}}
\pgfpathlineto{\pgfpoint{51.285014\du}{14.087080\du}}
\pgfpathlineto{\pgfpoint{51.295602\du}{14.080508\du}}
\pgfpathlineto{\pgfpoint{51.306920\du}{14.075397\du}}
\pgfpathlineto{\pgfpoint{51.316778\du}{14.069555\du}}
\pgfpathlineto{\pgfpoint{51.328461\du}{14.063714\du}}
\pgfpathlineto{\pgfpoint{51.339414\du}{14.057872\du}}
\pgfpathlineto{\pgfpoint{51.351097\du}{14.052031\du}}
\pgfpathlineto{\pgfpoint{51.362780\du}{14.046554\du}}
\pgfpathlineto{\pgfpoint{51.375558\du}{14.041443\du}}
\pgfpathlineto{\pgfpoint{51.386876\du}{14.035601\du}}
\pgfpathlineto{\pgfpoint{51.398925\du}{14.030125\du}}
\pgfpathlineto{\pgfpoint{51.412068\du}{14.025014\du}}
\pgfpathlineto{\pgfpoint{51.424481\du}{14.019537\du}}
\pgfpathlineto{\pgfpoint{51.437260\du}{14.014426\du}}
\pgfpathlineto{\pgfpoint{51.450768\du}{14.008584\du}}
\pgfpathlineto{\pgfpoint{51.463547\du}{14.003838\du}}
\pgfpathlineto{\pgfpoint{51.477055\du}{13.998727\du}}
\pgfpathlineto{\pgfpoint{51.490564\du}{13.993250\du}}
\pgfpathlineto{\pgfpoint{51.504438\du}{13.988869\du}}
\pgfpathlineto{\pgfpoint{51.518311\du}{13.983393\du}}
\pgfpathlineto{\pgfpoint{51.532550\du}{13.978646\du}}
\pgfpathlineto{\pgfpoint{51.546789\du}{13.973900\du}}
\pgfpathlineto{\pgfpoint{51.561028\du}{13.968789\du}}
\pgfpathlineto{\pgfpoint{51.575266\du}{13.964043\du}}
\pgfpathlineto{\pgfpoint{51.590600\du}{13.959296\du}}
\pgfpathlineto{\pgfpoint{51.620173\du}{13.950534\du}}
\pgfpathlineto{\pgfpoint{51.651572\du}{13.941407\du}}
\pgfpathlineto{\pgfpoint{51.682605\du}{13.933009\du}}
\pgfpathlineto{\pgfpoint{51.715098\du}{13.924247\du}}
\pgfpathlineto{\pgfpoint{51.746497\du}{13.916215\du}}
\pgfpathlineto{\pgfpoint{51.779721\du}{13.908548\du}}
\pgfpathlineto{\pgfpoint{51.813675\du}{13.900881\du}}
\pgfpathlineto{\pgfpoint{51.847994\du}{13.893214\du}}
\pgfpathlineto{\pgfpoint{51.882678\du}{13.886277\du}}
\pgfpathlineto{\pgfpoint{51.918822\du}{13.879340\du}}
\pgfpathlineto{\pgfpoint{51.953872\du}{13.873499\du}}
\pgfpathlineto{\pgfpoint{51.991112\du}{13.866927\du}}
\pgfpathlineto{\pgfpoint{52.027621\du}{13.861085\du}}
\pgfpathlineto{\pgfpoint{52.064861\du}{13.855244\du}}
\pgfpathlineto{\pgfpoint{52.102466\du}{13.850132\du}}
\pgfpathlineto{\pgfpoint{52.140801\du}{13.844656\du}}
\pgfpathlineto{\pgfpoint{52.179867\du}{13.839910\du}}
\pgfpathlineto{\pgfpoint{52.219297\du}{13.835894\du}}
\pgfpathlineto{\pgfpoint{52.258727\du}{13.831878\du}}
\pgfpathlineto{\pgfpoint{52.298523\du}{13.828227\du}}
\pgfpathlineto{\pgfpoint{52.338684\du}{13.824941\du}}
\pgfpathlineto{\pgfpoint{52.379575\du}{13.822020\du}}
\pgfpathlineto{\pgfpoint{52.420830\du}{13.819099\du}}
\pgfpathlineto{\pgfpoint{52.461721\du}{13.816543\du}}
\pgfpathlineto{\pgfpoint{52.503707\du}{13.814353\du}}
\pgfpathlineto{\pgfpoint{52.546059\du}{13.813258\du}}
\pgfpathlineto{\pgfpoint{52.588775\du}{13.811432\du}}
\pgfpathlineto{\pgfpoint{52.631126\du}{13.810702\du}}
\pgfpathlineto{\pgfpoint{52.674573\du}{13.810337\du}}
\pgfpathlineto{\pgfpoint{52.717289\du}{13.810337\du}}
\pgfpathlineto{\pgfpoint{52.717289\du}{13.810337\du}}
\pgfpathlineto{\pgfpoint{52.717289\du}{13.810337\du}}
\pgfpathlineto{\pgfpoint{52.718384\du}{13.809607\du}}
\pgfpathlineto{\pgfpoint{52.719480\du}{13.809607\du}}
\pgfpathlineto{\pgfpoint{52.720940\du}{13.809607\du}}
\pgfpathlineto{\pgfpoint{52.722035\du}{13.809242\du}}
\pgfpathlineto{\pgfpoint{52.722400\du}{13.808511\du}}
\pgfpathlineto{\pgfpoint{52.723496\du}{13.808511\du}}
\pgfpathlineto{\pgfpoint{52.724226\du}{13.807416\du}}
\pgfpathlineto{\pgfpoint{52.725321\du}{13.806686\du}}
\pgfpathlineto{\pgfpoint{52.726416\du}{13.805591\du}}
\pgfpathlineto{\pgfpoint{52.727147\du}{13.803765\du}}
\pgfpathlineto{\pgfpoint{52.727147\du}{13.801574\du}}
\pgfpathlineto{\pgfpoint{52.727877\du}{13.799749\du}}
\pgfpathlineto{\pgfpoint{52.727147\du}{13.797924\du}}
\pgfpathlineto{\pgfpoint{52.727147\du}{13.796098\du}}
\pgfpathlineto{\pgfpoint{52.726416\du}{13.794273\du}}
\pgfpathlineto{\pgfpoint{52.725321\du}{13.792812\du}}
\pgfpathlineto{\pgfpoint{52.724226\du}{13.792082\du}}
\pgfpathlineto{\pgfpoint{52.723496\du}{13.791352\du}}
\pgfpathlineto{\pgfpoint{52.722400\du}{13.790987\du}}
\pgfpathlineto{\pgfpoint{52.722035\du}{13.790256\du}}
\pgfpathlineto{\pgfpoint{52.720940\du}{13.789891\du}}
\pgfpathlineto{\pgfpoint{52.719480\du}{13.789891\du}}
\pgfpathlineto{\pgfpoint{52.718384\du}{13.789891\du}}
\pgfpathlineto{\pgfpoint{52.717289\du}{13.789891\du}}
\pgfusepath{fill}
\pgfsetbuttcap
\pgfsetmiterjoin
\pgfsetdash{}{0pt}
\definecolor{dialinecolor}{rgb}{0.678431, 0.839216, 0.905882}
\pgfsetfillcolor{dialinecolor}
\pgfpathmoveto{\pgfpoint{54.407687\du}{14.377332\du}}
\pgfpathlineto{\pgfpoint{54.407687\du}{14.368935\du}}
\pgfpathlineto{\pgfpoint{54.406957\du}{14.361268\du}}
\pgfpathlineto{\pgfpoint{54.406227\du}{14.353966\du}}
\pgfpathlineto{\pgfpoint{54.405496\du}{14.345569\du}}
\pgfpathlineto{\pgfpoint{54.404401\du}{14.337902\du}}
\pgfpathlineto{\pgfpoint{54.402211\du}{14.330235\du}}
\pgfpathlineto{\pgfpoint{54.401115\du}{14.322933\du}}
\pgfpathlineto{\pgfpoint{54.398194\du}{14.315266\du}}
\pgfpathlineto{\pgfpoint{54.396369\du}{14.307599\du}}
\pgfpathlineto{\pgfpoint{54.393448\du}{14.299932\du}}
\pgfpathlineto{\pgfpoint{54.390893\du}{14.292265\du}}
\pgfpathlineto{\pgfpoint{54.387607\du}{14.285328\du}}
\pgfpathlineto{\pgfpoint{54.383591\du}{14.277661\du}}
\pgfpathlineto{\pgfpoint{54.380305\du}{14.269994\du}}
\pgfpathlineto{\pgfpoint{54.376289\du}{14.262692\du}}
\pgfpathlineto{\pgfpoint{54.372273\du}{14.255390\du}}
\pgfpathlineto{\pgfpoint{54.367891\du}{14.248088\du}}
\pgfpathlineto{\pgfpoint{54.363145\du}{14.240786\du}}
\pgfpathlineto{\pgfpoint{54.357669\du}{14.233849\du}}
\pgfpathlineto{\pgfpoint{54.352922\du}{14.226912\du}}
\pgfpathlineto{\pgfpoint{54.347081\du}{14.219245\du}}
\pgfpathlineto{\pgfpoint{54.341604\du}{14.212308\du}}
\pgfpathlineto{\pgfpoint{54.335763\du}{14.205371\du}}
\pgfpathlineto{\pgfpoint{54.329556\du}{14.198070\du}}
\pgfpathlineto{\pgfpoint{54.322985\du}{14.191863\du}}
\pgfpathlineto{\pgfpoint{54.316048\du}{14.184561\du}}
\pgfpathlineto{\pgfpoint{54.309476\du}{14.177624\du}}
\pgfpathlineto{\pgfpoint{54.302174\du}{14.171417\du}}
\pgfpathlineto{\pgfpoint{54.295602\du}{14.164116\du}}
\pgfpathlineto{\pgfpoint{54.287570\du}{14.157909\du}}
\pgfpathlineto{\pgfpoint{54.279538\du}{14.151337\du}}
\pgfpathlineto{\pgfpoint{54.272236\du}{14.144400\du}}
\pgfpathlineto{\pgfpoint{54.263474\du}{14.137829\du}}
\pgfpathlineto{\pgfpoint{54.255442\du}{14.131622\du}}
\pgfpathlineto{\pgfpoint{54.246314\du}{14.125050\du}}
\pgfpathlineto{\pgfpoint{54.237552\du}{14.118478\du}}
\pgfpathlineto{\pgfpoint{54.228059\du}{14.112272\du}}
\pgfpathlineto{\pgfpoint{54.218932\du}{14.105700\du}}
\pgfpathlineto{\pgfpoint{54.209805\du}{14.099128\du}}
\pgfpathlineto{\pgfpoint{54.199947\du}{14.093287\du}}
\pgfpathlineto{\pgfpoint{54.190089\du}{14.087080\du}}
\pgfpathlineto{\pgfpoint{54.179867\du}{14.080508\du}}
\pgfpathlineto{\pgfpoint{54.169279\du}{14.074667\du}}
\pgfpathlineto{\pgfpoint{54.159421\du}{14.068825\du}}
\pgfpathlineto{\pgfpoint{54.148833\du}{14.062984\du}}
\pgfpathlineto{\pgfpoint{54.137515\du}{14.056412\du}}
\pgfpathlineto{\pgfpoint{54.126197\du}{14.051301\du}}
\pgfpathlineto{\pgfpoint{54.114514\du}{14.045459\du}}
\pgfpathlineto{\pgfpoint{54.103561\du}{14.039618\du}}
\pgfpathlineto{\pgfpoint{54.091513\du}{14.033776\du}}
\pgfpathlineto{\pgfpoint{54.080195\du}{14.027934\du}}
\pgfpathlineto{\pgfpoint{54.067782\du}{14.022823\du}}
\pgfpathlineto{\pgfpoint{54.055369\du}{14.016616\du}}
\pgfpathlineto{\pgfpoint{54.042590\du}{14.010775\du}}
\pgfpathlineto{\pgfpoint{54.030542\du}{14.005664\du}}
\pgfpathlineto{\pgfpoint{54.017764\du}{14.000552\du}}
\pgfpathlineto{\pgfpoint{54.004255\du}{13.995076\du}}
\pgfpathlineto{\pgfpoint{53.991112\du}{13.989964\du}}
\pgfpathlineto{\pgfpoint{53.977968\du}{13.984488\du}}
\pgfpathlineto{\pgfpoint{53.963729\du}{13.979377\du}}
\pgfpathlineto{\pgfpoint{53.950586\du}{13.973900\du}}
\pgfpathlineto{\pgfpoint{53.936347\du}{13.969519\du}}
\pgfpathlineto{\pgfpoint{53.922473\du}{13.964043\du}}
\pgfpathlineto{\pgfpoint{53.908235\du}{13.959296\du}}
\pgfpathlineto{\pgfpoint{53.894361\du}{13.954185\du}}
\pgfpathlineto{\pgfpoint{53.879392\du}{13.949439\du}}
\pgfpathlineto{\pgfpoint{53.864423\du}{13.944692\du}}
\pgfpathlineto{\pgfpoint{53.850184\du}{13.940311\du}}
\pgfpathlineto{\pgfpoint{53.819516\du}{13.930819\du}}
\pgfpathlineto{\pgfpoint{53.788848\du}{13.922056\du}}
\pgfpathlineto{\pgfpoint{53.757450\du}{13.913294\du}}
\pgfpathlineto{\pgfpoint{53.725686\du}{13.904532\du}}
\pgfpathlineto{\pgfpoint{53.692828\du}{13.896135\du}}
\pgfpathlineto{\pgfpoint{53.659239\du}{13.888468\du}}
\pgfpathlineto{\pgfpoint{53.625285\du}{13.880435\du}}
\pgfpathlineto{\pgfpoint{53.590600\du}{13.873499\du}}
\pgfpathlineto{\pgfpoint{53.555916\du}{13.866562\du}}
\pgfpathlineto{\pgfpoint{53.520502\du}{13.859260\du}}
\pgfpathlineto{\pgfpoint{53.483992\du}{13.853053\du}}
\pgfpathlineto{\pgfpoint{53.447483\du}{13.846481\du}}
\pgfpathlineto{\pgfpoint{53.410608\du}{13.840640\du}}
\pgfpathlineto{\pgfpoint{53.373368\du}{13.834798\du}}
\pgfpathlineto{\pgfpoint{53.334303\du}{13.829687\du}}
\pgfpathlineto{\pgfpoint{53.296332\du}{13.824941\du}}
\pgfpathlineto{\pgfpoint{53.256902\du}{13.820194\du}}
\pgfpathlineto{\pgfpoint{53.218202\du}{13.815448\du}}
\pgfpathlineto{\pgfpoint{53.178041\du}{13.811432\du}}
\pgfpathlineto{\pgfpoint{53.138246\du}{13.807416\du}}
\pgfpathlineto{\pgfpoint{53.097355\du}{13.804495\du}}
\pgfpathlineto{\pgfpoint{53.056829\du}{13.801574\du}}
\pgfpathlineto{\pgfpoint{53.015208\du}{13.798654\du}}
\pgfpathlineto{\pgfpoint{52.973587\du}{13.796098\du}}
\pgfpathlineto{\pgfpoint{52.931966\du}{13.794273\du}}
\pgfpathlineto{\pgfpoint{52.889250\du}{13.792812\du}}
\pgfpathlineto{\pgfpoint{52.847263\du}{13.791352\du}}
\pgfpathlineto{\pgfpoint{52.804182\du}{13.790256\du}}
\pgfpathlineto{\pgfpoint{52.760736\du}{13.789891\du}}
\pgfpathlineto{\pgfpoint{52.717289\du}{13.789891\du}}
\pgfpathlineto{\pgfpoint{52.717289\du}{13.810337\du}}
\pgfpathlineto{\pgfpoint{52.760370\du}{13.810337\du}}
\pgfpathlineto{\pgfpoint{52.803817\du}{13.810702\du}}
\pgfpathlineto{\pgfpoint{52.845803\du}{13.811432\du}}
\pgfpathlineto{\pgfpoint{52.888519\du}{13.813258\du}}
\pgfpathlineto{\pgfpoint{52.931236\du}{13.814353\du}}
\pgfpathlineto{\pgfpoint{52.972857\du}{13.816543\du}}
\pgfpathlineto{\pgfpoint{53.014113\du}{13.819099\du}}
\pgfpathlineto{\pgfpoint{53.055004\du}{13.822020\du}}
\pgfpathlineto{\pgfpoint{53.096259\du}{13.824941\du}}
\pgfpathlineto{\pgfpoint{53.136420\du}{13.828227\du}}
\pgfpathlineto{\pgfpoint{53.176581\du}{13.831878\du}}
\pgfpathlineto{\pgfpoint{53.215646\du}{13.835894\du}}
\pgfpathlineto{\pgfpoint{53.255077\du}{13.839910\du}}
\pgfpathlineto{\pgfpoint{53.293777\du}{13.844656\du}}
\pgfpathlineto{\pgfpoint{53.332477\du}{13.850132\du}}
\pgfpathlineto{\pgfpoint{53.370082\du}{13.855244\du}}
\pgfpathlineto{\pgfpoint{53.407322\du}{13.861085\du}}
\pgfpathlineto{\pgfpoint{53.443832\du}{13.866927\du}}
\pgfpathlineto{\pgfpoint{53.480706\du}{13.873499\du}}
\pgfpathlineto{\pgfpoint{53.516121\du}{13.879340\du}}
\pgfpathlineto{\pgfpoint{53.551900\du}{13.886277\du}}
\pgfpathlineto{\pgfpoint{53.586584\du}{13.893214\du}}
\pgfpathlineto{\pgfpoint{53.620538\du}{13.900881\du}}
\pgfpathlineto{\pgfpoint{53.654857\du}{13.908548\du}}
\pgfpathlineto{\pgfpoint{53.688081\du}{13.916215\du}}
\pgfpathlineto{\pgfpoint{53.720210\du}{13.924247\du}}
\pgfpathlineto{\pgfpoint{53.752338\du}{13.933009\du}}
\pgfpathlineto{\pgfpoint{53.783737\du}{13.941407\du}}
\pgfpathlineto{\pgfpoint{53.814405\du}{13.950534\du}}
\pgfpathlineto{\pgfpoint{53.844343\du}{13.959296\du}}
\pgfpathlineto{\pgfpoint{53.858581\du}{13.964043\du}}
\pgfpathlineto{\pgfpoint{53.873185\du}{13.968789\du}}
\pgfpathlineto{\pgfpoint{53.887059\du}{13.973900\du}}
\pgfpathlineto{\pgfpoint{53.901663\du}{13.978646\du}}
\pgfpathlineto{\pgfpoint{53.915902\du}{13.983393\du}}
\pgfpathlineto{\pgfpoint{53.929775\du}{13.988869\du}}
\pgfpathlineto{\pgfpoint{53.944014\du}{13.993250\du}}
\pgfpathlineto{\pgfpoint{53.957158\du}{13.998727\du}}
\pgfpathlineto{\pgfpoint{53.970666\du}{14.003838\du}}
\pgfpathlineto{\pgfpoint{53.983810\du}{14.008584\du}}
\pgfpathlineto{\pgfpoint{53.996588\du}{14.014426\du}}
\pgfpathlineto{\pgfpoint{54.009366\du}{14.019537\du}}
\pgfpathlineto{\pgfpoint{54.022510\du}{14.025014\du}}
\pgfpathlineto{\pgfpoint{54.034558\du}{14.030125\du}}
\pgfpathlineto{\pgfpoint{54.046971\du}{14.035601\du}}
\pgfpathlineto{\pgfpoint{54.059020\du}{14.041443\du}}
\pgfpathlineto{\pgfpoint{54.071433\du}{14.046554\du}}
\pgfpathlineto{\pgfpoint{54.082751\du}{14.052031\du}}
\pgfpathlineto{\pgfpoint{54.094799\du}{14.057872\du}}
\pgfpathlineto{\pgfpoint{54.105387\du}{14.063714\du}}
\pgfpathlineto{\pgfpoint{54.117435\du}{14.069555\du}}
\pgfpathlineto{\pgfpoint{54.127658\du}{14.075397\du}}
\pgfpathlineto{\pgfpoint{54.138611\du}{14.080508\du}}
\pgfpathlineto{\pgfpoint{54.149564\du}{14.087080\du}}
\pgfpathlineto{\pgfpoint{54.159421\du}{14.092922\du}}
\pgfpathlineto{\pgfpoint{54.169279\du}{14.098763\du}}
\pgfpathlineto{\pgfpoint{54.179501\du}{14.104605\du}}
\pgfpathlineto{\pgfpoint{54.188629\du}{14.110446\du}}
\pgfpathlineto{\pgfpoint{54.198852\du}{14.117018\du}}
\pgfpathlineto{\pgfpoint{54.208344\du}{14.122860\du}}
\pgfpathlineto{\pgfpoint{54.217106\du}{14.129066\du}}
\pgfpathlineto{\pgfpoint{54.225139\du}{14.134908\du}}
\pgfpathlineto{\pgfpoint{54.233901\du}{14.141480\du}}
\pgfpathlineto{\pgfpoint{54.242298\du}{14.147686\du}}
\pgfpathlineto{\pgfpoint{54.251060\du}{14.154258\du}}
\pgfpathlineto{\pgfpoint{54.258727\du}{14.160830\du}}
\pgfpathlineto{\pgfpoint{54.266760\du}{14.166671\du}}
\pgfpathlineto{\pgfpoint{54.274062\du}{14.172878\du}}
\pgfpathlineto{\pgfpoint{54.281729\du}{14.180180\du}}
\pgfpathlineto{\pgfpoint{54.288300\du}{14.186021\du}}
\pgfpathlineto{\pgfpoint{54.295602\du}{14.192958\du}}
\pgfpathlineto{\pgfpoint{54.301444\du}{14.199530\du}}
\pgfpathlineto{\pgfpoint{54.308381\du}{14.205737\du}}
\pgfpathlineto{\pgfpoint{54.314952\du}{14.212308\du}}
\pgfpathlineto{\pgfpoint{54.320064\du}{14.219245\du}}
\pgfpathlineto{\pgfpoint{54.325540\du}{14.225817\du}}
\pgfpathlineto{\pgfpoint{54.331747\du}{14.232754\du}}
\pgfpathlineto{\pgfpoint{54.336493\du}{14.238960\du}}
\pgfpathlineto{\pgfpoint{54.341604\du}{14.245532\du}}
\pgfpathlineto{\pgfpoint{54.346351\du}{14.252469\du}}
\pgfpathlineto{\pgfpoint{54.350732\du}{14.259041\du}}
\pgfpathlineto{\pgfpoint{54.355113\du}{14.265978\du}}
\pgfpathlineto{\pgfpoint{54.358034\du}{14.272549\du}}
\pgfpathlineto{\pgfpoint{54.362050\du}{14.279486\du}}
\pgfpathlineto{\pgfpoint{54.364971\du}{14.286423\du}}
\pgfpathlineto{\pgfpoint{54.368987\du}{14.293725\du}}
\pgfpathlineto{\pgfpoint{54.371542\du}{14.299932\du}}
\pgfpathlineto{\pgfpoint{54.374463\du}{14.306868\du}}
\pgfpathlineto{\pgfpoint{54.377019\du}{14.314170\du}}
\pgfpathlineto{\pgfpoint{54.379209\du}{14.321107\du}}
\pgfpathlineto{\pgfpoint{54.380670\du}{14.328044\du}}
\pgfpathlineto{\pgfpoint{54.382860\du}{14.334981\du}}
\pgfpathlineto{\pgfpoint{54.383591\du}{14.342283\du}}
\pgfpathlineto{\pgfpoint{54.384686\du}{14.348489\du}}
\pgfpathlineto{\pgfpoint{54.386511\du}{14.356156\du}}
\pgfpathlineto{\pgfpoint{54.386876\du}{14.363093\du}}
\pgfpathlineto{\pgfpoint{54.386876\du}{14.370030\du}}
\pgfpathlineto{\pgfpoint{54.387607\du}{14.377332\du}}
\pgfpathlineto{\pgfpoint{54.407687\du}{14.377332\du}}
\pgfusepath{fill}
\pgfsetbuttcap
\pgfsetmiterjoin
\pgfsetdash{}{0pt}
\definecolor{dialinecolor}{rgb}{0.027451, 0.486275, 0.682353}
\pgfsetfillcolor{dialinecolor}
\pgfpathmoveto{\pgfpoint{51.031637\du}{13.567913\du}}
\pgfpathlineto{\pgfpoint{51.031637\du}{14.392301\du}}
\pgfpathlineto{\pgfpoint{54.397099\du}{14.392301\du}}
\pgfpathlineto{\pgfpoint{54.397829\du}{13.568643\du}}
\pgfpathlineto{\pgfpoint{51.031637\du}{13.567913\du}}
\pgfusepath{fill}
\pgfsetbuttcap
\pgfsetmiterjoin
\pgfsetdash{}{0pt}
\definecolor{dialinecolor}{rgb}{0.235294, 0.686275, 0.894118}
\pgfsetfillcolor{dialinecolor}
\pgfpathmoveto{\pgfpoint{54.397099\du}{13.552213\du}}
\pgfpathlineto{\pgfpoint{54.395639\du}{13.582151\du}}
\pgfpathlineto{\pgfpoint{54.388337\du}{13.611359\du}}
\pgfpathlineto{\pgfpoint{54.378114\du}{13.640567\du}}
\pgfpathlineto{\pgfpoint{54.363510\du}{13.668679\du}}
\pgfpathlineto{\pgfpoint{54.344160\du}{13.696792\du}}
\pgfpathlineto{\pgfpoint{54.322254\du}{13.724174\du}}
\pgfpathlineto{\pgfpoint{54.295602\du}{13.750461\du}}
\pgfpathlineto{\pgfpoint{54.264934\du}{13.776748\du}}
\pgfpathlineto{\pgfpoint{54.232075\du}{13.802670\du}}
\pgfpathlineto{\pgfpoint{54.194836\du}{13.827861\du}}
\pgfpathlineto{\pgfpoint{54.153945\du}{13.851958\du}}
\pgfpathlineto{\pgfpoint{54.110133\du}{13.875324\du}}
\pgfpathlineto{\pgfpoint{54.063766\du}{13.897595\du}}
\pgfpathlineto{\pgfpoint{54.013382\du}{13.919501\du}}
\pgfpathlineto{\pgfpoint{53.960443\du}{13.940676\du}}
\pgfpathlineto{\pgfpoint{53.904949\du}{13.960757\du}}
\pgfpathlineto{\pgfpoint{53.846898\du}{13.979377\du}}
\pgfpathlineto{\pgfpoint{53.786292\du}{13.997997\du}}
\pgfpathlineto{\pgfpoint{53.722400\du}{14.015156\du}}
\pgfpathlineto{\pgfpoint{53.657413\du}{14.030855\du}}
\pgfpathlineto{\pgfpoint{53.588775\du}{14.046189\du}}
\pgfpathlineto{\pgfpoint{53.517946\du}{14.060063\du}}
\pgfpathlineto{\pgfpoint{53.446022\du}{14.072841\du}}
\pgfpathlineto{\pgfpoint{53.371177\du}{14.084159\du}}
\pgfpathlineto{\pgfpoint{53.295237\du}{14.094747\du}}
\pgfpathlineto{\pgfpoint{53.216741\du}{14.103875\du}}
\pgfpathlineto{\pgfpoint{53.137150\du}{14.111542\du}}
\pgfpathlineto{\pgfpoint{53.056099\du}{14.118113\du}}
\pgfpathlineto{\pgfpoint{52.972857\du}{14.123225\du}}
\pgfpathlineto{\pgfpoint{52.889250\du}{14.126876\du}}
\pgfpathlineto{\pgfpoint{52.803817\du}{14.128701\du}}
\pgfpathlineto{\pgfpoint{52.717289\du}{14.129796\du}}
\pgfpathlineto{\pgfpoint{52.631126\du}{14.128701\du}}
\pgfpathlineto{\pgfpoint{52.545328\du}{14.126876\du}}
\pgfpathlineto{\pgfpoint{52.461721\du}{14.123225\du}}
\pgfpathlineto{\pgfpoint{52.378844\du}{14.118113\du}}
\pgfpathlineto{\pgfpoint{52.297428\du}{14.111542\du}}
\pgfpathlineto{\pgfpoint{52.217837\du}{14.103875\du}}
\pgfpathlineto{\pgfpoint{52.140071\du}{14.094747\du}}
\pgfpathlineto{\pgfpoint{52.063401\du}{14.084159\du}}
\pgfpathlineto{\pgfpoint{51.989286\du}{14.072841\du}}
\pgfpathlineto{\pgfpoint{51.916632\du}{14.060063\du}}
\pgfpathlineto{\pgfpoint{51.846168\du}{14.046189\du}}
\pgfpathlineto{\pgfpoint{51.777530\du}{14.030855\du}}
\pgfpathlineto{\pgfpoint{51.711813\du}{14.015156\du}}
\pgfpathlineto{\pgfpoint{51.648286\du}{13.997997\du}}
\pgfpathlineto{\pgfpoint{51.587315\du}{13.979377\du}}
\pgfpathlineto{\pgfpoint{51.528899\du}{13.960757\du}}
\pgfpathlineto{\pgfpoint{51.473769\du}{13.940676\du}}
\pgfpathlineto{\pgfpoint{51.420830\du}{13.919501\du}}
\pgfpathlineto{\pgfpoint{51.370812\du}{13.897595\du}}
\pgfpathlineto{\pgfpoint{51.323715\du}{13.875324\du}}
\pgfpathlineto{\pgfpoint{51.280268\du}{13.851958\du}}
\pgfpathlineto{\pgfpoint{51.239377\du}{13.827861\du}}
\pgfpathlineto{\pgfpoint{51.202137\du}{13.802670\du}}
\pgfpathlineto{\pgfpoint{51.168914\du}{13.776748\du}}
\pgfpathlineto{\pgfpoint{51.138611\du}{13.750461\du}}
\pgfpathlineto{\pgfpoint{51.111959\du}{13.724174\du}}
\pgfpathlineto{\pgfpoint{51.090053\du}{13.696792\du}}
\pgfpathlineto{\pgfpoint{51.070703\du}{13.668679\du}}
\pgfpathlineto{\pgfpoint{51.056099\du}{13.640567\du}}
\pgfpathlineto{\pgfpoint{51.045511\du}{13.611359\du}}
\pgfpathlineto{\pgfpoint{51.038574\du}{13.582151\du}}
\pgfpathlineto{\pgfpoint{51.036749\du}{13.552213\du}}
\pgfpathlineto{\pgfpoint{51.038574\du}{13.523006\du}}
\pgfpathlineto{\pgfpoint{51.045511\du}{13.493068\du}}
\pgfpathlineto{\pgfpoint{51.056099\du}{13.464590\du}}
\pgfpathlineto{\pgfpoint{51.070703\du}{13.436478\du}}
\pgfpathlineto{\pgfpoint{51.090053\du}{13.408365\du}}
\pgfpathlineto{\pgfpoint{51.111959\du}{13.380618\du}}
\pgfpathlineto{\pgfpoint{51.138611\du}{13.353966\du}}
\pgfpathlineto{\pgfpoint{51.168914\du}{13.327679\du}}
\pgfpathlineto{\pgfpoint{51.202137\du}{13.302487\du}}
\pgfpathlineto{\pgfpoint{51.239377\du}{13.277296\du}}
\pgfpathlineto{\pgfpoint{51.280268\du}{13.253199\du}}
\pgfpathlineto{\pgfpoint{51.323715\du}{13.229833\du}}
\pgfpathlineto{\pgfpoint{51.370812\du}{13.206832\du}}
\pgfpathlineto{\pgfpoint{51.420830\du}{13.185291\du}}
\pgfpathlineto{\pgfpoint{51.473769\du}{13.164116\du}}
\pgfpathlineto{\pgfpoint{51.528899\du}{13.144400\du}}
\pgfpathlineto{\pgfpoint{51.587315\du}{13.125050\du}}
\pgfpathlineto{\pgfpoint{51.648286\du}{13.106795\du}}
\pgfpathlineto{\pgfpoint{51.711813\du}{13.090001\du}}
\pgfpathlineto{\pgfpoint{51.777530\du}{13.073572\du}}
\pgfpathlineto{\pgfpoint{51.846168\du}{13.058237\du}}
\pgfpathlineto{\pgfpoint{51.916632\du}{13.044729\du}}
\pgfpathlineto{\pgfpoint{51.989286\du}{13.031951\du}}
\pgfpathlineto{\pgfpoint{52.063401\du}{13.020267\du}}
\pgfpathlineto{\pgfpoint{52.140071\du}{13.010410\du}}
\pgfpathlineto{\pgfpoint{52.217837\du}{13.000917\du}}
\pgfpathlineto{\pgfpoint{52.297428\du}{12.993250\du}}
\pgfpathlineto{\pgfpoint{52.378844\du}{12.986679\du}}
\pgfpathlineto{\pgfpoint{52.461721\du}{12.981567\du}}
\pgfpathlineto{\pgfpoint{52.545328\du}{12.978281\du}}
\pgfpathlineto{\pgfpoint{52.631126\du}{12.975726\du}}
\pgfpathlineto{\pgfpoint{52.717289\du}{12.975361\du}}
\pgfpathlineto{\pgfpoint{52.803817\du}{12.975726\du}}
\pgfpathlineto{\pgfpoint{52.889250\du}{12.978281\du}}
\pgfpathlineto{\pgfpoint{52.972857\du}{12.981567\du}}
\pgfpathlineto{\pgfpoint{53.056099\du}{12.986679\du}}
\pgfpathlineto{\pgfpoint{53.137150\du}{12.993250\du}}
\pgfpathlineto{\pgfpoint{53.216741\du}{13.000917\du}}
\pgfpathlineto{\pgfpoint{53.295237\du}{13.010410\du}}
\pgfpathlineto{\pgfpoint{53.371177\du}{13.020267\du}}
\pgfpathlineto{\pgfpoint{53.446022\du}{13.031951\du}}
\pgfpathlineto{\pgfpoint{53.517946\du}{13.044729\du}}
\pgfpathlineto{\pgfpoint{53.588775\du}{13.058237\du}}
\pgfpathlineto{\pgfpoint{53.657413\du}{13.073572\du}}
\pgfpathlineto{\pgfpoint{53.722400\du}{13.090001\du}}
\pgfpathlineto{\pgfpoint{53.786292\du}{13.106795\du}}
\pgfpathlineto{\pgfpoint{53.846898\du}{13.125050\du}}
\pgfpathlineto{\pgfpoint{53.904949\du}{13.144400\du}}
\pgfpathlineto{\pgfpoint{53.960443\du}{13.164116\du}}
\pgfpathlineto{\pgfpoint{54.013382\du}{13.185291\du}}
\pgfpathlineto{\pgfpoint{54.063766\du}{13.206832\du}}
\pgfpathlineto{\pgfpoint{54.110133\du}{13.229833\du}}
\pgfpathlineto{\pgfpoint{54.153945\du}{13.253199\du}}
\pgfpathlineto{\pgfpoint{54.194836\du}{13.277296\du}}
\pgfpathlineto{\pgfpoint{54.232075\du}{13.302487\du}}
\pgfpathlineto{\pgfpoint{54.264934\du}{13.327679\du}}
\pgfpathlineto{\pgfpoint{54.295602\du}{13.353966\du}}
\pgfpathlineto{\pgfpoint{54.322254\du}{13.380618\du}}
\pgfpathlineto{\pgfpoint{54.344160\du}{13.408365\du}}
\pgfpathlineto{\pgfpoint{54.363510\du}{13.436478\du}}
\pgfpathlineto{\pgfpoint{54.378114\du}{13.464590\du}}
\pgfpathlineto{\pgfpoint{54.388337\du}{13.493068\du}}
\pgfpathlineto{\pgfpoint{54.395639\du}{13.523006\du}}
\pgfpathlineto{\pgfpoint{54.397099\du}{13.552213\du}}
\pgfusepath{fill}
\pgfsetbuttcap
\pgfsetmiterjoin
\pgfsetdash{}{0pt}
\definecolor{dialinecolor}{rgb}{0.678431, 0.839216, 0.905882}
\pgfsetfillcolor{dialinecolor}
\pgfpathmoveto{\pgfpoint{52.717289\du}{14.139654\du}}
\pgfpathlineto{\pgfpoint{52.717289\du}{14.139654\du}}
\pgfpathlineto{\pgfpoint{52.760736\du}{14.139654\du}}
\pgfpathlineto{\pgfpoint{52.804182\du}{14.138924\du}}
\pgfpathlineto{\pgfpoint{52.847263\du}{14.137829\du}}
\pgfpathlineto{\pgfpoint{52.889250\du}{14.136733\du}}
\pgfpathlineto{\pgfpoint{52.931966\du}{14.134908\du}}
\pgfpathlineto{\pgfpoint{52.973587\du}{14.133082\du}}
\pgfpathlineto{\pgfpoint{53.015208\du}{14.130892\du}}
\pgfpathlineto{\pgfpoint{53.056829\du}{14.127971\du}}
\pgfpathlineto{\pgfpoint{53.097355\du}{14.125050\du}}
\pgfpathlineto{\pgfpoint{53.138246\du}{14.122129\du}}
\pgfpathlineto{\pgfpoint{53.178041\du}{14.118113\du}}
\pgfpathlineto{\pgfpoint{53.218202\du}{14.114097\du}}
\pgfpathlineto{\pgfpoint{53.256902\du}{14.109351\du}}
\pgfpathlineto{\pgfpoint{53.296332\du}{14.104605\du}}
\pgfpathlineto{\pgfpoint{53.334303\du}{14.099859\du}}
\pgfpathlineto{\pgfpoint{53.373368\du}{14.094747\du}}
\pgfpathlineto{\pgfpoint{53.410608\du}{14.088906\du}}
\pgfpathlineto{\pgfpoint{53.447483\du}{14.083064\du}}
\pgfpathlineto{\pgfpoint{53.483992\du}{14.076492\du}}
\pgfpathlineto{\pgfpoint{53.520502\du}{14.069921\du}}
\pgfpathlineto{\pgfpoint{53.555916\du}{14.062984\du}}
\pgfpathlineto{\pgfpoint{53.590600\du}{14.056047\du}}
\pgfpathlineto{\pgfpoint{53.625285\du}{14.049110\du}}
\pgfpathlineto{\pgfpoint{53.659239\du}{14.041443\du}}
\pgfpathlineto{\pgfpoint{53.692828\du}{14.033046\du}}
\pgfpathlineto{\pgfpoint{53.725686\du}{14.025014\du}}
\pgfpathlineto{\pgfpoint{53.757450\du}{14.016616\du}}
\pgfpathlineto{\pgfpoint{53.788848\du}{14.007489\du}}
\pgfpathlineto{\pgfpoint{53.804182\du}{14.003473\du}}
\pgfpathlineto{\pgfpoint{53.819516\du}{13.998727\du}}
\pgfpathlineto{\pgfpoint{53.835580\du}{13.993980\du}}
\pgfpathlineto{\pgfpoint{53.850184\du}{13.989234\du}}
\pgfpathlineto{\pgfpoint{53.864423\du}{13.984488\du}}
\pgfpathlineto{\pgfpoint{53.879392\du}{13.980107\du}}
\pgfpathlineto{\pgfpoint{53.894361\du}{13.975361\du}}
\pgfpathlineto{\pgfpoint{53.908235\du}{13.969884\du}}
\pgfpathlineto{\pgfpoint{53.922473\du}{13.965138\du}}
\pgfpathlineto{\pgfpoint{53.936347\du}{13.960026\du}}
\pgfpathlineto{\pgfpoint{53.950586\du}{13.955280\du}}
\pgfpathlineto{\pgfpoint{53.963729\du}{13.950169\du}}
\pgfpathlineto{\pgfpoint{53.977968\du}{13.944692\du}}
\pgfpathlineto{\pgfpoint{53.991112\du}{13.939581\du}}
\pgfpathlineto{\pgfpoint{54.004255\du}{13.934470\du}}
\pgfpathlineto{\pgfpoint{54.017764\du}{13.928993\du}}
\pgfpathlineto{\pgfpoint{54.030542\du}{13.923882\du}}
\pgfpathlineto{\pgfpoint{54.042590\du}{13.918405\du}}
\pgfpathlineto{\pgfpoint{54.055369\du}{13.912564\du}}
\pgfpathlineto{\pgfpoint{54.067782\du}{13.906722\du}}
\pgfpathlineto{\pgfpoint{54.080195\du}{13.901611\du}}
\pgfpathlineto{\pgfpoint{54.091513\du}{13.895769\du}}
\pgfpathlineto{\pgfpoint{54.103561\du}{13.889928\du}}
\pgfpathlineto{\pgfpoint{54.114514\du}{13.884086\du}}
\pgfpathlineto{\pgfpoint{54.126197\du}{13.878610\du}}
\pgfpathlineto{\pgfpoint{54.137515\du}{13.872768\du}}
\pgfpathlineto{\pgfpoint{54.148833\du}{13.866562\du}}
\pgfpathlineto{\pgfpoint{54.159421\du}{13.860720\du}}
\pgfpathlineto{\pgfpoint{54.169279\du}{13.854879\du}}
\pgfpathlineto{\pgfpoint{54.179867\du}{13.849037\du}}
\pgfpathlineto{\pgfpoint{54.190089\du}{13.842465\du}}
\pgfpathlineto{\pgfpoint{54.199947\du}{13.835894\du}}
\pgfpathlineto{\pgfpoint{54.209805\du}{13.830052\du}}
\pgfpathlineto{\pgfpoint{54.218932\du}{13.823845\du}}
\pgfpathlineto{\pgfpoint{54.228059\du}{13.817274\du}}
\pgfpathlineto{\pgfpoint{54.237552\du}{13.810702\du}}
\pgfpathlineto{\pgfpoint{54.246314\du}{13.804495\du}}
\pgfpathlineto{\pgfpoint{54.255442\du}{13.797924\du}}
\pgfpathlineto{\pgfpoint{54.263474\du}{13.791352\du}}
\pgfpathlineto{\pgfpoint{54.272236\du}{13.785145\du}}
\pgfpathlineto{\pgfpoint{54.279538\du}{13.778573\du}}
\pgfpathlineto{\pgfpoint{54.287570\du}{13.771637\du}}
\pgfpathlineto{\pgfpoint{54.295602\du}{13.765065\du}}
\pgfpathlineto{\pgfpoint{54.302174\du}{13.758128\du}}
\pgfpathlineto{\pgfpoint{54.309476\du}{13.751556\du}}
\pgfpathlineto{\pgfpoint{54.316048\du}{13.744619\du}}
\pgfpathlineto{\pgfpoint{54.322985\du}{13.737683\du}}
\pgfpathlineto{\pgfpoint{54.329556\du}{13.731111\du}}
\pgfpathlineto{\pgfpoint{54.335763\du}{13.724174\du}}
\pgfpathlineto{\pgfpoint{54.341604\du}{13.717237\du}}
\pgfpathlineto{\pgfpoint{54.347081\du}{13.710300\du}}
\pgfpathlineto{\pgfpoint{54.352922\du}{13.702633\du}}
\pgfpathlineto{\pgfpoint{54.357669\du}{13.695696\du}}
\pgfpathlineto{\pgfpoint{54.363145\du}{13.688394\du}}
\pgfpathlineto{\pgfpoint{54.367891\du}{13.681458\du}}
\pgfpathlineto{\pgfpoint{54.372273\du}{13.673791\du}}
\pgfpathlineto{\pgfpoint{54.376289\du}{13.666854\du}}
\pgfpathlineto{\pgfpoint{54.380305\du}{13.659187\du}}
\pgfpathlineto{\pgfpoint{54.383591\du}{13.651520\du}}
\pgfpathlineto{\pgfpoint{54.387607\du}{13.644218\du}}
\pgfpathlineto{\pgfpoint{54.390893\du}{13.636916\du}}
\pgfpathlineto{\pgfpoint{54.393448\du}{13.629614\du}}
\pgfpathlineto{\pgfpoint{54.396369\du}{13.621947\du}}
\pgfpathlineto{\pgfpoint{54.398194\du}{13.614280\du}}
\pgfpathlineto{\pgfpoint{54.401115\du}{13.606613\du}}
\pgfpathlineto{\pgfpoint{54.402211\du}{13.598946\du}}
\pgfpathlineto{\pgfpoint{54.404401\du}{13.591279\du}}
\pgfpathlineto{\pgfpoint{54.405496\du}{13.583977\du}}
\pgfpathlineto{\pgfpoint{54.406227\du}{13.575580\du}}
\pgfpathlineto{\pgfpoint{54.406957\du}{13.567913\du}}
\pgfpathlineto{\pgfpoint{54.407687\du}{13.560246\du}}
\pgfpathlineto{\pgfpoint{54.407687\du}{13.552213\du}}
\pgfpathlineto{\pgfpoint{54.387607\du}{13.552213\du}}
\pgfpathlineto{\pgfpoint{54.386876\du}{13.559150\du}}
\pgfpathlineto{\pgfpoint{54.386876\du}{13.566817\du}}
\pgfpathlineto{\pgfpoint{54.386511\du}{13.573754\du}}
\pgfpathlineto{\pgfpoint{54.384686\du}{13.580691\du}}
\pgfpathlineto{\pgfpoint{54.383591\du}{13.587993\du}}
\pgfpathlineto{\pgfpoint{54.382860\du}{13.594200\du}}
\pgfpathlineto{\pgfpoint{54.380670\du}{13.601501\du}}
\pgfpathlineto{\pgfpoint{54.379209\du}{13.608438\du}}
\pgfpathlineto{\pgfpoint{54.377019\du}{13.615375\du}}
\pgfpathlineto{\pgfpoint{54.374463\du}{13.622312\du}}
\pgfpathlineto{\pgfpoint{54.371542\du}{13.629614\du}}
\pgfpathlineto{\pgfpoint{54.368987\du}{13.635821\du}}
\pgfpathlineto{\pgfpoint{54.364971\du}{13.642757\du}}
\pgfpathlineto{\pgfpoint{54.362050\du}{13.650059\du}}
\pgfpathlineto{\pgfpoint{54.358034\du}{13.656996\du}}
\pgfpathlineto{\pgfpoint{54.355113\du}{13.663203\du}}
\pgfpathlineto{\pgfpoint{54.350732\du}{13.670505\du}}
\pgfpathlineto{\pgfpoint{54.346351\du}{13.676711\du}}
\pgfpathlineto{\pgfpoint{54.341604\du}{13.684013\du}}
\pgfpathlineto{\pgfpoint{54.336493\du}{13.690220\du}}
\pgfpathlineto{\pgfpoint{54.331747\du}{13.697157\du}}
\pgfpathlineto{\pgfpoint{54.325540\du}{13.703729\du}}
\pgfpathlineto{\pgfpoint{54.320064\du}{13.710300\du}}
\pgfpathlineto{\pgfpoint{54.314952\du}{13.717237\du}}
\pgfpathlineto{\pgfpoint{54.308381\du}{13.723809\du}}
\pgfpathlineto{\pgfpoint{54.301444\du}{13.730016\du}}
\pgfpathlineto{\pgfpoint{54.295602\du}{13.736587\du}}
\pgfpathlineto{\pgfpoint{54.288300\du}{13.743524\du}}
\pgfpathlineto{\pgfpoint{54.281729\du}{13.750096\du}}
\pgfpathlineto{\pgfpoint{54.274062\du}{13.756302\du}}
\pgfpathlineto{\pgfpoint{54.266760\du}{13.762874\du}}
\pgfpathlineto{\pgfpoint{54.258727\du}{13.768716\du}}
\pgfpathlineto{\pgfpoint{54.251060\du}{13.775288\du}}
\pgfpathlineto{\pgfpoint{54.242298\du}{13.781494\du}}
\pgfpathlineto{\pgfpoint{54.233901\du}{13.788066\du}}
\pgfpathlineto{\pgfpoint{54.225139\du}{13.794273\du}}
\pgfpathlineto{\pgfpoint{54.217106\du}{13.800114\du}}
\pgfpathlineto{\pgfpoint{54.208344\du}{13.806686\du}}
\pgfpathlineto{\pgfpoint{54.198852\du}{13.812527\du}}
\pgfpathlineto{\pgfpoint{54.188629\du}{13.819099\du}}
\pgfpathlineto{\pgfpoint{54.179501\du}{13.824941\du}}
\pgfpathlineto{\pgfpoint{54.169279\du}{13.830782\du}}
\pgfpathlineto{\pgfpoint{54.159421\du}{13.836624\du}}
\pgfpathlineto{\pgfpoint{54.149564\du}{13.842465\du}}
\pgfpathlineto{\pgfpoint{54.138611\du}{13.849037\du}}
\pgfpathlineto{\pgfpoint{54.127658\du}{13.854148\du}}
\pgfpathlineto{\pgfpoint{54.117435\du}{13.859990\du}}
\pgfpathlineto{\pgfpoint{54.105387\du}{13.865832\du}}
\pgfpathlineto{\pgfpoint{54.094799\du}{13.871673\du}}
\pgfpathlineto{\pgfpoint{54.082751\du}{13.877515\du}}
\pgfpathlineto{\pgfpoint{54.071433\du}{13.882626\du}}
\pgfpathlineto{\pgfpoint{54.059020\du}{13.888102\du}}
\pgfpathlineto{\pgfpoint{54.046971\du}{13.893944\du}}
\pgfpathlineto{\pgfpoint{54.034558\du}{13.899055\du}}
\pgfpathlineto{\pgfpoint{54.022510\du}{13.904532\du}}
\pgfpathlineto{\pgfpoint{54.009366\du}{13.909643\du}}
\pgfpathlineto{\pgfpoint{53.996588\du}{13.915120\du}}
\pgfpathlineto{\pgfpoint{53.983810\du}{13.920961\du}}
\pgfpathlineto{\pgfpoint{53.970666\du}{13.925342\du}}
\pgfpathlineto{\pgfpoint{53.957158\du}{13.930819\du}}
\pgfpathlineto{\pgfpoint{53.944014\du}{13.935930\du}}
\pgfpathlineto{\pgfpoint{53.929775\du}{13.940676\du}}
\pgfpathlineto{\pgfpoint{53.915902\du}{13.946153\du}}
\pgfpathlineto{\pgfpoint{53.901663\du}{13.950534\du}}
\pgfpathlineto{\pgfpoint{53.887059\du}{13.955280\du}}
\pgfpathlineto{\pgfpoint{53.873185\du}{13.960757\du}}
\pgfpathlineto{\pgfpoint{53.858581\du}{13.965138\du}}
\pgfpathlineto{\pgfpoint{53.844343\du}{13.969884\du}}
\pgfpathlineto{\pgfpoint{53.828644\du}{13.974630\du}}
\pgfpathlineto{\pgfpoint{53.814405\du}{13.978646\du}}
\pgfpathlineto{\pgfpoint{53.798706\du}{13.983393\du}}
\pgfpathlineto{\pgfpoint{53.783737\du}{13.988139\du}}
\pgfpathlineto{\pgfpoint{53.752338\du}{13.996171\du}}
\pgfpathlineto{\pgfpoint{53.720210\du}{14.004933\du}}
\pgfpathlineto{\pgfpoint{53.688081\du}{14.013331\du}}
\pgfpathlineto{\pgfpoint{53.654857\du}{14.020998\du}}
\pgfpathlineto{\pgfpoint{53.620538\du}{14.028665\du}}
\pgfpathlineto{\pgfpoint{53.586584\du}{14.035967\du}}
\pgfpathlineto{\pgfpoint{53.551900\du}{14.043269\du}}
\pgfpathlineto{\pgfpoint{53.516121\du}{14.050205\du}}
\pgfpathlineto{\pgfpoint{53.480706\du}{14.056412\du}}
\pgfpathlineto{\pgfpoint{53.443832\du}{14.062254\du}}
\pgfpathlineto{\pgfpoint{53.407322\du}{14.068460\du}}
\pgfpathlineto{\pgfpoint{53.370082\du}{14.074302\du}}
\pgfpathlineto{\pgfpoint{53.332477\du}{14.079413\du}}
\pgfpathlineto{\pgfpoint{53.293777\du}{14.084524\du}}
\pgfpathlineto{\pgfpoint{53.255077\du}{14.089271\du}}
\pgfpathlineto{\pgfpoint{53.215646\du}{14.093287\du}}
\pgfpathlineto{\pgfpoint{53.176581\du}{14.097668\du}}
\pgfpathlineto{\pgfpoint{53.136420\du}{14.100954\du}}
\pgfpathlineto{\pgfpoint{53.096259\du}{14.104605\du}}
\pgfpathlineto{\pgfpoint{53.055004\du}{14.107526\du}}
\pgfpathlineto{\pgfpoint{53.014113\du}{14.110446\du}}
\pgfpathlineto{\pgfpoint{52.972857\du}{14.112637\du}}
\pgfpathlineto{\pgfpoint{52.931236\du}{14.115193\du}}
\pgfpathlineto{\pgfpoint{52.888519\du}{14.116288\du}}
\pgfpathlineto{\pgfpoint{52.845803\du}{14.118113\du}}
\pgfpathlineto{\pgfpoint{52.803817\du}{14.118478\du}}
\pgfpathlineto{\pgfpoint{52.760370\du}{14.119209\du}}
\pgfpathlineto{\pgfpoint{52.717289\du}{14.119209\du}}
\pgfpathlineto{\pgfpoint{52.717289\du}{14.119209\du}}
\pgfpathlineto{\pgfpoint{52.717289\du}{14.119209\du}}
\pgfpathlineto{\pgfpoint{52.716559\du}{14.119939\du}}
\pgfpathlineto{\pgfpoint{52.714733\du}{14.119939\du}}
\pgfpathlineto{\pgfpoint{52.713638\du}{14.119939\du}}
\pgfpathlineto{\pgfpoint{52.712908\du}{14.120304\du}}
\pgfpathlineto{\pgfpoint{52.712543\du}{14.121034\du}}
\pgfpathlineto{\pgfpoint{52.711082\du}{14.121034\du}}
\pgfpathlineto{\pgfpoint{52.710352\du}{14.122129\du}}
\pgfpathlineto{\pgfpoint{52.709622\du}{14.122860\du}}
\pgfpathlineto{\pgfpoint{52.708527\du}{14.123955\du}}
\pgfpathlineto{\pgfpoint{52.707796\du}{14.125780\du}}
\pgfpathlineto{\pgfpoint{52.707796\du}{14.127971\du}}
\pgfpathlineto{\pgfpoint{52.707066\du}{14.129796\du}}
\pgfpathlineto{\pgfpoint{52.707796\du}{14.131622\du}}
\pgfpathlineto{\pgfpoint{52.707796\du}{14.133082\du}}
\pgfpathlineto{\pgfpoint{52.708527\du}{14.134908\du}}
\pgfpathlineto{\pgfpoint{52.709622\du}{14.136733\du}}
\pgfpathlineto{\pgfpoint{52.710352\du}{14.137463\du}}
\pgfpathlineto{\pgfpoint{52.711082\du}{14.137829\du}}
\pgfpathlineto{\pgfpoint{52.712543\du}{14.138559\du}}
\pgfpathlineto{\pgfpoint{52.712908\du}{14.138924\du}}
\pgfpathlineto{\pgfpoint{52.713638\du}{14.139654\du}}
\pgfpathlineto{\pgfpoint{52.714733\du}{14.139654\du}}
\pgfpathlineto{\pgfpoint{52.716559\du}{14.139654\du}}
\pgfpathlineto{\pgfpoint{52.717289\du}{14.139654\du}}
\pgfusepath{fill}
\pgfsetbuttcap
\pgfsetmiterjoin
\pgfsetdash{}{0pt}
\definecolor{dialinecolor}{rgb}{0.678431, 0.839216, 0.905882}
\pgfsetfillcolor{dialinecolor}
\pgfpathmoveto{\pgfpoint{51.026526\du}{13.552213\du}}
\pgfpathlineto{\pgfpoint{51.026526\du}{13.552213\du}}
\pgfpathlineto{\pgfpoint{51.026526\du}{13.560246\du}}
\pgfpathlineto{\pgfpoint{51.026891\du}{13.567913\du}}
\pgfpathlineto{\pgfpoint{51.027621\du}{13.575580\du}}
\pgfpathlineto{\pgfpoint{51.028717\du}{13.583977\du}}
\pgfpathlineto{\pgfpoint{51.029812\du}{13.591279\du}}
\pgfpathlineto{\pgfpoint{51.031637\du}{13.598946\du}}
\pgfpathlineto{\pgfpoint{51.033463\du}{13.606613\du}}
\pgfpathlineto{\pgfpoint{51.035653\du}{13.614280\du}}
\pgfpathlineto{\pgfpoint{51.037844\du}{13.621947\du}}
\pgfpathlineto{\pgfpoint{51.040400\du}{13.629614\du}}
\pgfpathlineto{\pgfpoint{51.043320\du}{13.636916\du}}
\pgfpathlineto{\pgfpoint{51.046971\du}{13.644218\du}}
\pgfpathlineto{\pgfpoint{51.050257\du}{13.651520\du}}
\pgfpathlineto{\pgfpoint{51.053908\du}{13.659187\du}}
\pgfpathlineto{\pgfpoint{51.058289\du}{13.666854\du}}
\pgfpathlineto{\pgfpoint{51.061940\du}{13.673791\du}}
\pgfpathlineto{\pgfpoint{51.067052\du}{13.681458\du}}
\pgfpathlineto{\pgfpoint{51.071068\du}{13.688394\du}}
\pgfpathlineto{\pgfpoint{51.076544\du}{13.695696\du}}
\pgfpathlineto{\pgfpoint{51.081290\du}{13.702633\du}}
\pgfpathlineto{\pgfpoint{51.086767\du}{13.710300\du}}
\pgfpathlineto{\pgfpoint{51.092608\du}{13.717237\du}}
\pgfpathlineto{\pgfpoint{51.098450\du}{13.724174\du}}
\pgfpathlineto{\pgfpoint{51.104292\du}{13.731111\du}}
\pgfpathlineto{\pgfpoint{51.111228\du}{13.737683\du}}
\pgfpathlineto{\pgfpoint{51.117800\du}{13.744619\du}}
\pgfpathlineto{\pgfpoint{51.124737\du}{13.751556\du}}
\pgfpathlineto{\pgfpoint{51.131674\du}{13.758128\du}}
\pgfpathlineto{\pgfpoint{51.138611\du}{13.765065\du}}
\pgfpathlineto{\pgfpoint{51.147373\du}{13.771637\du}}
\pgfpathlineto{\pgfpoint{51.154310\du}{13.778573\du}}
\pgfpathlineto{\pgfpoint{51.161977\du}{13.785145\du}}
\pgfpathlineto{\pgfpoint{51.170739\du}{13.791352\du}}
\pgfpathlineto{\pgfpoint{51.179136\du}{13.797924\du}}
\pgfpathlineto{\pgfpoint{51.187899\du}{13.804495\du}}
\pgfpathlineto{\pgfpoint{51.196296\du}{13.810702\du}}
\pgfpathlineto{\pgfpoint{51.206154\du}{13.817274\du}}
\pgfpathlineto{\pgfpoint{51.215281\du}{13.823845\du}}
\pgfpathlineto{\pgfpoint{51.224773\du}{13.830052\du}}
\pgfpathlineto{\pgfpoint{51.234266\du}{13.835894\du}}
\pgfpathlineto{\pgfpoint{51.244124\du}{13.842465\du}}
\pgfpathlineto{\pgfpoint{51.254346\du}{13.849037\du}}
\pgfpathlineto{\pgfpoint{51.264569\du}{13.854879\du}}
\pgfpathlineto{\pgfpoint{51.275157\du}{13.860720\du}}
\pgfpathlineto{\pgfpoint{51.285380\du}{13.866562\du}}
\pgfpathlineto{\pgfpoint{51.296332\du}{13.872768\du}}
\pgfpathlineto{\pgfpoint{51.308016\du}{13.878610\du}}
\pgfpathlineto{\pgfpoint{51.319334\du}{13.884086\du}}
\pgfpathlineto{\pgfpoint{51.331017\du}{13.889928\du}}
\pgfpathlineto{\pgfpoint{51.342335\du}{13.895769\du}}
\pgfpathlineto{\pgfpoint{51.354018\du}{13.901611\du}}
\pgfpathlineto{\pgfpoint{51.366431\du}{13.906722\du}}
\pgfpathlineto{\pgfpoint{51.378479\du}{13.912564\du}}
\pgfpathlineto{\pgfpoint{51.391623\du}{13.918405\du}}
\pgfpathlineto{\pgfpoint{51.403671\du}{13.923882\du}}
\pgfpathlineto{\pgfpoint{51.416449\du}{13.928993\du}}
\pgfpathlineto{\pgfpoint{51.430323\du}{13.934470\du}}
\pgfpathlineto{\pgfpoint{51.442736\du}{13.939581\du}}
\pgfpathlineto{\pgfpoint{51.455880\du}{13.944692\du}}
\pgfpathlineto{\pgfpoint{51.470119\du}{13.950169\du}}
\pgfpathlineto{\pgfpoint{51.483262\du}{13.955280\du}}
\pgfpathlineto{\pgfpoint{51.497501\du}{13.960026\du}}
\pgfpathlineto{\pgfpoint{51.511374\du}{13.965138\du}}
\pgfpathlineto{\pgfpoint{51.525978\du}{13.969884\du}}
\pgfpathlineto{\pgfpoint{51.539852\du}{13.975361\du}}
\pgfpathlineto{\pgfpoint{51.555551\du}{13.980107\du}}
\pgfpathlineto{\pgfpoint{51.569425\du}{13.984488\du}}
\pgfpathlineto{\pgfpoint{51.584029\du}{13.989234\du}}
\pgfpathlineto{\pgfpoint{51.599363\du}{13.993980\du}}
\pgfpathlineto{\pgfpoint{51.615062\du}{13.998727\du}}
\pgfpathlineto{\pgfpoint{51.630031\du}{14.003473\du}}
\pgfpathlineto{\pgfpoint{51.645730\du}{14.007489\du}}
\pgfpathlineto{\pgfpoint{51.677493\du}{14.016616\du}}
\pgfpathlineto{\pgfpoint{51.709257\du}{14.025014\du}}
\pgfpathlineto{\pgfpoint{51.742481\du}{14.033046\du}}
\pgfpathlineto{\pgfpoint{51.774974\du}{14.041443\du}}
\pgfpathlineto{\pgfpoint{51.809658\du}{14.049110\du}}
\pgfpathlineto{\pgfpoint{51.843978\du}{14.056047\du}}
\pgfpathlineto{\pgfpoint{51.878662\du}{14.062984\du}}
\pgfpathlineto{\pgfpoint{51.914441\du}{14.069921\du}}
\pgfpathlineto{\pgfpoint{51.950586\du}{14.076492\du}}
\pgfpathlineto{\pgfpoint{51.987096\du}{14.083064\du}}
\pgfpathlineto{\pgfpoint{52.024335\du}{14.088906\du}}
\pgfpathlineto{\pgfpoint{52.061940\du}{14.094747\du}}
\pgfpathlineto{\pgfpoint{52.099910\du}{14.099859\du}}
\pgfpathlineto{\pgfpoint{52.138611\du}{14.104605\du}}
\pgfpathlineto{\pgfpoint{52.177311\du}{14.109351\du}}
\pgfpathlineto{\pgfpoint{52.216376\du}{14.114097\du}}
\pgfpathlineto{\pgfpoint{52.256537\du}{14.118113\du}}
\pgfpathlineto{\pgfpoint{52.296332\du}{14.122129\du}}
\pgfpathlineto{\pgfpoint{52.337223\du}{14.125050\du}}
\pgfpathlineto{\pgfpoint{52.378114\du}{14.127971\du}}
\pgfpathlineto{\pgfpoint{52.419370\du}{14.130892\du}}
\pgfpathlineto{\pgfpoint{52.461356\du}{14.133082\du}}
\pgfpathlineto{\pgfpoint{52.502977\du}{14.134908\du}}
\pgfpathlineto{\pgfpoint{52.545328\du}{14.136733\du}}
\pgfpathlineto{\pgfpoint{52.587315\du}{14.137829\du}}
\pgfpathlineto{\pgfpoint{52.630761\du}{14.138924\du}}
\pgfpathlineto{\pgfpoint{52.673477\du}{14.139654\du}}
\pgfpathlineto{\pgfpoint{52.717289\du}{14.139654\du}}
\pgfpathlineto{\pgfpoint{52.717289\du}{14.119209\du}}
\pgfpathlineto{\pgfpoint{52.674573\du}{14.119209\du}}
\pgfpathlineto{\pgfpoint{52.631126\du}{14.118478\du}}
\pgfpathlineto{\pgfpoint{52.588775\du}{14.118113\du}}
\pgfpathlineto{\pgfpoint{52.546059\du}{14.116288\du}}
\pgfpathlineto{\pgfpoint{52.503707\du}{14.115193\du}}
\pgfpathlineto{\pgfpoint{52.461721\du}{14.112637\du}}
\pgfpathlineto{\pgfpoint{52.420830\du}{14.110446\du}}
\pgfpathlineto{\pgfpoint{52.379575\du}{14.107526\du}}
\pgfpathlineto{\pgfpoint{52.338684\du}{14.104605\du}}
\pgfpathlineto{\pgfpoint{52.298523\du}{14.100954\du}}
\pgfpathlineto{\pgfpoint{52.258727\du}{14.097668\du}}
\pgfpathlineto{\pgfpoint{52.219297\du}{14.093287\du}}
\pgfpathlineto{\pgfpoint{52.179867\du}{14.089271\du}}
\pgfpathlineto{\pgfpoint{52.140801\du}{14.084524\du}}
\pgfpathlineto{\pgfpoint{52.102466\du}{14.079413\du}}
\pgfpathlineto{\pgfpoint{52.064861\du}{14.074302\du}}
\pgfpathlineto{\pgfpoint{52.027621\du}{14.068460\du}}
\pgfpathlineto{\pgfpoint{51.991112\du}{14.062254\du}}
\pgfpathlineto{\pgfpoint{51.953872\du}{14.056412\du}}
\pgfpathlineto{\pgfpoint{51.918822\du}{14.050205\du}}
\pgfpathlineto{\pgfpoint{51.882678\du}{14.043269\du}}
\pgfpathlineto{\pgfpoint{51.847994\du}{14.035967\du}}
\pgfpathlineto{\pgfpoint{51.813675\du}{14.028665\du}}
\pgfpathlineto{\pgfpoint{51.779721\du}{14.020998\du}}
\pgfpathlineto{\pgfpoint{51.746497\du}{14.013331\du}}
\pgfpathlineto{\pgfpoint{51.715098\du}{14.004933\du}}
\pgfpathlineto{\pgfpoint{51.682605\du}{13.996171\du}}
\pgfpathlineto{\pgfpoint{51.651572\du}{13.988139\du}}
\pgfpathlineto{\pgfpoint{51.635872\du}{13.983393\du}}
\pgfpathlineto{\pgfpoint{51.620173\du}{13.978646\du}}
\pgfpathlineto{\pgfpoint{51.605569\du}{13.974630\du}}
\pgfpathlineto{\pgfpoint{51.590600\du}{13.969884\du}}
\pgfpathlineto{\pgfpoint{51.575266\du}{13.965138\du}}
\pgfpathlineto{\pgfpoint{51.561028\du}{13.960757\du}}
\pgfpathlineto{\pgfpoint{51.546789\du}{13.955280\du}}
\pgfpathlineto{\pgfpoint{51.532550\du}{13.950534\du}}
\pgfpathlineto{\pgfpoint{51.518311\du}{13.946153\du}}
\pgfpathlineto{\pgfpoint{51.504438\du}{13.940676\du}}
\pgfpathlineto{\pgfpoint{51.490564\du}{13.935930\du}}
\pgfpathlineto{\pgfpoint{51.477055\du}{13.930819\du}}
\pgfpathlineto{\pgfpoint{51.463547\du}{13.925342\du}}
\pgfpathlineto{\pgfpoint{51.450768\du}{13.920961\du}}
\pgfpathlineto{\pgfpoint{51.437260\du}{13.915120\du}}
\pgfpathlineto{\pgfpoint{51.424481\du}{13.909643\du}}
\pgfpathlineto{\pgfpoint{51.412068\du}{13.904532\du}}
\pgfpathlineto{\pgfpoint{51.398925\du}{13.899055\du}}
\pgfpathlineto{\pgfpoint{51.386876\du}{13.893944\du}}
\pgfpathlineto{\pgfpoint{51.375558\du}{13.888102\du}}
\pgfpathlineto{\pgfpoint{51.362780\du}{13.882626\du}}
\pgfpathlineto{\pgfpoint{51.351097\du}{13.877515\du}}
\pgfpathlineto{\pgfpoint{51.339414\du}{13.871673\du}}
\pgfpathlineto{\pgfpoint{51.328461\du}{13.865832\du}}
\pgfpathlineto{\pgfpoint{51.316778\du}{13.859990\du}}
\pgfpathlineto{\pgfpoint{51.306920\du}{13.854148\du}}
\pgfpathlineto{\pgfpoint{51.295602\du}{13.849037\du}}
\pgfpathlineto{\pgfpoint{51.285014\du}{13.842465\du}}
\pgfpathlineto{\pgfpoint{51.275157\du}{13.836624\du}}
\pgfpathlineto{\pgfpoint{51.264569\du}{13.830782\du}}
\pgfpathlineto{\pgfpoint{51.254711\du}{13.824941\du}}
\pgfpathlineto{\pgfpoint{51.245219\du}{13.819099\du}}
\pgfpathlineto{\pgfpoint{51.235361\du}{13.812527\du}}
\pgfpathlineto{\pgfpoint{51.226234\du}{13.806686\du}}
\pgfpathlineto{\pgfpoint{51.217837\du}{13.800114\du}}
\pgfpathlineto{\pgfpoint{51.209074\du}{13.794273\du}}
\pgfpathlineto{\pgfpoint{51.199947\du}{13.788066\du}}
\pgfpathlineto{\pgfpoint{51.191185\du}{13.781494\du}}
\pgfpathlineto{\pgfpoint{51.182787\du}{13.775288\du}}
\pgfpathlineto{\pgfpoint{51.175485\du}{13.768716\du}}
\pgfpathlineto{\pgfpoint{51.167453\du}{13.762874\du}}
\pgfpathlineto{\pgfpoint{51.159786\du}{13.756302\du}}
\pgfpathlineto{\pgfpoint{51.152849\du}{13.750096\du}}
\pgfpathlineto{\pgfpoint{51.145547\du}{13.743524\du}}
\pgfpathlineto{\pgfpoint{51.138611\du}{13.736587\du}}
\pgfpathlineto{\pgfpoint{51.132404\du}{13.730016\du}}
\pgfpathlineto{\pgfpoint{51.125832\du}{13.723809\du}}
\pgfpathlineto{\pgfpoint{51.119991\du}{13.717237\du}}
\pgfpathlineto{\pgfpoint{51.113784\du}{13.710300\du}}
\pgfpathlineto{\pgfpoint{51.108673\du}{13.703729\du}}
\pgfpathlineto{\pgfpoint{51.102466\du}{13.697157\du}}
\pgfpathlineto{\pgfpoint{51.098085\du}{13.690220\du}}
\pgfpathlineto{\pgfpoint{51.092608\du}{13.684013\du}}
\pgfpathlineto{\pgfpoint{51.088227\du}{13.676711\du}}
\pgfpathlineto{\pgfpoint{51.083846\du}{13.670505\du}}
\pgfpathlineto{\pgfpoint{51.079465\du}{13.663203\du}}
\pgfpathlineto{\pgfpoint{51.076179\du}{13.656996\du}}
\pgfpathlineto{\pgfpoint{51.072163\du}{13.650059\du}}
\pgfpathlineto{\pgfpoint{51.068147\du}{13.642757\du}}
\pgfpathlineto{\pgfpoint{51.065226\du}{13.635821\du}}
\pgfpathlineto{\pgfpoint{51.062671\du}{13.629614\du}}
\pgfpathlineto{\pgfpoint{51.059385\du}{13.622312\du}}
\pgfpathlineto{\pgfpoint{51.057194\du}{13.615375\du}}
\pgfpathlineto{\pgfpoint{51.055004\du}{13.608438\du}}
\pgfpathlineto{\pgfpoint{51.053543\du}{13.601501\du}}
\pgfpathlineto{\pgfpoint{51.051353\du}{13.594200\du}}
\pgfpathlineto{\pgfpoint{51.050257\du}{13.587993\du}}
\pgfpathlineto{\pgfpoint{51.049162\du}{13.580691\du}}
\pgfpathlineto{\pgfpoint{51.047702\du}{13.573754\du}}
\pgfpathlineto{\pgfpoint{51.047336\du}{13.566817\du}}
\pgfpathlineto{\pgfpoint{51.047336\du}{13.559150\du}}
\pgfpathlineto{\pgfpoint{51.046971\du}{13.552213\du}}
\pgfpathlineto{\pgfpoint{51.046971\du}{13.552213\du}}
\pgfpathlineto{\pgfpoint{51.046971\du}{13.552213\du}}
\pgfpathlineto{\pgfpoint{51.046971\du}{13.551118\du}}
\pgfpathlineto{\pgfpoint{51.046971\du}{13.550023\du}}
\pgfpathlineto{\pgfpoint{51.046606\du}{13.548562\du}}
\pgfpathlineto{\pgfpoint{51.046606\du}{13.548197\du}}
\pgfpathlineto{\pgfpoint{51.045511\du}{13.547102\du}}
\pgfpathlineto{\pgfpoint{51.045146\du}{13.545642\du}}
\pgfpathlineto{\pgfpoint{51.044781\du}{13.545277\du}}
\pgfpathlineto{\pgfpoint{51.043686\du}{13.544546\du}}
\pgfpathlineto{\pgfpoint{51.042225\du}{13.543451\du}}
\pgfpathlineto{\pgfpoint{51.040400\du}{13.542721\du}}
\pgfpathlineto{\pgfpoint{51.038574\du}{13.542356\du}}
\pgfpathlineto{\pgfpoint{51.036749\du}{13.542356\du}}
\pgfpathlineto{\pgfpoint{51.034558\du}{13.542356\du}}
\pgfpathlineto{\pgfpoint{51.033098\du}{13.542721\du}}
\pgfpathlineto{\pgfpoint{51.030907\du}{13.543451\du}}
\pgfpathlineto{\pgfpoint{51.029082\du}{13.544546\du}}
\pgfpathlineto{\pgfpoint{51.028717\du}{13.545277\du}}
\pgfpathlineto{\pgfpoint{51.028351\du}{13.545642\du}}
\pgfpathlineto{\pgfpoint{51.027621\du}{13.547102\du}}
\pgfpathlineto{\pgfpoint{51.026891\du}{13.548197\du}}
\pgfpathlineto{\pgfpoint{51.026891\du}{13.548562\du}}
\pgfpathlineto{\pgfpoint{51.026526\du}{13.550023\du}}
\pgfpathlineto{\pgfpoint{51.026526\du}{13.551118\du}}
\pgfpathlineto{\pgfpoint{51.026526\du}{13.552213\du}}
\pgfusepath{fill}
\pgfsetbuttcap
\pgfsetmiterjoin
\pgfsetdash{}{0pt}
\definecolor{dialinecolor}{rgb}{0.678431, 0.839216, 0.905882}
\pgfsetfillcolor{dialinecolor}
\pgfpathmoveto{\pgfpoint{52.717289\du}{12.964773\du}}
\pgfpathlineto{\pgfpoint{52.717289\du}{12.964773\du}}
\pgfpathlineto{\pgfpoint{52.673477\du}{12.964773\du}}
\pgfpathlineto{\pgfpoint{52.630761\du}{12.965503\du}}
\pgfpathlineto{\pgfpoint{52.587315\du}{12.966598\du}}
\pgfpathlineto{\pgfpoint{52.545328\du}{12.967693\du}}
\pgfpathlineto{\pgfpoint{52.502977\du}{12.969519\du}}
\pgfpathlineto{\pgfpoint{52.461356\du}{12.971710\du}}
\pgfpathlineto{\pgfpoint{52.419370\du}{12.974265\du}}
\pgfpathlineto{\pgfpoint{52.378114\du}{12.976456\du}}
\pgfpathlineto{\pgfpoint{52.337223\du}{12.979377\du}}
\pgfpathlineto{\pgfpoint{52.296332\du}{12.983028\du}}
\pgfpathlineto{\pgfpoint{52.256537\du}{12.986679\du}}
\pgfpathlineto{\pgfpoint{52.216376\du}{12.990695\du}}
\pgfpathlineto{\pgfpoint{52.177311\du}{12.995076\du}}
\pgfpathlineto{\pgfpoint{52.138611\du}{12.999822\du}}
\pgfpathlineto{\pgfpoint{52.099910\du}{13.004933\du}}
\pgfpathlineto{\pgfpoint{52.061940\du}{13.010410\du}}
\pgfpathlineto{\pgfpoint{52.024335\du}{13.015521\du}}
\pgfpathlineto{\pgfpoint{51.987096\du}{13.022093\du}}
\pgfpathlineto{\pgfpoint{51.950586\du}{13.027934\du}}
\pgfpathlineto{\pgfpoint{51.914441\du}{13.034506\du}}
\pgfpathlineto{\pgfpoint{51.878662\du}{13.041443\du}}
\pgfpathlineto{\pgfpoint{51.843978\du}{13.048380\du}}
\pgfpathlineto{\pgfpoint{51.809658\du}{13.056047\du}}
\pgfpathlineto{\pgfpoint{51.774974\du}{13.063714\du}}
\pgfpathlineto{\pgfpoint{51.742481\du}{13.071746\du}}
\pgfpathlineto{\pgfpoint{51.709257\du}{13.080143\du}}
\pgfpathlineto{\pgfpoint{51.677493\du}{13.088175\du}}
\pgfpathlineto{\pgfpoint{51.645730\du}{13.096938\du}}
\pgfpathlineto{\pgfpoint{51.615062\du}{13.106430\du}}
\pgfpathlineto{\pgfpoint{51.584029\du}{13.115193\du}}
\pgfpathlineto{\pgfpoint{51.569425\du}{13.119939\du}}
\pgfpathlineto{\pgfpoint{51.555551\du}{13.125050\du}}
\pgfpathlineto{\pgfpoint{51.539852\du}{13.129796\du}}
\pgfpathlineto{\pgfpoint{51.525978\du}{13.134543\du}}
\pgfpathlineto{\pgfpoint{51.511374\du}{13.139654\du}}
\pgfpathlineto{\pgfpoint{51.497501\du}{13.144400\du}}
\pgfpathlineto{\pgfpoint{51.483262\du}{13.149512\du}}
\pgfpathlineto{\pgfpoint{51.470119\du}{13.154258\du}}
\pgfpathlineto{\pgfpoint{51.455880\du}{13.159734\du}}
\pgfpathlineto{\pgfpoint{51.442736\du}{13.164846\du}}
\pgfpathlineto{\pgfpoint{51.430323\du}{13.169957\du}}
\pgfpathlineto{\pgfpoint{51.416449\du}{13.175799\du}}
\pgfpathlineto{\pgfpoint{51.403671\du}{13.181275\du}}
\pgfpathlineto{\pgfpoint{51.391623\du}{13.186386\du}}
\pgfpathlineto{\pgfpoint{51.378479\du}{13.192228\du}}
\pgfpathlineto{\pgfpoint{51.366431\du}{13.197704\du}}
\pgfpathlineto{\pgfpoint{51.354018\du}{13.203546\du}}
\pgfpathlineto{\pgfpoint{51.342335\du}{13.208657\du}}
\pgfpathlineto{\pgfpoint{51.331017\du}{13.214499\du}}
\pgfpathlineto{\pgfpoint{51.319334\du}{13.220340\du}}
\pgfpathlineto{\pgfpoint{51.308016\du}{13.226182\du}}
\pgfpathlineto{\pgfpoint{51.296332\du}{13.232024\du}}
\pgfpathlineto{\pgfpoint{51.285380\du}{13.237865\du}}
\pgfpathlineto{\pgfpoint{51.275157\du}{13.244437\du}}
\pgfpathlineto{\pgfpoint{51.264569\du}{13.250278\du}}
\pgfpathlineto{\pgfpoint{51.254346\du}{13.256120\du}}
\pgfpathlineto{\pgfpoint{51.244124\du}{13.262692\du}}
\pgfpathlineto{\pgfpoint{51.234266\du}{13.268533\du}}
\pgfpathlineto{\pgfpoint{51.224773\du}{13.274740\du}}
\pgfpathlineto{\pgfpoint{51.215281\du}{13.281312\du}}
\pgfpathlineto{\pgfpoint{51.206154\du}{13.287883\du}}
\pgfpathlineto{\pgfpoint{51.196296\du}{13.293725\du}}
\pgfpathlineto{\pgfpoint{51.187899\du}{13.299932\du}}
\pgfpathlineto{\pgfpoint{51.179136\du}{13.306503\du}}
\pgfpathlineto{\pgfpoint{51.170739\du}{13.312710\du}}
\pgfpathlineto{\pgfpoint{51.161977\du}{13.320012\du}}
\pgfpathlineto{\pgfpoint{51.154310\du}{13.326219\du}}
\pgfpathlineto{\pgfpoint{51.147373\du}{13.332790\du}}
\pgfpathlineto{\pgfpoint{51.138611\du}{13.339727\du}}
\pgfpathlineto{\pgfpoint{51.131674\du}{13.346299\du}}
\pgfpathlineto{\pgfpoint{51.124737\du}{13.353236\du}}
\pgfpathlineto{\pgfpoint{51.117800\du}{13.360173\du}}
\pgfpathlineto{\pgfpoint{51.111228\du}{13.366744\du}}
\pgfpathlineto{\pgfpoint{51.104292\du}{13.373681\du}}
\pgfpathlineto{\pgfpoint{51.098450\du}{13.380618\du}}
\pgfpathlineto{\pgfpoint{51.092608\du}{13.387920\du}}
\pgfpathlineto{\pgfpoint{51.086767\du}{13.394857\du}}
\pgfpathlineto{\pgfpoint{51.081290\du}{13.401794\du}}
\pgfpathlineto{\pgfpoint{51.076544\du}{13.408730\du}}
\pgfpathlineto{\pgfpoint{51.071068\du}{13.416397\du}}
\pgfpathlineto{\pgfpoint{51.067052\du}{13.423334\du}}
\pgfpathlineto{\pgfpoint{51.061940\du}{13.430636\du}}
\pgfpathlineto{\pgfpoint{51.058289\du}{13.437938\du}}
\pgfpathlineto{\pgfpoint{51.053908\du}{13.445240\du}}
\pgfpathlineto{\pgfpoint{51.050257\du}{13.452907\du}}
\pgfpathlineto{\pgfpoint{51.046971\du}{13.460209\du}}
\pgfpathlineto{\pgfpoint{51.043320\du}{13.467876\du}}
\pgfpathlineto{\pgfpoint{51.040400\du}{13.475543\du}}
\pgfpathlineto{\pgfpoint{51.037844\du}{13.482480\du}}
\pgfpathlineto{\pgfpoint{51.035653\du}{13.490147\du}}
\pgfpathlineto{\pgfpoint{51.033463\du}{13.497814\du}}
\pgfpathlineto{\pgfpoint{51.031637\du}{13.505846\du}}
\pgfpathlineto{\pgfpoint{51.029812\du}{13.513513\du}}
\pgfpathlineto{\pgfpoint{51.028717\du}{13.521180\du}}
\pgfpathlineto{\pgfpoint{51.027621\du}{13.528847\du}}
\pgfpathlineto{\pgfpoint{51.026891\du}{13.536879\du}}
\pgfpathlineto{\pgfpoint{51.026526\du}{13.544546\du}}
\pgfpathlineto{\pgfpoint{51.026526\du}{13.552213\du}}
\pgfpathlineto{\pgfpoint{51.046971\du}{13.552213\du}}
\pgfpathlineto{\pgfpoint{51.047336\du}{13.545277\du}}
\pgfpathlineto{\pgfpoint{51.047336\du}{13.538340\du}}
\pgfpathlineto{\pgfpoint{51.047702\du}{13.531038\du}}
\pgfpathlineto{\pgfpoint{51.049162\du}{13.524101\du}}
\pgfpathlineto{\pgfpoint{51.050257\du}{13.517164\du}}
\pgfpathlineto{\pgfpoint{51.051353\du}{13.510227\du}}
\pgfpathlineto{\pgfpoint{51.053543\du}{13.502925\du}}
\pgfpathlineto{\pgfpoint{51.055004\du}{13.496719\du}}
\pgfpathlineto{\pgfpoint{51.057194\du}{13.489417\du}}
\pgfpathlineto{\pgfpoint{51.059385\du}{13.482480\du}}
\pgfpathlineto{\pgfpoint{51.062671\du}{13.475543\du}}
\pgfpathlineto{\pgfpoint{51.065226\du}{13.468606\du}}
\pgfpathlineto{\pgfpoint{51.068147\du}{13.461669\du}}
\pgfpathlineto{\pgfpoint{51.072163\du}{13.455098\du}}
\pgfpathlineto{\pgfpoint{51.076179\du}{13.448161\du}}
\pgfpathlineto{\pgfpoint{51.079465\du}{13.441589\du}}
\pgfpathlineto{\pgfpoint{51.083481\du}{13.434652\du}}
\pgfpathlineto{\pgfpoint{51.088227\du}{13.427715\du}}
\pgfpathlineto{\pgfpoint{51.092608\du}{13.421144\du}}
\pgfpathlineto{\pgfpoint{51.098085\du}{13.414572\du}}
\pgfpathlineto{\pgfpoint{51.102466\du}{13.407635\du}}
\pgfpathlineto{\pgfpoint{51.108673\du}{13.401063\du}}
\pgfpathlineto{\pgfpoint{51.113784\du}{13.394127\du}}
\pgfpathlineto{\pgfpoint{51.119991\du}{13.387920\du}}
\pgfpathlineto{\pgfpoint{51.125832\du}{13.381348\du}}
\pgfpathlineto{\pgfpoint{51.132404\du}{13.374411\du}}
\pgfpathlineto{\pgfpoint{51.138611\du}{13.367840\du}}
\pgfpathlineto{\pgfpoint{51.145547\du}{13.361633\du}}
\pgfpathlineto{\pgfpoint{51.152849\du}{13.355061\du}}
\pgfpathlineto{\pgfpoint{51.159786\du}{13.348489\du}}
\pgfpathlineto{\pgfpoint{51.167453\du}{13.342283\du}}
\pgfpathlineto{\pgfpoint{51.175485\du}{13.335711\du}}
\pgfpathlineto{\pgfpoint{51.182787\du}{13.329139\du}}
\pgfpathlineto{\pgfpoint{51.191185\du}{13.322933\du}}
\pgfpathlineto{\pgfpoint{51.199947\du}{13.317091\du}}
\pgfpathlineto{\pgfpoint{51.209074\du}{13.310154\du}}
\pgfpathlineto{\pgfpoint{51.217837\du}{13.304678\du}}
\pgfpathlineto{\pgfpoint{51.226234\du}{13.298106\du}}
\pgfpathlineto{\pgfpoint{51.235361\du}{13.291899\du}}
\pgfpathlineto{\pgfpoint{51.245219\du}{13.286058\du}}
\pgfpathlineto{\pgfpoint{51.254711\du}{13.280216\du}}
\pgfpathlineto{\pgfpoint{51.264569\du}{13.273645\du}}
\pgfpathlineto{\pgfpoint{51.275157\du}{13.267803\du}}
\pgfpathlineto{\pgfpoint{51.285014\du}{13.261961\du}}
\pgfpathlineto{\pgfpoint{51.295602\du}{13.256120\du}}
\pgfpathlineto{\pgfpoint{51.306920\du}{13.250278\du}}
\pgfpathlineto{\pgfpoint{51.316778\du}{13.244437\du}}
\pgfpathlineto{\pgfpoint{51.328461\du}{13.238595\du}}
\pgfpathlineto{\pgfpoint{51.339414\du}{13.233484\du}}
\pgfpathlineto{\pgfpoint{51.351097\du}{13.227277\du}}
\pgfpathlineto{\pgfpoint{51.362780\du}{13.221436\du}}
\pgfpathlineto{\pgfpoint{51.375558\du}{13.216324\du}}
\pgfpathlineto{\pgfpoint{51.386876\du}{13.211213\du}}
\pgfpathlineto{\pgfpoint{51.398925\du}{13.205371\du}}
\pgfpathlineto{\pgfpoint{51.412068\du}{13.199895\du}}
\pgfpathlineto{\pgfpoint{51.424481\du}{13.194784\du}}
\pgfpathlineto{\pgfpoint{51.437260\du}{13.189307\du}}
\pgfpathlineto{\pgfpoint{51.450768\du}{13.184196\du}}
\pgfpathlineto{\pgfpoint{51.463547\du}{13.178719\du}}
\pgfpathlineto{\pgfpoint{51.477055\du}{13.173243\du}}
\pgfpathlineto{\pgfpoint{51.490564\du}{13.168862\du}}
\pgfpathlineto{\pgfpoint{51.504438\du}{13.163750\du}}
\pgfpathlineto{\pgfpoint{51.518311\du}{13.159004\du}}
\pgfpathlineto{\pgfpoint{51.532550\du}{13.153893\du}}
\pgfpathlineto{\pgfpoint{51.546789\du}{13.149147\du}}
\pgfpathlineto{\pgfpoint{51.561028\du}{13.144400\du}}
\pgfpathlineto{\pgfpoint{51.575266\du}{13.139654\du}}
\pgfpathlineto{\pgfpoint{51.590600\du}{13.134908\du}}
\pgfpathlineto{\pgfpoint{51.620173\du}{13.125780\du}}
\pgfpathlineto{\pgfpoint{51.651572\du}{13.117018\du}}
\pgfpathlineto{\pgfpoint{51.682605\du}{13.108256\du}}
\pgfpathlineto{\pgfpoint{51.715098\du}{13.099859\du}}
\pgfpathlineto{\pgfpoint{51.746497\du}{13.091826\du}}
\pgfpathlineto{\pgfpoint{51.779721\du}{13.083429\du}}
\pgfpathlineto{\pgfpoint{51.813675\du}{13.075762\du}}
\pgfpathlineto{\pgfpoint{51.847994\du}{13.068825\du}}
\pgfpathlineto{\pgfpoint{51.882678\du}{13.061888\du}}
\pgfpathlineto{\pgfpoint{51.918822\du}{13.054587\du}}
\pgfpathlineto{\pgfpoint{51.953872\du}{13.048380\du}}
\pgfpathlineto{\pgfpoint{51.991112\du}{13.041443\du}}
\pgfpathlineto{\pgfpoint{52.027621\du}{13.035967\du}}
\pgfpathlineto{\pgfpoint{52.064861\du}{13.030125\du}}
\pgfpathlineto{\pgfpoint{52.102466\du}{13.025014\du}}
\pgfpathlineto{\pgfpoint{52.140801\du}{13.020267\du}}
\pgfpathlineto{\pgfpoint{52.179867\du}{13.015521\du}}
\pgfpathlineto{\pgfpoint{52.219297\du}{13.011140\du}}
\pgfpathlineto{\pgfpoint{52.258727\du}{13.007489\du}}
\pgfpathlineto{\pgfpoint{52.298523\du}{13.003473\du}}
\pgfpathlineto{\pgfpoint{52.338684\du}{13.000552\du}}
\pgfpathlineto{\pgfpoint{52.379575\du}{12.996901\du}}
\pgfpathlineto{\pgfpoint{52.420830\du}{12.993980\du}}
\pgfpathlineto{\pgfpoint{52.461721\du}{12.991790\du}}
\pgfpathlineto{\pgfpoint{52.503707\du}{12.989964\du}}
\pgfpathlineto{\pgfpoint{52.546059\du}{12.988139\du}}
\pgfpathlineto{\pgfpoint{52.588775\du}{12.986679\du}}
\pgfpathlineto{\pgfpoint{52.631126\du}{12.986313\du}}
\pgfpathlineto{\pgfpoint{52.674573\du}{12.985948\du}}
\pgfpathlineto{\pgfpoint{52.717289\du}{12.985218\du}}
\pgfpathlineto{\pgfpoint{52.717289\du}{12.985218\du}}
\pgfpathlineto{\pgfpoint{52.717289\du}{12.985218\du}}
\pgfpathlineto{\pgfpoint{52.718384\du}{12.985218\du}}
\pgfpathlineto{\pgfpoint{52.719480\du}{12.985218\du}}
\pgfpathlineto{\pgfpoint{52.720940\du}{12.984488\du}}
\pgfpathlineto{\pgfpoint{52.722035\du}{12.984488\du}}
\pgfpathlineto{\pgfpoint{52.722400\du}{12.984123\du}}
\pgfpathlineto{\pgfpoint{52.723496\du}{12.983393\du}}
\pgfpathlineto{\pgfpoint{52.724226\du}{12.983028\du}}
\pgfpathlineto{\pgfpoint{52.725321\du}{12.982297\du}}
\pgfpathlineto{\pgfpoint{52.726416\du}{12.980472\du}}
\pgfpathlineto{\pgfpoint{52.727147\du}{12.978646\du}}
\pgfpathlineto{\pgfpoint{52.727147\du}{12.977186\du}}
\pgfpathlineto{\pgfpoint{52.727877\du}{12.975361\du}}
\pgfpathlineto{\pgfpoint{52.727147\du}{12.973535\du}}
\pgfpathlineto{\pgfpoint{52.727147\du}{12.971344\du}}
\pgfpathlineto{\pgfpoint{52.726416\du}{12.969519\du}}
\pgfpathlineto{\pgfpoint{52.725321\du}{12.967693\du}}
\pgfpathlineto{\pgfpoint{52.724226\du}{12.966963\du}}
\pgfpathlineto{\pgfpoint{52.723496\du}{12.966598\du}}
\pgfpathlineto{\pgfpoint{52.722400\du}{12.965868\du}}
\pgfpathlineto{\pgfpoint{52.722035\du}{12.965503\du}}
\pgfpathlineto{\pgfpoint{52.720940\du}{12.965503\du}}
\pgfpathlineto{\pgfpoint{52.719480\du}{12.964773\du}}
\pgfpathlineto{\pgfpoint{52.718384\du}{12.964773\du}}
\pgfpathlineto{\pgfpoint{52.717289\du}{12.964773\du}}
\pgfusepath{fill}
\pgfsetbuttcap
\pgfsetmiterjoin
\pgfsetdash{}{0pt}
\definecolor{dialinecolor}{rgb}{0.678431, 0.839216, 0.905882}
\pgfsetfillcolor{dialinecolor}
\pgfpathmoveto{\pgfpoint{54.407687\du}{13.552213\du}}
\pgfpathlineto{\pgfpoint{54.407687\du}{13.544546\du}}
\pgfpathlineto{\pgfpoint{54.406957\du}{13.536879\du}}
\pgfpathlineto{\pgfpoint{54.406227\du}{13.528847\du}}
\pgfpathlineto{\pgfpoint{54.405496\du}{13.521180\du}}
\pgfpathlineto{\pgfpoint{54.404401\du}{13.513513\du}}
\pgfpathlineto{\pgfpoint{54.402211\du}{13.505846\du}}
\pgfpathlineto{\pgfpoint{54.401115\du}{13.497814\du}}
\pgfpathlineto{\pgfpoint{54.398194\du}{13.490147\du}}
\pgfpathlineto{\pgfpoint{54.396369\du}{13.482480\du}}
\pgfpathlineto{\pgfpoint{54.393448\du}{13.475543\du}}
\pgfpathlineto{\pgfpoint{54.390893\du}{13.467876\du}}
\pgfpathlineto{\pgfpoint{54.387607\du}{13.460209\du}}
\pgfpathlineto{\pgfpoint{54.383591\du}{13.452907\du}}
\pgfpathlineto{\pgfpoint{54.380305\du}{13.445240\du}}
\pgfpathlineto{\pgfpoint{54.376289\du}{13.437938\du}}
\pgfpathlineto{\pgfpoint{54.372273\du}{13.430636\du}}
\pgfpathlineto{\pgfpoint{54.367891\du}{13.423334\du}}
\pgfpathlineto{\pgfpoint{54.363145\du}{13.416397\du}}
\pgfpathlineto{\pgfpoint{54.357669\du}{13.408730\du}}
\pgfpathlineto{\pgfpoint{54.352922\du}{13.401794\du}}
\pgfpathlineto{\pgfpoint{54.347081\du}{13.394857\du}}
\pgfpathlineto{\pgfpoint{54.341604\du}{13.387920\du}}
\pgfpathlineto{\pgfpoint{54.335763\du}{13.380618\du}}
\pgfpathlineto{\pgfpoint{54.329556\du}{13.373681\du}}
\pgfpathlineto{\pgfpoint{54.322985\du}{13.366744\du}}
\pgfpathlineto{\pgfpoint{54.316048\du}{13.360173\du}}
\pgfpathlineto{\pgfpoint{54.309476\du}{13.353236\du}}
\pgfpathlineto{\pgfpoint{54.302174\du}{13.346299\du}}
\pgfpathlineto{\pgfpoint{54.295602\du}{13.339727\du}}
\pgfpathlineto{\pgfpoint{54.287570\du}{13.332790\du}}
\pgfpathlineto{\pgfpoint{54.279538\du}{13.326219\du}}
\pgfpathlineto{\pgfpoint{54.272236\du}{13.320012\du}}
\pgfpathlineto{\pgfpoint{54.263474\du}{13.312710\du}}
\pgfpathlineto{\pgfpoint{54.255442\du}{13.306503\du}}
\pgfpathlineto{\pgfpoint{54.246314\du}{13.299932\du}}
\pgfpathlineto{\pgfpoint{54.237552\du}{13.293725\du}}
\pgfpathlineto{\pgfpoint{54.228059\du}{13.287883\du}}
\pgfpathlineto{\pgfpoint{54.218932\du}{13.281312\du}}
\pgfpathlineto{\pgfpoint{54.209805\du}{13.274740\du}}
\pgfpathlineto{\pgfpoint{54.199947\du}{13.268533\du}}
\pgfpathlineto{\pgfpoint{54.190089\du}{13.262692\du}}
\pgfpathlineto{\pgfpoint{54.179867\du}{13.256120\du}}
\pgfpathlineto{\pgfpoint{54.169279\du}{13.250278\du}}
\pgfpathlineto{\pgfpoint{54.159421\du}{13.244437\du}}
\pgfpathlineto{\pgfpoint{54.148833\du}{13.237865\du}}
\pgfpathlineto{\pgfpoint{54.137515\du}{13.232024\du}}
\pgfpathlineto{\pgfpoint{54.126197\du}{13.226182\du}}
\pgfpathlineto{\pgfpoint{54.114514\du}{13.220340\du}}
\pgfpathlineto{\pgfpoint{54.103561\du}{13.214499\du}}
\pgfpathlineto{\pgfpoint{54.091513\du}{13.208657\du}}
\pgfpathlineto{\pgfpoint{54.080195\du}{13.203546\du}}
\pgfpathlineto{\pgfpoint{54.067782\du}{13.197704\du}}
\pgfpathlineto{\pgfpoint{54.055369\du}{13.192228\du}}
\pgfpathlineto{\pgfpoint{54.042590\du}{13.186386\du}}
\pgfpathlineto{\pgfpoint{54.030542\du}{13.181275\du}}
\pgfpathlineto{\pgfpoint{54.017764\du}{13.175799\du}}
\pgfpathlineto{\pgfpoint{54.004255\du}{13.169957\du}}
\pgfpathlineto{\pgfpoint{53.991112\du}{13.164846\du}}
\pgfpathlineto{\pgfpoint{53.977968\du}{13.159734\du}}
\pgfpathlineto{\pgfpoint{53.963729\du}{13.154258\du}}
\pgfpathlineto{\pgfpoint{53.950586\du}{13.149512\du}}
\pgfpathlineto{\pgfpoint{53.936347\du}{13.144400\du}}
\pgfpathlineto{\pgfpoint{53.922473\du}{13.139654\du}}
\pgfpathlineto{\pgfpoint{53.908235\du}{13.134543\du}}
\pgfpathlineto{\pgfpoint{53.894361\du}{13.129796\du}}
\pgfpathlineto{\pgfpoint{53.879392\du}{13.125050\du}}
\pgfpathlineto{\pgfpoint{53.864423\du}{13.119939\du}}
\pgfpathlineto{\pgfpoint{53.850184\du}{13.115193\du}}
\pgfpathlineto{\pgfpoint{53.819516\du}{13.106430\du}}
\pgfpathlineto{\pgfpoint{53.788848\du}{13.096938\du}}
\pgfpathlineto{\pgfpoint{53.757450\du}{13.088175\du}}
\pgfpathlineto{\pgfpoint{53.725686\du}{13.080143\du}}
\pgfpathlineto{\pgfpoint{53.692828\du}{13.071746\du}}
\pgfpathlineto{\pgfpoint{53.659239\du}{13.063714\du}}
\pgfpathlineto{\pgfpoint{53.625285\du}{13.056047\du}}
\pgfpathlineto{\pgfpoint{53.590600\du}{13.048380\du}}
\pgfpathlineto{\pgfpoint{53.555916\du}{13.041443\du}}
\pgfpathlineto{\pgfpoint{53.520502\du}{13.034506\du}}
\pgfpathlineto{\pgfpoint{53.483992\du}{13.027934\du}}
\pgfpathlineto{\pgfpoint{53.447483\du}{13.022093\du}}
\pgfpathlineto{\pgfpoint{53.410608\du}{13.015521\du}}
\pgfpathlineto{\pgfpoint{53.373368\du}{13.010410\du}}
\pgfpathlineto{\pgfpoint{53.334303\du}{13.004933\du}}
\pgfpathlineto{\pgfpoint{53.296332\du}{12.999822\du}}
\pgfpathlineto{\pgfpoint{53.256902\du}{12.995076\du}}
\pgfpathlineto{\pgfpoint{53.218202\du}{12.990695\du}}
\pgfpathlineto{\pgfpoint{53.178041\du}{12.986679\du}}
\pgfpathlineto{\pgfpoint{53.138246\du}{12.983028\du}}
\pgfpathlineto{\pgfpoint{53.097355\du}{12.979377\du}}
\pgfpathlineto{\pgfpoint{53.056829\du}{12.976456\du}}
\pgfpathlineto{\pgfpoint{53.015208\du}{12.974265\du}}
\pgfpathlineto{\pgfpoint{52.973587\du}{12.971710\du}}
\pgfpathlineto{\pgfpoint{52.931966\du}{12.969519\du}}
\pgfpathlineto{\pgfpoint{52.889250\du}{12.967693\du}}
\pgfpathlineto{\pgfpoint{52.847263\du}{12.966598\du}}
\pgfpathlineto{\pgfpoint{52.804182\du}{12.965503\du}}
\pgfpathlineto{\pgfpoint{52.760736\du}{12.964773\du}}
\pgfpathlineto{\pgfpoint{52.717289\du}{12.964773\du}}
\pgfpathlineto{\pgfpoint{52.717289\du}{12.985218\du}}
\pgfpathlineto{\pgfpoint{52.760370\du}{12.985948\du}}
\pgfpathlineto{\pgfpoint{52.803817\du}{12.986313\du}}
\pgfpathlineto{\pgfpoint{52.845803\du}{12.986679\du}}
\pgfpathlineto{\pgfpoint{52.888519\du}{12.988139\du}}
\pgfpathlineto{\pgfpoint{52.931236\du}{12.989964\du}}
\pgfpathlineto{\pgfpoint{52.972857\du}{12.991790\du}}
\pgfpathlineto{\pgfpoint{53.014113\du}{12.993980\du}}
\pgfpathlineto{\pgfpoint{53.055004\du}{12.996901\du}}
\pgfpathlineto{\pgfpoint{53.096259\du}{13.000552\du}}
\pgfpathlineto{\pgfpoint{53.136420\du}{13.003473\du}}
\pgfpathlineto{\pgfpoint{53.176581\du}{13.007489\du}}
\pgfpathlineto{\pgfpoint{53.215646\du}{13.011140\du}}
\pgfpathlineto{\pgfpoint{53.255077\du}{13.015521\du}}
\pgfpathlineto{\pgfpoint{53.293777\du}{13.020267\du}}
\pgfpathlineto{\pgfpoint{53.332477\du}{13.025014\du}}
\pgfpathlineto{\pgfpoint{53.370082\du}{13.030125\du}}
\pgfpathlineto{\pgfpoint{53.407322\du}{13.035967\du}}
\pgfpathlineto{\pgfpoint{53.443832\du}{13.041443\du}}
\pgfpathlineto{\pgfpoint{53.480706\du}{13.048380\du}}
\pgfpathlineto{\pgfpoint{53.516121\du}{13.054587\du}}
\pgfpathlineto{\pgfpoint{53.551900\du}{13.061888\du}}
\pgfpathlineto{\pgfpoint{53.586584\du}{13.068825\du}}
\pgfpathlineto{\pgfpoint{53.620538\du}{13.075762\du}}
\pgfpathlineto{\pgfpoint{53.654857\du}{13.083429\du}}
\pgfpathlineto{\pgfpoint{53.688081\du}{13.091826\du}}
\pgfpathlineto{\pgfpoint{53.720210\du}{13.099859\du}}
\pgfpathlineto{\pgfpoint{53.752338\du}{13.108256\du}}
\pgfpathlineto{\pgfpoint{53.783737\du}{13.117018\du}}
\pgfpathlineto{\pgfpoint{53.814405\du}{13.125780\du}}
\pgfpathlineto{\pgfpoint{53.844343\du}{13.134908\du}}
\pgfpathlineto{\pgfpoint{53.858581\du}{13.139654\du}}
\pgfpathlineto{\pgfpoint{53.873185\du}{13.144400\du}}
\pgfpathlineto{\pgfpoint{53.887059\du}{13.149147\du}}
\pgfpathlineto{\pgfpoint{53.901663\du}{13.153893\du}}
\pgfpathlineto{\pgfpoint{53.915902\du}{13.159004\du}}
\pgfpathlineto{\pgfpoint{53.929775\du}{13.163750\du}}
\pgfpathlineto{\pgfpoint{53.944014\du}{13.168862\du}}
\pgfpathlineto{\pgfpoint{53.957158\du}{13.173243\du}}
\pgfpathlineto{\pgfpoint{53.970666\du}{13.178719\du}}
\pgfpathlineto{\pgfpoint{53.983810\du}{13.184196\du}}
\pgfpathlineto{\pgfpoint{53.996588\du}{13.189307\du}}
\pgfpathlineto{\pgfpoint{54.009366\du}{13.194784\du}}
\pgfpathlineto{\pgfpoint{54.022510\du}{13.199895\du}}
\pgfpathlineto{\pgfpoint{54.034558\du}{13.205371\du}}
\pgfpathlineto{\pgfpoint{54.046971\du}{13.211213\du}}
\pgfpathlineto{\pgfpoint{54.059020\du}{13.216324\du}}
\pgfpathlineto{\pgfpoint{54.071433\du}{13.221436\du}}
\pgfpathlineto{\pgfpoint{54.082751\du}{13.227277\du}}
\pgfpathlineto{\pgfpoint{54.094799\du}{13.233484\du}}
\pgfpathlineto{\pgfpoint{54.105387\du}{13.238595\du}}
\pgfpathlineto{\pgfpoint{54.117435\du}{13.244437\du}}
\pgfpathlineto{\pgfpoint{54.127658\du}{13.250278\du}}
\pgfpathlineto{\pgfpoint{54.138611\du}{13.256120\du}}
\pgfpathlineto{\pgfpoint{54.149564\du}{13.261961\du}}
\pgfpathlineto{\pgfpoint{54.159421\du}{13.267803\du}}
\pgfpathlineto{\pgfpoint{54.169279\du}{13.273645\du}}
\pgfpathlineto{\pgfpoint{54.179501\du}{13.280216\du}}
\pgfpathlineto{\pgfpoint{54.188629\du}{13.286058\du}}
\pgfpathlineto{\pgfpoint{54.198852\du}{13.291899\du}}
\pgfpathlineto{\pgfpoint{54.208344\du}{13.298106\du}}
\pgfpathlineto{\pgfpoint{54.217106\du}{13.304678\du}}
\pgfpathlineto{\pgfpoint{54.225139\du}{13.310154\du}}
\pgfpathlineto{\pgfpoint{54.233901\du}{13.317091\du}}
\pgfpathlineto{\pgfpoint{54.242298\du}{13.322933\du}}
\pgfpathlineto{\pgfpoint{54.251060\du}{13.329139\du}}
\pgfpathlineto{\pgfpoint{54.258727\du}{13.335711\du}}
\pgfpathlineto{\pgfpoint{54.266760\du}{13.342283\du}}
\pgfpathlineto{\pgfpoint{54.274062\du}{13.348489\du}}
\pgfpathlineto{\pgfpoint{54.281729\du}{13.355061\du}}
\pgfpathlineto{\pgfpoint{54.288300\du}{13.361633\du}}
\pgfpathlineto{\pgfpoint{54.295602\du}{13.367840\du}}
\pgfpathlineto{\pgfpoint{54.301444\du}{13.374411\du}}
\pgfpathlineto{\pgfpoint{54.308381\du}{13.381348\du}}
\pgfpathlineto{\pgfpoint{54.314952\du}{13.387920\du}}
\pgfpathlineto{\pgfpoint{54.320064\du}{13.394127\du}}
\pgfpathlineto{\pgfpoint{54.325540\du}{13.401063\du}}
\pgfpathlineto{\pgfpoint{54.331747\du}{13.407635\du}}
\pgfpathlineto{\pgfpoint{54.336493\du}{13.414572\du}}
\pgfpathlineto{\pgfpoint{54.341604\du}{13.421144\du}}
\pgfpathlineto{\pgfpoint{54.346351\du}{13.427715\du}}
\pgfpathlineto{\pgfpoint{54.350732\du}{13.434652\du}}
\pgfpathlineto{\pgfpoint{54.355113\du}{13.441589\du}}
\pgfpathlineto{\pgfpoint{54.358034\du}{13.448161\du}}
\pgfpathlineto{\pgfpoint{54.362050\du}{13.455098\du}}
\pgfpathlineto{\pgfpoint{54.364971\du}{13.461669\du}}
\pgfpathlineto{\pgfpoint{54.368987\du}{13.468606\du}}
\pgfpathlineto{\pgfpoint{54.371542\du}{13.475543\du}}
\pgfpathlineto{\pgfpoint{54.374463\du}{13.482480\du}}
\pgfpathlineto{\pgfpoint{54.377019\du}{13.489417\du}}
\pgfpathlineto{\pgfpoint{54.379209\du}{13.496719\du}}
\pgfpathlineto{\pgfpoint{54.380670\du}{13.502925\du}}
\pgfpathlineto{\pgfpoint{54.382860\du}{13.510227\du}}
\pgfpathlineto{\pgfpoint{54.383591\du}{13.517164\du}}
\pgfpathlineto{\pgfpoint{54.384686\du}{13.524101\du}}
\pgfpathlineto{\pgfpoint{54.386511\du}{13.531038\du}}
\pgfpathlineto{\pgfpoint{54.386876\du}{13.538340\du}}
\pgfpathlineto{\pgfpoint{54.386876\du}{13.545277\du}}
\pgfpathlineto{\pgfpoint{54.387607\du}{13.552213\du}}
\pgfpathlineto{\pgfpoint{54.407687\du}{13.552213\du}}
\pgfusepath{fill}
\pgfsetbuttcap
\pgfsetmiterjoin
\pgfsetdash{}{0pt}
\definecolor{dialinecolor}{rgb}{0.074510, 0.082353, 0.086275}
\pgfsetfillcolor{dialinecolor}
\pgfpathmoveto{\pgfpoint{52.760370\du}{13.424795\du}}
\pgfpathlineto{\pgfpoint{53.008271\du}{13.507306\du}}
\pgfpathlineto{\pgfpoint{53.593521\du}{13.272914\du}}
\pgfpathlineto{\pgfpoint{53.866248\du}{13.340457\du}}
\pgfpathlineto{\pgfpoint{53.722400\du}{13.131987\du}}
\pgfpathlineto{\pgfpoint{53.018129\du}{13.131987\du}}
\pgfpathlineto{\pgfpoint{53.312397\du}{13.204641\du}}
\pgfpathlineto{\pgfpoint{52.760370\du}{13.424795\du}}
\pgfusepath{fill}
\pgfsetbuttcap
\pgfsetmiterjoin
\pgfsetdash{}{0pt}
\definecolor{dialinecolor}{rgb}{0.074510, 0.082353, 0.086275}
\pgfsetfillcolor{dialinecolor}
\pgfpathmoveto{\pgfpoint{52.658508\du}{13.662838\du}}
\pgfpathlineto{\pgfpoint{52.410608\du}{13.580691\du}}
\pgfpathlineto{\pgfpoint{51.825358\du}{13.814353\du}}
\pgfpathlineto{\pgfpoint{51.552265\du}{13.747540\du}}
\pgfpathlineto{\pgfpoint{51.696113\du}{13.955280\du}}
\pgfpathlineto{\pgfpoint{52.401480\du}{13.955280\du}}
\pgfpathlineto{\pgfpoint{52.106482\du}{13.883356\du}}
\pgfpathlineto{\pgfpoint{52.658508\du}{13.662838\du}}
\pgfusepath{fill}
\pgfsetbuttcap
\pgfsetmiterjoin
\pgfsetdash{}{0pt}
\definecolor{dialinecolor}{rgb}{0.074510, 0.082353, 0.086275}
\pgfsetfillcolor{dialinecolor}
\pgfpathmoveto{\pgfpoint{51.612506\du}{13.203911\du}}
\pgfpathlineto{\pgfpoint{51.860042\du}{13.122129\du}}
\pgfpathlineto{\pgfpoint{52.445292\du}{13.355426\du}}
\pgfpathlineto{\pgfpoint{52.718384\du}{13.288979\du}}
\pgfpathlineto{\pgfpoint{52.574536\du}{13.496719\du}}
\pgfpathlineto{\pgfpoint{51.869534\du}{13.496719\du}}
\pgfpathlineto{\pgfpoint{52.164533\du}{13.424795\du}}
\pgfpathlineto{\pgfpoint{51.612506\du}{13.203911\du}}
\pgfusepath{fill}
\pgfsetbuttcap
\pgfsetmiterjoin
\pgfsetdash{}{0pt}
\definecolor{dialinecolor}{rgb}{0.074510, 0.082353, 0.086275}
\pgfsetfillcolor{dialinecolor}
\pgfpathmoveto{\pgfpoint{53.830834\du}{13.899055\du}}
\pgfpathlineto{\pgfpoint{53.583299\du}{13.981202\du}}
\pgfpathlineto{\pgfpoint{52.998048\du}{13.747540\du}}
\pgfpathlineto{\pgfpoint{52.724226\du}{13.814353\du}}
\pgfpathlineto{\pgfpoint{52.868804\du}{13.606613\du}}
\pgfpathlineto{\pgfpoint{53.574171\du}{13.606613\du}}
\pgfpathlineto{\pgfpoint{53.278808\du}{13.678537\du}}
\pgfpathlineto{\pgfpoint{53.830834\du}{13.899055\du}}
\pgfusepath{fill}
\pgfsetbuttcap
\pgfsetmiterjoin
\pgfsetdash{}{0pt}
\definecolor{dialinecolor}{rgb}{1.000000, 1.000000, 1.000000}
\pgfsetfillcolor{dialinecolor}
\pgfpathmoveto{\pgfpoint{52.781181\du}{13.445240\du}}
\pgfpathlineto{\pgfpoint{53.028717\du}{13.527752\du}}
\pgfpathlineto{\pgfpoint{53.613967\du}{13.293725\du}}
\pgfpathlineto{\pgfpoint{53.886329\du}{13.360903\du}}
\pgfpathlineto{\pgfpoint{53.743576\du}{13.152432\du}}
\pgfpathlineto{\pgfpoint{53.038209\du}{13.152432\du}}
\pgfpathlineto{\pgfpoint{53.333207\du}{13.225087\du}}
\pgfpathlineto{\pgfpoint{52.781181\du}{13.445240\du}}
\pgfusepath{fill}
\pgfsetbuttcap
\pgfsetmiterjoin
\pgfsetdash{}{0pt}
\definecolor{dialinecolor}{rgb}{1.000000, 1.000000, 1.000000}
\pgfsetfillcolor{dialinecolor}
\pgfpathmoveto{\pgfpoint{52.679319\du}{13.684013\du}}
\pgfpathlineto{\pgfpoint{52.430688\du}{13.601501\du}}
\pgfpathlineto{\pgfpoint{51.845803\du}{13.835528\du}}
\pgfpathlineto{\pgfpoint{51.572346\du}{13.767986\du}}
\pgfpathlineto{\pgfpoint{51.717289\du}{13.976456\du}}
\pgfpathlineto{\pgfpoint{52.421926\du}{13.976456\du}}
\pgfpathlineto{\pgfpoint{52.127293\du}{13.903802\du}}
\pgfpathlineto{\pgfpoint{52.679319\du}{13.684013\du}}
\pgfusepath{fill}
\pgfsetbuttcap
\pgfsetmiterjoin
\pgfsetdash{}{0pt}
\definecolor{dialinecolor}{rgb}{1.000000, 1.000000, 1.000000}
\pgfsetfillcolor{dialinecolor}
\pgfpathmoveto{\pgfpoint{51.632952\du}{13.224357\du}}
\pgfpathlineto{\pgfpoint{51.880487\du}{13.142575\du}}
\pgfpathlineto{\pgfpoint{52.466102\du}{13.376602\du}}
\pgfpathlineto{\pgfpoint{52.739195\du}{13.309789\du}}
\pgfpathlineto{\pgfpoint{52.594251\du}{13.517164\du}}
\pgfpathlineto{\pgfpoint{51.889980\du}{13.517164\du}}
\pgfpathlineto{\pgfpoint{52.184613\du}{13.445240\du}}
\pgfpathlineto{\pgfpoint{51.632952\du}{13.224357\du}}
\pgfusepath{fill}
\pgfsetbuttcap
\pgfsetmiterjoin
\pgfsetdash{}{0pt}
\definecolor{dialinecolor}{rgb}{1.000000, 1.000000, 1.000000}
\pgfsetfillcolor{dialinecolor}
\pgfpathmoveto{\pgfpoint{53.850914\du}{13.919501\du}}
\pgfpathlineto{\pgfpoint{53.603379\du}{14.001647\du}}
\pgfpathlineto{\pgfpoint{53.018494\du}{13.767986\du}}
\pgfpathlineto{\pgfpoint{52.745036\du}{13.834798\du}}
\pgfpathlineto{\pgfpoint{52.889250\du}{13.627058\du}}
\pgfpathlineto{\pgfpoint{53.594251\du}{13.627058\du}}
\pgfpathlineto{\pgfpoint{53.299983\du}{13.698982\du}}
\pgfpathlineto{\pgfpoint{53.850914\du}{13.919501\du}}
\pgfusepath{fill}
\pgfsetbuttcap
\pgfsetmiterjoin
\pgfsetdash{}{0pt}
\definecolor{dialinecolor}{rgb}{0.678431, 0.839216, 0.905882}
\pgfsetfillcolor{dialinecolor}
\pgfpathmoveto{\pgfpoint{51.046971\du}{13.562801\du}}
\pgfpathlineto{\pgfpoint{51.046971\du}{13.552213\du}}
\pgfpathlineto{\pgfpoint{51.026526\du}{13.552213\du}}
\pgfpathlineto{\pgfpoint{51.026526\du}{13.562801\du}}
\pgfpathlineto{\pgfpoint{51.046971\du}{13.562801\du}}
\pgfusepath{fill}
\pgfsetbuttcap
\pgfsetmiterjoin
\pgfsetdash{}{0pt}
\definecolor{dialinecolor}{rgb}{0.678431, 0.839216, 0.905882}
\pgfsetfillcolor{dialinecolor}
\pgfpathmoveto{\pgfpoint{51.046971\du}{14.392301\du}}
\pgfpathlineto{\pgfpoint{51.046971\du}{13.562801\du}}
\pgfpathlineto{\pgfpoint{51.026526\du}{13.562801\du}}
\pgfpathlineto{\pgfpoint{51.026526\du}{14.392301\du}}
\pgfpathlineto{\pgfpoint{51.046971\du}{14.392301\du}}
\pgfusepath{fill}
\pgfsetbuttcap
\pgfsetmiterjoin
\pgfsetdash{}{0pt}
\definecolor{dialinecolor}{rgb}{0.678431, 0.839216, 0.905882}
\pgfsetfillcolor{dialinecolor}
\pgfpathmoveto{\pgfpoint{51.026526\du}{14.392301\du}}
\pgfpathlineto{\pgfpoint{51.026526\du}{14.402889\du}}
\pgfpathlineto{\pgfpoint{51.046971\du}{14.402889\du}}
\pgfpathlineto{\pgfpoint{51.046971\du}{14.392301\du}}
\pgfpathlineto{\pgfpoint{51.026526\du}{14.392301\du}}
\pgfusepath{fill}
\pgfsetbuttcap
\pgfsetmiterjoin
\pgfsetdash{}{0pt}
\definecolor{dialinecolor}{rgb}{0.678431, 0.839216, 0.905882}
\pgfsetfillcolor{dialinecolor}
\pgfpathmoveto{\pgfpoint{54.407687\du}{13.562801\du}}
\pgfpathlineto{\pgfpoint{54.407687\du}{13.552213\du}}
\pgfpathlineto{\pgfpoint{54.387607\du}{13.552213\du}}
\pgfpathlineto{\pgfpoint{54.387607\du}{13.562801\du}}
\pgfpathlineto{\pgfpoint{54.407687\du}{13.562801\du}}
\pgfusepath{fill}
\pgfsetbuttcap
\pgfsetmiterjoin
\pgfsetdash{}{0pt}
\definecolor{dialinecolor}{rgb}{0.678431, 0.839216, 0.905882}
\pgfsetfillcolor{dialinecolor}
\pgfpathmoveto{\pgfpoint{54.407687\du}{14.392301\du}}
\pgfpathlineto{\pgfpoint{54.407687\du}{13.562801\du}}
\pgfpathlineto{\pgfpoint{54.387607\du}{13.562801\du}}
\pgfpathlineto{\pgfpoint{54.387607\du}{14.392301\du}}
\pgfpathlineto{\pgfpoint{54.407687\du}{14.392301\du}}
\pgfusepath{fill}
\pgfsetbuttcap
\pgfsetmiterjoin
\pgfsetdash{}{0pt}
\definecolor{dialinecolor}{rgb}{0.678431, 0.839216, 0.905882}
\pgfsetfillcolor{dialinecolor}
\pgfpathmoveto{\pgfpoint{54.387607\du}{14.392301\du}}
\pgfpathlineto{\pgfpoint{54.387607\du}{14.402889\du}}
\pgfpathlineto{\pgfpoint{54.407687\du}{14.402889\du}}
\pgfpathlineto{\pgfpoint{54.407687\du}{14.392301\du}}
\pgfpathlineto{\pgfpoint{54.387607\du}{14.392301\du}}
\pgfusepath{fill}
\pgfsetbuttcap
\pgfsetmiterjoin
\pgfsetdash{}{0pt}
\definecolor{dialinecolor}{rgb}{0.027451, 0.372549, 0.529412}
\pgfsetfillcolor{dialinecolor}
\pgfpathmoveto{\pgfpoint{53.333937\du}{14.541626\du}}
\pgfpathlineto{\pgfpoint{53.333572\du}{14.559150\du}}
\pgfpathlineto{\pgfpoint{53.331017\du}{14.576310\du}}
\pgfpathlineto{\pgfpoint{53.327366\du}{14.592374\du}}
\pgfpathlineto{\pgfpoint{53.321524\du}{14.609534\du}}
\pgfpathlineto{\pgfpoint{53.314952\du}{14.625233\du}}
\pgfpathlineto{\pgfpoint{53.306555\du}{14.641297\du}}
\pgfpathlineto{\pgfpoint{53.297063\du}{14.656996\du}}
\pgfpathlineto{\pgfpoint{53.285380\du}{14.671965\du}}
\pgfpathlineto{\pgfpoint{53.273696\du}{14.686934\du}}
\pgfpathlineto{\pgfpoint{53.260188\du}{14.701538\du}}
\pgfpathlineto{\pgfpoint{53.245219\du}{14.715412\du}}
\pgfpathlineto{\pgfpoint{53.229155\du}{14.729650\du}}
\pgfpathlineto{\pgfpoint{53.211630\du}{14.742429\du}}
\pgfpathlineto{\pgfpoint{53.193375\du}{14.755207\du}}
\pgfpathlineto{\pgfpoint{53.174390\du}{14.767620\du}}
\pgfpathlineto{\pgfpoint{53.154310\du}{14.779304\du}}
\pgfpathlineto{\pgfpoint{53.132769\du}{14.789891\du}}
\pgfpathlineto{\pgfpoint{53.110133\du}{14.800844\du}}
\pgfpathlineto{\pgfpoint{53.087132\du}{14.810702\du}}
\pgfpathlineto{\pgfpoint{53.063401\du}{14.820194\du}}
\pgfpathlineto{\pgfpoint{53.038209\du}{14.828227\du}}
\pgfpathlineto{\pgfpoint{53.012652\du}{14.836624\du}}
\pgfpathlineto{\pgfpoint{52.986000\du}{14.844291\du}}
\pgfpathlineto{\pgfpoint{52.959348\du}{14.851228\du}}
\pgfpathlineto{\pgfpoint{52.931236\du}{14.857069\du}}
\pgfpathlineto{\pgfpoint{52.902758\du}{14.862181\du}}
\pgfpathlineto{\pgfpoint{52.873185\du}{14.866562\du}}
\pgfpathlineto{\pgfpoint{52.842882\du}{14.870578\du}}
\pgfpathlineto{\pgfpoint{52.812944\du}{14.873499\du}}
\pgfpathlineto{\pgfpoint{52.782276\du}{14.875689\du}}
\pgfpathlineto{\pgfpoint{52.750878\du}{14.876784\du}}
\pgfpathlineto{\pgfpoint{52.719480\du}{14.877515\du}}
\pgfpathlineto{\pgfpoint{52.688446\du}{14.876784\du}}
\pgfpathlineto{\pgfpoint{52.657048\du}{14.875689\du}}
\pgfpathlineto{\pgfpoint{52.626015\du}{14.873499\du}}
\pgfpathlineto{\pgfpoint{52.595712\du}{14.870578\du}}
\pgfpathlineto{\pgfpoint{52.566139\du}{14.866562\du}}
\pgfpathlineto{\pgfpoint{52.536566\du}{14.862181\du}}
\pgfpathlineto{\pgfpoint{52.508089\du}{14.857069\du}}
\pgfpathlineto{\pgfpoint{52.480341\du}{14.851228\du}}
\pgfpathlineto{\pgfpoint{52.452959\du}{14.844291\du}}
\pgfpathlineto{\pgfpoint{52.426672\du}{14.836624\du}}
\pgfpathlineto{\pgfpoint{52.401480\du}{14.828227\du}}
\pgfpathlineto{\pgfpoint{52.375924\du}{14.820194\du}}
\pgfpathlineto{\pgfpoint{52.351827\du}{14.810702\du}}
\pgfpathlineto{\pgfpoint{52.329191\du}{14.800844\du}}
\pgfpathlineto{\pgfpoint{52.306555\du}{14.789891\du}}
\pgfpathlineto{\pgfpoint{52.285745\du}{14.779304\du}}
\pgfpathlineto{\pgfpoint{52.265299\du}{14.767620\du}}
\pgfpathlineto{\pgfpoint{52.244854\du}{14.755207\du}}
\pgfpathlineto{\pgfpoint{52.226964\du}{14.742429\du}}
\pgfpathlineto{\pgfpoint{52.210535\du}{14.729650\du}}
\pgfpathlineto{\pgfpoint{52.193740\du}{14.715412\du}}
\pgfpathlineto{\pgfpoint{52.179136\du}{14.701538\du}}
\pgfpathlineto{\pgfpoint{52.165993\du}{14.686934\du}}
\pgfpathlineto{\pgfpoint{52.153580\du}{14.671965\du}}
\pgfpathlineto{\pgfpoint{52.142627\du}{14.656996\du}}
\pgfpathlineto{\pgfpoint{52.133134\du}{14.641297\du}}
\pgfpathlineto{\pgfpoint{52.124372\du}{14.625233\du}}
\pgfpathlineto{\pgfpoint{52.117435\du}{14.609534\du}}
\pgfpathlineto{\pgfpoint{52.112689\du}{14.592374\du}}
\pgfpathlineto{\pgfpoint{52.108308\du}{14.576310\du}}
\pgfpathlineto{\pgfpoint{52.105752\du}{14.559150\du}}
\pgfpathlineto{\pgfpoint{52.104657\du}{14.541626\du}}
\pgfpathlineto{\pgfpoint{52.105752\du}{14.524101\du}}
\pgfpathlineto{\pgfpoint{52.108308\du}{14.506941\du}}
\pgfpathlineto{\pgfpoint{52.112689\du}{14.490877\du}}
\pgfpathlineto{\pgfpoint{52.117435\du}{14.473718\du}}
\pgfpathlineto{\pgfpoint{52.124372\du}{14.458018\du}}
\pgfpathlineto{\pgfpoint{52.133134\du}{14.442319\du}}
\pgfpathlineto{\pgfpoint{52.142627\du}{14.426255\du}}
\pgfpathlineto{\pgfpoint{52.153580\du}{14.411286\du}}
\pgfpathlineto{\pgfpoint{52.165993\du}{14.396682\du}}
\pgfpathlineto{\pgfpoint{52.179136\du}{14.382078\du}}
\pgfpathlineto{\pgfpoint{52.193740\du}{14.367840\du}}
\pgfpathlineto{\pgfpoint{52.210535\du}{14.353966\du}}
\pgfpathlineto{\pgfpoint{52.226964\du}{14.340822\du}}
\pgfpathlineto{\pgfpoint{52.244854\du}{14.328774\du}}
\pgfpathlineto{\pgfpoint{52.265299\du}{14.316361\du}}
\pgfpathlineto{\pgfpoint{52.285745\du}{14.304678\du}}
\pgfpathlineto{\pgfpoint{52.306555\du}{14.293725\du}}
\pgfpathlineto{\pgfpoint{52.329191\du}{14.283137\du}}
\pgfpathlineto{\pgfpoint{52.351827\du}{14.272549\du}}
\pgfpathlineto{\pgfpoint{52.375924\du}{14.263787\du}}
\pgfpathlineto{\pgfpoint{52.401480\du}{14.255025\du}}
\pgfpathlineto{\pgfpoint{52.426672\du}{14.246627\du}}
\pgfpathlineto{\pgfpoint{52.452959\du}{14.238960\du}}
\pgfpathlineto{\pgfpoint{52.480341\du}{14.232754\du}}
\pgfpathlineto{\pgfpoint{52.508089\du}{14.226182\du}}
\pgfpathlineto{\pgfpoint{52.536566\du}{14.221071\du}}
\pgfpathlineto{\pgfpoint{52.566139\du}{14.217055\du}}
\pgfpathlineto{\pgfpoint{52.595712\du}{14.212673\du}}
\pgfpathlineto{\pgfpoint{52.626015\du}{14.209753\du}}
\pgfpathlineto{\pgfpoint{52.657048\du}{14.208292\du}}
\pgfpathlineto{\pgfpoint{52.688446\du}{14.206467\du}}
\pgfpathlineto{\pgfpoint{52.719480\du}{14.206467\du}}
\pgfpathlineto{\pgfpoint{52.750878\du}{14.206467\du}}
\pgfpathlineto{\pgfpoint{52.782276\du}{14.208292\du}}
\pgfpathlineto{\pgfpoint{52.812944\du}{14.209753\du}}
\pgfpathlineto{\pgfpoint{52.842882\du}{14.212673\du}}
\pgfpathlineto{\pgfpoint{52.873185\du}{14.217055\du}}
\pgfpathlineto{\pgfpoint{52.902758\du}{14.221071\du}}
\pgfpathlineto{\pgfpoint{52.931236\du}{14.226182\du}}
\pgfpathlineto{\pgfpoint{52.959348\du}{14.232754\du}}
\pgfpathlineto{\pgfpoint{52.986000\du}{14.238960\du}}
\pgfpathlineto{\pgfpoint{53.012652\du}{14.246627\du}}
\pgfpathlineto{\pgfpoint{53.038209\du}{14.255025\du}}
\pgfpathlineto{\pgfpoint{53.063401\du}{14.263787\du}}
\pgfpathlineto{\pgfpoint{53.087132\du}{14.272549\du}}
\pgfpathlineto{\pgfpoint{53.110133\du}{14.283137\du}}
\pgfpathlineto{\pgfpoint{53.132769\du}{14.293725\du}}
\pgfpathlineto{\pgfpoint{53.154310\du}{14.304678\du}}
\pgfpathlineto{\pgfpoint{53.174390\du}{14.316361\du}}
\pgfpathlineto{\pgfpoint{53.193375\du}{14.328774\du}}
\pgfpathlineto{\pgfpoint{53.211630\du}{14.340822\du}}
\pgfpathlineto{\pgfpoint{53.229155\du}{14.353966\du}}
\pgfpathlineto{\pgfpoint{53.245219\du}{14.367840\du}}
\pgfpathlineto{\pgfpoint{53.260188\du}{14.382078\du}}
\pgfpathlineto{\pgfpoint{53.273696\du}{14.396682\du}}
\pgfpathlineto{\pgfpoint{53.285380\du}{14.411286\du}}
\pgfpathlineto{\pgfpoint{53.297063\du}{14.426255\du}}
\pgfpathlineto{\pgfpoint{53.306555\du}{14.442319\du}}
\pgfpathlineto{\pgfpoint{53.314952\du}{14.458018\du}}
\pgfpathlineto{\pgfpoint{53.321524\du}{14.473718\du}}
\pgfpathlineto{\pgfpoint{53.327366\du}{14.490877\du}}
\pgfpathlineto{\pgfpoint{53.331017\du}{14.506941\du}}
\pgfpathlineto{\pgfpoint{53.333572\du}{14.524101\du}}
\pgfpathlineto{\pgfpoint{53.333937\du}{14.541626\du}}
\pgfusepath{fill}
\pgfsetbuttcap
\pgfsetmiterjoin
\pgfsetdash{}{0pt}
\definecolor{dialinecolor}{rgb}{0.678431, 0.839216, 0.905882}
\pgfsetfillcolor{dialinecolor}
\pgfpathmoveto{\pgfpoint{52.719480\du}{14.887372\du}}
\pgfpathlineto{\pgfpoint{52.719480\du}{14.887372\du}}
\pgfpathlineto{\pgfpoint{52.735544\du}{14.887372\du}}
\pgfpathlineto{\pgfpoint{52.751608\du}{14.887007\du}}
\pgfpathlineto{\pgfpoint{52.767672\du}{14.886277\du}}
\pgfpathlineto{\pgfpoint{52.783006\du}{14.885547\du}}
\pgfpathlineto{\pgfpoint{52.798706\du}{14.884451\du}}
\pgfpathlineto{\pgfpoint{52.814040\du}{14.883356\du}}
\pgfpathlineto{\pgfpoint{52.829009\du}{14.882261\du}}
\pgfpathlineto{\pgfpoint{52.845073\du}{14.880435\du}}
\pgfpathlineto{\pgfpoint{52.859312\du}{14.878610\du}}
\pgfpathlineto{\pgfpoint{52.874281\du}{14.876784\du}}
\pgfpathlineto{\pgfpoint{52.889250\du}{14.874594\du}}
\pgfpathlineto{\pgfpoint{52.904584\du}{14.872403\du}}
\pgfpathlineto{\pgfpoint{52.918822\du}{14.869482\du}}
\pgfpathlineto{\pgfpoint{52.932696\du}{14.866927\du}}
\pgfpathlineto{\pgfpoint{52.947300\du}{14.864006\du}}
\pgfpathlineto{\pgfpoint{52.961174\du}{14.861085\du}}
\pgfpathlineto{\pgfpoint{52.974682\du}{14.857434\du}}
\pgfpathlineto{\pgfpoint{52.988556\du}{14.854148\du}}
\pgfpathlineto{\pgfpoint{53.002064\du}{14.850497\du}}
\pgfpathlineto{\pgfpoint{53.015208\du}{14.846481\du}}
\pgfpathlineto{\pgfpoint{53.028351\du}{14.842465\du}}
\pgfpathlineto{\pgfpoint{53.041495\du}{14.838449\du}}
\pgfpathlineto{\pgfpoint{53.054273\du}{14.833703\du}}
\pgfpathlineto{\pgfpoint{53.067052\du}{14.829687\du}}
\pgfpathlineto{\pgfpoint{53.078735\du}{14.824941\du}}
\pgfpathlineto{\pgfpoint{53.091513\du}{14.820194\du}}
\pgfpathlineto{\pgfpoint{53.102831\du}{14.814718\du}}
\pgfpathlineto{\pgfpoint{53.114514\du}{14.809607\du}}
\pgfpathlineto{\pgfpoint{53.125832\du}{14.804495\du}}
\pgfpathlineto{\pgfpoint{53.137150\du}{14.799019\du}}
\pgfpathlineto{\pgfpoint{53.148103\du}{14.793907\du}}
\pgfpathlineto{\pgfpoint{53.159421\du}{14.788066\du}}
\pgfpathlineto{\pgfpoint{53.169279\du}{14.782224\du}}
\pgfpathlineto{\pgfpoint{53.179501\du}{14.775653\du}}
\pgfpathlineto{\pgfpoint{53.189359\du}{14.769811\du}}
\pgfpathlineto{\pgfpoint{53.199582\du}{14.763239\du}}
\pgfpathlineto{\pgfpoint{53.209074\du}{14.757033\du}}
\pgfpathlineto{\pgfpoint{53.218202\du}{14.750461\du}}
\pgfpathlineto{\pgfpoint{53.226964\du}{14.744254\du}}
\pgfpathlineto{\pgfpoint{53.235361\du}{14.736952\du}}
\pgfpathlineto{\pgfpoint{53.243759\du}{14.730016\du}}
\pgfpathlineto{\pgfpoint{53.251791\du}{14.723079\du}}
\pgfpathlineto{\pgfpoint{53.259823\du}{14.716142\du}}
\pgfpathlineto{\pgfpoint{53.266760\du}{14.708840\du}}
\pgfpathlineto{\pgfpoint{53.274427\du}{14.701538\du}}
\pgfpathlineto{\pgfpoint{53.280998\du}{14.693871\du}}
\pgfpathlineto{\pgfpoint{53.287570\du}{14.686204\du}}
\pgfpathlineto{\pgfpoint{53.293777\du}{14.678537\du}}
\pgfpathlineto{\pgfpoint{53.299983\du}{14.670505\du}}
\pgfpathlineto{\pgfpoint{53.305825\du}{14.662838\du}}
\pgfpathlineto{\pgfpoint{53.310571\du}{14.654440\du}}
\pgfpathlineto{\pgfpoint{53.315317\du}{14.646408\du}}
\pgfpathlineto{\pgfpoint{53.320064\du}{14.638011\du}}
\pgfpathlineto{\pgfpoint{53.324445\du}{14.629979\du}}
\pgfpathlineto{\pgfpoint{53.327731\du}{14.621217\du}}
\pgfpathlineto{\pgfpoint{53.331747\du}{14.613185\du}}
\pgfpathlineto{\pgfpoint{53.334303\du}{14.604422\du}}
\pgfpathlineto{\pgfpoint{53.337223\du}{14.595660\du}}
\pgfpathlineto{\pgfpoint{53.339049\du}{14.586532\du}}
\pgfpathlineto{\pgfpoint{53.341239\du}{14.577770\du}}
\pgfpathlineto{\pgfpoint{53.342700\du}{14.569008\du}}
\pgfpathlineto{\pgfpoint{53.343795\du}{14.559880\du}}
\pgfpathlineto{\pgfpoint{53.344525\du}{14.551118\du}}
\pgfpathlineto{\pgfpoint{53.344525\du}{14.541626\du}}
\pgfpathlineto{\pgfpoint{53.324445\du}{14.541626\du}}
\pgfpathlineto{\pgfpoint{53.323715\du}{14.549658\du}}
\pgfpathlineto{\pgfpoint{53.323350\du}{14.558055\du}}
\pgfpathlineto{\pgfpoint{53.322619\du}{14.566087\du}}
\pgfpathlineto{\pgfpoint{53.321159\du}{14.573754\du}}
\pgfpathlineto{\pgfpoint{53.319699\du}{14.582151\du}}
\pgfpathlineto{\pgfpoint{53.317143\du}{14.590183\du}}
\pgfpathlineto{\pgfpoint{53.314952\du}{14.597850\du}}
\pgfpathlineto{\pgfpoint{53.311667\du}{14.605518\du}}
\pgfpathlineto{\pgfpoint{53.309111\du}{14.613185\du}}
\pgfpathlineto{\pgfpoint{53.305825\du}{14.621217\du}}
\pgfpathlineto{\pgfpoint{53.301809\du}{14.628884\du}}
\pgfpathlineto{\pgfpoint{53.297063\du}{14.636551\du}}
\pgfpathlineto{\pgfpoint{53.293047\du}{14.644218\du}}
\pgfpathlineto{\pgfpoint{53.287935\du}{14.651155\du}}
\pgfpathlineto{\pgfpoint{53.283189\du}{14.658822\du}}
\pgfpathlineto{\pgfpoint{53.277713\du}{14.665758\du}}
\pgfpathlineto{\pgfpoint{53.272601\du}{14.673426\du}}
\pgfpathlineto{\pgfpoint{53.265664\du}{14.680362\du}}
\pgfpathlineto{\pgfpoint{53.259823\du}{14.687299\du}}
\pgfpathlineto{\pgfpoint{53.252521\du}{14.694236\du}}
\pgfpathlineto{\pgfpoint{53.245584\du}{14.701538\du}}
\pgfpathlineto{\pgfpoint{53.238282\du}{14.707745\du}}
\pgfpathlineto{\pgfpoint{53.231345\du}{14.714681\du}}
\pgfpathlineto{\pgfpoint{53.222948\du}{14.721253\du}}
\pgfpathlineto{\pgfpoint{53.214916\du}{14.727825\du}}
\pgfpathlineto{\pgfpoint{53.205788\du}{14.734032\du}}
\pgfpathlineto{\pgfpoint{53.197026\du}{14.739873\du}}
\pgfpathlineto{\pgfpoint{53.187534\du}{14.746445\du}}
\pgfpathlineto{\pgfpoint{53.178406\du}{14.753017\du}}
\pgfpathlineto{\pgfpoint{53.169279\du}{14.758128\du}}
\pgfpathlineto{\pgfpoint{53.159421\du}{14.763970\du}}
\pgfpathlineto{\pgfpoint{53.149564\du}{14.769811\du}}
\pgfpathlineto{\pgfpoint{53.138976\du}{14.775288\du}}
\pgfpathlineto{\pgfpoint{53.128388\du}{14.781129\du}}
\pgfpathlineto{\pgfpoint{53.117800\du}{14.785510\du}}
\pgfpathlineto{\pgfpoint{53.106117\du}{14.791352\du}}
\pgfpathlineto{\pgfpoint{53.094799\du}{14.796098\du}}
\pgfpathlineto{\pgfpoint{53.083481\du}{14.800844\du}}
\pgfpathlineto{\pgfpoint{53.071798\du}{14.805591\du}}
\pgfpathlineto{\pgfpoint{53.059750\du}{14.810337\du}}
\pgfpathlineto{\pgfpoint{53.047336\du}{14.814353\du}}
\pgfpathlineto{\pgfpoint{53.035653\du}{14.819099\du}}
\pgfpathlineto{\pgfpoint{53.022875\du}{14.823115\du}}
\pgfpathlineto{\pgfpoint{53.009366\du}{14.826766\du}}
\pgfpathlineto{\pgfpoint{52.996588\du}{14.830782\du}}
\pgfpathlineto{\pgfpoint{52.983445\du}{14.834068\du}}
\pgfpathlineto{\pgfpoint{52.969936\du}{14.837719\du}}
\pgfpathlineto{\pgfpoint{52.956427\du}{14.840640\du}}
\pgfpathlineto{\pgfpoint{52.942554\du}{14.844291\du}}
\pgfpathlineto{\pgfpoint{52.928680\du}{14.846481\du}}
\pgfpathlineto{\pgfpoint{52.914806\du}{14.849402\du}}
\pgfpathlineto{\pgfpoint{52.900568\du}{14.851593\du}}
\pgfpathlineto{\pgfpoint{52.886329\du}{14.854148\du}}
\pgfpathlineto{\pgfpoint{52.872090\du}{14.856339\du}}
\pgfpathlineto{\pgfpoint{52.857486\du}{14.858164\du}}
\pgfpathlineto{\pgfpoint{52.842517\du}{14.859990\du}}
\pgfpathlineto{\pgfpoint{52.827183\du}{14.861815\du}}
\pgfpathlineto{\pgfpoint{52.812579\du}{14.862911\du}}
\pgfpathlineto{\pgfpoint{52.796880\du}{14.864006\du}}
\pgfpathlineto{\pgfpoint{52.781911\du}{14.865101\du}}
\pgfpathlineto{\pgfpoint{52.766577\du}{14.865832\du}}
\pgfpathlineto{\pgfpoint{52.750878\du}{14.866562\du}}
\pgfpathlineto{\pgfpoint{52.735544\du}{14.866927\du}}
\pgfpathlineto{\pgfpoint{52.719480\du}{14.866927\du}}
\pgfpathlineto{\pgfpoint{52.719480\du}{14.866927\du}}
\pgfpathlineto{\pgfpoint{52.719480\du}{14.866927\du}}
\pgfpathlineto{\pgfpoint{52.718384\du}{14.866927\du}}
\pgfpathlineto{\pgfpoint{52.717289\du}{14.866927\du}}
\pgfpathlineto{\pgfpoint{52.716559\du}{14.867657\du}}
\pgfpathlineto{\pgfpoint{52.714733\du}{14.867657\du}}
\pgfpathlineto{\pgfpoint{52.714368\du}{14.868022\du}}
\pgfpathlineto{\pgfpoint{52.713273\du}{14.868752\du}}
\pgfpathlineto{\pgfpoint{52.712908\du}{14.869482\du}}
\pgfpathlineto{\pgfpoint{52.712543\du}{14.869848\du}}
\pgfpathlineto{\pgfpoint{52.711082\du}{14.871673\du}}
\pgfpathlineto{\pgfpoint{52.709622\du}{14.873499\du}}
\pgfpathlineto{\pgfpoint{52.709622\du}{14.875324\du}}
\pgfpathlineto{\pgfpoint{52.709257\du}{14.877515\du}}
\pgfpathlineto{\pgfpoint{52.709622\du}{14.879340\du}}
\pgfpathlineto{\pgfpoint{52.709622\du}{14.881166\du}}
\pgfpathlineto{\pgfpoint{52.711082\du}{14.882626\du}}
\pgfpathlineto{\pgfpoint{52.712543\du}{14.884086\du}}
\pgfpathlineto{\pgfpoint{52.712908\du}{14.885182\du}}
\pgfpathlineto{\pgfpoint{52.713273\du}{14.885547\du}}
\pgfpathlineto{\pgfpoint{52.714368\du}{14.886277\du}}
\pgfpathlineto{\pgfpoint{52.714733\du}{14.886277\du}}
\pgfpathlineto{\pgfpoint{52.716559\du}{14.887007\du}}
\pgfpathlineto{\pgfpoint{52.717289\du}{14.887372\du}}
\pgfpathlineto{\pgfpoint{52.718384\du}{14.887372\du}}
\pgfpathlineto{\pgfpoint{52.719480\du}{14.887372\du}}
\pgfusepath{fill}
\pgfsetbuttcap
\pgfsetmiterjoin
\pgfsetdash{}{0pt}
\definecolor{dialinecolor}{rgb}{0.678431, 0.839216, 0.905882}
\pgfsetfillcolor{dialinecolor}
\pgfpathmoveto{\pgfpoint{52.094799\du}{14.541626\du}}
\pgfpathlineto{\pgfpoint{52.094799\du}{14.541626\du}}
\pgfpathlineto{\pgfpoint{52.094799\du}{14.550388\du}}
\pgfpathlineto{\pgfpoint{52.095529\du}{14.559880\du}}
\pgfpathlineto{\pgfpoint{52.096625\du}{14.569008\du}}
\pgfpathlineto{\pgfpoint{52.098815\du}{14.577770\du}}
\pgfpathlineto{\pgfpoint{52.099910\du}{14.586532\du}}
\pgfpathlineto{\pgfpoint{52.102466\du}{14.595660\du}}
\pgfpathlineto{\pgfpoint{52.104657\du}{14.604422\du}}
\pgfpathlineto{\pgfpoint{52.107943\du}{14.613185\du}}
\pgfpathlineto{\pgfpoint{52.111228\du}{14.621217\du}}
\pgfpathlineto{\pgfpoint{52.115244\du}{14.629979\du}}
\pgfpathlineto{\pgfpoint{52.119626\du}{14.638011\du}}
\pgfpathlineto{\pgfpoint{52.124007\du}{14.646408\du}}
\pgfpathlineto{\pgfpoint{52.128753\du}{14.654440\du}}
\pgfpathlineto{\pgfpoint{52.133864\du}{14.662838\du}}
\pgfpathlineto{\pgfpoint{52.140071\du}{14.670505\du}}
\pgfpathlineto{\pgfpoint{52.144817\du}{14.678537\du}}
\pgfpathlineto{\pgfpoint{52.151389\du}{14.686204\du}}
\pgfpathlineto{\pgfpoint{52.158326\du}{14.693871\du}}
\pgfpathlineto{\pgfpoint{52.164898\du}{14.701538\du}}
\pgfpathlineto{\pgfpoint{52.171834\du}{14.708840\du}}
\pgfpathlineto{\pgfpoint{52.179136\du}{14.716142\du}}
\pgfpathlineto{\pgfpoint{52.187899\du}{14.723079\du}}
\pgfpathlineto{\pgfpoint{52.194836\du}{14.730016\du}}
\pgfpathlineto{\pgfpoint{52.203963\du}{14.736952\du}}
\pgfpathlineto{\pgfpoint{52.211995\du}{14.744254\du}}
\pgfpathlineto{\pgfpoint{52.221123\du}{14.750461\du}}
\pgfpathlineto{\pgfpoint{52.230980\du}{14.757033\du}}
\pgfpathlineto{\pgfpoint{52.240108\du}{14.763239\du}}
\pgfpathlineto{\pgfpoint{52.249600\du}{14.769811\du}}
\pgfpathlineto{\pgfpoint{52.259458\du}{14.775653\du}}
\pgfpathlineto{\pgfpoint{52.269680\du}{14.782224\du}}
\pgfpathlineto{\pgfpoint{52.279903\du}{14.788066\du}}
\pgfpathlineto{\pgfpoint{52.290856\du}{14.793907\du}}
\pgfpathlineto{\pgfpoint{52.302174\du}{14.799019\du}}
\pgfpathlineto{\pgfpoint{52.313492\du}{14.804495\du}}
\pgfpathlineto{\pgfpoint{52.324810\du}{14.809607\du}}
\pgfpathlineto{\pgfpoint{52.336858\du}{14.814718\du}}
\pgfpathlineto{\pgfpoint{52.348176\du}{14.820194\du}}
\pgfpathlineto{\pgfpoint{52.360589\du}{14.824941\du}}
\pgfpathlineto{\pgfpoint{52.372638\du}{14.829687\du}}
\pgfpathlineto{\pgfpoint{52.384686\du}{14.833703\du}}
\pgfpathlineto{\pgfpoint{52.397829\du}{14.838449\du}}
\pgfpathlineto{\pgfpoint{52.410608\du}{14.842465\du}}
\pgfpathlineto{\pgfpoint{52.424116\du}{14.846481\du}}
\pgfpathlineto{\pgfpoint{52.437625\du}{14.850497\du}}
\pgfpathlineto{\pgfpoint{52.450403\du}{14.854148\du}}
\pgfpathlineto{\pgfpoint{52.463912\du}{14.857434\du}}
\pgfpathlineto{\pgfpoint{52.477786\du}{14.861085\du}}
\pgfpathlineto{\pgfpoint{52.492389\du}{14.864006\du}}
\pgfpathlineto{\pgfpoint{52.506993\du}{14.866927\du}}
\pgfpathlineto{\pgfpoint{52.520867\du}{14.869482\du}}
\pgfpathlineto{\pgfpoint{52.535106\du}{14.872403\du}}
\pgfpathlineto{\pgfpoint{52.550075\du}{14.874594\du}}
\pgfpathlineto{\pgfpoint{52.565044\du}{14.876784\du}}
\pgfpathlineto{\pgfpoint{52.579648\du}{14.878610\du}}
\pgfpathlineto{\pgfpoint{52.594251\du}{14.880435\du}}
\pgfpathlineto{\pgfpoint{52.609951\du}{14.882261\du}}
\pgfpathlineto{\pgfpoint{52.625650\du}{14.883356\du}}
\pgfpathlineto{\pgfpoint{52.640254\du}{14.884451\du}}
\pgfpathlineto{\pgfpoint{52.656318\du}{14.885547\du}}
\pgfpathlineto{\pgfpoint{52.671652\du}{14.886277\du}}
\pgfpathlineto{\pgfpoint{52.687716\du}{14.887007\du}}
\pgfpathlineto{\pgfpoint{52.703780\du}{14.887372\du}}
\pgfpathlineto{\pgfpoint{52.719480\du}{14.887372\du}}
\pgfpathlineto{\pgfpoint{52.719480\du}{14.866927\du}}
\pgfpathlineto{\pgfpoint{52.703780\du}{14.866927\du}}
\pgfpathlineto{\pgfpoint{52.688446\du}{14.866562\du}}
\pgfpathlineto{\pgfpoint{52.672382\du}{14.865832\du}}
\pgfpathlineto{\pgfpoint{52.657778\du}{14.865101\du}}
\pgfpathlineto{\pgfpoint{52.642079\du}{14.864006\du}}
\pgfpathlineto{\pgfpoint{52.627110\du}{14.862911\du}}
\pgfpathlineto{\pgfpoint{52.612141\du}{14.861815\du}}
\pgfpathlineto{\pgfpoint{52.597172\du}{14.859990\du}}
\pgfpathlineto{\pgfpoint{52.582203\du}{14.858164\du}}
\pgfpathlineto{\pgfpoint{52.567599\du}{14.856339\du}}
\pgfpathlineto{\pgfpoint{52.552995\du}{14.854148\du}}
\pgfpathlineto{\pgfpoint{52.538757\du}{14.851593\du}}
\pgfpathlineto{\pgfpoint{52.524883\du}{14.849402\du}}
\pgfpathlineto{\pgfpoint{52.511009\du}{14.846481\du}}
\pgfpathlineto{\pgfpoint{52.496771\du}{14.844291\du}}
\pgfpathlineto{\pgfpoint{52.482897\du}{14.840640\du}}
\pgfpathlineto{\pgfpoint{52.469388\du}{14.837719\du}}
\pgfpathlineto{\pgfpoint{52.456245\du}{14.834068\du}}
\pgfpathlineto{\pgfpoint{52.442736\du}{14.830782\du}}
\pgfpathlineto{\pgfpoint{52.429958\du}{14.826766\du}}
\pgfpathlineto{\pgfpoint{52.416814\du}{14.823115\du}}
\pgfpathlineto{\pgfpoint{52.404036\du}{14.819099\du}}
\pgfpathlineto{\pgfpoint{52.391988\du}{14.814353\du}}
\pgfpathlineto{\pgfpoint{52.379575\du}{14.810337\du}}
\pgfpathlineto{\pgfpoint{52.367161\du}{14.805591\du}}
\pgfpathlineto{\pgfpoint{52.355843\du}{14.800844\du}}
\pgfpathlineto{\pgfpoint{52.344160\du}{14.796098\du}}
\pgfpathlineto{\pgfpoint{52.333207\du}{14.791352\du}}
\pgfpathlineto{\pgfpoint{52.321889\du}{14.785510\du}}
\pgfpathlineto{\pgfpoint{52.311301\du}{14.781129\du}}
\pgfpathlineto{\pgfpoint{52.300349\du}{14.775288\du}}
\pgfpathlineto{\pgfpoint{52.290126\du}{14.769811\du}}
\pgfpathlineto{\pgfpoint{52.279903\du}{14.763970\du}}
\pgfpathlineto{\pgfpoint{52.270045\du}{14.758128\du}}
\pgfpathlineto{\pgfpoint{52.260553\du}{14.753017\du}}
\pgfpathlineto{\pgfpoint{52.250695\du}{14.746445\du}}
\pgfpathlineto{\pgfpoint{52.242298\du}{14.739873\du}}
\pgfpathlineto{\pgfpoint{52.233171\du}{14.734032\du}}
\pgfpathlineto{\pgfpoint{52.224773\du}{14.727825\du}}
\pgfpathlineto{\pgfpoint{52.216376\du}{14.721253\du}}
\pgfpathlineto{\pgfpoint{52.208344\du}{14.714681\du}}
\pgfpathlineto{\pgfpoint{52.201042\du}{14.707745\du}}
\pgfpathlineto{\pgfpoint{52.193375\du}{14.701538\du}}
\pgfpathlineto{\pgfpoint{52.186803\du}{14.694236\du}}
\pgfpathlineto{\pgfpoint{52.179867\du}{14.687299\du}}
\pgfpathlineto{\pgfpoint{52.173660\du}{14.680362\du}}
\pgfpathlineto{\pgfpoint{52.167453\du}{14.673426\du}}
\pgfpathlineto{\pgfpoint{52.161247\du}{14.665758\du}}
\pgfpathlineto{\pgfpoint{52.156135\du}{14.658822\du}}
\pgfpathlineto{\pgfpoint{52.151024\du}{14.651155\du}}
\pgfpathlineto{\pgfpoint{52.146278\du}{14.644218\du}}
\pgfpathlineto{\pgfpoint{52.141897\du}{14.636551\du}}
\pgfpathlineto{\pgfpoint{52.137515\du}{14.628884\du}}
\pgfpathlineto{\pgfpoint{52.133864\du}{14.621217\du}}
\pgfpathlineto{\pgfpoint{52.130579\du}{14.613185\du}}
\pgfpathlineto{\pgfpoint{52.127293\du}{14.605518\du}}
\pgfpathlineto{\pgfpoint{52.124372\du}{14.597850\du}}
\pgfpathlineto{\pgfpoint{52.121816\du}{14.590183\du}}
\pgfpathlineto{\pgfpoint{52.119991\du}{14.582151\du}}
\pgfpathlineto{\pgfpoint{52.118165\du}{14.573754\du}}
\pgfpathlineto{\pgfpoint{52.117070\du}{14.566087\du}}
\pgfpathlineto{\pgfpoint{52.115610\du}{14.558055\du}}
\pgfpathlineto{\pgfpoint{52.115244\du}{14.549658\du}}
\pgfpathlineto{\pgfpoint{52.115244\du}{14.541626\du}}
\pgfpathlineto{\pgfpoint{52.115244\du}{14.541626\du}}
\pgfpathlineto{\pgfpoint{52.115244\du}{14.541626\du}}
\pgfpathlineto{\pgfpoint{52.115244\du}{14.540530\du}}
\pgfpathlineto{\pgfpoint{52.115244\du}{14.539435\du}}
\pgfpathlineto{\pgfpoint{52.114879\du}{14.537975\du}}
\pgfpathlineto{\pgfpoint{52.114879\du}{14.536879\du}}
\pgfpathlineto{\pgfpoint{52.114514\du}{14.536514\du}}
\pgfpathlineto{\pgfpoint{52.113419\du}{14.535054\du}}
\pgfpathlineto{\pgfpoint{52.112689\du}{14.534689\du}}
\pgfpathlineto{\pgfpoint{52.112689\du}{14.533959\du}}
\pgfpathlineto{\pgfpoint{52.110498\du}{14.532863\du}}
\pgfpathlineto{\pgfpoint{52.108673\du}{14.532133\du}}
\pgfpathlineto{\pgfpoint{52.106847\du}{14.531768\du}}
\pgfpathlineto{\pgfpoint{52.104657\du}{14.531038\du}}
\pgfpathlineto{\pgfpoint{52.103196\du}{14.531768\du}}
\pgfpathlineto{\pgfpoint{52.101371\du}{14.532133\du}}
\pgfpathlineto{\pgfpoint{52.099545\du}{14.532863\du}}
\pgfpathlineto{\pgfpoint{52.097720\du}{14.533959\du}}
\pgfpathlineto{\pgfpoint{52.096990\du}{14.534689\du}}
\pgfpathlineto{\pgfpoint{52.096625\du}{14.535054\du}}
\pgfpathlineto{\pgfpoint{52.096259\du}{14.536514\du}}
\pgfpathlineto{\pgfpoint{52.095529\du}{14.536879\du}}
\pgfpathlineto{\pgfpoint{52.095529\du}{14.537975\du}}
\pgfpathlineto{\pgfpoint{52.094799\du}{14.539435\du}}
\pgfpathlineto{\pgfpoint{52.094799\du}{14.540530\du}}
\pgfpathlineto{\pgfpoint{52.094799\du}{14.541626\du}}
\pgfusepath{fill}
\pgfsetbuttcap
\pgfsetmiterjoin
\pgfsetdash{}{0pt}
\definecolor{dialinecolor}{rgb}{0.678431, 0.839216, 0.905882}
\pgfsetfillcolor{dialinecolor}
\pgfpathmoveto{\pgfpoint{52.719480\du}{14.195879\du}}
\pgfpathlineto{\pgfpoint{52.719480\du}{14.195879\du}}
\pgfpathlineto{\pgfpoint{52.703780\du}{14.195879\du}}
\pgfpathlineto{\pgfpoint{52.687716\du}{14.196244\du}}
\pgfpathlineto{\pgfpoint{52.671652\du}{14.196974\du}}
\pgfpathlineto{\pgfpoint{52.656318\du}{14.197704\du}}
\pgfpathlineto{\pgfpoint{52.640254\du}{14.198800\du}}
\pgfpathlineto{\pgfpoint{52.625650\du}{14.199895\du}}
\pgfpathlineto{\pgfpoint{52.609951\du}{14.200990\du}}
\pgfpathlineto{\pgfpoint{52.594251\du}{14.202816\du}}
\pgfpathlineto{\pgfpoint{52.579648\du}{14.204641\du}}
\pgfpathlineto{\pgfpoint{52.565044\du}{14.206467\du}}
\pgfpathlineto{\pgfpoint{52.550075\du}{14.208657\du}}
\pgfpathlineto{\pgfpoint{52.535106\du}{14.211213\du}}
\pgfpathlineto{\pgfpoint{52.520867\du}{14.214134\du}}
\pgfpathlineto{\pgfpoint{52.506993\du}{14.216324\du}}
\pgfpathlineto{\pgfpoint{52.492389\du}{14.219245\du}}
\pgfpathlineto{\pgfpoint{52.477786\du}{14.222166\du}}
\pgfpathlineto{\pgfpoint{52.463912\du}{14.225817\du}}
\pgfpathlineto{\pgfpoint{52.450403\du}{14.229103\du}}
\pgfpathlineto{\pgfpoint{52.437625\du}{14.233119\du}}
\pgfpathlineto{\pgfpoint{52.424116\du}{14.236770\du}}
\pgfpathlineto{\pgfpoint{52.410608\du}{14.240786\du}}
\pgfpathlineto{\pgfpoint{52.397829\du}{14.245167\du}}
\pgfpathlineto{\pgfpoint{52.384686\du}{14.249548\du}}
\pgfpathlineto{\pgfpoint{52.372638\du}{14.253929\du}}
\pgfpathlineto{\pgfpoint{52.360589\du}{14.258311\du}}
\pgfpathlineto{\pgfpoint{52.348176\du}{14.263787\du}}
\pgfpathlineto{\pgfpoint{52.336858\du}{14.268533\du}}
\pgfpathlineto{\pgfpoint{52.324810\du}{14.273645\du}}
\pgfpathlineto{\pgfpoint{52.313492\du}{14.278756\du}}
\pgfpathlineto{\pgfpoint{52.302174\du}{14.284232\du}}
\pgfpathlineto{\pgfpoint{52.290856\du}{14.290074\du}}
\pgfpathlineto{\pgfpoint{52.279903\du}{14.295185\du}}
\pgfpathlineto{\pgfpoint{52.269680\du}{14.301027\du}}
\pgfpathlineto{\pgfpoint{52.259458\du}{14.307599\du}}
\pgfpathlineto{\pgfpoint{52.249600\du}{14.313440\du}}
\pgfpathlineto{\pgfpoint{52.240108\du}{14.320012\du}}
\pgfpathlineto{\pgfpoint{52.230980\du}{14.326219\du}}
\pgfpathlineto{\pgfpoint{52.221123\du}{14.332790\du}}
\pgfpathlineto{\pgfpoint{52.211995\du}{14.339362\du}}
\pgfpathlineto{\pgfpoint{52.203963\du}{14.346299\du}}
\pgfpathlineto{\pgfpoint{52.194836\du}{14.353236\du}}
\pgfpathlineto{\pgfpoint{52.187899\du}{14.360173\du}}
\pgfpathlineto{\pgfpoint{52.179136\du}{14.367109\du}}
\pgfpathlineto{\pgfpoint{52.171834\du}{14.374411\du}}
\pgfpathlineto{\pgfpoint{52.164898\du}{14.382078\du}}
\pgfpathlineto{\pgfpoint{52.158326\du}{14.389380\du}}
\pgfpathlineto{\pgfpoint{52.151389\du}{14.397047\du}}
\pgfpathlineto{\pgfpoint{52.144817\du}{14.404714\du}}
\pgfpathlineto{\pgfpoint{52.140071\du}{14.412746\du}}
\pgfpathlineto{\pgfpoint{52.133864\du}{14.420413\du}}
\pgfpathlineto{\pgfpoint{52.128753\du}{14.428811\du}}
\pgfpathlineto{\pgfpoint{52.124007\du}{14.436843\du}}
\pgfpathlineto{\pgfpoint{52.119626\du}{14.445240\du}}
\pgfpathlineto{\pgfpoint{52.115244\du}{14.453272\du}}
\pgfpathlineto{\pgfpoint{52.111228\du}{14.462034\du}}
\pgfpathlineto{\pgfpoint{52.107943\du}{14.470432\du}}
\pgfpathlineto{\pgfpoint{52.104657\du}{14.479194\du}}
\pgfpathlineto{\pgfpoint{52.102466\du}{14.487956\du}}
\pgfpathlineto{\pgfpoint{52.099910\du}{14.496719\du}}
\pgfpathlineto{\pgfpoint{52.098815\du}{14.505481\du}}
\pgfpathlineto{\pgfpoint{52.096625\du}{14.514243\du}}
\pgfpathlineto{\pgfpoint{52.095529\du}{14.523371\du}}
\pgfpathlineto{\pgfpoint{52.094799\du}{14.532863\du}}
\pgfpathlineto{\pgfpoint{52.094799\du}{14.541626\du}}
\pgfpathlineto{\pgfpoint{52.115244\du}{14.541626\du}}
\pgfpathlineto{\pgfpoint{52.115244\du}{14.533593\du}}
\pgfpathlineto{\pgfpoint{52.115610\du}{14.525196\du}}
\pgfpathlineto{\pgfpoint{52.117070\du}{14.517164\du}}
\pgfpathlineto{\pgfpoint{52.118165\du}{14.509497\du}}
\pgfpathlineto{\pgfpoint{52.119991\du}{14.501100\du}}
\pgfpathlineto{\pgfpoint{52.121816\du}{14.493068\du}}
\pgfpathlineto{\pgfpoint{52.124372\du}{14.485401\du}}
\pgfpathlineto{\pgfpoint{52.127293\du}{14.477734\du}}
\pgfpathlineto{\pgfpoint{52.130579\du}{14.470432\du}}
\pgfpathlineto{\pgfpoint{52.133864\du}{14.462034\du}}
\pgfpathlineto{\pgfpoint{52.137515\du}{14.454367\du}}
\pgfpathlineto{\pgfpoint{52.141897\du}{14.446700\du}}
\pgfpathlineto{\pgfpoint{52.146278\du}{14.439764\du}}
\pgfpathlineto{\pgfpoint{52.151024\du}{14.432097\du}}
\pgfpathlineto{\pgfpoint{52.156135\du}{14.424795\du}}
\pgfpathlineto{\pgfpoint{52.161247\du}{14.417493\du}}
\pgfpathlineto{\pgfpoint{52.167453\du}{14.409826\du}}
\pgfpathlineto{\pgfpoint{52.173660\du}{14.402889\du}}
\pgfpathlineto{\pgfpoint{52.179867\du}{14.395952\du}}
\pgfpathlineto{\pgfpoint{52.186803\du}{14.389015\du}}
\pgfpathlineto{\pgfpoint{52.193375\du}{14.382443\du}}
\pgfpathlineto{\pgfpoint{52.201042\du}{14.375507\du}}
\pgfpathlineto{\pgfpoint{52.208344\du}{14.368570\du}}
\pgfpathlineto{\pgfpoint{52.216376\du}{14.361998\du}}
\pgfpathlineto{\pgfpoint{52.224773\du}{14.355426\du}}
\pgfpathlineto{\pgfpoint{52.233171\du}{14.349220\du}}
\pgfpathlineto{\pgfpoint{52.242298\du}{14.343378\du}}
\pgfpathlineto{\pgfpoint{52.250695\du}{14.336806\du}}
\pgfpathlineto{\pgfpoint{52.260553\du}{14.330965\du}}
\pgfpathlineto{\pgfpoint{52.270045\du}{14.325123\du}}
\pgfpathlineto{\pgfpoint{52.279903\du}{14.319282\du}}
\pgfpathlineto{\pgfpoint{52.290126\du}{14.313440\du}}
\pgfpathlineto{\pgfpoint{52.300349\du}{14.308329\du}}
\pgfpathlineto{\pgfpoint{52.311301\du}{14.302487\du}}
\pgfpathlineto{\pgfpoint{52.321889\du}{14.297741\du}}
\pgfpathlineto{\pgfpoint{52.333207\du}{14.292265\du}}
\pgfpathlineto{\pgfpoint{52.344160\du}{14.287153\du}}
\pgfpathlineto{\pgfpoint{52.355843\du}{14.282407\du}}
\pgfpathlineto{\pgfpoint{52.367161\du}{14.277661\du}}
\pgfpathlineto{\pgfpoint{52.379575\du}{14.272914\du}}
\pgfpathlineto{\pgfpoint{52.391988\du}{14.268898\du}}
\pgfpathlineto{\pgfpoint{52.404036\du}{14.264152\du}}
\pgfpathlineto{\pgfpoint{52.416814\du}{14.260136\du}}
\pgfpathlineto{\pgfpoint{52.429958\du}{14.256850\du}}
\pgfpathlineto{\pgfpoint{52.442736\du}{14.252469\du}}
\pgfpathlineto{\pgfpoint{52.456245\du}{14.249183\du}}
\pgfpathlineto{\pgfpoint{52.469388\du}{14.245532\du}}
\pgfpathlineto{\pgfpoint{52.482897\du}{14.242611\du}}
\pgfpathlineto{\pgfpoint{52.496771\du}{14.239691\du}}
\pgfpathlineto{\pgfpoint{52.511009\du}{14.236770\du}}
\pgfpathlineto{\pgfpoint{52.524883\du}{14.233849\du}}
\pgfpathlineto{\pgfpoint{52.538757\du}{14.231658\du}}
\pgfpathlineto{\pgfpoint{52.552995\du}{14.229103\du}}
\pgfpathlineto{\pgfpoint{52.567599\du}{14.226912\du}}
\pgfpathlineto{\pgfpoint{52.582203\du}{14.225087\du}}
\pgfpathlineto{\pgfpoint{52.597172\du}{14.223261\du}}
\pgfpathlineto{\pgfpoint{52.612141\du}{14.221436\du}}
\pgfpathlineto{\pgfpoint{52.627110\du}{14.220340\du}}
\pgfpathlineto{\pgfpoint{52.642079\du}{14.219245\du}}
\pgfpathlineto{\pgfpoint{52.657778\du}{14.218150\du}}
\pgfpathlineto{\pgfpoint{52.672382\du}{14.217420\du}}
\pgfpathlineto{\pgfpoint{52.688446\du}{14.217055\du}}
\pgfpathlineto{\pgfpoint{52.703780\du}{14.216324\du}}
\pgfpathlineto{\pgfpoint{52.719480\du}{14.216324\du}}
\pgfpathlineto{\pgfpoint{52.719480\du}{14.216324\du}}
\pgfpathlineto{\pgfpoint{52.719480\du}{14.216324\du}}
\pgfpathlineto{\pgfpoint{52.720940\du}{14.216324\du}}
\pgfpathlineto{\pgfpoint{52.722035\du}{14.216324\du}}
\pgfpathlineto{\pgfpoint{52.723131\du}{14.216324\du}}
\pgfpathlineto{\pgfpoint{52.724226\du}{14.215594\du}}
\pgfpathlineto{\pgfpoint{52.725321\du}{14.215229\du}}
\pgfpathlineto{\pgfpoint{52.726416\du}{14.214499\du}}
\pgfpathlineto{\pgfpoint{52.726416\du}{14.214134\du}}
\pgfpathlineto{\pgfpoint{52.727147\du}{14.213404\du}}
\pgfpathlineto{\pgfpoint{52.728242\du}{14.211578\du}}
\pgfpathlineto{\pgfpoint{52.729337\du}{14.209753\du}}
\pgfpathlineto{\pgfpoint{52.729702\du}{14.208292\du}}
\pgfpathlineto{\pgfpoint{52.729702\du}{14.206467\du}}
\pgfpathlineto{\pgfpoint{52.729702\du}{14.203911\du}}
\pgfpathlineto{\pgfpoint{52.729337\du}{14.202451\du}}
\pgfpathlineto{\pgfpoint{52.728242\du}{14.200625\du}}
\pgfpathlineto{\pgfpoint{52.727147\du}{14.199530\du}}
\pgfpathlineto{\pgfpoint{52.726416\du}{14.198070\du}}
\pgfpathlineto{\pgfpoint{52.726416\du}{14.197704\du}}
\pgfpathlineto{\pgfpoint{52.725321\du}{14.196974\du}}
\pgfpathlineto{\pgfpoint{52.724226\du}{14.196974\du}}
\pgfpathlineto{\pgfpoint{52.723131\du}{14.196244\du}}
\pgfpathlineto{\pgfpoint{52.722035\du}{14.196244\du}}
\pgfpathlineto{\pgfpoint{52.720940\du}{14.195879\du}}
\pgfpathlineto{\pgfpoint{52.719480\du}{14.195879\du}}
\pgfusepath{fill}
\pgfsetbuttcap
\pgfsetmiterjoin
\pgfsetdash{}{0pt}
\definecolor{dialinecolor}{rgb}{0.678431, 0.839216, 0.905882}
\pgfsetfillcolor{dialinecolor}
\pgfpathmoveto{\pgfpoint{53.344525\du}{14.541626\du}}
\pgfpathlineto{\pgfpoint{53.344525\du}{14.532133\du}}
\pgfpathlineto{\pgfpoint{53.343795\du}{14.523371\du}}
\pgfpathlineto{\pgfpoint{53.342700\du}{14.514243\du}}
\pgfpathlineto{\pgfpoint{53.341239\du}{14.505481\du}}
\pgfpathlineto{\pgfpoint{53.339049\du}{14.496719\du}}
\pgfpathlineto{\pgfpoint{53.337223\du}{14.487956\du}}
\pgfpathlineto{\pgfpoint{53.334303\du}{14.479194\du}}
\pgfpathlineto{\pgfpoint{53.331747\du}{14.470432\du}}
\pgfpathlineto{\pgfpoint{53.327731\du}{14.462034\du}}
\pgfpathlineto{\pgfpoint{53.324445\du}{14.453272\du}}
\pgfpathlineto{\pgfpoint{53.320064\du}{14.445240\du}}
\pgfpathlineto{\pgfpoint{53.315317\du}{14.436843\du}}
\pgfpathlineto{\pgfpoint{53.310571\du}{14.428811\du}}
\pgfpathlineto{\pgfpoint{53.305825\du}{14.420413\du}}
\pgfpathlineto{\pgfpoint{53.299983\du}{14.412746\du}}
\pgfpathlineto{\pgfpoint{53.293777\du}{14.404714\du}}
\pgfpathlineto{\pgfpoint{53.287570\du}{14.397047\du}}
\pgfpathlineto{\pgfpoint{53.280998\du}{14.389380\du}}
\pgfpathlineto{\pgfpoint{53.274427\du}{14.382078\du}}
\pgfpathlineto{\pgfpoint{53.266760\du}{14.374411\du}}
\pgfpathlineto{\pgfpoint{53.259823\du}{14.367109\du}}
\pgfpathlineto{\pgfpoint{53.251791\du}{14.360173\du}}
\pgfpathlineto{\pgfpoint{53.243759\du}{14.353236\du}}
\pgfpathlineto{\pgfpoint{53.235361\du}{14.346299\du}}
\pgfpathlineto{\pgfpoint{53.226964\du}{14.339362\du}}
\pgfpathlineto{\pgfpoint{53.218202\du}{14.332790\du}}
\pgfpathlineto{\pgfpoint{53.209074\du}{14.326219\du}}
\pgfpathlineto{\pgfpoint{53.199582\du}{14.320012\du}}
\pgfpathlineto{\pgfpoint{53.189359\du}{14.313440\du}}
\pgfpathlineto{\pgfpoint{53.179501\du}{14.307599\du}}
\pgfpathlineto{\pgfpoint{53.169279\du}{14.301027\du}}
\pgfpathlineto{\pgfpoint{53.159421\du}{14.295185\du}}
\pgfpathlineto{\pgfpoint{53.148103\du}{14.290074\du}}
\pgfpathlineto{\pgfpoint{53.137150\du}{14.284232\du}}
\pgfpathlineto{\pgfpoint{53.125832\du}{14.278756\du}}
\pgfpathlineto{\pgfpoint{53.114514\du}{14.273645\du}}
\pgfpathlineto{\pgfpoint{53.102831\du}{14.268533\du}}
\pgfpathlineto{\pgfpoint{53.091513\du}{14.263787\du}}
\pgfpathlineto{\pgfpoint{53.078735\du}{14.258311\du}}
\pgfpathlineto{\pgfpoint{53.067052\du}{14.253929\du}}
\pgfpathlineto{\pgfpoint{53.054273\du}{14.249548\du}}
\pgfpathlineto{\pgfpoint{53.041495\du}{14.245167\du}}
\pgfpathlineto{\pgfpoint{53.028351\du}{14.240786\du}}
\pgfpathlineto{\pgfpoint{53.015208\du}{14.236770\du}}
\pgfpathlineto{\pgfpoint{53.002064\du}{14.233119\du}}
\pgfpathlineto{\pgfpoint{52.988556\du}{14.229103\du}}
\pgfpathlineto{\pgfpoint{52.974682\du}{14.225817\du}}
\pgfpathlineto{\pgfpoint{52.961174\du}{14.222166\du}}
\pgfpathlineto{\pgfpoint{52.947300\du}{14.219245\du}}
\pgfpathlineto{\pgfpoint{52.932696\du}{14.216324\du}}
\pgfpathlineto{\pgfpoint{52.918822\du}{14.214134\du}}
\pgfpathlineto{\pgfpoint{52.904584\du}{14.211213\du}}
\pgfpathlineto{\pgfpoint{52.889250\du}{14.208657\du}}
\pgfpathlineto{\pgfpoint{52.874281\du}{14.206467\du}}
\pgfpathlineto{\pgfpoint{52.859312\du}{14.204641\du}}
\pgfpathlineto{\pgfpoint{52.845073\du}{14.202816\du}}
\pgfpathlineto{\pgfpoint{52.829009\du}{14.200990\du}}
\pgfpathlineto{\pgfpoint{52.814040\du}{14.199895\du}}
\pgfpathlineto{\pgfpoint{52.798706\du}{14.198800\du}}
\pgfpathlineto{\pgfpoint{52.783006\du}{14.197704\du}}
\pgfpathlineto{\pgfpoint{52.767672\du}{14.196974\du}}
\pgfpathlineto{\pgfpoint{52.751608\du}{14.196244\du}}
\pgfpathlineto{\pgfpoint{52.735544\du}{14.195879\du}}
\pgfpathlineto{\pgfpoint{52.719480\du}{14.195879\du}}
\pgfpathlineto{\pgfpoint{52.719480\du}{14.216324\du}}
\pgfpathlineto{\pgfpoint{52.735544\du}{14.216324\du}}
\pgfpathlineto{\pgfpoint{52.750878\du}{14.217055\du}}
\pgfpathlineto{\pgfpoint{52.766577\du}{14.217420\du}}
\pgfpathlineto{\pgfpoint{52.781911\du}{14.218150\du}}
\pgfpathlineto{\pgfpoint{52.796880\du}{14.219245\du}}
\pgfpathlineto{\pgfpoint{52.812579\du}{14.220340\du}}
\pgfpathlineto{\pgfpoint{52.827183\du}{14.221436\du}}
\pgfpathlineto{\pgfpoint{52.842517\du}{14.223261\du}}
\pgfpathlineto{\pgfpoint{52.857486\du}{14.225087\du}}
\pgfpathlineto{\pgfpoint{52.872090\du}{14.226912\du}}
\pgfpathlineto{\pgfpoint{52.886329\du}{14.229103\du}}
\pgfpathlineto{\pgfpoint{52.900568\du}{14.231658\du}}
\pgfpathlineto{\pgfpoint{52.914806\du}{14.233849\du}}
\pgfpathlineto{\pgfpoint{52.928680\du}{14.236770\du}}
\pgfpathlineto{\pgfpoint{52.942554\du}{14.239691\du}}
\pgfpathlineto{\pgfpoint{52.956427\du}{14.242611\du}}
\pgfpathlineto{\pgfpoint{52.969936\du}{14.245532\du}}
\pgfpathlineto{\pgfpoint{52.983445\du}{14.249183\du}}
\pgfpathlineto{\pgfpoint{52.996588\du}{14.252469\du}}
\pgfpathlineto{\pgfpoint{53.009366\du}{14.256850\du}}
\pgfpathlineto{\pgfpoint{53.022875\du}{14.260136\du}}
\pgfpathlineto{\pgfpoint{53.035653\du}{14.264152\du}}
\pgfpathlineto{\pgfpoint{53.047336\du}{14.268898\du}}
\pgfpathlineto{\pgfpoint{53.059750\du}{14.272914\du}}
\pgfpathlineto{\pgfpoint{53.071798\du}{14.277661\du}}
\pgfpathlineto{\pgfpoint{53.083481\du}{14.282407\du}}
\pgfpathlineto{\pgfpoint{53.094799\du}{14.287153\du}}
\pgfpathlineto{\pgfpoint{53.106117\du}{14.292265\du}}
\pgfpathlineto{\pgfpoint{53.117800\du}{14.297741\du}}
\pgfpathlineto{\pgfpoint{53.128388\du}{14.302487\du}}
\pgfpathlineto{\pgfpoint{53.138976\du}{14.308329\du}}
\pgfpathlineto{\pgfpoint{53.149564\du}{14.313440\du}}
\pgfpathlineto{\pgfpoint{53.159421\du}{14.319282\du}}
\pgfpathlineto{\pgfpoint{53.169279\du}{14.325123\du}}
\pgfpathlineto{\pgfpoint{53.178406\du}{14.330965\du}}
\pgfpathlineto{\pgfpoint{53.187534\du}{14.336806\du}}
\pgfpathlineto{\pgfpoint{53.197026\du}{14.343378\du}}
\pgfpathlineto{\pgfpoint{53.205788\du}{14.349220\du}}
\pgfpathlineto{\pgfpoint{53.214916\du}{14.355426\du}}
\pgfpathlineto{\pgfpoint{53.222948\du}{14.361998\du}}
\pgfpathlineto{\pgfpoint{53.231345\du}{14.368570\du}}
\pgfpathlineto{\pgfpoint{53.238282\du}{14.375507\du}}
\pgfpathlineto{\pgfpoint{53.245584\du}{14.382443\du}}
\pgfpathlineto{\pgfpoint{53.252521\du}{14.389015\du}}
\pgfpathlineto{\pgfpoint{53.259823\du}{14.395952\du}}
\pgfpathlineto{\pgfpoint{53.265664\du}{14.402889\du}}
\pgfpathlineto{\pgfpoint{53.272601\du}{14.409826\du}}
\pgfpathlineto{\pgfpoint{53.277713\du}{14.417493\du}}
\pgfpathlineto{\pgfpoint{53.283189\du}{14.424795\du}}
\pgfpathlineto{\pgfpoint{53.287935\du}{14.432097\du}}
\pgfpathlineto{\pgfpoint{53.293047\du}{14.439764\du}}
\pgfpathlineto{\pgfpoint{53.297063\du}{14.446700\du}}
\pgfpathlineto{\pgfpoint{53.301809\du}{14.454367\du}}
\pgfpathlineto{\pgfpoint{53.305825\du}{14.462034\du}}
\pgfpathlineto{\pgfpoint{53.309111\du}{14.470432\du}}
\pgfpathlineto{\pgfpoint{53.311667\du}{14.477734\du}}
\pgfpathlineto{\pgfpoint{53.314952\du}{14.485401\du}}
\pgfpathlineto{\pgfpoint{53.317143\du}{14.493068\du}}
\pgfpathlineto{\pgfpoint{53.319699\du}{14.501100\du}}
\pgfpathlineto{\pgfpoint{53.321159\du}{14.509497\du}}
\pgfpathlineto{\pgfpoint{53.322619\du}{14.517164\du}}
\pgfpathlineto{\pgfpoint{53.323350\du}{14.525196\du}}
\pgfpathlineto{\pgfpoint{53.323715\du}{14.533593\du}}
\pgfpathlineto{\pgfpoint{53.324445\du}{14.541626\du}}
\pgfpathlineto{\pgfpoint{53.344525\du}{14.541626\du}}
\pgfusepath{fill}
\pgfsetbuttcap
\pgfsetmiterjoin
\pgfsetdash{}{0pt}
\definecolor{dialinecolor}{rgb}{0.074510, 0.082353, 0.086275}
\pgfsetfillcolor{dialinecolor}
\pgfpathmoveto{\pgfpoint{52.398560\du}{14.635821\du}}
\pgfpathlineto{\pgfpoint{52.626015\du}{14.407635\du}}
\pgfpathlineto{\pgfpoint{52.566139\du}{14.346664\du}}
\pgfpathlineto{\pgfpoint{52.746497\du}{14.346664\du}}
\pgfpathlineto{\pgfpoint{52.746497\du}{14.535054\du}}
\pgfpathlineto{\pgfpoint{52.686256\du}{14.474813\du}}
\pgfpathlineto{\pgfpoint{52.466102\du}{14.696062\du}}
\pgfpathlineto{\pgfpoint{52.398560\du}{14.635821\du}}
\pgfusepath{fill}
\pgfsetbuttcap
\pgfsetmiterjoin
\pgfsetdash{}{0pt}
\definecolor{dialinecolor}{rgb}{0.074510, 0.082353, 0.086275}
\pgfsetfillcolor{dialinecolor}
\pgfpathmoveto{\pgfpoint{52.666541\du}{14.756302\du}}
\pgfpathlineto{\pgfpoint{52.893631\du}{14.528117\du}}
\pgfpathlineto{\pgfpoint{52.833025\du}{14.467876\du}}
\pgfpathlineto{\pgfpoint{53.014113\du}{14.467876\du}}
\pgfpathlineto{\pgfpoint{53.014113\du}{14.655901\du}}
\pgfpathlineto{\pgfpoint{52.953507\du}{14.595660\du}}
\pgfpathlineto{\pgfpoint{52.732988\du}{14.816543\du}}
\pgfpathlineto{\pgfpoint{52.666541\du}{14.756302\du}}
\pgfusepath{fill}
\pgfsetbuttcap
\pgfsetmiterjoin
\pgfsetdash{}{0pt}
\definecolor{dialinecolor}{rgb}{1.000000, 1.000000, 1.000000}
\pgfsetfillcolor{dialinecolor}
\pgfpathmoveto{\pgfpoint{52.385416\du}{14.622312\du}}
\pgfpathlineto{\pgfpoint{52.612506\du}{14.394127\du}}
\pgfpathlineto{\pgfpoint{52.552995\du}{14.333886\du}}
\pgfpathlineto{\pgfpoint{52.732988\du}{14.333886\du}}
\pgfpathlineto{\pgfpoint{52.732988\du}{14.521910\du}}
\pgfpathlineto{\pgfpoint{52.673112\du}{14.460939\du}}
\pgfpathlineto{\pgfpoint{52.452594\du}{14.682553\du}}
\pgfpathlineto{\pgfpoint{52.385416\du}{14.622312\du}}
\pgfusepath{fill}
\pgfsetbuttcap
\pgfsetmiterjoin
\pgfsetdash{}{0pt}
\definecolor{dialinecolor}{rgb}{1.000000, 1.000000, 1.000000}
\pgfsetfillcolor{dialinecolor}
\pgfpathmoveto{\pgfpoint{52.653032\du}{14.742794\du}}
\pgfpathlineto{\pgfpoint{52.880122\du}{14.514608\du}}
\pgfpathlineto{\pgfpoint{52.819881\du}{14.454367\du}}
\pgfpathlineto{\pgfpoint{53.000239\du}{14.454367\du}}
\pgfpathlineto{\pgfpoint{53.000239\du}{14.642392\du}}
\pgfpathlineto{\pgfpoint{52.940728\du}{14.582151\du}}
\pgfpathlineto{\pgfpoint{52.719480\du}{14.803035\du}}
\pgfpathlineto{\pgfpoint{52.653032\du}{14.742794\du}}
\pgfusepath{fill}
\pgfsetlinewidth{0.000000\du}
\pgfsetdash{}{0pt}
\pgfsetdash{}{0pt}
\pgfsetbuttcap
{
\definecolor{dialinecolor}{rgb}{0.000000, 0.000000, 0.000000}
\pgfsetfillcolor{dialinecolor}
% was here!!!
\definecolor{dialinecolor}{rgb}{0.000000, 0.000000, 0.000000}
\pgfsetstrokecolor{dialinecolor}
\pgfpathmoveto{\pgfpoint{-7.136441\du}{15.345555\du}}
\pgfpatharc{101}{79}{20.615978\du and 20.615978\du}
\pgfusepath{stroke}
}
% setfont left to latex
\definecolor{dialinecolor}{rgb}{0.000000, 0.000000, 0.000000}
\pgfsetstrokecolor{dialinecolor}
\node[anchor=west] at (-9.796758\du,18.107798\du){Lien : ('89.105.200.57', '185.147.12.31')};
% setfont left to latex
\definecolor{dialinecolor}{rgb}{0.000000, 0.000000, 0.000000}
\pgfsetstrokecolor{dialinecolor}
\node[anchor=west] at (-7.896758\du,19.157798\du){RTT Diférentiel :  \ensuremath{[}1.463\ensuremath{]}};
% setfont left to latex
\definecolor{dialinecolor}{rgb}{0.000000, 0.000000, 0.000000}
\pgfsetstrokecolor{dialinecolor}
\node[anchor=west] at (-7.846758\du,20.057798\du){Probe': \ensuremath{[}'89.105.202.4'\ensuremath{]}};
% setfont left to latex
\definecolor{dialinecolor}{rgb}{0.000000, 0.000000, 0.000000}
\pgfsetstrokecolor{dialinecolor}
\node[anchor=west] at (-7.846758\du,21.007798\du){msmId': \{5004: \ensuremath{[}4247\ensuremath{]}\}};
\pgfsetlinewidth{0.000000\du}
\pgfsetdash{}{0pt}
\pgfsetdash{}{0pt}
\pgfsetbuttcap
{
\definecolor{dialinecolor}{rgb}{0.000000, 0.000000, 0.000000}
\pgfsetfillcolor{dialinecolor}
% was here!!!
\definecolor{dialinecolor}{rgb}{0.000000, 0.000000, 0.000000}
\pgfsetstrokecolor{dialinecolor}
\pgfpathmoveto{\pgfpoint{12.571091\du}{15.084705\du}}
\pgfpatharc{99}{83}{29.909945\du and 29.909945\du}
\pgfusepath{stroke}
}
% setfont left to latex
\definecolor{dialinecolor}{rgb}{0.000000, 0.000000, 0.000000}
\pgfsetstrokecolor{dialinecolor}
\node[anchor=west] at (12.766994\du,17.969607\du){Lien : ('185.147.12.19', '160.242.100.88') };
\pgfsetlinewidth{0.000000\du}
\pgfsetdash{}{0pt}
\pgfsetdash{}{0pt}
\pgfsetbuttcap
{
\definecolor{dialinecolor}{rgb}{0.000000, 0.000000, 0.000000}
\pgfsetfillcolor{dialinecolor}
% was here!!!
\definecolor{dialinecolor}{rgb}{0.000000, 0.000000, 0.000000}
\pgfsetstrokecolor{dialinecolor}
\pgfpathmoveto{\pgfpoint{23.802504\du}{15.541138\du}}
\pgfpatharc{104}{76}{15.817742\du and 15.817742\du}
\pgfusepath{stroke}
}
% setfont left to latex
\definecolor{dialinecolor}{rgb}{0.000000, 0.000000, 0.000000}
\pgfsetstrokecolor{dialinecolor}
\node[anchor=west] at (24.303484\du,24.998597\du){Lien : ('160.242.100.88', '196.216.48.144')};
% setfont left to latex
\definecolor{dialinecolor}{rgb}{0.000000, 0.000000, 0.000000}
\pgfsetstrokecolor{dialinecolor}
\node[anchor=west] at (26.672291\du,25.917836\du){RTT Différentiel: \ensuremath{[}2.404\ensuremath{]}};
% setfont left to latex
\definecolor{dialinecolor}{rgb}{0.000000, 0.000000, 0.000000}
\pgfsetstrokecolor{dialinecolor}
\node[anchor=west] at (26.601581\du,26.943141\du){probe: \ensuremath{[}'89.105.202.4'\ensuremath{]}};
% setfont left to latex
\definecolor{dialinecolor}{rgb}{0.000000, 0.000000, 0.000000}
\pgfsetstrokecolor{dialinecolor}
\node[anchor=west] at (26.601581\du,27.968445\du){msmId:  \{5004: \ensuremath{[}4247\ensuremath{]}\}};
% setfont left to latex
\definecolor{dialinecolor}{rgb}{0.000000, 0.000000, 0.000000}
\pgfsetstrokecolor{dialinecolor}
\node[anchor=west] at (31.455559\du,18.012268\du){Lien : ('196.216.48.144', '193.239.116.112')};
% setfont left to latex
\definecolor{dialinecolor}{rgb}{0.000000, 0.000000, 0.000000}
\pgfsetstrokecolor{dialinecolor}
\node[anchor=west] at (32.266172\du,18.964591\du){RTT différentiel : \ensuremath{[}1.6290000000000004\ensuremath{]}};
% setfont left to latex
\definecolor{dialinecolor}{rgb}{0.000000, 0.000000, 0.000000}
\pgfsetstrokecolor{dialinecolor}
\node[anchor=west] at (34.339166\du,19.892586\du){probe: \ensuremath{[}'89.105.202.4'\ensuremath{]}};
% setfont left to latex
\definecolor{dialinecolor}{rgb}{0.000000, 0.000000, 0.000000}
\pgfsetstrokecolor{dialinecolor}
\node[anchor=west] at (34.540270\du,20.977574\du){msmId: \{5004: \ensuremath{[}4247\ensuremath{]}\}};
\pgfsetlinewidth{0.000000\du}
\pgfsetdash{}{0pt}
\pgfsetdash{}{0pt}
\pgfsetbuttcap
{
\definecolor{dialinecolor}{rgb}{0.000000, 0.000000, 0.000000}
\pgfsetfillcolor{dialinecolor}
% was here!!!
\definecolor{dialinecolor}{rgb}{0.000000, 0.000000, 0.000000}
\pgfsetstrokecolor{dialinecolor}
\pgfpathmoveto{\pgfpoint{33.689906\du}{15.671511\du}}
\pgfpatharc{105}{76}{16.177941\du and 16.177941\du}
\pgfusepath{stroke}
}
% setfont left to latex
\definecolor{dialinecolor}{rgb}{0.000000, 0.000000, 0.000000}
\pgfsetstrokecolor{dialinecolor}
\node[anchor=west] at (42.551919\du,24.979239\du){Lien : ('193.239.116.112', '192.5.5.241')};
\pgfsetlinewidth{0.000000\du}
\pgfsetdash{}{0pt}
\pgfsetdash{}{0pt}
\pgfsetbuttcap
{
\definecolor{dialinecolor}{rgb}{0.000000, 0.000000, 0.000000}
\pgfsetfillcolor{dialinecolor}
% was here!!!
\definecolor{dialinecolor}{rgb}{0.000000, 0.000000, 0.000000}
\pgfsetstrokecolor{dialinecolor}
\pgfpathmoveto{\pgfpoint{43.208020\du}{15.606244\du}}
\pgfpatharc{102}{79}{23.954656\du and 23.954656\du}
\pgfusepath{stroke}
}
% setfont left to latex
\definecolor{dialinecolor}{rgb}{0.000000, 0.000000, 0.000000}
\pgfsetstrokecolor{dialinecolor}
\node[anchor=west] at (42.486723\du,25.909506\du){RTT Différentiel :  \ensuremath{[}0.5169999999999995\ensuremath{]}};
% setfont left to latex
\definecolor{dialinecolor}{rgb}{0.000000, 0.000000, 0.000000}
\pgfsetstrokecolor{dialinecolor}
\node[anchor=west] at (45.320020\du,26.828745\du){probe: \ensuremath{[}'89.105.202.4'\ensuremath{]}};
% setfont left to latex
\definecolor{dialinecolor}{rgb}{0.000000, 0.000000, 0.000000}
\pgfsetstrokecolor{dialinecolor}
\node[anchor=west] at (45.355376\du,27.889405\du){msmId': \{5004: \ensuremath{[}4247\ensuremath{]}\}};
\pgfsetlinewidth{0.000000\du}
\pgfsetdash{}{0pt}
\pgfsetdash{}{0pt}
\pgfsetbuttcap
\pgfsetmiterjoin
\pgfsetlinewidth{0.000000\du}
\pgfsetbuttcap
\pgfsetmiterjoin
\pgfsetdash{}{0pt}
\definecolor{dialinecolor}{rgb}{0.027451, 0.486275, 0.682353}
\pgfsetfillcolor{dialinecolor}
\pgfpathmoveto{\pgfpoint{12.707443\du}{14.124550\du}}
\pgfpathlineto{\pgfpoint{12.705983\du}{14.153758\du}}
\pgfpathlineto{\pgfpoint{12.698681\du}{14.183696\du}}
\pgfpathlineto{\pgfpoint{12.688458\du}{14.212173\du}}
\pgfpathlineto{\pgfpoint{12.673855\du}{14.240286\du}}
\pgfpathlineto{\pgfpoint{12.654504\du}{14.268398\du}}
\pgfpathlineto{\pgfpoint{12.632599\du}{14.295781\du}}
\pgfpathlineto{\pgfpoint{12.605947\du}{14.322798\du}}
\pgfpathlineto{\pgfpoint{12.575278\du}{14.349085\du}}
\pgfpathlineto{\pgfpoint{12.542420\du}{14.374276\du}}
\pgfpathlineto{\pgfpoint{12.505180\du}{14.399468\du}}
\pgfpathlineto{\pgfpoint{12.464289\du}{14.423565\du}}
\pgfpathlineto{\pgfpoint{12.420477\du}{14.446931\du}}
\pgfpathlineto{\pgfpoint{12.374110\du}{14.469567\du}}
\pgfpathlineto{\pgfpoint{12.323727\du}{14.491472\du}}
\pgfpathlineto{\pgfpoint{12.270788\du}{14.512283\du}}
\pgfpathlineto{\pgfpoint{12.215293\du}{14.532363\du}}
\pgfpathlineto{\pgfpoint{12.157243\du}{14.551713\du}}
\pgfpathlineto{\pgfpoint{12.096637\du}{14.569603\du}}
\pgfpathlineto{\pgfpoint{12.032745\du}{14.586763\du}}
\pgfpathlineto{\pgfpoint{11.967757\du}{14.603192\du}}
\pgfpathlineto{\pgfpoint{11.899119\du}{14.618161\du}}
\pgfpathlineto{\pgfpoint{11.828290\du}{14.631670\du}}
\pgfpathlineto{\pgfpoint{11.756366\du}{14.644448\du}}
\pgfpathlineto{\pgfpoint{11.681522\du}{14.656496\du}}
\pgfpathlineto{\pgfpoint{11.605581\du}{14.666354\du}}
\pgfpathlineto{\pgfpoint{11.527086\du}{14.675481\du}}
\pgfpathlineto{\pgfpoint{11.447495\du}{14.683148\du}}
\pgfpathlineto{\pgfpoint{11.366443\du}{14.689720\du}}
\pgfpathlineto{\pgfpoint{11.283201\du}{14.694831\du}}
\pgfpathlineto{\pgfpoint{11.199594\du}{14.698482\du}}
\pgfpathlineto{\pgfpoint{11.114161\du}{14.700673\du}}
\pgfpathlineto{\pgfpoint{11.027633\du}{14.701403\du}}
\pgfpathlineto{\pgfpoint{10.941470\du}{14.700673\du}}
\pgfpathlineto{\pgfpoint{10.855673\du}{14.698482\du}}
\pgfpathlineto{\pgfpoint{10.772066\du}{14.694831\du}}
\pgfpathlineto{\pgfpoint{10.689189\du}{14.689720\du}}
\pgfpathlineto{\pgfpoint{10.607772\du}{14.683148\du}}
\pgfpathlineto{\pgfpoint{10.528181\du}{14.675481\du}}
\pgfpathlineto{\pgfpoint{10.450415\du}{14.666354\du}}
\pgfpathlineto{\pgfpoint{10.373745\du}{14.656496\du}}
\pgfpathlineto{\pgfpoint{10.299630\du}{14.644448\du}}
\pgfpathlineto{\pgfpoint{10.226976\du}{14.631670\du}}
\pgfpathlineto{\pgfpoint{10.156512\du}{14.618161\du}}
\pgfpathlineto{\pgfpoint{10.087874\du}{14.603192\du}}
\pgfpathlineto{\pgfpoint{10.022157\du}{14.586763\du}}
\pgfpathlineto{\pgfpoint{9.958630\du}{14.569603\du}}
\pgfpathlineto{\pgfpoint{9.897659\du}{14.551713\du}}
\pgfpathlineto{\pgfpoint{9.839243\du}{14.532363\du}}
\pgfpathlineto{\pgfpoint{9.784114\du}{14.512283\du}}
\pgfpathlineto{\pgfpoint{9.731175\du}{14.491472\du}}
\pgfpathlineto{\pgfpoint{9.681156\du}{14.469567\du}}
\pgfpathlineto{\pgfpoint{9.634059\du}{14.446931\du}}
\pgfpathlineto{\pgfpoint{9.590612\du}{14.423565\du}}
\pgfpathlineto{\pgfpoint{9.549722\du}{14.399468\du}}
\pgfpathlineto{\pgfpoint{9.512482\du}{14.374276\du}}
\pgfpathlineto{\pgfpoint{9.479258\du}{14.349085\du}}
\pgfpathlineto{\pgfpoint{9.448955\du}{14.322798\du}}
\pgfpathlineto{\pgfpoint{9.422303\du}{14.295781\du}}
\pgfpathlineto{\pgfpoint{9.400397\du}{14.268398\du}}
\pgfpathlineto{\pgfpoint{9.381047\du}{14.240286\du}}
\pgfpathlineto{\pgfpoint{9.366443\du}{14.212173\du}}
\pgfpathlineto{\pgfpoint{9.355855\du}{14.183696\du}}
\pgfpathlineto{\pgfpoint{9.348918\du}{14.153758\du}}
\pgfpathlineto{\pgfpoint{9.347093\du}{14.124550\du}}
\pgfpathlineto{\pgfpoint{9.348918\du}{14.094612\du}}
\pgfpathlineto{\pgfpoint{9.355855\du}{14.065405\du}}
\pgfpathlineto{\pgfpoint{9.366443\du}{14.036197\du}}
\pgfpathlineto{\pgfpoint{9.381047\du}{14.008084\du}}
\pgfpathlineto{\pgfpoint{9.400397\du}{13.979972\du}}
\pgfpathlineto{\pgfpoint{9.422303\du}{13.952590\du}}
\pgfpathlineto{\pgfpoint{9.448955\du}{13.925938\du}}
\pgfpathlineto{\pgfpoint{9.479258\du}{13.899651\du}}
\pgfpathlineto{\pgfpoint{9.512482\du}{13.874094\du}}
\pgfpathlineto{\pgfpoint{9.549722\du}{13.849267\du}}
\pgfpathlineto{\pgfpoint{9.590612\du}{13.824806\du}}
\pgfpathlineto{\pgfpoint{9.634059\du}{13.801440\du}}
\pgfpathlineto{\pgfpoint{9.681156\du}{13.779169\du}}
\pgfpathlineto{\pgfpoint{9.731175\du}{13.756898\du}}
\pgfpathlineto{\pgfpoint{9.784114\du}{13.736087\du}}
\pgfpathlineto{\pgfpoint{9.839243\du}{13.716007\du}}
\pgfpathlineto{\pgfpoint{9.897659\du}{13.697387\du}}
\pgfpathlineto{\pgfpoint{9.958630\du}{13.678402\du}}
\pgfpathlineto{\pgfpoint{10.022157\du}{13.661608\du}}
\pgfpathlineto{\pgfpoint{10.087874\du}{13.645908\du}}
\pgfpathlineto{\pgfpoint{10.156512\du}{13.630574\du}}
\pgfpathlineto{\pgfpoint{10.226976\du}{13.616701\du}}
\pgfpathlineto{\pgfpoint{10.299630\du}{13.603557\du}}
\pgfpathlineto{\pgfpoint{10.373745\du}{13.592604\du}}
\pgfpathlineto{\pgfpoint{10.450415\du}{13.582016\du}}
\pgfpathlineto{\pgfpoint{10.528181\du}{13.572524\du}}
\pgfpathlineto{\pgfpoint{10.607772\du}{13.565222\du}}
\pgfpathlineto{\pgfpoint{10.689189\du}{13.558650\du}}
\pgfpathlineto{\pgfpoint{10.772066\du}{13.553539\du}}
\pgfpathlineto{\pgfpoint{10.855673\du}{13.549888\du}}
\pgfpathlineto{\pgfpoint{10.941470\du}{13.548062\du}}
\pgfpathlineto{\pgfpoint{11.027633\du}{13.546967\du}}
\pgfpathlineto{\pgfpoint{11.114161\du}{13.548062\du}}
\pgfpathlineto{\pgfpoint{11.199594\du}{13.549888\du}}
\pgfpathlineto{\pgfpoint{11.283201\du}{13.553539\du}}
\pgfpathlineto{\pgfpoint{11.366443\du}{13.558650\du}}
\pgfpathlineto{\pgfpoint{11.447495\du}{13.565222\du}}
\pgfpathlineto{\pgfpoint{11.527086\du}{13.572524\du}}
\pgfpathlineto{\pgfpoint{11.605581\du}{13.582016\du}}
\pgfpathlineto{\pgfpoint{11.681522\du}{13.592604\du}}
\pgfpathlineto{\pgfpoint{11.756366\du}{13.603557\du}}
\pgfpathlineto{\pgfpoint{11.828290\du}{13.616701\du}}
\pgfpathlineto{\pgfpoint{11.899119\du}{13.630574\du}}
\pgfpathlineto{\pgfpoint{11.967757\du}{13.645908\du}}
\pgfpathlineto{\pgfpoint{12.032745\du}{13.661608\du}}
\pgfpathlineto{\pgfpoint{12.096637\du}{13.678402\du}}
\pgfpathlineto{\pgfpoint{12.157243\du}{13.697387\du}}
\pgfpathlineto{\pgfpoint{12.215293\du}{13.716007\du}}
\pgfpathlineto{\pgfpoint{12.270788\du}{13.736087\du}}
\pgfpathlineto{\pgfpoint{12.323727\du}{13.756898\du}}
\pgfpathlineto{\pgfpoint{12.374110\du}{13.779169\du}}
\pgfpathlineto{\pgfpoint{12.420477\du}{13.801440\du}}
\pgfpathlineto{\pgfpoint{12.464289\du}{13.824806\du}}
\pgfpathlineto{\pgfpoint{12.505180\du}{13.849267\du}}
\pgfpathlineto{\pgfpoint{12.542420\du}{13.874094\du}}
\pgfpathlineto{\pgfpoint{12.575278\du}{13.899651\du}}
\pgfpathlineto{\pgfpoint{12.605947\du}{13.925938\du}}
\pgfpathlineto{\pgfpoint{12.632599\du}{13.952590\du}}
\pgfpathlineto{\pgfpoint{12.654504\du}{13.979972\du}}
\pgfpathlineto{\pgfpoint{12.673855\du}{14.008084\du}}
\pgfpathlineto{\pgfpoint{12.688458\du}{14.036197\du}}
\pgfpathlineto{\pgfpoint{12.698681\du}{14.065405\du}}
\pgfpathlineto{\pgfpoint{12.705983\du}{14.094612\du}}
\pgfpathlineto{\pgfpoint{12.707443\du}{14.124550\du}}
\pgfusepath{fill}
\pgfsetlinewidth{0.000000\du}
\pgfsetbuttcap
\pgfsetmiterjoin
\pgfsetdash{}{0pt}
\definecolor{dialinecolor}{rgb}{0.678431, 0.839216, 0.905882}
\pgfsetfillcolor{dialinecolor}
\pgfpathmoveto{\pgfpoint{11.027633\du}{14.711991\du}}
\pgfpathlineto{\pgfpoint{11.027633\du}{14.711991\du}}
\pgfpathlineto{\pgfpoint{11.071080\du}{14.711991\du}}
\pgfpathlineto{\pgfpoint{11.114526\du}{14.711261\du}}
\pgfpathlineto{\pgfpoint{11.157608\du}{14.710165\du}}
\pgfpathlineto{\pgfpoint{11.199594\du}{14.709070\du}}
\pgfpathlineto{\pgfpoint{11.242310\du}{14.707245\du}}
\pgfpathlineto{\pgfpoint{11.283931\du}{14.705054\du}}
\pgfpathlineto{\pgfpoint{11.325552\du}{14.702498\du}}
\pgfpathlineto{\pgfpoint{11.367173\du}{14.700308\du}}
\pgfpathlineto{\pgfpoint{11.407699\du}{14.697387\du}}
\pgfpathlineto{\pgfpoint{11.448590\du}{14.693736\du}}
\pgfpathlineto{\pgfpoint{11.488385\du}{14.689720\du}}
\pgfpathlineto{\pgfpoint{11.528546\du}{14.685704\du}}
\pgfpathlineto{\pgfpoint{11.567246\du}{14.681323\du}}
\pgfpathlineto{\pgfpoint{11.606677\du}{14.676942\du}}
\pgfpathlineto{\pgfpoint{11.644647\du}{14.671465\du}}
\pgfpathlineto{\pgfpoint{11.683712\du}{14.666354\du}}
\pgfpathlineto{\pgfpoint{11.720952\du}{14.660877\du}}
\pgfpathlineto{\pgfpoint{11.757827\du}{14.654671\du}}
\pgfpathlineto{\pgfpoint{11.794336\du}{14.648829\du}}
\pgfpathlineto{\pgfpoint{11.830846\du}{14.642257\du}}
\pgfpathlineto{\pgfpoint{11.866261\du}{14.635321\du}}
\pgfpathlineto{\pgfpoint{11.900945\du}{14.628384\du}}
\pgfpathlineto{\pgfpoint{11.935629\du}{14.620717\du}}
\pgfpathlineto{\pgfpoint{11.969583\du}{14.613050\du}}
\pgfpathlineto{\pgfpoint{12.003172\du}{14.604652\du}}
\pgfpathlineto{\pgfpoint{12.036031\du}{14.596620\du}}
\pgfpathlineto{\pgfpoint{12.067794\du}{14.588588\du}}
\pgfpathlineto{\pgfpoint{12.099192\du}{14.579826\du}}
\pgfpathlineto{\pgfpoint{12.114526\du}{14.575080\du}}
\pgfpathlineto{\pgfpoint{12.129860\du}{14.571064\du}}
\pgfpathlineto{\pgfpoint{12.145925\du}{14.566317\du}}
\pgfpathlineto{\pgfpoint{12.160528\du}{14.561571\du}}
\pgfpathlineto{\pgfpoint{12.174767\du}{14.556825\du}}
\pgfpathlineto{\pgfpoint{12.189736\du}{14.551713\du}}
\pgfpathlineto{\pgfpoint{12.204705\du}{14.546967\du}}
\pgfpathlineto{\pgfpoint{12.218579\du}{14.542221\du}}
\pgfpathlineto{\pgfpoint{12.232818\du}{14.537110\du}}
\pgfpathlineto{\pgfpoint{12.246691\du}{14.532363\du}}
\pgfpathlineto{\pgfpoint{12.260930\du}{14.526887\du}}
\pgfpathlineto{\pgfpoint{12.274074\du}{14.522506\du}}
\pgfpathlineto{\pgfpoint{12.288312\du}{14.517029\du}}
\pgfpathlineto{\pgfpoint{12.301456\du}{14.511918\du}}
\pgfpathlineto{\pgfpoint{12.314599\du}{14.506441\du}}
\pgfpathlineto{\pgfpoint{12.328108\du}{14.500600\du}}
\pgfpathlineto{\pgfpoint{12.340886\du}{14.495489\du}}
\pgfpathlineto{\pgfpoint{12.352934\du}{14.490012\du}}
\pgfpathlineto{\pgfpoint{12.365713\du}{14.484171\du}}
\pgfpathlineto{\pgfpoint{12.378126\du}{14.479059\du}}
\pgfpathlineto{\pgfpoint{12.390539\du}{14.473218\du}}
\pgfpathlineto{\pgfpoint{12.401857\du}{14.467741\du}}
\pgfpathlineto{\pgfpoint{12.413906\du}{14.461900\du}}
\pgfpathlineto{\pgfpoint{12.424859\du}{14.456058\du}}
\pgfpathlineto{\pgfpoint{12.436542\du}{14.450217\du}}
\pgfpathlineto{\pgfpoint{12.447860\du}{14.444375\du}}
\pgfpathlineto{\pgfpoint{12.459178\du}{14.438533\du}}
\pgfpathlineto{\pgfpoint{12.469765\du}{14.432327\du}}
\pgfpathlineto{\pgfpoint{12.479623\du}{14.426485\du}}
\pgfpathlineto{\pgfpoint{12.490211\du}{14.420644\du}}
\pgfpathlineto{\pgfpoint{12.500434\du}{14.414072\du}}
\pgfpathlineto{\pgfpoint{12.510291\du}{14.408230\du}}
\pgfpathlineto{\pgfpoint{12.520149\du}{14.401659\du}}
\pgfpathlineto{\pgfpoint{12.529276\du}{14.395452\du}}
\pgfpathlineto{\pgfpoint{12.538404\du}{14.388880\du}}
\pgfpathlineto{\pgfpoint{12.547896\du}{14.383039\du}}
\pgfpathlineto{\pgfpoint{12.556658\du}{14.376467\du}}
\pgfpathlineto{\pgfpoint{12.565786\du}{14.370260\du}}
\pgfpathlineto{\pgfpoint{12.573818\du}{14.363689\du}}
\pgfpathlineto{\pgfpoint{12.582580\du}{14.356752\du}}
\pgfpathlineto{\pgfpoint{12.589882\du}{14.350180\du}}
\pgfpathlineto{\pgfpoint{12.597914\du}{14.343973\du}}
\pgfpathlineto{\pgfpoint{12.605947\du}{14.336671\du}}
\pgfpathlineto{\pgfpoint{12.612518\du}{14.330465\du}}
\pgfpathlineto{\pgfpoint{12.619820\du}{14.323528\du}}
\pgfpathlineto{\pgfpoint{12.626392\du}{14.316226\du}}
\pgfpathlineto{\pgfpoint{12.633329\du}{14.310019\du}}
\pgfpathlineto{\pgfpoint{12.639901\du}{14.303083\du}}
\pgfpathlineto{\pgfpoint{12.646107\du}{14.295781\du}}
\pgfpathlineto{\pgfpoint{12.651949\du}{14.288844\du}}
\pgfpathlineto{\pgfpoint{12.657425\du}{14.281907\du}}
\pgfpathlineto{\pgfpoint{12.663267\du}{14.274970\du}}
\pgfpathlineto{\pgfpoint{12.668013\du}{14.267668\du}}
\pgfpathlineto{\pgfpoint{12.673489\du}{14.260366\du}}
\pgfpathlineto{\pgfpoint{12.678236\du}{14.253064\du}}
\pgfpathlineto{\pgfpoint{12.682617\du}{14.246127\du}}
\pgfpathlineto{\pgfpoint{12.686633\du}{14.238460\du}}
\pgfpathlineto{\pgfpoint{12.690649\du}{14.231524\du}}
\pgfpathlineto{\pgfpoint{12.693935\du}{14.223857\du}}
\pgfpathlineto{\pgfpoint{12.697951\du}{14.216190\du}}
\pgfpathlineto{\pgfpoint{12.701237\du}{14.208523\du}}
\pgfpathlineto{\pgfpoint{12.703792\du}{14.201221\du}}
\pgfpathlineto{\pgfpoint{12.706713\du}{14.193919\du}}
\pgfpathlineto{\pgfpoint{12.708539\du}{14.186617\du}}
\pgfpathlineto{\pgfpoint{12.711459\du}{14.178950\du}}
\pgfpathlineto{\pgfpoint{12.712555\du}{14.170552\du}}
\pgfpathlineto{\pgfpoint{12.714745\du}{14.162885\du}}
\pgfpathlineto{\pgfpoint{12.715841\du}{14.155583\du}}
\pgfpathlineto{\pgfpoint{12.716571\du}{14.147916\du}}
\pgfpathlineto{\pgfpoint{12.717301\du}{14.139519\du}}
\pgfpathlineto{\pgfpoint{12.718031\du}{14.132217\du}}
\pgfpathlineto{\pgfpoint{12.718031\du}{14.124550\du}}
\pgfpathlineto{\pgfpoint{12.697951\du}{14.124550\du}}
\pgfpathlineto{\pgfpoint{12.697221\du}{14.131487\du}}
\pgfpathlineto{\pgfpoint{12.697221\du}{14.138424\du}}
\pgfpathlineto{\pgfpoint{12.696856\du}{14.145361\du}}
\pgfpathlineto{\pgfpoint{12.695030\du}{14.152663\du}}
\pgfpathlineto{\pgfpoint{12.693935\du}{14.159600\du}}
\pgfpathlineto{\pgfpoint{12.693205\du}{14.166536\du}}
\pgfpathlineto{\pgfpoint{12.691014\du}{14.173473\du}}
\pgfpathlineto{\pgfpoint{12.689554\du}{14.180775\du}}
\pgfpathlineto{\pgfpoint{12.687363\du}{14.186982\du}}
\pgfpathlineto{\pgfpoint{12.684807\du}{14.193919\du}}
\pgfpathlineto{\pgfpoint{12.681887\du}{14.201221\du}}
\pgfpathlineto{\pgfpoint{12.679331\du}{14.208157\du}}
\pgfpathlineto{\pgfpoint{12.675315\du}{14.215094\du}}
\pgfpathlineto{\pgfpoint{12.672394\du}{14.221666\du}}
\pgfpathlineto{\pgfpoint{12.668378\du}{14.228603\du}}
\pgfpathlineto{\pgfpoint{12.665457\du}{14.235175\du}}
\pgfpathlineto{\pgfpoint{12.661076\du}{14.242111\du}}
\pgfpathlineto{\pgfpoint{12.656695\du}{14.249048\du}}
\pgfpathlineto{\pgfpoint{12.651949\du}{14.255620\du}}
\pgfpathlineto{\pgfpoint{12.646837\du}{14.261827\du}}
\pgfpathlineto{\pgfpoint{12.642091\du}{14.269129\du}}
\pgfpathlineto{\pgfpoint{12.635884\du}{14.276065\du}}
\pgfpathlineto{\pgfpoint{12.630408\du}{14.282272\du}}
\pgfpathlineto{\pgfpoint{12.625297\du}{14.288844\du}}
\pgfpathlineto{\pgfpoint{12.618725\du}{14.295416\du}}
\pgfpathlineto{\pgfpoint{12.611788\du}{14.302352\du}}
\pgfpathlineto{\pgfpoint{12.605947\du}{14.308924\du}}
\pgfpathlineto{\pgfpoint{12.598645\du}{14.315131\du}}
\pgfpathlineto{\pgfpoint{12.592073\du}{14.321703\du}}
\pgfpathlineto{\pgfpoint{12.584406\du}{14.327909\du}}
\pgfpathlineto{\pgfpoint{12.577104\du}{14.334481\du}}
\pgfpathlineto{\pgfpoint{12.569072\du}{14.341053\du}}
\pgfpathlineto{\pgfpoint{12.561405\du}{14.347259\du}}
\pgfpathlineto{\pgfpoint{12.552642\du}{14.353831\du}}
\pgfpathlineto{\pgfpoint{12.544245\du}{14.359673\du}}
\pgfpathlineto{\pgfpoint{12.535483\du}{14.366244\du}}
\pgfpathlineto{\pgfpoint{12.527451\du}{14.372451\du}}
\pgfpathlineto{\pgfpoint{12.518688\du}{14.378293\du}}
\pgfpathlineto{\pgfpoint{12.509196\du}{14.384864\du}}
\pgfpathlineto{\pgfpoint{12.498973\du}{14.390706\du}}
\pgfpathlineto{\pgfpoint{12.489846\du}{14.397278\du}}
\pgfpathlineto{\pgfpoint{12.479623\du}{14.403119\du}}
\pgfpathlineto{\pgfpoint{12.469765\du}{14.408961\du}}
\pgfpathlineto{\pgfpoint{12.459908\du}{14.414802\du}}
\pgfpathlineto{\pgfpoint{12.448955\du}{14.420644\du}}
\pgfpathlineto{\pgfpoint{12.438002\du}{14.426485\du}}
\pgfpathlineto{\pgfpoint{12.427779\du}{14.432327\du}}
\pgfpathlineto{\pgfpoint{12.415731\du}{14.438168\du}}
\pgfpathlineto{\pgfpoint{12.405143\du}{14.443280\du}}
\pgfpathlineto{\pgfpoint{12.393095\du}{14.449121\du}}
\pgfpathlineto{\pgfpoint{12.381777\du}{14.454963\du}}
\pgfpathlineto{\pgfpoint{12.369364\du}{14.460439\du}}
\pgfpathlineto{\pgfpoint{12.357316\du}{14.465551\du}}
\pgfpathlineto{\pgfpoint{12.344902\du}{14.471392\du}}
\pgfpathlineto{\pgfpoint{12.332854\du}{14.476869\du}}
\pgfpathlineto{\pgfpoint{12.319711\du}{14.481980\du}}
\pgfpathlineto{\pgfpoint{12.306932\du}{14.487091\du}}
\pgfpathlineto{\pgfpoint{12.294154\du}{14.492568\du}}
\pgfpathlineto{\pgfpoint{12.281010\du}{14.497679\du}}
\pgfpathlineto{\pgfpoint{12.267502\du}{14.503156\du}}
\pgfpathlineto{\pgfpoint{12.254358\du}{14.507537\du}}
\pgfpathlineto{\pgfpoint{12.240120\du}{14.513013\du}}
\pgfpathlineto{\pgfpoint{12.226246\du}{14.517759\du}}
\pgfpathlineto{\pgfpoint{12.212007\du}{14.522871\du}}
\pgfpathlineto{\pgfpoint{12.197403\du}{14.527617\du}}
\pgfpathlineto{\pgfpoint{12.183530\du}{14.532363\du}}
\pgfpathlineto{\pgfpoint{12.168926\du}{14.537110\du}}
\pgfpathlineto{\pgfpoint{12.154687\du}{14.541491\du}}
\pgfpathlineto{\pgfpoint{12.138988\du}{14.546237\du}}
\pgfpathlineto{\pgfpoint{12.124749\du}{14.550983\du}}
\pgfpathlineto{\pgfpoint{12.109050\du}{14.555730\du}}
\pgfpathlineto{\pgfpoint{12.094081\du}{14.559746\du}}
\pgfpathlineto{\pgfpoint{12.062683\du}{14.568508\du}}
\pgfpathlineto{\pgfpoint{12.030554\du}{14.576905\du}}
\pgfpathlineto{\pgfpoint{11.998426\du}{14.584937\du}}
\pgfpathlineto{\pgfpoint{11.965202\du}{14.592969\du}}
\pgfpathlineto{\pgfpoint{11.930883\du}{14.600636\du}}
\pgfpathlineto{\pgfpoint{11.896929\du}{14.607938\du}}
\pgfpathlineto{\pgfpoint{11.862244\du}{14.614875\du}}
\pgfpathlineto{\pgfpoint{11.826465\du}{14.621812\du}}
\pgfpathlineto{\pgfpoint{11.791051\du}{14.628384\du}}
\pgfpathlineto{\pgfpoint{11.754176\du}{14.634590\du}}
\pgfpathlineto{\pgfpoint{11.717666\du}{14.640432\du}}
\pgfpathlineto{\pgfpoint{11.680426\du}{14.646274\du}}
\pgfpathlineto{\pgfpoint{11.642821\du}{14.651750\du}}
\pgfpathlineto{\pgfpoint{11.604121\du}{14.656496\du}}
\pgfpathlineto{\pgfpoint{11.565421\du}{14.660877\du}}
\pgfpathlineto{\pgfpoint{11.525990\du}{14.665624\du}}
\pgfpathlineto{\pgfpoint{11.486925\du}{14.669275\du}}
\pgfpathlineto{\pgfpoint{11.446764\du}{14.673291\du}}
\pgfpathlineto{\pgfpoint{11.406604\du}{14.676211\du}}
\pgfpathlineto{\pgfpoint{11.365348\du}{14.679862\du}}
\pgfpathlineto{\pgfpoint{11.324457\du}{14.682783\du}}
\pgfpathlineto{\pgfpoint{11.283201\du}{14.684974\du}}
\pgfpathlineto{\pgfpoint{11.241580\du}{14.686799\du}}
\pgfpathlineto{\pgfpoint{11.198864\du}{14.688625\du}}
\pgfpathlineto{\pgfpoint{11.156147\du}{14.689720\du}}
\pgfpathlineto{\pgfpoint{11.114161\du}{14.690815\du}}
\pgfpathlineto{\pgfpoint{11.070715\du}{14.690815\du}}
\pgfpathlineto{\pgfpoint{11.027633\du}{14.691546\du}}
\pgfpathlineto{\pgfpoint{11.027633\du}{14.691546\du}}
\pgfpathlineto{\pgfpoint{11.027633\du}{14.691546\du}}
\pgfpathlineto{\pgfpoint{11.026903\du}{14.691546\du}}
\pgfpathlineto{\pgfpoint{11.025078\du}{14.691546\du}}
\pgfpathlineto{\pgfpoint{11.023982\du}{14.691911\du}}
\pgfpathlineto{\pgfpoint{11.023252\du}{14.691911\du}}
\pgfpathlineto{\pgfpoint{11.022887\du}{14.692641\du}}
\pgfpathlineto{\pgfpoint{11.021427\du}{14.693006\du}}
\pgfpathlineto{\pgfpoint{11.020696\du}{14.693736\du}}
\pgfpathlineto{\pgfpoint{11.019966\du}{14.694466\du}}
\pgfpathlineto{\pgfpoint{11.018871\du}{14.696292\du}}
\pgfpathlineto{\pgfpoint{11.018141\du}{14.697752\du}}
\pgfpathlineto{\pgfpoint{11.018141\du}{14.699578\du}}
\pgfpathlineto{\pgfpoint{11.017411\du}{14.701403\du}}
\pgfpathlineto{\pgfpoint{11.018141\du}{14.703594\du}}
\pgfpathlineto{\pgfpoint{11.018141\du}{14.705419\du}}
\pgfpathlineto{\pgfpoint{11.018871\du}{14.707245\du}}
\pgfpathlineto{\pgfpoint{11.019966\du}{14.709070\du}}
\pgfpathlineto{\pgfpoint{11.020696\du}{14.709435\du}}
\pgfpathlineto{\pgfpoint{11.021427\du}{14.710165\du}}
\pgfpathlineto{\pgfpoint{11.022887\du}{14.710896\du}}
\pgfpathlineto{\pgfpoint{11.023252\du}{14.711261\du}}
\pgfpathlineto{\pgfpoint{11.023982\du}{14.711261\du}}
\pgfpathlineto{\pgfpoint{11.025078\du}{14.711991\du}}
\pgfpathlineto{\pgfpoint{11.026903\du}{14.711991\du}}
\pgfpathlineto{\pgfpoint{11.027633\du}{14.711991\du}}
\pgfusepath{fill}
\pgfsetbuttcap
\pgfsetmiterjoin
\pgfsetdash{}{0pt}
\definecolor{dialinecolor}{rgb}{0.678431, 0.839216, 0.905882}
\pgfsetfillcolor{dialinecolor}
\pgfpathmoveto{\pgfpoint{9.336870\du}{14.124550\du}}
\pgfpathlineto{\pgfpoint{9.336870\du}{14.124550\du}}
\pgfpathlineto{\pgfpoint{9.336870\du}{14.132217\du}}
\pgfpathlineto{\pgfpoint{9.337235\du}{14.139519\du}}
\pgfpathlineto{\pgfpoint{9.337966\du}{14.147916\du}}
\pgfpathlineto{\pgfpoint{9.339061\du}{14.155583\du}}
\pgfpathlineto{\pgfpoint{9.340156\du}{14.162885\du}}
\pgfpathlineto{\pgfpoint{9.341982\du}{14.170552\du}}
\pgfpathlineto{\pgfpoint{9.343807\du}{14.178950\du}}
\pgfpathlineto{\pgfpoint{9.345998\du}{14.186617\du}}
\pgfpathlineto{\pgfpoint{9.348188\du}{14.193919\du}}
\pgfpathlineto{\pgfpoint{9.350744\du}{14.201221\du}}
\pgfpathlineto{\pgfpoint{9.353665\du}{14.208523\du}}
\pgfpathlineto{\pgfpoint{9.357316\du}{14.216190\du}}
\pgfpathlineto{\pgfpoint{9.360602\du}{14.223857\du}}
\pgfpathlineto{\pgfpoint{9.364252\du}{14.231524\du}}
\pgfpathlineto{\pgfpoint{9.368634\du}{14.238460\du}}
\pgfpathlineto{\pgfpoint{9.372285\du}{14.246127\du}}
\pgfpathlineto{\pgfpoint{9.377396\du}{14.253064\du}}
\pgfpathlineto{\pgfpoint{9.381412\du}{14.260366\du}}
\pgfpathlineto{\pgfpoint{9.386888\du}{14.267668\du}}
\pgfpathlineto{\pgfpoint{9.391635\du}{14.274970\du}}
\pgfpathlineto{\pgfpoint{9.397111\du}{14.281907\du}}
\pgfpathlineto{\pgfpoint{9.402953\du}{14.288844\du}}
\pgfpathlineto{\pgfpoint{9.408794\du}{14.295781\du}}
\pgfpathlineto{\pgfpoint{9.414636\du}{14.303083\du}}
\pgfpathlineto{\pgfpoint{9.421573\du}{14.310019\du}}
\pgfpathlineto{\pgfpoint{9.428144\du}{14.316226\du}}
\pgfpathlineto{\pgfpoint{9.435081\du}{14.323528\du}}
\pgfpathlineto{\pgfpoint{9.442018\du}{14.330465\du}}
\pgfpathlineto{\pgfpoint{9.448955\du}{14.336671\du}}
\pgfpathlineto{\pgfpoint{9.457717\du}{14.343973\du}}
\pgfpathlineto{\pgfpoint{9.464654\du}{14.350180\du}}
\pgfpathlineto{\pgfpoint{9.472321\du}{14.356752\du}}
\pgfpathlineto{\pgfpoint{9.481083\du}{14.363689\du}}
\pgfpathlineto{\pgfpoint{9.489481\du}{14.370260\du}}
\pgfpathlineto{\pgfpoint{9.498243\du}{14.376467\du}}
\pgfpathlineto{\pgfpoint{9.506640\du}{14.383039\du}}
\pgfpathlineto{\pgfpoint{9.516498\du}{14.388880\du}}
\pgfpathlineto{\pgfpoint{9.525625\du}{14.395452\du}}
\pgfpathlineto{\pgfpoint{9.535118\du}{14.401659\du}}
\pgfpathlineto{\pgfpoint{9.544610\du}{14.408230\du}}
\pgfpathlineto{\pgfpoint{9.554468\du}{14.414072\du}}
\pgfpathlineto{\pgfpoint{9.564691\du}{14.420644\du}}
\pgfpathlineto{\pgfpoint{9.574913\du}{14.426485\du}}
\pgfpathlineto{\pgfpoint{9.585501\du}{14.432327\du}}
\pgfpathlineto{\pgfpoint{9.595724\du}{14.438533\du}}
\pgfpathlineto{\pgfpoint{9.606677\du}{14.444375\du}}
\pgfpathlineto{\pgfpoint{9.618360\du}{14.450217\du}}
\pgfpathlineto{\pgfpoint{9.629678\du}{14.456058\du}}
\pgfpathlineto{\pgfpoint{9.641361\du}{14.461900\du}}
\pgfpathlineto{\pgfpoint{9.652679\du}{14.467741\du}}
\pgfpathlineto{\pgfpoint{9.664362\du}{14.473218\du}}
\pgfpathlineto{\pgfpoint{9.676775\du}{14.479059\du}}
\pgfpathlineto{\pgfpoint{9.688823\du}{14.484171\du}}
\pgfpathlineto{\pgfpoint{9.701967\du}{14.490012\du}}
\pgfpathlineto{\pgfpoint{9.714015\du}{14.495489\du}}
\pgfpathlineto{\pgfpoint{9.726794\du}{14.500600\du}}
\pgfpathlineto{\pgfpoint{9.740667\du}{14.506441\du}}
\pgfpathlineto{\pgfpoint{9.753081\du}{14.511918\du}}
\pgfpathlineto{\pgfpoint{9.766224\du}{14.517029\du}}
\pgfpathlineto{\pgfpoint{9.780463\du}{14.522506\du}}
\pgfpathlineto{\pgfpoint{9.793606\du}{14.526887\du}}
\pgfpathlineto{\pgfpoint{9.807845\du}{14.532363\du}}
\pgfpathlineto{\pgfpoint{9.821719\du}{14.537110\du}}
\pgfpathlineto{\pgfpoint{9.836323\du}{14.542221\du}}
\pgfpathlineto{\pgfpoint{9.850196\du}{14.546967\du}}
\pgfpathlineto{\pgfpoint{9.865895\du}{14.551713\du}}
\pgfpathlineto{\pgfpoint{9.879769\du}{14.556825\du}}
\pgfpathlineto{\pgfpoint{9.894373\du}{14.561571\du}}
\pgfpathlineto{\pgfpoint{9.909707\du}{14.566317\du}}
\pgfpathlineto{\pgfpoint{9.925406\du}{14.571064\du}}
\pgfpathlineto{\pgfpoint{9.940375\du}{14.575080\du}}
\pgfpathlineto{\pgfpoint{9.956074\du}{14.579826\du}}
\pgfpathlineto{\pgfpoint{9.987838\du}{14.588588\du}}
\pgfpathlineto{\pgfpoint{10.019601\du}{14.596620\du}}
\pgfpathlineto{\pgfpoint{10.052825\du}{14.604652\du}}
\pgfpathlineto{\pgfpoint{10.085319\du}{14.613050\du}}
\pgfpathlineto{\pgfpoint{10.120003\du}{14.620717\du}}
\pgfpathlineto{\pgfpoint{10.154322\du}{14.628384\du}}
\pgfpathlineto{\pgfpoint{10.189006\du}{14.635321\du}}
\pgfpathlineto{\pgfpoint{10.224786\du}{14.642257\du}}
\pgfpathlineto{\pgfpoint{10.260930\du}{14.648829\du}}
\pgfpathlineto{\pgfpoint{10.297440\du}{14.654671\du}}
\pgfpathlineto{\pgfpoint{10.334680\du}{14.660877\du}}
\pgfpathlineto{\pgfpoint{10.372285\du}{14.666354\du}}
\pgfpathlineto{\pgfpoint{10.410255\du}{14.671465\du}}
\pgfpathlineto{\pgfpoint{10.448955\du}{14.676942\du}}
\pgfpathlineto{\pgfpoint{10.487655\du}{14.681323\du}}
\pgfpathlineto{\pgfpoint{10.526721\du}{14.685704\du}}
\pgfpathlineto{\pgfpoint{10.566881\du}{14.689720\du}}
\pgfpathlineto{\pgfpoint{10.606677\du}{14.693736\du}}
\pgfpathlineto{\pgfpoint{10.647568\du}{14.697387\du}}
\pgfpathlineto{\pgfpoint{10.688458\du}{14.700308\du}}
\pgfpathlineto{\pgfpoint{10.729714\du}{14.702498\du}}
\pgfpathlineto{\pgfpoint{10.771700\du}{14.705054\du}}
\pgfpathlineto{\pgfpoint{10.813321\du}{14.707245\du}}
\pgfpathlineto{\pgfpoint{10.855673\du}{14.709070\du}}
\pgfpathlineto{\pgfpoint{10.897659\du}{14.710165\du}}
\pgfpathlineto{\pgfpoint{10.941105\du}{14.711261\du}}
\pgfpathlineto{\pgfpoint{10.983822\du}{14.711991\du}}
\pgfpathlineto{\pgfpoint{11.027633\du}{14.711991\du}}
\pgfpathlineto{\pgfpoint{11.027633\du}{14.691546\du}}
\pgfpathlineto{\pgfpoint{10.984917\du}{14.690815\du}}
\pgfpathlineto{\pgfpoint{10.941470\du}{14.690815\du}}
\pgfpathlineto{\pgfpoint{10.899119\du}{14.689720\du}}
\pgfpathlineto{\pgfpoint{10.856403\du}{14.688625\du}}
\pgfpathlineto{\pgfpoint{10.814052\du}{14.686799\du}}
\pgfpathlineto{\pgfpoint{10.772066\du}{14.684974\du}}
\pgfpathlineto{\pgfpoint{10.731175\du}{14.682783\du}}
\pgfpathlineto{\pgfpoint{10.689919\du}{14.679862\du}}
\pgfpathlineto{\pgfpoint{10.649028\du}{14.676211\du}}
\pgfpathlineto{\pgfpoint{10.608867\du}{14.673291\du}}
\pgfpathlineto{\pgfpoint{10.569072\du}{14.669275\du}}
\pgfpathlineto{\pgfpoint{10.529641\du}{14.665624\du}}
\pgfpathlineto{\pgfpoint{10.490211\du}{14.660877\du}}
\pgfpathlineto{\pgfpoint{10.451146\du}{14.656496\du}}
\pgfpathlineto{\pgfpoint{10.412810\du}{14.651750\du}}
\pgfpathlineto{\pgfpoint{10.375205\du}{14.646274\du}}
\pgfpathlineto{\pgfpoint{10.337966\du}{14.640432\du}}
\pgfpathlineto{\pgfpoint{10.301456\du}{14.634590\du}}
\pgfpathlineto{\pgfpoint{10.264216\du}{14.628384\du}}
\pgfpathlineto{\pgfpoint{10.229167\du}{14.621812\du}}
\pgfpathlineto{\pgfpoint{10.193022\du}{14.614875\du}}
\pgfpathlineto{\pgfpoint{10.158338\du}{14.607938\du}}
\pgfpathlineto{\pgfpoint{10.124019\du}{14.600636\du}}
\pgfpathlineto{\pgfpoint{10.090065\du}{14.592969\du}}
\pgfpathlineto{\pgfpoint{10.056841\du}{14.584937\du}}
\pgfpathlineto{\pgfpoint{10.025443\du}{14.576905\du}}
\pgfpathlineto{\pgfpoint{9.992949\du}{14.568508\du}}
\pgfpathlineto{\pgfpoint{9.961916\du}{14.559746\du}}
\pgfpathlineto{\pgfpoint{9.946217\du}{14.555730\du}}
\pgfpathlineto{\pgfpoint{9.930518\du}{14.550983\du}}
\pgfpathlineto{\pgfpoint{9.915914\du}{14.546237\du}}
\pgfpathlineto{\pgfpoint{9.900945\du}{14.541491\du}}
\pgfpathlineto{\pgfpoint{9.885611\du}{14.537110\du}}
\pgfpathlineto{\pgfpoint{9.871372\du}{14.532363\du}}
\pgfpathlineto{\pgfpoint{9.857133\du}{14.527617\du}}
\pgfpathlineto{\pgfpoint{9.842894\du}{14.522871\du}}
\pgfpathlineto{\pgfpoint{9.828656\du}{14.517759\du}}
\pgfpathlineto{\pgfpoint{9.814782\du}{14.513013\du}}
\pgfpathlineto{\pgfpoint{9.800908\du}{14.507537\du}}
\pgfpathlineto{\pgfpoint{9.787400\du}{14.503156\du}}
\pgfpathlineto{\pgfpoint{9.773891\du}{14.497679\du}}
\pgfpathlineto{\pgfpoint{9.761113\du}{14.492568\du}}
\pgfpathlineto{\pgfpoint{9.747604\du}{14.487091\du}}
\pgfpathlineto{\pgfpoint{9.734826\du}{14.481980\du}}
\pgfpathlineto{\pgfpoint{9.722412\du}{14.476869\du}}
\pgfpathlineto{\pgfpoint{9.709269\du}{14.471392\du}}
\pgfpathlineto{\pgfpoint{9.697221\du}{14.465551\du}}
\pgfpathlineto{\pgfpoint{9.685903\du}{14.460439\du}}
\pgfpathlineto{\pgfpoint{9.673124\du}{14.454963\du}}
\pgfpathlineto{\pgfpoint{9.661441\du}{14.449121\du}}
\pgfpathlineto{\pgfpoint{9.649758\du}{14.443280\du}}
\pgfpathlineto{\pgfpoint{9.638805\du}{14.438168\du}}
\pgfpathlineto{\pgfpoint{9.627122\du}{14.432327\du}}
\pgfpathlineto{\pgfpoint{9.617265\du}{14.426485\du}}
\pgfpathlineto{\pgfpoint{9.605947\du}{14.420644\du}}
\pgfpathlineto{\pgfpoint{9.595359\du}{14.414802\du}}
\pgfpathlineto{\pgfpoint{9.585501\du}{14.408961\du}}
\pgfpathlineto{\pgfpoint{9.574913\du}{14.403119\du}}
\pgfpathlineto{\pgfpoint{9.565056\du}{14.397278\du}}
\pgfpathlineto{\pgfpoint{9.555563\du}{14.390706\du}}
\pgfpathlineto{\pgfpoint{9.545706\du}{14.384864\du}}
\pgfpathlineto{\pgfpoint{9.536578\du}{14.378293\du}}
\pgfpathlineto{\pgfpoint{9.528181\du}{14.372451\du}}
\pgfpathlineto{\pgfpoint{9.519419\du}{14.366244\du}}
\pgfpathlineto{\pgfpoint{9.510291\du}{14.359673\du}}
\pgfpathlineto{\pgfpoint{9.501529\du}{14.353831\du}}
\pgfpathlineto{\pgfpoint{9.493132\du}{14.347259\du}}
\pgfpathlineto{\pgfpoint{9.485830\du}{14.341053\du}}
\pgfpathlineto{\pgfpoint{9.477798\du}{14.334481\du}}
\pgfpathlineto{\pgfpoint{9.470131\du}{14.327909\du}}
\pgfpathlineto{\pgfpoint{9.463194\du}{14.321703\du}}
\pgfpathlineto{\pgfpoint{9.455892\du}{14.315131\du}}
\pgfpathlineto{\pgfpoint{9.448955\du}{14.308924\du}}
\pgfpathlineto{\pgfpoint{9.442748\du}{14.302352\du}}
\pgfpathlineto{\pgfpoint{9.436177\du}{14.295416\du}}
\pgfpathlineto{\pgfpoint{9.430335\du}{14.288844\du}}
\pgfpathlineto{\pgfpoint{9.424128\du}{14.282272\du}}
\pgfpathlineto{\pgfpoint{9.419017\du}{14.276065\du}}
\pgfpathlineto{\pgfpoint{9.412810\du}{14.269129\du}}
\pgfpathlineto{\pgfpoint{9.408429\du}{14.262557\du}}
\pgfpathlineto{\pgfpoint{9.402953\du}{14.255620\du}}
\pgfpathlineto{\pgfpoint{9.398572\du}{14.249048\du}}
\pgfpathlineto{\pgfpoint{9.394190\du}{14.242111\du}}
\pgfpathlineto{\pgfpoint{9.389809\du}{14.235175\du}}
\pgfpathlineto{\pgfpoint{9.386523\du}{14.228603\du}}
\pgfpathlineto{\pgfpoint{9.382507\du}{14.221666\du}}
\pgfpathlineto{\pgfpoint{9.378491\du}{14.215094\du}}
\pgfpathlineto{\pgfpoint{9.375570\du}{14.208157\du}}
\pgfpathlineto{\pgfpoint{9.373015\du}{14.201221\du}}
\pgfpathlineto{\pgfpoint{9.369729\du}{14.193919\du}}
\pgfpathlineto{\pgfpoint{9.367538\du}{14.186982\du}}
\pgfpathlineto{\pgfpoint{9.365348\du}{14.180775\du}}
\pgfpathlineto{\pgfpoint{9.363887\du}{14.173473\du}}
\pgfpathlineto{\pgfpoint{9.361697\du}{14.166536\du}}
\pgfpathlineto{\pgfpoint{9.360602\du}{14.159600\du}}
\pgfpathlineto{\pgfpoint{9.359506\du}{14.152663\du}}
\pgfpathlineto{\pgfpoint{9.358046\du}{14.145361\du}}
\pgfpathlineto{\pgfpoint{9.357681\du}{14.138424\du}}
\pgfpathlineto{\pgfpoint{9.357681\du}{14.131487\du}}
\pgfpathlineto{\pgfpoint{9.357316\du}{14.124550\du}}
\pgfpathlineto{\pgfpoint{9.357316\du}{14.124550\du}}
\pgfpathlineto{\pgfpoint{9.357316\du}{14.124550\du}}
\pgfpathlineto{\pgfpoint{9.357316\du}{14.122725\du}}
\pgfpathlineto{\pgfpoint{9.357316\du}{14.121995\du}}
\pgfpathlineto{\pgfpoint{9.356951\du}{14.120899\du}}
\pgfpathlineto{\pgfpoint{9.356951\du}{14.119804\du}}
\pgfpathlineto{\pgfpoint{9.355855\du}{14.119074\du}}
\pgfpathlineto{\pgfpoint{9.355490\du}{14.117979\du}}
\pgfpathlineto{\pgfpoint{9.355125\du}{14.117248\du}}
\pgfpathlineto{\pgfpoint{9.354030\du}{14.116883\du}}
\pgfpathlineto{\pgfpoint{9.352569\du}{14.115788\du}}
\pgfpathlineto{\pgfpoint{9.350744\du}{14.114328\du}}
\pgfpathlineto{\pgfpoint{9.348918\du}{14.113962\du}}
\pgfpathlineto{\pgfpoint{9.347093\du}{14.113962\du}}
\pgfpathlineto{\pgfpoint{9.344902\du}{14.113962\du}}
\pgfpathlineto{\pgfpoint{9.343442\du}{14.114328\du}}
\pgfpathlineto{\pgfpoint{9.341251\du}{14.115788\du}}
\pgfpathlineto{\pgfpoint{9.339426\du}{14.116883\du}}
\pgfpathlineto{\pgfpoint{9.339061\du}{14.117248\du}}
\pgfpathlineto{\pgfpoint{9.338696\du}{14.117979\du}}
\pgfpathlineto{\pgfpoint{9.337966\du}{14.119074\du}}
\pgfpathlineto{\pgfpoint{9.337235\du}{14.119804\du}}
\pgfpathlineto{\pgfpoint{9.337235\du}{14.120899\du}}
\pgfpathlineto{\pgfpoint{9.336870\du}{14.121995\du}}
\pgfpathlineto{\pgfpoint{9.336870\du}{14.122725\du}}
\pgfpathlineto{\pgfpoint{9.336870\du}{14.124550\du}}
\pgfusepath{fill}
\pgfsetbuttcap
\pgfsetmiterjoin
\pgfsetdash{}{0pt}
\definecolor{dialinecolor}{rgb}{0.678431, 0.839216, 0.905882}
\pgfsetfillcolor{dialinecolor}
\pgfpathmoveto{\pgfpoint{11.027633\du}{13.537110\du}}
\pgfpathlineto{\pgfpoint{11.027633\du}{13.537110\du}}
\pgfpathlineto{\pgfpoint{10.983822\du}{13.537110\du}}
\pgfpathlineto{\pgfpoint{10.941105\du}{13.537475\du}}
\pgfpathlineto{\pgfpoint{10.897659\du}{13.538570\du}}
\pgfpathlineto{\pgfpoint{10.855673\du}{13.540030\du}}
\pgfpathlineto{\pgfpoint{10.813321\du}{13.541491\du}}
\pgfpathlineto{\pgfpoint{10.771700\du}{13.543316\du}}
\pgfpathlineto{\pgfpoint{10.729714\du}{13.545872\du}}
\pgfpathlineto{\pgfpoint{10.688458\du}{13.548793\du}}
\pgfpathlineto{\pgfpoint{10.647568\du}{13.551713\du}}
\pgfpathlineto{\pgfpoint{10.606677\du}{13.554634\du}}
\pgfpathlineto{\pgfpoint{10.566881\du}{13.558650\du}}
\pgfpathlineto{\pgfpoint{10.526721\du}{13.562666\du}}
\pgfpathlineto{\pgfpoint{10.487655\du}{13.567413\du}}
\pgfpathlineto{\pgfpoint{10.448955\du}{13.572159\du}}
\pgfpathlineto{\pgfpoint{10.410255\du}{13.576905\du}}
\pgfpathlineto{\pgfpoint{10.372285\du}{13.582016\du}}
\pgfpathlineto{\pgfpoint{10.334680\du}{13.587858\du}}
\pgfpathlineto{\pgfpoint{10.297440\du}{13.593700\du}}
\pgfpathlineto{\pgfpoint{10.260930\du}{13.600271\du}}
\pgfpathlineto{\pgfpoint{10.224786\du}{13.606478\du}}
\pgfpathlineto{\pgfpoint{10.189006\du}{13.613780\du}}
\pgfpathlineto{\pgfpoint{10.154322\du}{13.620717\du}}
\pgfpathlineto{\pgfpoint{10.120003\du}{13.627654\du}}
\pgfpathlineto{\pgfpoint{10.085319\du}{13.635686\du}}
\pgfpathlineto{\pgfpoint{10.052825\du}{13.643353\du}}
\pgfpathlineto{\pgfpoint{10.019601\du}{13.651750\du}}
\pgfpathlineto{\pgfpoint{9.987838\du}{13.660512\du}}
\pgfpathlineto{\pgfpoint{9.956074\du}{13.669275\du}}
\pgfpathlineto{\pgfpoint{9.925406\du}{13.678037\du}}
\pgfpathlineto{\pgfpoint{9.894373\du}{13.687529\du}}
\pgfpathlineto{\pgfpoint{9.879769\du}{13.691911\du}}
\pgfpathlineto{\pgfpoint{9.865895\du}{13.696657\du}}
\pgfpathlineto{\pgfpoint{9.850196\du}{13.701403\du}}
\pgfpathlineto{\pgfpoint{9.836323\du}{13.706514\du}}
\pgfpathlineto{\pgfpoint{9.821719\du}{13.711261\du}}
\pgfpathlineto{\pgfpoint{9.807845\du}{13.716737\du}}
\pgfpathlineto{\pgfpoint{9.793606\du}{13.721118\du}}
\pgfpathlineto{\pgfpoint{9.780463\du}{13.726595\du}}
\pgfpathlineto{\pgfpoint{9.766224\du}{13.731706\du}}
\pgfpathlineto{\pgfpoint{9.753081\du}{13.737183\du}}
\pgfpathlineto{\pgfpoint{9.740667\du}{13.742294\du}}
\pgfpathlineto{\pgfpoint{9.726794\du}{13.747770\du}}
\pgfpathlineto{\pgfpoint{9.714015\du}{13.752882\du}}
\pgfpathlineto{\pgfpoint{9.701967\du}{13.757993\du}}
\pgfpathlineto{\pgfpoint{9.688823\du}{13.763835\du}}
\pgfpathlineto{\pgfpoint{9.676775\du}{13.770041\du}}
\pgfpathlineto{\pgfpoint{9.664362\du}{13.775153\du}}
\pgfpathlineto{\pgfpoint{9.652679\du}{13.780994\du}}
\pgfpathlineto{\pgfpoint{9.641361\du}{13.786836\du}}
\pgfpathlineto{\pgfpoint{9.629678\du}{13.792677\du}}
\pgfpathlineto{\pgfpoint{9.618360\du}{13.798519\du}}
\pgfpathlineto{\pgfpoint{9.606677\du}{13.803630\du}}
\pgfpathlineto{\pgfpoint{9.595724\du}{13.810202\du}}
\pgfpathlineto{\pgfpoint{9.585501\du}{13.816044\du}}
\pgfpathlineto{\pgfpoint{9.574913\du}{13.821885\du}}
\pgfpathlineto{\pgfpoint{9.564691\du}{13.827727\du}}
\pgfpathlineto{\pgfpoint{9.554468\du}{13.834298\du}}
\pgfpathlineto{\pgfpoint{9.544610\du}{13.840505\du}}
\pgfpathlineto{\pgfpoint{9.535118\du}{13.846347\du}}
\pgfpathlineto{\pgfpoint{9.525625\du}{13.852918\du}}
\pgfpathlineto{\pgfpoint{9.516498\du}{13.859490\du}}
\pgfpathlineto{\pgfpoint{9.506640\du}{13.865697\du}}
\pgfpathlineto{\pgfpoint{9.498243\du}{13.872268\du}}
\pgfpathlineto{\pgfpoint{9.489481\du}{13.878840\du}}
\pgfpathlineto{\pgfpoint{9.481083\du}{13.885047\du}}
\pgfpathlineto{\pgfpoint{9.472321\du}{13.891619\du}}
\pgfpathlineto{\pgfpoint{9.464654\du}{13.898555\du}}
\pgfpathlineto{\pgfpoint{9.457717\du}{13.905127\du}}
\pgfpathlineto{\pgfpoint{9.448955\du}{13.911334\du}}
\pgfpathlineto{\pgfpoint{9.442018\du}{13.918636\du}}
\pgfpathlineto{\pgfpoint{9.435081\du}{13.924842\du}}
\pgfpathlineto{\pgfpoint{9.428144\du}{13.931779\du}}
\pgfpathlineto{\pgfpoint{9.421573\du}{13.939081\du}}
\pgfpathlineto{\pgfpoint{9.414636\du}{13.945288\du}}
\pgfpathlineto{\pgfpoint{9.408794\du}{13.952590\du}}
\pgfpathlineto{\pgfpoint{9.402953\du}{13.959527\du}}
\pgfpathlineto{\pgfpoint{9.397111\du}{13.967194\du}}
\pgfpathlineto{\pgfpoint{9.391635\du}{13.974130\du}}
\pgfpathlineto{\pgfpoint{9.386888\du}{13.981067\du}}
\pgfpathlineto{\pgfpoint{9.381412\du}{13.988004\du}}
\pgfpathlineto{\pgfpoint{9.377396\du}{13.995306\du}}
\pgfpathlineto{\pgfpoint{9.372285\du}{14.002608\du}}
\pgfpathlineto{\pgfpoint{9.368634\du}{14.009910\du}}
\pgfpathlineto{\pgfpoint{9.364252\du}{14.017212\du}}
\pgfpathlineto{\pgfpoint{9.360602\du}{14.024879\du}}
\pgfpathlineto{\pgfpoint{9.357316\du}{14.032546\du}}
\pgfpathlineto{\pgfpoint{9.353665\du}{14.039483\du}}
\pgfpathlineto{\pgfpoint{9.350744\du}{14.047150\du}}
\pgfpathlineto{\pgfpoint{9.348188\du}{14.054817\du}}
\pgfpathlineto{\pgfpoint{9.345998\du}{14.062484\du}}
\pgfpathlineto{\pgfpoint{9.343807\du}{14.070151\du}}
\pgfpathlineto{\pgfpoint{9.341982\du}{14.077453\du}}
\pgfpathlineto{\pgfpoint{9.340156\du}{14.085120\du}}
\pgfpathlineto{\pgfpoint{9.339061\du}{14.092787\du}}
\pgfpathlineto{\pgfpoint{9.337966\du}{14.101184\du}}
\pgfpathlineto{\pgfpoint{9.337235\du}{14.108486\du}}
\pgfpathlineto{\pgfpoint{9.336870\du}{14.116153\du}}
\pgfpathlineto{\pgfpoint{9.336870\du}{14.124550\du}}
\pgfpathlineto{\pgfpoint{9.357316\du}{14.124550\du}}
\pgfpathlineto{\pgfpoint{9.357681\du}{14.117248\du}}
\pgfpathlineto{\pgfpoint{9.357681\du}{14.110311\du}}
\pgfpathlineto{\pgfpoint{9.358046\du}{14.103375\du}}
\pgfpathlineto{\pgfpoint{9.359506\du}{14.095708\du}}
\pgfpathlineto{\pgfpoint{9.360602\du}{14.089501\du}}
\pgfpathlineto{\pgfpoint{9.361697\du}{14.082199\du}}
\pgfpathlineto{\pgfpoint{9.363887\du}{14.075262\du}}
\pgfpathlineto{\pgfpoint{9.365348\du}{14.068325\du}}
\pgfpathlineto{\pgfpoint{9.367538\du}{14.061389\du}}
\pgfpathlineto{\pgfpoint{9.369729\du}{14.054087\du}}
\pgfpathlineto{\pgfpoint{9.373015\du}{14.047880\du}}
\pgfpathlineto{\pgfpoint{9.375570\du}{14.040943\du}}
\pgfpathlineto{\pgfpoint{9.378491\du}{14.033641\du}}
\pgfpathlineto{\pgfpoint{9.382507\du}{14.026704\du}}
\pgfpathlineto{\pgfpoint{9.386523\du}{14.019767\du}}
\pgfpathlineto{\pgfpoint{9.389809\du}{14.013196\du}}
\pgfpathlineto{\pgfpoint{9.393825\du}{14.006259\du}}
\pgfpathlineto{\pgfpoint{9.398572\du}{13.999687\du}}
\pgfpathlineto{\pgfpoint{9.402953\du}{13.992750\du}}
\pgfpathlineto{\pgfpoint{9.408429\du}{13.986179\du}}
\pgfpathlineto{\pgfpoint{9.412810\du}{13.979972\du}}
\pgfpathlineto{\pgfpoint{9.419017\du}{13.973035\du}}
\pgfpathlineto{\pgfpoint{9.424128\du}{13.966463\du}}
\pgfpathlineto{\pgfpoint{9.430335\du}{13.959527\du}}
\pgfpathlineto{\pgfpoint{9.436177\du}{13.952955\du}}
\pgfpathlineto{\pgfpoint{9.442748\du}{13.946748\du}}
\pgfpathlineto{\pgfpoint{9.448955\du}{13.940176\du}}
\pgfpathlineto{\pgfpoint{9.455892\du}{13.933240\du}}
\pgfpathlineto{\pgfpoint{9.463194\du}{13.927398\du}}
\pgfpathlineto{\pgfpoint{9.470131\du}{13.920096\du}}
\pgfpathlineto{\pgfpoint{9.477798\du}{13.913889\du}}
\pgfpathlineto{\pgfpoint{9.485830\du}{13.908048\du}}
\pgfpathlineto{\pgfpoint{9.493132\du}{13.901476\du}}
\pgfpathlineto{\pgfpoint{9.501529\du}{13.894904\du}}
\pgfpathlineto{\pgfpoint{9.510291\du}{13.888698\du}}
\pgfpathlineto{\pgfpoint{9.519419\du}{13.882126\du}}
\pgfpathlineto{\pgfpoint{9.528181\du}{13.876284\du}}
\pgfpathlineto{\pgfpoint{9.536578\du}{13.870078\du}}
\pgfpathlineto{\pgfpoint{9.545706\du}{13.864236\du}}
\pgfpathlineto{\pgfpoint{9.555563\du}{13.857665\du}}
\pgfpathlineto{\pgfpoint{9.565056\du}{13.851823\du}}
\pgfpathlineto{\pgfpoint{9.574913\du}{13.845981\du}}
\pgfpathlineto{\pgfpoint{9.585501\du}{13.840140\du}}
\pgfpathlineto{\pgfpoint{9.595359\du}{13.834298\du}}
\pgfpathlineto{\pgfpoint{9.605947\du}{13.827727\du}}
\pgfpathlineto{\pgfpoint{9.617265\du}{13.822615\du}}
\pgfpathlineto{\pgfpoint{9.627122\du}{13.816774\du}}
\pgfpathlineto{\pgfpoint{9.638805\du}{13.810932\du}}
\pgfpathlineto{\pgfpoint{9.649758\du}{13.805091\du}}
\pgfpathlineto{\pgfpoint{9.661441\du}{13.799249\du}}
\pgfpathlineto{\pgfpoint{9.673124\du}{13.793773\du}}
\pgfpathlineto{\pgfpoint{9.685903\du}{13.788661\du}}
\pgfpathlineto{\pgfpoint{9.697221\du}{13.782820\du}}
\pgfpathlineto{\pgfpoint{9.709269\du}{13.777343\du}}
\pgfpathlineto{\pgfpoint{9.722412\du}{13.772232\du}}
\pgfpathlineto{\pgfpoint{9.734826\du}{13.766755\du}}
\pgfpathlineto{\pgfpoint{9.747604\du}{13.761644\du}}
\pgfpathlineto{\pgfpoint{9.761113\du}{13.755803\du}}
\pgfpathlineto{\pgfpoint{9.773891\du}{13.751056\du}}
\pgfpathlineto{\pgfpoint{9.787400\du}{13.745945\du}}
\pgfpathlineto{\pgfpoint{9.800908\du}{13.740468\du}}
\pgfpathlineto{\pgfpoint{9.814782\du}{13.736087\du}}
\pgfpathlineto{\pgfpoint{9.828656\du}{13.730611\du}}
\pgfpathlineto{\pgfpoint{9.842894\du}{13.725865\du}}
\pgfpathlineto{\pgfpoint{9.857133\du}{13.721118\du}}
\pgfpathlineto{\pgfpoint{9.871372\du}{13.716007\du}}
\pgfpathlineto{\pgfpoint{9.885611\du}{13.711261\du}}
\pgfpathlineto{\pgfpoint{9.900945\du}{13.706514\du}}
\pgfpathlineto{\pgfpoint{9.930518\du}{13.697752\du}}
\pgfpathlineto{\pgfpoint{9.961916\du}{13.688625\du}}
\pgfpathlineto{\pgfpoint{9.992949\du}{13.680228\du}}
\pgfpathlineto{\pgfpoint{10.025443\du}{13.671465\du}}
\pgfpathlineto{\pgfpoint{10.056841\du}{13.663433\du}}
\pgfpathlineto{\pgfpoint{10.090065\du}{13.655766\du}}
\pgfpathlineto{\pgfpoint{10.124019\du}{13.648099\du}}
\pgfpathlineto{\pgfpoint{10.158338\du}{13.640432\du}}
\pgfpathlineto{\pgfpoint{10.193022\du}{13.633495\du}}
\pgfpathlineto{\pgfpoint{10.229167\du}{13.626558\du}}
\pgfpathlineto{\pgfpoint{10.264216\du}{13.620717\du}}
\pgfpathlineto{\pgfpoint{10.301456\du}{13.614145\du}}
\pgfpathlineto{\pgfpoint{10.337966\du}{13.608303\du}}
\pgfpathlineto{\pgfpoint{10.375205\du}{13.602462\du}}
\pgfpathlineto{\pgfpoint{10.412810\du}{13.597351\du}}
\pgfpathlineto{\pgfpoint{10.451146\du}{13.591874\du}}
\pgfpathlineto{\pgfpoint{10.490211\du}{13.587128\du}}
\pgfpathlineto{\pgfpoint{10.529641\du}{13.583112\du}}
\pgfpathlineto{\pgfpoint{10.569072\du}{13.579096\du}}
\pgfpathlineto{\pgfpoint{10.608867\du}{13.575445\du}}
\pgfpathlineto{\pgfpoint{10.649028\du}{13.572159\du}}
\pgfpathlineto{\pgfpoint{10.689919\du}{13.569238\du}}
\pgfpathlineto{\pgfpoint{10.731175\du}{13.566317\du}}
\pgfpathlineto{\pgfpoint{10.772066\du}{13.563762\du}}
\pgfpathlineto{\pgfpoint{10.814052\du}{13.561571\du}}
\pgfpathlineto{\pgfpoint{10.856403\du}{13.560476\du}}
\pgfpathlineto{\pgfpoint{10.899119\du}{13.558650\du}}
\pgfpathlineto{\pgfpoint{10.941470\du}{13.557920\du}}
\pgfpathlineto{\pgfpoint{10.984917\du}{13.557555\du}}
\pgfpathlineto{\pgfpoint{11.027633\du}{13.557555\du}}
\pgfpathlineto{\pgfpoint{11.027633\du}{13.557555\du}}
\pgfpathlineto{\pgfpoint{11.027633\du}{13.557555\du}}
\pgfpathlineto{\pgfpoint{11.028729\du}{13.556825\du}}
\pgfpathlineto{\pgfpoint{11.029824\du}{13.556825\du}}
\pgfpathlineto{\pgfpoint{11.031284\du}{13.556825\du}}
\pgfpathlineto{\pgfpoint{11.032380\du}{13.556460\du}}
\pgfpathlineto{\pgfpoint{11.032745\du}{13.555730\du}}
\pgfpathlineto{\pgfpoint{11.033840\du}{13.555730\du}}
\pgfpathlineto{\pgfpoint{11.034570\du}{13.554634\du}}
\pgfpathlineto{\pgfpoint{11.035665\du}{13.553904\du}}
\pgfpathlineto{\pgfpoint{11.036761\du}{13.552809\du}}
\pgfpathlineto{\pgfpoint{11.037491\du}{13.550983\du}}
\pgfpathlineto{\pgfpoint{11.037491\du}{13.548793\du}}
\pgfpathlineto{\pgfpoint{11.038221\du}{13.546967\du}}
\pgfpathlineto{\pgfpoint{11.037491\du}{13.545142\du}}
\pgfpathlineto{\pgfpoint{11.037491\du}{13.543316\du}}
\pgfpathlineto{\pgfpoint{11.036761\du}{13.541491\du}}
\pgfpathlineto{\pgfpoint{11.035665\du}{13.540030\du}}
\pgfpathlineto{\pgfpoint{11.034570\du}{13.539300\du}}
\pgfpathlineto{\pgfpoint{11.033840\du}{13.538570\du}}
\pgfpathlineto{\pgfpoint{11.032745\du}{13.538205\du}}
\pgfpathlineto{\pgfpoint{11.032380\du}{13.537475\du}}
\pgfpathlineto{\pgfpoint{11.031284\du}{13.537110\du}}
\pgfpathlineto{\pgfpoint{11.029824\du}{13.537110\du}}
\pgfpathlineto{\pgfpoint{11.028729\du}{13.537110\du}}
\pgfpathlineto{\pgfpoint{11.027633\du}{13.537110\du}}
\pgfusepath{fill}
\pgfsetbuttcap
\pgfsetmiterjoin
\pgfsetdash{}{0pt}
\definecolor{dialinecolor}{rgb}{0.678431, 0.839216, 0.905882}
\pgfsetfillcolor{dialinecolor}
\pgfpathmoveto{\pgfpoint{12.718031\du}{14.124550\du}}
\pgfpathlineto{\pgfpoint{12.718031\du}{14.116153\du}}
\pgfpathlineto{\pgfpoint{12.717301\du}{14.108486\du}}
\pgfpathlineto{\pgfpoint{12.716571\du}{14.101184\du}}
\pgfpathlineto{\pgfpoint{12.715841\du}{14.092787\du}}
\pgfpathlineto{\pgfpoint{12.714745\du}{14.085120\du}}
\pgfpathlineto{\pgfpoint{12.712555\du}{14.077453\du}}
\pgfpathlineto{\pgfpoint{12.711459\du}{14.070151\du}}
\pgfpathlineto{\pgfpoint{12.708539\du}{14.062484\du}}
\pgfpathlineto{\pgfpoint{12.706713\du}{14.054817\du}}
\pgfpathlineto{\pgfpoint{12.703792\du}{14.047150\du}}
\pgfpathlineto{\pgfpoint{12.701237\du}{14.039483\du}}
\pgfpathlineto{\pgfpoint{12.697951\du}{14.032546\du}}
\pgfpathlineto{\pgfpoint{12.693935\du}{14.024879\du}}
\pgfpathlineto{\pgfpoint{12.690649\du}{14.017212\du}}
\pgfpathlineto{\pgfpoint{12.686633\du}{14.009910\du}}
\pgfpathlineto{\pgfpoint{12.682617\du}{14.002608\du}}
\pgfpathlineto{\pgfpoint{12.678236\du}{13.995306\du}}
\pgfpathlineto{\pgfpoint{12.673489\du}{13.988004\du}}
\pgfpathlineto{\pgfpoint{12.668013\du}{13.981067\du}}
\pgfpathlineto{\pgfpoint{12.663267\du}{13.974130\du}}
\pgfpathlineto{\pgfpoint{12.657425\du}{13.966463\du}}
\pgfpathlineto{\pgfpoint{12.651949\du}{13.959527\du}}
\pgfpathlineto{\pgfpoint{12.646107\du}{13.952590\du}}
\pgfpathlineto{\pgfpoint{12.639901\du}{13.945288\du}}
\pgfpathlineto{\pgfpoint{12.633329\du}{13.939081\du}}
\pgfpathlineto{\pgfpoint{12.626392\du}{13.931779\du}}
\pgfpathlineto{\pgfpoint{12.619820\du}{13.924842\du}}
\pgfpathlineto{\pgfpoint{12.612518\du}{13.918636\du}}
\pgfpathlineto{\pgfpoint{12.605947\du}{13.911334\du}}
\pgfpathlineto{\pgfpoint{12.597914\du}{13.905127\du}}
\pgfpathlineto{\pgfpoint{12.589882\du}{13.898555\du}}
\pgfpathlineto{\pgfpoint{12.582580\du}{13.891619\du}}
\pgfpathlineto{\pgfpoint{12.573818\du}{13.885047\du}}
\pgfpathlineto{\pgfpoint{12.565786\du}{13.878840\du}}
\pgfpathlineto{\pgfpoint{12.556658\du}{13.872268\du}}
\pgfpathlineto{\pgfpoint{12.547896\du}{13.865697\du}}
\pgfpathlineto{\pgfpoint{12.538404\du}{13.859490\du}}
\pgfpathlineto{\pgfpoint{12.529276\du}{13.852918\du}}
\pgfpathlineto{\pgfpoint{12.520149\du}{13.846347\du}}
\pgfpathlineto{\pgfpoint{12.510291\du}{13.840505\du}}
\pgfpathlineto{\pgfpoint{12.500434\du}{13.834298\du}}
\pgfpathlineto{\pgfpoint{12.490211\du}{13.827727\du}}
\pgfpathlineto{\pgfpoint{12.479623\du}{13.821885\du}}
\pgfpathlineto{\pgfpoint{12.469765\du}{13.816044\du}}
\pgfpathlineto{\pgfpoint{12.459178\du}{13.810202\du}}
\pgfpathlineto{\pgfpoint{12.447860\du}{13.803630\du}}
\pgfpathlineto{\pgfpoint{12.436542\du}{13.798519\du}}
\pgfpathlineto{\pgfpoint{12.424859\du}{13.792677\du}}
\pgfpathlineto{\pgfpoint{12.413906\du}{13.786836\du}}
\pgfpathlineto{\pgfpoint{12.401857\du}{13.780994\du}}
\pgfpathlineto{\pgfpoint{12.390539\du}{13.775153\du}}
\pgfpathlineto{\pgfpoint{12.378126\du}{13.770041\du}}
\pgfpathlineto{\pgfpoint{12.365713\du}{13.763835\du}}
\pgfpathlineto{\pgfpoint{12.352934\du}{13.757993\du}}
\pgfpathlineto{\pgfpoint{12.340886\du}{13.752882\du}}
\pgfpathlineto{\pgfpoint{12.328108\du}{13.747770\du}}
\pgfpathlineto{\pgfpoint{12.314599\du}{13.742294\du}}
\pgfpathlineto{\pgfpoint{12.301456\du}{13.737183\du}}
\pgfpathlineto{\pgfpoint{12.288312\du}{13.731706\du}}
\pgfpathlineto{\pgfpoint{12.274074\du}{13.726595\du}}
\pgfpathlineto{\pgfpoint{12.260930\du}{13.721118\du}}
\pgfpathlineto{\pgfpoint{12.246691\du}{13.716737\du}}
\pgfpathlineto{\pgfpoint{12.232818\du}{13.711261\du}}
\pgfpathlineto{\pgfpoint{12.218579\du}{13.706514\du}}
\pgfpathlineto{\pgfpoint{12.204705\du}{13.701403\du}}
\pgfpathlineto{\pgfpoint{12.189736\du}{13.696657\du}}
\pgfpathlineto{\pgfpoint{12.174767\du}{13.691911\du}}
\pgfpathlineto{\pgfpoint{12.160528\du}{13.687529\du}}
\pgfpathlineto{\pgfpoint{12.129860\du}{13.678037\du}}
\pgfpathlineto{\pgfpoint{12.099192\du}{13.669275\du}}
\pgfpathlineto{\pgfpoint{12.067794\du}{13.660512\du}}
\pgfpathlineto{\pgfpoint{12.036031\du}{13.651750\du}}
\pgfpathlineto{\pgfpoint{12.003172\du}{13.643353\du}}
\pgfpathlineto{\pgfpoint{11.969583\du}{13.635686\du}}
\pgfpathlineto{\pgfpoint{11.935629\du}{13.627654\du}}
\pgfpathlineto{\pgfpoint{11.900945\du}{13.620717\du}}
\pgfpathlineto{\pgfpoint{11.866261\du}{13.613780\du}}
\pgfpathlineto{\pgfpoint{11.830846\du}{13.606478\du}}
\pgfpathlineto{\pgfpoint{11.794336\du}{13.600271\du}}
\pgfpathlineto{\pgfpoint{11.757827\du}{13.593700\du}}
\pgfpathlineto{\pgfpoint{11.720952\du}{13.587858\du}}
\pgfpathlineto{\pgfpoint{11.683712\du}{13.582016\du}}
\pgfpathlineto{\pgfpoint{11.644647\du}{13.576905\du}}
\pgfpathlineto{\pgfpoint{11.606677\du}{13.572159\du}}
\pgfpathlineto{\pgfpoint{11.567246\du}{13.567413\du}}
\pgfpathlineto{\pgfpoint{11.528546\du}{13.562666\du}}
\pgfpathlineto{\pgfpoint{11.488385\du}{13.558650\du}}
\pgfpathlineto{\pgfpoint{11.448590\du}{13.554634\du}}
\pgfpathlineto{\pgfpoint{11.407699\du}{13.551713\du}}
\pgfpathlineto{\pgfpoint{11.367173\du}{13.548793\du}}
\pgfpathlineto{\pgfpoint{11.325552\du}{13.545872\du}}
\pgfpathlineto{\pgfpoint{11.283931\du}{13.543316\du}}
\pgfpathlineto{\pgfpoint{11.242310\du}{13.541491\du}}
\pgfpathlineto{\pgfpoint{11.199594\du}{13.540030\du}}
\pgfpathlineto{\pgfpoint{11.157608\du}{13.538570\du}}
\pgfpathlineto{\pgfpoint{11.114526\du}{13.537475\du}}
\pgfpathlineto{\pgfpoint{11.071080\du}{13.537110\du}}
\pgfpathlineto{\pgfpoint{11.027633\du}{13.537110\du}}
\pgfpathlineto{\pgfpoint{11.027633\du}{13.557555\du}}
\pgfpathlineto{\pgfpoint{11.070715\du}{13.557555\du}}
\pgfpathlineto{\pgfpoint{11.114161\du}{13.557920\du}}
\pgfpathlineto{\pgfpoint{11.156147\du}{13.558650\du}}
\pgfpathlineto{\pgfpoint{11.198864\du}{13.560476\du}}
\pgfpathlineto{\pgfpoint{11.241580\du}{13.561571\du}}
\pgfpathlineto{\pgfpoint{11.283201\du}{13.563762\du}}
\pgfpathlineto{\pgfpoint{11.324457\du}{13.566317\du}}
\pgfpathlineto{\pgfpoint{11.365348\du}{13.569238\du}}
\pgfpathlineto{\pgfpoint{11.406604\du}{13.572159\du}}
\pgfpathlineto{\pgfpoint{11.446764\du}{13.575445\du}}
\pgfpathlineto{\pgfpoint{11.486925\du}{13.579096\du}}
\pgfpathlineto{\pgfpoint{11.525990\du}{13.583112\du}}
\pgfpathlineto{\pgfpoint{11.565421\du}{13.587128\du}}
\pgfpathlineto{\pgfpoint{11.604121\du}{13.591874\du}}
\pgfpathlineto{\pgfpoint{11.642821\du}{13.597351\du}}
\pgfpathlineto{\pgfpoint{11.680426\du}{13.602462\du}}
\pgfpathlineto{\pgfpoint{11.717666\du}{13.608303\du}}
\pgfpathlineto{\pgfpoint{11.754176\du}{13.614145\du}}
\pgfpathlineto{\pgfpoint{11.791051\du}{13.620717\du}}
\pgfpathlineto{\pgfpoint{11.826465\du}{13.626558\du}}
\pgfpathlineto{\pgfpoint{11.862244\du}{13.633495\du}}
\pgfpathlineto{\pgfpoint{11.896929\du}{13.640432\du}}
\pgfpathlineto{\pgfpoint{11.930883\du}{13.648099\du}}
\pgfpathlineto{\pgfpoint{11.965202\du}{13.655766\du}}
\pgfpathlineto{\pgfpoint{11.998426\du}{13.663433\du}}
\pgfpathlineto{\pgfpoint{12.030554\du}{13.671465\du}}
\pgfpathlineto{\pgfpoint{12.062683\du}{13.680228\du}}
\pgfpathlineto{\pgfpoint{12.094081\du}{13.688625\du}}
\pgfpathlineto{\pgfpoint{12.124749\du}{13.697752\du}}
\pgfpathlineto{\pgfpoint{12.154687\du}{13.706514\du}}
\pgfpathlineto{\pgfpoint{12.168926\du}{13.711261\du}}
\pgfpathlineto{\pgfpoint{12.183530\du}{13.716007\du}}
\pgfpathlineto{\pgfpoint{12.197403\du}{13.721118\du}}
\pgfpathlineto{\pgfpoint{12.212007\du}{13.725865\du}}
\pgfpathlineto{\pgfpoint{12.226246\du}{13.730611\du}}
\pgfpathlineto{\pgfpoint{12.240120\du}{13.736087\du}}
\pgfpathlineto{\pgfpoint{12.254358\du}{13.740468\du}}
\pgfpathlineto{\pgfpoint{12.267502\du}{13.745945\du}}
\pgfpathlineto{\pgfpoint{12.281010\du}{13.751056\du}}
\pgfpathlineto{\pgfpoint{12.294154\du}{13.755803\du}}
\pgfpathlineto{\pgfpoint{12.306932\du}{13.761644\du}}
\pgfpathlineto{\pgfpoint{12.319711\du}{13.766755\du}}
\pgfpathlineto{\pgfpoint{12.332854\du}{13.772232\du}}
\pgfpathlineto{\pgfpoint{12.344902\du}{13.777343\du}}
\pgfpathlineto{\pgfpoint{12.357316\du}{13.782820\du}}
\pgfpathlineto{\pgfpoint{12.369364\du}{13.788661\du}}
\pgfpathlineto{\pgfpoint{12.381777\du}{13.793773\du}}
\pgfpathlineto{\pgfpoint{12.393095\du}{13.799249\du}}
\pgfpathlineto{\pgfpoint{12.405143\du}{13.805091\du}}
\pgfpathlineto{\pgfpoint{12.415731\du}{13.810932\du}}
\pgfpathlineto{\pgfpoint{12.427779\du}{13.816774\du}}
\pgfpathlineto{\pgfpoint{12.438002\du}{13.822615\du}}
\pgfpathlineto{\pgfpoint{12.448955\du}{13.827727\du}}
\pgfpathlineto{\pgfpoint{12.459908\du}{13.834298\du}}
\pgfpathlineto{\pgfpoint{12.469765\du}{13.840140\du}}
\pgfpathlineto{\pgfpoint{12.479623\du}{13.845981\du}}
\pgfpathlineto{\pgfpoint{12.489846\du}{13.851823\du}}
\pgfpathlineto{\pgfpoint{12.498973\du}{13.857665\du}}
\pgfpathlineto{\pgfpoint{12.509196\du}{13.864236\du}}
\pgfpathlineto{\pgfpoint{12.518688\du}{13.870078\du}}
\pgfpathlineto{\pgfpoint{12.527451\du}{13.876284\du}}
\pgfpathlineto{\pgfpoint{12.535483\du}{13.882126\du}}
\pgfpathlineto{\pgfpoint{12.544245\du}{13.888698\du}}
\pgfpathlineto{\pgfpoint{12.552642\du}{13.894904\du}}
\pgfpathlineto{\pgfpoint{12.561405\du}{13.901476\du}}
\pgfpathlineto{\pgfpoint{12.569072\du}{13.908048\du}}
\pgfpathlineto{\pgfpoint{12.577104\du}{13.913889\du}}
\pgfpathlineto{\pgfpoint{12.584406\du}{13.920096\du}}
\pgfpathlineto{\pgfpoint{12.592073\du}{13.927398\du}}
\pgfpathlineto{\pgfpoint{12.598645\du}{13.933240\du}}
\pgfpathlineto{\pgfpoint{12.605947\du}{13.940176\du}}
\pgfpathlineto{\pgfpoint{12.611788\du}{13.946748\du}}
\pgfpathlineto{\pgfpoint{12.618725\du}{13.952955\du}}
\pgfpathlineto{\pgfpoint{12.625297\du}{13.959527\du}}
\pgfpathlineto{\pgfpoint{12.630408\du}{13.966463\du}}
\pgfpathlineto{\pgfpoint{12.635884\du}{13.973035\du}}
\pgfpathlineto{\pgfpoint{12.642091\du}{13.979972\du}}
\pgfpathlineto{\pgfpoint{12.646837\du}{13.986179\du}}
\pgfpathlineto{\pgfpoint{12.651949\du}{13.992750\du}}
\pgfpathlineto{\pgfpoint{12.656695\du}{13.999687\du}}
\pgfpathlineto{\pgfpoint{12.661076\du}{14.006259\du}}
\pgfpathlineto{\pgfpoint{12.665457\du}{14.013196\du}}
\pgfpathlineto{\pgfpoint{12.668378\du}{14.019767\du}}
\pgfpathlineto{\pgfpoint{12.672394\du}{14.026704\du}}
\pgfpathlineto{\pgfpoint{12.675315\du}{14.033641\du}}
\pgfpathlineto{\pgfpoint{12.679331\du}{14.040943\du}}
\pgfpathlineto{\pgfpoint{12.681887\du}{14.047150\du}}
\pgfpathlineto{\pgfpoint{12.684807\du}{14.054087\du}}
\pgfpathlineto{\pgfpoint{12.687363\du}{14.061389\du}}
\pgfpathlineto{\pgfpoint{12.689554\du}{14.068325\du}}
\pgfpathlineto{\pgfpoint{12.691014\du}{14.075262\du}}
\pgfpathlineto{\pgfpoint{12.693205\du}{14.082199\du}}
\pgfpathlineto{\pgfpoint{12.693935\du}{14.089501\du}}
\pgfpathlineto{\pgfpoint{12.695030\du}{14.095708\du}}
\pgfpathlineto{\pgfpoint{12.696856\du}{14.103375\du}}
\pgfpathlineto{\pgfpoint{12.697221\du}{14.110311\du}}
\pgfpathlineto{\pgfpoint{12.697221\du}{14.117248\du}}
\pgfpathlineto{\pgfpoint{12.697951\du}{14.124550\du}}
\pgfpathlineto{\pgfpoint{12.718031\du}{14.124550\du}}
\pgfusepath{fill}
\pgfsetbuttcap
\pgfsetmiterjoin
\pgfsetdash{}{0pt}
\definecolor{dialinecolor}{rgb}{0.027451, 0.486275, 0.682353}
\pgfsetfillcolor{dialinecolor}
\pgfpathmoveto{\pgfpoint{9.341982\du}{13.315131\du}}
\pgfpathlineto{\pgfpoint{9.341982\du}{14.139519\du}}
\pgfpathlineto{\pgfpoint{12.707443\du}{14.139519\du}}
\pgfpathlineto{\pgfpoint{12.708174\du}{13.315861\du}}
\pgfpathlineto{\pgfpoint{9.341982\du}{13.315131\du}}
\pgfusepath{fill}
\pgfsetbuttcap
\pgfsetmiterjoin
\pgfsetdash{}{0pt}
\definecolor{dialinecolor}{rgb}{0.235294, 0.686275, 0.894118}
\pgfsetfillcolor{dialinecolor}
\pgfpathmoveto{\pgfpoint{12.707443\du}{13.299432\du}}
\pgfpathlineto{\pgfpoint{12.705983\du}{13.329370\du}}
\pgfpathlineto{\pgfpoint{12.698681\du}{13.358577\du}}
\pgfpathlineto{\pgfpoint{12.688458\du}{13.387785\du}}
\pgfpathlineto{\pgfpoint{12.673855\du}{13.415897\du}}
\pgfpathlineto{\pgfpoint{12.654504\du}{13.444010\du}}
\pgfpathlineto{\pgfpoint{12.632599\du}{13.471392\du}}
\pgfpathlineto{\pgfpoint{12.605947\du}{13.497679\du}}
\pgfpathlineto{\pgfpoint{12.575278\du}{13.523966\du}}
\pgfpathlineto{\pgfpoint{12.542420\du}{13.549888\du}}
\pgfpathlineto{\pgfpoint{12.505180\du}{13.575080\du}}
\pgfpathlineto{\pgfpoint{12.464289\du}{13.599176\du}}
\pgfpathlineto{\pgfpoint{12.420477\du}{13.622542\du}}
\pgfpathlineto{\pgfpoint{12.374110\du}{13.644813\du}}
\pgfpathlineto{\pgfpoint{12.323727\du}{13.666719\du}}
\pgfpathlineto{\pgfpoint{12.270788\du}{13.687895\du}}
\pgfpathlineto{\pgfpoint{12.215293\du}{13.707975\du}}
\pgfpathlineto{\pgfpoint{12.157243\du}{13.726595\du}}
\pgfpathlineto{\pgfpoint{12.096637\du}{13.745215\du}}
\pgfpathlineto{\pgfpoint{12.032745\du}{13.762374\du}}
\pgfpathlineto{\pgfpoint{11.967757\du}{13.778073\du}}
\pgfpathlineto{\pgfpoint{11.899119\du}{13.793408\du}}
\pgfpathlineto{\pgfpoint{11.828290\du}{13.807281\du}}
\pgfpathlineto{\pgfpoint{11.756366\du}{13.820060\du}}
\pgfpathlineto{\pgfpoint{11.681522\du}{13.831378\du}}
\pgfpathlineto{\pgfpoint{11.605581\du}{13.841965\du}}
\pgfpathlineto{\pgfpoint{11.527086\du}{13.851093\du}}
\pgfpathlineto{\pgfpoint{11.447495\du}{13.858760\du}}
\pgfpathlineto{\pgfpoint{11.366443\du}{13.865332\du}}
\pgfpathlineto{\pgfpoint{11.283201\du}{13.870443\du}}
\pgfpathlineto{\pgfpoint{11.199594\du}{13.874094\du}}
\pgfpathlineto{\pgfpoint{11.114161\du}{13.875919\du}}
\pgfpathlineto{\pgfpoint{11.027633\du}{13.877015\du}}
\pgfpathlineto{\pgfpoint{10.941470\du}{13.875919\du}}
\pgfpathlineto{\pgfpoint{10.855673\du}{13.874094\du}}
\pgfpathlineto{\pgfpoint{10.772066\du}{13.870443\du}}
\pgfpathlineto{\pgfpoint{10.689189\du}{13.865332\du}}
\pgfpathlineto{\pgfpoint{10.607772\du}{13.858760\du}}
\pgfpathlineto{\pgfpoint{10.528181\du}{13.851093\du}}
\pgfpathlineto{\pgfpoint{10.450415\du}{13.841965\du}}
\pgfpathlineto{\pgfpoint{10.373745\du}{13.831378\du}}
\pgfpathlineto{\pgfpoint{10.299630\du}{13.820060\du}}
\pgfpathlineto{\pgfpoint{10.226976\du}{13.807281\du}}
\pgfpathlineto{\pgfpoint{10.156512\du}{13.793408\du}}
\pgfpathlineto{\pgfpoint{10.087874\du}{13.778073\du}}
\pgfpathlineto{\pgfpoint{10.022157\du}{13.762374\du}}
\pgfpathlineto{\pgfpoint{9.958630\du}{13.745215\du}}
\pgfpathlineto{\pgfpoint{9.897659\du}{13.726595\du}}
\pgfpathlineto{\pgfpoint{9.839243\du}{13.707975\du}}
\pgfpathlineto{\pgfpoint{9.784114\du}{13.687895\du}}
\pgfpathlineto{\pgfpoint{9.731175\du}{13.666719\du}}
\pgfpathlineto{\pgfpoint{9.681156\du}{13.644813\du}}
\pgfpathlineto{\pgfpoint{9.634059\du}{13.622542\du}}
\pgfpathlineto{\pgfpoint{9.590612\du}{13.599176\du}}
\pgfpathlineto{\pgfpoint{9.549722\du}{13.575080\du}}
\pgfpathlineto{\pgfpoint{9.512482\du}{13.549888\du}}
\pgfpathlineto{\pgfpoint{9.479258\du}{13.523966\du}}
\pgfpathlineto{\pgfpoint{9.448955\du}{13.497679\du}}
\pgfpathlineto{\pgfpoint{9.422303\du}{13.471392\du}}
\pgfpathlineto{\pgfpoint{9.400397\du}{13.444010\du}}
\pgfpathlineto{\pgfpoint{9.381047\du}{13.415897\du}}
\pgfpathlineto{\pgfpoint{9.366443\du}{13.387785\du}}
\pgfpathlineto{\pgfpoint{9.355855\du}{13.358577\du}}
\pgfpathlineto{\pgfpoint{9.348918\du}{13.329370\du}}
\pgfpathlineto{\pgfpoint{9.347093\du}{13.299432\du}}
\pgfpathlineto{\pgfpoint{9.348918\du}{13.270224\du}}
\pgfpathlineto{\pgfpoint{9.355855\du}{13.240286\du}}
\pgfpathlineto{\pgfpoint{9.366443\du}{13.211808\du}}
\pgfpathlineto{\pgfpoint{9.381047\du}{13.183696\du}}
\pgfpathlineto{\pgfpoint{9.400397\du}{13.155583\du}}
\pgfpathlineto{\pgfpoint{9.422303\du}{13.127836\du}}
\pgfpathlineto{\pgfpoint{9.448955\du}{13.101184\du}}
\pgfpathlineto{\pgfpoint{9.479258\du}{13.074897\du}}
\pgfpathlineto{\pgfpoint{9.512482\du}{13.049705\du}}
\pgfpathlineto{\pgfpoint{9.549722\du}{13.024514\du}}
\pgfpathlineto{\pgfpoint{9.590612\du}{13.000417\du}}
\pgfpathlineto{\pgfpoint{9.634059\du}{12.977051\du}}
\pgfpathlineto{\pgfpoint{9.681156\du}{12.954050\du}}
\pgfpathlineto{\pgfpoint{9.731175\du}{12.932509\du}}
\pgfpathlineto{\pgfpoint{9.784114\du}{12.911334\du}}
\pgfpathlineto{\pgfpoint{9.839243\du}{12.891619\du}}
\pgfpathlineto{\pgfpoint{9.897659\du}{12.872268\du}}
\pgfpathlineto{\pgfpoint{9.958630\du}{12.854014\du}}
\pgfpathlineto{\pgfpoint{10.022157\du}{12.837219\du}}
\pgfpathlineto{\pgfpoint{10.087874\du}{12.820790\du}}
\pgfpathlineto{\pgfpoint{10.156512\du}{12.805456\du}}
\pgfpathlineto{\pgfpoint{10.226976\du}{12.791947\du}}
\pgfpathlineto{\pgfpoint{10.299630\du}{12.779169\du}}
\pgfpathlineto{\pgfpoint{10.373745\du}{12.767486\du}}
\pgfpathlineto{\pgfpoint{10.450415\du}{12.757628\du}}
\pgfpathlineto{\pgfpoint{10.528181\du}{12.748136\du}}
\pgfpathlineto{\pgfpoint{10.607772\du}{12.740468\du}}
\pgfpathlineto{\pgfpoint{10.689189\du}{12.733897\du}}
\pgfpathlineto{\pgfpoint{10.772066\du}{12.728785\du}}
\pgfpathlineto{\pgfpoint{10.855673\du}{12.725500\du}}
\pgfpathlineto{\pgfpoint{10.941470\du}{12.722944\du}}
\pgfpathlineto{\pgfpoint{11.027633\du}{12.722579\du}}
\pgfpathlineto{\pgfpoint{11.114161\du}{12.722944\du}}
\pgfpathlineto{\pgfpoint{11.199594\du}{12.725500\du}}
\pgfpathlineto{\pgfpoint{11.283201\du}{12.728785\du}}
\pgfpathlineto{\pgfpoint{11.366443\du}{12.733897\du}}
\pgfpathlineto{\pgfpoint{11.447495\du}{12.740468\du}}
\pgfpathlineto{\pgfpoint{11.527086\du}{12.748136\du}}
\pgfpathlineto{\pgfpoint{11.605581\du}{12.757628\du}}
\pgfpathlineto{\pgfpoint{11.681522\du}{12.767486\du}}
\pgfpathlineto{\pgfpoint{11.756366\du}{12.779169\du}}
\pgfpathlineto{\pgfpoint{11.828290\du}{12.791947\du}}
\pgfpathlineto{\pgfpoint{11.899119\du}{12.805456\du}}
\pgfpathlineto{\pgfpoint{11.967757\du}{12.820790\du}}
\pgfpathlineto{\pgfpoint{12.032745\du}{12.837219\du}}
\pgfpathlineto{\pgfpoint{12.096637\du}{12.854014\du}}
\pgfpathlineto{\pgfpoint{12.157243\du}{12.872268\du}}
\pgfpathlineto{\pgfpoint{12.215293\du}{12.891619\du}}
\pgfpathlineto{\pgfpoint{12.270788\du}{12.911334\du}}
\pgfpathlineto{\pgfpoint{12.323727\du}{12.932509\du}}
\pgfpathlineto{\pgfpoint{12.374110\du}{12.954050\du}}
\pgfpathlineto{\pgfpoint{12.420477\du}{12.977051\du}}
\pgfpathlineto{\pgfpoint{12.464289\du}{13.000417\du}}
\pgfpathlineto{\pgfpoint{12.505180\du}{13.024514\du}}
\pgfpathlineto{\pgfpoint{12.542420\du}{13.049705\du}}
\pgfpathlineto{\pgfpoint{12.575278\du}{13.074897\du}}
\pgfpathlineto{\pgfpoint{12.605947\du}{13.101184\du}}
\pgfpathlineto{\pgfpoint{12.632599\du}{13.127836\du}}
\pgfpathlineto{\pgfpoint{12.654504\du}{13.155583\du}}
\pgfpathlineto{\pgfpoint{12.673855\du}{13.183696\du}}
\pgfpathlineto{\pgfpoint{12.688458\du}{13.211808\du}}
\pgfpathlineto{\pgfpoint{12.698681\du}{13.240286\du}}
\pgfpathlineto{\pgfpoint{12.705983\du}{13.270224\du}}
\pgfpathlineto{\pgfpoint{12.707443\du}{13.299432\du}}
\pgfusepath{fill}
\pgfsetbuttcap
\pgfsetmiterjoin
\pgfsetdash{}{0pt}
\definecolor{dialinecolor}{rgb}{0.678431, 0.839216, 0.905882}
\pgfsetfillcolor{dialinecolor}
\pgfpathmoveto{\pgfpoint{11.027633\du}{13.886872\du}}
\pgfpathlineto{\pgfpoint{11.027633\du}{13.886872\du}}
\pgfpathlineto{\pgfpoint{11.071080\du}{13.886872\du}}
\pgfpathlineto{\pgfpoint{11.114526\du}{13.886142\du}}
\pgfpathlineto{\pgfpoint{11.157608\du}{13.885047\du}}
\pgfpathlineto{\pgfpoint{11.199594\du}{13.883952\du}}
\pgfpathlineto{\pgfpoint{11.242310\du}{13.882126\du}}
\pgfpathlineto{\pgfpoint{11.283931\du}{13.880301\du}}
\pgfpathlineto{\pgfpoint{11.325552\du}{13.878110\du}}
\pgfpathlineto{\pgfpoint{11.367173\du}{13.875189\du}}
\pgfpathlineto{\pgfpoint{11.407699\du}{13.872268\du}}
\pgfpathlineto{\pgfpoint{11.448590\du}{13.869348\du}}
\pgfpathlineto{\pgfpoint{11.488385\du}{13.865332\du}}
\pgfpathlineto{\pgfpoint{11.528546\du}{13.861316\du}}
\pgfpathlineto{\pgfpoint{11.567246\du}{13.856569\du}}
\pgfpathlineto{\pgfpoint{11.606677\du}{13.851823\du}}
\pgfpathlineto{\pgfpoint{11.644647\du}{13.847077\du}}
\pgfpathlineto{\pgfpoint{11.683712\du}{13.841965\du}}
\pgfpathlineto{\pgfpoint{11.720952\du}{13.836124\du}}
\pgfpathlineto{\pgfpoint{11.757827\du}{13.830282\du}}
\pgfpathlineto{\pgfpoint{11.794336\du}{13.823711\du}}
\pgfpathlineto{\pgfpoint{11.830846\du}{13.817139\du}}
\pgfpathlineto{\pgfpoint{11.866261\du}{13.810202\du}}
\pgfpathlineto{\pgfpoint{11.900945\du}{13.803265\du}}
\pgfpathlineto{\pgfpoint{11.935629\du}{13.796328\du}}
\pgfpathlineto{\pgfpoint{11.969583\du}{13.788661\du}}
\pgfpathlineto{\pgfpoint{12.003172\du}{13.780264\du}}
\pgfpathlineto{\pgfpoint{12.036031\du}{13.772232\du}}
\pgfpathlineto{\pgfpoint{12.067794\du}{13.763835\du}}
\pgfpathlineto{\pgfpoint{12.099192\du}{13.754707\du}}
\pgfpathlineto{\pgfpoint{12.114526\du}{13.750691\du}}
\pgfpathlineto{\pgfpoint{12.129860\du}{13.745945\du}}
\pgfpathlineto{\pgfpoint{12.145925\du}{13.741199\du}}
\pgfpathlineto{\pgfpoint{12.160528\du}{13.736452\du}}
\pgfpathlineto{\pgfpoint{12.174767\du}{13.731706\du}}
\pgfpathlineto{\pgfpoint{12.189736\du}{13.727325\du}}
\pgfpathlineto{\pgfpoint{12.204705\du}{13.722579\du}}
\pgfpathlineto{\pgfpoint{12.218579\du}{13.717102\du}}
\pgfpathlineto{\pgfpoint{12.232818\du}{13.712356\du}}
\pgfpathlineto{\pgfpoint{12.246691\du}{13.707245\du}}
\pgfpathlineto{\pgfpoint{12.260930\du}{13.702498\du}}
\pgfpathlineto{\pgfpoint{12.274074\du}{13.697387\du}}
\pgfpathlineto{\pgfpoint{12.288312\du}{13.691911\du}}
\pgfpathlineto{\pgfpoint{12.301456\du}{13.686799\du}}
\pgfpathlineto{\pgfpoint{12.314599\du}{13.681688\du}}
\pgfpathlineto{\pgfpoint{12.328108\du}{13.676211\du}}
\pgfpathlineto{\pgfpoint{12.340886\du}{13.671100\du}}
\pgfpathlineto{\pgfpoint{12.352934\du}{13.665624\du}}
\pgfpathlineto{\pgfpoint{12.365713\du}{13.659782\du}}
\pgfpathlineto{\pgfpoint{12.378126\du}{13.653941\du}}
\pgfpathlineto{\pgfpoint{12.390539\du}{13.648829\du}}
\pgfpathlineto{\pgfpoint{12.401857\du}{13.642988\du}}
\pgfpathlineto{\pgfpoint{12.413906\du}{13.637146\du}}
\pgfpathlineto{\pgfpoint{12.424859\du}{13.631305\du}}
\pgfpathlineto{\pgfpoint{12.436542\du}{13.625828\du}}
\pgfpathlineto{\pgfpoint{12.447860\du}{13.619987\du}}
\pgfpathlineto{\pgfpoint{12.459178\du}{13.613780\du}}
\pgfpathlineto{\pgfpoint{12.469765\du}{13.607938\du}}
\pgfpathlineto{\pgfpoint{12.479623\du}{13.602097\du}}
\pgfpathlineto{\pgfpoint{12.490211\du}{13.596255\du}}
\pgfpathlineto{\pgfpoint{12.500434\du}{13.589684\du}}
\pgfpathlineto{\pgfpoint{12.510291\du}{13.583112\du}}
\pgfpathlineto{\pgfpoint{12.520149\du}{13.577270\du}}
\pgfpathlineto{\pgfpoint{12.529276\du}{13.571064\du}}
\pgfpathlineto{\pgfpoint{12.538404\du}{13.564492\du}}
\pgfpathlineto{\pgfpoint{12.547896\du}{13.557920\du}}
\pgfpathlineto{\pgfpoint{12.556658\du}{13.551713\du}}
\pgfpathlineto{\pgfpoint{12.565786\du}{13.545142\du}}
\pgfpathlineto{\pgfpoint{12.573818\du}{13.538570\du}}
\pgfpathlineto{\pgfpoint{12.582580\du}{13.532363\du}}
\pgfpathlineto{\pgfpoint{12.589882\du}{13.525792\du}}
\pgfpathlineto{\pgfpoint{12.597914\du}{13.518855\du}}
\pgfpathlineto{\pgfpoint{12.605947\du}{13.512283\du}}
\pgfpathlineto{\pgfpoint{12.612518\du}{13.505346\du}}
\pgfpathlineto{\pgfpoint{12.619820\du}{13.498774\du}}
\pgfpathlineto{\pgfpoint{12.626392\du}{13.491838\du}}
\pgfpathlineto{\pgfpoint{12.633329\du}{13.484901\du}}
\pgfpathlineto{\pgfpoint{12.639901\du}{13.478329\du}}
\pgfpathlineto{\pgfpoint{12.646107\du}{13.471392\du}}
\pgfpathlineto{\pgfpoint{12.651949\du}{13.464455\du}}
\pgfpathlineto{\pgfpoint{12.657425\du}{13.457519\du}}
\pgfpathlineto{\pgfpoint{12.663267\du}{13.449851\du}}
\pgfpathlineto{\pgfpoint{12.668013\du}{13.442915\du}}
\pgfpathlineto{\pgfpoint{12.673489\du}{13.435613\du}}
\pgfpathlineto{\pgfpoint{12.678236\du}{13.428676\du}}
\pgfpathlineto{\pgfpoint{12.682617\du}{13.421009\du}}
\pgfpathlineto{\pgfpoint{12.686633\du}{13.414072\du}}
\pgfpathlineto{\pgfpoint{12.690649\du}{13.406405\du}}
\pgfpathlineto{\pgfpoint{12.693935\du}{13.398738\du}}
\pgfpathlineto{\pgfpoint{12.697951\du}{13.391436\du}}
\pgfpathlineto{\pgfpoint{12.701237\du}{13.384134\du}}
\pgfpathlineto{\pgfpoint{12.703792\du}{13.376832\du}}
\pgfpathlineto{\pgfpoint{12.706713\du}{13.369165\du}}
\pgfpathlineto{\pgfpoint{12.708539\du}{13.361498\du}}
\pgfpathlineto{\pgfpoint{12.711459\du}{13.353831\du}}
\pgfpathlineto{\pgfpoint{12.712555\du}{13.346164\du}}
\pgfpathlineto{\pgfpoint{12.714745\du}{13.338497\du}}
\pgfpathlineto{\pgfpoint{12.715841\du}{13.331195\du}}
\pgfpathlineto{\pgfpoint{12.716571\du}{13.322798\du}}
\pgfpathlineto{\pgfpoint{12.717301\du}{13.315131\du}}
\pgfpathlineto{\pgfpoint{12.718031\du}{13.307464\du}}
\pgfpathlineto{\pgfpoint{12.718031\du}{13.299432\du}}
\pgfpathlineto{\pgfpoint{12.697951\du}{13.299432\du}}
\pgfpathlineto{\pgfpoint{12.697221\du}{13.306368\du}}
\pgfpathlineto{\pgfpoint{12.697221\du}{13.314035\du}}
\pgfpathlineto{\pgfpoint{12.696856\du}{13.320972\du}}
\pgfpathlineto{\pgfpoint{12.695030\du}{13.327909\du}}
\pgfpathlineto{\pgfpoint{12.693935\du}{13.335211\du}}
\pgfpathlineto{\pgfpoint{12.693205\du}{13.341418\du}}
\pgfpathlineto{\pgfpoint{12.691014\du}{13.348720\du}}
\pgfpathlineto{\pgfpoint{12.689554\du}{13.355657\du}}
\pgfpathlineto{\pgfpoint{12.687363\du}{13.362593\du}}
\pgfpathlineto{\pgfpoint{12.684807\du}{13.369530\du}}
\pgfpathlineto{\pgfpoint{12.681887\du}{13.376832\du}}
\pgfpathlineto{\pgfpoint{12.679331\du}{13.383039\du}}
\pgfpathlineto{\pgfpoint{12.675315\du}{13.389976\du}}
\pgfpathlineto{\pgfpoint{12.672394\du}{13.397278\du}}
\pgfpathlineto{\pgfpoint{12.668378\du}{13.404214\du}}
\pgfpathlineto{\pgfpoint{12.665457\du}{13.410421\du}}
\pgfpathlineto{\pgfpoint{12.661076\du}{13.417723\du}}
\pgfpathlineto{\pgfpoint{12.656695\du}{13.423930\du}}
\pgfpathlineto{\pgfpoint{12.651949\du}{13.431232\du}}
\pgfpathlineto{\pgfpoint{12.646837\du}{13.437438\du}}
\pgfpathlineto{\pgfpoint{12.642091\du}{13.444375\du}}
\pgfpathlineto{\pgfpoint{12.635884\du}{13.450947\du}}
\pgfpathlineto{\pgfpoint{12.630408\du}{13.457519\du}}
\pgfpathlineto{\pgfpoint{12.625297\du}{13.464455\du}}
\pgfpathlineto{\pgfpoint{12.618725\du}{13.471027\du}}
\pgfpathlineto{\pgfpoint{12.611788\du}{13.477234\du}}
\pgfpathlineto{\pgfpoint{12.605947\du}{13.483805\du}}
\pgfpathlineto{\pgfpoint{12.598645\du}{13.490742\du}}
\pgfpathlineto{\pgfpoint{12.592073\du}{13.497314\du}}
\pgfpathlineto{\pgfpoint{12.584406\du}{13.503521\du}}
\pgfpathlineto{\pgfpoint{12.577104\du}{13.510092\du}}
\pgfpathlineto{\pgfpoint{12.569072\du}{13.515934\du}}
\pgfpathlineto{\pgfpoint{12.561405\du}{13.522506\du}}
\pgfpathlineto{\pgfpoint{12.552642\du}{13.528712\du}}
\pgfpathlineto{\pgfpoint{12.544245\du}{13.535284\du}}
\pgfpathlineto{\pgfpoint{12.535483\du}{13.541491\du}}
\pgfpathlineto{\pgfpoint{12.527451\du}{13.547332\du}}
\pgfpathlineto{\pgfpoint{12.518688\du}{13.553904\du}}
\pgfpathlineto{\pgfpoint{12.509196\du}{13.559746\du}}
\pgfpathlineto{\pgfpoint{12.498973\du}{13.566317\du}}
\pgfpathlineto{\pgfpoint{12.489846\du}{13.572159\du}}
\pgfpathlineto{\pgfpoint{12.479623\du}{13.578000\du}}
\pgfpathlineto{\pgfpoint{12.469765\du}{13.583842\du}}
\pgfpathlineto{\pgfpoint{12.459908\du}{13.589684\du}}
\pgfpathlineto{\pgfpoint{12.448955\du}{13.596255\du}}
\pgfpathlineto{\pgfpoint{12.438002\du}{13.601367\du}}
\pgfpathlineto{\pgfpoint{12.427779\du}{13.607208\du}}
\pgfpathlineto{\pgfpoint{12.415731\du}{13.613050\du}}
\pgfpathlineto{\pgfpoint{12.405143\du}{13.618891\du}}
\pgfpathlineto{\pgfpoint{12.393095\du}{13.624733\du}}
\pgfpathlineto{\pgfpoint{12.381777\du}{13.629844\du}}
\pgfpathlineto{\pgfpoint{12.369364\du}{13.635321\du}}
\pgfpathlineto{\pgfpoint{12.357316\du}{13.641162\du}}
\pgfpathlineto{\pgfpoint{12.344902\du}{13.646274\du}}
\pgfpathlineto{\pgfpoint{12.332854\du}{13.651750\du}}
\pgfpathlineto{\pgfpoint{12.319711\du}{13.656861\du}}
\pgfpathlineto{\pgfpoint{12.306932\du}{13.662338\du}}
\pgfpathlineto{\pgfpoint{12.294154\du}{13.668179\du}}
\pgfpathlineto{\pgfpoint{12.281010\du}{13.672560\du}}
\pgfpathlineto{\pgfpoint{12.267502\du}{13.678037\du}}
\pgfpathlineto{\pgfpoint{12.254358\du}{13.683148\du}}
\pgfpathlineto{\pgfpoint{12.240120\du}{13.687895\du}}
\pgfpathlineto{\pgfpoint{12.226246\du}{13.693371\du}}
\pgfpathlineto{\pgfpoint{12.212007\du}{13.697752\du}}
\pgfpathlineto{\pgfpoint{12.197403\du}{13.702498\du}}
\pgfpathlineto{\pgfpoint{12.183530\du}{13.707975\du}}
\pgfpathlineto{\pgfpoint{12.168926\du}{13.712356\du}}
\pgfpathlineto{\pgfpoint{12.154687\du}{13.717102\du}}
\pgfpathlineto{\pgfpoint{12.138988\du}{13.721849\du}}
\pgfpathlineto{\pgfpoint{12.124749\du}{13.725865\du}}
\pgfpathlineto{\pgfpoint{12.109050\du}{13.730611\du}}
\pgfpathlineto{\pgfpoint{12.094081\du}{13.735357\du}}
\pgfpathlineto{\pgfpoint{12.062683\du}{13.743389\du}}
\pgfpathlineto{\pgfpoint{12.030554\du}{13.752152\du}}
\pgfpathlineto{\pgfpoint{11.998426\du}{13.760549\du}}
\pgfpathlineto{\pgfpoint{11.965202\du}{13.768216\du}}
\pgfpathlineto{\pgfpoint{11.930883\du}{13.775883\du}}
\pgfpathlineto{\pgfpoint{11.896929\du}{13.783185\du}}
\pgfpathlineto{\pgfpoint{11.862244\du}{13.790487\du}}
\pgfpathlineto{\pgfpoint{11.826465\du}{13.797424\du}}
\pgfpathlineto{\pgfpoint{11.791051\du}{13.803630\du}}
\pgfpathlineto{\pgfpoint{11.754176\du}{13.809472\du}}
\pgfpathlineto{\pgfpoint{11.717666\du}{13.815678\du}}
\pgfpathlineto{\pgfpoint{11.680426\du}{13.821520\du}}
\pgfpathlineto{\pgfpoint{11.642821\du}{13.826631\du}}
\pgfpathlineto{\pgfpoint{11.604121\du}{13.831743\du}}
\pgfpathlineto{\pgfpoint{11.565421\du}{13.836489\du}}
\pgfpathlineto{\pgfpoint{11.525990\du}{13.840505\du}}
\pgfpathlineto{\pgfpoint{11.486925\du}{13.844886\du}}
\pgfpathlineto{\pgfpoint{11.446764\du}{13.848172\du}}
\pgfpathlineto{\pgfpoint{11.406604\du}{13.851823\du}}
\pgfpathlineto{\pgfpoint{11.365348\du}{13.854744\du}}
\pgfpathlineto{\pgfpoint{11.324457\du}{13.857665\du}}
\pgfpathlineto{\pgfpoint{11.283201\du}{13.859855\du}}
\pgfpathlineto{\pgfpoint{11.241580\du}{13.862411\du}}
\pgfpathlineto{\pgfpoint{11.198864\du}{13.863506\du}}
\pgfpathlineto{\pgfpoint{11.156147\du}{13.865332\du}}
\pgfpathlineto{\pgfpoint{11.114161\du}{13.865697\du}}
\pgfpathlineto{\pgfpoint{11.070715\du}{13.866427\du}}
\pgfpathlineto{\pgfpoint{11.027633\du}{13.866427\du}}
\pgfpathlineto{\pgfpoint{11.027633\du}{13.866427\du}}
\pgfpathlineto{\pgfpoint{11.027633\du}{13.866427\du}}
\pgfpathlineto{\pgfpoint{11.026903\du}{13.867157\du}}
\pgfpathlineto{\pgfpoint{11.025078\du}{13.867157\du}}
\pgfpathlineto{\pgfpoint{11.023982\du}{13.867157\du}}
\pgfpathlineto{\pgfpoint{11.023252\du}{13.867522\du}}
\pgfpathlineto{\pgfpoint{11.022887\du}{13.868252\du}}
\pgfpathlineto{\pgfpoint{11.021427\du}{13.868252\du}}
\pgfpathlineto{\pgfpoint{11.020696\du}{13.869348\du}}
\pgfpathlineto{\pgfpoint{11.019966\du}{13.870078\du}}
\pgfpathlineto{\pgfpoint{11.018871\du}{13.871173\du}}
\pgfpathlineto{\pgfpoint{11.018141\du}{13.872999\du}}
\pgfpathlineto{\pgfpoint{11.018141\du}{13.875189\du}}
\pgfpathlineto{\pgfpoint{11.017411\du}{13.877015\du}}
\pgfpathlineto{\pgfpoint{11.018141\du}{13.878840\du}}
\pgfpathlineto{\pgfpoint{11.018141\du}{13.880301\du}}
\pgfpathlineto{\pgfpoint{11.018871\du}{13.882126\du}}
\pgfpathlineto{\pgfpoint{11.019966\du}{13.883952\du}}
\pgfpathlineto{\pgfpoint{11.020696\du}{13.884682\du}}
\pgfpathlineto{\pgfpoint{11.021427\du}{13.885047\du}}
\pgfpathlineto{\pgfpoint{11.022887\du}{13.885777\du}}
\pgfpathlineto{\pgfpoint{11.023252\du}{13.886142\du}}
\pgfpathlineto{\pgfpoint{11.023982\du}{13.886872\du}}
\pgfpathlineto{\pgfpoint{11.025078\du}{13.886872\du}}
\pgfpathlineto{\pgfpoint{11.026903\du}{13.886872\du}}
\pgfpathlineto{\pgfpoint{11.027633\du}{13.886872\du}}
\pgfusepath{fill}
\pgfsetbuttcap
\pgfsetmiterjoin
\pgfsetdash{}{0pt}
\definecolor{dialinecolor}{rgb}{0.678431, 0.839216, 0.905882}
\pgfsetfillcolor{dialinecolor}
\pgfpathmoveto{\pgfpoint{9.336870\du}{13.299432\du}}
\pgfpathlineto{\pgfpoint{9.336870\du}{13.299432\du}}
\pgfpathlineto{\pgfpoint{9.336870\du}{13.307464\du}}
\pgfpathlineto{\pgfpoint{9.337235\du}{13.315131\du}}
\pgfpathlineto{\pgfpoint{9.337966\du}{13.322798\du}}
\pgfpathlineto{\pgfpoint{9.339061\du}{13.331195\du}}
\pgfpathlineto{\pgfpoint{9.340156\du}{13.338497\du}}
\pgfpathlineto{\pgfpoint{9.341982\du}{13.346164\du}}
\pgfpathlineto{\pgfpoint{9.343807\du}{13.353831\du}}
\pgfpathlineto{\pgfpoint{9.345998\du}{13.361498\du}}
\pgfpathlineto{\pgfpoint{9.348188\du}{13.369165\du}}
\pgfpathlineto{\pgfpoint{9.350744\du}{13.376832\du}}
\pgfpathlineto{\pgfpoint{9.353665\du}{13.384134\du}}
\pgfpathlineto{\pgfpoint{9.357316\du}{13.391436\du}}
\pgfpathlineto{\pgfpoint{9.360602\du}{13.398738\du}}
\pgfpathlineto{\pgfpoint{9.364252\du}{13.406405\du}}
\pgfpathlineto{\pgfpoint{9.368634\du}{13.414072\du}}
\pgfpathlineto{\pgfpoint{9.372285\du}{13.421009\du}}
\pgfpathlineto{\pgfpoint{9.377396\du}{13.428676\du}}
\pgfpathlineto{\pgfpoint{9.381412\du}{13.435613\du}}
\pgfpathlineto{\pgfpoint{9.386888\du}{13.442915\du}}
\pgfpathlineto{\pgfpoint{9.391635\du}{13.449851\du}}
\pgfpathlineto{\pgfpoint{9.397111\du}{13.457519\du}}
\pgfpathlineto{\pgfpoint{9.402953\du}{13.464455\du}}
\pgfpathlineto{\pgfpoint{9.408794\du}{13.471392\du}}
\pgfpathlineto{\pgfpoint{9.414636\du}{13.478329\du}}
\pgfpathlineto{\pgfpoint{9.421573\du}{13.484901\du}}
\pgfpathlineto{\pgfpoint{9.428144\du}{13.491838\du}}
\pgfpathlineto{\pgfpoint{9.435081\du}{13.498774\du}}
\pgfpathlineto{\pgfpoint{9.442018\du}{13.505346\du}}
\pgfpathlineto{\pgfpoint{9.448955\du}{13.512283\du}}
\pgfpathlineto{\pgfpoint{9.457717\du}{13.518855\du}}
\pgfpathlineto{\pgfpoint{9.464654\du}{13.525792\du}}
\pgfpathlineto{\pgfpoint{9.472321\du}{13.532363\du}}
\pgfpathlineto{\pgfpoint{9.481083\du}{13.538570\du}}
\pgfpathlineto{\pgfpoint{9.489481\du}{13.545142\du}}
\pgfpathlineto{\pgfpoint{9.498243\du}{13.551713\du}}
\pgfpathlineto{\pgfpoint{9.506640\du}{13.557920\du}}
\pgfpathlineto{\pgfpoint{9.516498\du}{13.564492\du}}
\pgfpathlineto{\pgfpoint{9.525625\du}{13.571064\du}}
\pgfpathlineto{\pgfpoint{9.535118\du}{13.577270\du}}
\pgfpathlineto{\pgfpoint{9.544610\du}{13.583112\du}}
\pgfpathlineto{\pgfpoint{9.554468\du}{13.589684\du}}
\pgfpathlineto{\pgfpoint{9.564691\du}{13.596255\du}}
\pgfpathlineto{\pgfpoint{9.574913\du}{13.602097\du}}
\pgfpathlineto{\pgfpoint{9.585501\du}{13.607938\du}}
\pgfpathlineto{\pgfpoint{9.595724\du}{13.613780\du}}
\pgfpathlineto{\pgfpoint{9.606677\du}{13.619987\du}}
\pgfpathlineto{\pgfpoint{9.618360\du}{13.625828\du}}
\pgfpathlineto{\pgfpoint{9.629678\du}{13.631305\du}}
\pgfpathlineto{\pgfpoint{9.641361\du}{13.637146\du}}
\pgfpathlineto{\pgfpoint{9.652679\du}{13.642988\du}}
\pgfpathlineto{\pgfpoint{9.664362\du}{13.648829\du}}
\pgfpathlineto{\pgfpoint{9.676775\du}{13.653941\du}}
\pgfpathlineto{\pgfpoint{9.688823\du}{13.659782\du}}
\pgfpathlineto{\pgfpoint{9.701967\du}{13.665624\du}}
\pgfpathlineto{\pgfpoint{9.714015\du}{13.671100\du}}
\pgfpathlineto{\pgfpoint{9.726794\du}{13.676211\du}}
\pgfpathlineto{\pgfpoint{9.740667\du}{13.681688\du}}
\pgfpathlineto{\pgfpoint{9.753081\du}{13.686799\du}}
\pgfpathlineto{\pgfpoint{9.766224\du}{13.691911\du}}
\pgfpathlineto{\pgfpoint{9.780463\du}{13.697387\du}}
\pgfpathlineto{\pgfpoint{9.793606\du}{13.702498\du}}
\pgfpathlineto{\pgfpoint{9.807845\du}{13.707245\du}}
\pgfpathlineto{\pgfpoint{9.821719\du}{13.712356\du}}
\pgfpathlineto{\pgfpoint{9.836323\du}{13.717102\du}}
\pgfpathlineto{\pgfpoint{9.850196\du}{13.722579\du}}
\pgfpathlineto{\pgfpoint{9.865895\du}{13.727325\du}}
\pgfpathlineto{\pgfpoint{9.879769\du}{13.731706\du}}
\pgfpathlineto{\pgfpoint{9.894373\du}{13.736452\du}}
\pgfpathlineto{\pgfpoint{9.909707\du}{13.741199\du}}
\pgfpathlineto{\pgfpoint{9.925406\du}{13.745945\du}}
\pgfpathlineto{\pgfpoint{9.940375\du}{13.750691\du}}
\pgfpathlineto{\pgfpoint{9.956074\du}{13.754707\du}}
\pgfpathlineto{\pgfpoint{9.987838\du}{13.763835\du}}
\pgfpathlineto{\pgfpoint{10.019601\du}{13.772232\du}}
\pgfpathlineto{\pgfpoint{10.052825\du}{13.780264\du}}
\pgfpathlineto{\pgfpoint{10.085319\du}{13.788661\du}}
\pgfpathlineto{\pgfpoint{10.120003\du}{13.796328\du}}
\pgfpathlineto{\pgfpoint{10.154322\du}{13.803265\du}}
\pgfpathlineto{\pgfpoint{10.189006\du}{13.810202\du}}
\pgfpathlineto{\pgfpoint{10.224786\du}{13.817139\du}}
\pgfpathlineto{\pgfpoint{10.260930\du}{13.823711\du}}
\pgfpathlineto{\pgfpoint{10.297440\du}{13.830282\du}}
\pgfpathlineto{\pgfpoint{10.334680\du}{13.836124\du}}
\pgfpathlineto{\pgfpoint{10.372285\du}{13.841965\du}}
\pgfpathlineto{\pgfpoint{10.410255\du}{13.847077\du}}
\pgfpathlineto{\pgfpoint{10.448955\du}{13.851823\du}}
\pgfpathlineto{\pgfpoint{10.487655\du}{13.856569\du}}
\pgfpathlineto{\pgfpoint{10.526721\du}{13.861316\du}}
\pgfpathlineto{\pgfpoint{10.566881\du}{13.865332\du}}
\pgfpathlineto{\pgfpoint{10.606677\du}{13.869348\du}}
\pgfpathlineto{\pgfpoint{10.647568\du}{13.872268\du}}
\pgfpathlineto{\pgfpoint{10.688458\du}{13.875189\du}}
\pgfpathlineto{\pgfpoint{10.729714\du}{13.878110\du}}
\pgfpathlineto{\pgfpoint{10.771700\du}{13.880301\du}}
\pgfpathlineto{\pgfpoint{10.813321\du}{13.882126\du}}
\pgfpathlineto{\pgfpoint{10.855673\du}{13.883952\du}}
\pgfpathlineto{\pgfpoint{10.897659\du}{13.885047\du}}
\pgfpathlineto{\pgfpoint{10.941105\du}{13.886142\du}}
\pgfpathlineto{\pgfpoint{10.983822\du}{13.886872\du}}
\pgfpathlineto{\pgfpoint{11.027633\du}{13.886872\du}}
\pgfpathlineto{\pgfpoint{11.027633\du}{13.866427\du}}
\pgfpathlineto{\pgfpoint{10.984917\du}{13.866427\du}}
\pgfpathlineto{\pgfpoint{10.941470\du}{13.865697\du}}
\pgfpathlineto{\pgfpoint{10.899119\du}{13.865332\du}}
\pgfpathlineto{\pgfpoint{10.856403\du}{13.863506\du}}
\pgfpathlineto{\pgfpoint{10.814052\du}{13.862411\du}}
\pgfpathlineto{\pgfpoint{10.772066\du}{13.859855\du}}
\pgfpathlineto{\pgfpoint{10.731175\du}{13.857665\du}}
\pgfpathlineto{\pgfpoint{10.689919\du}{13.854744\du}}
\pgfpathlineto{\pgfpoint{10.649028\du}{13.851823\du}}
\pgfpathlineto{\pgfpoint{10.608867\du}{13.848172\du}}
\pgfpathlineto{\pgfpoint{10.569072\du}{13.844886\du}}
\pgfpathlineto{\pgfpoint{10.529641\du}{13.840505\du}}
\pgfpathlineto{\pgfpoint{10.490211\du}{13.836489\du}}
\pgfpathlineto{\pgfpoint{10.451146\du}{13.831743\du}}
\pgfpathlineto{\pgfpoint{10.412810\du}{13.826631\du}}
\pgfpathlineto{\pgfpoint{10.375205\du}{13.821520\du}}
\pgfpathlineto{\pgfpoint{10.337966\du}{13.815678\du}}
\pgfpathlineto{\pgfpoint{10.301456\du}{13.809472\du}}
\pgfpathlineto{\pgfpoint{10.264216\du}{13.803630\du}}
\pgfpathlineto{\pgfpoint{10.229167\du}{13.797424\du}}
\pgfpathlineto{\pgfpoint{10.193022\du}{13.790487\du}}
\pgfpathlineto{\pgfpoint{10.158338\du}{13.783185\du}}
\pgfpathlineto{\pgfpoint{10.124019\du}{13.775883\du}}
\pgfpathlineto{\pgfpoint{10.090065\du}{13.768216\du}}
\pgfpathlineto{\pgfpoint{10.056841\du}{13.760549\du}}
\pgfpathlineto{\pgfpoint{10.025443\du}{13.752152\du}}
\pgfpathlineto{\pgfpoint{9.992949\du}{13.743389\du}}
\pgfpathlineto{\pgfpoint{9.961916\du}{13.735357\du}}
\pgfpathlineto{\pgfpoint{9.946217\du}{13.730611\du}}
\pgfpathlineto{\pgfpoint{9.930518\du}{13.725865\du}}
\pgfpathlineto{\pgfpoint{9.915914\du}{13.721849\du}}
\pgfpathlineto{\pgfpoint{9.900945\du}{13.717102\du}}
\pgfpathlineto{\pgfpoint{9.885611\du}{13.712356\du}}
\pgfpathlineto{\pgfpoint{9.871372\du}{13.707975\du}}
\pgfpathlineto{\pgfpoint{9.857133\du}{13.702498\du}}
\pgfpathlineto{\pgfpoint{9.842894\du}{13.697752\du}}
\pgfpathlineto{\pgfpoint{9.828656\du}{13.693371\du}}
\pgfpathlineto{\pgfpoint{9.814782\du}{13.687895\du}}
\pgfpathlineto{\pgfpoint{9.800908\du}{13.683148\du}}
\pgfpathlineto{\pgfpoint{9.787400\du}{13.678037\du}}
\pgfpathlineto{\pgfpoint{9.773891\du}{13.672560\du}}
\pgfpathlineto{\pgfpoint{9.761113\du}{13.668179\du}}
\pgfpathlineto{\pgfpoint{9.747604\du}{13.662338\du}}
\pgfpathlineto{\pgfpoint{9.734826\du}{13.656861\du}}
\pgfpathlineto{\pgfpoint{9.722412\du}{13.651750\du}}
\pgfpathlineto{\pgfpoint{9.709269\du}{13.646274\du}}
\pgfpathlineto{\pgfpoint{9.697221\du}{13.641162\du}}
\pgfpathlineto{\pgfpoint{9.685903\du}{13.635321\du}}
\pgfpathlineto{\pgfpoint{9.673124\du}{13.629844\du}}
\pgfpathlineto{\pgfpoint{9.661441\du}{13.624733\du}}
\pgfpathlineto{\pgfpoint{9.649758\du}{13.618891\du}}
\pgfpathlineto{\pgfpoint{9.638805\du}{13.613050\du}}
\pgfpathlineto{\pgfpoint{9.627122\du}{13.607208\du}}
\pgfpathlineto{\pgfpoint{9.617265\du}{13.601367\du}}
\pgfpathlineto{\pgfpoint{9.605947\du}{13.596255\du}}
\pgfpathlineto{\pgfpoint{9.595359\du}{13.589684\du}}
\pgfpathlineto{\pgfpoint{9.585501\du}{13.583842\du}}
\pgfpathlineto{\pgfpoint{9.574913\du}{13.578000\du}}
\pgfpathlineto{\pgfpoint{9.565056\du}{13.572159\du}}
\pgfpathlineto{\pgfpoint{9.555563\du}{13.566317\du}}
\pgfpathlineto{\pgfpoint{9.545706\du}{13.559746\du}}
\pgfpathlineto{\pgfpoint{9.536578\du}{13.553904\du}}
\pgfpathlineto{\pgfpoint{9.528181\du}{13.547332\du}}
\pgfpathlineto{\pgfpoint{9.519419\du}{13.541491\du}}
\pgfpathlineto{\pgfpoint{9.510291\du}{13.535284\du}}
\pgfpathlineto{\pgfpoint{9.501529\du}{13.528712\du}}
\pgfpathlineto{\pgfpoint{9.493132\du}{13.522506\du}}
\pgfpathlineto{\pgfpoint{9.485830\du}{13.515934\du}}
\pgfpathlineto{\pgfpoint{9.477798\du}{13.510092\du}}
\pgfpathlineto{\pgfpoint{9.470131\du}{13.503521\du}}
\pgfpathlineto{\pgfpoint{9.463194\du}{13.497314\du}}
\pgfpathlineto{\pgfpoint{9.455892\du}{13.490742\du}}
\pgfpathlineto{\pgfpoint{9.448955\du}{13.483805\du}}
\pgfpathlineto{\pgfpoint{9.442748\du}{13.477234\du}}
\pgfpathlineto{\pgfpoint{9.436177\du}{13.471027\du}}
\pgfpathlineto{\pgfpoint{9.430335\du}{13.464455\du}}
\pgfpathlineto{\pgfpoint{9.424128\du}{13.457519\du}}
\pgfpathlineto{\pgfpoint{9.419017\du}{13.450947\du}}
\pgfpathlineto{\pgfpoint{9.412810\du}{13.444375\du}}
\pgfpathlineto{\pgfpoint{9.408429\du}{13.437438\du}}
\pgfpathlineto{\pgfpoint{9.402953\du}{13.431232\du}}
\pgfpathlineto{\pgfpoint{9.398572\du}{13.423930\du}}
\pgfpathlineto{\pgfpoint{9.394190\du}{13.417723\du}}
\pgfpathlineto{\pgfpoint{9.389809\du}{13.410421\du}}
\pgfpathlineto{\pgfpoint{9.386523\du}{13.404214\du}}
\pgfpathlineto{\pgfpoint{9.382507\du}{13.397278\du}}
\pgfpathlineto{\pgfpoint{9.378491\du}{13.389976\du}}
\pgfpathlineto{\pgfpoint{9.375570\du}{13.383039\du}}
\pgfpathlineto{\pgfpoint{9.373015\du}{13.376832\du}}
\pgfpathlineto{\pgfpoint{9.369729\du}{13.369530\du}}
\pgfpathlineto{\pgfpoint{9.367538\du}{13.362593\du}}
\pgfpathlineto{\pgfpoint{9.365348\du}{13.355657\du}}
\pgfpathlineto{\pgfpoint{9.363887\du}{13.348720\du}}
\pgfpathlineto{\pgfpoint{9.361697\du}{13.341418\du}}
\pgfpathlineto{\pgfpoint{9.360602\du}{13.335211\du}}
\pgfpathlineto{\pgfpoint{9.359506\du}{13.327909\du}}
\pgfpathlineto{\pgfpoint{9.358046\du}{13.320972\du}}
\pgfpathlineto{\pgfpoint{9.357681\du}{13.314035\du}}
\pgfpathlineto{\pgfpoint{9.357681\du}{13.306368\du}}
\pgfpathlineto{\pgfpoint{9.357316\du}{13.299432\du}}
\pgfpathlineto{\pgfpoint{9.357316\du}{13.299432\du}}
\pgfpathlineto{\pgfpoint{9.357316\du}{13.299432\du}}
\pgfpathlineto{\pgfpoint{9.357316\du}{13.298336\du}}
\pgfpathlineto{\pgfpoint{9.357316\du}{13.297241\du}}
\pgfpathlineto{\pgfpoint{9.356951\du}{13.295781\du}}
\pgfpathlineto{\pgfpoint{9.356951\du}{13.295416\du}}
\pgfpathlineto{\pgfpoint{9.355855\du}{13.294320\du}}
\pgfpathlineto{\pgfpoint{9.355490\du}{13.292860\du}}
\pgfpathlineto{\pgfpoint{9.355125\du}{13.292495\du}}
\pgfpathlineto{\pgfpoint{9.354030\du}{13.291765\du}}
\pgfpathlineto{\pgfpoint{9.352569\du}{13.290669\du}}
\pgfpathlineto{\pgfpoint{9.350744\du}{13.289939\du}}
\pgfpathlineto{\pgfpoint{9.348918\du}{13.289574\du}}
\pgfpathlineto{\pgfpoint{9.347093\du}{13.289574\du}}
\pgfpathlineto{\pgfpoint{9.344902\du}{13.289574\du}}
\pgfpathlineto{\pgfpoint{9.343442\du}{13.289939\du}}
\pgfpathlineto{\pgfpoint{9.341251\du}{13.290669\du}}
\pgfpathlineto{\pgfpoint{9.339426\du}{13.291765\du}}
\pgfpathlineto{\pgfpoint{9.339061\du}{13.292495\du}}
\pgfpathlineto{\pgfpoint{9.338696\du}{13.292860\du}}
\pgfpathlineto{\pgfpoint{9.337966\du}{13.294320\du}}
\pgfpathlineto{\pgfpoint{9.337235\du}{13.295416\du}}
\pgfpathlineto{\pgfpoint{9.337235\du}{13.295781\du}}
\pgfpathlineto{\pgfpoint{9.336870\du}{13.297241\du}}
\pgfpathlineto{\pgfpoint{9.336870\du}{13.298336\du}}
\pgfpathlineto{\pgfpoint{9.336870\du}{13.299432\du}}
\pgfusepath{fill}
\pgfsetbuttcap
\pgfsetmiterjoin
\pgfsetdash{}{0pt}
\definecolor{dialinecolor}{rgb}{0.678431, 0.839216, 0.905882}
\pgfsetfillcolor{dialinecolor}
\pgfpathmoveto{\pgfpoint{11.027633\du}{12.711991\du}}
\pgfpathlineto{\pgfpoint{11.027633\du}{12.711991\du}}
\pgfpathlineto{\pgfpoint{10.983822\du}{12.711991\du}}
\pgfpathlineto{\pgfpoint{10.941105\du}{12.712721\du}}
\pgfpathlineto{\pgfpoint{10.897659\du}{12.713816\du}}
\pgfpathlineto{\pgfpoint{10.855673\du}{12.714912\du}}
\pgfpathlineto{\pgfpoint{10.813321\du}{12.716737\du}}
\pgfpathlineto{\pgfpoint{10.771700\du}{12.718928\du}}
\pgfpathlineto{\pgfpoint{10.729714\du}{12.721483\du}}
\pgfpathlineto{\pgfpoint{10.688458\du}{12.723674\du}}
\pgfpathlineto{\pgfpoint{10.647568\du}{12.726595\du}}
\pgfpathlineto{\pgfpoint{10.606677\du}{12.730246\du}}
\pgfpathlineto{\pgfpoint{10.566881\du}{12.733897\du}}
\pgfpathlineto{\pgfpoint{10.526721\du}{12.737913\du}}
\pgfpathlineto{\pgfpoint{10.487655\du}{12.742294\du}}
\pgfpathlineto{\pgfpoint{10.448955\du}{12.747040\du}}
\pgfpathlineto{\pgfpoint{10.410255\du}{12.752152\du}}
\pgfpathlineto{\pgfpoint{10.372285\du}{12.757628\du}}
\pgfpathlineto{\pgfpoint{10.334680\du}{12.762739\du}}
\pgfpathlineto{\pgfpoint{10.297440\du}{12.769311\du}}
\pgfpathlineto{\pgfpoint{10.260930\du}{12.775153\du}}
\pgfpathlineto{\pgfpoint{10.224786\du}{12.781724\du}}
\pgfpathlineto{\pgfpoint{10.189006\du}{12.788661\du}}
\pgfpathlineto{\pgfpoint{10.154322\du}{12.795598\du}}
\pgfpathlineto{\pgfpoint{10.120003\du}{12.803265\du}}
\pgfpathlineto{\pgfpoint{10.085319\du}{12.810932\du}}
\pgfpathlineto{\pgfpoint{10.052825\du}{12.818964\du}}
\pgfpathlineto{\pgfpoint{10.019601\du}{12.827362\du}}
\pgfpathlineto{\pgfpoint{9.987838\du}{12.835394\du}}
\pgfpathlineto{\pgfpoint{9.956074\du}{12.844156\du}}
\pgfpathlineto{\pgfpoint{9.925406\du}{12.853648\du}}
\pgfpathlineto{\pgfpoint{9.894373\du}{12.862411\du}}
\pgfpathlineto{\pgfpoint{9.879769\du}{12.867157\du}}
\pgfpathlineto{\pgfpoint{9.865895\du}{12.872268\du}}
\pgfpathlineto{\pgfpoint{9.850196\du}{12.877015\du}}
\pgfpathlineto{\pgfpoint{9.836323\du}{12.881761\du}}
\pgfpathlineto{\pgfpoint{9.821719\du}{12.886872\du}}
\pgfpathlineto{\pgfpoint{9.807845\du}{12.891619\du}}
\pgfpathlineto{\pgfpoint{9.793606\du}{12.896730\du}}
\pgfpathlineto{\pgfpoint{9.780463\du}{12.901476\du}}
\pgfpathlineto{\pgfpoint{9.766224\du}{12.906953\du}}
\pgfpathlineto{\pgfpoint{9.753081\du}{12.912064\du}}
\pgfpathlineto{\pgfpoint{9.740667\du}{12.917175\du}}
\pgfpathlineto{\pgfpoint{9.726794\du}{12.923017\du}}
\pgfpathlineto{\pgfpoint{9.714015\du}{12.928493\du}}
\pgfpathlineto{\pgfpoint{9.701967\du}{12.933605\du}}
\pgfpathlineto{\pgfpoint{9.688823\du}{12.939446\du}}
\pgfpathlineto{\pgfpoint{9.676775\du}{12.944923\du}}
\pgfpathlineto{\pgfpoint{9.664362\du}{12.950764\du}}
\pgfpathlineto{\pgfpoint{9.652679\du}{12.955876\du}}
\pgfpathlineto{\pgfpoint{9.641361\du}{12.961717\du}}
\pgfpathlineto{\pgfpoint{9.629678\du}{12.967559\du}}
\pgfpathlineto{\pgfpoint{9.618360\du}{12.973400\du}}
\pgfpathlineto{\pgfpoint{9.606677\du}{12.979242\du}}
\pgfpathlineto{\pgfpoint{9.595724\du}{12.985083\du}}
\pgfpathlineto{\pgfpoint{9.585501\du}{12.991655\du}}
\pgfpathlineto{\pgfpoint{9.574913\du}{12.997497\du}}
\pgfpathlineto{\pgfpoint{9.564691\du}{13.003338\du}}
\pgfpathlineto{\pgfpoint{9.554468\du}{13.009910\du}}
\pgfpathlineto{\pgfpoint{9.544610\du}{13.015751\du}}
\pgfpathlineto{\pgfpoint{9.535118\du}{13.021958\du}}
\pgfpathlineto{\pgfpoint{9.525625\du}{13.028530\du}}
\pgfpathlineto{\pgfpoint{9.516498\du}{13.035102\du}}
\pgfpathlineto{\pgfpoint{9.506640\du}{13.040943\du}}
\pgfpathlineto{\pgfpoint{9.498243\du}{13.047150\du}}
\pgfpathlineto{\pgfpoint{9.489481\du}{13.053721\du}}
\pgfpathlineto{\pgfpoint{9.481083\du}{13.059928\du}}
\pgfpathlineto{\pgfpoint{9.472321\du}{13.067230\du}}
\pgfpathlineto{\pgfpoint{9.464654\du}{13.073437\du}}
\pgfpathlineto{\pgfpoint{9.457717\du}{13.080008\du}}
\pgfpathlineto{\pgfpoint{9.448955\du}{13.086945\du}}
\pgfpathlineto{\pgfpoint{9.442018\du}{13.093517\du}}
\pgfpathlineto{\pgfpoint{9.435081\du}{13.100454\du}}
\pgfpathlineto{\pgfpoint{9.428144\du}{13.107391\du}}
\pgfpathlineto{\pgfpoint{9.421573\du}{13.113962\du}}
\pgfpathlineto{\pgfpoint{9.414636\du}{13.120899\du}}
\pgfpathlineto{\pgfpoint{9.408794\du}{13.127836\du}}
\pgfpathlineto{\pgfpoint{9.402953\du}{13.135138\du}}
\pgfpathlineto{\pgfpoint{9.397111\du}{13.142075\du}}
\pgfpathlineto{\pgfpoint{9.391635\du}{13.149012\du}}
\pgfpathlineto{\pgfpoint{9.386888\du}{13.155949\du}}
\pgfpathlineto{\pgfpoint{9.381412\du}{13.163616\du}}
\pgfpathlineto{\pgfpoint{9.377396\du}{13.170552\du}}
\pgfpathlineto{\pgfpoint{9.372285\du}{13.177854\du}}
\pgfpathlineto{\pgfpoint{9.368634\du}{13.185156\du}}
\pgfpathlineto{\pgfpoint{9.364252\du}{13.192458\du}}
\pgfpathlineto{\pgfpoint{9.360602\du}{13.200125\du}}
\pgfpathlineto{\pgfpoint{9.357316\du}{13.207427\du}}
\pgfpathlineto{\pgfpoint{9.353665\du}{13.215094\du}}
\pgfpathlineto{\pgfpoint{9.350744\du}{13.222761\du}}
\pgfpathlineto{\pgfpoint{9.348188\du}{13.229698\du}}
\pgfpathlineto{\pgfpoint{9.345998\du}{13.237365\du}}
\pgfpathlineto{\pgfpoint{9.343807\du}{13.245032\du}}
\pgfpathlineto{\pgfpoint{9.341982\du}{13.253064\du}}
\pgfpathlineto{\pgfpoint{9.340156\du}{13.260731\du}}
\pgfpathlineto{\pgfpoint{9.339061\du}{13.268398\du}}
\pgfpathlineto{\pgfpoint{9.337966\du}{13.276065\du}}
\pgfpathlineto{\pgfpoint{9.337235\du}{13.284098\du}}
\pgfpathlineto{\pgfpoint{9.336870\du}{13.291765\du}}
\pgfpathlineto{\pgfpoint{9.336870\du}{13.299432\du}}
\pgfpathlineto{\pgfpoint{9.357316\du}{13.299432\du}}
\pgfpathlineto{\pgfpoint{9.357681\du}{13.292495\du}}
\pgfpathlineto{\pgfpoint{9.357681\du}{13.285558\du}}
\pgfpathlineto{\pgfpoint{9.358046\du}{13.278256\du}}
\pgfpathlineto{\pgfpoint{9.359506\du}{13.271319\du}}
\pgfpathlineto{\pgfpoint{9.360602\du}{13.264382\du}}
\pgfpathlineto{\pgfpoint{9.361697\du}{13.257445\du}}
\pgfpathlineto{\pgfpoint{9.363887\du}{13.250144\du}}
\pgfpathlineto{\pgfpoint{9.365348\du}{13.243937\du}}
\pgfpathlineto{\pgfpoint{9.367538\du}{13.236635\du}}
\pgfpathlineto{\pgfpoint{9.369729\du}{13.229698\du}}
\pgfpathlineto{\pgfpoint{9.373015\du}{13.222761\du}}
\pgfpathlineto{\pgfpoint{9.375570\du}{13.215824\du}}
\pgfpathlineto{\pgfpoint{9.378491\du}{13.208888\du}}
\pgfpathlineto{\pgfpoint{9.382507\du}{13.202316\du}}
\pgfpathlineto{\pgfpoint{9.386523\du}{13.195379\du}}
\pgfpathlineto{\pgfpoint{9.389809\du}{13.188807\du}}
\pgfpathlineto{\pgfpoint{9.393825\du}{13.181870\du}}
\pgfpathlineto{\pgfpoint{9.398572\du}{13.174934\du}}
\pgfpathlineto{\pgfpoint{9.402953\du}{13.168362\du}}
\pgfpathlineto{\pgfpoint{9.408429\du}{13.161790\du}}
\pgfpathlineto{\pgfpoint{9.412810\du}{13.154853\du}}
\pgfpathlineto{\pgfpoint{9.419017\du}{13.148282\du}}
\pgfpathlineto{\pgfpoint{9.424128\du}{13.141345\du}}
\pgfpathlineto{\pgfpoint{9.430335\du}{13.135138\du}}
\pgfpathlineto{\pgfpoint{9.436177\du}{13.128566\du}}
\pgfpathlineto{\pgfpoint{9.442748\du}{13.121629\du}}
\pgfpathlineto{\pgfpoint{9.448955\du}{13.115058\du}}
\pgfpathlineto{\pgfpoint{9.455892\du}{13.108851\du}}
\pgfpathlineto{\pgfpoint{9.463194\du}{13.102279\du}}
\pgfpathlineto{\pgfpoint{9.470131\du}{13.095708\du}}
\pgfpathlineto{\pgfpoint{9.477798\du}{13.089501\du}}
\pgfpathlineto{\pgfpoint{9.485830\du}{13.082929\du}}
\pgfpathlineto{\pgfpoint{9.493132\du}{13.076357\du}}
\pgfpathlineto{\pgfpoint{9.501529\du}{13.070151\du}}
\pgfpathlineto{\pgfpoint{9.510291\du}{13.064309\du}}
\pgfpathlineto{\pgfpoint{9.519419\du}{13.057372\du}}
\pgfpathlineto{\pgfpoint{9.528181\du}{13.051896\du}}
\pgfpathlineto{\pgfpoint{9.536578\du}{13.045324\du}}
\pgfpathlineto{\pgfpoint{9.545706\du}{13.039118\du}}
\pgfpathlineto{\pgfpoint{9.555563\du}{13.033276\du}}
\pgfpathlineto{\pgfpoint{9.565056\du}{13.027435\du}}
\pgfpathlineto{\pgfpoint{9.574913\du}{13.020863\du}}
\pgfpathlineto{\pgfpoint{9.585501\du}{13.015021\du}}
\pgfpathlineto{\pgfpoint{9.595359\du}{13.009180\du}}
\pgfpathlineto{\pgfpoint{9.605947\du}{13.003338\du}}
\pgfpathlineto{\pgfpoint{9.617265\du}{12.997497\du}}
\pgfpathlineto{\pgfpoint{9.627122\du}{12.991655\du}}
\pgfpathlineto{\pgfpoint{9.638805\du}{12.985814\du}}
\pgfpathlineto{\pgfpoint{9.649758\du}{12.980702\du}}
\pgfpathlineto{\pgfpoint{9.661441\du}{12.974496\du}}
\pgfpathlineto{\pgfpoint{9.673124\du}{12.968654\du}}
\pgfpathlineto{\pgfpoint{9.685903\du}{12.963543\du}}
\pgfpathlineto{\pgfpoint{9.697221\du}{12.958431\du}}
\pgfpathlineto{\pgfpoint{9.709269\du}{12.952590\du}}
\pgfpathlineto{\pgfpoint{9.722412\du}{12.947113\du}}
\pgfpathlineto{\pgfpoint{9.734826\du}{12.942002\du}}
\pgfpathlineto{\pgfpoint{9.747604\du}{12.936525\du}}
\pgfpathlineto{\pgfpoint{9.761113\du}{12.931414\du}}
\pgfpathlineto{\pgfpoint{9.773891\du}{12.925938\du}}
\pgfpathlineto{\pgfpoint{9.787400\du}{12.920461\du}}
\pgfpathlineto{\pgfpoint{9.800908\du}{12.916080\du}}
\pgfpathlineto{\pgfpoint{9.814782\du}{12.910969\du}}
\pgfpathlineto{\pgfpoint{9.828656\du}{12.906222\du}}
\pgfpathlineto{\pgfpoint{9.842894\du}{12.901111\du}}
\pgfpathlineto{\pgfpoint{9.857133\du}{12.896365\du}}
\pgfpathlineto{\pgfpoint{9.871372\du}{12.891619\du}}
\pgfpathlineto{\pgfpoint{9.885611\du}{12.886872\du}}
\pgfpathlineto{\pgfpoint{9.900945\du}{12.882126\du}}
\pgfpathlineto{\pgfpoint{9.930518\du}{12.872999\du}}
\pgfpathlineto{\pgfpoint{9.961916\du}{12.864236\du}}
\pgfpathlineto{\pgfpoint{9.992949\du}{12.855474\du}}
\pgfpathlineto{\pgfpoint{10.025443\du}{12.847077\du}}
\pgfpathlineto{\pgfpoint{10.056841\du}{12.839045\du}}
\pgfpathlineto{\pgfpoint{10.090065\du}{12.830647\du}}
\pgfpathlineto{\pgfpoint{10.124019\du}{12.822980\du}}
\pgfpathlineto{\pgfpoint{10.158338\du}{12.816044\du}}
\pgfpathlineto{\pgfpoint{10.193022\du}{12.809107\du}}
\pgfpathlineto{\pgfpoint{10.229167\du}{12.801805\du}}
\pgfpathlineto{\pgfpoint{10.264216\du}{12.795598\du}}
\pgfpathlineto{\pgfpoint{10.301456\du}{12.788661\du}}
\pgfpathlineto{\pgfpoint{10.337966\du}{12.783185\du}}
\pgfpathlineto{\pgfpoint{10.375205\du}{12.777343\du}}
\pgfpathlineto{\pgfpoint{10.412810\du}{12.772232\du}}
\pgfpathlineto{\pgfpoint{10.451146\du}{12.767486\du}}
\pgfpathlineto{\pgfpoint{10.490211\du}{12.762739\du}}
\pgfpathlineto{\pgfpoint{10.529641\du}{12.758358\du}}
\pgfpathlineto{\pgfpoint{10.569072\du}{12.754707\du}}
\pgfpathlineto{\pgfpoint{10.608867\du}{12.750691\du}}
\pgfpathlineto{\pgfpoint{10.649028\du}{12.747770\du}}
\pgfpathlineto{\pgfpoint{10.689919\du}{12.744119\du}}
\pgfpathlineto{\pgfpoint{10.731175\du}{12.741199\du}}
\pgfpathlineto{\pgfpoint{10.772066\du}{12.739008\du}}
\pgfpathlineto{\pgfpoint{10.814052\du}{12.737183\du}}
\pgfpathlineto{\pgfpoint{10.856403\du}{12.735357\du}}
\pgfpathlineto{\pgfpoint{10.899119\du}{12.733897\du}}
\pgfpathlineto{\pgfpoint{10.941470\du}{12.733532\du}}
\pgfpathlineto{\pgfpoint{10.984917\du}{12.733167\du}}
\pgfpathlineto{\pgfpoint{11.027633\du}{12.732436\du}}
\pgfpathlineto{\pgfpoint{11.027633\du}{12.732436\du}}
\pgfpathlineto{\pgfpoint{11.027633\du}{12.732436\du}}
\pgfpathlineto{\pgfpoint{11.028729\du}{12.732436\du}}
\pgfpathlineto{\pgfpoint{11.029824\du}{12.732436\du}}
\pgfpathlineto{\pgfpoint{11.031284\du}{12.731706\du}}
\pgfpathlineto{\pgfpoint{11.032380\du}{12.731706\du}}
\pgfpathlineto{\pgfpoint{11.032745\du}{12.731341\du}}
\pgfpathlineto{\pgfpoint{11.033840\du}{12.730611\du}}
\pgfpathlineto{\pgfpoint{11.034570\du}{12.730246\du}}
\pgfpathlineto{\pgfpoint{11.035665\du}{12.729516\du}}
\pgfpathlineto{\pgfpoint{11.036761\du}{12.727690\du}}
\pgfpathlineto{\pgfpoint{11.037491\du}{12.725865\du}}
\pgfpathlineto{\pgfpoint{11.037491\du}{12.724404\du}}
\pgfpathlineto{\pgfpoint{11.038221\du}{12.722579\du}}
\pgfpathlineto{\pgfpoint{11.037491\du}{12.720753\du}}
\pgfpathlineto{\pgfpoint{11.037491\du}{12.718563\du}}
\pgfpathlineto{\pgfpoint{11.036761\du}{12.716737\du}}
\pgfpathlineto{\pgfpoint{11.035665\du}{12.714912\du}}
\pgfpathlineto{\pgfpoint{11.034570\du}{12.714182\du}}
\pgfpathlineto{\pgfpoint{11.033840\du}{12.713816\du}}
\pgfpathlineto{\pgfpoint{11.032745\du}{12.713086\du}}
\pgfpathlineto{\pgfpoint{11.032380\du}{12.712721\du}}
\pgfpathlineto{\pgfpoint{11.031284\du}{12.712721\du}}
\pgfpathlineto{\pgfpoint{11.029824\du}{12.711991\du}}
\pgfpathlineto{\pgfpoint{11.028729\du}{12.711991\du}}
\pgfpathlineto{\pgfpoint{11.027633\du}{12.711991\du}}
\pgfusepath{fill}
\pgfsetbuttcap
\pgfsetmiterjoin
\pgfsetdash{}{0pt}
\definecolor{dialinecolor}{rgb}{0.678431, 0.839216, 0.905882}
\pgfsetfillcolor{dialinecolor}
\pgfpathmoveto{\pgfpoint{12.718031\du}{13.299432\du}}
\pgfpathlineto{\pgfpoint{12.718031\du}{13.291765\du}}
\pgfpathlineto{\pgfpoint{12.717301\du}{13.284098\du}}
\pgfpathlineto{\pgfpoint{12.716571\du}{13.276065\du}}
\pgfpathlineto{\pgfpoint{12.715841\du}{13.268398\du}}
\pgfpathlineto{\pgfpoint{12.714745\du}{13.260731\du}}
\pgfpathlineto{\pgfpoint{12.712555\du}{13.253064\du}}
\pgfpathlineto{\pgfpoint{12.711459\du}{13.245032\du}}
\pgfpathlineto{\pgfpoint{12.708539\du}{13.237365\du}}
\pgfpathlineto{\pgfpoint{12.706713\du}{13.229698\du}}
\pgfpathlineto{\pgfpoint{12.703792\du}{13.222761\du}}
\pgfpathlineto{\pgfpoint{12.701237\du}{13.215094\du}}
\pgfpathlineto{\pgfpoint{12.697951\du}{13.207427\du}}
\pgfpathlineto{\pgfpoint{12.693935\du}{13.200125\du}}
\pgfpathlineto{\pgfpoint{12.690649\du}{13.192458\du}}
\pgfpathlineto{\pgfpoint{12.686633\du}{13.185156\du}}
\pgfpathlineto{\pgfpoint{12.682617\du}{13.177854\du}}
\pgfpathlineto{\pgfpoint{12.678236\du}{13.170552\du}}
\pgfpathlineto{\pgfpoint{12.673489\du}{13.163616\du}}
\pgfpathlineto{\pgfpoint{12.668013\du}{13.155949\du}}
\pgfpathlineto{\pgfpoint{12.663267\du}{13.149012\du}}
\pgfpathlineto{\pgfpoint{12.657425\du}{13.142075\du}}
\pgfpathlineto{\pgfpoint{12.651949\du}{13.135138\du}}
\pgfpathlineto{\pgfpoint{12.646107\du}{13.127836\du}}
\pgfpathlineto{\pgfpoint{12.639901\du}{13.120899\du}}
\pgfpathlineto{\pgfpoint{12.633329\du}{13.113962\du}}
\pgfpathlineto{\pgfpoint{12.626392\du}{13.107391\du}}
\pgfpathlineto{\pgfpoint{12.619820\du}{13.100454\du}}
\pgfpathlineto{\pgfpoint{12.612518\du}{13.093517\du}}
\pgfpathlineto{\pgfpoint{12.605947\du}{13.086945\du}}
\pgfpathlineto{\pgfpoint{12.597914\du}{13.080008\du}}
\pgfpathlineto{\pgfpoint{12.589882\du}{13.073437\du}}
\pgfpathlineto{\pgfpoint{12.582580\du}{13.067230\du}}
\pgfpathlineto{\pgfpoint{12.573818\du}{13.059928\du}}
\pgfpathlineto{\pgfpoint{12.565786\du}{13.053721\du}}
\pgfpathlineto{\pgfpoint{12.556658\du}{13.047150\du}}
\pgfpathlineto{\pgfpoint{12.547896\du}{13.040943\du}}
\pgfpathlineto{\pgfpoint{12.538404\du}{13.035102\du}}
\pgfpathlineto{\pgfpoint{12.529276\du}{13.028530\du}}
\pgfpathlineto{\pgfpoint{12.520149\du}{13.021958\du}}
\pgfpathlineto{\pgfpoint{12.510291\du}{13.015751\du}}
\pgfpathlineto{\pgfpoint{12.500434\du}{13.009910\du}}
\pgfpathlineto{\pgfpoint{12.490211\du}{13.003338\du}}
\pgfpathlineto{\pgfpoint{12.479623\du}{12.997497\du}}
\pgfpathlineto{\pgfpoint{12.469765\du}{12.991655\du}}
\pgfpathlineto{\pgfpoint{12.459178\du}{12.985083\du}}
\pgfpathlineto{\pgfpoint{12.447860\du}{12.979242\du}}
\pgfpathlineto{\pgfpoint{12.436542\du}{12.973400\du}}
\pgfpathlineto{\pgfpoint{12.424859\du}{12.967559\du}}
\pgfpathlineto{\pgfpoint{12.413906\du}{12.961717\du}}
\pgfpathlineto{\pgfpoint{12.401857\du}{12.955876\du}}
\pgfpathlineto{\pgfpoint{12.390539\du}{12.950764\du}}
\pgfpathlineto{\pgfpoint{12.378126\du}{12.944923\du}}
\pgfpathlineto{\pgfpoint{12.365713\du}{12.939446\du}}
\pgfpathlineto{\pgfpoint{12.352934\du}{12.933605\du}}
\pgfpathlineto{\pgfpoint{12.340886\du}{12.928493\du}}
\pgfpathlineto{\pgfpoint{12.328108\du}{12.923017\du}}
\pgfpathlineto{\pgfpoint{12.314599\du}{12.917175\du}}
\pgfpathlineto{\pgfpoint{12.301456\du}{12.912064\du}}
\pgfpathlineto{\pgfpoint{12.288312\du}{12.906953\du}}
\pgfpathlineto{\pgfpoint{12.274074\du}{12.901476\du}}
\pgfpathlineto{\pgfpoint{12.260930\du}{12.896730\du}}
\pgfpathlineto{\pgfpoint{12.246691\du}{12.891619\du}}
\pgfpathlineto{\pgfpoint{12.232818\du}{12.886872\du}}
\pgfpathlineto{\pgfpoint{12.218579\du}{12.881761\du}}
\pgfpathlineto{\pgfpoint{12.204705\du}{12.877015\du}}
\pgfpathlineto{\pgfpoint{12.189736\du}{12.872268\du}}
\pgfpathlineto{\pgfpoint{12.174767\du}{12.867157\du}}
\pgfpathlineto{\pgfpoint{12.160528\du}{12.862411\du}}
\pgfpathlineto{\pgfpoint{12.129860\du}{12.853648\du}}
\pgfpathlineto{\pgfpoint{12.099192\du}{12.844156\du}}
\pgfpathlineto{\pgfpoint{12.067794\du}{12.835394\du}}
\pgfpathlineto{\pgfpoint{12.036031\du}{12.827362\du}}
\pgfpathlineto{\pgfpoint{12.003172\du}{12.818964\du}}
\pgfpathlineto{\pgfpoint{11.969583\du}{12.810932\du}}
\pgfpathlineto{\pgfpoint{11.935629\du}{12.803265\du}}
\pgfpathlineto{\pgfpoint{11.900945\du}{12.795598\du}}
\pgfpathlineto{\pgfpoint{11.866261\du}{12.788661\du}}
\pgfpathlineto{\pgfpoint{11.830846\du}{12.781724\du}}
\pgfpathlineto{\pgfpoint{11.794336\du}{12.775153\du}}
\pgfpathlineto{\pgfpoint{11.757827\du}{12.769311\du}}
\pgfpathlineto{\pgfpoint{11.720952\du}{12.762739\du}}
\pgfpathlineto{\pgfpoint{11.683712\du}{12.757628\du}}
\pgfpathlineto{\pgfpoint{11.644647\du}{12.752152\du}}
\pgfpathlineto{\pgfpoint{11.606677\du}{12.747040\du}}
\pgfpathlineto{\pgfpoint{11.567246\du}{12.742294\du}}
\pgfpathlineto{\pgfpoint{11.528546\du}{12.737913\du}}
\pgfpathlineto{\pgfpoint{11.488385\du}{12.733897\du}}
\pgfpathlineto{\pgfpoint{11.448590\du}{12.730246\du}}
\pgfpathlineto{\pgfpoint{11.407699\du}{12.726595\du}}
\pgfpathlineto{\pgfpoint{11.367173\du}{12.723674\du}}
\pgfpathlineto{\pgfpoint{11.325552\du}{12.721483\du}}
\pgfpathlineto{\pgfpoint{11.283931\du}{12.718928\du}}
\pgfpathlineto{\pgfpoint{11.242310\du}{12.716737\du}}
\pgfpathlineto{\pgfpoint{11.199594\du}{12.714912\du}}
\pgfpathlineto{\pgfpoint{11.157608\du}{12.713816\du}}
\pgfpathlineto{\pgfpoint{11.114526\du}{12.712721\du}}
\pgfpathlineto{\pgfpoint{11.071080\du}{12.711991\du}}
\pgfpathlineto{\pgfpoint{11.027633\du}{12.711991\du}}
\pgfpathlineto{\pgfpoint{11.027633\du}{12.732436\du}}
\pgfpathlineto{\pgfpoint{11.070715\du}{12.733167\du}}
\pgfpathlineto{\pgfpoint{11.114161\du}{12.733532\du}}
\pgfpathlineto{\pgfpoint{11.156147\du}{12.733897\du}}
\pgfpathlineto{\pgfpoint{11.198864\du}{12.735357\du}}
\pgfpathlineto{\pgfpoint{11.241580\du}{12.737183\du}}
\pgfpathlineto{\pgfpoint{11.283201\du}{12.739008\du}}
\pgfpathlineto{\pgfpoint{11.324457\du}{12.741199\du}}
\pgfpathlineto{\pgfpoint{11.365348\du}{12.744119\du}}
\pgfpathlineto{\pgfpoint{11.406604\du}{12.747770\du}}
\pgfpathlineto{\pgfpoint{11.446764\du}{12.750691\du}}
\pgfpathlineto{\pgfpoint{11.486925\du}{12.754707\du}}
\pgfpathlineto{\pgfpoint{11.525990\du}{12.758358\du}}
\pgfpathlineto{\pgfpoint{11.565421\du}{12.762739\du}}
\pgfpathlineto{\pgfpoint{11.604121\du}{12.767486\du}}
\pgfpathlineto{\pgfpoint{11.642821\du}{12.772232\du}}
\pgfpathlineto{\pgfpoint{11.680426\du}{12.777343\du}}
\pgfpathlineto{\pgfpoint{11.717666\du}{12.783185\du}}
\pgfpathlineto{\pgfpoint{11.754176\du}{12.788661\du}}
\pgfpathlineto{\pgfpoint{11.791051\du}{12.795598\du}}
\pgfpathlineto{\pgfpoint{11.826465\du}{12.801805\du}}
\pgfpathlineto{\pgfpoint{11.862244\du}{12.809107\du}}
\pgfpathlineto{\pgfpoint{11.896929\du}{12.816044\du}}
\pgfpathlineto{\pgfpoint{11.930883\du}{12.822980\du}}
\pgfpathlineto{\pgfpoint{11.965202\du}{12.830647\du}}
\pgfpathlineto{\pgfpoint{11.998426\du}{12.839045\du}}
\pgfpathlineto{\pgfpoint{12.030554\du}{12.847077\du}}
\pgfpathlineto{\pgfpoint{12.062683\du}{12.855474\du}}
\pgfpathlineto{\pgfpoint{12.094081\du}{12.864236\du}}
\pgfpathlineto{\pgfpoint{12.124749\du}{12.872999\du}}
\pgfpathlineto{\pgfpoint{12.154687\du}{12.882126\du}}
\pgfpathlineto{\pgfpoint{12.168926\du}{12.886872\du}}
\pgfpathlineto{\pgfpoint{12.183530\du}{12.891619\du}}
\pgfpathlineto{\pgfpoint{12.197403\du}{12.896365\du}}
\pgfpathlineto{\pgfpoint{12.212007\du}{12.901111\du}}
\pgfpathlineto{\pgfpoint{12.226246\du}{12.906222\du}}
\pgfpathlineto{\pgfpoint{12.240120\du}{12.910969\du}}
\pgfpathlineto{\pgfpoint{12.254358\du}{12.916080\du}}
\pgfpathlineto{\pgfpoint{12.267502\du}{12.920461\du}}
\pgfpathlineto{\pgfpoint{12.281010\du}{12.925938\du}}
\pgfpathlineto{\pgfpoint{12.294154\du}{12.931414\du}}
\pgfpathlineto{\pgfpoint{12.306932\du}{12.936525\du}}
\pgfpathlineto{\pgfpoint{12.319711\du}{12.942002\du}}
\pgfpathlineto{\pgfpoint{12.332854\du}{12.947113\du}}
\pgfpathlineto{\pgfpoint{12.344902\du}{12.952590\du}}
\pgfpathlineto{\pgfpoint{12.357316\du}{12.958431\du}}
\pgfpathlineto{\pgfpoint{12.369364\du}{12.963543\du}}
\pgfpathlineto{\pgfpoint{12.381777\du}{12.968654\du}}
\pgfpathlineto{\pgfpoint{12.393095\du}{12.974496\du}}
\pgfpathlineto{\pgfpoint{12.405143\du}{12.980702\du}}
\pgfpathlineto{\pgfpoint{12.415731\du}{12.985814\du}}
\pgfpathlineto{\pgfpoint{12.427779\du}{12.991655\du}}
\pgfpathlineto{\pgfpoint{12.438002\du}{12.997497\du}}
\pgfpathlineto{\pgfpoint{12.448955\du}{13.003338\du}}
\pgfpathlineto{\pgfpoint{12.459908\du}{13.009180\du}}
\pgfpathlineto{\pgfpoint{12.469765\du}{13.015021\du}}
\pgfpathlineto{\pgfpoint{12.479623\du}{13.020863\du}}
\pgfpathlineto{\pgfpoint{12.489846\du}{13.027435\du}}
\pgfpathlineto{\pgfpoint{12.498973\du}{13.033276\du}}
\pgfpathlineto{\pgfpoint{12.509196\du}{13.039118\du}}
\pgfpathlineto{\pgfpoint{12.518688\du}{13.045324\du}}
\pgfpathlineto{\pgfpoint{12.527451\du}{13.051896\du}}
\pgfpathlineto{\pgfpoint{12.535483\du}{13.057372\du}}
\pgfpathlineto{\pgfpoint{12.544245\du}{13.064309\du}}
\pgfpathlineto{\pgfpoint{12.552642\du}{13.070151\du}}
\pgfpathlineto{\pgfpoint{12.561405\du}{13.076357\du}}
\pgfpathlineto{\pgfpoint{12.569072\du}{13.082929\du}}
\pgfpathlineto{\pgfpoint{12.577104\du}{13.089501\du}}
\pgfpathlineto{\pgfpoint{12.584406\du}{13.095708\du}}
\pgfpathlineto{\pgfpoint{12.592073\du}{13.102279\du}}
\pgfpathlineto{\pgfpoint{12.598645\du}{13.108851\du}}
\pgfpathlineto{\pgfpoint{12.605947\du}{13.115058\du}}
\pgfpathlineto{\pgfpoint{12.611788\du}{13.121629\du}}
\pgfpathlineto{\pgfpoint{12.618725\du}{13.128566\du}}
\pgfpathlineto{\pgfpoint{12.625297\du}{13.135138\du}}
\pgfpathlineto{\pgfpoint{12.630408\du}{13.141345\du}}
\pgfpathlineto{\pgfpoint{12.635884\du}{13.148282\du}}
\pgfpathlineto{\pgfpoint{12.642091\du}{13.154853\du}}
\pgfpathlineto{\pgfpoint{12.646837\du}{13.161790\du}}
\pgfpathlineto{\pgfpoint{12.651949\du}{13.168362\du}}
\pgfpathlineto{\pgfpoint{12.656695\du}{13.174934\du}}
\pgfpathlineto{\pgfpoint{12.661076\du}{13.181870\du}}
\pgfpathlineto{\pgfpoint{12.665457\du}{13.188807\du}}
\pgfpathlineto{\pgfpoint{12.668378\du}{13.195379\du}}
\pgfpathlineto{\pgfpoint{12.672394\du}{13.202316\du}}
\pgfpathlineto{\pgfpoint{12.675315\du}{13.208888\du}}
\pgfpathlineto{\pgfpoint{12.679331\du}{13.215824\du}}
\pgfpathlineto{\pgfpoint{12.681887\du}{13.222761\du}}
\pgfpathlineto{\pgfpoint{12.684807\du}{13.229698\du}}
\pgfpathlineto{\pgfpoint{12.687363\du}{13.236635\du}}
\pgfpathlineto{\pgfpoint{12.689554\du}{13.243937\du}}
\pgfpathlineto{\pgfpoint{12.691014\du}{13.250144\du}}
\pgfpathlineto{\pgfpoint{12.693205\du}{13.257445\du}}
\pgfpathlineto{\pgfpoint{12.693935\du}{13.264382\du}}
\pgfpathlineto{\pgfpoint{12.695030\du}{13.271319\du}}
\pgfpathlineto{\pgfpoint{12.696856\du}{13.278256\du}}
\pgfpathlineto{\pgfpoint{12.697221\du}{13.285558\du}}
\pgfpathlineto{\pgfpoint{12.697221\du}{13.292495\du}}
\pgfpathlineto{\pgfpoint{12.697951\du}{13.299432\du}}
\pgfpathlineto{\pgfpoint{12.718031\du}{13.299432\du}}
\pgfusepath{fill}
\pgfsetbuttcap
\pgfsetmiterjoin
\pgfsetdash{}{0pt}
\definecolor{dialinecolor}{rgb}{0.074510, 0.082353, 0.086275}
\pgfsetfillcolor{dialinecolor}
\pgfpathmoveto{\pgfpoint{11.070715\du}{13.172013\du}}
\pgfpathlineto{\pgfpoint{11.318615\du}{13.254525\du}}
\pgfpathlineto{\pgfpoint{11.903865\du}{13.020133\du}}
\pgfpathlineto{\pgfpoint{12.176593\du}{13.087675\du}}
\pgfpathlineto{\pgfpoint{12.032745\du}{12.879205\du}}
\pgfpathlineto{\pgfpoint{11.328473\du}{12.879205\du}}
\pgfpathlineto{\pgfpoint{11.622741\du}{12.951860\du}}
\pgfpathlineto{\pgfpoint{11.070715\du}{13.172013\du}}
\pgfusepath{fill}
\pgfsetbuttcap
\pgfsetmiterjoin
\pgfsetdash{}{0pt}
\definecolor{dialinecolor}{rgb}{0.074510, 0.082353, 0.086275}
\pgfsetfillcolor{dialinecolor}
\pgfpathmoveto{\pgfpoint{10.968853\du}{13.410056\du}}
\pgfpathlineto{\pgfpoint{10.720952\du}{13.327909\du}}
\pgfpathlineto{\pgfpoint{10.135702\du}{13.561571\du}}
\pgfpathlineto{\pgfpoint{9.862610\du}{13.494758\du}}
\pgfpathlineto{\pgfpoint{10.006458\du}{13.702498\du}}
\pgfpathlineto{\pgfpoint{10.711825\du}{13.702498\du}}
\pgfpathlineto{\pgfpoint{10.416826\du}{13.630574\du}}
\pgfpathlineto{\pgfpoint{10.968853\du}{13.410056\du}}
\pgfusepath{fill}
\pgfsetbuttcap
\pgfsetmiterjoin
\pgfsetdash{}{0pt}
\definecolor{dialinecolor}{rgb}{0.074510, 0.082353, 0.086275}
\pgfsetfillcolor{dialinecolor}
\pgfpathmoveto{\pgfpoint{9.922851\du}{12.951129\du}}
\pgfpathlineto{\pgfpoint{10.170386\du}{12.869348\du}}
\pgfpathlineto{\pgfpoint{10.755636\du}{13.102644\du}}
\pgfpathlineto{\pgfpoint{11.028729\du}{13.036197\du}}
\pgfpathlineto{\pgfpoint{10.884880\du}{13.243937\du}}
\pgfpathlineto{\pgfpoint{10.179879\du}{13.243937\du}}
\pgfpathlineto{\pgfpoint{10.474877\du}{13.172013\du}}
\pgfpathlineto{\pgfpoint{9.922851\du}{12.951129\du}}
\pgfusepath{fill}
\pgfsetbuttcap
\pgfsetmiterjoin
\pgfsetdash{}{0pt}
\definecolor{dialinecolor}{rgb}{0.074510, 0.082353, 0.086275}
\pgfsetfillcolor{dialinecolor}
\pgfpathmoveto{\pgfpoint{12.141178\du}{13.646274\du}}
\pgfpathlineto{\pgfpoint{11.893643\du}{13.728420\du}}
\pgfpathlineto{\pgfpoint{11.308393\du}{13.494758\du}}
\pgfpathlineto{\pgfpoint{11.034570\du}{13.561571\du}}
\pgfpathlineto{\pgfpoint{11.179148\du}{13.353831\du}}
\pgfpathlineto{\pgfpoint{11.884515\du}{13.353831\du}}
\pgfpathlineto{\pgfpoint{11.589152\du}{13.425755\du}}
\pgfpathlineto{\pgfpoint{12.141178\du}{13.646274\du}}
\pgfusepath{fill}
\pgfsetbuttcap
\pgfsetmiterjoin
\pgfsetdash{}{0pt}
\definecolor{dialinecolor}{rgb}{1.000000, 1.000000, 1.000000}
\pgfsetfillcolor{dialinecolor}
\pgfpathmoveto{\pgfpoint{11.091525\du}{13.192458\du}}
\pgfpathlineto{\pgfpoint{11.339061\du}{13.274970\du}}
\pgfpathlineto{\pgfpoint{11.924311\du}{13.040943\du}}
\pgfpathlineto{\pgfpoint{12.196673\du}{13.108121\du}}
\pgfpathlineto{\pgfpoint{12.053920\du}{12.899651\du}}
\pgfpathlineto{\pgfpoint{11.348553\du}{12.899651\du}}
\pgfpathlineto{\pgfpoint{11.643551\du}{12.972305\du}}
\pgfpathlineto{\pgfpoint{11.091525\du}{13.192458\du}}
\pgfusepath{fill}
\pgfsetbuttcap
\pgfsetmiterjoin
\pgfsetdash{}{0pt}
\definecolor{dialinecolor}{rgb}{1.000000, 1.000000, 1.000000}
\pgfsetfillcolor{dialinecolor}
\pgfpathmoveto{\pgfpoint{10.989663\du}{13.431232\du}}
\pgfpathlineto{\pgfpoint{10.741032\du}{13.348720\du}}
\pgfpathlineto{\pgfpoint{10.156147\du}{13.582747\du}}
\pgfpathlineto{\pgfpoint{9.882690\du}{13.515204\du}}
\pgfpathlineto{\pgfpoint{10.027633\du}{13.723674\du}}
\pgfpathlineto{\pgfpoint{10.732270\du}{13.723674\du}}
\pgfpathlineto{\pgfpoint{10.437637\du}{13.651020\du}}
\pgfpathlineto{\pgfpoint{10.989663\du}{13.431232\du}}
\pgfusepath{fill}
\pgfsetbuttcap
\pgfsetmiterjoin
\pgfsetdash{}{0pt}
\definecolor{dialinecolor}{rgb}{1.000000, 1.000000, 1.000000}
\pgfsetfillcolor{dialinecolor}
\pgfpathmoveto{\pgfpoint{9.943296\du}{12.971575\du}}
\pgfpathlineto{\pgfpoint{10.190832\du}{12.889793\du}}
\pgfpathlineto{\pgfpoint{10.776447\du}{13.123820\du}}
\pgfpathlineto{\pgfpoint{11.049539\du}{13.057007\du}}
\pgfpathlineto{\pgfpoint{10.904596\du}{13.264382\du}}
\pgfpathlineto{\pgfpoint{10.200324\du}{13.264382\du}}
\pgfpathlineto{\pgfpoint{10.494957\du}{13.192458\du}}
\pgfpathlineto{\pgfpoint{9.943296\du}{12.971575\du}}
\pgfusepath{fill}
\pgfsetbuttcap
\pgfsetmiterjoin
\pgfsetdash{}{0pt}
\definecolor{dialinecolor}{rgb}{1.000000, 1.000000, 1.000000}
\pgfsetfillcolor{dialinecolor}
\pgfpathmoveto{\pgfpoint{12.161259\du}{13.666719\du}}
\pgfpathlineto{\pgfpoint{11.913723\du}{13.748866\du}}
\pgfpathlineto{\pgfpoint{11.328838\du}{13.515204\du}}
\pgfpathlineto{\pgfpoint{11.055381\du}{13.582016\du}}
\pgfpathlineto{\pgfpoint{11.199594\du}{13.374276\du}}
\pgfpathlineto{\pgfpoint{11.904596\du}{13.374276\du}}
\pgfpathlineto{\pgfpoint{11.610328\du}{13.446201\du}}
\pgfpathlineto{\pgfpoint{12.161259\du}{13.666719\du}}
\pgfusepath{fill}
\pgfsetbuttcap
\pgfsetmiterjoin
\pgfsetdash{}{0pt}
\definecolor{dialinecolor}{rgb}{0.678431, 0.839216, 0.905882}
\pgfsetfillcolor{dialinecolor}
\pgfpathmoveto{\pgfpoint{9.357316\du}{13.310019\du}}
\pgfpathlineto{\pgfpoint{9.357316\du}{13.299432\du}}
\pgfpathlineto{\pgfpoint{9.336870\du}{13.299432\du}}
\pgfpathlineto{\pgfpoint{9.336870\du}{13.310019\du}}
\pgfpathlineto{\pgfpoint{9.357316\du}{13.310019\du}}
\pgfusepath{fill}
\pgfsetbuttcap
\pgfsetmiterjoin
\pgfsetdash{}{0pt}
\definecolor{dialinecolor}{rgb}{0.678431, 0.839216, 0.905882}
\pgfsetfillcolor{dialinecolor}
\pgfpathmoveto{\pgfpoint{9.357316\du}{14.139519\du}}
\pgfpathlineto{\pgfpoint{9.357316\du}{13.310019\du}}
\pgfpathlineto{\pgfpoint{9.336870\du}{13.310019\du}}
\pgfpathlineto{\pgfpoint{9.336870\du}{14.139519\du}}
\pgfpathlineto{\pgfpoint{9.357316\du}{14.139519\du}}
\pgfusepath{fill}
\pgfsetbuttcap
\pgfsetmiterjoin
\pgfsetdash{}{0pt}
\definecolor{dialinecolor}{rgb}{0.678431, 0.839216, 0.905882}
\pgfsetfillcolor{dialinecolor}
\pgfpathmoveto{\pgfpoint{9.336870\du}{14.139519\du}}
\pgfpathlineto{\pgfpoint{9.336870\du}{14.150107\du}}
\pgfpathlineto{\pgfpoint{9.357316\du}{14.150107\du}}
\pgfpathlineto{\pgfpoint{9.357316\du}{14.139519\du}}
\pgfpathlineto{\pgfpoint{9.336870\du}{14.139519\du}}
\pgfusepath{fill}
\pgfsetbuttcap
\pgfsetmiterjoin
\pgfsetdash{}{0pt}
\definecolor{dialinecolor}{rgb}{0.678431, 0.839216, 0.905882}
\pgfsetfillcolor{dialinecolor}
\pgfpathmoveto{\pgfpoint{12.718031\du}{13.310019\du}}
\pgfpathlineto{\pgfpoint{12.718031\du}{13.299432\du}}
\pgfpathlineto{\pgfpoint{12.697951\du}{13.299432\du}}
\pgfpathlineto{\pgfpoint{12.697951\du}{13.310019\du}}
\pgfpathlineto{\pgfpoint{12.718031\du}{13.310019\du}}
\pgfusepath{fill}
\pgfsetbuttcap
\pgfsetmiterjoin
\pgfsetdash{}{0pt}
\definecolor{dialinecolor}{rgb}{0.678431, 0.839216, 0.905882}
\pgfsetfillcolor{dialinecolor}
\pgfpathmoveto{\pgfpoint{12.718031\du}{14.139519\du}}
\pgfpathlineto{\pgfpoint{12.718031\du}{13.310019\du}}
\pgfpathlineto{\pgfpoint{12.697951\du}{13.310019\du}}
\pgfpathlineto{\pgfpoint{12.697951\du}{14.139519\du}}
\pgfpathlineto{\pgfpoint{12.718031\du}{14.139519\du}}
\pgfusepath{fill}
\pgfsetbuttcap
\pgfsetmiterjoin
\pgfsetdash{}{0pt}
\definecolor{dialinecolor}{rgb}{0.678431, 0.839216, 0.905882}
\pgfsetfillcolor{dialinecolor}
\pgfpathmoveto{\pgfpoint{12.697951\du}{14.139519\du}}
\pgfpathlineto{\pgfpoint{12.697951\du}{14.150107\du}}
\pgfpathlineto{\pgfpoint{12.718031\du}{14.150107\du}}
\pgfpathlineto{\pgfpoint{12.718031\du}{14.139519\du}}
\pgfpathlineto{\pgfpoint{12.697951\du}{14.139519\du}}
\pgfusepath{fill}
\pgfsetbuttcap
\pgfsetmiterjoin
\pgfsetdash{}{0pt}
\definecolor{dialinecolor}{rgb}{0.027451, 0.372549, 0.529412}
\pgfsetfillcolor{dialinecolor}
\pgfpathmoveto{\pgfpoint{11.644282\du}{14.288844\du}}
\pgfpathlineto{\pgfpoint{11.643917\du}{14.306368\du}}
\pgfpathlineto{\pgfpoint{11.641361\du}{14.323528\du}}
\pgfpathlineto{\pgfpoint{11.637710\du}{14.339592\du}}
\pgfpathlineto{\pgfpoint{11.631868\du}{14.356752\du}}
\pgfpathlineto{\pgfpoint{11.625297\du}{14.372451\du}}
\pgfpathlineto{\pgfpoint{11.616899\du}{14.388515\du}}
\pgfpathlineto{\pgfpoint{11.607407\du}{14.404214\du}}
\pgfpathlineto{\pgfpoint{11.595724\du}{14.419183\du}}
\pgfpathlineto{\pgfpoint{11.584041\du}{14.434152\du}}
\pgfpathlineto{\pgfpoint{11.570532\du}{14.448756\du}}
\pgfpathlineto{\pgfpoint{11.555563\du}{14.462630\du}}
\pgfpathlineto{\pgfpoint{11.539499\du}{14.476869\du}}
\pgfpathlineto{\pgfpoint{11.521974\du}{14.489647\du}}
\pgfpathlineto{\pgfpoint{11.503719\du}{14.502425\du}}
\pgfpathlineto{\pgfpoint{11.484734\du}{14.514839\du}}
\pgfpathlineto{\pgfpoint{11.464654\du}{14.526522\du}}
\pgfpathlineto{\pgfpoint{11.443113\du}{14.537110\du}}
\pgfpathlineto{\pgfpoint{11.420477\du}{14.548062\du}}
\pgfpathlineto{\pgfpoint{11.397476\du}{14.557920\du}}
\pgfpathlineto{\pgfpoint{11.373745\du}{14.567413\du}}
\pgfpathlineto{\pgfpoint{11.348553\du}{14.575445\du}}
\pgfpathlineto{\pgfpoint{11.322997\du}{14.583842\du}}
\pgfpathlineto{\pgfpoint{11.296344\du}{14.591509\du}}
\pgfpathlineto{\pgfpoint{11.269692\du}{14.598446\du}}
\pgfpathlineto{\pgfpoint{11.241580\du}{14.604287\du}}
\pgfpathlineto{\pgfpoint{11.213102\du}{14.609399\du}}
\pgfpathlineto{\pgfpoint{11.183530\du}{14.613780\du}}
\pgfpathlineto{\pgfpoint{11.153227\du}{14.617796\du}}
\pgfpathlineto{\pgfpoint{11.123289\du}{14.620717\du}}
\pgfpathlineto{\pgfpoint{11.092621\du}{14.622907\du}}
\pgfpathlineto{\pgfpoint{11.061222\du}{14.624003\du}}
\pgfpathlineto{\pgfpoint{11.029824\du}{14.624733\du}}
\pgfpathlineto{\pgfpoint{10.998791\du}{14.624003\du}}
\pgfpathlineto{\pgfpoint{10.967392\du}{14.622907\du}}
\pgfpathlineto{\pgfpoint{10.936359\du}{14.620717\du}}
\pgfpathlineto{\pgfpoint{10.906056\du}{14.617796\du}}
\pgfpathlineto{\pgfpoint{10.876483\du}{14.613780\du}}
\pgfpathlineto{\pgfpoint{10.846910\du}{14.609399\du}}
\pgfpathlineto{\pgfpoint{10.818433\du}{14.604287\du}}
\pgfpathlineto{\pgfpoint{10.790685\du}{14.598446\du}}
\pgfpathlineto{\pgfpoint{10.763303\du}{14.591509\du}}
\pgfpathlineto{\pgfpoint{10.737016\du}{14.583842\du}}
\pgfpathlineto{\pgfpoint{10.711825\du}{14.575445\du}}
\pgfpathlineto{\pgfpoint{10.686268\du}{14.567413\du}}
\pgfpathlineto{\pgfpoint{10.662171\du}{14.557920\du}}
\pgfpathlineto{\pgfpoint{10.639535\du}{14.548062\du}}
\pgfpathlineto{\pgfpoint{10.616899\du}{14.537110\du}}
\pgfpathlineto{\pgfpoint{10.596089\du}{14.526522\du}}
\pgfpathlineto{\pgfpoint{10.575644\du}{14.514839\du}}
\pgfpathlineto{\pgfpoint{10.555198\du}{14.502425\du}}
\pgfpathlineto{\pgfpoint{10.537308\du}{14.489647\du}}
\pgfpathlineto{\pgfpoint{10.520879\du}{14.476869\du}}
\pgfpathlineto{\pgfpoint{10.504085\du}{14.462630\du}}
\pgfpathlineto{\pgfpoint{10.489481\du}{14.448756\du}}
\pgfpathlineto{\pgfpoint{10.476337\du}{14.434152\du}}
\pgfpathlineto{\pgfpoint{10.463924\du}{14.419183\du}}
\pgfpathlineto{\pgfpoint{10.452971\du}{14.404214\du}}
\pgfpathlineto{\pgfpoint{10.443478\du}{14.388515\du}}
\pgfpathlineto{\pgfpoint{10.434716\du}{14.372451\du}}
\pgfpathlineto{\pgfpoint{10.427779\du}{14.356752\du}}
\pgfpathlineto{\pgfpoint{10.423033\du}{14.339592\du}}
\pgfpathlineto{\pgfpoint{10.418652\du}{14.323528\du}}
\pgfpathlineto{\pgfpoint{10.416096\du}{14.306368\du}}
\pgfpathlineto{\pgfpoint{10.415001\du}{14.288844\du}}
\pgfpathlineto{\pgfpoint{10.416096\du}{14.271319\du}}
\pgfpathlineto{\pgfpoint{10.418652\du}{14.254160\du}}
\pgfpathlineto{\pgfpoint{10.423033\du}{14.238095\du}}
\pgfpathlineto{\pgfpoint{10.427779\du}{14.220936\du}}
\pgfpathlineto{\pgfpoint{10.434716\du}{14.205237\du}}
\pgfpathlineto{\pgfpoint{10.443478\du}{14.189537\du}}
\pgfpathlineto{\pgfpoint{10.452971\du}{14.173473\du}}
\pgfpathlineto{\pgfpoint{10.463924\du}{14.158504\du}}
\pgfpathlineto{\pgfpoint{10.476337\du}{14.143900\du}}
\pgfpathlineto{\pgfpoint{10.489481\du}{14.129297\du}}
\pgfpathlineto{\pgfpoint{10.504085\du}{14.115058\du}}
\pgfpathlineto{\pgfpoint{10.520879\du}{14.101184\du}}
\pgfpathlineto{\pgfpoint{10.537308\du}{14.088041\du}}
\pgfpathlineto{\pgfpoint{10.555198\du}{14.075992\du}}
\pgfpathlineto{\pgfpoint{10.575644\du}{14.063579\du}}
\pgfpathlineto{\pgfpoint{10.596089\du}{14.051896\du}}
\pgfpathlineto{\pgfpoint{10.616899\du}{14.040943\du}}
\pgfpathlineto{\pgfpoint{10.639535\du}{14.030355\du}}
\pgfpathlineto{\pgfpoint{10.662171\du}{14.019767\du}}
\pgfpathlineto{\pgfpoint{10.686268\du}{14.011005\du}}
\pgfpathlineto{\pgfpoint{10.711825\du}{14.002243\du}}
\pgfpathlineto{\pgfpoint{10.737016\du}{13.993846\du}}
\pgfpathlineto{\pgfpoint{10.763303\du}{13.986179\du}}
\pgfpathlineto{\pgfpoint{10.790685\du}{13.979972\du}}
\pgfpathlineto{\pgfpoint{10.818433\du}{13.973400\du}}
\pgfpathlineto{\pgfpoint{10.846910\du}{13.968289\du}}
\pgfpathlineto{\pgfpoint{10.876483\du}{13.964273\du}}
\pgfpathlineto{\pgfpoint{10.906056\du}{13.959892\du}}
\pgfpathlineto{\pgfpoint{10.936359\du}{13.956971\du}}
\pgfpathlineto{\pgfpoint{10.967392\du}{13.955510\du}}
\pgfpathlineto{\pgfpoint{10.998791\du}{13.953685\du}}
\pgfpathlineto{\pgfpoint{11.029824\du}{13.953685\du}}
\pgfpathlineto{\pgfpoint{11.061222\du}{13.953685\du}}
\pgfpathlineto{\pgfpoint{11.092621\du}{13.955510\du}}
\pgfpathlineto{\pgfpoint{11.123289\du}{13.956971\du}}
\pgfpathlineto{\pgfpoint{11.153227\du}{13.959892\du}}
\pgfpathlineto{\pgfpoint{11.183530\du}{13.964273\du}}
\pgfpathlineto{\pgfpoint{11.213102\du}{13.968289\du}}
\pgfpathlineto{\pgfpoint{11.241580\du}{13.973400\du}}
\pgfpathlineto{\pgfpoint{11.269692\du}{13.979972\du}}
\pgfpathlineto{\pgfpoint{11.296344\du}{13.986179\du}}
\pgfpathlineto{\pgfpoint{11.322997\du}{13.993846\du}}
\pgfpathlineto{\pgfpoint{11.348553\du}{14.002243\du}}
\pgfpathlineto{\pgfpoint{11.373745\du}{14.011005\du}}
\pgfpathlineto{\pgfpoint{11.397476\du}{14.019767\du}}
\pgfpathlineto{\pgfpoint{11.420477\du}{14.030355\du}}
\pgfpathlineto{\pgfpoint{11.443113\du}{14.040943\du}}
\pgfpathlineto{\pgfpoint{11.464654\du}{14.051896\du}}
\pgfpathlineto{\pgfpoint{11.484734\du}{14.063579\du}}
\pgfpathlineto{\pgfpoint{11.503719\du}{14.075992\du}}
\pgfpathlineto{\pgfpoint{11.521974\du}{14.088041\du}}
\pgfpathlineto{\pgfpoint{11.539499\du}{14.101184\du}}
\pgfpathlineto{\pgfpoint{11.555563\du}{14.115058\du}}
\pgfpathlineto{\pgfpoint{11.570532\du}{14.129297\du}}
\pgfpathlineto{\pgfpoint{11.584041\du}{14.143900\du}}
\pgfpathlineto{\pgfpoint{11.595724\du}{14.158504\du}}
\pgfpathlineto{\pgfpoint{11.607407\du}{14.173473\du}}
\pgfpathlineto{\pgfpoint{11.616899\du}{14.189537\du}}
\pgfpathlineto{\pgfpoint{11.625297\du}{14.205237\du}}
\pgfpathlineto{\pgfpoint{11.631868\du}{14.220936\du}}
\pgfpathlineto{\pgfpoint{11.637710\du}{14.238095\du}}
\pgfpathlineto{\pgfpoint{11.641361\du}{14.254160\du}}
\pgfpathlineto{\pgfpoint{11.643917\du}{14.271319\du}}
\pgfpathlineto{\pgfpoint{11.644282\du}{14.288844\du}}
\pgfusepath{fill}
\pgfsetbuttcap
\pgfsetmiterjoin
\pgfsetdash{}{0pt}
\definecolor{dialinecolor}{rgb}{0.678431, 0.839216, 0.905882}
\pgfsetfillcolor{dialinecolor}
\pgfpathmoveto{\pgfpoint{11.029824\du}{14.634590\du}}
\pgfpathlineto{\pgfpoint{11.029824\du}{14.634590\du}}
\pgfpathlineto{\pgfpoint{11.045888\du}{14.634590\du}}
\pgfpathlineto{\pgfpoint{11.061952\du}{14.634225\du}}
\pgfpathlineto{\pgfpoint{11.078017\du}{14.633495\du}}
\pgfpathlineto{\pgfpoint{11.093351\du}{14.632765\du}}
\pgfpathlineto{\pgfpoint{11.109050\du}{14.631670\du}}
\pgfpathlineto{\pgfpoint{11.124384\du}{14.630574\du}}
\pgfpathlineto{\pgfpoint{11.139353\du}{14.629479\du}}
\pgfpathlineto{\pgfpoint{11.155417\du}{14.627654\du}}
\pgfpathlineto{\pgfpoint{11.169656\du}{14.625828\du}}
\pgfpathlineto{\pgfpoint{11.184625\du}{14.624003\du}}
\pgfpathlineto{\pgfpoint{11.199594\du}{14.621812\du}}
\pgfpathlineto{\pgfpoint{11.214928\du}{14.619621\du}}
\pgfpathlineto{\pgfpoint{11.229167\du}{14.616701\du}}
\pgfpathlineto{\pgfpoint{11.243040\du}{14.614145\du}}
\pgfpathlineto{\pgfpoint{11.257644\du}{14.611224\du}}
\pgfpathlineto{\pgfpoint{11.271518\du}{14.608303\du}}
\pgfpathlineto{\pgfpoint{11.285026\du}{14.604652\du}}
\pgfpathlineto{\pgfpoint{11.298900\du}{14.601367\du}}
\pgfpathlineto{\pgfpoint{11.312409\du}{14.597716\du}}
\pgfpathlineto{\pgfpoint{11.325552\du}{14.593700\du}}
\pgfpathlineto{\pgfpoint{11.338696\du}{14.589684\du}}
\pgfpathlineto{\pgfpoint{11.351839\du}{14.585667\du}}
\pgfpathlineto{\pgfpoint{11.364618\du}{14.580921\du}}
\pgfpathlineto{\pgfpoint{11.377396\du}{14.576905\du}}
\pgfpathlineto{\pgfpoint{11.389079\du}{14.572159\du}}
\pgfpathlineto{\pgfpoint{11.401857\du}{14.567413\du}}
\pgfpathlineto{\pgfpoint{11.413175\du}{14.561936\du}}
\pgfpathlineto{\pgfpoint{11.424859\du}{14.556825\du}}
\pgfpathlineto{\pgfpoint{11.436177\du}{14.551713\du}}
\pgfpathlineto{\pgfpoint{11.447495\du}{14.546237\du}}
\pgfpathlineto{\pgfpoint{11.458447\du}{14.541126\du}}
\pgfpathlineto{\pgfpoint{11.469765\du}{14.535284\du}}
\pgfpathlineto{\pgfpoint{11.479623\du}{14.529443\du}}
\pgfpathlineto{\pgfpoint{11.489846\du}{14.522871\du}}
\pgfpathlineto{\pgfpoint{11.499703\du}{14.517029\du}}
\pgfpathlineto{\pgfpoint{11.509926\du}{14.510458\du}}
\pgfpathlineto{\pgfpoint{11.519419\du}{14.504251\du}}
\pgfpathlineto{\pgfpoint{11.528546\du}{14.497679\du}}
\pgfpathlineto{\pgfpoint{11.537308\du}{14.491472\du}}
\pgfpathlineto{\pgfpoint{11.545706\du}{14.484171\du}}
\pgfpathlineto{\pgfpoint{11.554103\du}{14.477234\du}}
\pgfpathlineto{\pgfpoint{11.562135\du}{14.470297\du}}
\pgfpathlineto{\pgfpoint{11.570167\du}{14.463360\du}}
\pgfpathlineto{\pgfpoint{11.577104\du}{14.456058\du}}
\pgfpathlineto{\pgfpoint{11.584771\du}{14.448756\du}}
\pgfpathlineto{\pgfpoint{11.591343\du}{14.441089\du}}
\pgfpathlineto{\pgfpoint{11.597914\du}{14.433422\du}}
\pgfpathlineto{\pgfpoint{11.604121\du}{14.425755\du}}
\pgfpathlineto{\pgfpoint{11.610328\du}{14.417723\du}}
\pgfpathlineto{\pgfpoint{11.616169\du}{14.410056\du}}
\pgfpathlineto{\pgfpoint{11.620916\du}{14.401659\du}}
\pgfpathlineto{\pgfpoint{11.625662\du}{14.393627\du}}
\pgfpathlineto{\pgfpoint{11.630408\du}{14.385229\du}}
\pgfpathlineto{\pgfpoint{11.634789\du}{14.377197\du}}
\pgfpathlineto{\pgfpoint{11.638075\du}{14.368435\du}}
\pgfpathlineto{\pgfpoint{11.642091\du}{14.360403\du}}
\pgfpathlineto{\pgfpoint{11.644647\du}{14.351640\du}}
\pgfpathlineto{\pgfpoint{11.647568\du}{14.342878\du}}
\pgfpathlineto{\pgfpoint{11.649393\du}{14.333751\du}}
\pgfpathlineto{\pgfpoint{11.651584\du}{14.324988\du}}
\pgfpathlineto{\pgfpoint{11.653044\du}{14.316226\du}}
\pgfpathlineto{\pgfpoint{11.654139\du}{14.307099\du}}
\pgfpathlineto{\pgfpoint{11.654869\du}{14.298336\du}}
\pgfpathlineto{\pgfpoint{11.654869\du}{14.288844\du}}
\pgfpathlineto{\pgfpoint{11.634789\du}{14.288844\du}}
\pgfpathlineto{\pgfpoint{11.634059\du}{14.296876\du}}
\pgfpathlineto{\pgfpoint{11.633694\du}{14.305273\du}}
\pgfpathlineto{\pgfpoint{11.632964\du}{14.313305\du}}
\pgfpathlineto{\pgfpoint{11.631503\du}{14.320972\du}}
\pgfpathlineto{\pgfpoint{11.630043\du}{14.329370\du}}
\pgfpathlineto{\pgfpoint{11.627487\du}{14.337402\du}}
\pgfpathlineto{\pgfpoint{11.625297\du}{14.345069\du}}
\pgfpathlineto{\pgfpoint{11.622011\du}{14.352736\du}}
\pgfpathlineto{\pgfpoint{11.619455\du}{14.360403\du}}
\pgfpathlineto{\pgfpoint{11.616169\du}{14.368435\du}}
\pgfpathlineto{\pgfpoint{11.612153\du}{14.376102\du}}
\pgfpathlineto{\pgfpoint{11.607407\du}{14.383769\du}}
\pgfpathlineto{\pgfpoint{11.603391\du}{14.391436\du}}
\pgfpathlineto{\pgfpoint{11.598280\du}{14.398373\du}}
\pgfpathlineto{\pgfpoint{11.593533\du}{14.406040\du}}
\pgfpathlineto{\pgfpoint{11.588057\du}{14.412977\du}}
\pgfpathlineto{\pgfpoint{11.582945\du}{14.420644\du}}
\pgfpathlineto{\pgfpoint{11.576009\du}{14.427581\du}}
\pgfpathlineto{\pgfpoint{11.570167\du}{14.434517\du}}
\pgfpathlineto{\pgfpoint{11.562865\du}{14.441454\du}}
\pgfpathlineto{\pgfpoint{11.555928\du}{14.448756\du}}
\pgfpathlineto{\pgfpoint{11.548626\du}{14.454963\du}}
\pgfpathlineto{\pgfpoint{11.541690\du}{14.461900\du}}
\pgfpathlineto{\pgfpoint{11.533292\du}{14.468471\du}}
\pgfpathlineto{\pgfpoint{11.525260\du}{14.475043\du}}
\pgfpathlineto{\pgfpoint{11.516133\du}{14.481250\du}}
\pgfpathlineto{\pgfpoint{11.507370\du}{14.487091\du}}
\pgfpathlineto{\pgfpoint{11.497878\du}{14.493663\du}}
\pgfpathlineto{\pgfpoint{11.488750\du}{14.500235\du}}
\pgfpathlineto{\pgfpoint{11.479623\du}{14.505346\du}}
\pgfpathlineto{\pgfpoint{11.469765\du}{14.511188\du}}
\pgfpathlineto{\pgfpoint{11.459908\du}{14.517029\du}}
\pgfpathlineto{\pgfpoint{11.449320\du}{14.522506\du}}
\pgfpathlineto{\pgfpoint{11.438732\du}{14.528347\du}}
\pgfpathlineto{\pgfpoint{11.428144\du}{14.532728\du}}
\pgfpathlineto{\pgfpoint{11.416461\du}{14.538570\du}}
\pgfpathlineto{\pgfpoint{11.405143\du}{14.543316\du}}
\pgfpathlineto{\pgfpoint{11.393825\du}{14.548062\du}}
\pgfpathlineto{\pgfpoint{11.382142\du}{14.552809\du}}
\pgfpathlineto{\pgfpoint{11.370094\du}{14.557555\du}}
\pgfpathlineto{\pgfpoint{11.357681\du}{14.561571\du}}
\pgfpathlineto{\pgfpoint{11.345998\du}{14.566317\du}}
\pgfpathlineto{\pgfpoint{11.333219\du}{14.570333\du}}
\pgfpathlineto{\pgfpoint{11.319711\du}{14.573984\du}}
\pgfpathlineto{\pgfpoint{11.306932\du}{14.578000\du}}
\pgfpathlineto{\pgfpoint{11.293789\du}{14.581286\du}}
\pgfpathlineto{\pgfpoint{11.280280\du}{14.584937\du}}
\pgfpathlineto{\pgfpoint{11.266772\du}{14.587858\du}}
\pgfpathlineto{\pgfpoint{11.252898\du}{14.591509\du}}
\pgfpathlineto{\pgfpoint{11.239024\du}{14.593700\du}}
\pgfpathlineto{\pgfpoint{11.225151\du}{14.596620\du}}
\pgfpathlineto{\pgfpoint{11.210912\du}{14.598811\du}}
\pgfpathlineto{\pgfpoint{11.196673\du}{14.601367\du}}
\pgfpathlineto{\pgfpoint{11.182434\du}{14.603557\du}}
\pgfpathlineto{\pgfpoint{11.167830\du}{14.605383\du}}
\pgfpathlineto{\pgfpoint{11.152861\du}{14.607208\du}}
\pgfpathlineto{\pgfpoint{11.137527\du}{14.609034\du}}
\pgfpathlineto{\pgfpoint{11.122924\du}{14.610129\du}}
\pgfpathlineto{\pgfpoint{11.107224\du}{14.611224\du}}
\pgfpathlineto{\pgfpoint{11.092255\du}{14.612320\du}}
\pgfpathlineto{\pgfpoint{11.076921\du}{14.613050\du}}
\pgfpathlineto{\pgfpoint{11.061222\du}{14.613780\du}}
\pgfpathlineto{\pgfpoint{11.045888\du}{14.614145\du}}
\pgfpathlineto{\pgfpoint{11.029824\du}{14.614145\du}}
\pgfpathlineto{\pgfpoint{11.029824\du}{14.614145\du}}
\pgfpathlineto{\pgfpoint{11.029824\du}{14.614145\du}}
\pgfpathlineto{\pgfpoint{11.028729\du}{14.614145\du}}
\pgfpathlineto{\pgfpoint{11.027633\du}{14.614145\du}}
\pgfpathlineto{\pgfpoint{11.026903\du}{14.614875\du}}
\pgfpathlineto{\pgfpoint{11.025078\du}{14.614875\du}}
\pgfpathlineto{\pgfpoint{11.024713\du}{14.615240\du}}
\pgfpathlineto{\pgfpoint{11.023617\du}{14.615970\du}}
\pgfpathlineto{\pgfpoint{11.023252\du}{14.616701\du}}
\pgfpathlineto{\pgfpoint{11.022887\du}{14.617066\du}}
\pgfpathlineto{\pgfpoint{11.021427\du}{14.618891\du}}
\pgfpathlineto{\pgfpoint{11.019966\du}{14.620717\du}}
\pgfpathlineto{\pgfpoint{11.019966\du}{14.622542\du}}
\pgfpathlineto{\pgfpoint{11.019601\du}{14.624733\du}}
\pgfpathlineto{\pgfpoint{11.019966\du}{14.626558\du}}
\pgfpathlineto{\pgfpoint{11.019966\du}{14.628384\du}}
\pgfpathlineto{\pgfpoint{11.021427\du}{14.629844\du}}
\pgfpathlineto{\pgfpoint{11.022887\du}{14.631305\du}}
\pgfpathlineto{\pgfpoint{11.023252\du}{14.632400\du}}
\pgfpathlineto{\pgfpoint{11.023617\du}{14.632765\du}}
\pgfpathlineto{\pgfpoint{11.024713\du}{14.633495\du}}
\pgfpathlineto{\pgfpoint{11.025078\du}{14.633495\du}}
\pgfpathlineto{\pgfpoint{11.026903\du}{14.634225\du}}
\pgfpathlineto{\pgfpoint{11.027633\du}{14.634590\du}}
\pgfpathlineto{\pgfpoint{11.028729\du}{14.634590\du}}
\pgfpathlineto{\pgfpoint{11.029824\du}{14.634590\du}}
\pgfusepath{fill}
\pgfsetbuttcap
\pgfsetmiterjoin
\pgfsetdash{}{0pt}
\definecolor{dialinecolor}{rgb}{0.678431, 0.839216, 0.905882}
\pgfsetfillcolor{dialinecolor}
\pgfpathmoveto{\pgfpoint{10.405143\du}{14.288844\du}}
\pgfpathlineto{\pgfpoint{10.405143\du}{14.288844\du}}
\pgfpathlineto{\pgfpoint{10.405143\du}{14.297606\du}}
\pgfpathlineto{\pgfpoint{10.405874\du}{14.307099\du}}
\pgfpathlineto{\pgfpoint{10.406969\du}{14.316226\du}}
\pgfpathlineto{\pgfpoint{10.409159\du}{14.324988\du}}
\pgfpathlineto{\pgfpoint{10.410255\du}{14.333751\du}}
\pgfpathlineto{\pgfpoint{10.412810\du}{14.342878\du}}
\pgfpathlineto{\pgfpoint{10.415001\du}{14.351640\du}}
\pgfpathlineto{\pgfpoint{10.418287\du}{14.360403\du}}
\pgfpathlineto{\pgfpoint{10.421573\du}{14.368435\du}}
\pgfpathlineto{\pgfpoint{10.425589\du}{14.377197\du}}
\pgfpathlineto{\pgfpoint{10.429970\du}{14.385229\du}}
\pgfpathlineto{\pgfpoint{10.434351\du}{14.393627\du}}
\pgfpathlineto{\pgfpoint{10.439097\du}{14.401659\du}}
\pgfpathlineto{\pgfpoint{10.444209\du}{14.410056\du}}
\pgfpathlineto{\pgfpoint{10.450415\du}{14.417723\du}}
\pgfpathlineto{\pgfpoint{10.455162\du}{14.425755\du}}
\pgfpathlineto{\pgfpoint{10.461733\du}{14.433422\du}}
\pgfpathlineto{\pgfpoint{10.468670\du}{14.441089\du}}
\pgfpathlineto{\pgfpoint{10.475242\du}{14.448756\du}}
\pgfpathlineto{\pgfpoint{10.482179\du}{14.456058\du}}
\pgfpathlineto{\pgfpoint{10.489481\du}{14.463360\du}}
\pgfpathlineto{\pgfpoint{10.498243\du}{14.470297\du}}
\pgfpathlineto{\pgfpoint{10.505180\du}{14.477234\du}}
\pgfpathlineto{\pgfpoint{10.514307\du}{14.484171\du}}
\pgfpathlineto{\pgfpoint{10.522339\du}{14.491472\du}}
\pgfpathlineto{\pgfpoint{10.531467\du}{14.497679\du}}
\pgfpathlineto{\pgfpoint{10.541324\du}{14.504251\du}}
\pgfpathlineto{\pgfpoint{10.550452\du}{14.510458\du}}
\pgfpathlineto{\pgfpoint{10.559944\du}{14.517029\du}}
\pgfpathlineto{\pgfpoint{10.569802\du}{14.522871\du}}
\pgfpathlineto{\pgfpoint{10.580025\du}{14.529443\du}}
\pgfpathlineto{\pgfpoint{10.590247\du}{14.535284\du}}
\pgfpathlineto{\pgfpoint{10.601200\du}{14.541126\du}}
\pgfpathlineto{\pgfpoint{10.612518\du}{14.546237\du}}
\pgfpathlineto{\pgfpoint{10.623836\du}{14.551713\du}}
\pgfpathlineto{\pgfpoint{10.635154\du}{14.556825\du}}
\pgfpathlineto{\pgfpoint{10.647202\du}{14.561936\du}}
\pgfpathlineto{\pgfpoint{10.658520\du}{14.567413\du}}
\pgfpathlineto{\pgfpoint{10.670934\du}{14.572159\du}}
\pgfpathlineto{\pgfpoint{10.682982\du}{14.576905\du}}
\pgfpathlineto{\pgfpoint{10.695030\du}{14.580921\du}}
\pgfpathlineto{\pgfpoint{10.708174\du}{14.585667\du}}
\pgfpathlineto{\pgfpoint{10.720952\du}{14.589684\du}}
\pgfpathlineto{\pgfpoint{10.734461\du}{14.593700\du}}
\pgfpathlineto{\pgfpoint{10.747969\du}{14.597716\du}}
\pgfpathlineto{\pgfpoint{10.760748\du}{14.601367\du}}
\pgfpathlineto{\pgfpoint{10.774256\du}{14.604652\du}}
\pgfpathlineto{\pgfpoint{10.788130\du}{14.608303\du}}
\pgfpathlineto{\pgfpoint{10.802734\du}{14.611224\du}}
\pgfpathlineto{\pgfpoint{10.817338\du}{14.614145\du}}
\pgfpathlineto{\pgfpoint{10.831211\du}{14.616701\du}}
\pgfpathlineto{\pgfpoint{10.845450\du}{14.619621\du}}
\pgfpathlineto{\pgfpoint{10.860419\du}{14.621812\du}}
\pgfpathlineto{\pgfpoint{10.875388\du}{14.624003\du}}
\pgfpathlineto{\pgfpoint{10.889992\du}{14.625828\du}}
\pgfpathlineto{\pgfpoint{10.904596\du}{14.627654\du}}
\pgfpathlineto{\pgfpoint{10.920295\du}{14.629479\du}}
\pgfpathlineto{\pgfpoint{10.935994\du}{14.630574\du}}
\pgfpathlineto{\pgfpoint{10.950598\du}{14.631670\du}}
\pgfpathlineto{\pgfpoint{10.966662\du}{14.632765\du}}
\pgfpathlineto{\pgfpoint{10.981996\du}{14.633495\du}}
\pgfpathlineto{\pgfpoint{10.998060\du}{14.634225\du}}
\pgfpathlineto{\pgfpoint{11.014125\du}{14.634590\du}}
\pgfpathlineto{\pgfpoint{11.029824\du}{14.634590\du}}
\pgfpathlineto{\pgfpoint{11.029824\du}{14.614145\du}}
\pgfpathlineto{\pgfpoint{11.014125\du}{14.614145\du}}
\pgfpathlineto{\pgfpoint{10.998791\du}{14.613780\du}}
\pgfpathlineto{\pgfpoint{10.982726\du}{14.613050\du}}
\pgfpathlineto{\pgfpoint{10.968123\du}{14.612320\du}}
\pgfpathlineto{\pgfpoint{10.952423\du}{14.611224\du}}
\pgfpathlineto{\pgfpoint{10.937454\du}{14.610129\du}}
\pgfpathlineto{\pgfpoint{10.922485\du}{14.609034\du}}
\pgfpathlineto{\pgfpoint{10.907516\du}{14.607208\du}}
\pgfpathlineto{\pgfpoint{10.892547\du}{14.605383\du}}
\pgfpathlineto{\pgfpoint{10.877944\du}{14.603557\du}}
\pgfpathlineto{\pgfpoint{10.863340\du}{14.601367\du}}
\pgfpathlineto{\pgfpoint{10.849101\du}{14.598811\du}}
\pgfpathlineto{\pgfpoint{10.835227\du}{14.596620\du}}
\pgfpathlineto{\pgfpoint{10.821354\du}{14.593700\du}}
\pgfpathlineto{\pgfpoint{10.807115\du}{14.591509\du}}
\pgfpathlineto{\pgfpoint{10.793241\du}{14.587858\du}}
\pgfpathlineto{\pgfpoint{10.779733\du}{14.584937\du}}
\pgfpathlineto{\pgfpoint{10.766589\du}{14.581286\du}}
\pgfpathlineto{\pgfpoint{10.753081\du}{14.578000\du}}
\pgfpathlineto{\pgfpoint{10.740302\du}{14.573984\du}}
\pgfpathlineto{\pgfpoint{10.727159\du}{14.570333\du}}
\pgfpathlineto{\pgfpoint{10.714380\du}{14.566317\du}}
\pgfpathlineto{\pgfpoint{10.702332\du}{14.561571\du}}
\pgfpathlineto{\pgfpoint{10.689919\du}{14.557555\du}}
\pgfpathlineto{\pgfpoint{10.677505\du}{14.552809\du}}
\pgfpathlineto{\pgfpoint{10.666187\du}{14.548062\du}}
\pgfpathlineto{\pgfpoint{10.654504\du}{14.543316\du}}
\pgfpathlineto{\pgfpoint{10.643551\du}{14.538570\du}}
\pgfpathlineto{\pgfpoint{10.632234\du}{14.532728\du}}
\pgfpathlineto{\pgfpoint{10.621646\du}{14.528347\du}}
\pgfpathlineto{\pgfpoint{10.610693\du}{14.522506\du}}
\pgfpathlineto{\pgfpoint{10.600470\du}{14.517029\du}}
\pgfpathlineto{\pgfpoint{10.590247\du}{14.511188\du}}
\pgfpathlineto{\pgfpoint{10.580390\du}{14.505346\du}}
\pgfpathlineto{\pgfpoint{10.570897\du}{14.500235\du}}
\pgfpathlineto{\pgfpoint{10.561040\du}{14.493663\du}}
\pgfpathlineto{\pgfpoint{10.552642\du}{14.487091\du}}
\pgfpathlineto{\pgfpoint{10.543515\du}{14.481250\du}}
\pgfpathlineto{\pgfpoint{10.535118\du}{14.475043\du}}
\pgfpathlineto{\pgfpoint{10.526721\du}{14.468471\du}}
\pgfpathlineto{\pgfpoint{10.518688\du}{14.461900\du}}
\pgfpathlineto{\pgfpoint{10.511386\du}{14.454963\du}}
\pgfpathlineto{\pgfpoint{10.503719\du}{14.448756\du}}
\pgfpathlineto{\pgfpoint{10.497148\du}{14.441454\du}}
\pgfpathlineto{\pgfpoint{10.490211\du}{14.434517\du}}
\pgfpathlineto{\pgfpoint{10.484004\du}{14.427581\du}}
\pgfpathlineto{\pgfpoint{10.477798\du}{14.420644\du}}
\pgfpathlineto{\pgfpoint{10.471591\du}{14.412977\du}}
\pgfpathlineto{\pgfpoint{10.466480\du}{14.406040\du}}
\pgfpathlineto{\pgfpoint{10.461368\du}{14.398373\du}}
\pgfpathlineto{\pgfpoint{10.456622\du}{14.391436\du}}
\pgfpathlineto{\pgfpoint{10.452241\du}{14.383769\du}}
\pgfpathlineto{\pgfpoint{10.447860\du}{14.376102\du}}
\pgfpathlineto{\pgfpoint{10.444209\du}{14.368435\du}}
\pgfpathlineto{\pgfpoint{10.440923\du}{14.360403\du}}
\pgfpathlineto{\pgfpoint{10.437637\du}{14.352736\du}}
\pgfpathlineto{\pgfpoint{10.434716\du}{14.345069\du}}
\pgfpathlineto{\pgfpoint{10.432160\du}{14.337402\du}}
\pgfpathlineto{\pgfpoint{10.430335\du}{14.329370\du}}
\pgfpathlineto{\pgfpoint{10.428510\du}{14.320972\du}}
\pgfpathlineto{\pgfpoint{10.427414\du}{14.313305\du}}
\pgfpathlineto{\pgfpoint{10.425954\du}{14.305273\du}}
\pgfpathlineto{\pgfpoint{10.425589\du}{14.296876\du}}
\pgfpathlineto{\pgfpoint{10.425589\du}{14.288844\du}}
\pgfpathlineto{\pgfpoint{10.425589\du}{14.288844\du}}
\pgfpathlineto{\pgfpoint{10.425589\du}{14.288844\du}}
\pgfpathlineto{\pgfpoint{10.425589\du}{14.287749\du}}
\pgfpathlineto{\pgfpoint{10.425589\du}{14.286653\du}}
\pgfpathlineto{\pgfpoint{10.425224\du}{14.285193\du}}
\pgfpathlineto{\pgfpoint{10.425224\du}{14.284098\du}}
\pgfpathlineto{\pgfpoint{10.424859\du}{14.283732\du}}
\pgfpathlineto{\pgfpoint{10.423763\du}{14.282272\du}}
\pgfpathlineto{\pgfpoint{10.423033\du}{14.281907\du}}
\pgfpathlineto{\pgfpoint{10.423033\du}{14.281177\du}}
\pgfpathlineto{\pgfpoint{10.420842\du}{14.280081\du}}
\pgfpathlineto{\pgfpoint{10.419017\du}{14.279351\du}}
\pgfpathlineto{\pgfpoint{10.417192\du}{14.278986\du}}
\pgfpathlineto{\pgfpoint{10.415001\du}{14.278256\du}}
\pgfpathlineto{\pgfpoint{10.413541\du}{14.278986\du}}
\pgfpathlineto{\pgfpoint{10.411715\du}{14.279351\du}}
\pgfpathlineto{\pgfpoint{10.409890\du}{14.280081\du}}
\pgfpathlineto{\pgfpoint{10.408064\du}{14.281177\du}}
\pgfpathlineto{\pgfpoint{10.407334\du}{14.281907\du}}
\pgfpathlineto{\pgfpoint{10.406969\du}{14.282272\du}}
\pgfpathlineto{\pgfpoint{10.406604\du}{14.283732\du}}
\pgfpathlineto{\pgfpoint{10.405874\du}{14.284098\du}}
\pgfpathlineto{\pgfpoint{10.405874\du}{14.285193\du}}
\pgfpathlineto{\pgfpoint{10.405143\du}{14.286653\du}}
\pgfpathlineto{\pgfpoint{10.405143\du}{14.287749\du}}
\pgfpathlineto{\pgfpoint{10.405143\du}{14.288844\du}}
\pgfusepath{fill}
\pgfsetbuttcap
\pgfsetmiterjoin
\pgfsetdash{}{0pt}
\definecolor{dialinecolor}{rgb}{0.678431, 0.839216, 0.905882}
\pgfsetfillcolor{dialinecolor}
\pgfpathmoveto{\pgfpoint{11.029824\du}{13.943097\du}}
\pgfpathlineto{\pgfpoint{11.029824\du}{13.943097\du}}
\pgfpathlineto{\pgfpoint{11.014125\du}{13.943097\du}}
\pgfpathlineto{\pgfpoint{10.998060\du}{13.943462\du}}
\pgfpathlineto{\pgfpoint{10.981996\du}{13.944192\du}}
\pgfpathlineto{\pgfpoint{10.966662\du}{13.944923\du}}
\pgfpathlineto{\pgfpoint{10.950598\du}{13.946018\du}}
\pgfpathlineto{\pgfpoint{10.935994\du}{13.947113\du}}
\pgfpathlineto{\pgfpoint{10.920295\du}{13.948209\du}}
\pgfpathlineto{\pgfpoint{10.904596\du}{13.950034\du}}
\pgfpathlineto{\pgfpoint{10.889992\du}{13.951860\du}}
\pgfpathlineto{\pgfpoint{10.875388\du}{13.953685\du}}
\pgfpathlineto{\pgfpoint{10.860419\du}{13.955876\du}}
\pgfpathlineto{\pgfpoint{10.845450\du}{13.958431\du}}
\pgfpathlineto{\pgfpoint{10.831211\du}{13.961352\du}}
\pgfpathlineto{\pgfpoint{10.817338\du}{13.963543\du}}
\pgfpathlineto{\pgfpoint{10.802734\du}{13.966463\du}}
\pgfpathlineto{\pgfpoint{10.788130\du}{13.969384\du}}
\pgfpathlineto{\pgfpoint{10.774256\du}{13.973035\du}}
\pgfpathlineto{\pgfpoint{10.760748\du}{13.976321\du}}
\pgfpathlineto{\pgfpoint{10.747969\du}{13.980337\du}}
\pgfpathlineto{\pgfpoint{10.734461\du}{13.983988\du}}
\pgfpathlineto{\pgfpoint{10.720952\du}{13.988004\du}}
\pgfpathlineto{\pgfpoint{10.708174\du}{13.992385\du}}
\pgfpathlineto{\pgfpoint{10.695030\du}{13.996766\du}}
\pgfpathlineto{\pgfpoint{10.682982\du}{14.001148\du}}
\pgfpathlineto{\pgfpoint{10.670934\du}{14.005529\du}}
\pgfpathlineto{\pgfpoint{10.658520\du}{14.011005\du}}
\pgfpathlineto{\pgfpoint{10.647202\du}{14.015751\du}}
\pgfpathlineto{\pgfpoint{10.635154\du}{14.020863\du}}
\pgfpathlineto{\pgfpoint{10.623836\du}{14.025974\du}}
\pgfpathlineto{\pgfpoint{10.612518\du}{14.031451\du}}
\pgfpathlineto{\pgfpoint{10.601200\du}{14.037292\du}}
\pgfpathlineto{\pgfpoint{10.590247\du}{14.042403\du}}
\pgfpathlineto{\pgfpoint{10.580025\du}{14.048245\du}}
\pgfpathlineto{\pgfpoint{10.569802\du}{14.054817\du}}
\pgfpathlineto{\pgfpoint{10.559944\du}{14.060658\du}}
\pgfpathlineto{\pgfpoint{10.550452\du}{14.067230\du}}
\pgfpathlineto{\pgfpoint{10.541324\du}{14.073437\du}}
\pgfpathlineto{\pgfpoint{10.531467\du}{14.080008\du}}
\pgfpathlineto{\pgfpoint{10.522339\du}{14.086580\du}}
\pgfpathlineto{\pgfpoint{10.514307\du}{14.093517\du}}
\pgfpathlineto{\pgfpoint{10.505180\du}{14.100454\du}}
\pgfpathlineto{\pgfpoint{10.498243\du}{14.107391\du}}
\pgfpathlineto{\pgfpoint{10.489481\du}{14.114328\du}}
\pgfpathlineto{\pgfpoint{10.482179\du}{14.121629\du}}
\pgfpathlineto{\pgfpoint{10.475242\du}{14.129297\du}}
\pgfpathlineto{\pgfpoint{10.468670\du}{14.136598\du}}
\pgfpathlineto{\pgfpoint{10.461733\du}{14.144265\du}}
\pgfpathlineto{\pgfpoint{10.455162\du}{14.151933\du}}
\pgfpathlineto{\pgfpoint{10.450415\du}{14.159965\du}}
\pgfpathlineto{\pgfpoint{10.444209\du}{14.167632\du}}
\pgfpathlineto{\pgfpoint{10.439097\du}{14.176029\du}}
\pgfpathlineto{\pgfpoint{10.434351\du}{14.184061\du}}
\pgfpathlineto{\pgfpoint{10.429970\du}{14.192458\du}}
\pgfpathlineto{\pgfpoint{10.425589\du}{14.200490\du}}
\pgfpathlineto{\pgfpoint{10.421573\du}{14.209253\du}}
\pgfpathlineto{\pgfpoint{10.418287\du}{14.217650\du}}
\pgfpathlineto{\pgfpoint{10.415001\du}{14.226412\du}}
\pgfpathlineto{\pgfpoint{10.412810\du}{14.235175\du}}
\pgfpathlineto{\pgfpoint{10.410255\du}{14.243937\du}}
\pgfpathlineto{\pgfpoint{10.409159\du}{14.252699\du}}
\pgfpathlineto{\pgfpoint{10.406969\du}{14.261462\du}}
\pgfpathlineto{\pgfpoint{10.405874\du}{14.270589\du}}
\pgfpathlineto{\pgfpoint{10.405143\du}{14.280081\du}}
\pgfpathlineto{\pgfpoint{10.405143\du}{14.288844\du}}
\pgfpathlineto{\pgfpoint{10.425589\du}{14.288844\du}}
\pgfpathlineto{\pgfpoint{10.425589\du}{14.280812\du}}
\pgfpathlineto{\pgfpoint{10.425954\du}{14.272414\du}}
\pgfpathlineto{\pgfpoint{10.427414\du}{14.264382\du}}
\pgfpathlineto{\pgfpoint{10.428510\du}{14.256715\du}}
\pgfpathlineto{\pgfpoint{10.430335\du}{14.248318\du}}
\pgfpathlineto{\pgfpoint{10.432160\du}{14.240286\du}}
\pgfpathlineto{\pgfpoint{10.434716\du}{14.232619\du}}
\pgfpathlineto{\pgfpoint{10.437637\du}{14.224952\du}}
\pgfpathlineto{\pgfpoint{10.440923\du}{14.217650\du}}
\pgfpathlineto{\pgfpoint{10.444209\du}{14.209253\du}}
\pgfpathlineto{\pgfpoint{10.447860\du}{14.201586\du}}
\pgfpathlineto{\pgfpoint{10.452241\du}{14.193919\du}}
\pgfpathlineto{\pgfpoint{10.456622\du}{14.186982\du}}
\pgfpathlineto{\pgfpoint{10.461368\du}{14.179315\du}}
\pgfpathlineto{\pgfpoint{10.466480\du}{14.172013\du}}
\pgfpathlineto{\pgfpoint{10.471591\du}{14.164711\du}}
\pgfpathlineto{\pgfpoint{10.477798\du}{14.157044\du}}
\pgfpathlineto{\pgfpoint{10.484004\du}{14.150107\du}}
\pgfpathlineto{\pgfpoint{10.490211\du}{14.143170\du}}
\pgfpathlineto{\pgfpoint{10.497148\du}{14.136233\du}}
\pgfpathlineto{\pgfpoint{10.503719\du}{14.129662\du}}
\pgfpathlineto{\pgfpoint{10.511386\du}{14.122725\du}}
\pgfpathlineto{\pgfpoint{10.518688\du}{14.115788\du}}
\pgfpathlineto{\pgfpoint{10.526721\du}{14.109216\du}}
\pgfpathlineto{\pgfpoint{10.535118\du}{14.102644\du}}
\pgfpathlineto{\pgfpoint{10.543515\du}{14.096438\du}}
\pgfpathlineto{\pgfpoint{10.552642\du}{14.090596\du}}
\pgfpathlineto{\pgfpoint{10.561040\du}{14.084025\du}}
\pgfpathlineto{\pgfpoint{10.570897\du}{14.078183\du}}
\pgfpathlineto{\pgfpoint{10.580390\du}{14.072341\du}}
\pgfpathlineto{\pgfpoint{10.590247\du}{14.066500\du}}
\pgfpathlineto{\pgfpoint{10.600470\du}{14.060658\du}}
\pgfpathlineto{\pgfpoint{10.610693\du}{14.055547\du}}
\pgfpathlineto{\pgfpoint{10.621646\du}{14.049705\du}}
\pgfpathlineto{\pgfpoint{10.632234\du}{14.044959\du}}
\pgfpathlineto{\pgfpoint{10.643551\du}{14.039483\du}}
\pgfpathlineto{\pgfpoint{10.654504\du}{14.034371\du}}
\pgfpathlineto{\pgfpoint{10.666187\du}{14.029625\du}}
\pgfpathlineto{\pgfpoint{10.677505\du}{14.024879\du}}
\pgfpathlineto{\pgfpoint{10.689919\du}{14.020133\du}}
\pgfpathlineto{\pgfpoint{10.702332\du}{14.016117\du}}
\pgfpathlineto{\pgfpoint{10.714380\du}{14.011370\du}}
\pgfpathlineto{\pgfpoint{10.727159\du}{14.007354\du}}
\pgfpathlineto{\pgfpoint{10.740302\du}{14.004068\du}}
\pgfpathlineto{\pgfpoint{10.753081\du}{13.999687\du}}
\pgfpathlineto{\pgfpoint{10.766589\du}{13.996401\du}}
\pgfpathlineto{\pgfpoint{10.779733\du}{13.992750\du}}
\pgfpathlineto{\pgfpoint{10.793241\du}{13.989830\du}}
\pgfpathlineto{\pgfpoint{10.807115\du}{13.986909\du}}
\pgfpathlineto{\pgfpoint{10.821354\du}{13.983988\du}}
\pgfpathlineto{\pgfpoint{10.835227\du}{13.981067\du}}
\pgfpathlineto{\pgfpoint{10.849101\du}{13.978877\du}}
\pgfpathlineto{\pgfpoint{10.863340\du}{13.976321\du}}
\pgfpathlineto{\pgfpoint{10.877944\du}{13.974130\du}}
\pgfpathlineto{\pgfpoint{10.892547\du}{13.972305\du}}
\pgfpathlineto{\pgfpoint{10.907516\du}{13.970479\du}}
\pgfpathlineto{\pgfpoint{10.922485\du}{13.968654\du}}
\pgfpathlineto{\pgfpoint{10.937454\du}{13.967559\du}}
\pgfpathlineto{\pgfpoint{10.952423\du}{13.966463\du}}
\pgfpathlineto{\pgfpoint{10.968123\du}{13.965368\du}}
\pgfpathlineto{\pgfpoint{10.982726\du}{13.964638\du}}
\pgfpathlineto{\pgfpoint{10.998791\du}{13.964273\du}}
\pgfpathlineto{\pgfpoint{11.014125\du}{13.963543\du}}
\pgfpathlineto{\pgfpoint{11.029824\du}{13.963543\du}}
\pgfpathlineto{\pgfpoint{11.029824\du}{13.963543\du}}
\pgfpathlineto{\pgfpoint{11.029824\du}{13.963543\du}}
\pgfpathlineto{\pgfpoint{11.031284\du}{13.963543\du}}
\pgfpathlineto{\pgfpoint{11.032380\du}{13.963543\du}}
\pgfpathlineto{\pgfpoint{11.033475\du}{13.963543\du}}
\pgfpathlineto{\pgfpoint{11.034570\du}{13.962812\du}}
\pgfpathlineto{\pgfpoint{11.035665\du}{13.962447\du}}
\pgfpathlineto{\pgfpoint{11.036761\du}{13.961717\du}}
\pgfpathlineto{\pgfpoint{11.036761\du}{13.961352\du}}
\pgfpathlineto{\pgfpoint{11.037491\du}{13.960622\du}}
\pgfpathlineto{\pgfpoint{11.038586\du}{13.958796\du}}
\pgfpathlineto{\pgfpoint{11.039681\du}{13.956971\du}}
\pgfpathlineto{\pgfpoint{11.040047\du}{13.955510\du}}
\pgfpathlineto{\pgfpoint{11.040047\du}{13.953685\du}}
\pgfpathlineto{\pgfpoint{11.040047\du}{13.951129\du}}
\pgfpathlineto{\pgfpoint{11.039681\du}{13.949669\du}}
\pgfpathlineto{\pgfpoint{11.038586\du}{13.947843\du}}
\pgfpathlineto{\pgfpoint{11.037491\du}{13.946748\du}}
\pgfpathlineto{\pgfpoint{11.036761\du}{13.945288\du}}
\pgfpathlineto{\pgfpoint{11.036761\du}{13.944923\du}}
\pgfpathlineto{\pgfpoint{11.035665\du}{13.944192\du}}
\pgfpathlineto{\pgfpoint{11.034570\du}{13.944192\du}}
\pgfpathlineto{\pgfpoint{11.033475\du}{13.943462\du}}
\pgfpathlineto{\pgfpoint{11.032380\du}{13.943462\du}}
\pgfpathlineto{\pgfpoint{11.031284\du}{13.943097\du}}
\pgfpathlineto{\pgfpoint{11.029824\du}{13.943097\du}}
\pgfusepath{fill}
\pgfsetbuttcap
\pgfsetmiterjoin
\pgfsetdash{}{0pt}
\definecolor{dialinecolor}{rgb}{0.678431, 0.839216, 0.905882}
\pgfsetfillcolor{dialinecolor}
\pgfpathmoveto{\pgfpoint{11.654869\du}{14.288844\du}}
\pgfpathlineto{\pgfpoint{11.654869\du}{14.279351\du}}
\pgfpathlineto{\pgfpoint{11.654139\du}{14.270589\du}}
\pgfpathlineto{\pgfpoint{11.653044\du}{14.261462\du}}
\pgfpathlineto{\pgfpoint{11.651584\du}{14.252699\du}}
\pgfpathlineto{\pgfpoint{11.649393\du}{14.243937\du}}
\pgfpathlineto{\pgfpoint{11.647568\du}{14.235175\du}}
\pgfpathlineto{\pgfpoint{11.644647\du}{14.226412\du}}
\pgfpathlineto{\pgfpoint{11.642091\du}{14.217650\du}}
\pgfpathlineto{\pgfpoint{11.638075\du}{14.209253\du}}
\pgfpathlineto{\pgfpoint{11.634789\du}{14.200490\du}}
\pgfpathlineto{\pgfpoint{11.630408\du}{14.192458\du}}
\pgfpathlineto{\pgfpoint{11.625662\du}{14.184061\du}}
\pgfpathlineto{\pgfpoint{11.620916\du}{14.176029\du}}
\pgfpathlineto{\pgfpoint{11.616169\du}{14.167632\du}}
\pgfpathlineto{\pgfpoint{11.610328\du}{14.159965\du}}
\pgfpathlineto{\pgfpoint{11.604121\du}{14.151933\du}}
\pgfpathlineto{\pgfpoint{11.597914\du}{14.144265\du}}
\pgfpathlineto{\pgfpoint{11.591343\du}{14.136598\du}}
\pgfpathlineto{\pgfpoint{11.584771\du}{14.129297\du}}
\pgfpathlineto{\pgfpoint{11.577104\du}{14.121629\du}}
\pgfpathlineto{\pgfpoint{11.570167\du}{14.114328\du}}
\pgfpathlineto{\pgfpoint{11.562135\du}{14.107391\du}}
\pgfpathlineto{\pgfpoint{11.554103\du}{14.100454\du}}
\pgfpathlineto{\pgfpoint{11.545706\du}{14.093517\du}}
\pgfpathlineto{\pgfpoint{11.537308\du}{14.086580\du}}
\pgfpathlineto{\pgfpoint{11.528546\du}{14.080008\du}}
\pgfpathlineto{\pgfpoint{11.519419\du}{14.073437\du}}
\pgfpathlineto{\pgfpoint{11.509926\du}{14.067230\du}}
\pgfpathlineto{\pgfpoint{11.499703\du}{14.060658\du}}
\pgfpathlineto{\pgfpoint{11.489846\du}{14.054817\du}}
\pgfpathlineto{\pgfpoint{11.479623\du}{14.048245\du}}
\pgfpathlineto{\pgfpoint{11.469765\du}{14.042403\du}}
\pgfpathlineto{\pgfpoint{11.458447\du}{14.037292\du}}
\pgfpathlineto{\pgfpoint{11.447495\du}{14.031451\du}}
\pgfpathlineto{\pgfpoint{11.436177\du}{14.025974\du}}
\pgfpathlineto{\pgfpoint{11.424859\du}{14.020863\du}}
\pgfpathlineto{\pgfpoint{11.413175\du}{14.015751\du}}
\pgfpathlineto{\pgfpoint{11.401857\du}{14.011005\du}}
\pgfpathlineto{\pgfpoint{11.389079\du}{14.005529\du}}
\pgfpathlineto{\pgfpoint{11.377396\du}{14.001148\du}}
\pgfpathlineto{\pgfpoint{11.364618\du}{13.996766\du}}
\pgfpathlineto{\pgfpoint{11.351839\du}{13.992385\du}}
\pgfpathlineto{\pgfpoint{11.338696\du}{13.988004\du}}
\pgfpathlineto{\pgfpoint{11.325552\du}{13.983988\du}}
\pgfpathlineto{\pgfpoint{11.312409\du}{13.980337\du}}
\pgfpathlineto{\pgfpoint{11.298900\du}{13.976321\du}}
\pgfpathlineto{\pgfpoint{11.285026\du}{13.973035\du}}
\pgfpathlineto{\pgfpoint{11.271518\du}{13.969384\du}}
\pgfpathlineto{\pgfpoint{11.257644\du}{13.966463\du}}
\pgfpathlineto{\pgfpoint{11.243040\du}{13.963543\du}}
\pgfpathlineto{\pgfpoint{11.229167\du}{13.961352\du}}
\pgfpathlineto{\pgfpoint{11.214928\du}{13.958431\du}}
\pgfpathlineto{\pgfpoint{11.199594\du}{13.955876\du}}
\pgfpathlineto{\pgfpoint{11.184625\du}{13.953685\du}}
\pgfpathlineto{\pgfpoint{11.169656\du}{13.951860\du}}
\pgfpathlineto{\pgfpoint{11.155417\du}{13.950034\du}}
\pgfpathlineto{\pgfpoint{11.139353\du}{13.948209\du}}
\pgfpathlineto{\pgfpoint{11.124384\du}{13.947113\du}}
\pgfpathlineto{\pgfpoint{11.109050\du}{13.946018\du}}
\pgfpathlineto{\pgfpoint{11.093351\du}{13.944923\du}}
\pgfpathlineto{\pgfpoint{11.078017\du}{13.944192\du}}
\pgfpathlineto{\pgfpoint{11.061952\du}{13.943462\du}}
\pgfpathlineto{\pgfpoint{11.045888\du}{13.943097\du}}
\pgfpathlineto{\pgfpoint{11.029824\du}{13.943097\du}}
\pgfpathlineto{\pgfpoint{11.029824\du}{13.963543\du}}
\pgfpathlineto{\pgfpoint{11.045888\du}{13.963543\du}}
\pgfpathlineto{\pgfpoint{11.061222\du}{13.964273\du}}
\pgfpathlineto{\pgfpoint{11.076921\du}{13.964638\du}}
\pgfpathlineto{\pgfpoint{11.092255\du}{13.965368\du}}
\pgfpathlineto{\pgfpoint{11.107224\du}{13.966463\du}}
\pgfpathlineto{\pgfpoint{11.122924\du}{13.967559\du}}
\pgfpathlineto{\pgfpoint{11.137527\du}{13.968654\du}}
\pgfpathlineto{\pgfpoint{11.152861\du}{13.970479\du}}
\pgfpathlineto{\pgfpoint{11.167830\du}{13.972305\du}}
\pgfpathlineto{\pgfpoint{11.182434\du}{13.974130\du}}
\pgfpathlineto{\pgfpoint{11.196673\du}{13.976321\du}}
\pgfpathlineto{\pgfpoint{11.210912\du}{13.978877\du}}
\pgfpathlineto{\pgfpoint{11.225151\du}{13.981067\du}}
\pgfpathlineto{\pgfpoint{11.239024\du}{13.983988\du}}
\pgfpathlineto{\pgfpoint{11.252898\du}{13.986909\du}}
\pgfpathlineto{\pgfpoint{11.266772\du}{13.989830\du}}
\pgfpathlineto{\pgfpoint{11.280280\du}{13.992750\du}}
\pgfpathlineto{\pgfpoint{11.293789\du}{13.996401\du}}
\pgfpathlineto{\pgfpoint{11.306932\du}{13.999687\du}}
\pgfpathlineto{\pgfpoint{11.319711\du}{14.004068\du}}
\pgfpathlineto{\pgfpoint{11.333219\du}{14.007354\du}}
\pgfpathlineto{\pgfpoint{11.345998\du}{14.011370\du}}
\pgfpathlineto{\pgfpoint{11.357681\du}{14.016117\du}}
\pgfpathlineto{\pgfpoint{11.370094\du}{14.020133\du}}
\pgfpathlineto{\pgfpoint{11.382142\du}{14.024879\du}}
\pgfpathlineto{\pgfpoint{11.393825\du}{14.029625\du}}
\pgfpathlineto{\pgfpoint{11.405143\du}{14.034371\du}}
\pgfpathlineto{\pgfpoint{11.416461\du}{14.039483\du}}
\pgfpathlineto{\pgfpoint{11.428144\du}{14.044959\du}}
\pgfpathlineto{\pgfpoint{11.438732\du}{14.049705\du}}
\pgfpathlineto{\pgfpoint{11.449320\du}{14.055547\du}}
\pgfpathlineto{\pgfpoint{11.459908\du}{14.060658\du}}
\pgfpathlineto{\pgfpoint{11.469765\du}{14.066500\du}}
\pgfpathlineto{\pgfpoint{11.479623\du}{14.072341\du}}
\pgfpathlineto{\pgfpoint{11.488750\du}{14.078183\du}}
\pgfpathlineto{\pgfpoint{11.497878\du}{14.084025\du}}
\pgfpathlineto{\pgfpoint{11.507370\du}{14.090596\du}}
\pgfpathlineto{\pgfpoint{11.516133\du}{14.096438\du}}
\pgfpathlineto{\pgfpoint{11.525260\du}{14.102644\du}}
\pgfpathlineto{\pgfpoint{11.533292\du}{14.109216\du}}
\pgfpathlineto{\pgfpoint{11.541690\du}{14.115788\du}}
\pgfpathlineto{\pgfpoint{11.548626\du}{14.122725\du}}
\pgfpathlineto{\pgfpoint{11.555928\du}{14.129662\du}}
\pgfpathlineto{\pgfpoint{11.562865\du}{14.136233\du}}
\pgfpathlineto{\pgfpoint{11.570167\du}{14.143170\du}}
\pgfpathlineto{\pgfpoint{11.576009\du}{14.150107\du}}
\pgfpathlineto{\pgfpoint{11.582945\du}{14.157044\du}}
\pgfpathlineto{\pgfpoint{11.588057\du}{14.164711\du}}
\pgfpathlineto{\pgfpoint{11.593533\du}{14.172013\du}}
\pgfpathlineto{\pgfpoint{11.598280\du}{14.179315\du}}
\pgfpathlineto{\pgfpoint{11.603391\du}{14.186982\du}}
\pgfpathlineto{\pgfpoint{11.607407\du}{14.193919\du}}
\pgfpathlineto{\pgfpoint{11.612153\du}{14.201586\du}}
\pgfpathlineto{\pgfpoint{11.616169\du}{14.209253\du}}
\pgfpathlineto{\pgfpoint{11.619455\du}{14.217650\du}}
\pgfpathlineto{\pgfpoint{11.622011\du}{14.224952\du}}
\pgfpathlineto{\pgfpoint{11.625297\du}{14.232619\du}}
\pgfpathlineto{\pgfpoint{11.627487\du}{14.240286\du}}
\pgfpathlineto{\pgfpoint{11.630043\du}{14.248318\du}}
\pgfpathlineto{\pgfpoint{11.631503\du}{14.256715\du}}
\pgfpathlineto{\pgfpoint{11.632964\du}{14.264382\du}}
\pgfpathlineto{\pgfpoint{11.633694\du}{14.272414\du}}
\pgfpathlineto{\pgfpoint{11.634059\du}{14.280812\du}}
\pgfpathlineto{\pgfpoint{11.634789\du}{14.288844\du}}
\pgfpathlineto{\pgfpoint{11.654869\du}{14.288844\du}}
\pgfusepath{fill}
\pgfsetbuttcap
\pgfsetmiterjoin
\pgfsetdash{}{0pt}
\definecolor{dialinecolor}{rgb}{0.074510, 0.082353, 0.086275}
\pgfsetfillcolor{dialinecolor}
\pgfpathmoveto{\pgfpoint{10.708904\du}{14.383039\du}}
\pgfpathlineto{\pgfpoint{10.936359\du}{14.154853\du}}
\pgfpathlineto{\pgfpoint{10.876483\du}{14.093882\du}}
\pgfpathlineto{\pgfpoint{11.056841\du}{14.093882\du}}
\pgfpathlineto{\pgfpoint{11.056841\du}{14.282272\du}}
\pgfpathlineto{\pgfpoint{10.996600\du}{14.222031\du}}
\pgfpathlineto{\pgfpoint{10.776447\du}{14.443280\du}}
\pgfpathlineto{\pgfpoint{10.708904\du}{14.383039\du}}
\pgfusepath{fill}
\pgfsetbuttcap
\pgfsetmiterjoin
\pgfsetdash{}{0pt}
\definecolor{dialinecolor}{rgb}{0.074510, 0.082353, 0.086275}
\pgfsetfillcolor{dialinecolor}
\pgfpathmoveto{\pgfpoint{10.976885\du}{14.503521\du}}
\pgfpathlineto{\pgfpoint{11.203975\du}{14.275335\du}}
\pgfpathlineto{\pgfpoint{11.143369\du}{14.215094\du}}
\pgfpathlineto{\pgfpoint{11.324457\du}{14.215094\du}}
\pgfpathlineto{\pgfpoint{11.324457\du}{14.403119\du}}
\pgfpathlineto{\pgfpoint{11.263851\du}{14.342878\du}}
\pgfpathlineto{\pgfpoint{11.043332\du}{14.563762\du}}
\pgfpathlineto{\pgfpoint{10.976885\du}{14.503521\du}}
\pgfusepath{fill}
\pgfsetbuttcap
\pgfsetmiterjoin
\pgfsetdash{}{0pt}
\definecolor{dialinecolor}{rgb}{1.000000, 1.000000, 1.000000}
\pgfsetfillcolor{dialinecolor}
\pgfpathmoveto{\pgfpoint{10.695760\du}{14.369530\du}}
\pgfpathlineto{\pgfpoint{10.922851\du}{14.141345\du}}
\pgfpathlineto{\pgfpoint{10.863340\du}{14.081104\du}}
\pgfpathlineto{\pgfpoint{11.043332\du}{14.081104\du}}
\pgfpathlineto{\pgfpoint{11.043332\du}{14.269129\du}}
\pgfpathlineto{\pgfpoint{10.983457\du}{14.208157\du}}
\pgfpathlineto{\pgfpoint{10.762938\du}{14.429771\du}}
\pgfpathlineto{\pgfpoint{10.695760\du}{14.369530\du}}
\pgfusepath{fill}
\pgfsetbuttcap
\pgfsetmiterjoin
\pgfsetdash{}{0pt}
\definecolor{dialinecolor}{rgb}{1.000000, 1.000000, 1.000000}
\pgfsetfillcolor{dialinecolor}
\pgfpathmoveto{\pgfpoint{10.963376\du}{14.490012\du}}
\pgfpathlineto{\pgfpoint{11.190466\du}{14.261827\du}}
\pgfpathlineto{\pgfpoint{11.130225\du}{14.201586\du}}
\pgfpathlineto{\pgfpoint{11.310583\du}{14.201586\du}}
\pgfpathlineto{\pgfpoint{11.310583\du}{14.389611\du}}
\pgfpathlineto{\pgfpoint{11.251072\du}{14.329370\du}}
\pgfpathlineto{\pgfpoint{11.029824\du}{14.550253\du}}
\pgfpathlineto{\pgfpoint{10.963376\du}{14.490012\du}}
\pgfusepath{fill}
% setfont left to latex
\definecolor{dialinecolor}{rgb}{0.000000, 0.000000, 0.000000}
\pgfsetstrokecolor{dialinecolor}
\node[anchor=west] at (6.925249\du,9.896976\du){Hop : 3};
% setfont left to latex
\definecolor{dialinecolor}{rgb}{0.000000, 0.000000, 0.000000}
\pgfsetstrokecolor{dialinecolor}
\node[anchor=west] at (6.860052\du,10.900012\du){IP : 185.147.12.19};
% setfont left to latex
\definecolor{dialinecolor}{rgb}{0.000000, 0.000000, 0.000000}
\pgfsetstrokecolor{dialinecolor}
\node[anchor=west] at (6.675648\du,11.945873\du){RTTs : 4.347, 2.876, 3.143};
\pgfsetlinewidth{0.000000\du}
\pgfsetdash{}{0pt}
\pgfsetdash{}{0pt}
\pgfsetbuttcap
{
\definecolor{dialinecolor}{rgb}{0.000000, 0.000000, 0.000000}
\pgfsetfillcolor{dialinecolor}
% was here!!!
\definecolor{dialinecolor}{rgb}{0.000000, 0.000000, 0.000000}
\pgfsetstrokecolor{dialinecolor}
\pgfpathmoveto{\pgfpoint{2.447533\du}{15.149933\du}}
\pgfpatharc{102}{80}{19.994254\du and 19.994254\du}
\pgfusepath{stroke}
}
% setfont left to latex
\definecolor{dialinecolor}{rgb}{0.000000, 0.000000, 0.000000}
\pgfsetstrokecolor{dialinecolor}
\node[anchor=west] at (0.078656\du,24.994938\du){Lien : ('185.147.12.31', '185.147.12.19')};
% setfont left to latex
\definecolor{dialinecolor}{rgb}{0.000000, 0.000000, 0.000000}
\pgfsetstrokecolor{dialinecolor}
\node[anchor=west] at (-0.182131\du,26.038084\du){RTT différentiel : \ensuremath{[}-0.02900000000000036\ensuremath{]}};
% setfont left to latex
\definecolor{dialinecolor}{rgb}{0.000000, 0.000000, 0.000000}
\pgfsetstrokecolor{dialinecolor}
\node[anchor=west] at (2.425735\du,27.081231\du){probe': \ensuremath{[}'89.105.202.4'\ensuremath{]}};
% setfont left to latex
\definecolor{dialinecolor}{rgb}{0.000000, 0.000000, 0.000000}
\pgfsetstrokecolor{dialinecolor}
\node[anchor=west] at (2.621325\du,27.993984\du){msmId: \{5004: \ensuremath{[}4247\ensuremath{]}\}};
% setfont left to latex
\definecolor{dialinecolor}{rgb}{0.000000, 0.000000, 0.000000}
\pgfsetstrokecolor{dialinecolor}
\node[anchor=west] at (12.792005\du,18.996845\du){RTT différentiel : \ensuremath{[}0.5350000000000001\ensuremath{]}};
% setfont left to latex
\definecolor{dialinecolor}{rgb}{0.000000, 0.000000, 0.000000}
\pgfsetstrokecolor{dialinecolor}
\node[anchor=west] at (14.813101\du,19.974795\du){probe: \ensuremath{[}'89.105.202.4'\ensuremath{]}};
% setfont left to latex
\definecolor{dialinecolor}{rgb}{0.000000, 0.000000, 0.000000}
\pgfsetstrokecolor{dialinecolor}
\node[anchor=west] at (15.008691\du,20.887548\du){msmId : \{5004: \ensuremath{[}4247\ensuremath{]}\}};
\pgfsetlinewidth{0.000000\du}
\pgfsetdash{}{0pt}
\pgfsetdash{}{0pt}
\pgfsetbuttcap
{
\definecolor{dialinecolor}{rgb}{0.000000, 0.000000, 0.000000}
\pgfsetfillcolor{dialinecolor}
% was here!!!
\pgfsetarrowsend{to}
\definecolor{dialinecolor}{rgb}{0.000000, 0.000000, 0.000000}
\pgfsetstrokecolor{dialinecolor}
\draw (-3.940262\du,16.780158\du)--(-3.940262\du,16.128192\du);
}
\pgfsetlinewidth{0.000000\du}
\pgfsetdash{}{0pt}
\pgfsetdash{}{0pt}
\pgfsetbuttcap
{
\definecolor{dialinecolor}{rgb}{0.000000, 0.000000, 0.000000}
\pgfsetfillcolor{dialinecolor}
% was here!!!
\pgfsetarrowsend{to}
\definecolor{dialinecolor}{rgb}{0.000000, 0.000000, 0.000000}
\pgfsetstrokecolor{dialinecolor}
\draw (6.751991\du,23.756201\du)--(6.751991\du,16.519372\du);
}
\pgfsetlinewidth{0.000000\du}
\pgfsetdash{}{0pt}
\pgfsetdash{}{0pt}
\pgfsetbuttcap
{
\definecolor{dialinecolor}{rgb}{0.000000, 0.000000, 0.000000}
\pgfsetfillcolor{dialinecolor}
% was here!!!
\pgfsetarrowsend{to}
\definecolor{dialinecolor}{rgb}{0.000000, 0.000000, 0.000000}
\pgfsetstrokecolor{dialinecolor}
\draw (17.053064\du,16.845355\du)--(17.053064\du,15.997798\du);
}
\pgfsetlinewidth{0.000000\du}
\pgfsetdash{}{0pt}
\pgfsetdash{}{0pt}
\pgfsetbuttcap
{
\definecolor{dialinecolor}{rgb}{0.000000, 0.000000, 0.000000}
\pgfsetfillcolor{dialinecolor}
% was here!!!
\pgfsetarrowsend{to}
\definecolor{dialinecolor}{rgb}{0.000000, 0.000000, 0.000000}
\pgfsetstrokecolor{dialinecolor}
\draw (27.875710\du,23.756201\du)--(27.875710\du,16.845355\du);
}
\pgfsetlinewidth{0.000000\du}
\pgfsetdash{}{0pt}
\pgfsetdash{}{0pt}
\pgfsetbuttcap
{
\definecolor{dialinecolor}{rgb}{0.000000, 0.000000, 0.000000}
\pgfsetfillcolor{dialinecolor}
% was here!!!
\pgfsetarrowsend{to}
\definecolor{dialinecolor}{rgb}{0.000000, 0.000000, 0.000000}
\pgfsetstrokecolor{dialinecolor}
\draw (47.816775\du,23.769241\du)--(47.816775\du,16.858394\du);
}
\pgfsetlinewidth{0.000000\du}
\pgfsetdash{}{0pt}
\pgfsetdash{}{0pt}
\pgfsetbuttcap
{
\definecolor{dialinecolor}{rgb}{0.000000, 0.000000, 0.000000}
\pgfsetfillcolor{dialinecolor}
% was here!!!
\pgfsetarrowsend{to}
\definecolor{dialinecolor}{rgb}{0.000000, 0.000000, 0.000000}
\pgfsetstrokecolor{dialinecolor}
\draw (37.937228\du,17.106142\du)--(38.002424\du,16.519372\du);
}
\end{tikzpicture}
 
 		}
 		\caption{Caractérisation des liens dans un traceroute}
 		\label{fig:links-inference}
 	\end{figure}
 \end{landscape}

  
\paragraph{Fusion des détails des liens}~
 
 La méthode conçue pour la détection des anomalies dans les délais des liens a démontré ses performances si le lien suivi a été identifié plusieurs fois. Autrement dit, l'échantillon des RTTs différentiels caractérisant un lien est assez grande. L'ordre des deux routeurs d'un lien est indifférent, de ce fait, le lien (R1, R2) est similaire au lien (R2, R1) en terme du RTT différentiel. Donc après avoir caractérisé chaque lien, à l'étape de la fusion, les RTTs différentiels de chaque lien sont fusionnés afin d'avoir un seul échantillon caractérisant un lien durant une période donnée.
 
 
\subsection{Exploitation des données}

\section{Application de l'algorithme de la détection}



\chapter{La détection des anomalies dans les délais d'un lien}

Dans le présent chapitre, nous allons présenter l'outil de détection des anomalies dans les délais d'un lien  conçu dans le cadre de l'étude  de R. Fontugne et al. \cite{DBLP:journals/corr/FontugneAPB16}. Nous avons choisi ce travail qui exploite des données massives en vue d'évaluer quelques technologies du Big Data.

\section{Introduction}
Le travail de R. Fontugne \cite{DBLP:journals/corr/FontugneAPB16} et al. exploite une des mesures effectuées par les sondes Atlas: la requête traceroute. L'idée de ce travail est de collecter les résultats des requêtes traceroutes effectuées par les sondes Atlas, d'en déterminer une valeur de référence sur base de l'historique, et ensuite de comparer la référence avec la valeur courante. La référence pour le délai  d'un lien donné   est mise à jour au fur et à mesure de l'analyse.



Dans leur travail \cite{DBLP:journals/corr/FontugneAPB16},  R. Fontugne et al. ont exploité la  distribution répandue des sondes Atlas dans le monde afin d'étudier un des problèmes relatifs aux performances des réseaux informatiques. 

En pratique, il est  difficile  d'avoir une idée globale et exacte sur la topologie de l'Internet. Toutefois, les opérateurs des réseaux informatiques  disposent d'un aperçu de l'état des entités qui forment leurs réseaux, les relations entre ces entités ainsi que les éventuels problèmes. Avec la distribution abondante des sondes Atlas dans le monde en terme de type d'adressage : sondes Atlas supportant seulement l'adressage IPv4, d'autres qui supportent en plus l'adressage IPv6, en terme de  la diversité géographique, la diversité en terme d'ASs hébergeant les sondes Atlas, etc, il était  possible d'aborder  les délais dans les réseaux informatiques à travers de nouvelles approches, reposées sur des fondements statistiques. Parmi les points forts de l'analyse menée par R. Fontugne et al., c'était la possibilité de valider les   méthodes proposées avec des événements  marquants sur Internet.

Le travail de R. Fontugne et al. reprend trois méthodes basées sur les données collectées par les sondes Atlas. Chaque méthode reflète l'approche utilisée pour étudier les performances des réseaux informatiques. Ces méthodes sont les suivantes:

\begin{enumerate}
	\item la détection des changements des délais que subissent les liens intermédiaires dans les traceroutes; 
	
	\item la conception d'un modèle de forwording pour un routeur donné. Ce modèle  prédit l'acheminement du trafic afin d'identifier les routeurs  et les liens en panne dans le cas  d'un problème  de perte de paquets;
	
	\item la création d'un score par Système Autonome afin d'évaluer l'état de ce dernier.
	
\end{enumerate}

Dans la suite de ce travail, nous allons reprendre seulement la première méthode.  Il s'agit d'étudier le délai d'un lien topologique; c'est le délai entre deux routeurs adjacents sur Internet.



\section{Pourquoi analyser les délais des liens réseaux}

Il existe un bon nombre de sujets à traiter en exploitant les données collectées par les sondes Atlas. Quelques exemples de sujets ont été déjà présentés dans la section \ref{use-cases-atlas}. Notre choix de reprendre le travail de Fontugne et al. a été fait après avoir parcouru un ensemble de travaux basés sur le projet  RIPE Atlas. 

Pour certains travaux, il n'était pas possible  de les reprendre suite au manque de détails et de l'accès à  certaines ressources réseaux utilisées dans ces travaux \footnote{Par exemple, les informations du peering entre les Systèmes Autonomes, la géolocalisation des adresses IP, etc.}. Par exemple, afin d'étudier la censure dans un pays donné, il faut être au courant des spécificités de ce pays, les circonstances politiques et sociales, les opposants,  etc.  Un autre exemple concerne le sujet de la détection des détours du trafic local dans un pays.  Anant Shah  et al. ont travaillé sur la détection des détours \cite{anant-shah} où ils ont présenté  les contraintes relatives à ce sujet: la difficulté d'avoir une géolocalisation exacte d'une adresse IP d'une part et l'absence des informations de peering entre les ASs d'autre part. Ces dernières peuvent changer complètement les conclusions finales. 

Le travail sur lequel nous nous basons \cite{DBLP:journals/corr/FontugneAPB16} s'inscrit dans les travaux traitant les performances des réseaux. Dans la suite de ce document, ce travail sera utilisé comme travail de référence.  Ce travail a été choisi comme référence pour plusieurs raisons:


	
\paragraph{Les données utilisées}  Les auteurs de ce travail ont exploité des données déjà présentes dans la base de données du RIPE Atlas. Ainsi, il n'y a pas besoin de lancer des mesures qui nécessitent la possession d'assez de crédits \footnote{Voir la section \ref{credits-atlas}.}. De plus, les mesures intégrées montrent plus de stabilité par rapport aux mesures personnalisées. Les destinations des mesures intégrées sont prédéfinies, généralement ce sont des instances des serveurs DNS et des serveurs gérés par  RIPE NCC. Cependant, les mesures personnalisées peuvent concerner des destinations moins stables en terme de disponibilité.
	
\paragraph{La clarté du travail} La communauté RIPE Atlas est active en nombre de travaux publiés. Cependant, ces travaux reprennent seulement les résultats. Pour certains,  la méthodologie a été bien détaillée. Pour d'autres, la méthodologie adoptée était très brève. En ce qui concerne le  travail choisi, il  est bien détaillé.
	
\paragraph{La disponibilité du code source} La détection des anomalies proposée a été mise en pratique à travers un outil. Les code source de cet outil est disponible sur GitHub\cite{InternetHealthReport}. L'accès au code source de l'outil nous a permis de bien comprendre la démarche de la détection.

%Avoir le code source disponible sur GitHub est avantageux. L'objectif n'était pas de réécrire ce que les auteurs ont fait, mais plutôt le comprendre et éventuellement l'ajuster suivant les besoins.  De plus, l'onglet \textit{issues} du projet permet de  demander des détails aux contributeurs mais aussi cela permet de partager les réflexions.
	
\paragraph{La possibilité de la validation des résultats}  Comme il était cité dans le travail \cite{DBLP:journals/corr/FontugneAPB16}, les auteurs ont démontré la cohérence de l'outil de détection avec des éventements réels comme les  attaques DDOS.
	
	\begin{tcolorbox}
		Une attaque de type Distributed Denial of Service (\textbf{\textit{DDOS}}) vise la disponibilité d'un serveur en surchargeant ce dernier avec un trafic depuis différentes sources, d'où le terme \textit{Distributed}.
	\end{tcolorbox}
	

\section{L'étude des délais des liens } 

\subsection{Les données utilisées dans l'analyse des délais}~

La méthode conçue pour la détection des changements des délais se base sur des fondements statistiques. Ces derniers sont capables de montrer leurs performances si la taille des échantillons \footnote{L'échantillon de la métrique qui caractérise un lien : RTT différentiel.} considérés est grande.   Afin de surveiller un grand nombre de liens sur Internet, il faut avoir un grand nombre de sondes Atlas avec une certaine diversité.

Le travail de référence implique principalement les mesures de traceroutes. On distingue  deux catégories de mesures  utilisées :

\begin{itemize}
	\item \textit{builtin} : ce sont les traceroutes effectués par toutes les sondes Atlas vers les instances des  $13$ serveurs DNS racines. Les traceroutes sont effectués chaque $30$ minutes. En pratique, certains serveurs racines DNS déploient l'anycast. Au moment de la réalisation du travail de référence\footnote{Année $2017$.}, c'étaient des traceroutes vers $ 500 $ instances des serveurs DNS racines;
	\begin{tcolorbox}
		\textbf{DNS Anycast} est une solution   utilisée pour accélérer le fonctionnement  des serveurs DNS. Les serveurs DNS adoptant cette approche fournissent des temps de réponse plus courts, et ce partout dans le monde. Les requêtes en provenance de l'utilisateur sont redirigées vers un n\oe{}ud adéquat suivant un algorithme prédéfini. 
	\end{tcolorbox}
	
	\item \textit{anchoring} : ce sont les traceroutes effectués par environ $400$ sondes Atlas à destination de $189$ serveurs\footnote{Sondes Atlas ayant des fonctionnalités avancées.} et ce chaque $15$ minutes.
\end{itemize}

En ce qui concerne le nombre de  traceroutes analysés, le tableau \ref{tab:dataset} reprend plus de détails; le nombre de traceroutes ainsi que le nombre de sondes impliquées dans ces traceroutes, pour les deux types d'adressages (IPv4 et IPv6). 

\begin{table}[H]
	\centering
	\begin{tabular}{|l|l|l|}
		\hline
		& \textbf{Nombre de traceroute}s& \textbf{Nombre de sondes}\\ \hline
		IPv4		&$ 2.8 $ billion & $ 11,538 $\\ \hline
		IPv6	&	$ 1.2 $ billion & $ 4,30 $ \\ \hline
	\end{tabular}
	\caption{Récapitulatif des traceroutes utilisés dans le travail de référence }
	\label{tab:dataset}
\end{table}


Pour précision, l'étude des délais des liens ne concerne pas  les adresses privées, ainsi, le suivi des délais ne concerne pas les réseaux privés.  De plus, ce  suivi  se base sur les requêtes de type traceroute, et traceroute reprend une partie de la topologie de l'Internet. En effet, les liens considérés sont ceux topologiques et ne sont pas  les liens physiques. 

\subsection{Le principe de la détection des changements des délais} \label{principe-de-detection}
\paragraph{Définition du RTT différentiel }~

Avant de définir le RTT différentiel, soit la définition du RTT :
%Il est indispensable de présenter la définition du RTT  (Round Trip Time)  différentiel d'un lien avant de procéder à la description de l'algorithme de la détection des anomalies. 


\begin{tcolorbox}
	
	\textbf{ICMP} (Internet Control Message Protocol) est un protocole utilisé pour véhiculer des messages de contrôle sur Internet.
	
	\textbf{RTT} est obtenu en calculant la différence entre le timestamp associé à l'envoi du paquet sondé  et le timestamp associé à la réception de la réponse ICMP. C'est une métrique pour évaluer les performances d'un réseau en matière de temps de réponse. Les mesures du RTT sont fournies par les utilitaires traceroute et ping. En ce qui concerne traceroute,  ce dernier fournit les sauts impliqués dans le  chemin de forwarding, c'est le chemin parcouru par le trafic entre la source et la destination.  RTT inclut le temps de transmission, du quering et  du traitement. 
\end{tcolorbox}

La figure 	\ref{fig:rtt-differ} (a)  illustre le RTT entre la sonde P et les deux routeurs B et C. Le RTT différentiel  entre deux routeurs $B$ et $C$ adjacents, noté $\Delta_{PBC}$, est la différence entre le RTT entre la sonde $P$ et $B$ (bleu) d'une part, et le RTT entre la sonde $P$ et $C$ (rouge) dans la figure 	\ref{fig:rtt-differ} (b). 

\begin{align*}
\Delta_{PBC} &= RTT_{PC} - RTT_{PB} \\
&= \delta_{BC} + \delta_{CD} + \delta_{DA}  - \delta_{BA} \\
&= \delta_{BC} + \varepsilon_{PBC}
\end{align*}

où $\delta_{BC}$ est le délai du lien $BC$ et $\varepsilon_{PBC}$ est la différence entre les deux chemins de retour ($B$ vers $P$ et $C$ vers $P$). 
%La première composante dépend de l'état des routeurs $B$ et $C$. La deuxième composante dépend de la sonde $P$. L'analyse du RTT différentiel repose sur la variation de valeurs qu'il prend, au lieu des valeurs exactes, dans  le cas des valeurs exactes, elles peuvent dévier l'interprétation.
\begin{figure}[H]
	\centering
		\captionsetup{justification= centering}
	\includegraphics[width=0.7\linewidth]{illustrations/rtt-differ}
	\caption{(a) Le RTT entre la sonde P et les routeurs B et C. (b) La différence entre les  chemins de retour depuis les routeurs B et C vers la sonde P. Source : \cite{DBLP:journals/corr/FontugneAPB16}}
	\label{fig:rtt-differ}
\end{figure}

\paragraph{Le principe de la détection des changements des délais}

L'évolution du délai d'un lien est déduit de l'évolution de son RTT différentiel. Reprenons d'abord la formule du RTT différentiel du lien BC : 

 $\delta_{BC}$ + $\varepsilon_{PBC}$. 
 
 Supposons qu'on dispose d'un nombre $n$ de sondes Atlas P$_i$, $i$ $\in$ [$1$, $n$], telles que toutes les sondes ont un chemin de retour différent depuis B et depuis C.  En effet, les RTTs différentiels pour chacune des sondes Atlas $\Delta_{P{_i}BC}$ partagent la même composante $\delta_{BC}$.

On sait que les $n$ sondes Atlas sont indépendantes; le chemin de retour de chacune est indépendant, ainsi les valeurs des  $\varepsilon_{P_{i}BC}$ sont  indépendantes. L'indépendance de ces valeurs implique que la distribution $\Delta_{P_{i}BC}$ est estimé d'être stable au cours du temps si $\delta_{BC}$ est constant. Cependant, un changement significatif de la valeur de $\delta_{BC}$ influence les valeurs des RTTs différentiels. Dans ce cas, la distribution des RTTs différentiels changes. 

%Enfin, les changements des délais sont déduits des changements des RTTs différentiel qu'on peut les quantifier.

La détection des anomalies des délais repose sur un théorème très important en statistiques, c'est le théorème  central limite (TCL). Ce théorème  annonce que si on a une suite de variables aléatoires $X_i$ indépendantes ayant la même espérance $\mu$ et la même variance $\sigma^2$, la moyenne de ces variables aléatoires est une variable aléatoire qui suit une loi normale. 

%De manière générale, le théorème central limite explique la distribution des moyennes des échantillons. Ce théorème peut être appliquer aux différents lois. Par exemple la loi normale \footnote{Un exemple illustratif dans \ref{appendix:clt-exemple}.}, binomiale, etc. 



\section{L'évolution du RTT différentiel des liens}

L'évolution des RTTs différentiels est une des applications du principe décrit dans  \ref{principe-de-detection}. Pour un lien donné, le suivi est fait sur plusieurs  périodes. En plus du suivi du RTT différentiel, il y a aussi l'identification d'éventuelles anomalies pour ce lien.

\subsection{Description des paramètres de l'analyse des délais} \label{par:parametre-de-lanalyse}~

La détection des changements des délais nécessite l'ajustement d'un nombre de paramètres. La valeur de chaque paramètre est relative au  fondement utilisé théorique ou bien empirique, qui a été  justifié par les auteurs du travail de référence.   Ci-dessous les paramètres à ajuster afin d'obtenir l'évolution des RTTs différentiels d'un lien ainsi que les éventuelles anomalies.  

\textbf{start} : c'est la date de début de l'analyse. Ce sont les traceroutes effectués par les sondes Atlas à partir de cette date qui sont analysés.

\textbf{end} : c'est la date marquant la fin de l'analyse. Comme le paramètre \textit{start}, c'est la date des derniers traceroutes effectués par les sondes Atlas à considérer.

\textbf{timeWindow} :  ce paramètre est exprimé en seconde. La durée de l'analyse, qui est le temps écoulé entre \textit{start} et \textit{end}, est divisée sur des périodes de même taille : \textit{timeWindow}, c'est qui est illustré dans la Figure  \ref{fig:timing_tex}.

%illustre le contexte des trois paramètres \textit{start}, \textit{end} et \textit{timeWindow} avec les étapes principales.



\begin{figure}[h]
	\centering
	\captionsetup{justification=centering}
	% Graphic for TeX using PGF
% Title: /home/hayat/RipeAtlasTraceroutesAnalysis/report/illustrations/timing.dia
% Creator: Dia v0.97+git
% CreationDate: Thu Nov 29 15:39:51 2018
% For: hayat
% \usepackage{tikz}
% The following commands are not supported in PSTricks at present
% We define them conditionally, so when they are implemented,
% this pgf file will use them.
\ifx\du\undefined
  \newlength{\du}
\fi
\setlength{\du}{15\unitlength}
\begin{tikzpicture}[even odd rule]
\pgftransformxscale{1.000000}
\pgftransformyscale{-1.000000}
\definecolor{dialinecolor}{rgb}{0.000000, 0.000000, 0.000000}
\pgfsetstrokecolor{dialinecolor}
\pgfsetstrokeopacity{1.000000}
\definecolor{diafillcolor}{rgb}{1.000000, 1.000000, 1.000000}
\pgfsetfillcolor{diafillcolor}
\pgfsetfillopacity{1.000000}
\pgfsetlinewidth{0.100000\du}
\pgfsetdash{}{0pt}
\pgfsetbuttcap
{
\definecolor{diafillcolor}{rgb}{0.000000, 0.000000, 0.000000}
\pgfsetfillcolor{diafillcolor}
\pgfsetfillopacity{1.000000}
% was here!!!
}
\definecolor{dialinecolor}{rgb}{0.000000, 0.000000, 0.000000}
\pgfsetstrokecolor{dialinecolor}
\pgfsetstrokeopacity{1.000000}
\draw (34.600000\du,6.050000\du)--(40.550000\du,6.000000\du);
\pgfsetlinewidth{0.100000\du}
\pgfsetdash{}{0pt}
\pgfsetmiterjoin
\pgfsetbuttcap
\definecolor{dialinecolor}{rgb}{0.000000, 0.000000, 0.000000}
\pgfsetstrokecolor{dialinecolor}
\pgfsetstrokeopacity{1.000000}
\draw (35.097882\du,5.795807\du)--(35.102083\du,6.295790\du);
\pgfsetlinewidth{0.100000\du}
\pgfsetdash{}{0pt}
\pgfsetmiterjoin
\pgfsetbuttcap
\definecolor{dialinecolor}{rgb}{0.000000, 0.000000, 0.000000}
\pgfsetstrokecolor{dialinecolor}
\pgfsetstrokeopacity{1.000000}
\draw (40.052118\du,6.254193\du)--(40.047917\du,5.754210\du);
% setfont left to latex
\definecolor{dialinecolor}{rgb}{0.000000, 0.000000, 0.000000}
\pgfsetstrokecolor{dialinecolor}
\pgfsetstrokeopacity{1.000000}
\definecolor{diafillcolor}{rgb}{0.000000, 0.000000, 0.000000}
\pgfsetfillcolor{diafillcolor}
\pgfsetfillopacity{1.000000}
\node[anchor=base west,inner sep=0pt,outer sep=0pt,color=dialinecolor] at (14.800166\du,8.464650\du){start};
% setfont left to latex
\definecolor{dialinecolor}{rgb}{0.000000, 0.000000, 0.000000}
\pgfsetstrokecolor{dialinecolor}
\pgfsetstrokeopacity{1.000000}
\definecolor{diafillcolor}{rgb}{0.000000, 0.000000, 0.000000}
\pgfsetfillcolor{diafillcolor}
\pgfsetfillopacity{1.000000}
\node[anchor=base west,inner sep=0pt,outer sep=0pt,color=dialinecolor] at (39.375580\du,7.990892\du){end};
% setfont left to latex
\definecolor{dialinecolor}{rgb}{0.000000, 0.000000, 0.000000}
\pgfsetstrokecolor{dialinecolor}
\pgfsetstrokeopacity{1.000000}
\definecolor{diafillcolor}{rgb}{0.000000, 0.000000, 0.000000}
\pgfsetfillcolor{diafillcolor}
\pgfsetfillopacity{1.000000}
\node[anchor=base west,inner sep=0pt,outer sep=0pt,color=dialinecolor] at (20.674752\du,8.627312\du){Périodes de l'analyse (Unix time)};
% setfont left to latex
\definecolor{dialinecolor}{rgb}{0.000000, 0.000000, 0.000000}
\pgfsetstrokecolor{dialinecolor}
\pgfsetstrokeopacity{1.000000}
\definecolor{diafillcolor}{rgb}{0.000000, 0.000000, 0.000000}
\pgfsetfillcolor{diafillcolor}
\pgfsetfillopacity{1.000000}
\node[anchor=base west,inner sep=0pt,outer sep=0pt,color=dialinecolor] at (14.950000\du,7.200000\du){d1};
\pgfsetlinewidth{0.100000\du}
\pgfsetdash{}{0pt}
\pgfsetbuttcap
{
\definecolor{diafillcolor}{rgb}{0.000000, 0.000000, 0.000000}
\pgfsetfillcolor{diafillcolor}
\pgfsetfillopacity{1.000000}
% was here!!!
}
\definecolor{dialinecolor}{rgb}{0.000000, 0.000000, 0.000000}
\pgfsetstrokecolor{dialinecolor}
\pgfsetstrokeopacity{1.000000}
\draw (29.656799\du,6.066268\du)--(35.606799\du,6.016268\du);
\pgfsetlinewidth{0.100000\du}
\pgfsetdash{}{0pt}
\pgfsetmiterjoin
\pgfsetbuttcap
\definecolor{dialinecolor}{rgb}{0.000000, 0.000000, 0.000000}
\pgfsetstrokecolor{dialinecolor}
\pgfsetstrokeopacity{1.000000}
\draw (30.154681\du,5.812076\du)--(30.158883\du,6.312058\du);
\pgfsetlinewidth{0.100000\du}
\pgfsetdash{}{0pt}
\pgfsetmiterjoin
\pgfsetbuttcap
\definecolor{dialinecolor}{rgb}{0.000000, 0.000000, 0.000000}
\pgfsetstrokecolor{dialinecolor}
\pgfsetstrokeopacity{1.000000}
\draw (35.108918\du,6.270461\du)--(35.104716\du,5.770479\du);
\pgfsetlinewidth{0.100000\du}
\pgfsetdash{}{0pt}
\pgfsetbuttcap
{
\definecolor{diafillcolor}{rgb}{0.000000, 0.000000, 0.000000}
\pgfsetfillcolor{diafillcolor}
\pgfsetfillopacity{1.000000}
% was here!!!
}
\definecolor{dialinecolor}{rgb}{0.000000, 0.000000, 0.000000}
\pgfsetstrokecolor{dialinecolor}
\pgfsetstrokeopacity{1.000000}
\draw (24.701799\du,6.106103\du)--(30.651799\du,6.056103\du);
\pgfsetlinewidth{0.100000\du}
\pgfsetdash{}{0pt}
\pgfsetmiterjoin
\pgfsetbuttcap
\definecolor{dialinecolor}{rgb}{0.000000, 0.000000, 0.000000}
\pgfsetstrokecolor{dialinecolor}
\pgfsetstrokeopacity{1.000000}
\draw (25.199681\du,5.851910\du)--(25.203883\du,6.351892\du);
\pgfsetlinewidth{0.100000\du}
\pgfsetdash{}{0pt}
\pgfsetmiterjoin
\pgfsetbuttcap
\definecolor{dialinecolor}{rgb}{0.000000, 0.000000, 0.000000}
\pgfsetstrokecolor{dialinecolor}
\pgfsetstrokeopacity{1.000000}
\draw (30.153918\du,6.310295\du)--(30.149716\du,5.810313\du);
\pgfsetlinewidth{0.100000\du}
\pgfsetdash{}{0pt}
\pgfsetbuttcap
{
\definecolor{diafillcolor}{rgb}{0.000000, 0.000000, 0.000000}
\pgfsetfillcolor{diafillcolor}
\pgfsetfillopacity{1.000000}
% was here!!!
}
\definecolor{dialinecolor}{rgb}{0.000000, 0.000000, 0.000000}
\pgfsetstrokecolor{dialinecolor}
\pgfsetstrokeopacity{1.000000}
\draw (19.796634\du,6.121020\du)--(25.746634\du,6.071020\du);
\pgfsetlinewidth{0.100000\du}
\pgfsetdash{}{0pt}
\pgfsetmiterjoin
\pgfsetbuttcap
\definecolor{dialinecolor}{rgb}{0.000000, 0.000000, 0.000000}
\pgfsetstrokecolor{dialinecolor}
\pgfsetstrokeopacity{1.000000}
\draw (20.294515\du,5.866827\du)--(20.298717\du,6.366810\du);
\pgfsetlinewidth{0.100000\du}
\pgfsetdash{}{0pt}
\pgfsetmiterjoin
\pgfsetbuttcap
\definecolor{dialinecolor}{rgb}{0.000000, 0.000000, 0.000000}
\pgfsetstrokecolor{dialinecolor}
\pgfsetstrokeopacity{1.000000}
\draw (25.248752\du,6.325213\du)--(25.244551\du,5.825230\du);
\pgfsetlinewidth{0.100000\du}
\pgfsetdash{}{0pt}
\pgfsetbuttcap
{
\definecolor{diafillcolor}{rgb}{0.000000, 0.000000, 0.000000}
\pgfsetfillcolor{diafillcolor}
\pgfsetfillopacity{1.000000}
% was here!!!
}
\definecolor{dialinecolor}{rgb}{0.000000, 0.000000, 0.000000}
\pgfsetstrokecolor{dialinecolor}
\pgfsetstrokeopacity{1.000000}
\draw (14.841634\du,6.135937\du)--(20.791634\du,6.085937\du);
\pgfsetlinewidth{0.100000\du}
\pgfsetdash{}{0pt}
\pgfsetmiterjoin
\pgfsetbuttcap
\definecolor{dialinecolor}{rgb}{0.000000, 0.000000, 0.000000}
\pgfsetstrokecolor{dialinecolor}
\pgfsetstrokeopacity{1.000000}
\draw (15.339515\du,5.881744\du)--(15.343717\du,6.381727\du);
\pgfsetlinewidth{0.100000\du}
\pgfsetdash{}{0pt}
\pgfsetmiterjoin
\pgfsetbuttcap
\definecolor{dialinecolor}{rgb}{0.000000, 0.000000, 0.000000}
\pgfsetstrokecolor{dialinecolor}
\pgfsetstrokeopacity{1.000000}
\draw (20.293752\du,6.340130\du)--(20.289551\du,5.840147\du);
% setfont left to latex
\definecolor{dialinecolor}{rgb}{0.000000, 0.000000, 0.000000}
\pgfsetstrokecolor{dialinecolor}
\pgfsetstrokeopacity{1.000000}
\definecolor{diafillcolor}{rgb}{0.000000, 0.000000, 0.000000}
\pgfsetfillcolor{diafillcolor}
\pgfsetfillopacity{1.000000}
\node[anchor=base west,inner sep=0pt,outer sep=0pt,color=dialinecolor] at (19.845000\du,7.235000\du){d2};
% setfont left to latex
\definecolor{dialinecolor}{rgb}{0.000000, 0.000000, 0.000000}
\pgfsetstrokecolor{dialinecolor}
\pgfsetstrokeopacity{1.000000}
\definecolor{diafillcolor}{rgb}{0.000000, 0.000000, 0.000000}
\pgfsetfillcolor{diafillcolor}
\pgfsetfillopacity{1.000000}
\node[anchor=base west,inner sep=0pt,outer sep=0pt,color=dialinecolor] at (24.750000\du,7.300000\du){d3};
% setfont left to latex
\definecolor{dialinecolor}{rgb}{0.000000, 0.000000, 0.000000}
\pgfsetstrokecolor{dialinecolor}
\pgfsetstrokeopacity{1.000000}
\definecolor{diafillcolor}{rgb}{0.000000, 0.000000, 0.000000}
\pgfsetfillcolor{diafillcolor}
\pgfsetfillopacity{1.000000}
\node[anchor=base west,inner sep=0pt,outer sep=0pt,color=dialinecolor] at (29.740000\du,7.175000\du){di};
% setfont left to latex
\definecolor{dialinecolor}{rgb}{0.000000, 0.000000, 0.000000}
\pgfsetstrokecolor{dialinecolor}
\pgfsetstrokeopacity{1.000000}
\definecolor{diafillcolor}{rgb}{0.000000, 0.000000, 0.000000}
\pgfsetfillcolor{diafillcolor}
\pgfsetfillopacity{1.000000}
\node[anchor=base west,inner sep=0pt,outer sep=0pt,color=dialinecolor] at (34.685000\du,7.165000\du){di+1};
\pgfsetlinewidth{0.100000\du}
\pgfsetdash{}{0pt}
\pgfsetbuttcap
\definecolor{dialinecolor}{rgb}{0.000000, 0.000000, 0.000000}
\pgfsetstrokecolor{dialinecolor}
\pgfsetstrokeopacity{1.000000}
\pgfpathmoveto{\pgfpoint{25.180033\du}{4.900033\du}}
\pgfpatharc{315}{226}{3.501250\du and 3.501250\du}
\pgfusepath{stroke}
% setfont left to latex
\definecolor{dialinecolor}{rgb}{0.000000, 0.000000, 0.000000}
\pgfsetstrokecolor{dialinecolor}
\pgfsetstrokeopacity{1.000000}
\definecolor{diafillcolor}{rgb}{0.000000, 0.000000, 0.000000}
\pgfsetfillcolor{diafillcolor}
\pgfsetfillopacity{1.000000}
\node[anchor=base west,inner sep=0pt,outer sep=0pt,color=dialinecolor] at (20.780000\du,3.400000\du){timeWindow};
\pgfsetlinewidth{0.100000\du}
\pgfsetdash{}{0pt}
\pgfsetbuttcap
\definecolor{dialinecolor}{rgb}{0.000000, 0.000000, 0.000000}
\pgfsetstrokecolor{dialinecolor}
\pgfsetstrokeopacity{1.000000}
\pgfpathmoveto{\pgfpoint{29.975033\du}{4.885033\du}}
\pgfpatharc{315}{226}{3.501250\du and 3.501250\du}
\pgfusepath{stroke}
% setfont left to latex
\definecolor{dialinecolor}{rgb}{0.000000, 0.000000, 0.000000}
\pgfsetstrokecolor{dialinecolor}
\pgfsetstrokeopacity{1.000000}
\definecolor{diafillcolor}{rgb}{0.000000, 0.000000, 0.000000}
\pgfsetfillcolor{diafillcolor}
\pgfsetfillopacity{1.000000}
\node[anchor=base west,inner sep=0pt,outer sep=0pt,color=dialinecolor] at (25.675000\du,3.485000\du){timeWindow};
\pgfsetlinewidth{0.100000\du}
\pgfsetdash{}{0pt}
\pgfsetbuttcap
\definecolor{dialinecolor}{rgb}{0.000000, 0.000000, 0.000000}
\pgfsetstrokecolor{dialinecolor}
\pgfsetstrokeopacity{1.000000}
\pgfpathmoveto{\pgfpoint{20.275033\du}{4.785033\du}}
\pgfpatharc{315}{226}{3.501250\du and 3.501250\du}
\pgfusepath{stroke}
% setfont left to latex
\definecolor{dialinecolor}{rgb}{0.000000, 0.000000, 0.000000}
\pgfsetstrokecolor{dialinecolor}
\pgfsetstrokeopacity{1.000000}
\definecolor{diafillcolor}{rgb}{0.000000, 0.000000, 0.000000}
\pgfsetfillcolor{diafillcolor}
\pgfsetfillopacity{1.000000}
\node[anchor=base west,inner sep=0pt,outer sep=0pt,color=dialinecolor] at (16.025000\du,3.435000\du){timeWindow};
\end{tikzpicture}

	\caption{Illustration des périodes de l'analyse entre la date de début et la date de fin}
	\label{fig:timing_tex}
\end{figure}
%\begin{figure}[h]	
%	\centering
%	\resizebox{\textwidth}{!}{
%		% Graphic for TeX using PGF
% Title: /home/bellafkih/Documents/2018-2019/memoire/rapport_memoire/dia/timewindow.dia
% Creator: Dia v0.97.3
% CreationDate: Sun Oct  7 15:34:20 2018
% For: bellafkih
% \usepackage{tikz}
% The following commands are not supported in PSTricks at present
% We define them conditionally, so when they are implemented,
% this pgf file will use them.
\ifx\du\undefined
  \newlength{\du}
\fi
\setlength{\du}{15\unitlength}
\begin{tikzpicture}
\pgftransformxscale{1.000000}
\pgftransformyscale{-1.000000}
\definecolor{dialinecolor}{rgb}{0.000000, 0.000000, 0.000000}
\pgfsetstrokecolor{dialinecolor}
\definecolor{dialinecolor}{rgb}{1.000000, 1.000000, 1.000000}
\pgfsetfillcolor{dialinecolor}
\pgfsetlinewidth{0.200000\du}
\pgfsetdash{}{0pt}
\pgfsetdash{}{0pt}
\pgfsetbuttcap
{
\definecolor{dialinecolor}{rgb}{0.000000, 0.000000, 0.000000}
\pgfsetfillcolor{dialinecolor}
% was here!!!
\pgfsetarrowsend{to}
\definecolor{dialinecolor}{rgb}{0.000000, 0.000000, 0.000000}
\pgfsetstrokecolor{dialinecolor}
\draw (10.000000\du,2.000000\du)--(10.000000\du,26.650000\du);
}
% setfont left to latex
\definecolor{dialinecolor}{rgb}{0.000000, 0.000000, 0.000000}
\pgfsetstrokecolor{dialinecolor}
\node[anchor=west] at (7.650000\du,2.050000\du){start};
% setfont left to latex
\definecolor{dialinecolor}{rgb}{0.000000, 0.000000, 0.000000}
\pgfsetstrokecolor{dialinecolor}
\node[anchor=west] at (7.850000\du,22.100000\du){end};
\pgfsetlinewidth{0.000000\du}
\pgfsetdash{}{0pt}
\pgfsetdash{}{0pt}
\pgfsetbuttcap
{
\definecolor{dialinecolor}{rgb}{0.000000, 0.000000, 0.000000}
\pgfsetfillcolor{dialinecolor}
% was here!!!
\definecolor{dialinecolor}{rgb}{0.000000, 0.000000, 0.000000}
\pgfsetstrokecolor{dialinecolor}
\draw (10.600000\du,2.000000\du)--(9.450000\du,2.000000\du);
}
\pgfsetlinewidth{0.000000\du}
\pgfsetdash{}{0pt}
\pgfsetdash{}{0pt}
\pgfsetbuttcap
{
\definecolor{dialinecolor}{rgb}{0.000000, 0.000000, 0.000000}
\pgfsetfillcolor{dialinecolor}
% was here!!!
\definecolor{dialinecolor}{rgb}{0.000000, 0.000000, 0.000000}
\pgfsetstrokecolor{dialinecolor}
\draw (10.585000\du,5.960000\du)--(9.435000\du,5.960000\du);
}
\pgfsetlinewidth{0.000000\du}
\pgfsetdash{}{0pt}
\pgfsetdash{}{0pt}
\pgfsetbuttcap
{
\definecolor{dialinecolor}{rgb}{0.000000, 0.000000, 0.000000}
\pgfsetfillcolor{dialinecolor}
% was here!!!
\definecolor{dialinecolor}{rgb}{0.000000, 0.000000, 0.000000}
\pgfsetstrokecolor{dialinecolor}
\draw (10.570000\du,10.020000\du)--(9.420000\du,10.020000\du);
}
\pgfsetlinewidth{0.000000\du}
\pgfsetdash{}{0pt}
\pgfsetdash{}{0pt}
\pgfsetbuttcap
{
\definecolor{dialinecolor}{rgb}{0.000000, 0.000000, 0.000000}
\pgfsetfillcolor{dialinecolor}
% was here!!!
\definecolor{dialinecolor}{rgb}{0.000000, 0.000000, 0.000000}
\pgfsetstrokecolor{dialinecolor}
\draw (10.505000\du,13.930000\du)--(9.355000\du,13.930000\du);
}
\pgfsetlinewidth{0.000000\du}
\pgfsetdash{}{0pt}
\pgfsetdash{}{0pt}
\pgfsetbuttcap
{
\definecolor{dialinecolor}{rgb}{0.000000, 0.000000, 0.000000}
\pgfsetfillcolor{dialinecolor}
% was here!!!
\definecolor{dialinecolor}{rgb}{0.000000, 0.000000, 0.000000}
\pgfsetstrokecolor{dialinecolor}
\draw (10.540000\du,17.940000\du)--(9.390000\du,17.940000\du);
}
\pgfsetlinewidth{0.000000\du}
\pgfsetdash{}{0pt}
\pgfsetdash{}{0pt}
\pgfsetbuttcap
{
\definecolor{dialinecolor}{rgb}{0.000000, 0.000000, 0.000000}
\pgfsetfillcolor{dialinecolor}
% was here!!!
\definecolor{dialinecolor}{rgb}{0.000000, 0.000000, 0.000000}
\pgfsetstrokecolor{dialinecolor}
\draw (10.575000\du,22.050000\du)--(9.425000\du,22.050000\du);
}
\pgfsetlinewidth{0.000000\du}
\pgfsetdash{}{0pt}
\pgfsetdash{}{0pt}
\pgfsetbuttcap
{
\definecolor{dialinecolor}{rgb}{0.000000, 0.000000, 0.000000}
\pgfsetfillcolor{dialinecolor}
% was here!!!
\pgfsetarrowsstart{to}
\pgfsetarrowsend{to}
\definecolor{dialinecolor}{rgb}{0.000000, 0.000000, 0.000000}
\pgfsetstrokecolor{dialinecolor}
\pgfpathmoveto{\pgfpoint{6.950071\du}{2.049940\du}}
\pgfpatharc{231}{129}{2.448949\du and 2.448949\du}
\pgfusepath{stroke}
}
% setfont left to latex
\definecolor{dialinecolor}{rgb}{0.000000, 0.000000, 0.000000}
\pgfsetstrokecolor{dialinecolor}
\node[anchor=west] at (2.150000\du,26.900000\du){Temps en secondes (s)};
\pgfsetlinewidth{0.000000\du}
\pgfsetdash{}{0pt}
\pgfsetdash{}{0pt}
\pgfsetbuttcap
{
\definecolor{dialinecolor}{rgb}{0.000000, 0.000000, 0.000000}
\pgfsetfillcolor{dialinecolor}
% was here!!!
\pgfsetarrowsstart{to}
\pgfsetarrowsend{to}
\definecolor{dialinecolor}{rgb}{0.000000, 0.000000, 0.000000}
\pgfsetstrokecolor{dialinecolor}
\pgfpathmoveto{\pgfpoint{7.049251\du}{6.119940\du}}
\pgfpatharc{231}{129}{2.448949\du and 2.448949\du}
\pgfusepath{stroke}
}
\pgfsetlinewidth{0.000000\du}
\pgfsetdash{}{0pt}
\pgfsetdash{}{0pt}
\pgfsetbuttcap
{
\definecolor{dialinecolor}{rgb}{0.000000, 0.000000, 0.000000}
\pgfsetfillcolor{dialinecolor}
% was here!!!
\pgfsetarrowsstart{to}
\pgfsetarrowsend{to}
\definecolor{dialinecolor}{rgb}{0.000000, 0.000000, 0.000000}
\pgfsetstrokecolor{dialinecolor}
\pgfpathmoveto{\pgfpoint{6.984251\du}{10.079940\du}}
\pgfpatharc{231}{129}{2.448949\du and 2.448949\du}
\pgfusepath{stroke}
}
\pgfsetlinewidth{0.000000\du}
\pgfsetdash{}{0pt}
\pgfsetdash{}{0pt}
\pgfsetbuttcap
{
\definecolor{dialinecolor}{rgb}{0.000000, 0.000000, 0.000000}
\pgfsetfillcolor{dialinecolor}
% was here!!!
\pgfsetarrowsstart{to}
\pgfsetarrowsend{to}
\definecolor{dialinecolor}{rgb}{0.000000, 0.000000, 0.000000}
\pgfsetstrokecolor{dialinecolor}
\pgfpathmoveto{\pgfpoint{6.869251\du}{14.089940\du}}
\pgfpatharc{231}{129}{2.448949\du and 2.448949\du}
\pgfusepath{stroke}
}
\pgfsetlinewidth{0.000000\du}
\pgfsetdash{}{0pt}
\pgfsetdash{}{0pt}
\pgfsetbuttcap
{
\definecolor{dialinecolor}{rgb}{0.000000, 0.000000, 0.000000}
\pgfsetfillcolor{dialinecolor}
% was here!!!
\pgfsetarrowsstart{to}
\pgfsetarrowsend{to}
\definecolor{dialinecolor}{rgb}{0.000000, 0.000000, 0.000000}
\pgfsetstrokecolor{dialinecolor}
\pgfpathmoveto{\pgfpoint{6.954251\du}{18.099940\du}}
\pgfpatharc{231}{129}{2.448949\du and 2.448949\du}
\pgfusepath{stroke}
}
% setfont left to latex
\definecolor{dialinecolor}{rgb}{0.000000, 0.000000, 0.000000}
\pgfsetstrokecolor{dialinecolor}
\node[anchor=west] at (0.650000\du,4.050000\du){timeWindow s};
% setfont left to latex
\definecolor{dialinecolor}{rgb}{0.000000, 0.000000, 0.000000}
\pgfsetstrokecolor{dialinecolor}
\node[anchor=west] at (0.685100\du,8.005000\du){timeWindow s};
% setfont left to latex
\definecolor{dialinecolor}{rgb}{0.000000, 0.000000, 0.000000}
\pgfsetstrokecolor{dialinecolor}
\node[anchor=west] at (0.870100\du,12.015000\du){timeWindow s};
% setfont left to latex
\definecolor{dialinecolor}{rgb}{0.000000, 0.000000, 0.000000}
\pgfsetstrokecolor{dialinecolor}
\node[anchor=west] at (0.905100\du,16.025000\du){timeWindow s};
% setfont left to latex
\definecolor{dialinecolor}{rgb}{0.000000, 0.000000, 0.000000}
\pgfsetstrokecolor{dialinecolor}
\node[anchor=west] at (1.040100\du,19.935000\du){timeWindow s};
% setfont left to latex
\definecolor{dialinecolor}{rgb}{0.000000, 0.000000, 0.000000}
\pgfsetstrokecolor{dialinecolor}
\node[anchor=west] at (11.200100\du,3.850000\du){appliquer : computeRtt(), mergeRttResults(), outlierDetection()};
% setfont left to latex
\definecolor{dialinecolor}{rgb}{0.000000, 0.000000, 0.000000}
\pgfsetstrokecolor{dialinecolor}
\node[anchor=west] at (11.035100\du,7.905000\du){appliquer : computeRtt(), mergeRttResults(), outlierDetection()};
% setfont left to latex
\definecolor{dialinecolor}{rgb}{0.000000, 0.000000, 0.000000}
\pgfsetstrokecolor{dialinecolor}
\node[anchor=west] at (11.120100\du,12.015000\du){appliquer : computeRtt(), mergeRttResults(), outlierDetection()};
% setfont left to latex
\definecolor{dialinecolor}{rgb}{0.000000, 0.000000, 0.000000}
\pgfsetstrokecolor{dialinecolor}
\node[anchor=west] at (11.105100\du,15.875000\du){appliquer : computeRtt(), mergeRttResults(), outlierDetection()};
% setfont left to latex
\definecolor{dialinecolor}{rgb}{0.000000, 0.000000, 0.000000}
\pgfsetstrokecolor{dialinecolor}
\node[anchor=west] at (11.040100\du,19.885000\du){appliquer : computeRtt(), mergeRttResults(), outlierDetection()};
% setfont left to latex
\definecolor{dialinecolor}{rgb}{0.000000, 0.000000, 0.000000}
\pgfsetstrokecolor{dialinecolor}
\node[anchor=west] at (11.950100\du,2.000000\du){inputs : paramère de l'expérience};
% setfont left to latex
\definecolor{dialinecolor}{rgb}{0.000000, 0.000000, 0.000000}
\pgfsetstrokecolor{dialinecolor}
\node[anchor=west] at (12.050100\du,22.000000\du){output : les changements détéctés et ses caractériqtiques};
\end{tikzpicture}
 
%	}
%	\caption{Illustration du paramètre timeWindow}
%	\label{fig:timewindow}
%\end{figure}

\textbf{minSeen} : comme l'analyse est faite sur plusieurs périodes de durée \textit{timeWindow}, le paramètre \textit{minSeen} indique le nombre de fois où le lien doit avoir été identifié. Par exemple, un lien peut être identifié dans $3$ $d_i$, ou bien être identifié  une seule fois durant toute la période de l'analyse.

\textbf{alpha }: noté $\alpha$, c'est le paramètre de la  moyenne mobile exponentielle calculée.

\guillemotleft \textit{ Les méthodes de lissage exponentiel  sont un ensemble de techniques empiriques de prévision qui accordent plus ou moins d'importance aux valeurs du passé d'une série temporelle.\footnote{Source : \url{https://perso.math.univ-toulouse.fr/lagnoux/files/2013/12/Chap6.pdf}, consultée le $30/09/2018.$}} \guillemotright

 La  moyenne mobile exponentielle est utilisée pour calculer la médiane des RTTs différentiels durant une période $d_i$. Cette médiane constitue une référence  sur laquelle se base la détection des anomalies.
Pour calculer la prochaine  valeur de la médiane des RTTs différentiels de référence $ \overline{m}_{t}$   courant la période $ t $ et pour le lien $l$, soient: 

$m_t$ la médiane des RTTs différentiels observée pour $l$ durant le \textit{timeWindow} $t$. 

$ \overline{m}_{t-1}$  la médiane des  RTTs différentiels  de référence durant le \textit{timeWindow} $ t-1 $.  



Pour précision, \{$m_t$\} et \{$ \overline{m}_{t}$\} désignent deux ensembles différents. Le premier est l'ensemble est la médiane des RTTs différentiels de chaque période $d_i$. Toutefois, le deuxième est l'ensemble des médianes des RTTs différentiels construites en utilisant la méthode de la moyenne mobile exponentielle. Le calcul de cette dernière prend en compte les médianes des RTTs différentiels précédentes ainsi que la médiane des RTTs différentiels courante. La participation de ces dernières dans le calcul de la référence est dirigé par le paramètre $\alpha$.

La  valeur de la médiane de référence  $ \overline{m}_{t}$ de la période $t$ est obtenue par : 

\begin{center}
	$ \overline{m}_{t}$ =  $\alpha$ ${m}_{t}$ + (1-  $\alpha$) $ \overline{m}_{t-1}$
\end{center}

Le paramètre  $\alpha \in (0, 1)$  contrôle l'importance  des mesures précédentes par rapport aux mesures récentes. De ce fait, \guillemotleft \textit{plus $\alpha$ est proche de $ 1 $ plus les observations récentes influent sur la prévision, à l'inverse un $\alpha$ proche de $0$ conduit à une prévision très stable prenant en compte un passé lointain\footnote{Source : \url{https://www.math.u-psud.fr/~goude/Materials/time_series/cours3_lissage_expo.pdf}, consultée le $30/09/2018$.}}.\guillemotright.  Dans la présente étude, le paramètre $\alpha$ est préféré d'être petit, précisément, il prend par défaut $0.05$ comme valeur.

\subsection{L'évolution du RTT différentiel d'un lien et  la détection des anomalies} \label{rttevolution}


L'entrée de l'algorithme de détection est un ensemble de traceroutes. Un traceroute  est un ensemble de sauts auxquels sont jointes l'identifiant de la sonde ayant effectué la requête traceroute et la destination de la requête. Chaque saut est décrit par un ensemble de signaux.  Chaque signal décrit le routeur ayant émis une réponse à la sonde parmi les routeurs traversés avant d'atteindre la destination finale.  Pour le saut $i$, on note trois signaux $S_{i, j}; j\in [1,3]$ dont le routeur émettant le signal est $from_{i,j}$ et le RTT est égal à $rtt_{i,j}$, avec $j \in [1,3]$. C'est ce que   illustre dans la Figure \ref{fig:traceroute}.

\begin{figure}[H]
	\centering
	\captionsetup{justification=centering}
	\resizebox{\textwidth}{!}{
	% Graphic for TeX using PGF
% Title: /home/bellafkih/Documents/2018-2019/memoire/rapport_memoire/dia/traceroute.dia
% Creator: Dia v0.97.3
% CreationDate: Sun Sep 23 22:12:05 2018
% For: bellafkih
% \usepackage{tikz}
% The following commands are not supported in PSTricks at present
% We define them conditionally, so when they are implemented,
% this pgf file will use them.
\ifx\du\undefined
  \newlength{\du}
\fi
\setlength{\du}{15\unitlength}
\begin{tikzpicture}
\pgftransformxscale{1.000000}
\pgftransformyscale{-1.000000}
\definecolor{dialinecolor}{rgb}{0.000000, 0.000000, 0.000000}
\pgfsetstrokecolor{dialinecolor}
\definecolor{dialinecolor}{rgb}{1.000000, 1.000000, 1.000000}
\pgfsetfillcolor{dialinecolor}
\definecolor{dialinecolor}{rgb}{0.847059, 0.898039, 0.898039}
\pgfsetfillcolor{dialinecolor}
\fill (6.953088\du,8.550000\du)--(8.629264\du,8.550000\du)--(7.974118\du,10.350000\du)--(6.297942\du,10.350000\du)--cycle;
\pgfsetlinewidth{0.000000\du}
\pgfsetdash{}{0pt}
\pgfsetdash{}{0pt}
\pgfsetmiterjoin
\definecolor{dialinecolor}{rgb}{0.000000, 0.000000, 0.000000}
\pgfsetstrokecolor{dialinecolor}
\draw (6.953088\du,8.550000\du)--(8.629264\du,8.550000\du)--(7.974118\du,10.350000\du)--(6.297942\du,10.350000\du)--cycle;
% setfont left to latex
\definecolor{dialinecolor}{rgb}{0.000000, 0.000000, 0.000000}
\pgfsetstrokecolor{dialinecolor}
\node at (7.463603\du,9.645000\du){};
\pgfsetlinewidth{0.000000\du}
\pgfsetdash{}{0pt}
\pgfsetdash{}{0pt}
\pgfsetbuttcap
\pgfsetmiterjoin
\pgfsetlinewidth{0.000000\du}
\pgfsetbuttcap
\pgfsetmiterjoin
\pgfsetdash{}{0pt}
\definecolor{dialinecolor}{rgb}{0.027451, 0.486275, 0.682353}
\pgfsetfillcolor{dialinecolor}
\pgfpathmoveto{\pgfpoint{18.279993\du}{10.312559\du}}
\pgfpathlineto{\pgfpoint{18.278532\du}{10.341767\du}}
\pgfpathlineto{\pgfpoint{18.271230\du}{10.371705\du}}
\pgfpathlineto{\pgfpoint{18.261008\du}{10.400183\du}}
\pgfpathlineto{\pgfpoint{18.246404\du}{10.428295\du}}
\pgfpathlineto{\pgfpoint{18.227054\du}{10.456407\du}}
\pgfpathlineto{\pgfpoint{18.205148\du}{10.483790\du}}
\pgfpathlineto{\pgfpoint{18.178496\du}{10.510807\du}}
\pgfpathlineto{\pgfpoint{18.147828\du}{10.537094\du}}
\pgfpathlineto{\pgfpoint{18.114969\du}{10.562286\du}}
\pgfpathlineto{\pgfpoint{18.077729\du}{10.587477\du}}
\pgfpathlineto{\pgfpoint{18.036838\du}{10.611574\du}}
\pgfpathlineto{\pgfpoint{17.993027\du}{10.634940\du}}
\pgfpathlineto{\pgfpoint{17.946659\du}{10.657576\du}}
\pgfpathlineto{\pgfpoint{17.896276\du}{10.679482\du}}
\pgfpathlineto{\pgfpoint{17.843337\du}{10.700292\du}}
\pgfpathlineto{\pgfpoint{17.787842\du}{10.720372\du}}
\pgfpathlineto{\pgfpoint{17.729792\du}{10.739723\du}}
\pgfpathlineto{\pgfpoint{17.669186\du}{10.757612\du}}
\pgfpathlineto{\pgfpoint{17.605294\du}{10.774772\du}}
\pgfpathlineto{\pgfpoint{17.540307\du}{10.791201\du}}
\pgfpathlineto{\pgfpoint{17.471668\du}{10.806170\du}}
\pgfpathlineto{\pgfpoint{17.400840\du}{10.819679\du}}
\pgfpathlineto{\pgfpoint{17.328916\du}{10.832457\du}}
\pgfpathlineto{\pgfpoint{17.254071\du}{10.844505\du}}
\pgfpathlineto{\pgfpoint{17.178131\du}{10.854363\du}}
\pgfpathlineto{\pgfpoint{17.099635\du}{10.863490\du}}
\pgfpathlineto{\pgfpoint{17.020044\du}{10.871157\du}}
\pgfpathlineto{\pgfpoint{16.938992\du}{10.877729\du}}
\pgfpathlineto{\pgfpoint{16.855750\du}{10.882840\du}}
\pgfpathlineto{\pgfpoint{16.772143\du}{10.886491\du}}
\pgfpathlineto{\pgfpoint{16.686710\du}{10.888682\du}}
\pgfpathlineto{\pgfpoint{16.600183\du}{10.889412\du}}
\pgfpathlineto{\pgfpoint{16.514020\du}{10.888682\du}}
\pgfpathlineto{\pgfpoint{16.428222\du}{10.886491\du}}
\pgfpathlineto{\pgfpoint{16.344615\du}{10.882840\du}}
\pgfpathlineto{\pgfpoint{16.261738\du}{10.877729\du}}
\pgfpathlineto{\pgfpoint{16.180321\du}{10.871157\du}}
\pgfpathlineto{\pgfpoint{16.100730\du}{10.863490\du}}
\pgfpathlineto{\pgfpoint{16.022965\du}{10.854363\du}}
\pgfpathlineto{\pgfpoint{15.946294\du}{10.844505\du}}
\pgfpathlineto{\pgfpoint{15.872180\du}{10.832457\du}}
\pgfpathlineto{\pgfpoint{15.799525\du}{10.819679\du}}
\pgfpathlineto{\pgfpoint{15.729062\du}{10.806170\du}}
\pgfpathlineto{\pgfpoint{15.660424\du}{10.791201\du}}
\pgfpathlineto{\pgfpoint{15.594706\du}{10.774772\du}}
\pgfpathlineto{\pgfpoint{15.531179\du}{10.757612\du}}
\pgfpathlineto{\pgfpoint{15.470208\du}{10.739723\du}}
\pgfpathlineto{\pgfpoint{15.411793\du}{10.720372\du}}
\pgfpathlineto{\pgfpoint{15.356663\du}{10.700292\du}}
\pgfpathlineto{\pgfpoint{15.303724\du}{10.679482\du}}
\pgfpathlineto{\pgfpoint{15.253706\du}{10.657576\du}}
\pgfpathlineto{\pgfpoint{15.206608\du}{10.634940\du}}
\pgfpathlineto{\pgfpoint{15.163162\du}{10.611574\du}}
\pgfpathlineto{\pgfpoint{15.122271\du}{10.587477\du}}
\pgfpathlineto{\pgfpoint{15.085031\du}{10.562286\du}}
\pgfpathlineto{\pgfpoint{15.051807\du}{10.537094\du}}
\pgfpathlineto{\pgfpoint{15.021504\du}{10.510807\du}}
\pgfpathlineto{\pgfpoint{14.994852\du}{10.483790\du}}
\pgfpathlineto{\pgfpoint{14.972946\du}{10.456407\du}}
\pgfpathlineto{\pgfpoint{14.953596\du}{10.428295\du}}
\pgfpathlineto{\pgfpoint{14.938992\du}{10.400183\du}}
\pgfpathlineto{\pgfpoint{14.928405\du}{10.371705\du}}
\pgfpathlineto{\pgfpoint{14.921468\du}{10.341767\du}}
\pgfpathlineto{\pgfpoint{14.919642\du}{10.312559\du}}
\pgfpathlineto{\pgfpoint{14.921468\du}{10.282621\du}}
\pgfpathlineto{\pgfpoint{14.928405\du}{10.253414\du}}
\pgfpathlineto{\pgfpoint{14.938992\du}{10.224206\du}}
\pgfpathlineto{\pgfpoint{14.953596\du}{10.196093\du}}
\pgfpathlineto{\pgfpoint{14.972946\du}{10.167981\du}}
\pgfpathlineto{\pgfpoint{14.994852\du}{10.140599\du}}
\pgfpathlineto{\pgfpoint{15.021504\du}{10.113947\du}}
\pgfpathlineto{\pgfpoint{15.051807\du}{10.087660\du}}
\pgfpathlineto{\pgfpoint{15.085031\du}{10.062103\du}}
\pgfpathlineto{\pgfpoint{15.122271\du}{10.037276\du}}
\pgfpathlineto{\pgfpoint{15.163162\du}{10.012815\du}}
\pgfpathlineto{\pgfpoint{15.206608\du}{9.989449\du}}
\pgfpathlineto{\pgfpoint{15.253706\du}{9.967178\du}}
\pgfpathlineto{\pgfpoint{15.303724\du}{9.944907\du}}
\pgfpathlineto{\pgfpoint{15.356663\du}{9.924096\du}}
\pgfpathlineto{\pgfpoint{15.411793\du}{9.904016\du}}
\pgfpathlineto{\pgfpoint{15.470208\du}{9.885396\du}}
\pgfpathlineto{\pgfpoint{15.531179\du}{9.866411\du}}
\pgfpathlineto{\pgfpoint{15.594706\du}{9.849617\du}}
\pgfpathlineto{\pgfpoint{15.660424\du}{9.833917\du}}
\pgfpathlineto{\pgfpoint{15.729062\du}{9.818583\du}}
\pgfpathlineto{\pgfpoint{15.799525\du}{9.804710\du}}
\pgfpathlineto{\pgfpoint{15.872180\du}{9.791566\du}}
\pgfpathlineto{\pgfpoint{15.946294\du}{9.780613\du}}
\pgfpathlineto{\pgfpoint{16.022965\du}{9.770026\du}}
\pgfpathlineto{\pgfpoint{16.100730\du}{9.760533\du}}
\pgfpathlineto{\pgfpoint{16.180321\du}{9.753231\du}}
\pgfpathlineto{\pgfpoint{16.261738\du}{9.746659\du}}
\pgfpathlineto{\pgfpoint{16.344615\du}{9.741548\du}}
\pgfpathlineto{\pgfpoint{16.428222\du}{9.737897\du}}
\pgfpathlineto{\pgfpoint{16.514020\du}{9.736072\du}}
\pgfpathlineto{\pgfpoint{16.600183\du}{9.734976\du}}
\pgfpathlineto{\pgfpoint{16.686710\du}{9.736072\du}}
\pgfpathlineto{\pgfpoint{16.772143\du}{9.737897\du}}
\pgfpathlineto{\pgfpoint{16.855750\du}{9.741548\du}}
\pgfpathlineto{\pgfpoint{16.938992\du}{9.746659\du}}
\pgfpathlineto{\pgfpoint{17.020044\du}{9.753231\du}}
\pgfpathlineto{\pgfpoint{17.099635\du}{9.760533\du}}
\pgfpathlineto{\pgfpoint{17.178131\du}{9.770026\du}}
\pgfpathlineto{\pgfpoint{17.254071\du}{9.780613\du}}
\pgfpathlineto{\pgfpoint{17.328916\du}{9.791566\du}}
\pgfpathlineto{\pgfpoint{17.400840\du}{9.804710\du}}
\pgfpathlineto{\pgfpoint{17.471668\du}{9.818583\du}}
\pgfpathlineto{\pgfpoint{17.540307\du}{9.833917\du}}
\pgfpathlineto{\pgfpoint{17.605294\du}{9.849617\du}}
\pgfpathlineto{\pgfpoint{17.669186\du}{9.866411\du}}
\pgfpathlineto{\pgfpoint{17.729792\du}{9.885396\du}}
\pgfpathlineto{\pgfpoint{17.787842\du}{9.904016\du}}
\pgfpathlineto{\pgfpoint{17.843337\du}{9.924096\du}}
\pgfpathlineto{\pgfpoint{17.896276\du}{9.944907\du}}
\pgfpathlineto{\pgfpoint{17.946659\du}{9.967178\du}}
\pgfpathlineto{\pgfpoint{17.993027\du}{9.989449\du}}
\pgfpathlineto{\pgfpoint{18.036838\du}{10.012815\du}}
\pgfpathlineto{\pgfpoint{18.077729\du}{10.037276\du}}
\pgfpathlineto{\pgfpoint{18.114969\du}{10.062103\du}}
\pgfpathlineto{\pgfpoint{18.147828\du}{10.087660\du}}
\pgfpathlineto{\pgfpoint{18.178496\du}{10.113947\du}}
\pgfpathlineto{\pgfpoint{18.205148\du}{10.140599\du}}
\pgfpathlineto{\pgfpoint{18.227054\du}{10.167981\du}}
\pgfpathlineto{\pgfpoint{18.246404\du}{10.196093\du}}
\pgfpathlineto{\pgfpoint{18.261008\du}{10.224206\du}}
\pgfpathlineto{\pgfpoint{18.271230\du}{10.253414\du}}
\pgfpathlineto{\pgfpoint{18.278532\du}{10.282621\du}}
\pgfpathlineto{\pgfpoint{18.279993\du}{10.312559\du}}
\pgfusepath{fill}
\pgfsetlinewidth{0.000000\du}
\pgfsetbuttcap
\pgfsetmiterjoin
\pgfsetdash{}{0pt}
\definecolor{dialinecolor}{rgb}{0.678431, 0.839216, 0.905882}
\pgfsetfillcolor{dialinecolor}
\pgfpathmoveto{\pgfpoint{16.600183\du}{10.900000\du}}
\pgfpathlineto{\pgfpoint{16.600183\du}{10.900000\du}}
\pgfpathlineto{\pgfpoint{16.643629\du}{10.900000\du}}
\pgfpathlineto{\pgfpoint{16.687076\du}{10.899270\du}}
\pgfpathlineto{\pgfpoint{16.730157\du}{10.898175\du}}
\pgfpathlineto{\pgfpoint{16.772143\du}{10.897079\du}}
\pgfpathlineto{\pgfpoint{16.814859\du}{10.895254\du}}
\pgfpathlineto{\pgfpoint{16.856480\du}{10.893063\du}}
\pgfpathlineto{\pgfpoint{16.898101\du}{10.890507\du}}
\pgfpathlineto{\pgfpoint{16.939723\du}{10.888317\du}}
\pgfpathlineto{\pgfpoint{16.980248\du}{10.885396\du}}
\pgfpathlineto{\pgfpoint{17.021139\du}{10.881745\du}}
\pgfpathlineto{\pgfpoint{17.060935\du}{10.877729\du}}
\pgfpathlineto{\pgfpoint{17.101095\du}{10.873713\du}}
\pgfpathlineto{\pgfpoint{17.139796\du}{10.869332\du}}
\pgfpathlineto{\pgfpoint{17.179226\du}{10.864951\du}}
\pgfpathlineto{\pgfpoint{17.217196\du}{10.859474\du}}
\pgfpathlineto{\pgfpoint{17.256261\du}{10.854363\du}}
\pgfpathlineto{\pgfpoint{17.293501\du}{10.848886\du}}
\pgfpathlineto{\pgfpoint{17.330376\du}{10.842680\du}}
\pgfpathlineto{\pgfpoint{17.366886\du}{10.836838\du}}
\pgfpathlineto{\pgfpoint{17.403395\du}{10.830267\du}}
\pgfpathlineto{\pgfpoint{17.438810\du}{10.823330\du}}
\pgfpathlineto{\pgfpoint{17.473494\du}{10.816393\du}}
\pgfpathlineto{\pgfpoint{17.508178\du}{10.808726\du}}
\pgfpathlineto{\pgfpoint{17.542132\du}{10.801059\du}}
\pgfpathlineto{\pgfpoint{17.575721\du}{10.792662\du}}
\pgfpathlineto{\pgfpoint{17.608580\du}{10.784629\du}}
\pgfpathlineto{\pgfpoint{17.640343\du}{10.776597\du}}
\pgfpathlineto{\pgfpoint{17.671742\du}{10.767835\du}}
\pgfpathlineto{\pgfpoint{17.687076\du}{10.763089\du}}
\pgfpathlineto{\pgfpoint{17.702410\du}{10.759073\du}}
\pgfpathlineto{\pgfpoint{17.718474\du}{10.754326\du}}
\pgfpathlineto{\pgfpoint{17.733078\du}{10.749580\du}}
\pgfpathlineto{\pgfpoint{17.747317\du}{10.744834\du}}
\pgfpathlineto{\pgfpoint{17.762286\du}{10.739723\du}}
\pgfpathlineto{\pgfpoint{17.777254\du}{10.734976\du}}
\pgfpathlineto{\pgfpoint{17.791128\du}{10.730230\du}}
\pgfpathlineto{\pgfpoint{17.805367\du}{10.725119\du}}
\pgfpathlineto{\pgfpoint{17.819241\du}{10.720372\du}}
\pgfpathlineto{\pgfpoint{17.833479\du}{10.714896\du}}
\pgfpathlineto{\pgfpoint{17.846623\du}{10.710515\du}}
\pgfpathlineto{\pgfpoint{17.860862\du}{10.705038\du}}
\pgfpathlineto{\pgfpoint{17.874005\du}{10.699927\du}}
\pgfpathlineto{\pgfpoint{17.887149\du}{10.694451\du}}
\pgfpathlineto{\pgfpoint{17.900657\du}{10.688609\du}}
\pgfpathlineto{\pgfpoint{17.913436\du}{10.683498\du}}
\pgfpathlineto{\pgfpoint{17.925484\du}{10.678021\du}}
\pgfpathlineto{\pgfpoint{17.938262\du}{10.672180\du}}
\pgfpathlineto{\pgfpoint{17.950675\du}{10.667068\du}}
\pgfpathlineto{\pgfpoint{17.963089\du}{10.661227\du}}
\pgfpathlineto{\pgfpoint{17.974407\du}{10.655750\du}}
\pgfpathlineto{\pgfpoint{17.986455\du}{10.649909\du}}
\pgfpathlineto{\pgfpoint{17.997408\du}{10.644067\du}}
\pgfpathlineto{\pgfpoint{18.009091\du}{10.638226\du}}
\pgfpathlineto{\pgfpoint{18.020409\du}{10.632384\du}}
\pgfpathlineto{\pgfpoint{18.031727\du}{10.626543\du}}
\pgfpathlineto{\pgfpoint{18.042315\du}{10.620336\du}}
\pgfpathlineto{\pgfpoint{18.052172\du}{10.614494\du}}
\pgfpathlineto{\pgfpoint{18.062760\du}{10.608653\du}}
\pgfpathlineto{\pgfpoint{18.072983\du}{10.602081\du}}
\pgfpathlineto{\pgfpoint{18.082840\du}{10.596240\du}}
\pgfpathlineto{\pgfpoint{18.092698\du}{10.589668\du}}
\pgfpathlineto{\pgfpoint{18.101825\du}{10.583461\du}}
\pgfpathlineto{\pgfpoint{18.110953\du}{10.576889\du}}
\pgfpathlineto{\pgfpoint{18.120445\du}{10.571048\du}}
\pgfpathlineto{\pgfpoint{18.129208\du}{10.564476\du}}
\pgfpathlineto{\pgfpoint{18.138335\du}{10.558269\du}}
\pgfpathlineto{\pgfpoint{18.146367\du}{10.551698\du}}
\pgfpathlineto{\pgfpoint{18.155130\du}{10.544761\du}}
\pgfpathlineto{\pgfpoint{18.162432\du}{10.538189\du}}
\pgfpathlineto{\pgfpoint{18.170464\du}{10.531982\du}}
\pgfpathlineto{\pgfpoint{18.178496\du}{10.524681\du}}
\pgfpathlineto{\pgfpoint{18.185068\du}{10.518474\du}}
\pgfpathlineto{\pgfpoint{18.192369\du}{10.511537\du}}
\pgfpathlineto{\pgfpoint{18.198941\du}{10.504235\du}}
\pgfpathlineto{\pgfpoint{18.205878\du}{10.498028\du}}
\pgfpathlineto{\pgfpoint{18.212450\du}{10.491092\du}}
\pgfpathlineto{\pgfpoint{18.218656\du}{10.483790\du}}
\pgfpathlineto{\pgfpoint{18.224498\du}{10.476853\du}}
\pgfpathlineto{\pgfpoint{18.229974\du}{10.469916\du}}
\pgfpathlineto{\pgfpoint{18.235816\du}{10.462979\du}}
\pgfpathlineto{\pgfpoint{18.240562\du}{10.455677\du}}
\pgfpathlineto{\pgfpoint{18.246039\du}{10.448375\du}}
\pgfpathlineto{\pgfpoint{18.250785\du}{10.441073\du}}
\pgfpathlineto{\pgfpoint{18.255166\du}{10.434137\du}}
\pgfpathlineto{\pgfpoint{18.259182\du}{10.426470\du}}
\pgfpathlineto{\pgfpoint{18.263198\du}{10.419533\du}}
\pgfpathlineto{\pgfpoint{18.266484\du}{10.411866\du}}
\pgfpathlineto{\pgfpoint{18.270500\du}{10.404199\du}}
\pgfpathlineto{\pgfpoint{18.273786\du}{10.396532\du}}
\pgfpathlineto{\pgfpoint{18.276342\du}{10.389230\du}}
\pgfpathlineto{\pgfpoint{18.279263\du}{10.381928\du}}
\pgfpathlineto{\pgfpoint{18.281088\du}{10.374626\du}}
\pgfpathlineto{\pgfpoint{18.284009\du}{10.366959\du}}
\pgfpathlineto{\pgfpoint{18.285104\du}{10.358562\du}}
\pgfpathlineto{\pgfpoint{18.287295\du}{10.350894\du}}
\pgfpathlineto{\pgfpoint{18.288390\du}{10.343593\du}}
\pgfpathlineto{\pgfpoint{18.289120\du}{10.335926\du}}
\pgfpathlineto{\pgfpoint{18.289850\du}{10.327528\du}}
\pgfpathlineto{\pgfpoint{18.290581\du}{10.320226\du}}
\pgfpathlineto{\pgfpoint{18.290581\du}{10.312559\du}}
\pgfpathlineto{\pgfpoint{18.270500\du}{10.312559\du}}
\pgfpathlineto{\pgfpoint{18.269770\du}{10.319496\du}}
\pgfpathlineto{\pgfpoint{18.269770\du}{10.326433\du}}
\pgfpathlineto{\pgfpoint{18.269405\du}{10.333370\du}}
\pgfpathlineto{\pgfpoint{18.267579\du}{10.340672\du}}
\pgfpathlineto{\pgfpoint{18.266484\du}{10.347609\du}}
\pgfpathlineto{\pgfpoint{18.265754\du}{10.354545\du}}
\pgfpathlineto{\pgfpoint{18.263563\du}{10.361482\du}}
\pgfpathlineto{\pgfpoint{18.262103\du}{10.368784\du}}
\pgfpathlineto{\pgfpoint{18.259912\du}{10.374991\du}}
\pgfpathlineto{\pgfpoint{18.257357\du}{10.381928\du}}
\pgfpathlineto{\pgfpoint{18.254436\du}{10.389230\du}}
\pgfpathlineto{\pgfpoint{18.251880\du}{10.396166\du}}
\pgfpathlineto{\pgfpoint{18.247864\du}{10.403103\du}}
\pgfpathlineto{\pgfpoint{18.244943\du}{10.409675\du}}
\pgfpathlineto{\pgfpoint{18.240927\du}{10.416612\du}}
\pgfpathlineto{\pgfpoint{18.238007\du}{10.423184\du}}
\pgfpathlineto{\pgfpoint{18.233625\du}{10.430120\du}}
\pgfpathlineto{\pgfpoint{18.229244\du}{10.437057\du}}
\pgfpathlineto{\pgfpoint{18.224498\du}{10.443629\du}}
\pgfpathlineto{\pgfpoint{18.219387\du}{10.449836\du}}
\pgfpathlineto{\pgfpoint{18.214640\du}{10.457138\du}}
\pgfpathlineto{\pgfpoint{18.208434\du}{10.464074\du}}
\pgfpathlineto{\pgfpoint{18.202957\du}{10.470281\du}}
\pgfpathlineto{\pgfpoint{18.197846\du}{10.476853\du}}
\pgfpathlineto{\pgfpoint{18.191274\du}{10.483425\du}}
\pgfpathlineto{\pgfpoint{18.184337\du}{10.490361\du}}
\pgfpathlineto{\pgfpoint{18.178496\du}{10.496933\du}}
\pgfpathlineto{\pgfpoint{18.171194\du}{10.503140\du}}
\pgfpathlineto{\pgfpoint{18.164622\du}{10.509712\du}}
\pgfpathlineto{\pgfpoint{18.156955\du}{10.515918\du}}
\pgfpathlineto{\pgfpoint{18.149653\du}{10.522490\du}}
\pgfpathlineto{\pgfpoint{18.141621\du}{10.529062\du}}
\pgfpathlineto{\pgfpoint{18.133954\du}{10.535268\du}}
\pgfpathlineto{\pgfpoint{18.125192\du}{10.541840\du}}
\pgfpathlineto{\pgfpoint{18.116794\du}{10.547682\du}}
\pgfpathlineto{\pgfpoint{18.108032\du}{10.554253\du}}
\pgfpathlineto{\pgfpoint{18.100000\du}{10.560460\du}}
\pgfpathlineto{\pgfpoint{18.091238\du}{10.566302\du}}
\pgfpathlineto{\pgfpoint{18.081745\du}{10.572873\du}}
\pgfpathlineto{\pgfpoint{18.071522\du}{10.578715\du}}
\pgfpathlineto{\pgfpoint{18.062395\du}{10.585287\du}}
\pgfpathlineto{\pgfpoint{18.052172\du}{10.591128\du}}
\pgfpathlineto{\pgfpoint{18.042315\du}{10.596970\du}}
\pgfpathlineto{\pgfpoint{18.032457\du}{10.602811\du}}
\pgfpathlineto{\pgfpoint{18.021504\du}{10.608653\du}}
\pgfpathlineto{\pgfpoint{18.010551\du}{10.614494\du}}
\pgfpathlineto{\pgfpoint{18.000329\du}{10.620336\du}}
\pgfpathlineto{\pgfpoint{17.988280\du}{10.626177\du}}
\pgfpathlineto{\pgfpoint{17.977693\du}{10.631289\du}}
\pgfpathlineto{\pgfpoint{17.965644\du}{10.637130\du}}
\pgfpathlineto{\pgfpoint{17.954326\du}{10.642972\du}}
\pgfpathlineto{\pgfpoint{17.941913\du}{10.648448\du}}
\pgfpathlineto{\pgfpoint{17.929865\du}{10.653560\du}}
\pgfpathlineto{\pgfpoint{17.917452\du}{10.659401\du}}
\pgfpathlineto{\pgfpoint{17.905403\du}{10.664878\du}}
\pgfpathlineto{\pgfpoint{17.892260\du}{10.669989\du}}
\pgfpathlineto{\pgfpoint{17.879482\du}{10.675100\du}}
\pgfpathlineto{\pgfpoint{17.866703\du}{10.680577\du}}
\pgfpathlineto{\pgfpoint{17.853560\du}{10.685688\du}}
\pgfpathlineto{\pgfpoint{17.840051\du}{10.691165\du}}
\pgfpathlineto{\pgfpoint{17.826908\du}{10.695546\du}}
\pgfpathlineto{\pgfpoint{17.812669\du}{10.701022\du}}
\pgfpathlineto{\pgfpoint{17.798795\du}{10.705769\du}}
\pgfpathlineto{\pgfpoint{17.784556\du}{10.710880\du}}
\pgfpathlineto{\pgfpoint{17.769953\du}{10.715626\du}}
\pgfpathlineto{\pgfpoint{17.756079\du}{10.720372\du}}
\pgfpathlineto{\pgfpoint{17.741475\du}{10.725119\du}}
\pgfpathlineto{\pgfpoint{17.727236\du}{10.729500\du}}
\pgfpathlineto{\pgfpoint{17.711537\du}{10.734246\du}}
\pgfpathlineto{\pgfpoint{17.697298\du}{10.738992\du}}
\pgfpathlineto{\pgfpoint{17.681599\du}{10.743739\du}}
\pgfpathlineto{\pgfpoint{17.666630\du}{10.747755\du}}
\pgfpathlineto{\pgfpoint{17.635232\du}{10.756517\du}}
\pgfpathlineto{\pgfpoint{17.603103\du}{10.764914\du}}
\pgfpathlineto{\pgfpoint{17.570975\du}{10.772946\du}}
\pgfpathlineto{\pgfpoint{17.537751\du}{10.780978\du}}
\pgfpathlineto{\pgfpoint{17.503432\du}{10.788645\du}}
\pgfpathlineto{\pgfpoint{17.469478\du}{10.795947\du}}
\pgfpathlineto{\pgfpoint{17.434794\du}{10.802884\du}}
\pgfpathlineto{\pgfpoint{17.399014\du}{10.809821\du}}
\pgfpathlineto{\pgfpoint{17.363600\du}{10.816393\du}}
\pgfpathlineto{\pgfpoint{17.326725\du}{10.822599\du}}
\pgfpathlineto{\pgfpoint{17.290215\du}{10.828441\du}}
\pgfpathlineto{\pgfpoint{17.252976\du}{10.834283\du}}
\pgfpathlineto{\pgfpoint{17.215371\du}{10.839759\du}}
\pgfpathlineto{\pgfpoint{17.176670\du}{10.844505\du}}
\pgfpathlineto{\pgfpoint{17.137970\du}{10.848886\du}}
\pgfpathlineto{\pgfpoint{17.098540\du}{10.853633\du}}
\pgfpathlineto{\pgfpoint{17.059474\du}{10.857284\du}}
\pgfpathlineto{\pgfpoint{17.019314\du}{10.861300\du}}
\pgfpathlineto{\pgfpoint{16.979153\du}{10.864221\du}}
\pgfpathlineto{\pgfpoint{16.937897\du}{10.867871\du}}
\pgfpathlineto{\pgfpoint{16.897006\du}{10.870792\du}}
\pgfpathlineto{\pgfpoint{16.855750\du}{10.872983\du}}
\pgfpathlineto{\pgfpoint{16.814129\du}{10.874808\du}}
\pgfpathlineto{\pgfpoint{16.771413\du}{10.876634\du}}
\pgfpathlineto{\pgfpoint{16.728697\du}{10.877729\du}}
\pgfpathlineto{\pgfpoint{16.686710\du}{10.878824\du}}
\pgfpathlineto{\pgfpoint{16.643264\du}{10.878824\du}}
\pgfpathlineto{\pgfpoint{16.600183\du}{10.879555\du}}
\pgfpathlineto{\pgfpoint{16.600183\du}{10.879555\du}}
\pgfpathlineto{\pgfpoint{16.600183\du}{10.879555\du}}
\pgfpathlineto{\pgfpoint{16.599452\du}{10.879555\du}}
\pgfpathlineto{\pgfpoint{16.597627\du}{10.879555\du}}
\pgfpathlineto{\pgfpoint{16.596532\du}{10.879920\du}}
\pgfpathlineto{\pgfpoint{16.595801\du}{10.879920\du}}
\pgfpathlineto{\pgfpoint{16.595436\du}{10.880650\du}}
\pgfpathlineto{\pgfpoint{16.593976\du}{10.881015\du}}
\pgfpathlineto{\pgfpoint{16.593246\du}{10.881745\du}}
\pgfpathlineto{\pgfpoint{16.592516\du}{10.882475\du}}
\pgfpathlineto{\pgfpoint{16.591420\du}{10.884301\du}}
\pgfpathlineto{\pgfpoint{16.590690\du}{10.885761\du}}
\pgfpathlineto{\pgfpoint{16.590690\du}{10.887587\du}}
\pgfpathlineto{\pgfpoint{16.589960\du}{10.889412\du}}
\pgfpathlineto{\pgfpoint{16.590690\du}{10.891603\du}}
\pgfpathlineto{\pgfpoint{16.590690\du}{10.893428\du}}
\pgfpathlineto{\pgfpoint{16.591420\du}{10.895254\du}}
\pgfpathlineto{\pgfpoint{16.592516\du}{10.897079\du}}
\pgfpathlineto{\pgfpoint{16.593246\du}{10.897444\du}}
\pgfpathlineto{\pgfpoint{16.593976\du}{10.898175\du}}
\pgfpathlineto{\pgfpoint{16.595436\du}{10.898905\du}}
\pgfpathlineto{\pgfpoint{16.595801\du}{10.899270\du}}
\pgfpathlineto{\pgfpoint{16.596532\du}{10.899270\du}}
\pgfpathlineto{\pgfpoint{16.597627\du}{10.900000\du}}
\pgfpathlineto{\pgfpoint{16.599452\du}{10.900000\du}}
\pgfpathlineto{\pgfpoint{16.600183\du}{10.900000\du}}
\pgfusepath{fill}
\pgfsetbuttcap
\pgfsetmiterjoin
\pgfsetdash{}{0pt}
\definecolor{dialinecolor}{rgb}{0.678431, 0.839216, 0.905882}
\pgfsetfillcolor{dialinecolor}
\pgfpathmoveto{\pgfpoint{14.909419\du}{10.312559\du}}
\pgfpathlineto{\pgfpoint{14.909419\du}{10.312559\du}}
\pgfpathlineto{\pgfpoint{14.909419\du}{10.320226\du}}
\pgfpathlineto{\pgfpoint{14.909785\du}{10.327528\du}}
\pgfpathlineto{\pgfpoint{14.910515\du}{10.335926\du}}
\pgfpathlineto{\pgfpoint{14.911610\du}{10.343593\du}}
\pgfpathlineto{\pgfpoint{14.912705\du}{10.350894\du}}
\pgfpathlineto{\pgfpoint{14.914531\du}{10.358562\du}}
\pgfpathlineto{\pgfpoint{14.916356\du}{10.366959\du}}
\pgfpathlineto{\pgfpoint{14.918547\du}{10.374626\du}}
\pgfpathlineto{\pgfpoint{14.920737\du}{10.381928\du}}
\pgfpathlineto{\pgfpoint{14.923293\du}{10.389230\du}}
\pgfpathlineto{\pgfpoint{14.926214\du}{10.396532\du}}
\pgfpathlineto{\pgfpoint{14.929865\du}{10.404199\du}}
\pgfpathlineto{\pgfpoint{14.933151\du}{10.411866\du}}
\pgfpathlineto{\pgfpoint{14.936802\du}{10.419533\du}}
\pgfpathlineto{\pgfpoint{14.941183\du}{10.426470\du}}
\pgfpathlineto{\pgfpoint{14.944834\du}{10.434137\du}}
\pgfpathlineto{\pgfpoint{14.949945\du}{10.441073\du}}
\pgfpathlineto{\pgfpoint{14.953961\du}{10.448375\du}}
\pgfpathlineto{\pgfpoint{14.959438\du}{10.455677\du}}
\pgfpathlineto{\pgfpoint{14.964184\du}{10.462979\du}}
\pgfpathlineto{\pgfpoint{14.969660\du}{10.469916\du}}
\pgfpathlineto{\pgfpoint{14.975502\du}{10.476853\du}}
\pgfpathlineto{\pgfpoint{14.981344\du}{10.483790\du}}
\pgfpathlineto{\pgfpoint{14.987185\du}{10.491092\du}}
\pgfpathlineto{\pgfpoint{14.994122\du}{10.498028\du}}
\pgfpathlineto{\pgfpoint{15.000694\du}{10.504235\du}}
\pgfpathlineto{\pgfpoint{15.007631\du}{10.511537\du}}
\pgfpathlineto{\pgfpoint{15.014567\du}{10.518474\du}}
\pgfpathlineto{\pgfpoint{15.021504\du}{10.524681\du}}
\pgfpathlineto{\pgfpoint{15.030267\du}{10.531982\du}}
\pgfpathlineto{\pgfpoint{15.037203\du}{10.538189\du}}
\pgfpathlineto{\pgfpoint{15.044870\du}{10.544761\du}}
\pgfpathlineto{\pgfpoint{15.053633\du}{10.551698\du}}
\pgfpathlineto{\pgfpoint{15.062030\du}{10.558269\du}}
\pgfpathlineto{\pgfpoint{15.070792\du}{10.564476\du}}
\pgfpathlineto{\pgfpoint{15.079189\du}{10.571048\du}}
\pgfpathlineto{\pgfpoint{15.089047\du}{10.576889\du}}
\pgfpathlineto{\pgfpoint{15.098175\du}{10.583461\du}}
\pgfpathlineto{\pgfpoint{15.107667\du}{10.589668\du}}
\pgfpathlineto{\pgfpoint{15.117160\du}{10.596240\du}}
\pgfpathlineto{\pgfpoint{15.127017\du}{10.602081\du}}
\pgfpathlineto{\pgfpoint{15.137240\du}{10.608653\du}}
\pgfpathlineto{\pgfpoint{15.147463\du}{10.614494\du}}
\pgfpathlineto{\pgfpoint{15.158050\du}{10.620336\du}}
\pgfpathlineto{\pgfpoint{15.168273\du}{10.626543\du}}
\pgfpathlineto{\pgfpoint{15.179226\du}{10.632384\du}}
\pgfpathlineto{\pgfpoint{15.190909\du}{10.638226\du}}
\pgfpathlineto{\pgfpoint{15.202227\du}{10.644067\du}}
\pgfpathlineto{\pgfpoint{15.213910\du}{10.649909\du}}
\pgfpathlineto{\pgfpoint{15.225228\du}{10.655750\du}}
\pgfpathlineto{\pgfpoint{15.236911\du}{10.661227\du}}
\pgfpathlineto{\pgfpoint{15.249325\du}{10.667068\du}}
\pgfpathlineto{\pgfpoint{15.261373\du}{10.672180\du}}
\pgfpathlineto{\pgfpoint{15.274516\du}{10.678021\du}}
\pgfpathlineto{\pgfpoint{15.286564\du}{10.683498\du}}
\pgfpathlineto{\pgfpoint{15.299343\du}{10.688609\du}}
\pgfpathlineto{\pgfpoint{15.313217\du}{10.694451\du}}
\pgfpathlineto{\pgfpoint{15.325630\du}{10.699927\du}}
\pgfpathlineto{\pgfpoint{15.338773\du}{10.705038\du}}
\pgfpathlineto{\pgfpoint{15.353012\du}{10.710515\du}}
\pgfpathlineto{\pgfpoint{15.366156\du}{10.714896\du}}
\pgfpathlineto{\pgfpoint{15.380394\du}{10.720372\du}}
\pgfpathlineto{\pgfpoint{15.394268\du}{10.725119\du}}
\pgfpathlineto{\pgfpoint{15.408872\du}{10.730230\du}}
\pgfpathlineto{\pgfpoint{15.422746\du}{10.734976\du}}
\pgfpathlineto{\pgfpoint{15.438445\du}{10.739723\du}}
\pgfpathlineto{\pgfpoint{15.452318\du}{10.744834\du}}
\pgfpathlineto{\pgfpoint{15.466922\du}{10.749580\du}}
\pgfpathlineto{\pgfpoint{15.482256\du}{10.754326\du}}
\pgfpathlineto{\pgfpoint{15.497955\du}{10.759073\du}}
\pgfpathlineto{\pgfpoint{15.512924\du}{10.763089\du}}
\pgfpathlineto{\pgfpoint{15.528624\du}{10.767835\du}}
\pgfpathlineto{\pgfpoint{15.560387\du}{10.776597\du}}
\pgfpathlineto{\pgfpoint{15.592150\du}{10.784629\du}}
\pgfpathlineto{\pgfpoint{15.625374\du}{10.792662\du}}
\pgfpathlineto{\pgfpoint{15.657868\du}{10.801059\du}}
\pgfpathlineto{\pgfpoint{15.692552\du}{10.808726\du}}
\pgfpathlineto{\pgfpoint{15.726871\du}{10.816393\du}}
\pgfpathlineto{\pgfpoint{15.761555\du}{10.823330\du}}
\pgfpathlineto{\pgfpoint{15.797335\du}{10.830267\du}}
\pgfpathlineto{\pgfpoint{15.833479\du}{10.836838\du}}
\pgfpathlineto{\pgfpoint{15.869989\du}{10.842680\du}}
\pgfpathlineto{\pgfpoint{15.907229\du}{10.848886\du}}
\pgfpathlineto{\pgfpoint{15.944834\du}{10.854363\du}}
\pgfpathlineto{\pgfpoint{15.982804\du}{10.859474\du}}
\pgfpathlineto{\pgfpoint{16.021504\du}{10.864951\du}}
\pgfpathlineto{\pgfpoint{16.060204\du}{10.869332\du}}
\pgfpathlineto{\pgfpoint{16.099270\du}{10.873713\du}}
\pgfpathlineto{\pgfpoint{16.139430\du}{10.877729\du}}
\pgfpathlineto{\pgfpoint{16.179226\du}{10.881745\du}}
\pgfpathlineto{\pgfpoint{16.220117\du}{10.885396\du}}
\pgfpathlineto{\pgfpoint{16.261008\du}{10.888317\du}}
\pgfpathlineto{\pgfpoint{16.302264\du}{10.890507\du}}
\pgfpathlineto{\pgfpoint{16.344250\du}{10.893063\du}}
\pgfpathlineto{\pgfpoint{16.385871\du}{10.895254\du}}
\pgfpathlineto{\pgfpoint{16.428222\du}{10.897079\du}}
\pgfpathlineto{\pgfpoint{16.470208\du}{10.898175\du}}
\pgfpathlineto{\pgfpoint{16.513655\du}{10.899270\du}}
\pgfpathlineto{\pgfpoint{16.556371\du}{10.900000\du}}
\pgfpathlineto{\pgfpoint{16.600183\du}{10.900000\du}}
\pgfpathlineto{\pgfpoint{16.600183\du}{10.879555\du}}
\pgfpathlineto{\pgfpoint{16.557466\du}{10.878824\du}}
\pgfpathlineto{\pgfpoint{16.514020\du}{10.878824\du}}
\pgfpathlineto{\pgfpoint{16.471668\du}{10.877729\du}}
\pgfpathlineto{\pgfpoint{16.428952\du}{10.876634\du}}
\pgfpathlineto{\pgfpoint{16.386601\du}{10.874808\du}}
\pgfpathlineto{\pgfpoint{16.344615\du}{10.872983\du}}
\pgfpathlineto{\pgfpoint{16.303724\du}{10.870792\du}}
\pgfpathlineto{\pgfpoint{16.262468\du}{10.867871\du}}
\pgfpathlineto{\pgfpoint{16.221577\du}{10.864221\du}}
\pgfpathlineto{\pgfpoint{16.181417\du}{10.861300\du}}
\pgfpathlineto{\pgfpoint{16.141621\du}{10.857284\du}}
\pgfpathlineto{\pgfpoint{16.102191\du}{10.853633\du}}
\pgfpathlineto{\pgfpoint{16.062760\du}{10.848886\du}}
\pgfpathlineto{\pgfpoint{16.023695\du}{10.844505\du}}
\pgfpathlineto{\pgfpoint{15.985360\du}{10.839759\du}}
\pgfpathlineto{\pgfpoint{15.947755\du}{10.834283\du}}
\pgfpathlineto{\pgfpoint{15.910515\du}{10.828441\du}}
\pgfpathlineto{\pgfpoint{15.874005\du}{10.822599\du}}
\pgfpathlineto{\pgfpoint{15.836765\du}{10.816393\du}}
\pgfpathlineto{\pgfpoint{15.801716\du}{10.809821\du}}
\pgfpathlineto{\pgfpoint{15.765571\du}{10.802884\du}}
\pgfpathlineto{\pgfpoint{15.730887\du}{10.795947\du}}
\pgfpathlineto{\pgfpoint{15.696568\du}{10.788645\du}}
\pgfpathlineto{\pgfpoint{15.662614\du}{10.780978\du}}
\pgfpathlineto{\pgfpoint{15.629390\du}{10.772946\du}}
\pgfpathlineto{\pgfpoint{15.597992\du}{10.764914\du}}
\pgfpathlineto{\pgfpoint{15.565498\du}{10.756517\du}}
\pgfpathlineto{\pgfpoint{15.534465\du}{10.747755\du}}
\pgfpathlineto{\pgfpoint{15.518766\du}{10.743739\du}}
\pgfpathlineto{\pgfpoint{15.503067\du}{10.738992\du}}
\pgfpathlineto{\pgfpoint{15.488463\du}{10.734246\du}}
\pgfpathlineto{\pgfpoint{15.473494\du}{10.729500\du}}
\pgfpathlineto{\pgfpoint{15.458160\du}{10.725119\du}}
\pgfpathlineto{\pgfpoint{15.443921\du}{10.720372\du}}
\pgfpathlineto{\pgfpoint{15.429682\du}{10.715626\du}}
\pgfpathlineto{\pgfpoint{15.415444\du}{10.710880\du}}
\pgfpathlineto{\pgfpoint{15.401205\du}{10.705769\du}}
\pgfpathlineto{\pgfpoint{15.387331\du}{10.701022\du}}
\pgfpathlineto{\pgfpoint{15.373457\du}{10.695546\du}}
\pgfpathlineto{\pgfpoint{15.359949\du}{10.691165\du}}
\pgfpathlineto{\pgfpoint{15.346440\du}{10.685688\du}}
\pgfpathlineto{\pgfpoint{15.333662\du}{10.680577\du}}
\pgfpathlineto{\pgfpoint{15.320153\du}{10.675100\du}}
\pgfpathlineto{\pgfpoint{15.307375\du}{10.669989\du}}
\pgfpathlineto{\pgfpoint{15.294962\du}{10.664878\du}}
\pgfpathlineto{\pgfpoint{15.281818\du}{10.659401\du}}
\pgfpathlineto{\pgfpoint{15.269770\du}{10.653560\du}}
\pgfpathlineto{\pgfpoint{15.258452\du}{10.648448\du}}
\pgfpathlineto{\pgfpoint{15.245674\du}{10.642972\du}}
\pgfpathlineto{\pgfpoint{15.233991\du}{10.637130\du}}
\pgfpathlineto{\pgfpoint{15.222307\du}{10.631289\du}}
\pgfpathlineto{\pgfpoint{15.211355\du}{10.626177\du}}
\pgfpathlineto{\pgfpoint{15.199671\du}{10.620336\du}}
\pgfpathlineto{\pgfpoint{15.189814\du}{10.614494\du}}
\pgfpathlineto{\pgfpoint{15.178496\du}{10.608653\du}}
\pgfpathlineto{\pgfpoint{15.167908\du}{10.602811\du}}
\pgfpathlineto{\pgfpoint{15.158050\du}{10.596970\du}}
\pgfpathlineto{\pgfpoint{15.147463\du}{10.591128\du}}
\pgfpathlineto{\pgfpoint{15.137605\du}{10.585287\du}}
\pgfpathlineto{\pgfpoint{15.128112\du}{10.578715\du}}
\pgfpathlineto{\pgfpoint{15.118255\du}{10.572873\du}}
\pgfpathlineto{\pgfpoint{15.109127\du}{10.566302\du}}
\pgfpathlineto{\pgfpoint{15.100730\du}{10.560460\du}}
\pgfpathlineto{\pgfpoint{15.091968\du}{10.554253\du}}
\pgfpathlineto{\pgfpoint{15.082840\du}{10.547682\du}}
\pgfpathlineto{\pgfpoint{15.074078\du}{10.541840\du}}
\pgfpathlineto{\pgfpoint{15.065681\du}{10.535268\du}}
\pgfpathlineto{\pgfpoint{15.058379\du}{10.529062\du}}
\pgfpathlineto{\pgfpoint{15.050347\du}{10.522490\du}}
\pgfpathlineto{\pgfpoint{15.042680\du}{10.515918\du}}
\pgfpathlineto{\pgfpoint{15.035743\du}{10.509712\du}}
\pgfpathlineto{\pgfpoint{15.028441\du}{10.503140\du}}
\pgfpathlineto{\pgfpoint{15.021504\du}{10.496933\du}}
\pgfpathlineto{\pgfpoint{15.015298\du}{10.490361\du}}
\pgfpathlineto{\pgfpoint{15.008726\du}{10.483425\du}}
\pgfpathlineto{\pgfpoint{15.002884\du}{10.476853\du}}
\pgfpathlineto{\pgfpoint{14.996678\du}{10.470281\du}}
\pgfpathlineto{\pgfpoint{14.991566\du}{10.464074\du}}
\pgfpathlineto{\pgfpoint{14.985360\du}{10.457138\du}}
\pgfpathlineto{\pgfpoint{14.980978\du}{10.450566\du}}
\pgfpathlineto{\pgfpoint{14.975502\du}{10.443629\du}}
\pgfpathlineto{\pgfpoint{14.971121\du}{10.437057\du}}
\pgfpathlineto{\pgfpoint{14.966740\du}{10.430120\du}}
\pgfpathlineto{\pgfpoint{14.962359\du}{10.423184\du}}
\pgfpathlineto{\pgfpoint{14.959073\du}{10.416612\du}}
\pgfpathlineto{\pgfpoint{14.955057\du}{10.409675\du}}
\pgfpathlineto{\pgfpoint{14.951041\du}{10.403103\du}}
\pgfpathlineto{\pgfpoint{14.948120\du}{10.396166\du}}
\pgfpathlineto{\pgfpoint{14.945564\du}{10.389230\du}}
\pgfpathlineto{\pgfpoint{14.942278\du}{10.381928\du}}
\pgfpathlineto{\pgfpoint{14.940088\du}{10.374991\du}}
\pgfpathlineto{\pgfpoint{14.937897\du}{10.368784\du}}
\pgfpathlineto{\pgfpoint{14.936437\du}{10.361482\du}}
\pgfpathlineto{\pgfpoint{14.934246\du}{10.354545\du}}
\pgfpathlineto{\pgfpoint{14.933151\du}{10.347609\du}}
\pgfpathlineto{\pgfpoint{14.932055\du}{10.340672\du}}
\pgfpathlineto{\pgfpoint{14.930595\du}{10.333370\du}}
\pgfpathlineto{\pgfpoint{14.930230\du}{10.326433\du}}
\pgfpathlineto{\pgfpoint{14.930230\du}{10.319496\du}}
\pgfpathlineto{\pgfpoint{14.929865\du}{10.312559\du}}
\pgfpathlineto{\pgfpoint{14.929865\du}{10.312559\du}}
\pgfpathlineto{\pgfpoint{14.929865\du}{10.312559\du}}
\pgfpathlineto{\pgfpoint{14.929865\du}{10.310734\du}}
\pgfpathlineto{\pgfpoint{14.929865\du}{10.310004\du}}
\pgfpathlineto{\pgfpoint{14.929500\du}{10.308908\du}}
\pgfpathlineto{\pgfpoint{14.929500\du}{10.307813\du}}
\pgfpathlineto{\pgfpoint{14.928405\du}{10.307083\du}}
\pgfpathlineto{\pgfpoint{14.928039\du}{10.305988\du}}
\pgfpathlineto{\pgfpoint{14.927674\du}{10.305257\du}}
\pgfpathlineto{\pgfpoint{14.926579\du}{10.304892\du}}
\pgfpathlineto{\pgfpoint{14.925119\du}{10.303797\du}}
\pgfpathlineto{\pgfpoint{14.923293\du}{10.302337\du}}
\pgfpathlineto{\pgfpoint{14.921468\du}{10.301972\du}}
\pgfpathlineto{\pgfpoint{14.919642\du}{10.301972\du}}
\pgfpathlineto{\pgfpoint{14.917452\du}{10.301972\du}}
\pgfpathlineto{\pgfpoint{14.915991\du}{10.302337\du}}
\pgfpathlineto{\pgfpoint{14.913801\du}{10.303797\du}}
\pgfpathlineto{\pgfpoint{14.911975\du}{10.304892\du}}
\pgfpathlineto{\pgfpoint{14.911610\du}{10.305257\du}}
\pgfpathlineto{\pgfpoint{14.911245\du}{10.305988\du}}
\pgfpathlineto{\pgfpoint{14.910515\du}{10.307083\du}}
\pgfpathlineto{\pgfpoint{14.909785\du}{10.307813\du}}
\pgfpathlineto{\pgfpoint{14.909785\du}{10.308908\du}}
\pgfpathlineto{\pgfpoint{14.909419\du}{10.310004\du}}
\pgfpathlineto{\pgfpoint{14.909419\du}{10.310734\du}}
\pgfpathlineto{\pgfpoint{14.909419\du}{10.312559\du}}
\pgfusepath{fill}
\pgfsetbuttcap
\pgfsetmiterjoin
\pgfsetdash{}{0pt}
\definecolor{dialinecolor}{rgb}{0.678431, 0.839216, 0.905882}
\pgfsetfillcolor{dialinecolor}
\pgfpathmoveto{\pgfpoint{16.600183\du}{9.725119\du}}
\pgfpathlineto{\pgfpoint{16.600183\du}{9.725119\du}}
\pgfpathlineto{\pgfpoint{16.556371\du}{9.725119\du}}
\pgfpathlineto{\pgfpoint{16.513655\du}{9.725484\du}}
\pgfpathlineto{\pgfpoint{16.470208\du}{9.726579\du}}
\pgfpathlineto{\pgfpoint{16.428222\du}{9.728039\du}}
\pgfpathlineto{\pgfpoint{16.385871\du}{9.729500\du}}
\pgfpathlineto{\pgfpoint{16.344250\du}{9.731325\du}}
\pgfpathlineto{\pgfpoint{16.302264\du}{9.733881\du}}
\pgfpathlineto{\pgfpoint{16.261008\du}{9.736802\du}}
\pgfpathlineto{\pgfpoint{16.220117\du}{9.739723\du}}
\pgfpathlineto{\pgfpoint{16.179226\du}{9.742643\du}}
\pgfpathlineto{\pgfpoint{16.139430\du}{9.746659\du}}
\pgfpathlineto{\pgfpoint{16.099270\du}{9.750675\du}}
\pgfpathlineto{\pgfpoint{16.060204\du}{9.755422\du}}
\pgfpathlineto{\pgfpoint{16.021504\du}{9.760168\du}}
\pgfpathlineto{\pgfpoint{15.982804\du}{9.764914\du}}
\pgfpathlineto{\pgfpoint{15.944834\du}{9.770026\du}}
\pgfpathlineto{\pgfpoint{15.907229\du}{9.775867\du}}
\pgfpathlineto{\pgfpoint{15.869989\du}{9.781709\du}}
\pgfpathlineto{\pgfpoint{15.833479\du}{9.788280\du}}
\pgfpathlineto{\pgfpoint{15.797335\du}{9.794487\du}}
\pgfpathlineto{\pgfpoint{15.761555\du}{9.801789\du}}
\pgfpathlineto{\pgfpoint{15.726871\du}{9.808726\du}}
\pgfpathlineto{\pgfpoint{15.692552\du}{9.815663\du}}
\pgfpathlineto{\pgfpoint{15.657868\du}{9.823695\du}}
\pgfpathlineto{\pgfpoint{15.625374\du}{9.831362\du}}
\pgfpathlineto{\pgfpoint{15.592150\du}{9.839759\du}}
\pgfpathlineto{\pgfpoint{15.560387\du}{9.848521\du}}
\pgfpathlineto{\pgfpoint{15.528624\du}{9.857284\du}}
\pgfpathlineto{\pgfpoint{15.497955\du}{9.866046\du}}
\pgfpathlineto{\pgfpoint{15.466922\du}{9.875539\du}}
\pgfpathlineto{\pgfpoint{15.452318\du}{9.879920\du}}
\pgfpathlineto{\pgfpoint{15.438445\du}{9.884666\du}}
\pgfpathlineto{\pgfpoint{15.422746\du}{9.889412\du}}
\pgfpathlineto{\pgfpoint{15.408872\du}{9.894524\du}}
\pgfpathlineto{\pgfpoint{15.394268\du}{9.899270\du}}
\pgfpathlineto{\pgfpoint{15.380394\du}{9.904746\du}}
\pgfpathlineto{\pgfpoint{15.366156\du}{9.909127\du}}
\pgfpathlineto{\pgfpoint{15.353012\du}{9.914604\du}}
\pgfpathlineto{\pgfpoint{15.338773\du}{9.919715\du}}
\pgfpathlineto{\pgfpoint{15.325630\du}{9.925192\du}}
\pgfpathlineto{\pgfpoint{15.313217\du}{9.930303\du}}
\pgfpathlineto{\pgfpoint{15.299343\du}{9.935779\du}}
\pgfpathlineto{\pgfpoint{15.286564\du}{9.940891\du}}
\pgfpathlineto{\pgfpoint{15.274516\du}{9.946002\du}}
\pgfpathlineto{\pgfpoint{15.261373\du}{9.951844\du}}
\pgfpathlineto{\pgfpoint{15.249325\du}{9.958050\du}}
\pgfpathlineto{\pgfpoint{15.236911\du}{9.963162\du}}
\pgfpathlineto{\pgfpoint{15.225228\du}{9.969003\du}}
\pgfpathlineto{\pgfpoint{15.213910\du}{9.974845\du}}
\pgfpathlineto{\pgfpoint{15.202227\du}{9.980686\du}}
\pgfpathlineto{\pgfpoint{15.190909\du}{9.986528\du}}
\pgfpathlineto{\pgfpoint{15.179226\du}{9.991639\du}}
\pgfpathlineto{\pgfpoint{15.168273\du}{9.998211\du}}
\pgfpathlineto{\pgfpoint{15.158050\du}{10.004053\du}}
\pgfpathlineto{\pgfpoint{15.147463\du}{10.009894\du}}
\pgfpathlineto{\pgfpoint{15.137240\du}{10.015736\du}}
\pgfpathlineto{\pgfpoint{15.127017\du}{10.022307\du}}
\pgfpathlineto{\pgfpoint{15.117160\du}{10.028514\du}}
\pgfpathlineto{\pgfpoint{15.107667\du}{10.034356\du}}
\pgfpathlineto{\pgfpoint{15.098175\du}{10.040927\du}}
\pgfpathlineto{\pgfpoint{15.089047\du}{10.047499\du}}
\pgfpathlineto{\pgfpoint{15.079189\du}{10.053706\du}}
\pgfpathlineto{\pgfpoint{15.070792\du}{10.060277\du}}
\pgfpathlineto{\pgfpoint{15.062030\du}{10.066849\du}}
\pgfpathlineto{\pgfpoint{15.053633\du}{10.073056\du}}
\pgfpathlineto{\pgfpoint{15.044870\du}{10.079628\du}}
\pgfpathlineto{\pgfpoint{15.037203\du}{10.086564\du}}
\pgfpathlineto{\pgfpoint{15.030267\du}{10.093136\du}}
\pgfpathlineto{\pgfpoint{15.021504\du}{10.099343\du}}
\pgfpathlineto{\pgfpoint{15.014567\du}{10.106645\du}}
\pgfpathlineto{\pgfpoint{15.007631\du}{10.112851\du}}
\pgfpathlineto{\pgfpoint{15.000694\du}{10.119788\du}}
\pgfpathlineto{\pgfpoint{14.994122\du}{10.127090\du}}
\pgfpathlineto{\pgfpoint{14.987185\du}{10.133297\du}}
\pgfpathlineto{\pgfpoint{14.981344\du}{10.140599\du}}
\pgfpathlineto{\pgfpoint{14.975502\du}{10.147536\du}}
\pgfpathlineto{\pgfpoint{14.969660\du}{10.155203\du}}
\pgfpathlineto{\pgfpoint{14.964184\du}{10.162139\du}}
\pgfpathlineto{\pgfpoint{14.959438\du}{10.169076\du}}
\pgfpathlineto{\pgfpoint{14.953961\du}{10.176013\du}}
\pgfpathlineto{\pgfpoint{14.949945\du}{10.183315\du}}
\pgfpathlineto{\pgfpoint{14.944834\du}{10.190617\du}}
\pgfpathlineto{\pgfpoint{14.941183\du}{10.197919\du}}
\pgfpathlineto{\pgfpoint{14.936802\du}{10.205221\du}}
\pgfpathlineto{\pgfpoint{14.933151\du}{10.212888\du}}
\pgfpathlineto{\pgfpoint{14.929865\du}{10.220555\du}}
\pgfpathlineto{\pgfpoint{14.926214\du}{10.227492\du}}
\pgfpathlineto{\pgfpoint{14.923293\du}{10.235159\du}}
\pgfpathlineto{\pgfpoint{14.920737\du}{10.242826\du}}
\pgfpathlineto{\pgfpoint{14.918547\du}{10.250493\du}}
\pgfpathlineto{\pgfpoint{14.916356\du}{10.258160\du}}
\pgfpathlineto{\pgfpoint{14.914531\du}{10.265462\du}}
\pgfpathlineto{\pgfpoint{14.912705\du}{10.273129\du}}
\pgfpathlineto{\pgfpoint{14.911610\du}{10.280796\du}}
\pgfpathlineto{\pgfpoint{14.910515\du}{10.289193\du}}
\pgfpathlineto{\pgfpoint{14.909785\du}{10.296495\du}}
\pgfpathlineto{\pgfpoint{14.909419\du}{10.304162\du}}
\pgfpathlineto{\pgfpoint{14.909419\du}{10.312559\du}}
\pgfpathlineto{\pgfpoint{14.929865\du}{10.312559\du}}
\pgfpathlineto{\pgfpoint{14.930230\du}{10.305257\du}}
\pgfpathlineto{\pgfpoint{14.930230\du}{10.298321\du}}
\pgfpathlineto{\pgfpoint{14.930595\du}{10.291384\du}}
\pgfpathlineto{\pgfpoint{14.932055\du}{10.283717\du}}
\pgfpathlineto{\pgfpoint{14.933151\du}{10.277510\du}}
\pgfpathlineto{\pgfpoint{14.934246\du}{10.270208\du}}
\pgfpathlineto{\pgfpoint{14.936437\du}{10.263271\du}}
\pgfpathlineto{\pgfpoint{14.937897\du}{10.256334\du}}
\pgfpathlineto{\pgfpoint{14.940088\du}{10.249398\du}}
\pgfpathlineto{\pgfpoint{14.942278\du}{10.242096\du}}
\pgfpathlineto{\pgfpoint{14.945564\du}{10.235889\du}}
\pgfpathlineto{\pgfpoint{14.948120\du}{10.228952\du}}
\pgfpathlineto{\pgfpoint{14.951041\du}{10.221650\du}}
\pgfpathlineto{\pgfpoint{14.955057\du}{10.214713\du}}
\pgfpathlineto{\pgfpoint{14.959073\du}{10.207777\du}}
\pgfpathlineto{\pgfpoint{14.962359\du}{10.201205\du}}
\pgfpathlineto{\pgfpoint{14.966375\du}{10.194268\du}}
\pgfpathlineto{\pgfpoint{14.971121\du}{10.187696\du}}
\pgfpathlineto{\pgfpoint{14.975502\du}{10.180759\du}}
\pgfpathlineto{\pgfpoint{14.980978\du}{10.174188\du}}
\pgfpathlineto{\pgfpoint{14.985360\du}{10.167981\du}}
\pgfpathlineto{\pgfpoint{14.991566\du}{10.161044\du}}
\pgfpathlineto{\pgfpoint{14.996678\du}{10.154472\du}}
\pgfpathlineto{\pgfpoint{15.002884\du}{10.147536\du}}
\pgfpathlineto{\pgfpoint{15.008726\du}{10.140964\du}}
\pgfpathlineto{\pgfpoint{15.015298\du}{10.134757\du}}
\pgfpathlineto{\pgfpoint{15.021504\du}{10.128185\du}}
\pgfpathlineto{\pgfpoint{15.028441\du}{10.121249\du}}
\pgfpathlineto{\pgfpoint{15.035743\du}{10.115407\du}}
\pgfpathlineto{\pgfpoint{15.042680\du}{10.108105\du}}
\pgfpathlineto{\pgfpoint{15.050347\du}{10.101899\du}}
\pgfpathlineto{\pgfpoint{15.058379\du}{10.096057\du}}
\pgfpathlineto{\pgfpoint{15.065681\du}{10.089485\du}}
\pgfpathlineto{\pgfpoint{15.074078\du}{10.082913\du}}
\pgfpathlineto{\pgfpoint{15.082840\du}{10.076707\du}}
\pgfpathlineto{\pgfpoint{15.091968\du}{10.070135\du}}
\pgfpathlineto{\pgfpoint{15.100730\du}{10.064294\du}}
\pgfpathlineto{\pgfpoint{15.109127\du}{10.058087\du}}
\pgfpathlineto{\pgfpoint{15.118255\du}{10.052245\du}}
\pgfpathlineto{\pgfpoint{15.128112\du}{10.045674\du}}
\pgfpathlineto{\pgfpoint{15.137605\du}{10.039832\du}}
\pgfpathlineto{\pgfpoint{15.147463\du}{10.033991\du}}
\pgfpathlineto{\pgfpoint{15.158050\du}{10.028149\du}}
\pgfpathlineto{\pgfpoint{15.167908\du}{10.022307\du}}
\pgfpathlineto{\pgfpoint{15.178496\du}{10.015736\du}}
\pgfpathlineto{\pgfpoint{15.189814\du}{10.010624\du}}
\pgfpathlineto{\pgfpoint{15.199671\du}{10.004783\du}}
\pgfpathlineto{\pgfpoint{15.211355\du}{9.998941\du}}
\pgfpathlineto{\pgfpoint{15.222307\du}{9.993100\du}}
\pgfpathlineto{\pgfpoint{15.233991\du}{9.987258\du}}
\pgfpathlineto{\pgfpoint{15.245674\du}{9.981782\du}}
\pgfpathlineto{\pgfpoint{15.258452\du}{9.976670\du}}
\pgfpathlineto{\pgfpoint{15.269770\du}{9.970829\du}}
\pgfpathlineto{\pgfpoint{15.281818\du}{9.965352\du}}
\pgfpathlineto{\pgfpoint{15.294962\du}{9.960241\du}}
\pgfpathlineto{\pgfpoint{15.307375\du}{9.954765\du}}
\pgfpathlineto{\pgfpoint{15.320153\du}{9.949653\du}}
\pgfpathlineto{\pgfpoint{15.333662\du}{9.943812\du}}
\pgfpathlineto{\pgfpoint{15.346440\du}{9.939065\du}}
\pgfpathlineto{\pgfpoint{15.359949\du}{9.933954\du}}
\pgfpathlineto{\pgfpoint{15.373457\du}{9.928478\du}}
\pgfpathlineto{\pgfpoint{15.387331\du}{9.924096\du}}
\pgfpathlineto{\pgfpoint{15.401205\du}{9.918620\du}}
\pgfpathlineto{\pgfpoint{15.415444\du}{9.913874\du}}
\pgfpathlineto{\pgfpoint{15.429682\du}{9.909127\du}}
\pgfpathlineto{\pgfpoint{15.443921\du}{9.904016\du}}
\pgfpathlineto{\pgfpoint{15.458160\du}{9.899270\du}}
\pgfpathlineto{\pgfpoint{15.473494\du}{9.894524\du}}
\pgfpathlineto{\pgfpoint{15.503067\du}{9.885761\du}}
\pgfpathlineto{\pgfpoint{15.534465\du}{9.876634\du}}
\pgfpathlineto{\pgfpoint{15.565498\du}{9.868237\du}}
\pgfpathlineto{\pgfpoint{15.597992\du}{9.859474\du}}
\pgfpathlineto{\pgfpoint{15.629390\du}{9.851442\du}}
\pgfpathlineto{\pgfpoint{15.662614\du}{9.843775\du}}
\pgfpathlineto{\pgfpoint{15.696568\du}{9.836108\du}}
\pgfpathlineto{\pgfpoint{15.730887\du}{9.828441\du}}
\pgfpathlineto{\pgfpoint{15.765571\du}{9.821504\du}}
\pgfpathlineto{\pgfpoint{15.801716\du}{9.814567\du}}
\pgfpathlineto{\pgfpoint{15.836765\du}{9.808726\du}}
\pgfpathlineto{\pgfpoint{15.874005\du}{9.802154\du}}
\pgfpathlineto{\pgfpoint{15.910515\du}{9.796313\du}}
\pgfpathlineto{\pgfpoint{15.947755\du}{9.790471\du}}
\pgfpathlineto{\pgfpoint{15.985360\du}{9.785360\du}}
\pgfpathlineto{\pgfpoint{16.023695\du}{9.779883\du}}
\pgfpathlineto{\pgfpoint{16.062760\du}{9.775137\du}}
\pgfpathlineto{\pgfpoint{16.102191\du}{9.771121\du}}
\pgfpathlineto{\pgfpoint{16.141621\du}{9.767105\du}}
\pgfpathlineto{\pgfpoint{16.181417\du}{9.763454\du}}
\pgfpathlineto{\pgfpoint{16.221577\du}{9.760168\du}}
\pgfpathlineto{\pgfpoint{16.262468\du}{9.757247\du}}
\pgfpathlineto{\pgfpoint{16.303724\du}{9.754326\du}}
\pgfpathlineto{\pgfpoint{16.344615\du}{9.751771\du}}
\pgfpathlineto{\pgfpoint{16.386601\du}{9.749580\du}}
\pgfpathlineto{\pgfpoint{16.428952\du}{9.748485\du}}
\pgfpathlineto{\pgfpoint{16.471668\du}{9.746659\du}}
\pgfpathlineto{\pgfpoint{16.514020\du}{9.745929\du}}
\pgfpathlineto{\pgfpoint{16.557466\du}{9.745564\du}}
\pgfpathlineto{\pgfpoint{16.600183\du}{9.745564\du}}
\pgfpathlineto{\pgfpoint{16.600183\du}{9.745564\du}}
\pgfpathlineto{\pgfpoint{16.600183\du}{9.745564\du}}
\pgfpathlineto{\pgfpoint{16.601278\du}{9.744834\du}}
\pgfpathlineto{\pgfpoint{16.602373\du}{9.744834\du}}
\pgfpathlineto{\pgfpoint{16.603834\du}{9.744834\du}}
\pgfpathlineto{\pgfpoint{16.604929\du}{9.744469\du}}
\pgfpathlineto{\pgfpoint{16.605294\du}{9.743739\du}}
\pgfpathlineto{\pgfpoint{16.606389\du}{9.743739\du}}
\pgfpathlineto{\pgfpoint{16.607119\du}{9.742643\du}}
\pgfpathlineto{\pgfpoint{16.608215\du}{9.741913\du}}
\pgfpathlineto{\pgfpoint{16.609310\du}{9.740818\du}}
\pgfpathlineto{\pgfpoint{16.610040\du}{9.738992\du}}
\pgfpathlineto{\pgfpoint{16.610040\du}{9.736802\du}}
\pgfpathlineto{\pgfpoint{16.610770\du}{9.734976\du}}
\pgfpathlineto{\pgfpoint{16.610040\du}{9.733151\du}}
\pgfpathlineto{\pgfpoint{16.610040\du}{9.731325\du}}
\pgfpathlineto{\pgfpoint{16.609310\du}{9.729500\du}}
\pgfpathlineto{\pgfpoint{16.608215\du}{9.728039\du}}
\pgfpathlineto{\pgfpoint{16.607119\du}{9.727309\du}}
\pgfpathlineto{\pgfpoint{16.606389\du}{9.726579\du}}
\pgfpathlineto{\pgfpoint{16.605294\du}{9.726214\du}}
\pgfpathlineto{\pgfpoint{16.604929\du}{9.725484\du}}
\pgfpathlineto{\pgfpoint{16.603834\du}{9.725119\du}}
\pgfpathlineto{\pgfpoint{16.602373\du}{9.725119\du}}
\pgfpathlineto{\pgfpoint{16.601278\du}{9.725119\du}}
\pgfpathlineto{\pgfpoint{16.600183\du}{9.725119\du}}
\pgfusepath{fill}
\pgfsetbuttcap
\pgfsetmiterjoin
\pgfsetdash{}{0pt}
\definecolor{dialinecolor}{rgb}{0.678431, 0.839216, 0.905882}
\pgfsetfillcolor{dialinecolor}
\pgfpathmoveto{\pgfpoint{18.290581\du}{10.312559\du}}
\pgfpathlineto{\pgfpoint{18.290581\du}{10.304162\du}}
\pgfpathlineto{\pgfpoint{18.289850\du}{10.296495\du}}
\pgfpathlineto{\pgfpoint{18.289120\du}{10.289193\du}}
\pgfpathlineto{\pgfpoint{18.288390\du}{10.280796\du}}
\pgfpathlineto{\pgfpoint{18.287295\du}{10.273129\du}}
\pgfpathlineto{\pgfpoint{18.285104\du}{10.265462\du}}
\pgfpathlineto{\pgfpoint{18.284009\du}{10.258160\du}}
\pgfpathlineto{\pgfpoint{18.281088\du}{10.250493\du}}
\pgfpathlineto{\pgfpoint{18.279263\du}{10.242826\du}}
\pgfpathlineto{\pgfpoint{18.276342\du}{10.235159\du}}
\pgfpathlineto{\pgfpoint{18.273786\du}{10.227492\du}}
\pgfpathlineto{\pgfpoint{18.270500\du}{10.220555\du}}
\pgfpathlineto{\pgfpoint{18.266484\du}{10.212888\du}}
\pgfpathlineto{\pgfpoint{18.263198\du}{10.205221\du}}
\pgfpathlineto{\pgfpoint{18.259182\du}{10.197919\du}}
\pgfpathlineto{\pgfpoint{18.255166\du}{10.190617\du}}
\pgfpathlineto{\pgfpoint{18.250785\du}{10.183315\du}}
\pgfpathlineto{\pgfpoint{18.246039\du}{10.176013\du}}
\pgfpathlineto{\pgfpoint{18.240562\du}{10.169076\du}}
\pgfpathlineto{\pgfpoint{18.235816\du}{10.162139\du}}
\pgfpathlineto{\pgfpoint{18.229974\du}{10.154472\du}}
\pgfpathlineto{\pgfpoint{18.224498\du}{10.147536\du}}
\pgfpathlineto{\pgfpoint{18.218656\du}{10.140599\du}}
\pgfpathlineto{\pgfpoint{18.212450\du}{10.133297\du}}
\pgfpathlineto{\pgfpoint{18.205878\du}{10.127090\du}}
\pgfpathlineto{\pgfpoint{18.198941\du}{10.119788\du}}
\pgfpathlineto{\pgfpoint{18.192369\du}{10.112851\du}}
\pgfpathlineto{\pgfpoint{18.185068\du}{10.106645\du}}
\pgfpathlineto{\pgfpoint{18.178496\du}{10.099343\du}}
\pgfpathlineto{\pgfpoint{18.170464\du}{10.093136\du}}
\pgfpathlineto{\pgfpoint{18.162432\du}{10.086564\du}}
\pgfpathlineto{\pgfpoint{18.155130\du}{10.079628\du}}
\pgfpathlineto{\pgfpoint{18.146367\du}{10.073056\du}}
\pgfpathlineto{\pgfpoint{18.138335\du}{10.066849\du}}
\pgfpathlineto{\pgfpoint{18.129208\du}{10.060277\du}}
\pgfpathlineto{\pgfpoint{18.120445\du}{10.053706\du}}
\pgfpathlineto{\pgfpoint{18.110953\du}{10.047499\du}}
\pgfpathlineto{\pgfpoint{18.101825\du}{10.040927\du}}
\pgfpathlineto{\pgfpoint{18.092698\du}{10.034356\du}}
\pgfpathlineto{\pgfpoint{18.082840\du}{10.028514\du}}
\pgfpathlineto{\pgfpoint{18.072983\du}{10.022307\du}}
\pgfpathlineto{\pgfpoint{18.062760\du}{10.015736\du}}
\pgfpathlineto{\pgfpoint{18.052172\du}{10.009894\du}}
\pgfpathlineto{\pgfpoint{18.042315\du}{10.004053\du}}
\pgfpathlineto{\pgfpoint{18.031727\du}{9.998211\du}}
\pgfpathlineto{\pgfpoint{18.020409\du}{9.991639\du}}
\pgfpathlineto{\pgfpoint{18.009091\du}{9.986528\du}}
\pgfpathlineto{\pgfpoint{17.997408\du}{9.980686\du}}
\pgfpathlineto{\pgfpoint{17.986455\du}{9.974845\du}}
\pgfpathlineto{\pgfpoint{17.974407\du}{9.969003\du}}
\pgfpathlineto{\pgfpoint{17.963089\du}{9.963162\du}}
\pgfpathlineto{\pgfpoint{17.950675\du}{9.958050\du}}
\pgfpathlineto{\pgfpoint{17.938262\du}{9.951844\du}}
\pgfpathlineto{\pgfpoint{17.925484\du}{9.946002\du}}
\pgfpathlineto{\pgfpoint{17.913436\du}{9.940891\du}}
\pgfpathlineto{\pgfpoint{17.900657\du}{9.935779\du}}
\pgfpathlineto{\pgfpoint{17.887149\du}{9.930303\du}}
\pgfpathlineto{\pgfpoint{17.874005\du}{9.925192\du}}
\pgfpathlineto{\pgfpoint{17.860862\du}{9.919715\du}}
\pgfpathlineto{\pgfpoint{17.846623\du}{9.914604\du}}
\pgfpathlineto{\pgfpoint{17.833479\du}{9.909127\du}}
\pgfpathlineto{\pgfpoint{17.819241\du}{9.904746\du}}
\pgfpathlineto{\pgfpoint{17.805367\du}{9.899270\du}}
\pgfpathlineto{\pgfpoint{17.791128\du}{9.894524\du}}
\pgfpathlineto{\pgfpoint{17.777254\du}{9.889412\du}}
\pgfpathlineto{\pgfpoint{17.762286\du}{9.884666\du}}
\pgfpathlineto{\pgfpoint{17.747317\du}{9.879920\du}}
\pgfpathlineto{\pgfpoint{17.733078\du}{9.875539\du}}
\pgfpathlineto{\pgfpoint{17.702410\du}{9.866046\du}}
\pgfpathlineto{\pgfpoint{17.671742\du}{9.857284\du}}
\pgfpathlineto{\pgfpoint{17.640343\du}{9.848521\du}}
\pgfpathlineto{\pgfpoint{17.608580\du}{9.839759\du}}
\pgfpathlineto{\pgfpoint{17.575721\du}{9.831362\du}}
\pgfpathlineto{\pgfpoint{17.542132\du}{9.823695\du}}
\pgfpathlineto{\pgfpoint{17.508178\du}{9.815663\du}}
\pgfpathlineto{\pgfpoint{17.473494\du}{9.808726\du}}
\pgfpathlineto{\pgfpoint{17.438810\du}{9.801789\du}}
\pgfpathlineto{\pgfpoint{17.403395\du}{9.794487\du}}
\pgfpathlineto{\pgfpoint{17.366886\du}{9.788280\du}}
\pgfpathlineto{\pgfpoint{17.330376\du}{9.781709\du}}
\pgfpathlineto{\pgfpoint{17.293501\du}{9.775867\du}}
\pgfpathlineto{\pgfpoint{17.256261\du}{9.770026\du}}
\pgfpathlineto{\pgfpoint{17.217196\du}{9.764914\du}}
\pgfpathlineto{\pgfpoint{17.179226\du}{9.760168\du}}
\pgfpathlineto{\pgfpoint{17.139796\du}{9.755422\du}}
\pgfpathlineto{\pgfpoint{17.101095\du}{9.750675\du}}
\pgfpathlineto{\pgfpoint{17.060935\du}{9.746659\du}}
\pgfpathlineto{\pgfpoint{17.021139\du}{9.742643\du}}
\pgfpathlineto{\pgfpoint{16.980248\du}{9.739723\du}}
\pgfpathlineto{\pgfpoint{16.939723\du}{9.736802\du}}
\pgfpathlineto{\pgfpoint{16.898101\du}{9.733881\du}}
\pgfpathlineto{\pgfpoint{16.856480\du}{9.731325\du}}
\pgfpathlineto{\pgfpoint{16.814859\du}{9.729500\du}}
\pgfpathlineto{\pgfpoint{16.772143\du}{9.728039\du}}
\pgfpathlineto{\pgfpoint{16.730157\du}{9.726579\du}}
\pgfpathlineto{\pgfpoint{16.687076\du}{9.725484\du}}
\pgfpathlineto{\pgfpoint{16.643629\du}{9.725119\du}}
\pgfpathlineto{\pgfpoint{16.600183\du}{9.725119\du}}
\pgfpathlineto{\pgfpoint{16.600183\du}{9.745564\du}}
\pgfpathlineto{\pgfpoint{16.643264\du}{9.745564\du}}
\pgfpathlineto{\pgfpoint{16.686710\du}{9.745929\du}}
\pgfpathlineto{\pgfpoint{16.728697\du}{9.746659\du}}
\pgfpathlineto{\pgfpoint{16.771413\du}{9.748485\du}}
\pgfpathlineto{\pgfpoint{16.814129\du}{9.749580\du}}
\pgfpathlineto{\pgfpoint{16.855750\du}{9.751771\du}}
\pgfpathlineto{\pgfpoint{16.897006\du}{9.754326\du}}
\pgfpathlineto{\pgfpoint{16.937897\du}{9.757247\du}}
\pgfpathlineto{\pgfpoint{16.979153\du}{9.760168\du}}
\pgfpathlineto{\pgfpoint{17.019314\du}{9.763454\du}}
\pgfpathlineto{\pgfpoint{17.059474\du}{9.767105\du}}
\pgfpathlineto{\pgfpoint{17.098540\du}{9.771121\du}}
\pgfpathlineto{\pgfpoint{17.137970\du}{9.775137\du}}
\pgfpathlineto{\pgfpoint{17.176670\du}{9.779883\du}}
\pgfpathlineto{\pgfpoint{17.215371\du}{9.785360\du}}
\pgfpathlineto{\pgfpoint{17.252976\du}{9.790471\du}}
\pgfpathlineto{\pgfpoint{17.290215\du}{9.796313\du}}
\pgfpathlineto{\pgfpoint{17.326725\du}{9.802154\du}}
\pgfpathlineto{\pgfpoint{17.363600\du}{9.808726\du}}
\pgfpathlineto{\pgfpoint{17.399014\du}{9.814567\du}}
\pgfpathlineto{\pgfpoint{17.434794\du}{9.821504\du}}
\pgfpathlineto{\pgfpoint{17.469478\du}{9.828441\du}}
\pgfpathlineto{\pgfpoint{17.503432\du}{9.836108\du}}
\pgfpathlineto{\pgfpoint{17.537751\du}{9.843775\du}}
\pgfpathlineto{\pgfpoint{17.570975\du}{9.851442\du}}
\pgfpathlineto{\pgfpoint{17.603103\du}{9.859474\du}}
\pgfpathlineto{\pgfpoint{17.635232\du}{9.868237\du}}
\pgfpathlineto{\pgfpoint{17.666630\du}{9.876634\du}}
\pgfpathlineto{\pgfpoint{17.697298\du}{9.885761\du}}
\pgfpathlineto{\pgfpoint{17.727236\du}{9.894524\du}}
\pgfpathlineto{\pgfpoint{17.741475\du}{9.899270\du}}
\pgfpathlineto{\pgfpoint{17.756079\du}{9.904016\du}}
\pgfpathlineto{\pgfpoint{17.769953\du}{9.909127\du}}
\pgfpathlineto{\pgfpoint{17.784556\du}{9.913874\du}}
\pgfpathlineto{\pgfpoint{17.798795\du}{9.918620\du}}
\pgfpathlineto{\pgfpoint{17.812669\du}{9.924096\du}}
\pgfpathlineto{\pgfpoint{17.826908\du}{9.928478\du}}
\pgfpathlineto{\pgfpoint{17.840051\du}{9.933954\du}}
\pgfpathlineto{\pgfpoint{17.853560\du}{9.939065\du}}
\pgfpathlineto{\pgfpoint{17.866703\du}{9.943812\du}}
\pgfpathlineto{\pgfpoint{17.879482\du}{9.949653\du}}
\pgfpathlineto{\pgfpoint{17.892260\du}{9.954765\du}}
\pgfpathlineto{\pgfpoint{17.905403\du}{9.960241\du}}
\pgfpathlineto{\pgfpoint{17.917452\du}{9.965352\du}}
\pgfpathlineto{\pgfpoint{17.929865\du}{9.970829\du}}
\pgfpathlineto{\pgfpoint{17.941913\du}{9.976670\du}}
\pgfpathlineto{\pgfpoint{17.954326\du}{9.981782\du}}
\pgfpathlineto{\pgfpoint{17.965644\du}{9.987258\du}}
\pgfpathlineto{\pgfpoint{17.977693\du}{9.993100\du}}
\pgfpathlineto{\pgfpoint{17.988280\du}{9.998941\du}}
\pgfpathlineto{\pgfpoint{18.000329\du}{10.004783\du}}
\pgfpathlineto{\pgfpoint{18.010551\du}{10.010624\du}}
\pgfpathlineto{\pgfpoint{18.021504\du}{10.015736\du}}
\pgfpathlineto{\pgfpoint{18.032457\du}{10.022307\du}}
\pgfpathlineto{\pgfpoint{18.042315\du}{10.028149\du}}
\pgfpathlineto{\pgfpoint{18.052172\du}{10.033991\du}}
\pgfpathlineto{\pgfpoint{18.062395\du}{10.039832\du}}
\pgfpathlineto{\pgfpoint{18.071522\du}{10.045674\du}}
\pgfpathlineto{\pgfpoint{18.081745\du}{10.052245\du}}
\pgfpathlineto{\pgfpoint{18.091238\du}{10.058087\du}}
\pgfpathlineto{\pgfpoint{18.100000\du}{10.064294\du}}
\pgfpathlineto{\pgfpoint{18.108032\du}{10.070135\du}}
\pgfpathlineto{\pgfpoint{18.116794\du}{10.076707\du}}
\pgfpathlineto{\pgfpoint{18.125192\du}{10.082913\du}}
\pgfpathlineto{\pgfpoint{18.133954\du}{10.089485\du}}
\pgfpathlineto{\pgfpoint{18.141621\du}{10.096057\du}}
\pgfpathlineto{\pgfpoint{18.149653\du}{10.101899\du}}
\pgfpathlineto{\pgfpoint{18.156955\du}{10.108105\du}}
\pgfpathlineto{\pgfpoint{18.164622\du}{10.115407\du}}
\pgfpathlineto{\pgfpoint{18.171194\du}{10.121249\du}}
\pgfpathlineto{\pgfpoint{18.178496\du}{10.128185\du}}
\pgfpathlineto{\pgfpoint{18.184337\du}{10.134757\du}}
\pgfpathlineto{\pgfpoint{18.191274\du}{10.140964\du}}
\pgfpathlineto{\pgfpoint{18.197846\du}{10.147536\du}}
\pgfpathlineto{\pgfpoint{18.202957\du}{10.154472\du}}
\pgfpathlineto{\pgfpoint{18.208434\du}{10.161044\du}}
\pgfpathlineto{\pgfpoint{18.214640\du}{10.167981\du}}
\pgfpathlineto{\pgfpoint{18.219387\du}{10.174188\du}}
\pgfpathlineto{\pgfpoint{18.224498\du}{10.180759\du}}
\pgfpathlineto{\pgfpoint{18.229244\du}{10.187696\du}}
\pgfpathlineto{\pgfpoint{18.233625\du}{10.194268\du}}
\pgfpathlineto{\pgfpoint{18.238007\du}{10.201205\du}}
\pgfpathlineto{\pgfpoint{18.240927\du}{10.207777\du}}
\pgfpathlineto{\pgfpoint{18.244943\du}{10.214713\du}}
\pgfpathlineto{\pgfpoint{18.247864\du}{10.221650\du}}
\pgfpathlineto{\pgfpoint{18.251880\du}{10.228952\du}}
\pgfpathlineto{\pgfpoint{18.254436\du}{10.235159\du}}
\pgfpathlineto{\pgfpoint{18.257357\du}{10.242096\du}}
\pgfpathlineto{\pgfpoint{18.259912\du}{10.249398\du}}
\pgfpathlineto{\pgfpoint{18.262103\du}{10.256334\du}}
\pgfpathlineto{\pgfpoint{18.263563\du}{10.263271\du}}
\pgfpathlineto{\pgfpoint{18.265754\du}{10.270208\du}}
\pgfpathlineto{\pgfpoint{18.266484\du}{10.277510\du}}
\pgfpathlineto{\pgfpoint{18.267579\du}{10.283717\du}}
\pgfpathlineto{\pgfpoint{18.269405\du}{10.291384\du}}
\pgfpathlineto{\pgfpoint{18.269770\du}{10.298321\du}}
\pgfpathlineto{\pgfpoint{18.269770\du}{10.305257\du}}
\pgfpathlineto{\pgfpoint{18.270500\du}{10.312559\du}}
\pgfpathlineto{\pgfpoint{18.290581\du}{10.312559\du}}
\pgfusepath{fill}
\pgfsetbuttcap
\pgfsetmiterjoin
\pgfsetdash{}{0pt}
\definecolor{dialinecolor}{rgb}{0.027451, 0.486275, 0.682353}
\pgfsetfillcolor{dialinecolor}
\pgfpathmoveto{\pgfpoint{14.914531\du}{9.503140\du}}
\pgfpathlineto{\pgfpoint{14.914531\du}{10.327528\du}}
\pgfpathlineto{\pgfpoint{18.279993\du}{10.327528\du}}
\pgfpathlineto{\pgfpoint{18.280723\du}{9.503870\du}}
\pgfpathlineto{\pgfpoint{14.914531\du}{9.503140\du}}
\pgfusepath{fill}
\pgfsetbuttcap
\pgfsetmiterjoin
\pgfsetdash{}{0pt}
\definecolor{dialinecolor}{rgb}{0.235294, 0.686275, 0.894118}
\pgfsetfillcolor{dialinecolor}
\pgfpathmoveto{\pgfpoint{18.279993\du}{9.487441\du}}
\pgfpathlineto{\pgfpoint{18.278532\du}{9.517379\du}}
\pgfpathlineto{\pgfpoint{18.271230\du}{9.546586\du}}
\pgfpathlineto{\pgfpoint{18.261008\du}{9.575794\du}}
\pgfpathlineto{\pgfpoint{18.246404\du}{9.603907\du}}
\pgfpathlineto{\pgfpoint{18.227054\du}{9.632019\du}}
\pgfpathlineto{\pgfpoint{18.205148\du}{9.659401\du}}
\pgfpathlineto{\pgfpoint{18.178496\du}{9.685688\du}}
\pgfpathlineto{\pgfpoint{18.147828\du}{9.711975\du}}
\pgfpathlineto{\pgfpoint{18.114969\du}{9.737897\du}}
\pgfpathlineto{\pgfpoint{18.077729\du}{9.763089\du}}
\pgfpathlineto{\pgfpoint{18.036838\du}{9.787185\du}}
\pgfpathlineto{\pgfpoint{17.993027\du}{9.810551\du}}
\pgfpathlineto{\pgfpoint{17.946659\du}{9.832822\du}}
\pgfpathlineto{\pgfpoint{17.896276\du}{9.854728\du}}
\pgfpathlineto{\pgfpoint{17.843337\du}{9.875904\du}}
\pgfpathlineto{\pgfpoint{17.787842\du}{9.895984\du}}
\pgfpathlineto{\pgfpoint{17.729792\du}{9.914604\du}}
\pgfpathlineto{\pgfpoint{17.669186\du}{9.933224\du}}
\pgfpathlineto{\pgfpoint{17.605294\du}{9.950383\du}}
\pgfpathlineto{\pgfpoint{17.540307\du}{9.966083\du}}
\pgfpathlineto{\pgfpoint{17.471668\du}{9.981417\du}}
\pgfpathlineto{\pgfpoint{17.400840\du}{9.995290\du}}
\pgfpathlineto{\pgfpoint{17.328916\du}{10.008069\du}}
\pgfpathlineto{\pgfpoint{17.254071\du}{10.019387\du}}
\pgfpathlineto{\pgfpoint{17.178131\du}{10.029974\du}}
\pgfpathlineto{\pgfpoint{17.099635\du}{10.039102\du}}
\pgfpathlineto{\pgfpoint{17.020044\du}{10.046769\du}}
\pgfpathlineto{\pgfpoint{16.938992\du}{10.053341\du}}
\pgfpathlineto{\pgfpoint{16.855750\du}{10.058452\du}}
\pgfpathlineto{\pgfpoint{16.772143\du}{10.062103\du}}
\pgfpathlineto{\pgfpoint{16.686710\du}{10.063928\du}}
\pgfpathlineto{\pgfpoint{16.600183\du}{10.065024\du}}
\pgfpathlineto{\pgfpoint{16.514020\du}{10.063928\du}}
\pgfpathlineto{\pgfpoint{16.428222\du}{10.062103\du}}
\pgfpathlineto{\pgfpoint{16.344615\du}{10.058452\du}}
\pgfpathlineto{\pgfpoint{16.261738\du}{10.053341\du}}
\pgfpathlineto{\pgfpoint{16.180321\du}{10.046769\du}}
\pgfpathlineto{\pgfpoint{16.100730\du}{10.039102\du}}
\pgfpathlineto{\pgfpoint{16.022965\du}{10.029974\du}}
\pgfpathlineto{\pgfpoint{15.946294\du}{10.019387\du}}
\pgfpathlineto{\pgfpoint{15.872180\du}{10.008069\du}}
\pgfpathlineto{\pgfpoint{15.799525\du}{9.995290\du}}
\pgfpathlineto{\pgfpoint{15.729062\du}{9.981417\du}}
\pgfpathlineto{\pgfpoint{15.660424\du}{9.966083\du}}
\pgfpathlineto{\pgfpoint{15.594706\du}{9.950383\du}}
\pgfpathlineto{\pgfpoint{15.531179\du}{9.933224\du}}
\pgfpathlineto{\pgfpoint{15.470208\du}{9.914604\du}}
\pgfpathlineto{\pgfpoint{15.411793\du}{9.895984\du}}
\pgfpathlineto{\pgfpoint{15.356663\du}{9.875904\du}}
\pgfpathlineto{\pgfpoint{15.303724\du}{9.854728\du}}
\pgfpathlineto{\pgfpoint{15.253706\du}{9.832822\du}}
\pgfpathlineto{\pgfpoint{15.206608\du}{9.810551\du}}
\pgfpathlineto{\pgfpoint{15.163162\du}{9.787185\du}}
\pgfpathlineto{\pgfpoint{15.122271\du}{9.763089\du}}
\pgfpathlineto{\pgfpoint{15.085031\du}{9.737897\du}}
\pgfpathlineto{\pgfpoint{15.051807\du}{9.711975\du}}
\pgfpathlineto{\pgfpoint{15.021504\du}{9.685688\du}}
\pgfpathlineto{\pgfpoint{14.994852\du}{9.659401\du}}
\pgfpathlineto{\pgfpoint{14.972946\du}{9.632019\du}}
\pgfpathlineto{\pgfpoint{14.953596\du}{9.603907\du}}
\pgfpathlineto{\pgfpoint{14.938992\du}{9.575794\du}}
\pgfpathlineto{\pgfpoint{14.928405\du}{9.546586\du}}
\pgfpathlineto{\pgfpoint{14.921468\du}{9.517379\du}}
\pgfpathlineto{\pgfpoint{14.919642\du}{9.487441\du}}
\pgfpathlineto{\pgfpoint{14.921468\du}{9.458233\du}}
\pgfpathlineto{\pgfpoint{14.928405\du}{9.428295\du}}
\pgfpathlineto{\pgfpoint{14.938992\du}{9.399817\du}}
\pgfpathlineto{\pgfpoint{14.953596\du}{9.371705\du}}
\pgfpathlineto{\pgfpoint{14.972946\du}{9.343593\du}}
\pgfpathlineto{\pgfpoint{14.994852\du}{9.315845\du}}
\pgfpathlineto{\pgfpoint{15.021504\du}{9.289193\du}}
\pgfpathlineto{\pgfpoint{15.051807\du}{9.262906\du}}
\pgfpathlineto{\pgfpoint{15.085031\du}{9.237714\du}}
\pgfpathlineto{\pgfpoint{15.122271\du}{9.212523\du}}
\pgfpathlineto{\pgfpoint{15.163162\du}{9.188426\du}}
\pgfpathlineto{\pgfpoint{15.206608\du}{9.165060\du}}
\pgfpathlineto{\pgfpoint{15.253706\du}{9.142059\du}}
\pgfpathlineto{\pgfpoint{15.303724\du}{9.120518\du}}
\pgfpathlineto{\pgfpoint{15.356663\du}{9.099343\du}}
\pgfpathlineto{\pgfpoint{15.411793\du}{9.079628\du}}
\pgfpathlineto{\pgfpoint{15.470208\du}{9.060277\du}}
\pgfpathlineto{\pgfpoint{15.531179\du}{9.042023\du}}
\pgfpathlineto{\pgfpoint{15.594706\du}{9.025228\du}}
\pgfpathlineto{\pgfpoint{15.660424\du}{9.008799\du}}
\pgfpathlineto{\pgfpoint{15.729062\du}{8.993465\du}}
\pgfpathlineto{\pgfpoint{15.799525\du}{8.979956\du}}
\pgfpathlineto{\pgfpoint{15.872180\du}{8.967178\du}}
\pgfpathlineto{\pgfpoint{15.946294\du}{8.955495\du}}
\pgfpathlineto{\pgfpoint{16.022965\du}{8.945637\du}}
\pgfpathlineto{\pgfpoint{16.100730\du}{8.936145\du}}
\pgfpathlineto{\pgfpoint{16.180321\du}{8.928478\du}}
\pgfpathlineto{\pgfpoint{16.261738\du}{8.921906\du}}
\pgfpathlineto{\pgfpoint{16.344615\du}{8.916794\du}}
\pgfpathlineto{\pgfpoint{16.428222\du}{8.913509\du}}
\pgfpathlineto{\pgfpoint{16.514020\du}{8.910953\du}}
\pgfpathlineto{\pgfpoint{16.600183\du}{8.910588\du}}
\pgfpathlineto{\pgfpoint{16.686710\du}{8.910953\du}}
\pgfpathlineto{\pgfpoint{16.772143\du}{8.913509\du}}
\pgfpathlineto{\pgfpoint{16.855750\du}{8.916794\du}}
\pgfpathlineto{\pgfpoint{16.938992\du}{8.921906\du}}
\pgfpathlineto{\pgfpoint{17.020044\du}{8.928478\du}}
\pgfpathlineto{\pgfpoint{17.099635\du}{8.936145\du}}
\pgfpathlineto{\pgfpoint{17.178131\du}{8.945637\du}}
\pgfpathlineto{\pgfpoint{17.254071\du}{8.955495\du}}
\pgfpathlineto{\pgfpoint{17.328916\du}{8.967178\du}}
\pgfpathlineto{\pgfpoint{17.400840\du}{8.979956\du}}
\pgfpathlineto{\pgfpoint{17.471668\du}{8.993465\du}}
\pgfpathlineto{\pgfpoint{17.540307\du}{9.008799\du}}
\pgfpathlineto{\pgfpoint{17.605294\du}{9.025228\du}}
\pgfpathlineto{\pgfpoint{17.669186\du}{9.042023\du}}
\pgfpathlineto{\pgfpoint{17.729792\du}{9.060277\du}}
\pgfpathlineto{\pgfpoint{17.787842\du}{9.079628\du}}
\pgfpathlineto{\pgfpoint{17.843337\du}{9.099343\du}}
\pgfpathlineto{\pgfpoint{17.896276\du}{9.120518\du}}
\pgfpathlineto{\pgfpoint{17.946659\du}{9.142059\du}}
\pgfpathlineto{\pgfpoint{17.993027\du}{9.165060\du}}
\pgfpathlineto{\pgfpoint{18.036838\du}{9.188426\du}}
\pgfpathlineto{\pgfpoint{18.077729\du}{9.212523\du}}
\pgfpathlineto{\pgfpoint{18.114969\du}{9.237714\du}}
\pgfpathlineto{\pgfpoint{18.147828\du}{9.262906\du}}
\pgfpathlineto{\pgfpoint{18.178496\du}{9.289193\du}}
\pgfpathlineto{\pgfpoint{18.205148\du}{9.315845\du}}
\pgfpathlineto{\pgfpoint{18.227054\du}{9.343593\du}}
\pgfpathlineto{\pgfpoint{18.246404\du}{9.371705\du}}
\pgfpathlineto{\pgfpoint{18.261008\du}{9.399817\du}}
\pgfpathlineto{\pgfpoint{18.271230\du}{9.428295\du}}
\pgfpathlineto{\pgfpoint{18.278532\du}{9.458233\du}}
\pgfpathlineto{\pgfpoint{18.279993\du}{9.487441\du}}
\pgfusepath{fill}
\pgfsetbuttcap
\pgfsetmiterjoin
\pgfsetdash{}{0pt}
\definecolor{dialinecolor}{rgb}{0.678431, 0.839216, 0.905882}
\pgfsetfillcolor{dialinecolor}
\pgfpathmoveto{\pgfpoint{16.600183\du}{10.074881\du}}
\pgfpathlineto{\pgfpoint{16.600183\du}{10.074881\du}}
\pgfpathlineto{\pgfpoint{16.643629\du}{10.074881\du}}
\pgfpathlineto{\pgfpoint{16.687076\du}{10.074151\du}}
\pgfpathlineto{\pgfpoint{16.730157\du}{10.073056\du}}
\pgfpathlineto{\pgfpoint{16.772143\du}{10.071961\du}}
\pgfpathlineto{\pgfpoint{16.814859\du}{10.070135\du}}
\pgfpathlineto{\pgfpoint{16.856480\du}{10.068310\du}}
\pgfpathlineto{\pgfpoint{16.898101\du}{10.066119\du}}
\pgfpathlineto{\pgfpoint{16.939723\du}{10.063198\du}}
\pgfpathlineto{\pgfpoint{16.980248\du}{10.060277\du}}
\pgfpathlineto{\pgfpoint{17.021139\du}{10.057357\du}}
\pgfpathlineto{\pgfpoint{17.060935\du}{10.053341\du}}
\pgfpathlineto{\pgfpoint{17.101095\du}{10.049325\du}}
\pgfpathlineto{\pgfpoint{17.139796\du}{10.044578\du}}
\pgfpathlineto{\pgfpoint{17.179226\du}{10.039832\du}}
\pgfpathlineto{\pgfpoint{17.217196\du}{10.035086\du}}
\pgfpathlineto{\pgfpoint{17.256261\du}{10.029974\du}}
\pgfpathlineto{\pgfpoint{17.293501\du}{10.024133\du}}
\pgfpathlineto{\pgfpoint{17.330376\du}{10.018291\du}}
\pgfpathlineto{\pgfpoint{17.366886\du}{10.011720\du}}
\pgfpathlineto{\pgfpoint{17.403395\du}{10.005148\du}}
\pgfpathlineto{\pgfpoint{17.438810\du}{9.998211\du}}
\pgfpathlineto{\pgfpoint{17.473494\du}{9.991274\du}}
\pgfpathlineto{\pgfpoint{17.508178\du}{9.984337\du}}
\pgfpathlineto{\pgfpoint{17.542132\du}{9.976670\du}}
\pgfpathlineto{\pgfpoint{17.575721\du}{9.968273\du}}
\pgfpathlineto{\pgfpoint{17.608580\du}{9.960241\du}}
\pgfpathlineto{\pgfpoint{17.640343\du}{9.951844\du}}
\pgfpathlineto{\pgfpoint{17.671742\du}{9.942716\du}}
\pgfpathlineto{\pgfpoint{17.687076\du}{9.938700\du}}
\pgfpathlineto{\pgfpoint{17.702410\du}{9.933954\du}}
\pgfpathlineto{\pgfpoint{17.718474\du}{9.929208\du}}
\pgfpathlineto{\pgfpoint{17.733078\du}{9.924461\du}}
\pgfpathlineto{\pgfpoint{17.747317\du}{9.919715\du}}
\pgfpathlineto{\pgfpoint{17.762286\du}{9.915334\du}}
\pgfpathlineto{\pgfpoint{17.777254\du}{9.910588\du}}
\pgfpathlineto{\pgfpoint{17.791128\du}{9.905111\du}}
\pgfpathlineto{\pgfpoint{17.805367\du}{9.900365\du}}
\pgfpathlineto{\pgfpoint{17.819241\du}{9.895254\du}}
\pgfpathlineto{\pgfpoint{17.833479\du}{9.890507\du}}
\pgfpathlineto{\pgfpoint{17.846623\du}{9.885396\du}}
\pgfpathlineto{\pgfpoint{17.860862\du}{9.879920\du}}
\pgfpathlineto{\pgfpoint{17.874005\du}{9.874808\du}}
\pgfpathlineto{\pgfpoint{17.887149\du}{9.869697\du}}
\pgfpathlineto{\pgfpoint{17.900657\du}{9.864221\du}}
\pgfpathlineto{\pgfpoint{17.913436\du}{9.859109\du}}
\pgfpathlineto{\pgfpoint{17.925484\du}{9.853633\du}}
\pgfpathlineto{\pgfpoint{17.938262\du}{9.847791\du}}
\pgfpathlineto{\pgfpoint{17.950675\du}{9.841950\du}}
\pgfpathlineto{\pgfpoint{17.963089\du}{9.836838\du}}
\pgfpathlineto{\pgfpoint{17.974407\du}{9.830997\du}}
\pgfpathlineto{\pgfpoint{17.986455\du}{9.825155\du}}
\pgfpathlineto{\pgfpoint{17.997408\du}{9.819314\du}}
\pgfpathlineto{\pgfpoint{18.009091\du}{9.813837\du}}
\pgfpathlineto{\pgfpoint{18.020409\du}{9.807996\du}}
\pgfpathlineto{\pgfpoint{18.031727\du}{9.801789\du}}
\pgfpathlineto{\pgfpoint{18.042315\du}{9.795947\du}}
\pgfpathlineto{\pgfpoint{18.052172\du}{9.790106\du}}
\pgfpathlineto{\pgfpoint{18.062760\du}{9.784264\du}}
\pgfpathlineto{\pgfpoint{18.072983\du}{9.777693\du}}
\pgfpathlineto{\pgfpoint{18.082840\du}{9.771121\du}}
\pgfpathlineto{\pgfpoint{18.092698\du}{9.765279\du}}
\pgfpathlineto{\pgfpoint{18.101825\du}{9.759073\du}}
\pgfpathlineto{\pgfpoint{18.110953\du}{9.752501\du}}
\pgfpathlineto{\pgfpoint{18.120445\du}{9.745929\du}}
\pgfpathlineto{\pgfpoint{18.129208\du}{9.739723\du}}
\pgfpathlineto{\pgfpoint{18.138335\du}{9.733151\du}}
\pgfpathlineto{\pgfpoint{18.146367\du}{9.726579\du}}
\pgfpathlineto{\pgfpoint{18.155130\du}{9.720372\du}}
\pgfpathlineto{\pgfpoint{18.162432\du}{9.713801\du}}
\pgfpathlineto{\pgfpoint{18.170464\du}{9.706864\du}}
\pgfpathlineto{\pgfpoint{18.178496\du}{9.700292\du}}
\pgfpathlineto{\pgfpoint{18.185068\du}{9.693355\du}}
\pgfpathlineto{\pgfpoint{18.192369\du}{9.686783\du}}
\pgfpathlineto{\pgfpoint{18.198941\du}{9.679847\du}}
\pgfpathlineto{\pgfpoint{18.205878\du}{9.672910\du}}
\pgfpathlineto{\pgfpoint{18.212450\du}{9.666338\du}}
\pgfpathlineto{\pgfpoint{18.218656\du}{9.659401\du}}
\pgfpathlineto{\pgfpoint{18.224498\du}{9.652464\du}}
\pgfpathlineto{\pgfpoint{18.229974\du}{9.645528\du}}
\pgfpathlineto{\pgfpoint{18.235816\du}{9.637861\du}}
\pgfpathlineto{\pgfpoint{18.240562\du}{9.630924\du}}
\pgfpathlineto{\pgfpoint{18.246039\du}{9.623622\du}}
\pgfpathlineto{\pgfpoint{18.250785\du}{9.616685\du}}
\pgfpathlineto{\pgfpoint{18.255166\du}{9.609018\du}}
\pgfpathlineto{\pgfpoint{18.259182\du}{9.602081\du}}
\pgfpathlineto{\pgfpoint{18.263198\du}{9.594414\du}}
\pgfpathlineto{\pgfpoint{18.266484\du}{9.586747\du}}
\pgfpathlineto{\pgfpoint{18.270500\du}{9.579445\du}}
\pgfpathlineto{\pgfpoint{18.273786\du}{9.572143\du}}
\pgfpathlineto{\pgfpoint{18.276342\du}{9.564841\du}}
\pgfpathlineto{\pgfpoint{18.279263\du}{9.557174\du}}
\pgfpathlineto{\pgfpoint{18.281088\du}{9.549507\du}}
\pgfpathlineto{\pgfpoint{18.284009\du}{9.541840\du}}
\pgfpathlineto{\pgfpoint{18.285104\du}{9.534173\du}}
\pgfpathlineto{\pgfpoint{18.287295\du}{9.526506\du}}
\pgfpathlineto{\pgfpoint{18.288390\du}{9.519204\du}}
\pgfpathlineto{\pgfpoint{18.289120\du}{9.510807\du}}
\pgfpathlineto{\pgfpoint{18.289850\du}{9.503140\du}}
\pgfpathlineto{\pgfpoint{18.290581\du}{9.495473\du}}
\pgfpathlineto{\pgfpoint{18.290581\du}{9.487441\du}}
\pgfpathlineto{\pgfpoint{18.270500\du}{9.487441\du}}
\pgfpathlineto{\pgfpoint{18.269770\du}{9.494378\du}}
\pgfpathlineto{\pgfpoint{18.269770\du}{9.502045\du}}
\pgfpathlineto{\pgfpoint{18.269405\du}{9.508981\du}}
\pgfpathlineto{\pgfpoint{18.267579\du}{9.515918\du}}
\pgfpathlineto{\pgfpoint{18.266484\du}{9.523220\du}}
\pgfpathlineto{\pgfpoint{18.265754\du}{9.529427\du}}
\pgfpathlineto{\pgfpoint{18.263563\du}{9.536729\du}}
\pgfpathlineto{\pgfpoint{18.262103\du}{9.543666\du}}
\pgfpathlineto{\pgfpoint{18.259912\du}{9.550602\du}}
\pgfpathlineto{\pgfpoint{18.257357\du}{9.557539\du}}
\pgfpathlineto{\pgfpoint{18.254436\du}{9.564841\du}}
\pgfpathlineto{\pgfpoint{18.251880\du}{9.571048\du}}
\pgfpathlineto{\pgfpoint{18.247864\du}{9.577985\du}}
\pgfpathlineto{\pgfpoint{18.244943\du}{9.585287\du}}
\pgfpathlineto{\pgfpoint{18.240927\du}{9.592223\du}}
\pgfpathlineto{\pgfpoint{18.238007\du}{9.598430\du}}
\pgfpathlineto{\pgfpoint{18.233625\du}{9.605732\du}}
\pgfpathlineto{\pgfpoint{18.229244\du}{9.611939\du}}
\pgfpathlineto{\pgfpoint{18.224498\du}{9.619241\du}}
\pgfpathlineto{\pgfpoint{18.219387\du}{9.625447\du}}
\pgfpathlineto{\pgfpoint{18.214640\du}{9.632384\du}}
\pgfpathlineto{\pgfpoint{18.208434\du}{9.638956\du}}
\pgfpathlineto{\pgfpoint{18.202957\du}{9.645528\du}}
\pgfpathlineto{\pgfpoint{18.197846\du}{9.652464\du}}
\pgfpathlineto{\pgfpoint{18.191274\du}{9.659036\du}}
\pgfpathlineto{\pgfpoint{18.184337\du}{9.665243\du}}
\pgfpathlineto{\pgfpoint{18.178496\du}{9.671815\du}}
\pgfpathlineto{\pgfpoint{18.171194\du}{9.678751\du}}
\pgfpathlineto{\pgfpoint{18.164622\du}{9.685323\du}}
\pgfpathlineto{\pgfpoint{18.156955\du}{9.691530\du}}
\pgfpathlineto{\pgfpoint{18.149653\du}{9.698101\du}}
\pgfpathlineto{\pgfpoint{18.141621\du}{9.703943\du}}
\pgfpathlineto{\pgfpoint{18.133954\du}{9.710515\du}}
\pgfpathlineto{\pgfpoint{18.125192\du}{9.716721\du}}
\pgfpathlineto{\pgfpoint{18.116794\du}{9.723293\du}}
\pgfpathlineto{\pgfpoint{18.108032\du}{9.729500\du}}
\pgfpathlineto{\pgfpoint{18.100000\du}{9.735341\du}}
\pgfpathlineto{\pgfpoint{18.091238\du}{9.741913\du}}
\pgfpathlineto{\pgfpoint{18.081745\du}{9.747755\du}}
\pgfpathlineto{\pgfpoint{18.071522\du}{9.754326\du}}
\pgfpathlineto{\pgfpoint{18.062395\du}{9.760168\du}}
\pgfpathlineto{\pgfpoint{18.052172\du}{9.766009\du}}
\pgfpathlineto{\pgfpoint{18.042315\du}{9.771851\du}}
\pgfpathlineto{\pgfpoint{18.032457\du}{9.777693\du}}
\pgfpathlineto{\pgfpoint{18.021504\du}{9.784264\du}}
\pgfpathlineto{\pgfpoint{18.010551\du}{9.789376\du}}
\pgfpathlineto{\pgfpoint{18.000329\du}{9.795217\du}}
\pgfpathlineto{\pgfpoint{17.988280\du}{9.801059\du}}
\pgfpathlineto{\pgfpoint{17.977693\du}{9.806900\du}}
\pgfpathlineto{\pgfpoint{17.965644\du}{9.812742\du}}
\pgfpathlineto{\pgfpoint{17.954326\du}{9.817853\du}}
\pgfpathlineto{\pgfpoint{17.941913\du}{9.823330\du}}
\pgfpathlineto{\pgfpoint{17.929865\du}{9.829171\du}}
\pgfpathlineto{\pgfpoint{17.917452\du}{9.834283\du}}
\pgfpathlineto{\pgfpoint{17.905403\du}{9.839759\du}}
\pgfpathlineto{\pgfpoint{17.892260\du}{9.844870\du}}
\pgfpathlineto{\pgfpoint{17.879482\du}{9.850347\du}}
\pgfpathlineto{\pgfpoint{17.866703\du}{9.856188\du}}
\pgfpathlineto{\pgfpoint{17.853560\du}{9.860570\du}}
\pgfpathlineto{\pgfpoint{17.840051\du}{9.866046\du}}
\pgfpathlineto{\pgfpoint{17.826908\du}{9.871157\du}}
\pgfpathlineto{\pgfpoint{17.812669\du}{9.875904\du}}
\pgfpathlineto{\pgfpoint{17.798795\du}{9.881380\du}}
\pgfpathlineto{\pgfpoint{17.784556\du}{9.885761\du}}
\pgfpathlineto{\pgfpoint{17.769953\du}{9.890507\du}}
\pgfpathlineto{\pgfpoint{17.756079\du}{9.895984\du}}
\pgfpathlineto{\pgfpoint{17.741475\du}{9.900365\du}}
\pgfpathlineto{\pgfpoint{17.727236\du}{9.905111\du}}
\pgfpathlineto{\pgfpoint{17.711537\du}{9.909858\du}}
\pgfpathlineto{\pgfpoint{17.697298\du}{9.913874\du}}
\pgfpathlineto{\pgfpoint{17.681599\du}{9.918620\du}}
\pgfpathlineto{\pgfpoint{17.666630\du}{9.923366\du}}
\pgfpathlineto{\pgfpoint{17.635232\du}{9.931398\du}}
\pgfpathlineto{\pgfpoint{17.603103\du}{9.940161\du}}
\pgfpathlineto{\pgfpoint{17.570975\du}{9.948558\du}}
\pgfpathlineto{\pgfpoint{17.537751\du}{9.956225\du}}
\pgfpathlineto{\pgfpoint{17.503432\du}{9.963892\du}}
\pgfpathlineto{\pgfpoint{17.469478\du}{9.971194\du}}
\pgfpathlineto{\pgfpoint{17.434794\du}{9.978496\du}}
\pgfpathlineto{\pgfpoint{17.399014\du}{9.985433\du}}
\pgfpathlineto{\pgfpoint{17.363600\du}{9.991639\du}}
\pgfpathlineto{\pgfpoint{17.326725\du}{9.997481\du}}
\pgfpathlineto{\pgfpoint{17.290215\du}{10.003687\du}}
\pgfpathlineto{\pgfpoint{17.252976\du}{10.009529\du}}
\pgfpathlineto{\pgfpoint{17.215371\du}{10.014640\du}}
\pgfpathlineto{\pgfpoint{17.176670\du}{10.019752\du}}
\pgfpathlineto{\pgfpoint{17.137970\du}{10.024498\du}}
\pgfpathlineto{\pgfpoint{17.098540\du}{10.028514\du}}
\pgfpathlineto{\pgfpoint{17.059474\du}{10.032895\du}}
\pgfpathlineto{\pgfpoint{17.019314\du}{10.036181\du}}
\pgfpathlineto{\pgfpoint{16.979153\du}{10.039832\du}}
\pgfpathlineto{\pgfpoint{16.937897\du}{10.042753\du}}
\pgfpathlineto{\pgfpoint{16.897006\du}{10.045674\du}}
\pgfpathlineto{\pgfpoint{16.855750\du}{10.047864\du}}
\pgfpathlineto{\pgfpoint{16.814129\du}{10.050420\du}}
\pgfpathlineto{\pgfpoint{16.771413\du}{10.051515\du}}
\pgfpathlineto{\pgfpoint{16.728697\du}{10.053341\du}}
\pgfpathlineto{\pgfpoint{16.686710\du}{10.053706\du}}
\pgfpathlineto{\pgfpoint{16.643264\du}{10.054436\du}}
\pgfpathlineto{\pgfpoint{16.600183\du}{10.054436\du}}
\pgfpathlineto{\pgfpoint{16.600183\du}{10.054436\du}}
\pgfpathlineto{\pgfpoint{16.600183\du}{10.054436\du}}
\pgfpathlineto{\pgfpoint{16.599452\du}{10.055166\du}}
\pgfpathlineto{\pgfpoint{16.597627\du}{10.055166\du}}
\pgfpathlineto{\pgfpoint{16.596532\du}{10.055166\du}}
\pgfpathlineto{\pgfpoint{16.595801\du}{10.055531\du}}
\pgfpathlineto{\pgfpoint{16.595436\du}{10.056261\du}}
\pgfpathlineto{\pgfpoint{16.593976\du}{10.056261\du}}
\pgfpathlineto{\pgfpoint{16.593246\du}{10.057357\du}}
\pgfpathlineto{\pgfpoint{16.592516\du}{10.058087\du}}
\pgfpathlineto{\pgfpoint{16.591420\du}{10.059182\du}}
\pgfpathlineto{\pgfpoint{16.590690\du}{10.061008\du}}
\pgfpathlineto{\pgfpoint{16.590690\du}{10.063198\du}}
\pgfpathlineto{\pgfpoint{16.589960\du}{10.065024\du}}
\pgfpathlineto{\pgfpoint{16.590690\du}{10.066849\du}}
\pgfpathlineto{\pgfpoint{16.590690\du}{10.068310\du}}
\pgfpathlineto{\pgfpoint{16.591420\du}{10.070135\du}}
\pgfpathlineto{\pgfpoint{16.592516\du}{10.071961\du}}
\pgfpathlineto{\pgfpoint{16.593246\du}{10.072691\du}}
\pgfpathlineto{\pgfpoint{16.593976\du}{10.073056\du}}
\pgfpathlineto{\pgfpoint{16.595436\du}{10.073786\du}}
\pgfpathlineto{\pgfpoint{16.595801\du}{10.074151\du}}
\pgfpathlineto{\pgfpoint{16.596532\du}{10.074881\du}}
\pgfpathlineto{\pgfpoint{16.597627\du}{10.074881\du}}
\pgfpathlineto{\pgfpoint{16.599452\du}{10.074881\du}}
\pgfpathlineto{\pgfpoint{16.600183\du}{10.074881\du}}
\pgfusepath{fill}
\pgfsetbuttcap
\pgfsetmiterjoin
\pgfsetdash{}{0pt}
\definecolor{dialinecolor}{rgb}{0.678431, 0.839216, 0.905882}
\pgfsetfillcolor{dialinecolor}
\pgfpathmoveto{\pgfpoint{14.909419\du}{9.487441\du}}
\pgfpathlineto{\pgfpoint{14.909419\du}{9.487441\du}}
\pgfpathlineto{\pgfpoint{14.909419\du}{9.495473\du}}
\pgfpathlineto{\pgfpoint{14.909785\du}{9.503140\du}}
\pgfpathlineto{\pgfpoint{14.910515\du}{9.510807\du}}
\pgfpathlineto{\pgfpoint{14.911610\du}{9.519204\du}}
\pgfpathlineto{\pgfpoint{14.912705\du}{9.526506\du}}
\pgfpathlineto{\pgfpoint{14.914531\du}{9.534173\du}}
\pgfpathlineto{\pgfpoint{14.916356\du}{9.541840\du}}
\pgfpathlineto{\pgfpoint{14.918547\du}{9.549507\du}}
\pgfpathlineto{\pgfpoint{14.920737\du}{9.557174\du}}
\pgfpathlineto{\pgfpoint{14.923293\du}{9.564841\du}}
\pgfpathlineto{\pgfpoint{14.926214\du}{9.572143\du}}
\pgfpathlineto{\pgfpoint{14.929865\du}{9.579445\du}}
\pgfpathlineto{\pgfpoint{14.933151\du}{9.586747\du}}
\pgfpathlineto{\pgfpoint{14.936802\du}{9.594414\du}}
\pgfpathlineto{\pgfpoint{14.941183\du}{9.602081\du}}
\pgfpathlineto{\pgfpoint{14.944834\du}{9.609018\du}}
\pgfpathlineto{\pgfpoint{14.949945\du}{9.616685\du}}
\pgfpathlineto{\pgfpoint{14.953961\du}{9.623622\du}}
\pgfpathlineto{\pgfpoint{14.959438\du}{9.630924\du}}
\pgfpathlineto{\pgfpoint{14.964184\du}{9.637861\du}}
\pgfpathlineto{\pgfpoint{14.969660\du}{9.645528\du}}
\pgfpathlineto{\pgfpoint{14.975502\du}{9.652464\du}}
\pgfpathlineto{\pgfpoint{14.981344\du}{9.659401\du}}
\pgfpathlineto{\pgfpoint{14.987185\du}{9.666338\du}}
\pgfpathlineto{\pgfpoint{14.994122\du}{9.672910\du}}
\pgfpathlineto{\pgfpoint{15.000694\du}{9.679847\du}}
\pgfpathlineto{\pgfpoint{15.007631\du}{9.686783\du}}
\pgfpathlineto{\pgfpoint{15.014567\du}{9.693355\du}}
\pgfpathlineto{\pgfpoint{15.021504\du}{9.700292\du}}
\pgfpathlineto{\pgfpoint{15.030267\du}{9.706864\du}}
\pgfpathlineto{\pgfpoint{15.037203\du}{9.713801\du}}
\pgfpathlineto{\pgfpoint{15.044870\du}{9.720372\du}}
\pgfpathlineto{\pgfpoint{15.053633\du}{9.726579\du}}
\pgfpathlineto{\pgfpoint{15.062030\du}{9.733151\du}}
\pgfpathlineto{\pgfpoint{15.070792\du}{9.739723\du}}
\pgfpathlineto{\pgfpoint{15.079189\du}{9.745929\du}}
\pgfpathlineto{\pgfpoint{15.089047\du}{9.752501\du}}
\pgfpathlineto{\pgfpoint{15.098175\du}{9.759073\du}}
\pgfpathlineto{\pgfpoint{15.107667\du}{9.765279\du}}
\pgfpathlineto{\pgfpoint{15.117160\du}{9.771121\du}}
\pgfpathlineto{\pgfpoint{15.127017\du}{9.777693\du}}
\pgfpathlineto{\pgfpoint{15.137240\du}{9.784264\du}}
\pgfpathlineto{\pgfpoint{15.147463\du}{9.790106\du}}
\pgfpathlineto{\pgfpoint{15.158050\du}{9.795947\du}}
\pgfpathlineto{\pgfpoint{15.168273\du}{9.801789\du}}
\pgfpathlineto{\pgfpoint{15.179226\du}{9.807996\du}}
\pgfpathlineto{\pgfpoint{15.190909\du}{9.813837\du}}
\pgfpathlineto{\pgfpoint{15.202227\du}{9.819314\du}}
\pgfpathlineto{\pgfpoint{15.213910\du}{9.825155\du}}
\pgfpathlineto{\pgfpoint{15.225228\du}{9.830997\du}}
\pgfpathlineto{\pgfpoint{15.236911\du}{9.836838\du}}
\pgfpathlineto{\pgfpoint{15.249325\du}{9.841950\du}}
\pgfpathlineto{\pgfpoint{15.261373\du}{9.847791\du}}
\pgfpathlineto{\pgfpoint{15.274516\du}{9.853633\du}}
\pgfpathlineto{\pgfpoint{15.286564\du}{9.859109\du}}
\pgfpathlineto{\pgfpoint{15.299343\du}{9.864221\du}}
\pgfpathlineto{\pgfpoint{15.313217\du}{9.869697\du}}
\pgfpathlineto{\pgfpoint{15.325630\du}{9.874808\du}}
\pgfpathlineto{\pgfpoint{15.338773\du}{9.879920\du}}
\pgfpathlineto{\pgfpoint{15.353012\du}{9.885396\du}}
\pgfpathlineto{\pgfpoint{15.366156\du}{9.890507\du}}
\pgfpathlineto{\pgfpoint{15.380394\du}{9.895254\du}}
\pgfpathlineto{\pgfpoint{15.394268\du}{9.900365\du}}
\pgfpathlineto{\pgfpoint{15.408872\du}{9.905111\du}}
\pgfpathlineto{\pgfpoint{15.422746\du}{9.910588\du}}
\pgfpathlineto{\pgfpoint{15.438445\du}{9.915334\du}}
\pgfpathlineto{\pgfpoint{15.452318\du}{9.919715\du}}
\pgfpathlineto{\pgfpoint{15.466922\du}{9.924461\du}}
\pgfpathlineto{\pgfpoint{15.482256\du}{9.929208\du}}
\pgfpathlineto{\pgfpoint{15.497955\du}{9.933954\du}}
\pgfpathlineto{\pgfpoint{15.512924\du}{9.938700\du}}
\pgfpathlineto{\pgfpoint{15.528624\du}{9.942716\du}}
\pgfpathlineto{\pgfpoint{15.560387\du}{9.951844\du}}
\pgfpathlineto{\pgfpoint{15.592150\du}{9.960241\du}}
\pgfpathlineto{\pgfpoint{15.625374\du}{9.968273\du}}
\pgfpathlineto{\pgfpoint{15.657868\du}{9.976670\du}}
\pgfpathlineto{\pgfpoint{15.692552\du}{9.984337\du}}
\pgfpathlineto{\pgfpoint{15.726871\du}{9.991274\du}}
\pgfpathlineto{\pgfpoint{15.761555\du}{9.998211\du}}
\pgfpathlineto{\pgfpoint{15.797335\du}{10.005148\du}}
\pgfpathlineto{\pgfpoint{15.833479\du}{10.011720\du}}
\pgfpathlineto{\pgfpoint{15.869989\du}{10.018291\du}}
\pgfpathlineto{\pgfpoint{15.907229\du}{10.024133\du}}
\pgfpathlineto{\pgfpoint{15.944834\du}{10.029974\du}}
\pgfpathlineto{\pgfpoint{15.982804\du}{10.035086\du}}
\pgfpathlineto{\pgfpoint{16.021504\du}{10.039832\du}}
\pgfpathlineto{\pgfpoint{16.060204\du}{10.044578\du}}
\pgfpathlineto{\pgfpoint{16.099270\du}{10.049325\du}}
\pgfpathlineto{\pgfpoint{16.139430\du}{10.053341\du}}
\pgfpathlineto{\pgfpoint{16.179226\du}{10.057357\du}}
\pgfpathlineto{\pgfpoint{16.220117\du}{10.060277\du}}
\pgfpathlineto{\pgfpoint{16.261008\du}{10.063198\du}}
\pgfpathlineto{\pgfpoint{16.302264\du}{10.066119\du}}
\pgfpathlineto{\pgfpoint{16.344250\du}{10.068310\du}}
\pgfpathlineto{\pgfpoint{16.385871\du}{10.070135\du}}
\pgfpathlineto{\pgfpoint{16.428222\du}{10.071961\du}}
\pgfpathlineto{\pgfpoint{16.470208\du}{10.073056\du}}
\pgfpathlineto{\pgfpoint{16.513655\du}{10.074151\du}}
\pgfpathlineto{\pgfpoint{16.556371\du}{10.074881\du}}
\pgfpathlineto{\pgfpoint{16.600183\du}{10.074881\du}}
\pgfpathlineto{\pgfpoint{16.600183\du}{10.054436\du}}
\pgfpathlineto{\pgfpoint{16.557466\du}{10.054436\du}}
\pgfpathlineto{\pgfpoint{16.514020\du}{10.053706\du}}
\pgfpathlineto{\pgfpoint{16.471668\du}{10.053341\du}}
\pgfpathlineto{\pgfpoint{16.428952\du}{10.051515\du}}
\pgfpathlineto{\pgfpoint{16.386601\du}{10.050420\du}}
\pgfpathlineto{\pgfpoint{16.344615\du}{10.047864\du}}
\pgfpathlineto{\pgfpoint{16.303724\du}{10.045674\du}}
\pgfpathlineto{\pgfpoint{16.262468\du}{10.042753\du}}
\pgfpathlineto{\pgfpoint{16.221577\du}{10.039832\du}}
\pgfpathlineto{\pgfpoint{16.181417\du}{10.036181\du}}
\pgfpathlineto{\pgfpoint{16.141621\du}{10.032895\du}}
\pgfpathlineto{\pgfpoint{16.102191\du}{10.028514\du}}
\pgfpathlineto{\pgfpoint{16.062760\du}{10.024498\du}}
\pgfpathlineto{\pgfpoint{16.023695\du}{10.019752\du}}
\pgfpathlineto{\pgfpoint{15.985360\du}{10.014640\du}}
\pgfpathlineto{\pgfpoint{15.947755\du}{10.009529\du}}
\pgfpathlineto{\pgfpoint{15.910515\du}{10.003687\du}}
\pgfpathlineto{\pgfpoint{15.874005\du}{9.997481\du}}
\pgfpathlineto{\pgfpoint{15.836765\du}{9.991639\du}}
\pgfpathlineto{\pgfpoint{15.801716\du}{9.985433\du}}
\pgfpathlineto{\pgfpoint{15.765571\du}{9.978496\du}}
\pgfpathlineto{\pgfpoint{15.730887\du}{9.971194\du}}
\pgfpathlineto{\pgfpoint{15.696568\du}{9.963892\du}}
\pgfpathlineto{\pgfpoint{15.662614\du}{9.956225\du}}
\pgfpathlineto{\pgfpoint{15.629390\du}{9.948558\du}}
\pgfpathlineto{\pgfpoint{15.597992\du}{9.940161\du}}
\pgfpathlineto{\pgfpoint{15.565498\du}{9.931398\du}}
\pgfpathlineto{\pgfpoint{15.534465\du}{9.923366\du}}
\pgfpathlineto{\pgfpoint{15.518766\du}{9.918620\du}}
\pgfpathlineto{\pgfpoint{15.503067\du}{9.913874\du}}
\pgfpathlineto{\pgfpoint{15.488463\du}{9.909858\du}}
\pgfpathlineto{\pgfpoint{15.473494\du}{9.905111\du}}
\pgfpathlineto{\pgfpoint{15.458160\du}{9.900365\du}}
\pgfpathlineto{\pgfpoint{15.443921\du}{9.895984\du}}
\pgfpathlineto{\pgfpoint{15.429682\du}{9.890507\du}}
\pgfpathlineto{\pgfpoint{15.415444\du}{9.885761\du}}
\pgfpathlineto{\pgfpoint{15.401205\du}{9.881380\du}}
\pgfpathlineto{\pgfpoint{15.387331\du}{9.875904\du}}
\pgfpathlineto{\pgfpoint{15.373457\du}{9.871157\du}}
\pgfpathlineto{\pgfpoint{15.359949\du}{9.866046\du}}
\pgfpathlineto{\pgfpoint{15.346440\du}{9.860570\du}}
\pgfpathlineto{\pgfpoint{15.333662\du}{9.856188\du}}
\pgfpathlineto{\pgfpoint{15.320153\du}{9.850347\du}}
\pgfpathlineto{\pgfpoint{15.307375\du}{9.844870\du}}
\pgfpathlineto{\pgfpoint{15.294962\du}{9.839759\du}}
\pgfpathlineto{\pgfpoint{15.281818\du}{9.834283\du}}
\pgfpathlineto{\pgfpoint{15.269770\du}{9.829171\du}}
\pgfpathlineto{\pgfpoint{15.258452\du}{9.823330\du}}
\pgfpathlineto{\pgfpoint{15.245674\du}{9.817853\du}}
\pgfpathlineto{\pgfpoint{15.233991\du}{9.812742\du}}
\pgfpathlineto{\pgfpoint{15.222307\du}{9.806900\du}}
\pgfpathlineto{\pgfpoint{15.211355\du}{9.801059\du}}
\pgfpathlineto{\pgfpoint{15.199671\du}{9.795217\du}}
\pgfpathlineto{\pgfpoint{15.189814\du}{9.789376\du}}
\pgfpathlineto{\pgfpoint{15.178496\du}{9.784264\du}}
\pgfpathlineto{\pgfpoint{15.167908\du}{9.777693\du}}
\pgfpathlineto{\pgfpoint{15.158050\du}{9.771851\du}}
\pgfpathlineto{\pgfpoint{15.147463\du}{9.766009\du}}
\pgfpathlineto{\pgfpoint{15.137605\du}{9.760168\du}}
\pgfpathlineto{\pgfpoint{15.128112\du}{9.754326\du}}
\pgfpathlineto{\pgfpoint{15.118255\du}{9.747755\du}}
\pgfpathlineto{\pgfpoint{15.109127\du}{9.741913\du}}
\pgfpathlineto{\pgfpoint{15.100730\du}{9.735341\du}}
\pgfpathlineto{\pgfpoint{15.091968\du}{9.729500\du}}
\pgfpathlineto{\pgfpoint{15.082840\du}{9.723293\du}}
\pgfpathlineto{\pgfpoint{15.074078\du}{9.716721\du}}
\pgfpathlineto{\pgfpoint{15.065681\du}{9.710515\du}}
\pgfpathlineto{\pgfpoint{15.058379\du}{9.703943\du}}
\pgfpathlineto{\pgfpoint{15.050347\du}{9.698101\du}}
\pgfpathlineto{\pgfpoint{15.042680\du}{9.691530\du}}
\pgfpathlineto{\pgfpoint{15.035743\du}{9.685323\du}}
\pgfpathlineto{\pgfpoint{15.028441\du}{9.678751\du}}
\pgfpathlineto{\pgfpoint{15.021504\du}{9.671815\du}}
\pgfpathlineto{\pgfpoint{15.015298\du}{9.665243\du}}
\pgfpathlineto{\pgfpoint{15.008726\du}{9.659036\du}}
\pgfpathlineto{\pgfpoint{15.002884\du}{9.652464\du}}
\pgfpathlineto{\pgfpoint{14.996678\du}{9.645528\du}}
\pgfpathlineto{\pgfpoint{14.991566\du}{9.638956\du}}
\pgfpathlineto{\pgfpoint{14.985360\du}{9.632384\du}}
\pgfpathlineto{\pgfpoint{14.980978\du}{9.625447\du}}
\pgfpathlineto{\pgfpoint{14.975502\du}{9.619241\du}}
\pgfpathlineto{\pgfpoint{14.971121\du}{9.611939\du}}
\pgfpathlineto{\pgfpoint{14.966740\du}{9.605732\du}}
\pgfpathlineto{\pgfpoint{14.962359\du}{9.598430\du}}
\pgfpathlineto{\pgfpoint{14.959073\du}{9.592223\du}}
\pgfpathlineto{\pgfpoint{14.955057\du}{9.585287\du}}
\pgfpathlineto{\pgfpoint{14.951041\du}{9.577985\du}}
\pgfpathlineto{\pgfpoint{14.948120\du}{9.571048\du}}
\pgfpathlineto{\pgfpoint{14.945564\du}{9.564841\du}}
\pgfpathlineto{\pgfpoint{14.942278\du}{9.557539\du}}
\pgfpathlineto{\pgfpoint{14.940088\du}{9.550602\du}}
\pgfpathlineto{\pgfpoint{14.937897\du}{9.543666\du}}
\pgfpathlineto{\pgfpoint{14.936437\du}{9.536729\du}}
\pgfpathlineto{\pgfpoint{14.934246\du}{9.529427\du}}
\pgfpathlineto{\pgfpoint{14.933151\du}{9.523220\du}}
\pgfpathlineto{\pgfpoint{14.932055\du}{9.515918\du}}
\pgfpathlineto{\pgfpoint{14.930595\du}{9.508981\du}}
\pgfpathlineto{\pgfpoint{14.930230\du}{9.502045\du}}
\pgfpathlineto{\pgfpoint{14.930230\du}{9.494378\du}}
\pgfpathlineto{\pgfpoint{14.929865\du}{9.487441\du}}
\pgfpathlineto{\pgfpoint{14.929865\du}{9.487441\du}}
\pgfpathlineto{\pgfpoint{14.929865\du}{9.487441\du}}
\pgfpathlineto{\pgfpoint{14.929865\du}{9.486345\du}}
\pgfpathlineto{\pgfpoint{14.929865\du}{9.485250\du}}
\pgfpathlineto{\pgfpoint{14.929500\du}{9.483790\du}}
\pgfpathlineto{\pgfpoint{14.929500\du}{9.483425\du}}
\pgfpathlineto{\pgfpoint{14.928405\du}{9.482329\du}}
\pgfpathlineto{\pgfpoint{14.928039\du}{9.480869\du}}
\pgfpathlineto{\pgfpoint{14.927674\du}{9.480504\du}}
\pgfpathlineto{\pgfpoint{14.926579\du}{9.479774\du}}
\pgfpathlineto{\pgfpoint{14.925119\du}{9.478678\du}}
\pgfpathlineto{\pgfpoint{14.923293\du}{9.477948\du}}
\pgfpathlineto{\pgfpoint{14.921468\du}{9.477583\du}}
\pgfpathlineto{\pgfpoint{14.919642\du}{9.477583\du}}
\pgfpathlineto{\pgfpoint{14.917452\du}{9.477583\du}}
\pgfpathlineto{\pgfpoint{14.915991\du}{9.477948\du}}
\pgfpathlineto{\pgfpoint{14.913801\du}{9.478678\du}}
\pgfpathlineto{\pgfpoint{14.911975\du}{9.479774\du}}
\pgfpathlineto{\pgfpoint{14.911610\du}{9.480504\du}}
\pgfpathlineto{\pgfpoint{14.911245\du}{9.480869\du}}
\pgfpathlineto{\pgfpoint{14.910515\du}{9.482329\du}}
\pgfpathlineto{\pgfpoint{14.909785\du}{9.483425\du}}
\pgfpathlineto{\pgfpoint{14.909785\du}{9.483790\du}}
\pgfpathlineto{\pgfpoint{14.909419\du}{9.485250\du}}
\pgfpathlineto{\pgfpoint{14.909419\du}{9.486345\du}}
\pgfpathlineto{\pgfpoint{14.909419\du}{9.487441\du}}
\pgfusepath{fill}
\pgfsetbuttcap
\pgfsetmiterjoin
\pgfsetdash{}{0pt}
\definecolor{dialinecolor}{rgb}{0.678431, 0.839216, 0.905882}
\pgfsetfillcolor{dialinecolor}
\pgfpathmoveto{\pgfpoint{16.600183\du}{8.900000\du}}
\pgfpathlineto{\pgfpoint{16.600183\du}{8.900000\du}}
\pgfpathlineto{\pgfpoint{16.556371\du}{8.900000\du}}
\pgfpathlineto{\pgfpoint{16.513655\du}{8.900730\du}}
\pgfpathlineto{\pgfpoint{16.470208\du}{8.901825\du}}
\pgfpathlineto{\pgfpoint{16.428222\du}{8.902921\du}}
\pgfpathlineto{\pgfpoint{16.385871\du}{8.904746\du}}
\pgfpathlineto{\pgfpoint{16.344250\du}{8.906937\du}}
\pgfpathlineto{\pgfpoint{16.302264\du}{8.909493\du}}
\pgfpathlineto{\pgfpoint{16.261008\du}{8.911683\du}}
\pgfpathlineto{\pgfpoint{16.220117\du}{8.914604\du}}
\pgfpathlineto{\pgfpoint{16.179226\du}{8.918255\du}}
\pgfpathlineto{\pgfpoint{16.139430\du}{8.921906\du}}
\pgfpathlineto{\pgfpoint{16.099270\du}{8.925922\du}}
\pgfpathlineto{\pgfpoint{16.060204\du}{8.930303\du}}
\pgfpathlineto{\pgfpoint{16.021504\du}{8.935049\du}}
\pgfpathlineto{\pgfpoint{15.982804\du}{8.940161\du}}
\pgfpathlineto{\pgfpoint{15.944834\du}{8.945637\du}}
\pgfpathlineto{\pgfpoint{15.907229\du}{8.950748\du}}
\pgfpathlineto{\pgfpoint{15.869989\du}{8.957320\du}}
\pgfpathlineto{\pgfpoint{15.833479\du}{8.963162\du}}
\pgfpathlineto{\pgfpoint{15.797335\du}{8.969733\du}}
\pgfpathlineto{\pgfpoint{15.761555\du}{8.976670\du}}
\pgfpathlineto{\pgfpoint{15.726871\du}{8.983607\du}}
\pgfpathlineto{\pgfpoint{15.692552\du}{8.991274\du}}
\pgfpathlineto{\pgfpoint{15.657868\du}{8.998941\du}}
\pgfpathlineto{\pgfpoint{15.625374\du}{9.006973\du}}
\pgfpathlineto{\pgfpoint{15.592150\du}{9.015371\du}}
\pgfpathlineto{\pgfpoint{15.560387\du}{9.023403\du}}
\pgfpathlineto{\pgfpoint{15.528624\du}{9.032165\du}}
\pgfpathlineto{\pgfpoint{15.497955\du}{9.041658\du}}
\pgfpathlineto{\pgfpoint{15.466922\du}{9.050420\du}}
\pgfpathlineto{\pgfpoint{15.452318\du}{9.055166\du}}
\pgfpathlineto{\pgfpoint{15.438445\du}{9.060277\du}}
\pgfpathlineto{\pgfpoint{15.422746\du}{9.065024\du}}
\pgfpathlineto{\pgfpoint{15.408872\du}{9.069770\du}}
\pgfpathlineto{\pgfpoint{15.394268\du}{9.074881\du}}
\pgfpathlineto{\pgfpoint{15.380394\du}{9.079628\du}}
\pgfpathlineto{\pgfpoint{15.366156\du}{9.084739\du}}
\pgfpathlineto{\pgfpoint{15.353012\du}{9.089485\du}}
\pgfpathlineto{\pgfpoint{15.338773\du}{9.094962\du}}
\pgfpathlineto{\pgfpoint{15.325630\du}{9.100073\du}}
\pgfpathlineto{\pgfpoint{15.313217\du}{9.105184\du}}
\pgfpathlineto{\pgfpoint{15.299343\du}{9.111026\du}}
\pgfpathlineto{\pgfpoint{15.286564\du}{9.116502\du}}
\pgfpathlineto{\pgfpoint{15.274516\du}{9.121614\du}}
\pgfpathlineto{\pgfpoint{15.261373\du}{9.127455\du}}
\pgfpathlineto{\pgfpoint{15.249325\du}{9.132932\du}}
\pgfpathlineto{\pgfpoint{15.236911\du}{9.138773\du}}
\pgfpathlineto{\pgfpoint{15.225228\du}{9.143885\du}}
\pgfpathlineto{\pgfpoint{15.213910\du}{9.149726\du}}
\pgfpathlineto{\pgfpoint{15.202227\du}{9.155568\du}}
\pgfpathlineto{\pgfpoint{15.190909\du}{9.161409\du}}
\pgfpathlineto{\pgfpoint{15.179226\du}{9.167251\du}}
\pgfpathlineto{\pgfpoint{15.168273\du}{9.173092\du}}
\pgfpathlineto{\pgfpoint{15.158050\du}{9.179664\du}}
\pgfpathlineto{\pgfpoint{15.147463\du}{9.185506\du}}
\pgfpathlineto{\pgfpoint{15.137240\du}{9.191347\du}}
\pgfpathlineto{\pgfpoint{15.127017\du}{9.197919\du}}
\pgfpathlineto{\pgfpoint{15.117160\du}{9.203760\du}}
\pgfpathlineto{\pgfpoint{15.107667\du}{9.209967\du}}
\pgfpathlineto{\pgfpoint{15.098175\du}{9.216539\du}}
\pgfpathlineto{\pgfpoint{15.089047\du}{9.223111\du}}
\pgfpathlineto{\pgfpoint{15.079189\du}{9.228952\du}}
\pgfpathlineto{\pgfpoint{15.070792\du}{9.235159\du}}
\pgfpathlineto{\pgfpoint{15.062030\du}{9.241731\du}}
\pgfpathlineto{\pgfpoint{15.053633\du}{9.247937\du}}
\pgfpathlineto{\pgfpoint{15.044870\du}{9.255239\du}}
\pgfpathlineto{\pgfpoint{15.037203\du}{9.261446\du}}
\pgfpathlineto{\pgfpoint{15.030267\du}{9.268018\du}}
\pgfpathlineto{\pgfpoint{15.021504\du}{9.274954\du}}
\pgfpathlineto{\pgfpoint{15.014567\du}{9.281526\du}}
\pgfpathlineto{\pgfpoint{15.007631\du}{9.288463\du}}
\pgfpathlineto{\pgfpoint{15.000694\du}{9.295400\du}}
\pgfpathlineto{\pgfpoint{14.994122\du}{9.301972\du}}
\pgfpathlineto{\pgfpoint{14.987185\du}{9.308908\du}}
\pgfpathlineto{\pgfpoint{14.981344\du}{9.315845\du}}
\pgfpathlineto{\pgfpoint{14.975502\du}{9.323147\du}}
\pgfpathlineto{\pgfpoint{14.969660\du}{9.330084\du}}
\pgfpathlineto{\pgfpoint{14.964184\du}{9.337021\du}}
\pgfpathlineto{\pgfpoint{14.959438\du}{9.343958\du}}
\pgfpathlineto{\pgfpoint{14.953961\du}{9.351625\du}}
\pgfpathlineto{\pgfpoint{14.949945\du}{9.358562\du}}
\pgfpathlineto{\pgfpoint{14.944834\du}{9.365863\du}}
\pgfpathlineto{\pgfpoint{14.941183\du}{9.373165\du}}
\pgfpathlineto{\pgfpoint{14.936802\du}{9.380467\du}}
\pgfpathlineto{\pgfpoint{14.933151\du}{9.388134\du}}
\pgfpathlineto{\pgfpoint{14.929865\du}{9.395436\du}}
\pgfpathlineto{\pgfpoint{14.926214\du}{9.403103\du}}
\pgfpathlineto{\pgfpoint{14.923293\du}{9.410770\du}}
\pgfpathlineto{\pgfpoint{14.920737\du}{9.417707\du}}
\pgfpathlineto{\pgfpoint{14.918547\du}{9.425374\du}}
\pgfpathlineto{\pgfpoint{14.916356\du}{9.433041\du}}
\pgfpathlineto{\pgfpoint{14.914531\du}{9.441073\du}}
\pgfpathlineto{\pgfpoint{14.912705\du}{9.448740\du}}
\pgfpathlineto{\pgfpoint{14.911610\du}{9.456407\du}}
\pgfpathlineto{\pgfpoint{14.910515\du}{9.464074\du}}
\pgfpathlineto{\pgfpoint{14.909785\du}{9.472107\du}}
\pgfpathlineto{\pgfpoint{14.909419\du}{9.479774\du}}
\pgfpathlineto{\pgfpoint{14.909419\du}{9.487441\du}}
\pgfpathlineto{\pgfpoint{14.929865\du}{9.487441\du}}
\pgfpathlineto{\pgfpoint{14.930230\du}{9.480504\du}}
\pgfpathlineto{\pgfpoint{14.930230\du}{9.473567\du}}
\pgfpathlineto{\pgfpoint{14.930595\du}{9.466265\du}}
\pgfpathlineto{\pgfpoint{14.932055\du}{9.459328\du}}
\pgfpathlineto{\pgfpoint{14.933151\du}{9.452391\du}}
\pgfpathlineto{\pgfpoint{14.934246\du}{9.445455\du}}
\pgfpathlineto{\pgfpoint{14.936437\du}{9.438153\du}}
\pgfpathlineto{\pgfpoint{14.937897\du}{9.431946\du}}
\pgfpathlineto{\pgfpoint{14.940088\du}{9.424644\du}}
\pgfpathlineto{\pgfpoint{14.942278\du}{9.417707\du}}
\pgfpathlineto{\pgfpoint{14.945564\du}{9.410770\du}}
\pgfpathlineto{\pgfpoint{14.948120\du}{9.403834\du}}
\pgfpathlineto{\pgfpoint{14.951041\du}{9.396897\du}}
\pgfpathlineto{\pgfpoint{14.955057\du}{9.390325\du}}
\pgfpathlineto{\pgfpoint{14.959073\du}{9.383388\du}}
\pgfpathlineto{\pgfpoint{14.962359\du}{9.376816\du}}
\pgfpathlineto{\pgfpoint{14.966375\du}{9.369880\du}}
\pgfpathlineto{\pgfpoint{14.971121\du}{9.362943\du}}
\pgfpathlineto{\pgfpoint{14.975502\du}{9.356371\du}}
\pgfpathlineto{\pgfpoint{14.980978\du}{9.349799\du}}
\pgfpathlineto{\pgfpoint{14.985360\du}{9.342862\du}}
\pgfpathlineto{\pgfpoint{14.991566\du}{9.336291\du}}
\pgfpathlineto{\pgfpoint{14.996678\du}{9.329354\du}}
\pgfpathlineto{\pgfpoint{15.002884\du}{9.323147\du}}
\pgfpathlineto{\pgfpoint{15.008726\du}{9.316575\du}}
\pgfpathlineto{\pgfpoint{15.015298\du}{9.309639\du}}
\pgfpathlineto{\pgfpoint{15.021504\du}{9.303067\du}}
\pgfpathlineto{\pgfpoint{15.028441\du}{9.296860\du}}
\pgfpathlineto{\pgfpoint{15.035743\du}{9.290288\du}}
\pgfpathlineto{\pgfpoint{15.042680\du}{9.283717\du}}
\pgfpathlineto{\pgfpoint{15.050347\du}{9.277510\du}}
\pgfpathlineto{\pgfpoint{15.058379\du}{9.270938\du}}
\pgfpathlineto{\pgfpoint{15.065681\du}{9.264367\du}}
\pgfpathlineto{\pgfpoint{15.074078\du}{9.258160\du}}
\pgfpathlineto{\pgfpoint{15.082840\du}{9.252318\du}}
\pgfpathlineto{\pgfpoint{15.091968\du}{9.245382\du}}
\pgfpathlineto{\pgfpoint{15.100730\du}{9.239905\du}}
\pgfpathlineto{\pgfpoint{15.109127\du}{9.233333\du}}
\pgfpathlineto{\pgfpoint{15.118255\du}{9.227127\du}}
\pgfpathlineto{\pgfpoint{15.128112\du}{9.221285\du}}
\pgfpathlineto{\pgfpoint{15.137605\du}{9.215444\du}}
\pgfpathlineto{\pgfpoint{15.147463\du}{9.208872\du}}
\pgfpathlineto{\pgfpoint{15.158050\du}{9.203030\du}}
\pgfpathlineto{\pgfpoint{15.167908\du}{9.197189\du}}
\pgfpathlineto{\pgfpoint{15.178496\du}{9.191347\du}}
\pgfpathlineto{\pgfpoint{15.189814\du}{9.185506\du}}
\pgfpathlineto{\pgfpoint{15.199671\du}{9.179664\du}}
\pgfpathlineto{\pgfpoint{15.211355\du}{9.173823\du}}
\pgfpathlineto{\pgfpoint{15.222307\du}{9.168711\du}}
\pgfpathlineto{\pgfpoint{15.233991\du}{9.162505\du}}
\pgfpathlineto{\pgfpoint{15.245674\du}{9.156663\du}}
\pgfpathlineto{\pgfpoint{15.258452\du}{9.151552\du}}
\pgfpathlineto{\pgfpoint{15.269770\du}{9.146440\du}}
\pgfpathlineto{\pgfpoint{15.281818\du}{9.140599\du}}
\pgfpathlineto{\pgfpoint{15.294962\du}{9.135122\du}}
\pgfpathlineto{\pgfpoint{15.307375\du}{9.130011\du}}
\pgfpathlineto{\pgfpoint{15.320153\du}{9.124535\du}}
\pgfpathlineto{\pgfpoint{15.333662\du}{9.119423\du}}
\pgfpathlineto{\pgfpoint{15.346440\du}{9.113947\du}}
\pgfpathlineto{\pgfpoint{15.359949\du}{9.108470\du}}
\pgfpathlineto{\pgfpoint{15.373457\du}{9.104089\du}}
\pgfpathlineto{\pgfpoint{15.387331\du}{9.098978\du}}
\pgfpathlineto{\pgfpoint{15.401205\du}{9.094231\du}}
\pgfpathlineto{\pgfpoint{15.415444\du}{9.089120\du}}
\pgfpathlineto{\pgfpoint{15.429682\du}{9.084374\du}}
\pgfpathlineto{\pgfpoint{15.443921\du}{9.079628\du}}
\pgfpathlineto{\pgfpoint{15.458160\du}{9.074881\du}}
\pgfpathlineto{\pgfpoint{15.473494\du}{9.070135\du}}
\pgfpathlineto{\pgfpoint{15.503067\du}{9.061008\du}}
\pgfpathlineto{\pgfpoint{15.534465\du}{9.052245\du}}
\pgfpathlineto{\pgfpoint{15.565498\du}{9.043483\du}}
\pgfpathlineto{\pgfpoint{15.597992\du}{9.035086\du}}
\pgfpathlineto{\pgfpoint{15.629390\du}{9.027054\du}}
\pgfpathlineto{\pgfpoint{15.662614\du}{9.018656\du}}
\pgfpathlineto{\pgfpoint{15.696568\du}{9.010989\du}}
\pgfpathlineto{\pgfpoint{15.730887\du}{9.004053\du}}
\pgfpathlineto{\pgfpoint{15.765571\du}{8.997116\du}}
\pgfpathlineto{\pgfpoint{15.801716\du}{8.989814\du}}
\pgfpathlineto{\pgfpoint{15.836765\du}{8.983607\du}}
\pgfpathlineto{\pgfpoint{15.874005\du}{8.976670\du}}
\pgfpathlineto{\pgfpoint{15.910515\du}{8.971194\du}}
\pgfpathlineto{\pgfpoint{15.947755\du}{8.965352\du}}
\pgfpathlineto{\pgfpoint{15.985360\du}{8.960241\du}}
\pgfpathlineto{\pgfpoint{16.023695\du}{8.955495\du}}
\pgfpathlineto{\pgfpoint{16.062760\du}{8.950748\du}}
\pgfpathlineto{\pgfpoint{16.102191\du}{8.946367\du}}
\pgfpathlineto{\pgfpoint{16.141621\du}{8.942716\du}}
\pgfpathlineto{\pgfpoint{16.181417\du}{8.938700\du}}
\pgfpathlineto{\pgfpoint{16.221577\du}{8.935779\du}}
\pgfpathlineto{\pgfpoint{16.262468\du}{8.932129\du}}
\pgfpathlineto{\pgfpoint{16.303724\du}{8.929208\du}}
\pgfpathlineto{\pgfpoint{16.344615\du}{8.927017\du}}
\pgfpathlineto{\pgfpoint{16.386601\du}{8.925192\du}}
\pgfpathlineto{\pgfpoint{16.428952\du}{8.923366\du}}
\pgfpathlineto{\pgfpoint{16.471668\du}{8.921906\du}}
\pgfpathlineto{\pgfpoint{16.514020\du}{8.921541\du}}
\pgfpathlineto{\pgfpoint{16.557466\du}{8.921176\du}}
\pgfpathlineto{\pgfpoint{16.600183\du}{8.920445\du}}
\pgfpathlineto{\pgfpoint{16.600183\du}{8.920445\du}}
\pgfpathlineto{\pgfpoint{16.600183\du}{8.920445\du}}
\pgfpathlineto{\pgfpoint{16.601278\du}{8.920445\du}}
\pgfpathlineto{\pgfpoint{16.602373\du}{8.920445\du}}
\pgfpathlineto{\pgfpoint{16.603834\du}{8.919715\du}}
\pgfpathlineto{\pgfpoint{16.604929\du}{8.919715\du}}
\pgfpathlineto{\pgfpoint{16.605294\du}{8.919350\du}}
\pgfpathlineto{\pgfpoint{16.606389\du}{8.918620\du}}
\pgfpathlineto{\pgfpoint{16.607119\du}{8.918255\du}}
\pgfpathlineto{\pgfpoint{16.608215\du}{8.917525\du}}
\pgfpathlineto{\pgfpoint{16.609310\du}{8.915699\du}}
\pgfpathlineto{\pgfpoint{16.610040\du}{8.913874\du}}
\pgfpathlineto{\pgfpoint{16.610040\du}{8.912413\du}}
\pgfpathlineto{\pgfpoint{16.610770\du}{8.910588\du}}
\pgfpathlineto{\pgfpoint{16.610040\du}{8.908762\du}}
\pgfpathlineto{\pgfpoint{16.610040\du}{8.906572\du}}
\pgfpathlineto{\pgfpoint{16.609310\du}{8.904746\du}}
\pgfpathlineto{\pgfpoint{16.608215\du}{8.902921\du}}
\pgfpathlineto{\pgfpoint{16.607119\du}{8.902191\du}}
\pgfpathlineto{\pgfpoint{16.606389\du}{8.901825\du}}
\pgfpathlineto{\pgfpoint{16.605294\du}{8.901095\du}}
\pgfpathlineto{\pgfpoint{16.604929\du}{8.900730\du}}
\pgfpathlineto{\pgfpoint{16.603834\du}{8.900730\du}}
\pgfpathlineto{\pgfpoint{16.602373\du}{8.900000\du}}
\pgfpathlineto{\pgfpoint{16.601278\du}{8.900000\du}}
\pgfpathlineto{\pgfpoint{16.600183\du}{8.900000\du}}
\pgfusepath{fill}
\pgfsetbuttcap
\pgfsetmiterjoin
\pgfsetdash{}{0pt}
\definecolor{dialinecolor}{rgb}{0.678431, 0.839216, 0.905882}
\pgfsetfillcolor{dialinecolor}
\pgfpathmoveto{\pgfpoint{18.290581\du}{9.487441\du}}
\pgfpathlineto{\pgfpoint{18.290581\du}{9.479774\du}}
\pgfpathlineto{\pgfpoint{18.289850\du}{9.472107\du}}
\pgfpathlineto{\pgfpoint{18.289120\du}{9.464074\du}}
\pgfpathlineto{\pgfpoint{18.288390\du}{9.456407\du}}
\pgfpathlineto{\pgfpoint{18.287295\du}{9.448740\du}}
\pgfpathlineto{\pgfpoint{18.285104\du}{9.441073\du}}
\pgfpathlineto{\pgfpoint{18.284009\du}{9.433041\du}}
\pgfpathlineto{\pgfpoint{18.281088\du}{9.425374\du}}
\pgfpathlineto{\pgfpoint{18.279263\du}{9.417707\du}}
\pgfpathlineto{\pgfpoint{18.276342\du}{9.410770\du}}
\pgfpathlineto{\pgfpoint{18.273786\du}{9.403103\du}}
\pgfpathlineto{\pgfpoint{18.270500\du}{9.395436\du}}
\pgfpathlineto{\pgfpoint{18.266484\du}{9.388134\du}}
\pgfpathlineto{\pgfpoint{18.263198\du}{9.380467\du}}
\pgfpathlineto{\pgfpoint{18.259182\du}{9.373165\du}}
\pgfpathlineto{\pgfpoint{18.255166\du}{9.365863\du}}
\pgfpathlineto{\pgfpoint{18.250785\du}{9.358562\du}}
\pgfpathlineto{\pgfpoint{18.246039\du}{9.351625\du}}
\pgfpathlineto{\pgfpoint{18.240562\du}{9.343958\du}}
\pgfpathlineto{\pgfpoint{18.235816\du}{9.337021\du}}
\pgfpathlineto{\pgfpoint{18.229974\du}{9.330084\du}}
\pgfpathlineto{\pgfpoint{18.224498\du}{9.323147\du}}
\pgfpathlineto{\pgfpoint{18.218656\du}{9.315845\du}}
\pgfpathlineto{\pgfpoint{18.212450\du}{9.308908\du}}
\pgfpathlineto{\pgfpoint{18.205878\du}{9.301972\du}}
\pgfpathlineto{\pgfpoint{18.198941\du}{9.295400\du}}
\pgfpathlineto{\pgfpoint{18.192369\du}{9.288463\du}}
\pgfpathlineto{\pgfpoint{18.185068\du}{9.281526\du}}
\pgfpathlineto{\pgfpoint{18.178496\du}{9.274954\du}}
\pgfpathlineto{\pgfpoint{18.170464\du}{9.268018\du}}
\pgfpathlineto{\pgfpoint{18.162432\du}{9.261446\du}}
\pgfpathlineto{\pgfpoint{18.155130\du}{9.255239\du}}
\pgfpathlineto{\pgfpoint{18.146367\du}{9.247937\du}}
\pgfpathlineto{\pgfpoint{18.138335\du}{9.241731\du}}
\pgfpathlineto{\pgfpoint{18.129208\du}{9.235159\du}}
\pgfpathlineto{\pgfpoint{18.120445\du}{9.228952\du}}
\pgfpathlineto{\pgfpoint{18.110953\du}{9.223111\du}}
\pgfpathlineto{\pgfpoint{18.101825\du}{9.216539\du}}
\pgfpathlineto{\pgfpoint{18.092698\du}{9.209967\du}}
\pgfpathlineto{\pgfpoint{18.082840\du}{9.203760\du}}
\pgfpathlineto{\pgfpoint{18.072983\du}{9.197919\du}}
\pgfpathlineto{\pgfpoint{18.062760\du}{9.191347\du}}
\pgfpathlineto{\pgfpoint{18.052172\du}{9.185506\du}}
\pgfpathlineto{\pgfpoint{18.042315\du}{9.179664\du}}
\pgfpathlineto{\pgfpoint{18.031727\du}{9.173092\du}}
\pgfpathlineto{\pgfpoint{18.020409\du}{9.167251\du}}
\pgfpathlineto{\pgfpoint{18.009091\du}{9.161409\du}}
\pgfpathlineto{\pgfpoint{17.997408\du}{9.155568\du}}
\pgfpathlineto{\pgfpoint{17.986455\du}{9.149726\du}}
\pgfpathlineto{\pgfpoint{17.974407\du}{9.143885\du}}
\pgfpathlineto{\pgfpoint{17.963089\du}{9.138773\du}}
\pgfpathlineto{\pgfpoint{17.950675\du}{9.132932\du}}
\pgfpathlineto{\pgfpoint{17.938262\du}{9.127455\du}}
\pgfpathlineto{\pgfpoint{17.925484\du}{9.121614\du}}
\pgfpathlineto{\pgfpoint{17.913436\du}{9.116502\du}}
\pgfpathlineto{\pgfpoint{17.900657\du}{9.111026\du}}
\pgfpathlineto{\pgfpoint{17.887149\du}{9.105184\du}}
\pgfpathlineto{\pgfpoint{17.874005\du}{9.100073\du}}
\pgfpathlineto{\pgfpoint{17.860862\du}{9.094962\du}}
\pgfpathlineto{\pgfpoint{17.846623\du}{9.089485\du}}
\pgfpathlineto{\pgfpoint{17.833479\du}{9.084739\du}}
\pgfpathlineto{\pgfpoint{17.819241\du}{9.079628\du}}
\pgfpathlineto{\pgfpoint{17.805367\du}{9.074881\du}}
\pgfpathlineto{\pgfpoint{17.791128\du}{9.069770\du}}
\pgfpathlineto{\pgfpoint{17.777254\du}{9.065024\du}}
\pgfpathlineto{\pgfpoint{17.762286\du}{9.060277\du}}
\pgfpathlineto{\pgfpoint{17.747317\du}{9.055166\du}}
\pgfpathlineto{\pgfpoint{17.733078\du}{9.050420\du}}
\pgfpathlineto{\pgfpoint{17.702410\du}{9.041658\du}}
\pgfpathlineto{\pgfpoint{17.671742\du}{9.032165\du}}
\pgfpathlineto{\pgfpoint{17.640343\du}{9.023403\du}}
\pgfpathlineto{\pgfpoint{17.608580\du}{9.015371\du}}
\pgfpathlineto{\pgfpoint{17.575721\du}{9.006973\du}}
\pgfpathlineto{\pgfpoint{17.542132\du}{8.998941\du}}
\pgfpathlineto{\pgfpoint{17.508178\du}{8.991274\du}}
\pgfpathlineto{\pgfpoint{17.473494\du}{8.983607\du}}
\pgfpathlineto{\pgfpoint{17.438810\du}{8.976670\du}}
\pgfpathlineto{\pgfpoint{17.403395\du}{8.969733\du}}
\pgfpathlineto{\pgfpoint{17.366886\du}{8.963162\du}}
\pgfpathlineto{\pgfpoint{17.330376\du}{8.957320\du}}
\pgfpathlineto{\pgfpoint{17.293501\du}{8.950748\du}}
\pgfpathlineto{\pgfpoint{17.256261\du}{8.945637\du}}
\pgfpathlineto{\pgfpoint{17.217196\du}{8.940161\du}}
\pgfpathlineto{\pgfpoint{17.179226\du}{8.935049\du}}
\pgfpathlineto{\pgfpoint{17.139796\du}{8.930303\du}}
\pgfpathlineto{\pgfpoint{17.101095\du}{8.925922\du}}
\pgfpathlineto{\pgfpoint{17.060935\du}{8.921906\du}}
\pgfpathlineto{\pgfpoint{17.021139\du}{8.918255\du}}
\pgfpathlineto{\pgfpoint{16.980248\du}{8.914604\du}}
\pgfpathlineto{\pgfpoint{16.939723\du}{8.911683\du}}
\pgfpathlineto{\pgfpoint{16.898101\du}{8.909493\du}}
\pgfpathlineto{\pgfpoint{16.856480\du}{8.906937\du}}
\pgfpathlineto{\pgfpoint{16.814859\du}{8.904746\du}}
\pgfpathlineto{\pgfpoint{16.772143\du}{8.902921\du}}
\pgfpathlineto{\pgfpoint{16.730157\du}{8.901825\du}}
\pgfpathlineto{\pgfpoint{16.687076\du}{8.900730\du}}
\pgfpathlineto{\pgfpoint{16.643629\du}{8.900000\du}}
\pgfpathlineto{\pgfpoint{16.600183\du}{8.900000\du}}
\pgfpathlineto{\pgfpoint{16.600183\du}{8.920445\du}}
\pgfpathlineto{\pgfpoint{16.643264\du}{8.921176\du}}
\pgfpathlineto{\pgfpoint{16.686710\du}{8.921541\du}}
\pgfpathlineto{\pgfpoint{16.728697\du}{8.921906\du}}
\pgfpathlineto{\pgfpoint{16.771413\du}{8.923366\du}}
\pgfpathlineto{\pgfpoint{16.814129\du}{8.925192\du}}
\pgfpathlineto{\pgfpoint{16.855750\du}{8.927017\du}}
\pgfpathlineto{\pgfpoint{16.897006\du}{8.929208\du}}
\pgfpathlineto{\pgfpoint{16.937897\du}{8.932129\du}}
\pgfpathlineto{\pgfpoint{16.979153\du}{8.935779\du}}
\pgfpathlineto{\pgfpoint{17.019314\du}{8.938700\du}}
\pgfpathlineto{\pgfpoint{17.059474\du}{8.942716\du}}
\pgfpathlineto{\pgfpoint{17.098540\du}{8.946367\du}}
\pgfpathlineto{\pgfpoint{17.137970\du}{8.950748\du}}
\pgfpathlineto{\pgfpoint{17.176670\du}{8.955495\du}}
\pgfpathlineto{\pgfpoint{17.215371\du}{8.960241\du}}
\pgfpathlineto{\pgfpoint{17.252976\du}{8.965352\du}}
\pgfpathlineto{\pgfpoint{17.290215\du}{8.971194\du}}
\pgfpathlineto{\pgfpoint{17.326725\du}{8.976670\du}}
\pgfpathlineto{\pgfpoint{17.363600\du}{8.983607\du}}
\pgfpathlineto{\pgfpoint{17.399014\du}{8.989814\du}}
\pgfpathlineto{\pgfpoint{17.434794\du}{8.997116\du}}
\pgfpathlineto{\pgfpoint{17.469478\du}{9.004053\du}}
\pgfpathlineto{\pgfpoint{17.503432\du}{9.010989\du}}
\pgfpathlineto{\pgfpoint{17.537751\du}{9.018656\du}}
\pgfpathlineto{\pgfpoint{17.570975\du}{9.027054\du}}
\pgfpathlineto{\pgfpoint{17.603103\du}{9.035086\du}}
\pgfpathlineto{\pgfpoint{17.635232\du}{9.043483\du}}
\pgfpathlineto{\pgfpoint{17.666630\du}{9.052245\du}}
\pgfpathlineto{\pgfpoint{17.697298\du}{9.061008\du}}
\pgfpathlineto{\pgfpoint{17.727236\du}{9.070135\du}}
\pgfpathlineto{\pgfpoint{17.741475\du}{9.074881\du}}
\pgfpathlineto{\pgfpoint{17.756079\du}{9.079628\du}}
\pgfpathlineto{\pgfpoint{17.769953\du}{9.084374\du}}
\pgfpathlineto{\pgfpoint{17.784556\du}{9.089120\du}}
\pgfpathlineto{\pgfpoint{17.798795\du}{9.094231\du}}
\pgfpathlineto{\pgfpoint{17.812669\du}{9.098978\du}}
\pgfpathlineto{\pgfpoint{17.826908\du}{9.104089\du}}
\pgfpathlineto{\pgfpoint{17.840051\du}{9.108470\du}}
\pgfpathlineto{\pgfpoint{17.853560\du}{9.113947\du}}
\pgfpathlineto{\pgfpoint{17.866703\du}{9.119423\du}}
\pgfpathlineto{\pgfpoint{17.879482\du}{9.124535\du}}
\pgfpathlineto{\pgfpoint{17.892260\du}{9.130011\du}}
\pgfpathlineto{\pgfpoint{17.905403\du}{9.135122\du}}
\pgfpathlineto{\pgfpoint{17.917452\du}{9.140599\du}}
\pgfpathlineto{\pgfpoint{17.929865\du}{9.146440\du}}
\pgfpathlineto{\pgfpoint{17.941913\du}{9.151552\du}}
\pgfpathlineto{\pgfpoint{17.954326\du}{9.156663\du}}
\pgfpathlineto{\pgfpoint{17.965644\du}{9.162505\du}}
\pgfpathlineto{\pgfpoint{17.977693\du}{9.168711\du}}
\pgfpathlineto{\pgfpoint{17.988280\du}{9.173823\du}}
\pgfpathlineto{\pgfpoint{18.000329\du}{9.179664\du}}
\pgfpathlineto{\pgfpoint{18.010551\du}{9.185506\du}}
\pgfpathlineto{\pgfpoint{18.021504\du}{9.191347\du}}
\pgfpathlineto{\pgfpoint{18.032457\du}{9.197189\du}}
\pgfpathlineto{\pgfpoint{18.042315\du}{9.203030\du}}
\pgfpathlineto{\pgfpoint{18.052172\du}{9.208872\du}}
\pgfpathlineto{\pgfpoint{18.062395\du}{9.215444\du}}
\pgfpathlineto{\pgfpoint{18.071522\du}{9.221285\du}}
\pgfpathlineto{\pgfpoint{18.081745\du}{9.227127\du}}
\pgfpathlineto{\pgfpoint{18.091238\du}{9.233333\du}}
\pgfpathlineto{\pgfpoint{18.100000\du}{9.239905\du}}
\pgfpathlineto{\pgfpoint{18.108032\du}{9.245382\du}}
\pgfpathlineto{\pgfpoint{18.116794\du}{9.252318\du}}
\pgfpathlineto{\pgfpoint{18.125192\du}{9.258160\du}}
\pgfpathlineto{\pgfpoint{18.133954\du}{9.264367\du}}
\pgfpathlineto{\pgfpoint{18.141621\du}{9.270938\du}}
\pgfpathlineto{\pgfpoint{18.149653\du}{9.277510\du}}
\pgfpathlineto{\pgfpoint{18.156955\du}{9.283717\du}}
\pgfpathlineto{\pgfpoint{18.164622\du}{9.290288\du}}
\pgfpathlineto{\pgfpoint{18.171194\du}{9.296860\du}}
\pgfpathlineto{\pgfpoint{18.178496\du}{9.303067\du}}
\pgfpathlineto{\pgfpoint{18.184337\du}{9.309639\du}}
\pgfpathlineto{\pgfpoint{18.191274\du}{9.316575\du}}
\pgfpathlineto{\pgfpoint{18.197846\du}{9.323147\du}}
\pgfpathlineto{\pgfpoint{18.202957\du}{9.329354\du}}
\pgfpathlineto{\pgfpoint{18.208434\du}{9.336291\du}}
\pgfpathlineto{\pgfpoint{18.214640\du}{9.342862\du}}
\pgfpathlineto{\pgfpoint{18.219387\du}{9.349799\du}}
\pgfpathlineto{\pgfpoint{18.224498\du}{9.356371\du}}
\pgfpathlineto{\pgfpoint{18.229244\du}{9.362943\du}}
\pgfpathlineto{\pgfpoint{18.233625\du}{9.369880\du}}
\pgfpathlineto{\pgfpoint{18.238007\du}{9.376816\du}}
\pgfpathlineto{\pgfpoint{18.240927\du}{9.383388\du}}
\pgfpathlineto{\pgfpoint{18.244943\du}{9.390325\du}}
\pgfpathlineto{\pgfpoint{18.247864\du}{9.396897\du}}
\pgfpathlineto{\pgfpoint{18.251880\du}{9.403834\du}}
\pgfpathlineto{\pgfpoint{18.254436\du}{9.410770\du}}
\pgfpathlineto{\pgfpoint{18.257357\du}{9.417707\du}}
\pgfpathlineto{\pgfpoint{18.259912\du}{9.424644\du}}
\pgfpathlineto{\pgfpoint{18.262103\du}{9.431946\du}}
\pgfpathlineto{\pgfpoint{18.263563\du}{9.438153\du}}
\pgfpathlineto{\pgfpoint{18.265754\du}{9.445455\du}}
\pgfpathlineto{\pgfpoint{18.266484\du}{9.452391\du}}
\pgfpathlineto{\pgfpoint{18.267579\du}{9.459328\du}}
\pgfpathlineto{\pgfpoint{18.269405\du}{9.466265\du}}
\pgfpathlineto{\pgfpoint{18.269770\du}{9.473567\du}}
\pgfpathlineto{\pgfpoint{18.269770\du}{9.480504\du}}
\pgfpathlineto{\pgfpoint{18.270500\du}{9.487441\du}}
\pgfpathlineto{\pgfpoint{18.290581\du}{9.487441\du}}
\pgfusepath{fill}
\pgfsetbuttcap
\pgfsetmiterjoin
\pgfsetdash{}{0pt}
\definecolor{dialinecolor}{rgb}{0.074510, 0.082353, 0.086275}
\pgfsetfillcolor{dialinecolor}
\pgfpathmoveto{\pgfpoint{16.643264\du}{9.360022\du}}
\pgfpathlineto{\pgfpoint{16.891165\du}{9.442534\du}}
\pgfpathlineto{\pgfpoint{17.476415\du}{9.208142\du}}
\pgfpathlineto{\pgfpoint{17.749142\du}{9.275685\du}}
\pgfpathlineto{\pgfpoint{17.605294\du}{9.067214\du}}
\pgfpathlineto{\pgfpoint{16.901022\du}{9.067214\du}}
\pgfpathlineto{\pgfpoint{17.195290\du}{9.139869\du}}
\pgfpathlineto{\pgfpoint{16.643264\du}{9.360022\du}}
\pgfusepath{fill}
\pgfsetbuttcap
\pgfsetmiterjoin
\pgfsetdash{}{0pt}
\definecolor{dialinecolor}{rgb}{0.074510, 0.082353, 0.086275}
\pgfsetfillcolor{dialinecolor}
\pgfpathmoveto{\pgfpoint{16.541402\du}{9.598065\du}}
\pgfpathlineto{\pgfpoint{16.293501\du}{9.515918\du}}
\pgfpathlineto{\pgfpoint{15.708251\du}{9.749580\du}}
\pgfpathlineto{\pgfpoint{15.435159\du}{9.682767\du}}
\pgfpathlineto{\pgfpoint{15.579007\du}{9.890507\du}}
\pgfpathlineto{\pgfpoint{16.284374\du}{9.890507\du}}
\pgfpathlineto{\pgfpoint{15.989376\du}{9.818583\du}}
\pgfpathlineto{\pgfpoint{16.541402\du}{9.598065\du}}
\pgfusepath{fill}
\pgfsetbuttcap
\pgfsetmiterjoin
\pgfsetdash{}{0pt}
\definecolor{dialinecolor}{rgb}{0.074510, 0.082353, 0.086275}
\pgfsetfillcolor{dialinecolor}
\pgfpathmoveto{\pgfpoint{15.495400\du}{9.139138\du}}
\pgfpathlineto{\pgfpoint{15.742935\du}{9.057357\du}}
\pgfpathlineto{\pgfpoint{16.328185\du}{9.290654\du}}
\pgfpathlineto{\pgfpoint{16.601278\du}{9.224206\du}}
\pgfpathlineto{\pgfpoint{16.457430\du}{9.431946\du}}
\pgfpathlineto{\pgfpoint{15.752428\du}{9.431946\du}}
\pgfpathlineto{\pgfpoint{16.047426\du}{9.360022\du}}
\pgfpathlineto{\pgfpoint{15.495400\du}{9.139138\du}}
\pgfusepath{fill}
\pgfsetbuttcap
\pgfsetmiterjoin
\pgfsetdash{}{0pt}
\definecolor{dialinecolor}{rgb}{0.074510, 0.082353, 0.086275}
\pgfsetfillcolor{dialinecolor}
\pgfpathmoveto{\pgfpoint{17.713728\du}{9.834283\du}}
\pgfpathlineto{\pgfpoint{17.466192\du}{9.916429\du}}
\pgfpathlineto{\pgfpoint{16.880942\du}{9.682767\du}}
\pgfpathlineto{\pgfpoint{16.607119\du}{9.749580\du}}
\pgfpathlineto{\pgfpoint{16.751698\du}{9.541840\du}}
\pgfpathlineto{\pgfpoint{17.457065\du}{9.541840\du}}
\pgfpathlineto{\pgfpoint{17.161701\du}{9.613764\du}}
\pgfpathlineto{\pgfpoint{17.713728\du}{9.834283\du}}
\pgfusepath{fill}
\pgfsetbuttcap
\pgfsetmiterjoin
\pgfsetdash{}{0pt}
\definecolor{dialinecolor}{rgb}{1.000000, 1.000000, 1.000000}
\pgfsetfillcolor{dialinecolor}
\pgfpathmoveto{\pgfpoint{16.664074\du}{9.380467\du}}
\pgfpathlineto{\pgfpoint{16.911610\du}{9.462979\du}}
\pgfpathlineto{\pgfpoint{17.496860\du}{9.228952\du}}
\pgfpathlineto{\pgfpoint{17.769222\du}{9.296130\du}}
\pgfpathlineto{\pgfpoint{17.626470\du}{9.087660\du}}
\pgfpathlineto{\pgfpoint{16.921103\du}{9.087660\du}}
\pgfpathlineto{\pgfpoint{17.216101\du}{9.160314\du}}
\pgfpathlineto{\pgfpoint{16.664074\du}{9.380467\du}}
\pgfusepath{fill}
\pgfsetbuttcap
\pgfsetmiterjoin
\pgfsetdash{}{0pt}
\definecolor{dialinecolor}{rgb}{1.000000, 1.000000, 1.000000}
\pgfsetfillcolor{dialinecolor}
\pgfpathmoveto{\pgfpoint{16.562212\du}{9.619241\du}}
\pgfpathlineto{\pgfpoint{16.313582\du}{9.536729\du}}
\pgfpathlineto{\pgfpoint{15.728697\du}{9.770756\du}}
\pgfpathlineto{\pgfpoint{15.455239\du}{9.703213\du}}
\pgfpathlineto{\pgfpoint{15.600183\du}{9.911683\du}}
\pgfpathlineto{\pgfpoint{16.304819\du}{9.911683\du}}
\pgfpathlineto{\pgfpoint{16.010186\du}{9.839029\du}}
\pgfpathlineto{\pgfpoint{16.562212\du}{9.619241\du}}
\pgfusepath{fill}
\pgfsetbuttcap
\pgfsetmiterjoin
\pgfsetdash{}{0pt}
\definecolor{dialinecolor}{rgb}{1.000000, 1.000000, 1.000000}
\pgfsetfillcolor{dialinecolor}
\pgfpathmoveto{\pgfpoint{15.515845\du}{9.159584\du}}
\pgfpathlineto{\pgfpoint{15.763381\du}{9.077802\du}}
\pgfpathlineto{\pgfpoint{16.348996\du}{9.311829\du}}
\pgfpathlineto{\pgfpoint{16.622088\du}{9.245016\du}}
\pgfpathlineto{\pgfpoint{16.477145\du}{9.452391\du}}
\pgfpathlineto{\pgfpoint{15.772873\du}{9.452391\du}}
\pgfpathlineto{\pgfpoint{16.067506\du}{9.380467\du}}
\pgfpathlineto{\pgfpoint{15.515845\du}{9.159584\du}}
\pgfusepath{fill}
\pgfsetbuttcap
\pgfsetmiterjoin
\pgfsetdash{}{0pt}
\definecolor{dialinecolor}{rgb}{1.000000, 1.000000, 1.000000}
\pgfsetfillcolor{dialinecolor}
\pgfpathmoveto{\pgfpoint{17.733808\du}{9.854728\du}}
\pgfpathlineto{\pgfpoint{17.486272\du}{9.936875\du}}
\pgfpathlineto{\pgfpoint{16.901387\du}{9.703213\du}}
\pgfpathlineto{\pgfpoint{16.627930\du}{9.770026\du}}
\pgfpathlineto{\pgfpoint{16.772143\du}{9.562286\du}}
\pgfpathlineto{\pgfpoint{17.477145\du}{9.562286\du}}
\pgfpathlineto{\pgfpoint{17.182877\du}{9.634210\du}}
\pgfpathlineto{\pgfpoint{17.733808\du}{9.854728\du}}
\pgfusepath{fill}
\pgfsetbuttcap
\pgfsetmiterjoin
\pgfsetdash{}{0pt}
\definecolor{dialinecolor}{rgb}{0.678431, 0.839216, 0.905882}
\pgfsetfillcolor{dialinecolor}
\pgfpathmoveto{\pgfpoint{14.929865\du}{9.498028\du}}
\pgfpathlineto{\pgfpoint{14.929865\du}{9.487441\du}}
\pgfpathlineto{\pgfpoint{14.909419\du}{9.487441\du}}
\pgfpathlineto{\pgfpoint{14.909419\du}{9.498028\du}}
\pgfpathlineto{\pgfpoint{14.929865\du}{9.498028\du}}
\pgfusepath{fill}
\pgfsetbuttcap
\pgfsetmiterjoin
\pgfsetdash{}{0pt}
\definecolor{dialinecolor}{rgb}{0.678431, 0.839216, 0.905882}
\pgfsetfillcolor{dialinecolor}
\pgfpathmoveto{\pgfpoint{14.929865\du}{10.327528\du}}
\pgfpathlineto{\pgfpoint{14.929865\du}{9.498028\du}}
\pgfpathlineto{\pgfpoint{14.909419\du}{9.498028\du}}
\pgfpathlineto{\pgfpoint{14.909419\du}{10.327528\du}}
\pgfpathlineto{\pgfpoint{14.929865\du}{10.327528\du}}
\pgfusepath{fill}
\pgfsetbuttcap
\pgfsetmiterjoin
\pgfsetdash{}{0pt}
\definecolor{dialinecolor}{rgb}{0.678431, 0.839216, 0.905882}
\pgfsetfillcolor{dialinecolor}
\pgfpathmoveto{\pgfpoint{14.909419\du}{10.327528\du}}
\pgfpathlineto{\pgfpoint{14.909419\du}{10.338116\du}}
\pgfpathlineto{\pgfpoint{14.929865\du}{10.338116\du}}
\pgfpathlineto{\pgfpoint{14.929865\du}{10.327528\du}}
\pgfpathlineto{\pgfpoint{14.909419\du}{10.327528\du}}
\pgfusepath{fill}
\pgfsetbuttcap
\pgfsetmiterjoin
\pgfsetdash{}{0pt}
\definecolor{dialinecolor}{rgb}{0.678431, 0.839216, 0.905882}
\pgfsetfillcolor{dialinecolor}
\pgfpathmoveto{\pgfpoint{18.290581\du}{9.498028\du}}
\pgfpathlineto{\pgfpoint{18.290581\du}{9.487441\du}}
\pgfpathlineto{\pgfpoint{18.270500\du}{9.487441\du}}
\pgfpathlineto{\pgfpoint{18.270500\du}{9.498028\du}}
\pgfpathlineto{\pgfpoint{18.290581\du}{9.498028\du}}
\pgfusepath{fill}
\pgfsetbuttcap
\pgfsetmiterjoin
\pgfsetdash{}{0pt}
\definecolor{dialinecolor}{rgb}{0.678431, 0.839216, 0.905882}
\pgfsetfillcolor{dialinecolor}
\pgfpathmoveto{\pgfpoint{18.290581\du}{10.327528\du}}
\pgfpathlineto{\pgfpoint{18.290581\du}{9.498028\du}}
\pgfpathlineto{\pgfpoint{18.270500\du}{9.498028\du}}
\pgfpathlineto{\pgfpoint{18.270500\du}{10.327528\du}}
\pgfpathlineto{\pgfpoint{18.290581\du}{10.327528\du}}
\pgfusepath{fill}
\pgfsetbuttcap
\pgfsetmiterjoin
\pgfsetdash{}{0pt}
\definecolor{dialinecolor}{rgb}{0.678431, 0.839216, 0.905882}
\pgfsetfillcolor{dialinecolor}
\pgfpathmoveto{\pgfpoint{18.270500\du}{10.327528\du}}
\pgfpathlineto{\pgfpoint{18.270500\du}{10.338116\du}}
\pgfpathlineto{\pgfpoint{18.290581\du}{10.338116\du}}
\pgfpathlineto{\pgfpoint{18.290581\du}{10.327528\du}}
\pgfpathlineto{\pgfpoint{18.270500\du}{10.327528\du}}
\pgfusepath{fill}
\pgfsetbuttcap
\pgfsetmiterjoin
\pgfsetdash{}{0pt}
\definecolor{dialinecolor}{rgb}{0.027451, 0.372549, 0.529412}
\pgfsetfillcolor{dialinecolor}
\pgfpathmoveto{\pgfpoint{17.216831\du}{10.476853\du}}
\pgfpathlineto{\pgfpoint{17.216466\du}{10.494378\du}}
\pgfpathlineto{\pgfpoint{17.213910\du}{10.511537\du}}
\pgfpathlineto{\pgfpoint{17.210259\du}{10.527601\du}}
\pgfpathlineto{\pgfpoint{17.204418\du}{10.544761\du}}
\pgfpathlineto{\pgfpoint{17.197846\du}{10.560460\du}}
\pgfpathlineto{\pgfpoint{17.189449\du}{10.576524\du}}
\pgfpathlineto{\pgfpoint{17.179956\du}{10.592223\du}}
\pgfpathlineto{\pgfpoint{17.168273\du}{10.607192\du}}
\pgfpathlineto{\pgfpoint{17.156590\du}{10.622161\du}}
\pgfpathlineto{\pgfpoint{17.143081\du}{10.636765\du}}
\pgfpathlineto{\pgfpoint{17.128112\du}{10.650639\du}}
\pgfpathlineto{\pgfpoint{17.112048\du}{10.664878\du}}
\pgfpathlineto{\pgfpoint{17.094524\du}{10.677656\du}}
\pgfpathlineto{\pgfpoint{17.076269\du}{10.690434\du}}
\pgfpathlineto{\pgfpoint{17.057284\du}{10.702848\du}}
\pgfpathlineto{\pgfpoint{17.037203\du}{10.714531\du}}
\pgfpathlineto{\pgfpoint{17.015663\du}{10.725119\du}}
\pgfpathlineto{\pgfpoint{16.993027\du}{10.736072\du}}
\pgfpathlineto{\pgfpoint{16.970026\du}{10.745929\du}}
\pgfpathlineto{\pgfpoint{16.946294\du}{10.755422\du}}
\pgfpathlineto{\pgfpoint{16.921103\du}{10.763454\du}}
\pgfpathlineto{\pgfpoint{16.895546\du}{10.771851\du}}
\pgfpathlineto{\pgfpoint{16.868894\du}{10.779518\du}}
\pgfpathlineto{\pgfpoint{16.842242\du}{10.786455\du}}
\pgfpathlineto{\pgfpoint{16.814129\du}{10.792296\du}}
\pgfpathlineto{\pgfpoint{16.785652\du}{10.797408\du}}
\pgfpathlineto{\pgfpoint{16.756079\du}{10.801789\du}}
\pgfpathlineto{\pgfpoint{16.725776\du}{10.805805\du}}
\pgfpathlineto{\pgfpoint{16.695838\du}{10.808726\du}}
\pgfpathlineto{\pgfpoint{16.665170\du}{10.810916\du}}
\pgfpathlineto{\pgfpoint{16.633771\du}{10.812012\du}}
\pgfpathlineto{\pgfpoint{16.602373\du}{10.812742\du}}
\pgfpathlineto{\pgfpoint{16.571340\du}{10.812012\du}}
\pgfpathlineto{\pgfpoint{16.539942\du}{10.810916\du}}
\pgfpathlineto{\pgfpoint{16.508908\du}{10.808726\du}}
\pgfpathlineto{\pgfpoint{16.478605\du}{10.805805\du}}
\pgfpathlineto{\pgfpoint{16.449032\du}{10.801789\du}}
\pgfpathlineto{\pgfpoint{16.419460\du}{10.797408\du}}
\pgfpathlineto{\pgfpoint{16.390982\du}{10.792296\du}}
\pgfpathlineto{\pgfpoint{16.363235\du}{10.786455\du}}
\pgfpathlineto{\pgfpoint{16.335853\du}{10.779518\du}}
\pgfpathlineto{\pgfpoint{16.309566\du}{10.771851\du}}
\pgfpathlineto{\pgfpoint{16.284374\du}{10.763454\du}}
\pgfpathlineto{\pgfpoint{16.258817\du}{10.755422\du}}
\pgfpathlineto{\pgfpoint{16.234721\du}{10.745929\du}}
\pgfpathlineto{\pgfpoint{16.212085\du}{10.736072\du}}
\pgfpathlineto{\pgfpoint{16.189449\du}{10.725119\du}}
\pgfpathlineto{\pgfpoint{16.168638\du}{10.714531\du}}
\pgfpathlineto{\pgfpoint{16.148193\du}{10.702848\du}}
\pgfpathlineto{\pgfpoint{16.127747\du}{10.690434\du}}
\pgfpathlineto{\pgfpoint{16.109858\du}{10.677656\du}}
\pgfpathlineto{\pgfpoint{16.093428\du}{10.664878\du}}
\pgfpathlineto{\pgfpoint{16.076634\du}{10.650639\du}}
\pgfpathlineto{\pgfpoint{16.062030\du}{10.636765\du}}
\pgfpathlineto{\pgfpoint{16.048886\du}{10.622161\du}}
\pgfpathlineto{\pgfpoint{16.036473\du}{10.607192\du}}
\pgfpathlineto{\pgfpoint{16.025520\du}{10.592223\du}}
\pgfpathlineto{\pgfpoint{16.016028\du}{10.576524\du}}
\pgfpathlineto{\pgfpoint{16.007265\du}{10.560460\du}}
\pgfpathlineto{\pgfpoint{16.000329\du}{10.544761\du}}
\pgfpathlineto{\pgfpoint{15.995582\du}{10.527601\du}}
\pgfpathlineto{\pgfpoint{15.991201\du}{10.511537\du}}
\pgfpathlineto{\pgfpoint{15.988645\du}{10.494378\du}}
\pgfpathlineto{\pgfpoint{15.987550\du}{10.476853\du}}
\pgfpathlineto{\pgfpoint{15.988645\du}{10.459328\du}}
\pgfpathlineto{\pgfpoint{15.991201\du}{10.442169\du}}
\pgfpathlineto{\pgfpoint{15.995582\du}{10.426104\du}}
\pgfpathlineto{\pgfpoint{16.000329\du}{10.408945\du}}
\pgfpathlineto{\pgfpoint{16.007265\du}{10.393246\du}}
\pgfpathlineto{\pgfpoint{16.016028\du}{10.377547\du}}
\pgfpathlineto{\pgfpoint{16.025520\du}{10.361482\du}}
\pgfpathlineto{\pgfpoint{16.036473\du}{10.346513\du}}
\pgfpathlineto{\pgfpoint{16.048886\du}{10.331909\du}}
\pgfpathlineto{\pgfpoint{16.062030\du}{10.317306\du}}
\pgfpathlineto{\pgfpoint{16.076634\du}{10.303067\du}}
\pgfpathlineto{\pgfpoint{16.093428\du}{10.289193\du}}
\pgfpathlineto{\pgfpoint{16.109858\du}{10.276050\du}}
\pgfpathlineto{\pgfpoint{16.127747\du}{10.264001\du}}
\pgfpathlineto{\pgfpoint{16.148193\du}{10.251588\du}}
\pgfpathlineto{\pgfpoint{16.168638\du}{10.239905\du}}
\pgfpathlineto{\pgfpoint{16.189449\du}{10.228952\du}}
\pgfpathlineto{\pgfpoint{16.212085\du}{10.218364\du}}
\pgfpathlineto{\pgfpoint{16.234721\du}{10.207777\du}}
\pgfpathlineto{\pgfpoint{16.258817\du}{10.199014\du}}
\pgfpathlineto{\pgfpoint{16.284374\du}{10.190252\du}}
\pgfpathlineto{\pgfpoint{16.309566\du}{10.181855\du}}
\pgfpathlineto{\pgfpoint{16.335853\du}{10.174188\du}}
\pgfpathlineto{\pgfpoint{16.363235\du}{10.167981\du}}
\pgfpathlineto{\pgfpoint{16.390982\du}{10.161409\du}}
\pgfpathlineto{\pgfpoint{16.419460\du}{10.156298\du}}
\pgfpathlineto{\pgfpoint{16.449032\du}{10.152282\du}}
\pgfpathlineto{\pgfpoint{16.478605\du}{10.147901\du}}
\pgfpathlineto{\pgfpoint{16.508908\du}{10.144980\du}}
\pgfpathlineto{\pgfpoint{16.539942\du}{10.143520\du}}
\pgfpathlineto{\pgfpoint{16.571340\du}{10.141694\du}}
\pgfpathlineto{\pgfpoint{16.602373\du}{10.141694\du}}
\pgfpathlineto{\pgfpoint{16.633771\du}{10.141694\du}}
\pgfpathlineto{\pgfpoint{16.665170\du}{10.143520\du}}
\pgfpathlineto{\pgfpoint{16.695838\du}{10.144980\du}}
\pgfpathlineto{\pgfpoint{16.725776\du}{10.147901\du}}
\pgfpathlineto{\pgfpoint{16.756079\du}{10.152282\du}}
\pgfpathlineto{\pgfpoint{16.785652\du}{10.156298\du}}
\pgfpathlineto{\pgfpoint{16.814129\du}{10.161409\du}}
\pgfpathlineto{\pgfpoint{16.842242\du}{10.167981\du}}
\pgfpathlineto{\pgfpoint{16.868894\du}{10.174188\du}}
\pgfpathlineto{\pgfpoint{16.895546\du}{10.181855\du}}
\pgfpathlineto{\pgfpoint{16.921103\du}{10.190252\du}}
\pgfpathlineto{\pgfpoint{16.946294\du}{10.199014\du}}
\pgfpathlineto{\pgfpoint{16.970026\du}{10.207777\du}}
\pgfpathlineto{\pgfpoint{16.993027\du}{10.218364\du}}
\pgfpathlineto{\pgfpoint{17.015663\du}{10.228952\du}}
\pgfpathlineto{\pgfpoint{17.037203\du}{10.239905\du}}
\pgfpathlineto{\pgfpoint{17.057284\du}{10.251588\du}}
\pgfpathlineto{\pgfpoint{17.076269\du}{10.264001\du}}
\pgfpathlineto{\pgfpoint{17.094524\du}{10.276050\du}}
\pgfpathlineto{\pgfpoint{17.112048\du}{10.289193\du}}
\pgfpathlineto{\pgfpoint{17.128112\du}{10.303067\du}}
\pgfpathlineto{\pgfpoint{17.143081\du}{10.317306\du}}
\pgfpathlineto{\pgfpoint{17.156590\du}{10.331909\du}}
\pgfpathlineto{\pgfpoint{17.168273\du}{10.346513\du}}
\pgfpathlineto{\pgfpoint{17.179956\du}{10.361482\du}}
\pgfpathlineto{\pgfpoint{17.189449\du}{10.377547\du}}
\pgfpathlineto{\pgfpoint{17.197846\du}{10.393246\du}}
\pgfpathlineto{\pgfpoint{17.204418\du}{10.408945\du}}
\pgfpathlineto{\pgfpoint{17.210259\du}{10.426104\du}}
\pgfpathlineto{\pgfpoint{17.213910\du}{10.442169\du}}
\pgfpathlineto{\pgfpoint{17.216466\du}{10.459328\du}}
\pgfpathlineto{\pgfpoint{17.216831\du}{10.476853\du}}
\pgfusepath{fill}
\pgfsetbuttcap
\pgfsetmiterjoin
\pgfsetdash{}{0pt}
\definecolor{dialinecolor}{rgb}{0.678431, 0.839216, 0.905882}
\pgfsetfillcolor{dialinecolor}
\pgfpathmoveto{\pgfpoint{16.602373\du}{10.822599\du}}
\pgfpathlineto{\pgfpoint{16.602373\du}{10.822599\du}}
\pgfpathlineto{\pgfpoint{16.618437\du}{10.822599\du}}
\pgfpathlineto{\pgfpoint{16.634502\du}{10.822234\du}}
\pgfpathlineto{\pgfpoint{16.650566\du}{10.821504\du}}
\pgfpathlineto{\pgfpoint{16.665900\du}{10.820774\du}}
\pgfpathlineto{\pgfpoint{16.681599\du}{10.819679\du}}
\pgfpathlineto{\pgfpoint{16.696933\du}{10.818583\du}}
\pgfpathlineto{\pgfpoint{16.711902\du}{10.817488\du}}
\pgfpathlineto{\pgfpoint{16.727966\du}{10.815663\du}}
\pgfpathlineto{\pgfpoint{16.742205\du}{10.813837\du}}
\pgfpathlineto{\pgfpoint{16.757174\du}{10.812012\du}}
\pgfpathlineto{\pgfpoint{16.772143\du}{10.809821\du}}
\pgfpathlineto{\pgfpoint{16.787477\du}{10.807631\du}}
\pgfpathlineto{\pgfpoint{16.801716\du}{10.804710\du}}
\pgfpathlineto{\pgfpoint{16.815590\du}{10.802154\du}}
\pgfpathlineto{\pgfpoint{16.830194\du}{10.799233\du}}
\pgfpathlineto{\pgfpoint{16.844067\du}{10.796313\du}}
\pgfpathlineto{\pgfpoint{16.857576\du}{10.792662\du}}
\pgfpathlineto{\pgfpoint{16.871449\du}{10.789376\du}}
\pgfpathlineto{\pgfpoint{16.884958\du}{10.785725\du}}
\pgfpathlineto{\pgfpoint{16.898101\du}{10.781709\du}}
\pgfpathlineto{\pgfpoint{16.911245\du}{10.777693\du}}
\pgfpathlineto{\pgfpoint{16.924388\du}{10.773677\du}}
\pgfpathlineto{\pgfpoint{16.937167\du}{10.768930\du}}
\pgfpathlineto{\pgfpoint{16.949945\du}{10.764914\du}}
\pgfpathlineto{\pgfpoint{16.961628\du}{10.760168\du}}
\pgfpathlineto{\pgfpoint{16.974407\du}{10.755422\du}}
\pgfpathlineto{\pgfpoint{16.985725\du}{10.749945\du}}
\pgfpathlineto{\pgfpoint{16.997408\du}{10.744834\du}}
\pgfpathlineto{\pgfpoint{17.008726\du}{10.739723\du}}
\pgfpathlineto{\pgfpoint{17.020044\du}{10.734246\du}}
\pgfpathlineto{\pgfpoint{17.030997\du}{10.729135\du}}
\pgfpathlineto{\pgfpoint{17.042315\du}{10.723293\du}}
\pgfpathlineto{\pgfpoint{17.052172\du}{10.717452\du}}
\pgfpathlineto{\pgfpoint{17.062395\du}{10.710880\du}}
\pgfpathlineto{\pgfpoint{17.072253\du}{10.705038\du}}
\pgfpathlineto{\pgfpoint{17.082475\du}{10.698467\du}}
\pgfpathlineto{\pgfpoint{17.091968\du}{10.692260\du}}
\pgfpathlineto{\pgfpoint{17.101095\du}{10.685688\du}}
\pgfpathlineto{\pgfpoint{17.109858\du}{10.679482\du}}
\pgfpathlineto{\pgfpoint{17.118255\du}{10.672180\du}}
\pgfpathlineto{\pgfpoint{17.126652\du}{10.665243\du}}
\pgfpathlineto{\pgfpoint{17.134684\du}{10.658306\du}}
\pgfpathlineto{\pgfpoint{17.142716\du}{10.651369\du}}
\pgfpathlineto{\pgfpoint{17.149653\du}{10.644067\du}}
\pgfpathlineto{\pgfpoint{17.157320\du}{10.636765\du}}
\pgfpathlineto{\pgfpoint{17.163892\du}{10.629098\du}}
\pgfpathlineto{\pgfpoint{17.170464\du}{10.621431\du}}
\pgfpathlineto{\pgfpoint{17.176670\du}{10.613764\du}}
\pgfpathlineto{\pgfpoint{17.182877\du}{10.605732\du}}
\pgfpathlineto{\pgfpoint{17.188719\du}{10.598065\du}}
\pgfpathlineto{\pgfpoint{17.193465\du}{10.589668\du}}
\pgfpathlineto{\pgfpoint{17.198211\du}{10.581636\du}}
\pgfpathlineto{\pgfpoint{17.202957\du}{10.573238\du}}
\pgfpathlineto{\pgfpoint{17.207338\du}{10.565206\du}}
\pgfpathlineto{\pgfpoint{17.210624\du}{10.556444\du}}
\pgfpathlineto{\pgfpoint{17.214640\du}{10.548412\du}}
\pgfpathlineto{\pgfpoint{17.217196\du}{10.539650\du}}
\pgfpathlineto{\pgfpoint{17.220117\du}{10.530887\du}}
\pgfpathlineto{\pgfpoint{17.221942\du}{10.521760\du}}
\pgfpathlineto{\pgfpoint{17.224133\du}{10.512997\du}}
\pgfpathlineto{\pgfpoint{17.225593\du}{10.504235\du}}
\pgfpathlineto{\pgfpoint{17.226689\du}{10.495108\du}}
\pgfpathlineto{\pgfpoint{17.227419\du}{10.486345\du}}
\pgfpathlineto{\pgfpoint{17.227419\du}{10.476853\du}}
\pgfpathlineto{\pgfpoint{17.207338\du}{10.476853\du}}
\pgfpathlineto{\pgfpoint{17.206608\du}{10.484885\du}}
\pgfpathlineto{\pgfpoint{17.206243\du}{10.493282\du}}
\pgfpathlineto{\pgfpoint{17.205513\du}{10.501314\du}}
\pgfpathlineto{\pgfpoint{17.204053\du}{10.508981\du}}
\pgfpathlineto{\pgfpoint{17.202592\du}{10.517379\du}}
\pgfpathlineto{\pgfpoint{17.200037\du}{10.525411\du}}
\pgfpathlineto{\pgfpoint{17.197846\du}{10.533078\du}}
\pgfpathlineto{\pgfpoint{17.194560\du}{10.540745\du}}
\pgfpathlineto{\pgfpoint{17.192004\du}{10.548412\du}}
\pgfpathlineto{\pgfpoint{17.188719\du}{10.556444\du}}
\pgfpathlineto{\pgfpoint{17.184702\du}{10.564111\du}}
\pgfpathlineto{\pgfpoint{17.179956\du}{10.571778\du}}
\pgfpathlineto{\pgfpoint{17.175940\du}{10.579445\du}}
\pgfpathlineto{\pgfpoint{17.170829\du}{10.586382\du}}
\pgfpathlineto{\pgfpoint{17.166083\du}{10.594049\du}}
\pgfpathlineto{\pgfpoint{17.160606\du}{10.600986\du}}
\pgfpathlineto{\pgfpoint{17.155495\du}{10.608653\du}}
\pgfpathlineto{\pgfpoint{17.148558\du}{10.615590\du}}
\pgfpathlineto{\pgfpoint{17.142716\du}{10.622526\du}}
\pgfpathlineto{\pgfpoint{17.135414\du}{10.629463\du}}
\pgfpathlineto{\pgfpoint{17.128478\du}{10.636765\du}}
\pgfpathlineto{\pgfpoint{17.121176\du}{10.642972\du}}
\pgfpathlineto{\pgfpoint{17.114239\du}{10.649909\du}}
\pgfpathlineto{\pgfpoint{17.105842\du}{10.656480\du}}
\pgfpathlineto{\pgfpoint{17.097809\du}{10.663052\du}}
\pgfpathlineto{\pgfpoint{17.088682\du}{10.669259\du}}
\pgfpathlineto{\pgfpoint{17.079920\du}{10.675100\du}}
\pgfpathlineto{\pgfpoint{17.070427\du}{10.681672\du}}
\pgfpathlineto{\pgfpoint{17.061300\du}{10.688244\du}}
\pgfpathlineto{\pgfpoint{17.052172\du}{10.693355\du}}
\pgfpathlineto{\pgfpoint{17.042315\du}{10.699197\du}}
\pgfpathlineto{\pgfpoint{17.032457\du}{10.705038\du}}
\pgfpathlineto{\pgfpoint{17.021869\du}{10.710515\du}}
\pgfpathlineto{\pgfpoint{17.011281\du}{10.716356\du}}
\pgfpathlineto{\pgfpoint{17.000694\du}{10.720737\du}}
\pgfpathlineto{\pgfpoint{16.989011\du}{10.726579\du}}
\pgfpathlineto{\pgfpoint{16.977693\du}{10.731325\du}}
\pgfpathlineto{\pgfpoint{16.966375\du}{10.736072\du}}
\pgfpathlineto{\pgfpoint{16.954691\du}{10.740818\du}}
\pgfpathlineto{\pgfpoint{16.942643\du}{10.745564\du}}
\pgfpathlineto{\pgfpoint{16.930230\du}{10.749580\du}}
\pgfpathlineto{\pgfpoint{16.918547\du}{10.754326\du}}
\pgfpathlineto{\pgfpoint{16.905769\du}{10.758342\du}}
\pgfpathlineto{\pgfpoint{16.892260\du}{10.761993\du}}
\pgfpathlineto{\pgfpoint{16.879482\du}{10.766009\du}}
\pgfpathlineto{\pgfpoint{16.866338\du}{10.769295\du}}
\pgfpathlineto{\pgfpoint{16.852829\du}{10.772946\du}}
\pgfpathlineto{\pgfpoint{16.839321\du}{10.775867\du}}
\pgfpathlineto{\pgfpoint{16.825447\du}{10.779518\du}}
\pgfpathlineto{\pgfpoint{16.811574\du}{10.781709\du}}
\pgfpathlineto{\pgfpoint{16.797700\du}{10.784629\du}}
\pgfpathlineto{\pgfpoint{16.783461\du}{10.786820\du}}
\pgfpathlineto{\pgfpoint{16.769222\du}{10.789376\du}}
\pgfpathlineto{\pgfpoint{16.754984\du}{10.791566\du}}
\pgfpathlineto{\pgfpoint{16.740380\du}{10.793392\du}}
\pgfpathlineto{\pgfpoint{16.725411\du}{10.795217\du}}
\pgfpathlineto{\pgfpoint{16.710077\du}{10.797043\du}}
\pgfpathlineto{\pgfpoint{16.695473\du}{10.798138\du}}
\pgfpathlineto{\pgfpoint{16.679774\du}{10.799233\du}}
\pgfpathlineto{\pgfpoint{16.664805\du}{10.800329\du}}
\pgfpathlineto{\pgfpoint{16.649471\du}{10.801059\du}}
\pgfpathlineto{\pgfpoint{16.633771\du}{10.801789\du}}
\pgfpathlineto{\pgfpoint{16.618437\du}{10.802154\du}}
\pgfpathlineto{\pgfpoint{16.602373\du}{10.802154\du}}
\pgfpathlineto{\pgfpoint{16.602373\du}{10.802154\du}}
\pgfpathlineto{\pgfpoint{16.602373\du}{10.802154\du}}
\pgfpathlineto{\pgfpoint{16.601278\du}{10.802154\du}}
\pgfpathlineto{\pgfpoint{16.600183\du}{10.802154\du}}
\pgfpathlineto{\pgfpoint{16.599452\du}{10.802884\du}}
\pgfpathlineto{\pgfpoint{16.597627\du}{10.802884\du}}
\pgfpathlineto{\pgfpoint{16.597262\du}{10.803249\du}}
\pgfpathlineto{\pgfpoint{16.596166\du}{10.803980\du}}
\pgfpathlineto{\pgfpoint{16.595801\du}{10.804710\du}}
\pgfpathlineto{\pgfpoint{16.595436\du}{10.805075\du}}
\pgfpathlineto{\pgfpoint{16.593976\du}{10.806900\du}}
\pgfpathlineto{\pgfpoint{16.592516\du}{10.808726\du}}
\pgfpathlineto{\pgfpoint{16.592516\du}{10.810551\du}}
\pgfpathlineto{\pgfpoint{16.592150\du}{10.812742\du}}
\pgfpathlineto{\pgfpoint{16.592516\du}{10.814567\du}}
\pgfpathlineto{\pgfpoint{16.592516\du}{10.816393\du}}
\pgfpathlineto{\pgfpoint{16.593976\du}{10.817853\du}}
\pgfpathlineto{\pgfpoint{16.595436\du}{10.819314\du}}
\pgfpathlineto{\pgfpoint{16.595801\du}{10.820409\du}}
\pgfpathlineto{\pgfpoint{16.596166\du}{10.820774\du}}
\pgfpathlineto{\pgfpoint{16.597262\du}{10.821504\du}}
\pgfpathlineto{\pgfpoint{16.597627\du}{10.821504\du}}
\pgfpathlineto{\pgfpoint{16.599452\du}{10.822234\du}}
\pgfpathlineto{\pgfpoint{16.600183\du}{10.822599\du}}
\pgfpathlineto{\pgfpoint{16.601278\du}{10.822599\du}}
\pgfpathlineto{\pgfpoint{16.602373\du}{10.822599\du}}
\pgfusepath{fill}
\pgfsetbuttcap
\pgfsetmiterjoin
\pgfsetdash{}{0pt}
\definecolor{dialinecolor}{rgb}{0.678431, 0.839216, 0.905882}
\pgfsetfillcolor{dialinecolor}
\pgfpathmoveto{\pgfpoint{15.977693\du}{10.476853\du}}
\pgfpathlineto{\pgfpoint{15.977693\du}{10.476853\du}}
\pgfpathlineto{\pgfpoint{15.977693\du}{10.485615\du}}
\pgfpathlineto{\pgfpoint{15.978423\du}{10.495108\du}}
\pgfpathlineto{\pgfpoint{15.979518\du}{10.504235\du}}
\pgfpathlineto{\pgfpoint{15.981709\du}{10.512997\du}}
\pgfpathlineto{\pgfpoint{15.982804\du}{10.521760\du}}
\pgfpathlineto{\pgfpoint{15.985360\du}{10.530887\du}}
\pgfpathlineto{\pgfpoint{15.987550\du}{10.539650\du}}
\pgfpathlineto{\pgfpoint{15.990836\du}{10.548412\du}}
\pgfpathlineto{\pgfpoint{15.994122\du}{10.556444\du}}
\pgfpathlineto{\pgfpoint{15.998138\du}{10.565206\du}}
\pgfpathlineto{\pgfpoint{16.002519\du}{10.573238\du}}
\pgfpathlineto{\pgfpoint{16.006900\du}{10.581636\du}}
\pgfpathlineto{\pgfpoint{16.011647\du}{10.589668\du}}
\pgfpathlineto{\pgfpoint{16.016758\du}{10.598065\du}}
\pgfpathlineto{\pgfpoint{16.022965\du}{10.605732\du}}
\pgfpathlineto{\pgfpoint{16.027711\du}{10.613764\du}}
\pgfpathlineto{\pgfpoint{16.034283\du}{10.621431\du}}
\pgfpathlineto{\pgfpoint{16.041219\du}{10.629098\du}}
\pgfpathlineto{\pgfpoint{16.047791\du}{10.636765\du}}
\pgfpathlineto{\pgfpoint{16.054728\du}{10.644067\du}}
\pgfpathlineto{\pgfpoint{16.062030\du}{10.651369\du}}
\pgfpathlineto{\pgfpoint{16.070792\du}{10.658306\du}}
\pgfpathlineto{\pgfpoint{16.077729\du}{10.665243\du}}
\pgfpathlineto{\pgfpoint{16.086857\du}{10.672180\du}}
\pgfpathlineto{\pgfpoint{16.094889\du}{10.679482\du}}
\pgfpathlineto{\pgfpoint{16.104016\du}{10.685688\du}}
\pgfpathlineto{\pgfpoint{16.113874\du}{10.692260\du}}
\pgfpathlineto{\pgfpoint{16.123001\du}{10.698467\du}}
\pgfpathlineto{\pgfpoint{16.132494\du}{10.705038\du}}
\pgfpathlineto{\pgfpoint{16.142351\du}{10.710880\du}}
\pgfpathlineto{\pgfpoint{16.152574\du}{10.717452\du}}
\pgfpathlineto{\pgfpoint{16.162797\du}{10.723293\du}}
\pgfpathlineto{\pgfpoint{16.173750\du}{10.729135\du}}
\pgfpathlineto{\pgfpoint{16.185068\du}{10.734246\du}}
\pgfpathlineto{\pgfpoint{16.196386\du}{10.739723\du}}
\pgfpathlineto{\pgfpoint{16.207704\du}{10.744834\du}}
\pgfpathlineto{\pgfpoint{16.219752\du}{10.749945\du}}
\pgfpathlineto{\pgfpoint{16.231070\du}{10.755422\du}}
\pgfpathlineto{\pgfpoint{16.243483\du}{10.760168\du}}
\pgfpathlineto{\pgfpoint{16.255531\du}{10.764914\du}}
\pgfpathlineto{\pgfpoint{16.267579\du}{10.768930\du}}
\pgfpathlineto{\pgfpoint{16.280723\du}{10.773677\du}}
\pgfpathlineto{\pgfpoint{16.293501\du}{10.777693\du}}
\pgfpathlineto{\pgfpoint{16.307010\du}{10.781709\du}}
\pgfpathlineto{\pgfpoint{16.320518\du}{10.785725\du}}
\pgfpathlineto{\pgfpoint{16.333297\du}{10.789376\du}}
\pgfpathlineto{\pgfpoint{16.346805\du}{10.792662\du}}
\pgfpathlineto{\pgfpoint{16.360679\du}{10.796313\du}}
\pgfpathlineto{\pgfpoint{16.375283\du}{10.799233\du}}
\pgfpathlineto{\pgfpoint{16.389887\du}{10.802154\du}}
\pgfpathlineto{\pgfpoint{16.403760\du}{10.804710\du}}
\pgfpathlineto{\pgfpoint{16.417999\du}{10.807631\du}}
\pgfpathlineto{\pgfpoint{16.432968\du}{10.809821\du}}
\pgfpathlineto{\pgfpoint{16.447937\du}{10.812012\du}}
\pgfpathlineto{\pgfpoint{16.462541\du}{10.813837\du}}
\pgfpathlineto{\pgfpoint{16.477145\du}{10.815663\du}}
\pgfpathlineto{\pgfpoint{16.492844\du}{10.817488\du}}
\pgfpathlineto{\pgfpoint{16.508543\du}{10.818583\du}}
\pgfpathlineto{\pgfpoint{16.523147\du}{10.819679\du}}
\pgfpathlineto{\pgfpoint{16.539211\du}{10.820774\du}}
\pgfpathlineto{\pgfpoint{16.554545\du}{10.821504\du}}
\pgfpathlineto{\pgfpoint{16.570610\du}{10.822234\du}}
\pgfpathlineto{\pgfpoint{16.586674\du}{10.822599\du}}
\pgfpathlineto{\pgfpoint{16.602373\du}{10.822599\du}}
\pgfpathlineto{\pgfpoint{16.602373\du}{10.802154\du}}
\pgfpathlineto{\pgfpoint{16.586674\du}{10.802154\du}}
\pgfpathlineto{\pgfpoint{16.571340\du}{10.801789\du}}
\pgfpathlineto{\pgfpoint{16.555276\du}{10.801059\du}}
\pgfpathlineto{\pgfpoint{16.540672\du}{10.800329\du}}
\pgfpathlineto{\pgfpoint{16.524973\du}{10.799233\du}}
\pgfpathlineto{\pgfpoint{16.510004\du}{10.798138\du}}
\pgfpathlineto{\pgfpoint{16.495035\du}{10.797043\du}}
\pgfpathlineto{\pgfpoint{16.480066\du}{10.795217\du}}
\pgfpathlineto{\pgfpoint{16.465097\du}{10.793392\du}}
\pgfpathlineto{\pgfpoint{16.450493\du}{10.791566\du}}
\pgfpathlineto{\pgfpoint{16.435889\du}{10.789376\du}}
\pgfpathlineto{\pgfpoint{16.421650\du}{10.786820\du}}
\pgfpathlineto{\pgfpoint{16.407777\du}{10.784629\du}}
\pgfpathlineto{\pgfpoint{16.393903\du}{10.781709\du}}
\pgfpathlineto{\pgfpoint{16.379664\du}{10.779518\du}}
\pgfpathlineto{\pgfpoint{16.365790\du}{10.775867\du}}
\pgfpathlineto{\pgfpoint{16.352282\du}{10.772946\du}}
\pgfpathlineto{\pgfpoint{16.339138\du}{10.769295\du}}
\pgfpathlineto{\pgfpoint{16.325630\du}{10.766009\du}}
\pgfpathlineto{\pgfpoint{16.312851\du}{10.761993\du}}
\pgfpathlineto{\pgfpoint{16.299708\du}{10.758342\du}}
\pgfpathlineto{\pgfpoint{16.286930\du}{10.754326\du}}
\pgfpathlineto{\pgfpoint{16.274881\du}{10.749580\du}}
\pgfpathlineto{\pgfpoint{16.262468\du}{10.745564\du}}
\pgfpathlineto{\pgfpoint{16.250055\du}{10.740818\du}}
\pgfpathlineto{\pgfpoint{16.238737\du}{10.736072\du}}
\pgfpathlineto{\pgfpoint{16.227054\du}{10.731325\du}}
\pgfpathlineto{\pgfpoint{16.216101\du}{10.726579\du}}
\pgfpathlineto{\pgfpoint{16.204783\du}{10.720737\du}}
\pgfpathlineto{\pgfpoint{16.194195\du}{10.716356\du}}
\pgfpathlineto{\pgfpoint{16.183242\du}{10.710515\du}}
\pgfpathlineto{\pgfpoint{16.173019\du}{10.705038\du}}
\pgfpathlineto{\pgfpoint{16.162797\du}{10.699197\du}}
\pgfpathlineto{\pgfpoint{16.152939\du}{10.693355\du}}
\pgfpathlineto{\pgfpoint{16.143447\du}{10.688244\du}}
\pgfpathlineto{\pgfpoint{16.133589\du}{10.681672\du}}
\pgfpathlineto{\pgfpoint{16.125192\du}{10.675100\du}}
\pgfpathlineto{\pgfpoint{16.116064\du}{10.669259\du}}
\pgfpathlineto{\pgfpoint{16.107667\du}{10.663052\du}}
\pgfpathlineto{\pgfpoint{16.099270\du}{10.656480\du}}
\pgfpathlineto{\pgfpoint{16.091238\du}{10.649909\du}}
\pgfpathlineto{\pgfpoint{16.083936\du}{10.642972\du}}
\pgfpathlineto{\pgfpoint{16.076269\du}{10.636765\du}}
\pgfpathlineto{\pgfpoint{16.069697\du}{10.629463\du}}
\pgfpathlineto{\pgfpoint{16.062760\du}{10.622526\du}}
\pgfpathlineto{\pgfpoint{16.056553\du}{10.615590\du}}
\pgfpathlineto{\pgfpoint{16.050347\du}{10.608653\du}}
\pgfpathlineto{\pgfpoint{16.044140\du}{10.600986\du}}
\pgfpathlineto{\pgfpoint{16.039029\du}{10.594049\du}}
\pgfpathlineto{\pgfpoint{16.033917\du}{10.586382\du}}
\pgfpathlineto{\pgfpoint{16.029171\du}{10.579445\du}}
\pgfpathlineto{\pgfpoint{16.024790\du}{10.571778\du}}
\pgfpathlineto{\pgfpoint{16.020409\du}{10.564111\du}}
\pgfpathlineto{\pgfpoint{16.016758\du}{10.556444\du}}
\pgfpathlineto{\pgfpoint{16.013472\du}{10.548412\du}}
\pgfpathlineto{\pgfpoint{16.010186\du}{10.540745\du}}
\pgfpathlineto{\pgfpoint{16.007265\du}{10.533078\du}}
\pgfpathlineto{\pgfpoint{16.004710\du}{10.525411\du}}
\pgfpathlineto{\pgfpoint{16.002884\du}{10.517379\du}}
\pgfpathlineto{\pgfpoint{16.001059\du}{10.508981\du}}
\pgfpathlineto{\pgfpoint{15.999963\du}{10.501314\du}}
\pgfpathlineto{\pgfpoint{15.998503\du}{10.493282\du}}
\pgfpathlineto{\pgfpoint{15.998138\du}{10.484885\du}}
\pgfpathlineto{\pgfpoint{15.998138\du}{10.476853\du}}
\pgfpathlineto{\pgfpoint{15.998138\du}{10.476853\du}}
\pgfpathlineto{\pgfpoint{15.998138\du}{10.476853\du}}
\pgfpathlineto{\pgfpoint{15.998138\du}{10.475758\du}}
\pgfpathlineto{\pgfpoint{15.998138\du}{10.474662\du}}
\pgfpathlineto{\pgfpoint{15.997773\du}{10.473202\du}}
\pgfpathlineto{\pgfpoint{15.997773\du}{10.472107\du}}
\pgfpathlineto{\pgfpoint{15.997408\du}{10.471742\du}}
\pgfpathlineto{\pgfpoint{15.996313\du}{10.470281\du}}
\pgfpathlineto{\pgfpoint{15.995582\du}{10.469916\du}}
\pgfpathlineto{\pgfpoint{15.995582\du}{10.469186\du}}
\pgfpathlineto{\pgfpoint{15.993392\du}{10.468091\du}}
\pgfpathlineto{\pgfpoint{15.991566\du}{10.467360\du}}
\pgfpathlineto{\pgfpoint{15.989741\du}{10.466995\du}}
\pgfpathlineto{\pgfpoint{15.987550\du}{10.466265\du}}
\pgfpathlineto{\pgfpoint{15.986090\du}{10.466995\du}}
\pgfpathlineto{\pgfpoint{15.984264\du}{10.467360\du}}
\pgfpathlineto{\pgfpoint{15.982439\du}{10.468091\du}}
\pgfpathlineto{\pgfpoint{15.980613\du}{10.469186\du}}
\pgfpathlineto{\pgfpoint{15.979883\du}{10.469916\du}}
\pgfpathlineto{\pgfpoint{15.979518\du}{10.470281\du}}
\pgfpathlineto{\pgfpoint{15.979153\du}{10.471742\du}}
\pgfpathlineto{\pgfpoint{15.978423\du}{10.472107\du}}
\pgfpathlineto{\pgfpoint{15.978423\du}{10.473202\du}}
\pgfpathlineto{\pgfpoint{15.977693\du}{10.474662\du}}
\pgfpathlineto{\pgfpoint{15.977693\du}{10.475758\du}}
\pgfpathlineto{\pgfpoint{15.977693\du}{10.476853\du}}
\pgfusepath{fill}
\pgfsetbuttcap
\pgfsetmiterjoin
\pgfsetdash{}{0pt}
\definecolor{dialinecolor}{rgb}{0.678431, 0.839216, 0.905882}
\pgfsetfillcolor{dialinecolor}
\pgfpathmoveto{\pgfpoint{16.602373\du}{10.131106\du}}
\pgfpathlineto{\pgfpoint{16.602373\du}{10.131106\du}}
\pgfpathlineto{\pgfpoint{16.586674\du}{10.131106\du}}
\pgfpathlineto{\pgfpoint{16.570610\du}{10.131471\du}}
\pgfpathlineto{\pgfpoint{16.554545\du}{10.132202\du}}
\pgfpathlineto{\pgfpoint{16.539211\du}{10.132932\du}}
\pgfpathlineto{\pgfpoint{16.523147\du}{10.134027\du}}
\pgfpathlineto{\pgfpoint{16.508543\du}{10.135122\du}}
\pgfpathlineto{\pgfpoint{16.492844\du}{10.136218\du}}
\pgfpathlineto{\pgfpoint{16.477145\du}{10.138043\du}}
\pgfpathlineto{\pgfpoint{16.462541\du}{10.139869\du}}
\pgfpathlineto{\pgfpoint{16.447937\du}{10.141694\du}}
\pgfpathlineto{\pgfpoint{16.432968\du}{10.143885\du}}
\pgfpathlineto{\pgfpoint{16.417999\du}{10.146440\du}}
\pgfpathlineto{\pgfpoint{16.403760\du}{10.149361\du}}
\pgfpathlineto{\pgfpoint{16.389887\du}{10.151552\du}}
\pgfpathlineto{\pgfpoint{16.375283\du}{10.154472\du}}
\pgfpathlineto{\pgfpoint{16.360679\du}{10.157393\du}}
\pgfpathlineto{\pgfpoint{16.346805\du}{10.161044\du}}
\pgfpathlineto{\pgfpoint{16.333297\du}{10.164330\du}}
\pgfpathlineto{\pgfpoint{16.320518\du}{10.168346\du}}
\pgfpathlineto{\pgfpoint{16.307010\du}{10.171997\du}}
\pgfpathlineto{\pgfpoint{16.293501\du}{10.176013\du}}
\pgfpathlineto{\pgfpoint{16.280723\du}{10.180394\du}}
\pgfpathlineto{\pgfpoint{16.267579\du}{10.184775\du}}
\pgfpathlineto{\pgfpoint{16.255531\du}{10.189157\du}}
\pgfpathlineto{\pgfpoint{16.243483\du}{10.193538\du}}
\pgfpathlineto{\pgfpoint{16.231070\du}{10.199014\du}}
\pgfpathlineto{\pgfpoint{16.219752\du}{10.203760\du}}
\pgfpathlineto{\pgfpoint{16.207704\du}{10.208872\du}}
\pgfpathlineto{\pgfpoint{16.196386\du}{10.213983\du}}
\pgfpathlineto{\pgfpoint{16.185068\du}{10.219460\du}}
\pgfpathlineto{\pgfpoint{16.173750\du}{10.225301\du}}
\pgfpathlineto{\pgfpoint{16.162797\du}{10.230413\du}}
\pgfpathlineto{\pgfpoint{16.152574\du}{10.236254\du}}
\pgfpathlineto{\pgfpoint{16.142351\du}{10.242826\du}}
\pgfpathlineto{\pgfpoint{16.132494\du}{10.248667\du}}
\pgfpathlineto{\pgfpoint{16.123001\du}{10.255239\du}}
\pgfpathlineto{\pgfpoint{16.113874\du}{10.261446\du}}
\pgfpathlineto{\pgfpoint{16.104016\du}{10.268018\du}}
\pgfpathlineto{\pgfpoint{16.094889\du}{10.274589\du}}
\pgfpathlineto{\pgfpoint{16.086857\du}{10.281526\du}}
\pgfpathlineto{\pgfpoint{16.077729\du}{10.288463\du}}
\pgfpathlineto{\pgfpoint{16.070792\du}{10.295400\du}}
\pgfpathlineto{\pgfpoint{16.062030\du}{10.302337\du}}
\pgfpathlineto{\pgfpoint{16.054728\du}{10.309639\du}}
\pgfpathlineto{\pgfpoint{16.047791\du}{10.317306\du}}
\pgfpathlineto{\pgfpoint{16.041219\du}{10.324608\du}}
\pgfpathlineto{\pgfpoint{16.034283\du}{10.332275\du}}
\pgfpathlineto{\pgfpoint{16.027711\du}{10.339942\du}}
\pgfpathlineto{\pgfpoint{16.022965\du}{10.347974\du}}
\pgfpathlineto{\pgfpoint{16.016758\du}{10.355641\du}}
\pgfpathlineto{\pgfpoint{16.011647\du}{10.364038\du}}
\pgfpathlineto{\pgfpoint{16.006900\du}{10.372070\du}}
\pgfpathlineto{\pgfpoint{16.002519\du}{10.380467\du}}
\pgfpathlineto{\pgfpoint{15.998138\du}{10.388499\du}}
\pgfpathlineto{\pgfpoint{15.994122\du}{10.397262\du}}
\pgfpathlineto{\pgfpoint{15.990836\du}{10.405659\du}}
\pgfpathlineto{\pgfpoint{15.987550\du}{10.414421\du}}
\pgfpathlineto{\pgfpoint{15.985360\du}{10.423184\du}}
\pgfpathlineto{\pgfpoint{15.982804\du}{10.431946\du}}
\pgfpathlineto{\pgfpoint{15.981709\du}{10.440708\du}}
\pgfpathlineto{\pgfpoint{15.979518\du}{10.449471\du}}
\pgfpathlineto{\pgfpoint{15.978423\du}{10.458598\du}}
\pgfpathlineto{\pgfpoint{15.977693\du}{10.468091\du}}
\pgfpathlineto{\pgfpoint{15.977693\du}{10.476853\du}}
\pgfpathlineto{\pgfpoint{15.998138\du}{10.476853\du}}
\pgfpathlineto{\pgfpoint{15.998138\du}{10.468821\du}}
\pgfpathlineto{\pgfpoint{15.998503\du}{10.460424\du}}
\pgfpathlineto{\pgfpoint{15.999963\du}{10.452391\du}}
\pgfpathlineto{\pgfpoint{16.001059\du}{10.444724\du}}
\pgfpathlineto{\pgfpoint{16.002884\du}{10.436327\du}}
\pgfpathlineto{\pgfpoint{16.004710\du}{10.428295\du}}
\pgfpathlineto{\pgfpoint{16.007265\du}{10.420628\du}}
\pgfpathlineto{\pgfpoint{16.010186\du}{10.412961\du}}
\pgfpathlineto{\pgfpoint{16.013472\du}{10.405659\du}}
\pgfpathlineto{\pgfpoint{16.016758\du}{10.397262\du}}
\pgfpathlineto{\pgfpoint{16.020409\du}{10.389595\du}}
\pgfpathlineto{\pgfpoint{16.024790\du}{10.381928\du}}
\pgfpathlineto{\pgfpoint{16.029171\du}{10.374991\du}}
\pgfpathlineto{\pgfpoint{16.033917\du}{10.367324\du}}
\pgfpathlineto{\pgfpoint{16.039029\du}{10.360022\du}}
\pgfpathlineto{\pgfpoint{16.044140\du}{10.352720\du}}
\pgfpathlineto{\pgfpoint{16.050347\du}{10.345053\du}}
\pgfpathlineto{\pgfpoint{16.056553\du}{10.338116\du}}
\pgfpathlineto{\pgfpoint{16.062760\du}{10.331179\du}}
\pgfpathlineto{\pgfpoint{16.069697\du}{10.324242\du}}
\pgfpathlineto{\pgfpoint{16.076269\du}{10.317671\du}}
\pgfpathlineto{\pgfpoint{16.083936\du}{10.310734\du}}
\pgfpathlineto{\pgfpoint{16.091238\du}{10.303797\du}}
\pgfpathlineto{\pgfpoint{16.099270\du}{10.297225\du}}
\pgfpathlineto{\pgfpoint{16.107667\du}{10.290654\du}}
\pgfpathlineto{\pgfpoint{16.116064\du}{10.284447\du}}
\pgfpathlineto{\pgfpoint{16.125192\du}{10.278605\du}}
\pgfpathlineto{\pgfpoint{16.133589\du}{10.272034\du}}
\pgfpathlineto{\pgfpoint{16.143447\du}{10.266192\du}}
\pgfpathlineto{\pgfpoint{16.152939\du}{10.260350\du}}
\pgfpathlineto{\pgfpoint{16.162797\du}{10.254509\du}}
\pgfpathlineto{\pgfpoint{16.173019\du}{10.248667\du}}
\pgfpathlineto{\pgfpoint{16.183242\du}{10.243556\du}}
\pgfpathlineto{\pgfpoint{16.194195\du}{10.237714\du}}
\pgfpathlineto{\pgfpoint{16.204783\du}{10.232968\du}}
\pgfpathlineto{\pgfpoint{16.216101\du}{10.227492\du}}
\pgfpathlineto{\pgfpoint{16.227054\du}{10.222380\du}}
\pgfpathlineto{\pgfpoint{16.238737\du}{10.217634\du}}
\pgfpathlineto{\pgfpoint{16.250055\du}{10.212888\du}}
\pgfpathlineto{\pgfpoint{16.262468\du}{10.208142\du}}
\pgfpathlineto{\pgfpoint{16.274881\du}{10.204126\du}}
\pgfpathlineto{\pgfpoint{16.286930\du}{10.199379\du}}
\pgfpathlineto{\pgfpoint{16.299708\du}{10.195363\du}}
\pgfpathlineto{\pgfpoint{16.312851\du}{10.192077\du}}
\pgfpathlineto{\pgfpoint{16.325630\du}{10.187696\du}}
\pgfpathlineto{\pgfpoint{16.339138\du}{10.184410\du}}
\pgfpathlineto{\pgfpoint{16.352282\du}{10.180759\du}}
\pgfpathlineto{\pgfpoint{16.365790\du}{10.177839\du}}
\pgfpathlineto{\pgfpoint{16.379664\du}{10.174918\du}}
\pgfpathlineto{\pgfpoint{16.393903\du}{10.171997\du}}
\pgfpathlineto{\pgfpoint{16.407777\du}{10.169076\du}}
\pgfpathlineto{\pgfpoint{16.421650\du}{10.166886\du}}
\pgfpathlineto{\pgfpoint{16.435889\du}{10.164330\du}}
\pgfpathlineto{\pgfpoint{16.450493\du}{10.162139\du}}
\pgfpathlineto{\pgfpoint{16.465097\du}{10.160314\du}}
\pgfpathlineto{\pgfpoint{16.480066\du}{10.158488\du}}
\pgfpathlineto{\pgfpoint{16.495035\du}{10.156663\du}}
\pgfpathlineto{\pgfpoint{16.510004\du}{10.155568\du}}
\pgfpathlineto{\pgfpoint{16.524973\du}{10.154472\du}}
\pgfpathlineto{\pgfpoint{16.540672\du}{10.153377\du}}
\pgfpathlineto{\pgfpoint{16.555276\du}{10.152647\du}}
\pgfpathlineto{\pgfpoint{16.571340\du}{10.152282\du}}
\pgfpathlineto{\pgfpoint{16.586674\du}{10.151552\du}}
\pgfpathlineto{\pgfpoint{16.602373\du}{10.151552\du}}
\pgfpathlineto{\pgfpoint{16.602373\du}{10.151552\du}}
\pgfpathlineto{\pgfpoint{16.602373\du}{10.151552\du}}
\pgfpathlineto{\pgfpoint{16.603834\du}{10.151552\du}}
\pgfpathlineto{\pgfpoint{16.604929\du}{10.151552\du}}
\pgfpathlineto{\pgfpoint{16.606024\du}{10.151552\du}}
\pgfpathlineto{\pgfpoint{16.607119\du}{10.150821\du}}
\pgfpathlineto{\pgfpoint{16.608215\du}{10.150456\du}}
\pgfpathlineto{\pgfpoint{16.609310\du}{10.149726\du}}
\pgfpathlineto{\pgfpoint{16.609310\du}{10.149361\du}}
\pgfpathlineto{\pgfpoint{16.610040\du}{10.148631\du}}
\pgfpathlineto{\pgfpoint{16.611135\du}{10.146805\du}}
\pgfpathlineto{\pgfpoint{16.612231\du}{10.144980\du}}
\pgfpathlineto{\pgfpoint{16.612596\du}{10.143520\du}}
\pgfpathlineto{\pgfpoint{16.612596\du}{10.141694\du}}
\pgfpathlineto{\pgfpoint{16.612596\du}{10.139138\du}}
\pgfpathlineto{\pgfpoint{16.612231\du}{10.137678\du}}
\pgfpathlineto{\pgfpoint{16.611135\du}{10.135853\du}}
\pgfpathlineto{\pgfpoint{16.610040\du}{10.134757\du}}
\pgfpathlineto{\pgfpoint{16.609310\du}{10.133297\du}}
\pgfpathlineto{\pgfpoint{16.609310\du}{10.132932\du}}
\pgfpathlineto{\pgfpoint{16.608215\du}{10.132202\du}}
\pgfpathlineto{\pgfpoint{16.607119\du}{10.132202\du}}
\pgfpathlineto{\pgfpoint{16.606024\du}{10.131471\du}}
\pgfpathlineto{\pgfpoint{16.604929\du}{10.131471\du}}
\pgfpathlineto{\pgfpoint{16.603834\du}{10.131106\du}}
\pgfpathlineto{\pgfpoint{16.602373\du}{10.131106\du}}
\pgfusepath{fill}
\pgfsetbuttcap
\pgfsetmiterjoin
\pgfsetdash{}{0pt}
\definecolor{dialinecolor}{rgb}{0.678431, 0.839216, 0.905882}
\pgfsetfillcolor{dialinecolor}
\pgfpathmoveto{\pgfpoint{17.227419\du}{10.476853\du}}
\pgfpathlineto{\pgfpoint{17.227419\du}{10.467360\du}}
\pgfpathlineto{\pgfpoint{17.226689\du}{10.458598\du}}
\pgfpathlineto{\pgfpoint{17.225593\du}{10.449471\du}}
\pgfpathlineto{\pgfpoint{17.224133\du}{10.440708\du}}
\pgfpathlineto{\pgfpoint{17.221942\du}{10.431946\du}}
\pgfpathlineto{\pgfpoint{17.220117\du}{10.423184\du}}
\pgfpathlineto{\pgfpoint{17.217196\du}{10.414421\du}}
\pgfpathlineto{\pgfpoint{17.214640\du}{10.405659\du}}
\pgfpathlineto{\pgfpoint{17.210624\du}{10.397262\du}}
\pgfpathlineto{\pgfpoint{17.207338\du}{10.388499\du}}
\pgfpathlineto{\pgfpoint{17.202957\du}{10.380467\du}}
\pgfpathlineto{\pgfpoint{17.198211\du}{10.372070\du}}
\pgfpathlineto{\pgfpoint{17.193465\du}{10.364038\du}}
\pgfpathlineto{\pgfpoint{17.188719\du}{10.355641\du}}
\pgfpathlineto{\pgfpoint{17.182877\du}{10.347974\du}}
\pgfpathlineto{\pgfpoint{17.176670\du}{10.339942\du}}
\pgfpathlineto{\pgfpoint{17.170464\du}{10.332275\du}}
\pgfpathlineto{\pgfpoint{17.163892\du}{10.324608\du}}
\pgfpathlineto{\pgfpoint{17.157320\du}{10.317306\du}}
\pgfpathlineto{\pgfpoint{17.149653\du}{10.309639\du}}
\pgfpathlineto{\pgfpoint{17.142716\du}{10.302337\du}}
\pgfpathlineto{\pgfpoint{17.134684\du}{10.295400\du}}
\pgfpathlineto{\pgfpoint{17.126652\du}{10.288463\du}}
\pgfpathlineto{\pgfpoint{17.118255\du}{10.281526\du}}
\pgfpathlineto{\pgfpoint{17.109858\du}{10.274589\du}}
\pgfpathlineto{\pgfpoint{17.101095\du}{10.268018\du}}
\pgfpathlineto{\pgfpoint{17.091968\du}{10.261446\du}}
\pgfpathlineto{\pgfpoint{17.082475\du}{10.255239\du}}
\pgfpathlineto{\pgfpoint{17.072253\du}{10.248667\du}}
\pgfpathlineto{\pgfpoint{17.062395\du}{10.242826\du}}
\pgfpathlineto{\pgfpoint{17.052172\du}{10.236254\du}}
\pgfpathlineto{\pgfpoint{17.042315\du}{10.230413\du}}
\pgfpathlineto{\pgfpoint{17.030997\du}{10.225301\du}}
\pgfpathlineto{\pgfpoint{17.020044\du}{10.219460\du}}
\pgfpathlineto{\pgfpoint{17.008726\du}{10.213983\du}}
\pgfpathlineto{\pgfpoint{16.997408\du}{10.208872\du}}
\pgfpathlineto{\pgfpoint{16.985725\du}{10.203760\du}}
\pgfpathlineto{\pgfpoint{16.974407\du}{10.199014\du}}
\pgfpathlineto{\pgfpoint{16.961628\du}{10.193538\du}}
\pgfpathlineto{\pgfpoint{16.949945\du}{10.189157\du}}
\pgfpathlineto{\pgfpoint{16.937167\du}{10.184775\du}}
\pgfpathlineto{\pgfpoint{16.924388\du}{10.180394\du}}
\pgfpathlineto{\pgfpoint{16.911245\du}{10.176013\du}}
\pgfpathlineto{\pgfpoint{16.898101\du}{10.171997\du}}
\pgfpathlineto{\pgfpoint{16.884958\du}{10.168346\du}}
\pgfpathlineto{\pgfpoint{16.871449\du}{10.164330\du}}
\pgfpathlineto{\pgfpoint{16.857576\du}{10.161044\du}}
\pgfpathlineto{\pgfpoint{16.844067\du}{10.157393\du}}
\pgfpathlineto{\pgfpoint{16.830194\du}{10.154472\du}}
\pgfpathlineto{\pgfpoint{16.815590\du}{10.151552\du}}
\pgfpathlineto{\pgfpoint{16.801716\du}{10.149361\du}}
\pgfpathlineto{\pgfpoint{16.787477\du}{10.146440\du}}
\pgfpathlineto{\pgfpoint{16.772143\du}{10.143885\du}}
\pgfpathlineto{\pgfpoint{16.757174\du}{10.141694\du}}
\pgfpathlineto{\pgfpoint{16.742205\du}{10.139869\du}}
\pgfpathlineto{\pgfpoint{16.727966\du}{10.138043\du}}
\pgfpathlineto{\pgfpoint{16.711902\du}{10.136218\du}}
\pgfpathlineto{\pgfpoint{16.696933\du}{10.135122\du}}
\pgfpathlineto{\pgfpoint{16.681599\du}{10.134027\du}}
\pgfpathlineto{\pgfpoint{16.665900\du}{10.132932\du}}
\pgfpathlineto{\pgfpoint{16.650566\du}{10.132202\du}}
\pgfpathlineto{\pgfpoint{16.634502\du}{10.131471\du}}
\pgfpathlineto{\pgfpoint{16.618437\du}{10.131106\du}}
\pgfpathlineto{\pgfpoint{16.602373\du}{10.131106\du}}
\pgfpathlineto{\pgfpoint{16.602373\du}{10.151552\du}}
\pgfpathlineto{\pgfpoint{16.618437\du}{10.151552\du}}
\pgfpathlineto{\pgfpoint{16.633771\du}{10.152282\du}}
\pgfpathlineto{\pgfpoint{16.649471\du}{10.152647\du}}
\pgfpathlineto{\pgfpoint{16.664805\du}{10.153377\du}}
\pgfpathlineto{\pgfpoint{16.679774\du}{10.154472\du}}
\pgfpathlineto{\pgfpoint{16.695473\du}{10.155568\du}}
\pgfpathlineto{\pgfpoint{16.710077\du}{10.156663\du}}
\pgfpathlineto{\pgfpoint{16.725411\du}{10.158488\du}}
\pgfpathlineto{\pgfpoint{16.740380\du}{10.160314\du}}
\pgfpathlineto{\pgfpoint{16.754984\du}{10.162139\du}}
\pgfpathlineto{\pgfpoint{16.769222\du}{10.164330\du}}
\pgfpathlineto{\pgfpoint{16.783461\du}{10.166886\du}}
\pgfpathlineto{\pgfpoint{16.797700\du}{10.169076\du}}
\pgfpathlineto{\pgfpoint{16.811574\du}{10.171997\du}}
\pgfpathlineto{\pgfpoint{16.825447\du}{10.174918\du}}
\pgfpathlineto{\pgfpoint{16.839321\du}{10.177839\du}}
\pgfpathlineto{\pgfpoint{16.852829\du}{10.180759\du}}
\pgfpathlineto{\pgfpoint{16.866338\du}{10.184410\du}}
\pgfpathlineto{\pgfpoint{16.879482\du}{10.187696\du}}
\pgfpathlineto{\pgfpoint{16.892260\du}{10.192077\du}}
\pgfpathlineto{\pgfpoint{16.905769\du}{10.195363\du}}
\pgfpathlineto{\pgfpoint{16.918547\du}{10.199379\du}}
\pgfpathlineto{\pgfpoint{16.930230\du}{10.204126\du}}
\pgfpathlineto{\pgfpoint{16.942643\du}{10.208142\du}}
\pgfpathlineto{\pgfpoint{16.954691\du}{10.212888\du}}
\pgfpathlineto{\pgfpoint{16.966375\du}{10.217634\du}}
\pgfpathlineto{\pgfpoint{16.977693\du}{10.222380\du}}
\pgfpathlineto{\pgfpoint{16.989011\du}{10.227492\du}}
\pgfpathlineto{\pgfpoint{17.000694\du}{10.232968\du}}
\pgfpathlineto{\pgfpoint{17.011281\du}{10.237714\du}}
\pgfpathlineto{\pgfpoint{17.021869\du}{10.243556\du}}
\pgfpathlineto{\pgfpoint{17.032457\du}{10.248667\du}}
\pgfpathlineto{\pgfpoint{17.042315\du}{10.254509\du}}
\pgfpathlineto{\pgfpoint{17.052172\du}{10.260350\du}}
\pgfpathlineto{\pgfpoint{17.061300\du}{10.266192\du}}
\pgfpathlineto{\pgfpoint{17.070427\du}{10.272034\du}}
\pgfpathlineto{\pgfpoint{17.079920\du}{10.278605\du}}
\pgfpathlineto{\pgfpoint{17.088682\du}{10.284447\du}}
\pgfpathlineto{\pgfpoint{17.097809\du}{10.290654\du}}
\pgfpathlineto{\pgfpoint{17.105842\du}{10.297225\du}}
\pgfpathlineto{\pgfpoint{17.114239\du}{10.303797\du}}
\pgfpathlineto{\pgfpoint{17.121176\du}{10.310734\du}}
\pgfpathlineto{\pgfpoint{17.128478\du}{10.317671\du}}
\pgfpathlineto{\pgfpoint{17.135414\du}{10.324242\du}}
\pgfpathlineto{\pgfpoint{17.142716\du}{10.331179\du}}
\pgfpathlineto{\pgfpoint{17.148558\du}{10.338116\du}}
\pgfpathlineto{\pgfpoint{17.155495\du}{10.345053\du}}
\pgfpathlineto{\pgfpoint{17.160606\du}{10.352720\du}}
\pgfpathlineto{\pgfpoint{17.166083\du}{10.360022\du}}
\pgfpathlineto{\pgfpoint{17.170829\du}{10.367324\du}}
\pgfpathlineto{\pgfpoint{17.175940\du}{10.374991\du}}
\pgfpathlineto{\pgfpoint{17.179956\du}{10.381928\du}}
\pgfpathlineto{\pgfpoint{17.184702\du}{10.389595\du}}
\pgfpathlineto{\pgfpoint{17.188719\du}{10.397262\du}}
\pgfpathlineto{\pgfpoint{17.192004\du}{10.405659\du}}
\pgfpathlineto{\pgfpoint{17.194560\du}{10.412961\du}}
\pgfpathlineto{\pgfpoint{17.197846\du}{10.420628\du}}
\pgfpathlineto{\pgfpoint{17.200037\du}{10.428295\du}}
\pgfpathlineto{\pgfpoint{17.202592\du}{10.436327\du}}
\pgfpathlineto{\pgfpoint{17.204053\du}{10.444724\du}}
\pgfpathlineto{\pgfpoint{17.205513\du}{10.452391\du}}
\pgfpathlineto{\pgfpoint{17.206243\du}{10.460424\du}}
\pgfpathlineto{\pgfpoint{17.206608\du}{10.468821\du}}
\pgfpathlineto{\pgfpoint{17.207338\du}{10.476853\du}}
\pgfpathlineto{\pgfpoint{17.227419\du}{10.476853\du}}
\pgfusepath{fill}
\pgfsetbuttcap
\pgfsetmiterjoin
\pgfsetdash{}{0pt}
\definecolor{dialinecolor}{rgb}{0.074510, 0.082353, 0.086275}
\pgfsetfillcolor{dialinecolor}
\pgfpathmoveto{\pgfpoint{16.281453\du}{10.571048\du}}
\pgfpathlineto{\pgfpoint{16.508908\du}{10.342862\du}}
\pgfpathlineto{\pgfpoint{16.449032\du}{10.281891\du}}
\pgfpathlineto{\pgfpoint{16.629390\du}{10.281891\du}}
\pgfpathlineto{\pgfpoint{16.629390\du}{10.470281\du}}
\pgfpathlineto{\pgfpoint{16.569149\du}{10.410040\du}}
\pgfpathlineto{\pgfpoint{16.348996\du}{10.631289\du}}
\pgfpathlineto{\pgfpoint{16.281453\du}{10.571048\du}}
\pgfusepath{fill}
\pgfsetbuttcap
\pgfsetmiterjoin
\pgfsetdash{}{0pt}
\definecolor{dialinecolor}{rgb}{0.074510, 0.082353, 0.086275}
\pgfsetfillcolor{dialinecolor}
\pgfpathmoveto{\pgfpoint{16.549434\du}{10.691530\du}}
\pgfpathlineto{\pgfpoint{16.776524\du}{10.463344\du}}
\pgfpathlineto{\pgfpoint{16.715918\du}{10.403103\du}}
\pgfpathlineto{\pgfpoint{16.897006\du}{10.403103\du}}
\pgfpathlineto{\pgfpoint{16.897006\du}{10.591128\du}}
\pgfpathlineto{\pgfpoint{16.836400\du}{10.530887\du}}
\pgfpathlineto{\pgfpoint{16.615882\du}{10.751771\du}}
\pgfpathlineto{\pgfpoint{16.549434\du}{10.691530\du}}
\pgfusepath{fill}
\pgfsetbuttcap
\pgfsetmiterjoin
\pgfsetdash{}{0pt}
\definecolor{dialinecolor}{rgb}{1.000000, 1.000000, 1.000000}
\pgfsetfillcolor{dialinecolor}
\pgfpathmoveto{\pgfpoint{16.268310\du}{10.557539\du}}
\pgfpathlineto{\pgfpoint{16.495400\du}{10.329354\du}}
\pgfpathlineto{\pgfpoint{16.435889\du}{10.269113\du}}
\pgfpathlineto{\pgfpoint{16.615882\du}{10.269113\du}}
\pgfpathlineto{\pgfpoint{16.615882\du}{10.457138\du}}
\pgfpathlineto{\pgfpoint{16.556006\du}{10.396166\du}}
\pgfpathlineto{\pgfpoint{16.335487\du}{10.617780\du}}
\pgfpathlineto{\pgfpoint{16.268310\du}{10.557539\du}}
\pgfusepath{fill}
\pgfsetbuttcap
\pgfsetmiterjoin
\pgfsetdash{}{0pt}
\definecolor{dialinecolor}{rgb}{1.000000, 1.000000, 1.000000}
\pgfsetfillcolor{dialinecolor}
\pgfpathmoveto{\pgfpoint{16.535926\du}{10.678021\du}}
\pgfpathlineto{\pgfpoint{16.763016\du}{10.449836\du}}
\pgfpathlineto{\pgfpoint{16.702775\du}{10.389595\du}}
\pgfpathlineto{\pgfpoint{16.883133\du}{10.389595\du}}
\pgfpathlineto{\pgfpoint{16.883133\du}{10.577620\du}}
\pgfpathlineto{\pgfpoint{16.823622\du}{10.517379\du}}
\pgfpathlineto{\pgfpoint{16.602373\du}{10.738262\du}}
\pgfpathlineto{\pgfpoint{16.535926\du}{10.678021\du}}
\pgfusepath{fill}
\pgfsetlinewidth{0.000000\du}
\pgfsetdash{}{0pt}
\pgfsetdash{}{0pt}
\pgfsetbuttcap
\pgfsetmiterjoin
\pgfsetlinewidth{0.000000\du}
\pgfsetbuttcap
\pgfsetmiterjoin
\pgfsetdash{}{0pt}
\definecolor{dialinecolor}{rgb}{0.027451, 0.486275, 0.682353}
\pgfsetfillcolor{dialinecolor}
\pgfpathmoveto{\pgfpoint{27.879993\du}{10.562559\du}}
\pgfpathlineto{\pgfpoint{27.878532\du}{10.591767\du}}
\pgfpathlineto{\pgfpoint{27.871230\du}{10.621705\du}}
\pgfpathlineto{\pgfpoint{27.861008\du}{10.650183\du}}
\pgfpathlineto{\pgfpoint{27.846404\du}{10.678295\du}}
\pgfpathlineto{\pgfpoint{27.827054\du}{10.706407\du}}
\pgfpathlineto{\pgfpoint{27.805148\du}{10.733790\du}}
\pgfpathlineto{\pgfpoint{27.778496\du}{10.760807\du}}
\pgfpathlineto{\pgfpoint{27.747828\du}{10.787094\du}}
\pgfpathlineto{\pgfpoint{27.714969\du}{10.812286\du}}
\pgfpathlineto{\pgfpoint{27.677729\du}{10.837477\du}}
\pgfpathlineto{\pgfpoint{27.636838\du}{10.861574\du}}
\pgfpathlineto{\pgfpoint{27.593027\du}{10.884940\du}}
\pgfpathlineto{\pgfpoint{27.546659\du}{10.907576\du}}
\pgfpathlineto{\pgfpoint{27.496276\du}{10.929482\du}}
\pgfpathlineto{\pgfpoint{27.443337\du}{10.950292\du}}
\pgfpathlineto{\pgfpoint{27.387842\du}{10.970372\du}}
\pgfpathlineto{\pgfpoint{27.329792\du}{10.989723\du}}
\pgfpathlineto{\pgfpoint{27.269186\du}{11.007612\du}}
\pgfpathlineto{\pgfpoint{27.205294\du}{11.024772\du}}
\pgfpathlineto{\pgfpoint{27.140307\du}{11.041201\du}}
\pgfpathlineto{\pgfpoint{27.071668\du}{11.056170\du}}
\pgfpathlineto{\pgfpoint{27.000840\du}{11.069679\du}}
\pgfpathlineto{\pgfpoint{26.928916\du}{11.082457\du}}
\pgfpathlineto{\pgfpoint{26.854071\du}{11.094505\du}}
\pgfpathlineto{\pgfpoint{26.778131\du}{11.104363\du}}
\pgfpathlineto{\pgfpoint{26.699635\du}{11.113490\du}}
\pgfpathlineto{\pgfpoint{26.620044\du}{11.121157\du}}
\pgfpathlineto{\pgfpoint{26.538992\du}{11.127729\du}}
\pgfpathlineto{\pgfpoint{26.455750\du}{11.132840\du}}
\pgfpathlineto{\pgfpoint{26.372143\du}{11.136491\du}}
\pgfpathlineto{\pgfpoint{26.286710\du}{11.138682\du}}
\pgfpathlineto{\pgfpoint{26.200183\du}{11.139412\du}}
\pgfpathlineto{\pgfpoint{26.114020\du}{11.138682\du}}
\pgfpathlineto{\pgfpoint{26.028222\du}{11.136491\du}}
\pgfpathlineto{\pgfpoint{25.944615\du}{11.132840\du}}
\pgfpathlineto{\pgfpoint{25.861738\du}{11.127729\du}}
\pgfpathlineto{\pgfpoint{25.780321\du}{11.121157\du}}
\pgfpathlineto{\pgfpoint{25.700730\du}{11.113490\du}}
\pgfpathlineto{\pgfpoint{25.622965\du}{11.104363\du}}
\pgfpathlineto{\pgfpoint{25.546294\du}{11.094505\du}}
\pgfpathlineto{\pgfpoint{25.472180\du}{11.082457\du}}
\pgfpathlineto{\pgfpoint{25.399525\du}{11.069679\du}}
\pgfpathlineto{\pgfpoint{25.329062\du}{11.056170\du}}
\pgfpathlineto{\pgfpoint{25.260424\du}{11.041201\du}}
\pgfpathlineto{\pgfpoint{25.194706\du}{11.024772\du}}
\pgfpathlineto{\pgfpoint{25.131179\du}{11.007612\du}}
\pgfpathlineto{\pgfpoint{25.070208\du}{10.989723\du}}
\pgfpathlineto{\pgfpoint{25.011793\du}{10.970372\du}}
\pgfpathlineto{\pgfpoint{24.956663\du}{10.950292\du}}
\pgfpathlineto{\pgfpoint{24.903724\du}{10.929482\du}}
\pgfpathlineto{\pgfpoint{24.853706\du}{10.907576\du}}
\pgfpathlineto{\pgfpoint{24.806608\du}{10.884940\du}}
\pgfpathlineto{\pgfpoint{24.763162\du}{10.861574\du}}
\pgfpathlineto{\pgfpoint{24.722271\du}{10.837477\du}}
\pgfpathlineto{\pgfpoint{24.685031\du}{10.812286\du}}
\pgfpathlineto{\pgfpoint{24.651807\du}{10.787094\du}}
\pgfpathlineto{\pgfpoint{24.621504\du}{10.760807\du}}
\pgfpathlineto{\pgfpoint{24.594852\du}{10.733790\du}}
\pgfpathlineto{\pgfpoint{24.572946\du}{10.706407\du}}
\pgfpathlineto{\pgfpoint{24.553596\du}{10.678295\du}}
\pgfpathlineto{\pgfpoint{24.538992\du}{10.650183\du}}
\pgfpathlineto{\pgfpoint{24.528405\du}{10.621705\du}}
\pgfpathlineto{\pgfpoint{24.521468\du}{10.591767\du}}
\pgfpathlineto{\pgfpoint{24.519642\du}{10.562559\du}}
\pgfpathlineto{\pgfpoint{24.521468\du}{10.532621\du}}
\pgfpathlineto{\pgfpoint{24.528405\du}{10.503414\du}}
\pgfpathlineto{\pgfpoint{24.538992\du}{10.474206\du}}
\pgfpathlineto{\pgfpoint{24.553596\du}{10.446093\du}}
\pgfpathlineto{\pgfpoint{24.572946\du}{10.417981\du}}
\pgfpathlineto{\pgfpoint{24.594852\du}{10.390599\du}}
\pgfpathlineto{\pgfpoint{24.621504\du}{10.363947\du}}
\pgfpathlineto{\pgfpoint{24.651807\du}{10.337660\du}}
\pgfpathlineto{\pgfpoint{24.685031\du}{10.312103\du}}
\pgfpathlineto{\pgfpoint{24.722271\du}{10.287276\du}}
\pgfpathlineto{\pgfpoint{24.763162\du}{10.262815\du}}
\pgfpathlineto{\pgfpoint{24.806608\du}{10.239449\du}}
\pgfpathlineto{\pgfpoint{24.853706\du}{10.217178\du}}
\pgfpathlineto{\pgfpoint{24.903724\du}{10.194907\du}}
\pgfpathlineto{\pgfpoint{24.956663\du}{10.174096\du}}
\pgfpathlineto{\pgfpoint{25.011793\du}{10.154016\du}}
\pgfpathlineto{\pgfpoint{25.070208\du}{10.135396\du}}
\pgfpathlineto{\pgfpoint{25.131179\du}{10.116411\du}}
\pgfpathlineto{\pgfpoint{25.194706\du}{10.099617\du}}
\pgfpathlineto{\pgfpoint{25.260424\du}{10.083917\du}}
\pgfpathlineto{\pgfpoint{25.329062\du}{10.068583\du}}
\pgfpathlineto{\pgfpoint{25.399525\du}{10.054710\du}}
\pgfpathlineto{\pgfpoint{25.472180\du}{10.041566\du}}
\pgfpathlineto{\pgfpoint{25.546294\du}{10.030613\du}}
\pgfpathlineto{\pgfpoint{25.622965\du}{10.020026\du}}
\pgfpathlineto{\pgfpoint{25.700730\du}{10.010533\du}}
\pgfpathlineto{\pgfpoint{25.780321\du}{10.003231\du}}
\pgfpathlineto{\pgfpoint{25.861738\du}{9.996659\du}}
\pgfpathlineto{\pgfpoint{25.944615\du}{9.991548\du}}
\pgfpathlineto{\pgfpoint{26.028222\du}{9.987897\du}}
\pgfpathlineto{\pgfpoint{26.114020\du}{9.986072\du}}
\pgfpathlineto{\pgfpoint{26.200183\du}{9.984976\du}}
\pgfpathlineto{\pgfpoint{26.286710\du}{9.986072\du}}
\pgfpathlineto{\pgfpoint{26.372143\du}{9.987897\du}}
\pgfpathlineto{\pgfpoint{26.455750\du}{9.991548\du}}
\pgfpathlineto{\pgfpoint{26.538992\du}{9.996659\du}}
\pgfpathlineto{\pgfpoint{26.620044\du}{10.003231\du}}
\pgfpathlineto{\pgfpoint{26.699635\du}{10.010533\du}}
\pgfpathlineto{\pgfpoint{26.778131\du}{10.020026\du}}
\pgfpathlineto{\pgfpoint{26.854071\du}{10.030613\du}}
\pgfpathlineto{\pgfpoint{26.928916\du}{10.041566\du}}
\pgfpathlineto{\pgfpoint{27.000840\du}{10.054710\du}}
\pgfpathlineto{\pgfpoint{27.071668\du}{10.068583\du}}
\pgfpathlineto{\pgfpoint{27.140307\du}{10.083917\du}}
\pgfpathlineto{\pgfpoint{27.205294\du}{10.099617\du}}
\pgfpathlineto{\pgfpoint{27.269186\du}{10.116411\du}}
\pgfpathlineto{\pgfpoint{27.329792\du}{10.135396\du}}
\pgfpathlineto{\pgfpoint{27.387842\du}{10.154016\du}}
\pgfpathlineto{\pgfpoint{27.443337\du}{10.174096\du}}
\pgfpathlineto{\pgfpoint{27.496276\du}{10.194907\du}}
\pgfpathlineto{\pgfpoint{27.546659\du}{10.217178\du}}
\pgfpathlineto{\pgfpoint{27.593027\du}{10.239449\du}}
\pgfpathlineto{\pgfpoint{27.636838\du}{10.262815\du}}
\pgfpathlineto{\pgfpoint{27.677729\du}{10.287276\du}}
\pgfpathlineto{\pgfpoint{27.714969\du}{10.312103\du}}
\pgfpathlineto{\pgfpoint{27.747828\du}{10.337660\du}}
\pgfpathlineto{\pgfpoint{27.778496\du}{10.363947\du}}
\pgfpathlineto{\pgfpoint{27.805148\du}{10.390599\du}}
\pgfpathlineto{\pgfpoint{27.827054\du}{10.417981\du}}
\pgfpathlineto{\pgfpoint{27.846404\du}{10.446093\du}}
\pgfpathlineto{\pgfpoint{27.861008\du}{10.474206\du}}
\pgfpathlineto{\pgfpoint{27.871230\du}{10.503414\du}}
\pgfpathlineto{\pgfpoint{27.878532\du}{10.532621\du}}
\pgfpathlineto{\pgfpoint{27.879993\du}{10.562559\du}}
\pgfusepath{fill}
\pgfsetlinewidth{0.000000\du}
\pgfsetbuttcap
\pgfsetmiterjoin
\pgfsetdash{}{0pt}
\definecolor{dialinecolor}{rgb}{0.678431, 0.839216, 0.905882}
\pgfsetfillcolor{dialinecolor}
\pgfpathmoveto{\pgfpoint{26.200183\du}{11.150000\du}}
\pgfpathlineto{\pgfpoint{26.200183\du}{11.150000\du}}
\pgfpathlineto{\pgfpoint{26.243629\du}{11.150000\du}}
\pgfpathlineto{\pgfpoint{26.287076\du}{11.149270\du}}
\pgfpathlineto{\pgfpoint{26.330157\du}{11.148175\du}}
\pgfpathlineto{\pgfpoint{26.372143\du}{11.147079\du}}
\pgfpathlineto{\pgfpoint{26.414859\du}{11.145254\du}}
\pgfpathlineto{\pgfpoint{26.456480\du}{11.143063\du}}
\pgfpathlineto{\pgfpoint{26.498101\du}{11.140507\du}}
\pgfpathlineto{\pgfpoint{26.539723\du}{11.138317\du}}
\pgfpathlineto{\pgfpoint{26.580248\du}{11.135396\du}}
\pgfpathlineto{\pgfpoint{26.621139\du}{11.131745\du}}
\pgfpathlineto{\pgfpoint{26.660935\du}{11.127729\du}}
\pgfpathlineto{\pgfpoint{26.701095\du}{11.123713\du}}
\pgfpathlineto{\pgfpoint{26.739796\du}{11.119332\du}}
\pgfpathlineto{\pgfpoint{26.779226\du}{11.114951\du}}
\pgfpathlineto{\pgfpoint{26.817196\du}{11.109474\du}}
\pgfpathlineto{\pgfpoint{26.856261\du}{11.104363\du}}
\pgfpathlineto{\pgfpoint{26.893501\du}{11.098886\du}}
\pgfpathlineto{\pgfpoint{26.930376\du}{11.092680\du}}
\pgfpathlineto{\pgfpoint{26.966886\du}{11.086838\du}}
\pgfpathlineto{\pgfpoint{27.003395\du}{11.080267\du}}
\pgfpathlineto{\pgfpoint{27.038810\du}{11.073330\du}}
\pgfpathlineto{\pgfpoint{27.073494\du}{11.066393\du}}
\pgfpathlineto{\pgfpoint{27.108178\du}{11.058726\du}}
\pgfpathlineto{\pgfpoint{27.142132\du}{11.051059\du}}
\pgfpathlineto{\pgfpoint{27.175721\du}{11.042662\du}}
\pgfpathlineto{\pgfpoint{27.208580\du}{11.034629\du}}
\pgfpathlineto{\pgfpoint{27.240343\du}{11.026597\du}}
\pgfpathlineto{\pgfpoint{27.271742\du}{11.017835\du}}
\pgfpathlineto{\pgfpoint{27.287076\du}{11.013089\du}}
\pgfpathlineto{\pgfpoint{27.302410\du}{11.009073\du}}
\pgfpathlineto{\pgfpoint{27.318474\du}{11.004326\du}}
\pgfpathlineto{\pgfpoint{27.333078\du}{10.999580\du}}
\pgfpathlineto{\pgfpoint{27.347317\du}{10.994834\du}}
\pgfpathlineto{\pgfpoint{27.362286\du}{10.989723\du}}
\pgfpathlineto{\pgfpoint{27.377254\du}{10.984976\du}}
\pgfpathlineto{\pgfpoint{27.391128\du}{10.980230\du}}
\pgfpathlineto{\pgfpoint{27.405367\du}{10.975119\du}}
\pgfpathlineto{\pgfpoint{27.419241\du}{10.970372\du}}
\pgfpathlineto{\pgfpoint{27.433479\du}{10.964896\du}}
\pgfpathlineto{\pgfpoint{27.446623\du}{10.960515\du}}
\pgfpathlineto{\pgfpoint{27.460862\du}{10.955038\du}}
\pgfpathlineto{\pgfpoint{27.474005\du}{10.949927\du}}
\pgfpathlineto{\pgfpoint{27.487149\du}{10.944451\du}}
\pgfpathlineto{\pgfpoint{27.500657\du}{10.938609\du}}
\pgfpathlineto{\pgfpoint{27.513436\du}{10.933498\du}}
\pgfpathlineto{\pgfpoint{27.525484\du}{10.928021\du}}
\pgfpathlineto{\pgfpoint{27.538262\du}{10.922180\du}}
\pgfpathlineto{\pgfpoint{27.550675\du}{10.917068\du}}
\pgfpathlineto{\pgfpoint{27.563089\du}{10.911227\du}}
\pgfpathlineto{\pgfpoint{27.574407\du}{10.905750\du}}
\pgfpathlineto{\pgfpoint{27.586455\du}{10.899909\du}}
\pgfpathlineto{\pgfpoint{27.597408\du}{10.894067\du}}
\pgfpathlineto{\pgfpoint{27.609091\du}{10.888226\du}}
\pgfpathlineto{\pgfpoint{27.620409\du}{10.882384\du}}
\pgfpathlineto{\pgfpoint{27.631727\du}{10.876543\du}}
\pgfpathlineto{\pgfpoint{27.642315\du}{10.870336\du}}
\pgfpathlineto{\pgfpoint{27.652172\du}{10.864494\du}}
\pgfpathlineto{\pgfpoint{27.662760\du}{10.858653\du}}
\pgfpathlineto{\pgfpoint{27.672983\du}{10.852081\du}}
\pgfpathlineto{\pgfpoint{27.682840\du}{10.846240\du}}
\pgfpathlineto{\pgfpoint{27.692698\du}{10.839668\du}}
\pgfpathlineto{\pgfpoint{27.701825\du}{10.833461\du}}
\pgfpathlineto{\pgfpoint{27.710953\du}{10.826889\du}}
\pgfpathlineto{\pgfpoint{27.720445\du}{10.821048\du}}
\pgfpathlineto{\pgfpoint{27.729208\du}{10.814476\du}}
\pgfpathlineto{\pgfpoint{27.738335\du}{10.808269\du}}
\pgfpathlineto{\pgfpoint{27.746367\du}{10.801698\du}}
\pgfpathlineto{\pgfpoint{27.755130\du}{10.794761\du}}
\pgfpathlineto{\pgfpoint{27.762432\du}{10.788189\du}}
\pgfpathlineto{\pgfpoint{27.770464\du}{10.781982\du}}
\pgfpathlineto{\pgfpoint{27.778496\du}{10.774681\du}}
\pgfpathlineto{\pgfpoint{27.785068\du}{10.768474\du}}
\pgfpathlineto{\pgfpoint{27.792369\du}{10.761537\du}}
\pgfpathlineto{\pgfpoint{27.798941\du}{10.754235\du}}
\pgfpathlineto{\pgfpoint{27.805878\du}{10.748028\du}}
\pgfpathlineto{\pgfpoint{27.812450\du}{10.741092\du}}
\pgfpathlineto{\pgfpoint{27.818656\du}{10.733790\du}}
\pgfpathlineto{\pgfpoint{27.824498\du}{10.726853\du}}
\pgfpathlineto{\pgfpoint{27.829974\du}{10.719916\du}}
\pgfpathlineto{\pgfpoint{27.835816\du}{10.712979\du}}
\pgfpathlineto{\pgfpoint{27.840562\du}{10.705677\du}}
\pgfpathlineto{\pgfpoint{27.846039\du}{10.698375\du}}
\pgfpathlineto{\pgfpoint{27.850785\du}{10.691073\du}}
\pgfpathlineto{\pgfpoint{27.855166\du}{10.684137\du}}
\pgfpathlineto{\pgfpoint{27.859182\du}{10.676470\du}}
\pgfpathlineto{\pgfpoint{27.863198\du}{10.669533\du}}
\pgfpathlineto{\pgfpoint{27.866484\du}{10.661866\du}}
\pgfpathlineto{\pgfpoint{27.870500\du}{10.654199\du}}
\pgfpathlineto{\pgfpoint{27.873786\du}{10.646532\du}}
\pgfpathlineto{\pgfpoint{27.876342\du}{10.639230\du}}
\pgfpathlineto{\pgfpoint{27.879263\du}{10.631928\du}}
\pgfpathlineto{\pgfpoint{27.881088\du}{10.624626\du}}
\pgfpathlineto{\pgfpoint{27.884009\du}{10.616959\du}}
\pgfpathlineto{\pgfpoint{27.885104\du}{10.608562\du}}
\pgfpathlineto{\pgfpoint{27.887295\du}{10.600894\du}}
\pgfpathlineto{\pgfpoint{27.888390\du}{10.593593\du}}
\pgfpathlineto{\pgfpoint{27.889120\du}{10.585926\du}}
\pgfpathlineto{\pgfpoint{27.889850\du}{10.577528\du}}
\pgfpathlineto{\pgfpoint{27.890581\du}{10.570226\du}}
\pgfpathlineto{\pgfpoint{27.890581\du}{10.562559\du}}
\pgfpathlineto{\pgfpoint{27.870500\du}{10.562559\du}}
\pgfpathlineto{\pgfpoint{27.869770\du}{10.569496\du}}
\pgfpathlineto{\pgfpoint{27.869770\du}{10.576433\du}}
\pgfpathlineto{\pgfpoint{27.869405\du}{10.583370\du}}
\pgfpathlineto{\pgfpoint{27.867579\du}{10.590672\du}}
\pgfpathlineto{\pgfpoint{27.866484\du}{10.597609\du}}
\pgfpathlineto{\pgfpoint{27.865754\du}{10.604545\du}}
\pgfpathlineto{\pgfpoint{27.863563\du}{10.611482\du}}
\pgfpathlineto{\pgfpoint{27.862103\du}{10.618784\du}}
\pgfpathlineto{\pgfpoint{27.859912\du}{10.624991\du}}
\pgfpathlineto{\pgfpoint{27.857357\du}{10.631928\du}}
\pgfpathlineto{\pgfpoint{27.854436\du}{10.639230\du}}
\pgfpathlineto{\pgfpoint{27.851880\du}{10.646166\du}}
\pgfpathlineto{\pgfpoint{27.847864\du}{10.653103\du}}
\pgfpathlineto{\pgfpoint{27.844943\du}{10.659675\du}}
\pgfpathlineto{\pgfpoint{27.840927\du}{10.666612\du}}
\pgfpathlineto{\pgfpoint{27.838007\du}{10.673184\du}}
\pgfpathlineto{\pgfpoint{27.833625\du}{10.680120\du}}
\pgfpathlineto{\pgfpoint{27.829244\du}{10.687057\du}}
\pgfpathlineto{\pgfpoint{27.824498\du}{10.693629\du}}
\pgfpathlineto{\pgfpoint{27.819387\du}{10.699836\du}}
\pgfpathlineto{\pgfpoint{27.814640\du}{10.707138\du}}
\pgfpathlineto{\pgfpoint{27.808434\du}{10.714074\du}}
\pgfpathlineto{\pgfpoint{27.802957\du}{10.720281\du}}
\pgfpathlineto{\pgfpoint{27.797846\du}{10.726853\du}}
\pgfpathlineto{\pgfpoint{27.791274\du}{10.733425\du}}
\pgfpathlineto{\pgfpoint{27.784337\du}{10.740361\du}}
\pgfpathlineto{\pgfpoint{27.778496\du}{10.746933\du}}
\pgfpathlineto{\pgfpoint{27.771194\du}{10.753140\du}}
\pgfpathlineto{\pgfpoint{27.764622\du}{10.759712\du}}
\pgfpathlineto{\pgfpoint{27.756955\du}{10.765918\du}}
\pgfpathlineto{\pgfpoint{27.749653\du}{10.772490\du}}
\pgfpathlineto{\pgfpoint{27.741621\du}{10.779062\du}}
\pgfpathlineto{\pgfpoint{27.733954\du}{10.785268\du}}
\pgfpathlineto{\pgfpoint{27.725192\du}{10.791840\du}}
\pgfpathlineto{\pgfpoint{27.716794\du}{10.797682\du}}
\pgfpathlineto{\pgfpoint{27.708032\du}{10.804253\du}}
\pgfpathlineto{\pgfpoint{27.700000\du}{10.810460\du}}
\pgfpathlineto{\pgfpoint{27.691238\du}{10.816302\du}}
\pgfpathlineto{\pgfpoint{27.681745\du}{10.822873\du}}
\pgfpathlineto{\pgfpoint{27.671522\du}{10.828715\du}}
\pgfpathlineto{\pgfpoint{27.662395\du}{10.835287\du}}
\pgfpathlineto{\pgfpoint{27.652172\du}{10.841128\du}}
\pgfpathlineto{\pgfpoint{27.642315\du}{10.846970\du}}
\pgfpathlineto{\pgfpoint{27.632457\du}{10.852811\du}}
\pgfpathlineto{\pgfpoint{27.621504\du}{10.858653\du}}
\pgfpathlineto{\pgfpoint{27.610551\du}{10.864494\du}}
\pgfpathlineto{\pgfpoint{27.600329\du}{10.870336\du}}
\pgfpathlineto{\pgfpoint{27.588280\du}{10.876177\du}}
\pgfpathlineto{\pgfpoint{27.577693\du}{10.881289\du}}
\pgfpathlineto{\pgfpoint{27.565644\du}{10.887130\du}}
\pgfpathlineto{\pgfpoint{27.554326\du}{10.892972\du}}
\pgfpathlineto{\pgfpoint{27.541913\du}{10.898448\du}}
\pgfpathlineto{\pgfpoint{27.529865\du}{10.903560\du}}
\pgfpathlineto{\pgfpoint{27.517452\du}{10.909401\du}}
\pgfpathlineto{\pgfpoint{27.505403\du}{10.914878\du}}
\pgfpathlineto{\pgfpoint{27.492260\du}{10.919989\du}}
\pgfpathlineto{\pgfpoint{27.479482\du}{10.925100\du}}
\pgfpathlineto{\pgfpoint{27.466703\du}{10.930577\du}}
\pgfpathlineto{\pgfpoint{27.453560\du}{10.935688\du}}
\pgfpathlineto{\pgfpoint{27.440051\du}{10.941165\du}}
\pgfpathlineto{\pgfpoint{27.426908\du}{10.945546\du}}
\pgfpathlineto{\pgfpoint{27.412669\du}{10.951022\du}}
\pgfpathlineto{\pgfpoint{27.398795\du}{10.955769\du}}
\pgfpathlineto{\pgfpoint{27.384556\du}{10.960880\du}}
\pgfpathlineto{\pgfpoint{27.369953\du}{10.965626\du}}
\pgfpathlineto{\pgfpoint{27.356079\du}{10.970372\du}}
\pgfpathlineto{\pgfpoint{27.341475\du}{10.975119\du}}
\pgfpathlineto{\pgfpoint{27.327236\du}{10.979500\du}}
\pgfpathlineto{\pgfpoint{27.311537\du}{10.984246\du}}
\pgfpathlineto{\pgfpoint{27.297298\du}{10.988992\du}}
\pgfpathlineto{\pgfpoint{27.281599\du}{10.993739\du}}
\pgfpathlineto{\pgfpoint{27.266630\du}{10.997755\du}}
\pgfpathlineto{\pgfpoint{27.235232\du}{11.006517\du}}
\pgfpathlineto{\pgfpoint{27.203103\du}{11.014914\du}}
\pgfpathlineto{\pgfpoint{27.170975\du}{11.022946\du}}
\pgfpathlineto{\pgfpoint{27.137751\du}{11.030978\du}}
\pgfpathlineto{\pgfpoint{27.103432\du}{11.038645\du}}
\pgfpathlineto{\pgfpoint{27.069478\du}{11.045947\du}}
\pgfpathlineto{\pgfpoint{27.034794\du}{11.052884\du}}
\pgfpathlineto{\pgfpoint{26.999014\du}{11.059821\du}}
\pgfpathlineto{\pgfpoint{26.963600\du}{11.066393\du}}
\pgfpathlineto{\pgfpoint{26.926725\du}{11.072599\du}}
\pgfpathlineto{\pgfpoint{26.890215\du}{11.078441\du}}
\pgfpathlineto{\pgfpoint{26.852976\du}{11.084283\du}}
\pgfpathlineto{\pgfpoint{26.815371\du}{11.089759\du}}
\pgfpathlineto{\pgfpoint{26.776670\du}{11.094505\du}}
\pgfpathlineto{\pgfpoint{26.737970\du}{11.098886\du}}
\pgfpathlineto{\pgfpoint{26.698540\du}{11.103633\du}}
\pgfpathlineto{\pgfpoint{26.659474\du}{11.107284\du}}
\pgfpathlineto{\pgfpoint{26.619314\du}{11.111300\du}}
\pgfpathlineto{\pgfpoint{26.579153\du}{11.114221\du}}
\pgfpathlineto{\pgfpoint{26.537897\du}{11.117871\du}}
\pgfpathlineto{\pgfpoint{26.497006\du}{11.120792\du}}
\pgfpathlineto{\pgfpoint{26.455750\du}{11.122983\du}}
\pgfpathlineto{\pgfpoint{26.414129\du}{11.124808\du}}
\pgfpathlineto{\pgfpoint{26.371413\du}{11.126634\du}}
\pgfpathlineto{\pgfpoint{26.328697\du}{11.127729\du}}
\pgfpathlineto{\pgfpoint{26.286710\du}{11.128824\du}}
\pgfpathlineto{\pgfpoint{26.243264\du}{11.128824\du}}
\pgfpathlineto{\pgfpoint{26.200183\du}{11.129555\du}}
\pgfpathlineto{\pgfpoint{26.200183\du}{11.129555\du}}
\pgfpathlineto{\pgfpoint{26.200183\du}{11.129555\du}}
\pgfpathlineto{\pgfpoint{26.199452\du}{11.129555\du}}
\pgfpathlineto{\pgfpoint{26.197627\du}{11.129555\du}}
\pgfpathlineto{\pgfpoint{26.196532\du}{11.129920\du}}
\pgfpathlineto{\pgfpoint{26.195801\du}{11.129920\du}}
\pgfpathlineto{\pgfpoint{26.195436\du}{11.130650\du}}
\pgfpathlineto{\pgfpoint{26.193976\du}{11.131015\du}}
\pgfpathlineto{\pgfpoint{26.193246\du}{11.131745\du}}
\pgfpathlineto{\pgfpoint{26.192516\du}{11.132475\du}}
\pgfpathlineto{\pgfpoint{26.191420\du}{11.134301\du}}
\pgfpathlineto{\pgfpoint{26.190690\du}{11.135761\du}}
\pgfpathlineto{\pgfpoint{26.190690\du}{11.137587\du}}
\pgfpathlineto{\pgfpoint{26.189960\du}{11.139412\du}}
\pgfpathlineto{\pgfpoint{26.190690\du}{11.141603\du}}
\pgfpathlineto{\pgfpoint{26.190690\du}{11.143428\du}}
\pgfpathlineto{\pgfpoint{26.191420\du}{11.145254\du}}
\pgfpathlineto{\pgfpoint{26.192516\du}{11.147079\du}}
\pgfpathlineto{\pgfpoint{26.193246\du}{11.147444\du}}
\pgfpathlineto{\pgfpoint{26.193976\du}{11.148175\du}}
\pgfpathlineto{\pgfpoint{26.195436\du}{11.148905\du}}
\pgfpathlineto{\pgfpoint{26.195801\du}{11.149270\du}}
\pgfpathlineto{\pgfpoint{26.196532\du}{11.149270\du}}
\pgfpathlineto{\pgfpoint{26.197627\du}{11.150000\du}}
\pgfpathlineto{\pgfpoint{26.199452\du}{11.150000\du}}
\pgfpathlineto{\pgfpoint{26.200183\du}{11.150000\du}}
\pgfusepath{fill}
\pgfsetbuttcap
\pgfsetmiterjoin
\pgfsetdash{}{0pt}
\definecolor{dialinecolor}{rgb}{0.678431, 0.839216, 0.905882}
\pgfsetfillcolor{dialinecolor}
\pgfpathmoveto{\pgfpoint{24.509419\du}{10.562559\du}}
\pgfpathlineto{\pgfpoint{24.509419\du}{10.562559\du}}
\pgfpathlineto{\pgfpoint{24.509419\du}{10.570226\du}}
\pgfpathlineto{\pgfpoint{24.509785\du}{10.577528\du}}
\pgfpathlineto{\pgfpoint{24.510515\du}{10.585926\du}}
\pgfpathlineto{\pgfpoint{24.511610\du}{10.593593\du}}
\pgfpathlineto{\pgfpoint{24.512705\du}{10.600894\du}}
\pgfpathlineto{\pgfpoint{24.514531\du}{10.608562\du}}
\pgfpathlineto{\pgfpoint{24.516356\du}{10.616959\du}}
\pgfpathlineto{\pgfpoint{24.518547\du}{10.624626\du}}
\pgfpathlineto{\pgfpoint{24.520737\du}{10.631928\du}}
\pgfpathlineto{\pgfpoint{24.523293\du}{10.639230\du}}
\pgfpathlineto{\pgfpoint{24.526214\du}{10.646532\du}}
\pgfpathlineto{\pgfpoint{24.529865\du}{10.654199\du}}
\pgfpathlineto{\pgfpoint{24.533151\du}{10.661866\du}}
\pgfpathlineto{\pgfpoint{24.536802\du}{10.669533\du}}
\pgfpathlineto{\pgfpoint{24.541183\du}{10.676470\du}}
\pgfpathlineto{\pgfpoint{24.544834\du}{10.684137\du}}
\pgfpathlineto{\pgfpoint{24.549945\du}{10.691073\du}}
\pgfpathlineto{\pgfpoint{24.553961\du}{10.698375\du}}
\pgfpathlineto{\pgfpoint{24.559438\du}{10.705677\du}}
\pgfpathlineto{\pgfpoint{24.564184\du}{10.712979\du}}
\pgfpathlineto{\pgfpoint{24.569660\du}{10.719916\du}}
\pgfpathlineto{\pgfpoint{24.575502\du}{10.726853\du}}
\pgfpathlineto{\pgfpoint{24.581344\du}{10.733790\du}}
\pgfpathlineto{\pgfpoint{24.587185\du}{10.741092\du}}
\pgfpathlineto{\pgfpoint{24.594122\du}{10.748028\du}}
\pgfpathlineto{\pgfpoint{24.600694\du}{10.754235\du}}
\pgfpathlineto{\pgfpoint{24.607631\du}{10.761537\du}}
\pgfpathlineto{\pgfpoint{24.614567\du}{10.768474\du}}
\pgfpathlineto{\pgfpoint{24.621504\du}{10.774681\du}}
\pgfpathlineto{\pgfpoint{24.630267\du}{10.781982\du}}
\pgfpathlineto{\pgfpoint{24.637203\du}{10.788189\du}}
\pgfpathlineto{\pgfpoint{24.644870\du}{10.794761\du}}
\pgfpathlineto{\pgfpoint{24.653633\du}{10.801698\du}}
\pgfpathlineto{\pgfpoint{24.662030\du}{10.808269\du}}
\pgfpathlineto{\pgfpoint{24.670792\du}{10.814476\du}}
\pgfpathlineto{\pgfpoint{24.679189\du}{10.821048\du}}
\pgfpathlineto{\pgfpoint{24.689047\du}{10.826889\du}}
\pgfpathlineto{\pgfpoint{24.698175\du}{10.833461\du}}
\pgfpathlineto{\pgfpoint{24.707667\du}{10.839668\du}}
\pgfpathlineto{\pgfpoint{24.717160\du}{10.846240\du}}
\pgfpathlineto{\pgfpoint{24.727017\du}{10.852081\du}}
\pgfpathlineto{\pgfpoint{24.737240\du}{10.858653\du}}
\pgfpathlineto{\pgfpoint{24.747463\du}{10.864494\du}}
\pgfpathlineto{\pgfpoint{24.758050\du}{10.870336\du}}
\pgfpathlineto{\pgfpoint{24.768273\du}{10.876543\du}}
\pgfpathlineto{\pgfpoint{24.779226\du}{10.882384\du}}
\pgfpathlineto{\pgfpoint{24.790909\du}{10.888226\du}}
\pgfpathlineto{\pgfpoint{24.802227\du}{10.894067\du}}
\pgfpathlineto{\pgfpoint{24.813910\du}{10.899909\du}}
\pgfpathlineto{\pgfpoint{24.825228\du}{10.905750\du}}
\pgfpathlineto{\pgfpoint{24.836911\du}{10.911227\du}}
\pgfpathlineto{\pgfpoint{24.849325\du}{10.917068\du}}
\pgfpathlineto{\pgfpoint{24.861373\du}{10.922180\du}}
\pgfpathlineto{\pgfpoint{24.874516\du}{10.928021\du}}
\pgfpathlineto{\pgfpoint{24.886564\du}{10.933498\du}}
\pgfpathlineto{\pgfpoint{24.899343\du}{10.938609\du}}
\pgfpathlineto{\pgfpoint{24.913217\du}{10.944451\du}}
\pgfpathlineto{\pgfpoint{24.925630\du}{10.949927\du}}
\pgfpathlineto{\pgfpoint{24.938773\du}{10.955038\du}}
\pgfpathlineto{\pgfpoint{24.953012\du}{10.960515\du}}
\pgfpathlineto{\pgfpoint{24.966156\du}{10.964896\du}}
\pgfpathlineto{\pgfpoint{24.980394\du}{10.970372\du}}
\pgfpathlineto{\pgfpoint{24.994268\du}{10.975119\du}}
\pgfpathlineto{\pgfpoint{25.008872\du}{10.980230\du}}
\pgfpathlineto{\pgfpoint{25.022746\du}{10.984976\du}}
\pgfpathlineto{\pgfpoint{25.038445\du}{10.989723\du}}
\pgfpathlineto{\pgfpoint{25.052318\du}{10.994834\du}}
\pgfpathlineto{\pgfpoint{25.066922\du}{10.999580\du}}
\pgfpathlineto{\pgfpoint{25.082256\du}{11.004326\du}}
\pgfpathlineto{\pgfpoint{25.097955\du}{11.009073\du}}
\pgfpathlineto{\pgfpoint{25.112924\du}{11.013089\du}}
\pgfpathlineto{\pgfpoint{25.128624\du}{11.017835\du}}
\pgfpathlineto{\pgfpoint{25.160387\du}{11.026597\du}}
\pgfpathlineto{\pgfpoint{25.192150\du}{11.034629\du}}
\pgfpathlineto{\pgfpoint{25.225374\du}{11.042662\du}}
\pgfpathlineto{\pgfpoint{25.257868\du}{11.051059\du}}
\pgfpathlineto{\pgfpoint{25.292552\du}{11.058726\du}}
\pgfpathlineto{\pgfpoint{25.326871\du}{11.066393\du}}
\pgfpathlineto{\pgfpoint{25.361555\du}{11.073330\du}}
\pgfpathlineto{\pgfpoint{25.397335\du}{11.080267\du}}
\pgfpathlineto{\pgfpoint{25.433479\du}{11.086838\du}}
\pgfpathlineto{\pgfpoint{25.469989\du}{11.092680\du}}
\pgfpathlineto{\pgfpoint{25.507229\du}{11.098886\du}}
\pgfpathlineto{\pgfpoint{25.544834\du}{11.104363\du}}
\pgfpathlineto{\pgfpoint{25.582804\du}{11.109474\du}}
\pgfpathlineto{\pgfpoint{25.621504\du}{11.114951\du}}
\pgfpathlineto{\pgfpoint{25.660204\du}{11.119332\du}}
\pgfpathlineto{\pgfpoint{25.699270\du}{11.123713\du}}
\pgfpathlineto{\pgfpoint{25.739430\du}{11.127729\du}}
\pgfpathlineto{\pgfpoint{25.779226\du}{11.131745\du}}
\pgfpathlineto{\pgfpoint{25.820117\du}{11.135396\du}}
\pgfpathlineto{\pgfpoint{25.861008\du}{11.138317\du}}
\pgfpathlineto{\pgfpoint{25.902264\du}{11.140507\du}}
\pgfpathlineto{\pgfpoint{25.944250\du}{11.143063\du}}
\pgfpathlineto{\pgfpoint{25.985871\du}{11.145254\du}}
\pgfpathlineto{\pgfpoint{26.028222\du}{11.147079\du}}
\pgfpathlineto{\pgfpoint{26.070208\du}{11.148175\du}}
\pgfpathlineto{\pgfpoint{26.113655\du}{11.149270\du}}
\pgfpathlineto{\pgfpoint{26.156371\du}{11.150000\du}}
\pgfpathlineto{\pgfpoint{26.200183\du}{11.150000\du}}
\pgfpathlineto{\pgfpoint{26.200183\du}{11.129555\du}}
\pgfpathlineto{\pgfpoint{26.157466\du}{11.128824\du}}
\pgfpathlineto{\pgfpoint{26.114020\du}{11.128824\du}}
\pgfpathlineto{\pgfpoint{26.071668\du}{11.127729\du}}
\pgfpathlineto{\pgfpoint{26.028952\du}{11.126634\du}}
\pgfpathlineto{\pgfpoint{25.986601\du}{11.124808\du}}
\pgfpathlineto{\pgfpoint{25.944615\du}{11.122983\du}}
\pgfpathlineto{\pgfpoint{25.903724\du}{11.120792\du}}
\pgfpathlineto{\pgfpoint{25.862468\du}{11.117871\du}}
\pgfpathlineto{\pgfpoint{25.821577\du}{11.114221\du}}
\pgfpathlineto{\pgfpoint{25.781417\du}{11.111300\du}}
\pgfpathlineto{\pgfpoint{25.741621\du}{11.107284\du}}
\pgfpathlineto{\pgfpoint{25.702191\du}{11.103633\du}}
\pgfpathlineto{\pgfpoint{25.662760\du}{11.098886\du}}
\pgfpathlineto{\pgfpoint{25.623695\du}{11.094505\du}}
\pgfpathlineto{\pgfpoint{25.585360\du}{11.089759\du}}
\pgfpathlineto{\pgfpoint{25.547755\du}{11.084283\du}}
\pgfpathlineto{\pgfpoint{25.510515\du}{11.078441\du}}
\pgfpathlineto{\pgfpoint{25.474005\du}{11.072599\du}}
\pgfpathlineto{\pgfpoint{25.436765\du}{11.066393\du}}
\pgfpathlineto{\pgfpoint{25.401716\du}{11.059821\du}}
\pgfpathlineto{\pgfpoint{25.365571\du}{11.052884\du}}
\pgfpathlineto{\pgfpoint{25.330887\du}{11.045947\du}}
\pgfpathlineto{\pgfpoint{25.296568\du}{11.038645\du}}
\pgfpathlineto{\pgfpoint{25.262614\du}{11.030978\du}}
\pgfpathlineto{\pgfpoint{25.229390\du}{11.022946\du}}
\pgfpathlineto{\pgfpoint{25.197992\du}{11.014914\du}}
\pgfpathlineto{\pgfpoint{25.165498\du}{11.006517\du}}
\pgfpathlineto{\pgfpoint{25.134465\du}{10.997755\du}}
\pgfpathlineto{\pgfpoint{25.118766\du}{10.993739\du}}
\pgfpathlineto{\pgfpoint{25.103067\du}{10.988992\du}}
\pgfpathlineto{\pgfpoint{25.088463\du}{10.984246\du}}
\pgfpathlineto{\pgfpoint{25.073494\du}{10.979500\du}}
\pgfpathlineto{\pgfpoint{25.058160\du}{10.975119\du}}
\pgfpathlineto{\pgfpoint{25.043921\du}{10.970372\du}}
\pgfpathlineto{\pgfpoint{25.029682\du}{10.965626\du}}
\pgfpathlineto{\pgfpoint{25.015444\du}{10.960880\du}}
\pgfpathlineto{\pgfpoint{25.001205\du}{10.955769\du}}
\pgfpathlineto{\pgfpoint{24.987331\du}{10.951022\du}}
\pgfpathlineto{\pgfpoint{24.973457\du}{10.945546\du}}
\pgfpathlineto{\pgfpoint{24.959949\du}{10.941165\du}}
\pgfpathlineto{\pgfpoint{24.946440\du}{10.935688\du}}
\pgfpathlineto{\pgfpoint{24.933662\du}{10.930577\du}}
\pgfpathlineto{\pgfpoint{24.920153\du}{10.925100\du}}
\pgfpathlineto{\pgfpoint{24.907375\du}{10.919989\du}}
\pgfpathlineto{\pgfpoint{24.894962\du}{10.914878\du}}
\pgfpathlineto{\pgfpoint{24.881818\du}{10.909401\du}}
\pgfpathlineto{\pgfpoint{24.869770\du}{10.903560\du}}
\pgfpathlineto{\pgfpoint{24.858452\du}{10.898448\du}}
\pgfpathlineto{\pgfpoint{24.845674\du}{10.892972\du}}
\pgfpathlineto{\pgfpoint{24.833991\du}{10.887130\du}}
\pgfpathlineto{\pgfpoint{24.822307\du}{10.881289\du}}
\pgfpathlineto{\pgfpoint{24.811355\du}{10.876177\du}}
\pgfpathlineto{\pgfpoint{24.799671\du}{10.870336\du}}
\pgfpathlineto{\pgfpoint{24.789814\du}{10.864494\du}}
\pgfpathlineto{\pgfpoint{24.778496\du}{10.858653\du}}
\pgfpathlineto{\pgfpoint{24.767908\du}{10.852811\du}}
\pgfpathlineto{\pgfpoint{24.758050\du}{10.846970\du}}
\pgfpathlineto{\pgfpoint{24.747463\du}{10.841128\du}}
\pgfpathlineto{\pgfpoint{24.737605\du}{10.835287\du}}
\pgfpathlineto{\pgfpoint{24.728112\du}{10.828715\du}}
\pgfpathlineto{\pgfpoint{24.718255\du}{10.822873\du}}
\pgfpathlineto{\pgfpoint{24.709127\du}{10.816302\du}}
\pgfpathlineto{\pgfpoint{24.700730\du}{10.810460\du}}
\pgfpathlineto{\pgfpoint{24.691968\du}{10.804253\du}}
\pgfpathlineto{\pgfpoint{24.682840\du}{10.797682\du}}
\pgfpathlineto{\pgfpoint{24.674078\du}{10.791840\du}}
\pgfpathlineto{\pgfpoint{24.665681\du}{10.785268\du}}
\pgfpathlineto{\pgfpoint{24.658379\du}{10.779062\du}}
\pgfpathlineto{\pgfpoint{24.650347\du}{10.772490\du}}
\pgfpathlineto{\pgfpoint{24.642680\du}{10.765918\du}}
\pgfpathlineto{\pgfpoint{24.635743\du}{10.759712\du}}
\pgfpathlineto{\pgfpoint{24.628441\du}{10.753140\du}}
\pgfpathlineto{\pgfpoint{24.621504\du}{10.746933\du}}
\pgfpathlineto{\pgfpoint{24.615298\du}{10.740361\du}}
\pgfpathlineto{\pgfpoint{24.608726\du}{10.733425\du}}
\pgfpathlineto{\pgfpoint{24.602884\du}{10.726853\du}}
\pgfpathlineto{\pgfpoint{24.596678\du}{10.720281\du}}
\pgfpathlineto{\pgfpoint{24.591566\du}{10.714074\du}}
\pgfpathlineto{\pgfpoint{24.585360\du}{10.707138\du}}
\pgfpathlineto{\pgfpoint{24.580978\du}{10.700566\du}}
\pgfpathlineto{\pgfpoint{24.575502\du}{10.693629\du}}
\pgfpathlineto{\pgfpoint{24.571121\du}{10.687057\du}}
\pgfpathlineto{\pgfpoint{24.566740\du}{10.680120\du}}
\pgfpathlineto{\pgfpoint{24.562359\du}{10.673184\du}}
\pgfpathlineto{\pgfpoint{24.559073\du}{10.666612\du}}
\pgfpathlineto{\pgfpoint{24.555057\du}{10.659675\du}}
\pgfpathlineto{\pgfpoint{24.551041\du}{10.653103\du}}
\pgfpathlineto{\pgfpoint{24.548120\du}{10.646166\du}}
\pgfpathlineto{\pgfpoint{24.545564\du}{10.639230\du}}
\pgfpathlineto{\pgfpoint{24.542278\du}{10.631928\du}}
\pgfpathlineto{\pgfpoint{24.540088\du}{10.624991\du}}
\pgfpathlineto{\pgfpoint{24.537897\du}{10.618784\du}}
\pgfpathlineto{\pgfpoint{24.536437\du}{10.611482\du}}
\pgfpathlineto{\pgfpoint{24.534246\du}{10.604545\du}}
\pgfpathlineto{\pgfpoint{24.533151\du}{10.597609\du}}
\pgfpathlineto{\pgfpoint{24.532055\du}{10.590672\du}}
\pgfpathlineto{\pgfpoint{24.530595\du}{10.583370\du}}
\pgfpathlineto{\pgfpoint{24.530230\du}{10.576433\du}}
\pgfpathlineto{\pgfpoint{24.530230\du}{10.569496\du}}
\pgfpathlineto{\pgfpoint{24.529865\du}{10.562559\du}}
\pgfpathlineto{\pgfpoint{24.529865\du}{10.562559\du}}
\pgfpathlineto{\pgfpoint{24.529865\du}{10.562559\du}}
\pgfpathlineto{\pgfpoint{24.529865\du}{10.560734\du}}
\pgfpathlineto{\pgfpoint{24.529865\du}{10.560004\du}}
\pgfpathlineto{\pgfpoint{24.529500\du}{10.558908\du}}
\pgfpathlineto{\pgfpoint{24.529500\du}{10.557813\du}}
\pgfpathlineto{\pgfpoint{24.528405\du}{10.557083\du}}
\pgfpathlineto{\pgfpoint{24.528039\du}{10.555988\du}}
\pgfpathlineto{\pgfpoint{24.527674\du}{10.555257\du}}
\pgfpathlineto{\pgfpoint{24.526579\du}{10.554892\du}}
\pgfpathlineto{\pgfpoint{24.525119\du}{10.553797\du}}
\pgfpathlineto{\pgfpoint{24.523293\du}{10.552337\du}}
\pgfpathlineto{\pgfpoint{24.521468\du}{10.551972\du}}
\pgfpathlineto{\pgfpoint{24.519642\du}{10.551972\du}}
\pgfpathlineto{\pgfpoint{24.517452\du}{10.551972\du}}
\pgfpathlineto{\pgfpoint{24.515991\du}{10.552337\du}}
\pgfpathlineto{\pgfpoint{24.513801\du}{10.553797\du}}
\pgfpathlineto{\pgfpoint{24.511975\du}{10.554892\du}}
\pgfpathlineto{\pgfpoint{24.511610\du}{10.555257\du}}
\pgfpathlineto{\pgfpoint{24.511245\du}{10.555988\du}}
\pgfpathlineto{\pgfpoint{24.510515\du}{10.557083\du}}
\pgfpathlineto{\pgfpoint{24.509785\du}{10.557813\du}}
\pgfpathlineto{\pgfpoint{24.509785\du}{10.558908\du}}
\pgfpathlineto{\pgfpoint{24.509419\du}{10.560004\du}}
\pgfpathlineto{\pgfpoint{24.509419\du}{10.560734\du}}
\pgfpathlineto{\pgfpoint{24.509419\du}{10.562559\du}}
\pgfusepath{fill}
\pgfsetbuttcap
\pgfsetmiterjoin
\pgfsetdash{}{0pt}
\definecolor{dialinecolor}{rgb}{0.678431, 0.839216, 0.905882}
\pgfsetfillcolor{dialinecolor}
\pgfpathmoveto{\pgfpoint{26.200183\du}{9.975119\du}}
\pgfpathlineto{\pgfpoint{26.200183\du}{9.975119\du}}
\pgfpathlineto{\pgfpoint{26.156371\du}{9.975119\du}}
\pgfpathlineto{\pgfpoint{26.113655\du}{9.975484\du}}
\pgfpathlineto{\pgfpoint{26.070208\du}{9.976579\du}}
\pgfpathlineto{\pgfpoint{26.028222\du}{9.978039\du}}
\pgfpathlineto{\pgfpoint{25.985871\du}{9.979500\du}}
\pgfpathlineto{\pgfpoint{25.944250\du}{9.981325\du}}
\pgfpathlineto{\pgfpoint{25.902264\du}{9.983881\du}}
\pgfpathlineto{\pgfpoint{25.861008\du}{9.986802\du}}
\pgfpathlineto{\pgfpoint{25.820117\du}{9.989723\du}}
\pgfpathlineto{\pgfpoint{25.779226\du}{9.992643\du}}
\pgfpathlineto{\pgfpoint{25.739430\du}{9.996659\du}}
\pgfpathlineto{\pgfpoint{25.699270\du}{10.000675\du}}
\pgfpathlineto{\pgfpoint{25.660204\du}{10.005422\du}}
\pgfpathlineto{\pgfpoint{25.621504\du}{10.010168\du}}
\pgfpathlineto{\pgfpoint{25.582804\du}{10.014914\du}}
\pgfpathlineto{\pgfpoint{25.544834\du}{10.020026\du}}
\pgfpathlineto{\pgfpoint{25.507229\du}{10.025867\du}}
\pgfpathlineto{\pgfpoint{25.469989\du}{10.031709\du}}
\pgfpathlineto{\pgfpoint{25.433479\du}{10.038280\du}}
\pgfpathlineto{\pgfpoint{25.397335\du}{10.044487\du}}
\pgfpathlineto{\pgfpoint{25.361555\du}{10.051789\du}}
\pgfpathlineto{\pgfpoint{25.326871\du}{10.058726\du}}
\pgfpathlineto{\pgfpoint{25.292552\du}{10.065663\du}}
\pgfpathlineto{\pgfpoint{25.257868\du}{10.073695\du}}
\pgfpathlineto{\pgfpoint{25.225374\du}{10.081362\du}}
\pgfpathlineto{\pgfpoint{25.192150\du}{10.089759\du}}
\pgfpathlineto{\pgfpoint{25.160387\du}{10.098521\du}}
\pgfpathlineto{\pgfpoint{25.128624\du}{10.107284\du}}
\pgfpathlineto{\pgfpoint{25.097955\du}{10.116046\du}}
\pgfpathlineto{\pgfpoint{25.066922\du}{10.125539\du}}
\pgfpathlineto{\pgfpoint{25.052318\du}{10.129920\du}}
\pgfpathlineto{\pgfpoint{25.038445\du}{10.134666\du}}
\pgfpathlineto{\pgfpoint{25.022746\du}{10.139412\du}}
\pgfpathlineto{\pgfpoint{25.008872\du}{10.144524\du}}
\pgfpathlineto{\pgfpoint{24.994268\du}{10.149270\du}}
\pgfpathlineto{\pgfpoint{24.980394\du}{10.154746\du}}
\pgfpathlineto{\pgfpoint{24.966156\du}{10.159127\du}}
\pgfpathlineto{\pgfpoint{24.953012\du}{10.164604\du}}
\pgfpathlineto{\pgfpoint{24.938773\du}{10.169715\du}}
\pgfpathlineto{\pgfpoint{24.925630\du}{10.175192\du}}
\pgfpathlineto{\pgfpoint{24.913217\du}{10.180303\du}}
\pgfpathlineto{\pgfpoint{24.899343\du}{10.185779\du}}
\pgfpathlineto{\pgfpoint{24.886564\du}{10.190891\du}}
\pgfpathlineto{\pgfpoint{24.874516\du}{10.196002\du}}
\pgfpathlineto{\pgfpoint{24.861373\du}{10.201844\du}}
\pgfpathlineto{\pgfpoint{24.849325\du}{10.208050\du}}
\pgfpathlineto{\pgfpoint{24.836911\du}{10.213162\du}}
\pgfpathlineto{\pgfpoint{24.825228\du}{10.219003\du}}
\pgfpathlineto{\pgfpoint{24.813910\du}{10.224845\du}}
\pgfpathlineto{\pgfpoint{24.802227\du}{10.230686\du}}
\pgfpathlineto{\pgfpoint{24.790909\du}{10.236528\du}}
\pgfpathlineto{\pgfpoint{24.779226\du}{10.241639\du}}
\pgfpathlineto{\pgfpoint{24.768273\du}{10.248211\du}}
\pgfpathlineto{\pgfpoint{24.758050\du}{10.254053\du}}
\pgfpathlineto{\pgfpoint{24.747463\du}{10.259894\du}}
\pgfpathlineto{\pgfpoint{24.737240\du}{10.265736\du}}
\pgfpathlineto{\pgfpoint{24.727017\du}{10.272307\du}}
\pgfpathlineto{\pgfpoint{24.717160\du}{10.278514\du}}
\pgfpathlineto{\pgfpoint{24.707667\du}{10.284356\du}}
\pgfpathlineto{\pgfpoint{24.698175\du}{10.290927\du}}
\pgfpathlineto{\pgfpoint{24.689047\du}{10.297499\du}}
\pgfpathlineto{\pgfpoint{24.679189\du}{10.303706\du}}
\pgfpathlineto{\pgfpoint{24.670792\du}{10.310277\du}}
\pgfpathlineto{\pgfpoint{24.662030\du}{10.316849\du}}
\pgfpathlineto{\pgfpoint{24.653633\du}{10.323056\du}}
\pgfpathlineto{\pgfpoint{24.644870\du}{10.329628\du}}
\pgfpathlineto{\pgfpoint{24.637203\du}{10.336564\du}}
\pgfpathlineto{\pgfpoint{24.630267\du}{10.343136\du}}
\pgfpathlineto{\pgfpoint{24.621504\du}{10.349343\du}}
\pgfpathlineto{\pgfpoint{24.614567\du}{10.356645\du}}
\pgfpathlineto{\pgfpoint{24.607631\du}{10.362851\du}}
\pgfpathlineto{\pgfpoint{24.600694\du}{10.369788\du}}
\pgfpathlineto{\pgfpoint{24.594122\du}{10.377090\du}}
\pgfpathlineto{\pgfpoint{24.587185\du}{10.383297\du}}
\pgfpathlineto{\pgfpoint{24.581344\du}{10.390599\du}}
\pgfpathlineto{\pgfpoint{24.575502\du}{10.397536\du}}
\pgfpathlineto{\pgfpoint{24.569660\du}{10.405203\du}}
\pgfpathlineto{\pgfpoint{24.564184\du}{10.412139\du}}
\pgfpathlineto{\pgfpoint{24.559438\du}{10.419076\du}}
\pgfpathlineto{\pgfpoint{24.553961\du}{10.426013\du}}
\pgfpathlineto{\pgfpoint{24.549945\du}{10.433315\du}}
\pgfpathlineto{\pgfpoint{24.544834\du}{10.440617\du}}
\pgfpathlineto{\pgfpoint{24.541183\du}{10.447919\du}}
\pgfpathlineto{\pgfpoint{24.536802\du}{10.455221\du}}
\pgfpathlineto{\pgfpoint{24.533151\du}{10.462888\du}}
\pgfpathlineto{\pgfpoint{24.529865\du}{10.470555\du}}
\pgfpathlineto{\pgfpoint{24.526214\du}{10.477492\du}}
\pgfpathlineto{\pgfpoint{24.523293\du}{10.485159\du}}
\pgfpathlineto{\pgfpoint{24.520737\du}{10.492826\du}}
\pgfpathlineto{\pgfpoint{24.518547\du}{10.500493\du}}
\pgfpathlineto{\pgfpoint{24.516356\du}{10.508160\du}}
\pgfpathlineto{\pgfpoint{24.514531\du}{10.515462\du}}
\pgfpathlineto{\pgfpoint{24.512705\du}{10.523129\du}}
\pgfpathlineto{\pgfpoint{24.511610\du}{10.530796\du}}
\pgfpathlineto{\pgfpoint{24.510515\du}{10.539193\du}}
\pgfpathlineto{\pgfpoint{24.509785\du}{10.546495\du}}
\pgfpathlineto{\pgfpoint{24.509419\du}{10.554162\du}}
\pgfpathlineto{\pgfpoint{24.509419\du}{10.562559\du}}
\pgfpathlineto{\pgfpoint{24.529865\du}{10.562559\du}}
\pgfpathlineto{\pgfpoint{24.530230\du}{10.555257\du}}
\pgfpathlineto{\pgfpoint{24.530230\du}{10.548321\du}}
\pgfpathlineto{\pgfpoint{24.530595\du}{10.541384\du}}
\pgfpathlineto{\pgfpoint{24.532055\du}{10.533717\du}}
\pgfpathlineto{\pgfpoint{24.533151\du}{10.527510\du}}
\pgfpathlineto{\pgfpoint{24.534246\du}{10.520208\du}}
\pgfpathlineto{\pgfpoint{24.536437\du}{10.513271\du}}
\pgfpathlineto{\pgfpoint{24.537897\du}{10.506334\du}}
\pgfpathlineto{\pgfpoint{24.540088\du}{10.499398\du}}
\pgfpathlineto{\pgfpoint{24.542278\du}{10.492096\du}}
\pgfpathlineto{\pgfpoint{24.545564\du}{10.485889\du}}
\pgfpathlineto{\pgfpoint{24.548120\du}{10.478952\du}}
\pgfpathlineto{\pgfpoint{24.551041\du}{10.471650\du}}
\pgfpathlineto{\pgfpoint{24.555057\du}{10.464713\du}}
\pgfpathlineto{\pgfpoint{24.559073\du}{10.457777\du}}
\pgfpathlineto{\pgfpoint{24.562359\du}{10.451205\du}}
\pgfpathlineto{\pgfpoint{24.566375\du}{10.444268\du}}
\pgfpathlineto{\pgfpoint{24.571121\du}{10.437696\du}}
\pgfpathlineto{\pgfpoint{24.575502\du}{10.430759\du}}
\pgfpathlineto{\pgfpoint{24.580978\du}{10.424188\du}}
\pgfpathlineto{\pgfpoint{24.585360\du}{10.417981\du}}
\pgfpathlineto{\pgfpoint{24.591566\du}{10.411044\du}}
\pgfpathlineto{\pgfpoint{24.596678\du}{10.404472\du}}
\pgfpathlineto{\pgfpoint{24.602884\du}{10.397536\du}}
\pgfpathlineto{\pgfpoint{24.608726\du}{10.390964\du}}
\pgfpathlineto{\pgfpoint{24.615298\du}{10.384757\du}}
\pgfpathlineto{\pgfpoint{24.621504\du}{10.378185\du}}
\pgfpathlineto{\pgfpoint{24.628441\du}{10.371249\du}}
\pgfpathlineto{\pgfpoint{24.635743\du}{10.365407\du}}
\pgfpathlineto{\pgfpoint{24.642680\du}{10.358105\du}}
\pgfpathlineto{\pgfpoint{24.650347\du}{10.351899\du}}
\pgfpathlineto{\pgfpoint{24.658379\du}{10.346057\du}}
\pgfpathlineto{\pgfpoint{24.665681\du}{10.339485\du}}
\pgfpathlineto{\pgfpoint{24.674078\du}{10.332913\du}}
\pgfpathlineto{\pgfpoint{24.682840\du}{10.326707\du}}
\pgfpathlineto{\pgfpoint{24.691968\du}{10.320135\du}}
\pgfpathlineto{\pgfpoint{24.700730\du}{10.314294\du}}
\pgfpathlineto{\pgfpoint{24.709127\du}{10.308087\du}}
\pgfpathlineto{\pgfpoint{24.718255\du}{10.302245\du}}
\pgfpathlineto{\pgfpoint{24.728112\du}{10.295674\du}}
\pgfpathlineto{\pgfpoint{24.737605\du}{10.289832\du}}
\pgfpathlineto{\pgfpoint{24.747463\du}{10.283991\du}}
\pgfpathlineto{\pgfpoint{24.758050\du}{10.278149\du}}
\pgfpathlineto{\pgfpoint{24.767908\du}{10.272307\du}}
\pgfpathlineto{\pgfpoint{24.778496\du}{10.265736\du}}
\pgfpathlineto{\pgfpoint{24.789814\du}{10.260624\du}}
\pgfpathlineto{\pgfpoint{24.799671\du}{10.254783\du}}
\pgfpathlineto{\pgfpoint{24.811355\du}{10.248941\du}}
\pgfpathlineto{\pgfpoint{24.822307\du}{10.243100\du}}
\pgfpathlineto{\pgfpoint{24.833991\du}{10.237258\du}}
\pgfpathlineto{\pgfpoint{24.845674\du}{10.231782\du}}
\pgfpathlineto{\pgfpoint{24.858452\du}{10.226670\du}}
\pgfpathlineto{\pgfpoint{24.869770\du}{10.220829\du}}
\pgfpathlineto{\pgfpoint{24.881818\du}{10.215352\du}}
\pgfpathlineto{\pgfpoint{24.894962\du}{10.210241\du}}
\pgfpathlineto{\pgfpoint{24.907375\du}{10.204765\du}}
\pgfpathlineto{\pgfpoint{24.920153\du}{10.199653\du}}
\pgfpathlineto{\pgfpoint{24.933662\du}{10.193812\du}}
\pgfpathlineto{\pgfpoint{24.946440\du}{10.189065\du}}
\pgfpathlineto{\pgfpoint{24.959949\du}{10.183954\du}}
\pgfpathlineto{\pgfpoint{24.973457\du}{10.178478\du}}
\pgfpathlineto{\pgfpoint{24.987331\du}{10.174096\du}}
\pgfpathlineto{\pgfpoint{25.001205\du}{10.168620\du}}
\pgfpathlineto{\pgfpoint{25.015444\du}{10.163874\du}}
\pgfpathlineto{\pgfpoint{25.029682\du}{10.159127\du}}
\pgfpathlineto{\pgfpoint{25.043921\du}{10.154016\du}}
\pgfpathlineto{\pgfpoint{25.058160\du}{10.149270\du}}
\pgfpathlineto{\pgfpoint{25.073494\du}{10.144524\du}}
\pgfpathlineto{\pgfpoint{25.103067\du}{10.135761\du}}
\pgfpathlineto{\pgfpoint{25.134465\du}{10.126634\du}}
\pgfpathlineto{\pgfpoint{25.165498\du}{10.118237\du}}
\pgfpathlineto{\pgfpoint{25.197992\du}{10.109474\du}}
\pgfpathlineto{\pgfpoint{25.229390\du}{10.101442\du}}
\pgfpathlineto{\pgfpoint{25.262614\du}{10.093775\du}}
\pgfpathlineto{\pgfpoint{25.296568\du}{10.086108\du}}
\pgfpathlineto{\pgfpoint{25.330887\du}{10.078441\du}}
\pgfpathlineto{\pgfpoint{25.365571\du}{10.071504\du}}
\pgfpathlineto{\pgfpoint{25.401716\du}{10.064567\du}}
\pgfpathlineto{\pgfpoint{25.436765\du}{10.058726\du}}
\pgfpathlineto{\pgfpoint{25.474005\du}{10.052154\du}}
\pgfpathlineto{\pgfpoint{25.510515\du}{10.046313\du}}
\pgfpathlineto{\pgfpoint{25.547755\du}{10.040471\du}}
\pgfpathlineto{\pgfpoint{25.585360\du}{10.035360\du}}
\pgfpathlineto{\pgfpoint{25.623695\du}{10.029883\du}}
\pgfpathlineto{\pgfpoint{25.662760\du}{10.025137\du}}
\pgfpathlineto{\pgfpoint{25.702191\du}{10.021121\du}}
\pgfpathlineto{\pgfpoint{25.741621\du}{10.017105\du}}
\pgfpathlineto{\pgfpoint{25.781417\du}{10.013454\du}}
\pgfpathlineto{\pgfpoint{25.821577\du}{10.010168\du}}
\pgfpathlineto{\pgfpoint{25.862468\du}{10.007247\du}}
\pgfpathlineto{\pgfpoint{25.903724\du}{10.004326\du}}
\pgfpathlineto{\pgfpoint{25.944615\du}{10.001771\du}}
\pgfpathlineto{\pgfpoint{25.986601\du}{9.999580\du}}
\pgfpathlineto{\pgfpoint{26.028952\du}{9.998485\du}}
\pgfpathlineto{\pgfpoint{26.071668\du}{9.996659\du}}
\pgfpathlineto{\pgfpoint{26.114020\du}{9.995929\du}}
\pgfpathlineto{\pgfpoint{26.157466\du}{9.995564\du}}
\pgfpathlineto{\pgfpoint{26.200183\du}{9.995564\du}}
\pgfpathlineto{\pgfpoint{26.200183\du}{9.995564\du}}
\pgfpathlineto{\pgfpoint{26.200183\du}{9.995564\du}}
\pgfpathlineto{\pgfpoint{26.201278\du}{9.994834\du}}
\pgfpathlineto{\pgfpoint{26.202373\du}{9.994834\du}}
\pgfpathlineto{\pgfpoint{26.203834\du}{9.994834\du}}
\pgfpathlineto{\pgfpoint{26.204929\du}{9.994469\du}}
\pgfpathlineto{\pgfpoint{26.205294\du}{9.993739\du}}
\pgfpathlineto{\pgfpoint{26.206389\du}{9.993739\du}}
\pgfpathlineto{\pgfpoint{26.207119\du}{9.992643\du}}
\pgfpathlineto{\pgfpoint{26.208215\du}{9.991913\du}}
\pgfpathlineto{\pgfpoint{26.209310\du}{9.990818\du}}
\pgfpathlineto{\pgfpoint{26.210040\du}{9.988992\du}}
\pgfpathlineto{\pgfpoint{26.210040\du}{9.986802\du}}
\pgfpathlineto{\pgfpoint{26.210770\du}{9.984976\du}}
\pgfpathlineto{\pgfpoint{26.210040\du}{9.983151\du}}
\pgfpathlineto{\pgfpoint{26.210040\du}{9.981325\du}}
\pgfpathlineto{\pgfpoint{26.209310\du}{9.979500\du}}
\pgfpathlineto{\pgfpoint{26.208215\du}{9.978039\du}}
\pgfpathlineto{\pgfpoint{26.207119\du}{9.977309\du}}
\pgfpathlineto{\pgfpoint{26.206389\du}{9.976579\du}}
\pgfpathlineto{\pgfpoint{26.205294\du}{9.976214\du}}
\pgfpathlineto{\pgfpoint{26.204929\du}{9.975484\du}}
\pgfpathlineto{\pgfpoint{26.203834\du}{9.975119\du}}
\pgfpathlineto{\pgfpoint{26.202373\du}{9.975119\du}}
\pgfpathlineto{\pgfpoint{26.201278\du}{9.975119\du}}
\pgfpathlineto{\pgfpoint{26.200183\du}{9.975119\du}}
\pgfusepath{fill}
\pgfsetbuttcap
\pgfsetmiterjoin
\pgfsetdash{}{0pt}
\definecolor{dialinecolor}{rgb}{0.678431, 0.839216, 0.905882}
\pgfsetfillcolor{dialinecolor}
\pgfpathmoveto{\pgfpoint{27.890581\du}{10.562559\du}}
\pgfpathlineto{\pgfpoint{27.890581\du}{10.554162\du}}
\pgfpathlineto{\pgfpoint{27.889850\du}{10.546495\du}}
\pgfpathlineto{\pgfpoint{27.889120\du}{10.539193\du}}
\pgfpathlineto{\pgfpoint{27.888390\du}{10.530796\du}}
\pgfpathlineto{\pgfpoint{27.887295\du}{10.523129\du}}
\pgfpathlineto{\pgfpoint{27.885104\du}{10.515462\du}}
\pgfpathlineto{\pgfpoint{27.884009\du}{10.508160\du}}
\pgfpathlineto{\pgfpoint{27.881088\du}{10.500493\du}}
\pgfpathlineto{\pgfpoint{27.879263\du}{10.492826\du}}
\pgfpathlineto{\pgfpoint{27.876342\du}{10.485159\du}}
\pgfpathlineto{\pgfpoint{27.873786\du}{10.477492\du}}
\pgfpathlineto{\pgfpoint{27.870500\du}{10.470555\du}}
\pgfpathlineto{\pgfpoint{27.866484\du}{10.462888\du}}
\pgfpathlineto{\pgfpoint{27.863198\du}{10.455221\du}}
\pgfpathlineto{\pgfpoint{27.859182\du}{10.447919\du}}
\pgfpathlineto{\pgfpoint{27.855166\du}{10.440617\du}}
\pgfpathlineto{\pgfpoint{27.850785\du}{10.433315\du}}
\pgfpathlineto{\pgfpoint{27.846039\du}{10.426013\du}}
\pgfpathlineto{\pgfpoint{27.840562\du}{10.419076\du}}
\pgfpathlineto{\pgfpoint{27.835816\du}{10.412139\du}}
\pgfpathlineto{\pgfpoint{27.829974\du}{10.404472\du}}
\pgfpathlineto{\pgfpoint{27.824498\du}{10.397536\du}}
\pgfpathlineto{\pgfpoint{27.818656\du}{10.390599\du}}
\pgfpathlineto{\pgfpoint{27.812450\du}{10.383297\du}}
\pgfpathlineto{\pgfpoint{27.805878\du}{10.377090\du}}
\pgfpathlineto{\pgfpoint{27.798941\du}{10.369788\du}}
\pgfpathlineto{\pgfpoint{27.792369\du}{10.362851\du}}
\pgfpathlineto{\pgfpoint{27.785068\du}{10.356645\du}}
\pgfpathlineto{\pgfpoint{27.778496\du}{10.349343\du}}
\pgfpathlineto{\pgfpoint{27.770464\du}{10.343136\du}}
\pgfpathlineto{\pgfpoint{27.762432\du}{10.336564\du}}
\pgfpathlineto{\pgfpoint{27.755130\du}{10.329628\du}}
\pgfpathlineto{\pgfpoint{27.746367\du}{10.323056\du}}
\pgfpathlineto{\pgfpoint{27.738335\du}{10.316849\du}}
\pgfpathlineto{\pgfpoint{27.729208\du}{10.310277\du}}
\pgfpathlineto{\pgfpoint{27.720445\du}{10.303706\du}}
\pgfpathlineto{\pgfpoint{27.710953\du}{10.297499\du}}
\pgfpathlineto{\pgfpoint{27.701825\du}{10.290927\du}}
\pgfpathlineto{\pgfpoint{27.692698\du}{10.284356\du}}
\pgfpathlineto{\pgfpoint{27.682840\du}{10.278514\du}}
\pgfpathlineto{\pgfpoint{27.672983\du}{10.272307\du}}
\pgfpathlineto{\pgfpoint{27.662760\du}{10.265736\du}}
\pgfpathlineto{\pgfpoint{27.652172\du}{10.259894\du}}
\pgfpathlineto{\pgfpoint{27.642315\du}{10.254053\du}}
\pgfpathlineto{\pgfpoint{27.631727\du}{10.248211\du}}
\pgfpathlineto{\pgfpoint{27.620409\du}{10.241639\du}}
\pgfpathlineto{\pgfpoint{27.609091\du}{10.236528\du}}
\pgfpathlineto{\pgfpoint{27.597408\du}{10.230686\du}}
\pgfpathlineto{\pgfpoint{27.586455\du}{10.224845\du}}
\pgfpathlineto{\pgfpoint{27.574407\du}{10.219003\du}}
\pgfpathlineto{\pgfpoint{27.563089\du}{10.213162\du}}
\pgfpathlineto{\pgfpoint{27.550675\du}{10.208050\du}}
\pgfpathlineto{\pgfpoint{27.538262\du}{10.201844\du}}
\pgfpathlineto{\pgfpoint{27.525484\du}{10.196002\du}}
\pgfpathlineto{\pgfpoint{27.513436\du}{10.190891\du}}
\pgfpathlineto{\pgfpoint{27.500657\du}{10.185779\du}}
\pgfpathlineto{\pgfpoint{27.487149\du}{10.180303\du}}
\pgfpathlineto{\pgfpoint{27.474005\du}{10.175192\du}}
\pgfpathlineto{\pgfpoint{27.460862\du}{10.169715\du}}
\pgfpathlineto{\pgfpoint{27.446623\du}{10.164604\du}}
\pgfpathlineto{\pgfpoint{27.433479\du}{10.159127\du}}
\pgfpathlineto{\pgfpoint{27.419241\du}{10.154746\du}}
\pgfpathlineto{\pgfpoint{27.405367\du}{10.149270\du}}
\pgfpathlineto{\pgfpoint{27.391128\du}{10.144524\du}}
\pgfpathlineto{\pgfpoint{27.377254\du}{10.139412\du}}
\pgfpathlineto{\pgfpoint{27.362286\du}{10.134666\du}}
\pgfpathlineto{\pgfpoint{27.347317\du}{10.129920\du}}
\pgfpathlineto{\pgfpoint{27.333078\du}{10.125539\du}}
\pgfpathlineto{\pgfpoint{27.302410\du}{10.116046\du}}
\pgfpathlineto{\pgfpoint{27.271742\du}{10.107284\du}}
\pgfpathlineto{\pgfpoint{27.240343\du}{10.098521\du}}
\pgfpathlineto{\pgfpoint{27.208580\du}{10.089759\du}}
\pgfpathlineto{\pgfpoint{27.175721\du}{10.081362\du}}
\pgfpathlineto{\pgfpoint{27.142132\du}{10.073695\du}}
\pgfpathlineto{\pgfpoint{27.108178\du}{10.065663\du}}
\pgfpathlineto{\pgfpoint{27.073494\du}{10.058726\du}}
\pgfpathlineto{\pgfpoint{27.038810\du}{10.051789\du}}
\pgfpathlineto{\pgfpoint{27.003395\du}{10.044487\du}}
\pgfpathlineto{\pgfpoint{26.966886\du}{10.038280\du}}
\pgfpathlineto{\pgfpoint{26.930376\du}{10.031709\du}}
\pgfpathlineto{\pgfpoint{26.893501\du}{10.025867\du}}
\pgfpathlineto{\pgfpoint{26.856261\du}{10.020026\du}}
\pgfpathlineto{\pgfpoint{26.817196\du}{10.014914\du}}
\pgfpathlineto{\pgfpoint{26.779226\du}{10.010168\du}}
\pgfpathlineto{\pgfpoint{26.739796\du}{10.005422\du}}
\pgfpathlineto{\pgfpoint{26.701095\du}{10.000675\du}}
\pgfpathlineto{\pgfpoint{26.660935\du}{9.996659\du}}
\pgfpathlineto{\pgfpoint{26.621139\du}{9.992643\du}}
\pgfpathlineto{\pgfpoint{26.580248\du}{9.989723\du}}
\pgfpathlineto{\pgfpoint{26.539723\du}{9.986802\du}}
\pgfpathlineto{\pgfpoint{26.498101\du}{9.983881\du}}
\pgfpathlineto{\pgfpoint{26.456480\du}{9.981325\du}}
\pgfpathlineto{\pgfpoint{26.414859\du}{9.979500\du}}
\pgfpathlineto{\pgfpoint{26.372143\du}{9.978039\du}}
\pgfpathlineto{\pgfpoint{26.330157\du}{9.976579\du}}
\pgfpathlineto{\pgfpoint{26.287076\du}{9.975484\du}}
\pgfpathlineto{\pgfpoint{26.243629\du}{9.975119\du}}
\pgfpathlineto{\pgfpoint{26.200183\du}{9.975119\du}}
\pgfpathlineto{\pgfpoint{26.200183\du}{9.995564\du}}
\pgfpathlineto{\pgfpoint{26.243264\du}{9.995564\du}}
\pgfpathlineto{\pgfpoint{26.286710\du}{9.995929\du}}
\pgfpathlineto{\pgfpoint{26.328697\du}{9.996659\du}}
\pgfpathlineto{\pgfpoint{26.371413\du}{9.998485\du}}
\pgfpathlineto{\pgfpoint{26.414129\du}{9.999580\du}}
\pgfpathlineto{\pgfpoint{26.455750\du}{10.001771\du}}
\pgfpathlineto{\pgfpoint{26.497006\du}{10.004326\du}}
\pgfpathlineto{\pgfpoint{26.537897\du}{10.007247\du}}
\pgfpathlineto{\pgfpoint{26.579153\du}{10.010168\du}}
\pgfpathlineto{\pgfpoint{26.619314\du}{10.013454\du}}
\pgfpathlineto{\pgfpoint{26.659474\du}{10.017105\du}}
\pgfpathlineto{\pgfpoint{26.698540\du}{10.021121\du}}
\pgfpathlineto{\pgfpoint{26.737970\du}{10.025137\du}}
\pgfpathlineto{\pgfpoint{26.776670\du}{10.029883\du}}
\pgfpathlineto{\pgfpoint{26.815371\du}{10.035360\du}}
\pgfpathlineto{\pgfpoint{26.852976\du}{10.040471\du}}
\pgfpathlineto{\pgfpoint{26.890215\du}{10.046313\du}}
\pgfpathlineto{\pgfpoint{26.926725\du}{10.052154\du}}
\pgfpathlineto{\pgfpoint{26.963600\du}{10.058726\du}}
\pgfpathlineto{\pgfpoint{26.999014\du}{10.064567\du}}
\pgfpathlineto{\pgfpoint{27.034794\du}{10.071504\du}}
\pgfpathlineto{\pgfpoint{27.069478\du}{10.078441\du}}
\pgfpathlineto{\pgfpoint{27.103432\du}{10.086108\du}}
\pgfpathlineto{\pgfpoint{27.137751\du}{10.093775\du}}
\pgfpathlineto{\pgfpoint{27.170975\du}{10.101442\du}}
\pgfpathlineto{\pgfpoint{27.203103\du}{10.109474\du}}
\pgfpathlineto{\pgfpoint{27.235232\du}{10.118237\du}}
\pgfpathlineto{\pgfpoint{27.266630\du}{10.126634\du}}
\pgfpathlineto{\pgfpoint{27.297298\du}{10.135761\du}}
\pgfpathlineto{\pgfpoint{27.327236\du}{10.144524\du}}
\pgfpathlineto{\pgfpoint{27.341475\du}{10.149270\du}}
\pgfpathlineto{\pgfpoint{27.356079\du}{10.154016\du}}
\pgfpathlineto{\pgfpoint{27.369953\du}{10.159127\du}}
\pgfpathlineto{\pgfpoint{27.384556\du}{10.163874\du}}
\pgfpathlineto{\pgfpoint{27.398795\du}{10.168620\du}}
\pgfpathlineto{\pgfpoint{27.412669\du}{10.174096\du}}
\pgfpathlineto{\pgfpoint{27.426908\du}{10.178478\du}}
\pgfpathlineto{\pgfpoint{27.440051\du}{10.183954\du}}
\pgfpathlineto{\pgfpoint{27.453560\du}{10.189065\du}}
\pgfpathlineto{\pgfpoint{27.466703\du}{10.193812\du}}
\pgfpathlineto{\pgfpoint{27.479482\du}{10.199653\du}}
\pgfpathlineto{\pgfpoint{27.492260\du}{10.204765\du}}
\pgfpathlineto{\pgfpoint{27.505403\du}{10.210241\du}}
\pgfpathlineto{\pgfpoint{27.517452\du}{10.215352\du}}
\pgfpathlineto{\pgfpoint{27.529865\du}{10.220829\du}}
\pgfpathlineto{\pgfpoint{27.541913\du}{10.226670\du}}
\pgfpathlineto{\pgfpoint{27.554326\du}{10.231782\du}}
\pgfpathlineto{\pgfpoint{27.565644\du}{10.237258\du}}
\pgfpathlineto{\pgfpoint{27.577693\du}{10.243100\du}}
\pgfpathlineto{\pgfpoint{27.588280\du}{10.248941\du}}
\pgfpathlineto{\pgfpoint{27.600329\du}{10.254783\du}}
\pgfpathlineto{\pgfpoint{27.610551\du}{10.260624\du}}
\pgfpathlineto{\pgfpoint{27.621504\du}{10.265736\du}}
\pgfpathlineto{\pgfpoint{27.632457\du}{10.272307\du}}
\pgfpathlineto{\pgfpoint{27.642315\du}{10.278149\du}}
\pgfpathlineto{\pgfpoint{27.652172\du}{10.283991\du}}
\pgfpathlineto{\pgfpoint{27.662395\du}{10.289832\du}}
\pgfpathlineto{\pgfpoint{27.671522\du}{10.295674\du}}
\pgfpathlineto{\pgfpoint{27.681745\du}{10.302245\du}}
\pgfpathlineto{\pgfpoint{27.691238\du}{10.308087\du}}
\pgfpathlineto{\pgfpoint{27.700000\du}{10.314294\du}}
\pgfpathlineto{\pgfpoint{27.708032\du}{10.320135\du}}
\pgfpathlineto{\pgfpoint{27.716794\du}{10.326707\du}}
\pgfpathlineto{\pgfpoint{27.725192\du}{10.332913\du}}
\pgfpathlineto{\pgfpoint{27.733954\du}{10.339485\du}}
\pgfpathlineto{\pgfpoint{27.741621\du}{10.346057\du}}
\pgfpathlineto{\pgfpoint{27.749653\du}{10.351899\du}}
\pgfpathlineto{\pgfpoint{27.756955\du}{10.358105\du}}
\pgfpathlineto{\pgfpoint{27.764622\du}{10.365407\du}}
\pgfpathlineto{\pgfpoint{27.771194\du}{10.371249\du}}
\pgfpathlineto{\pgfpoint{27.778496\du}{10.378185\du}}
\pgfpathlineto{\pgfpoint{27.784337\du}{10.384757\du}}
\pgfpathlineto{\pgfpoint{27.791274\du}{10.390964\du}}
\pgfpathlineto{\pgfpoint{27.797846\du}{10.397536\du}}
\pgfpathlineto{\pgfpoint{27.802957\du}{10.404472\du}}
\pgfpathlineto{\pgfpoint{27.808434\du}{10.411044\du}}
\pgfpathlineto{\pgfpoint{27.814640\du}{10.417981\du}}
\pgfpathlineto{\pgfpoint{27.819387\du}{10.424188\du}}
\pgfpathlineto{\pgfpoint{27.824498\du}{10.430759\du}}
\pgfpathlineto{\pgfpoint{27.829244\du}{10.437696\du}}
\pgfpathlineto{\pgfpoint{27.833625\du}{10.444268\du}}
\pgfpathlineto{\pgfpoint{27.838007\du}{10.451205\du}}
\pgfpathlineto{\pgfpoint{27.840927\du}{10.457777\du}}
\pgfpathlineto{\pgfpoint{27.844943\du}{10.464713\du}}
\pgfpathlineto{\pgfpoint{27.847864\du}{10.471650\du}}
\pgfpathlineto{\pgfpoint{27.851880\du}{10.478952\du}}
\pgfpathlineto{\pgfpoint{27.854436\du}{10.485159\du}}
\pgfpathlineto{\pgfpoint{27.857357\du}{10.492096\du}}
\pgfpathlineto{\pgfpoint{27.859912\du}{10.499398\du}}
\pgfpathlineto{\pgfpoint{27.862103\du}{10.506334\du}}
\pgfpathlineto{\pgfpoint{27.863563\du}{10.513271\du}}
\pgfpathlineto{\pgfpoint{27.865754\du}{10.520208\du}}
\pgfpathlineto{\pgfpoint{27.866484\du}{10.527510\du}}
\pgfpathlineto{\pgfpoint{27.867579\du}{10.533717\du}}
\pgfpathlineto{\pgfpoint{27.869405\du}{10.541384\du}}
\pgfpathlineto{\pgfpoint{27.869770\du}{10.548321\du}}
\pgfpathlineto{\pgfpoint{27.869770\du}{10.555257\du}}
\pgfpathlineto{\pgfpoint{27.870500\du}{10.562559\du}}
\pgfpathlineto{\pgfpoint{27.890581\du}{10.562559\du}}
\pgfusepath{fill}
\pgfsetbuttcap
\pgfsetmiterjoin
\pgfsetdash{}{0pt}
\definecolor{dialinecolor}{rgb}{0.027451, 0.486275, 0.682353}
\pgfsetfillcolor{dialinecolor}
\pgfpathmoveto{\pgfpoint{24.514531\du}{9.753140\du}}
\pgfpathlineto{\pgfpoint{24.514531\du}{10.577528\du}}
\pgfpathlineto{\pgfpoint{27.879993\du}{10.577528\du}}
\pgfpathlineto{\pgfpoint{27.880723\du}{9.753870\du}}
\pgfpathlineto{\pgfpoint{24.514531\du}{9.753140\du}}
\pgfusepath{fill}
\pgfsetbuttcap
\pgfsetmiterjoin
\pgfsetdash{}{0pt}
\definecolor{dialinecolor}{rgb}{0.235294, 0.686275, 0.894118}
\pgfsetfillcolor{dialinecolor}
\pgfpathmoveto{\pgfpoint{27.879993\du}{9.737441\du}}
\pgfpathlineto{\pgfpoint{27.878532\du}{9.767379\du}}
\pgfpathlineto{\pgfpoint{27.871230\du}{9.796586\du}}
\pgfpathlineto{\pgfpoint{27.861008\du}{9.825794\du}}
\pgfpathlineto{\pgfpoint{27.846404\du}{9.853907\du}}
\pgfpathlineto{\pgfpoint{27.827054\du}{9.882019\du}}
\pgfpathlineto{\pgfpoint{27.805148\du}{9.909401\du}}
\pgfpathlineto{\pgfpoint{27.778496\du}{9.935688\du}}
\pgfpathlineto{\pgfpoint{27.747828\du}{9.961975\du}}
\pgfpathlineto{\pgfpoint{27.714969\du}{9.987897\du}}
\pgfpathlineto{\pgfpoint{27.677729\du}{10.013089\du}}
\pgfpathlineto{\pgfpoint{27.636838\du}{10.037185\du}}
\pgfpathlineto{\pgfpoint{27.593027\du}{10.060551\du}}
\pgfpathlineto{\pgfpoint{27.546659\du}{10.082822\du}}
\pgfpathlineto{\pgfpoint{27.496276\du}{10.104728\du}}
\pgfpathlineto{\pgfpoint{27.443337\du}{10.125904\du}}
\pgfpathlineto{\pgfpoint{27.387842\du}{10.145984\du}}
\pgfpathlineto{\pgfpoint{27.329792\du}{10.164604\du}}
\pgfpathlineto{\pgfpoint{27.269186\du}{10.183224\du}}
\pgfpathlineto{\pgfpoint{27.205294\du}{10.200383\du}}
\pgfpathlineto{\pgfpoint{27.140307\du}{10.216083\du}}
\pgfpathlineto{\pgfpoint{27.071668\du}{10.231417\du}}
\pgfpathlineto{\pgfpoint{27.000840\du}{10.245290\du}}
\pgfpathlineto{\pgfpoint{26.928916\du}{10.258069\du}}
\pgfpathlineto{\pgfpoint{26.854071\du}{10.269387\du}}
\pgfpathlineto{\pgfpoint{26.778131\du}{10.279974\du}}
\pgfpathlineto{\pgfpoint{26.699635\du}{10.289102\du}}
\pgfpathlineto{\pgfpoint{26.620044\du}{10.296769\du}}
\pgfpathlineto{\pgfpoint{26.538992\du}{10.303341\du}}
\pgfpathlineto{\pgfpoint{26.455750\du}{10.308452\du}}
\pgfpathlineto{\pgfpoint{26.372143\du}{10.312103\du}}
\pgfpathlineto{\pgfpoint{26.286710\du}{10.313928\du}}
\pgfpathlineto{\pgfpoint{26.200183\du}{10.315024\du}}
\pgfpathlineto{\pgfpoint{26.114020\du}{10.313928\du}}
\pgfpathlineto{\pgfpoint{26.028222\du}{10.312103\du}}
\pgfpathlineto{\pgfpoint{25.944615\du}{10.308452\du}}
\pgfpathlineto{\pgfpoint{25.861738\du}{10.303341\du}}
\pgfpathlineto{\pgfpoint{25.780321\du}{10.296769\du}}
\pgfpathlineto{\pgfpoint{25.700730\du}{10.289102\du}}
\pgfpathlineto{\pgfpoint{25.622965\du}{10.279974\du}}
\pgfpathlineto{\pgfpoint{25.546294\du}{10.269387\du}}
\pgfpathlineto{\pgfpoint{25.472180\du}{10.258069\du}}
\pgfpathlineto{\pgfpoint{25.399525\du}{10.245290\du}}
\pgfpathlineto{\pgfpoint{25.329062\du}{10.231417\du}}
\pgfpathlineto{\pgfpoint{25.260424\du}{10.216083\du}}
\pgfpathlineto{\pgfpoint{25.194706\du}{10.200383\du}}
\pgfpathlineto{\pgfpoint{25.131179\du}{10.183224\du}}
\pgfpathlineto{\pgfpoint{25.070208\du}{10.164604\du}}
\pgfpathlineto{\pgfpoint{25.011793\du}{10.145984\du}}
\pgfpathlineto{\pgfpoint{24.956663\du}{10.125904\du}}
\pgfpathlineto{\pgfpoint{24.903724\du}{10.104728\du}}
\pgfpathlineto{\pgfpoint{24.853706\du}{10.082822\du}}
\pgfpathlineto{\pgfpoint{24.806608\du}{10.060551\du}}
\pgfpathlineto{\pgfpoint{24.763162\du}{10.037185\du}}
\pgfpathlineto{\pgfpoint{24.722271\du}{10.013089\du}}
\pgfpathlineto{\pgfpoint{24.685031\du}{9.987897\du}}
\pgfpathlineto{\pgfpoint{24.651807\du}{9.961975\du}}
\pgfpathlineto{\pgfpoint{24.621504\du}{9.935688\du}}
\pgfpathlineto{\pgfpoint{24.594852\du}{9.909401\du}}
\pgfpathlineto{\pgfpoint{24.572946\du}{9.882019\du}}
\pgfpathlineto{\pgfpoint{24.553596\du}{9.853907\du}}
\pgfpathlineto{\pgfpoint{24.538992\du}{9.825794\du}}
\pgfpathlineto{\pgfpoint{24.528405\du}{9.796586\du}}
\pgfpathlineto{\pgfpoint{24.521468\du}{9.767379\du}}
\pgfpathlineto{\pgfpoint{24.519642\du}{9.737441\du}}
\pgfpathlineto{\pgfpoint{24.521468\du}{9.708233\du}}
\pgfpathlineto{\pgfpoint{24.528405\du}{9.678295\du}}
\pgfpathlineto{\pgfpoint{24.538992\du}{9.649817\du}}
\pgfpathlineto{\pgfpoint{24.553596\du}{9.621705\du}}
\pgfpathlineto{\pgfpoint{24.572946\du}{9.593593\du}}
\pgfpathlineto{\pgfpoint{24.594852\du}{9.565845\du}}
\pgfpathlineto{\pgfpoint{24.621504\du}{9.539193\du}}
\pgfpathlineto{\pgfpoint{24.651807\du}{9.512906\du}}
\pgfpathlineto{\pgfpoint{24.685031\du}{9.487714\du}}
\pgfpathlineto{\pgfpoint{24.722271\du}{9.462523\du}}
\pgfpathlineto{\pgfpoint{24.763162\du}{9.438426\du}}
\pgfpathlineto{\pgfpoint{24.806608\du}{9.415060\du}}
\pgfpathlineto{\pgfpoint{24.853706\du}{9.392059\du}}
\pgfpathlineto{\pgfpoint{24.903724\du}{9.370518\du}}
\pgfpathlineto{\pgfpoint{24.956663\du}{9.349343\du}}
\pgfpathlineto{\pgfpoint{25.011793\du}{9.329628\du}}
\pgfpathlineto{\pgfpoint{25.070208\du}{9.310277\du}}
\pgfpathlineto{\pgfpoint{25.131179\du}{9.292023\du}}
\pgfpathlineto{\pgfpoint{25.194706\du}{9.275228\du}}
\pgfpathlineto{\pgfpoint{25.260424\du}{9.258799\du}}
\pgfpathlineto{\pgfpoint{25.329062\du}{9.243465\du}}
\pgfpathlineto{\pgfpoint{25.399525\du}{9.229956\du}}
\pgfpathlineto{\pgfpoint{25.472180\du}{9.217178\du}}
\pgfpathlineto{\pgfpoint{25.546294\du}{9.205495\du}}
\pgfpathlineto{\pgfpoint{25.622965\du}{9.195637\du}}
\pgfpathlineto{\pgfpoint{25.700730\du}{9.186145\du}}
\pgfpathlineto{\pgfpoint{25.780321\du}{9.178478\du}}
\pgfpathlineto{\pgfpoint{25.861738\du}{9.171906\du}}
\pgfpathlineto{\pgfpoint{25.944615\du}{9.166794\du}}
\pgfpathlineto{\pgfpoint{26.028222\du}{9.163509\du}}
\pgfpathlineto{\pgfpoint{26.114020\du}{9.160953\du}}
\pgfpathlineto{\pgfpoint{26.200183\du}{9.160588\du}}
\pgfpathlineto{\pgfpoint{26.286710\du}{9.160953\du}}
\pgfpathlineto{\pgfpoint{26.372143\du}{9.163509\du}}
\pgfpathlineto{\pgfpoint{26.455750\du}{9.166794\du}}
\pgfpathlineto{\pgfpoint{26.538992\du}{9.171906\du}}
\pgfpathlineto{\pgfpoint{26.620044\du}{9.178478\du}}
\pgfpathlineto{\pgfpoint{26.699635\du}{9.186145\du}}
\pgfpathlineto{\pgfpoint{26.778131\du}{9.195637\du}}
\pgfpathlineto{\pgfpoint{26.854071\du}{9.205495\du}}
\pgfpathlineto{\pgfpoint{26.928916\du}{9.217178\du}}
\pgfpathlineto{\pgfpoint{27.000840\du}{9.229956\du}}
\pgfpathlineto{\pgfpoint{27.071668\du}{9.243465\du}}
\pgfpathlineto{\pgfpoint{27.140307\du}{9.258799\du}}
\pgfpathlineto{\pgfpoint{27.205294\du}{9.275228\du}}
\pgfpathlineto{\pgfpoint{27.269186\du}{9.292023\du}}
\pgfpathlineto{\pgfpoint{27.329792\du}{9.310277\du}}
\pgfpathlineto{\pgfpoint{27.387842\du}{9.329628\du}}
\pgfpathlineto{\pgfpoint{27.443337\du}{9.349343\du}}
\pgfpathlineto{\pgfpoint{27.496276\du}{9.370518\du}}
\pgfpathlineto{\pgfpoint{27.546659\du}{9.392059\du}}
\pgfpathlineto{\pgfpoint{27.593027\du}{9.415060\du}}
\pgfpathlineto{\pgfpoint{27.636838\du}{9.438426\du}}
\pgfpathlineto{\pgfpoint{27.677729\du}{9.462523\du}}
\pgfpathlineto{\pgfpoint{27.714969\du}{9.487714\du}}
\pgfpathlineto{\pgfpoint{27.747828\du}{9.512906\du}}
\pgfpathlineto{\pgfpoint{27.778496\du}{9.539193\du}}
\pgfpathlineto{\pgfpoint{27.805148\du}{9.565845\du}}
\pgfpathlineto{\pgfpoint{27.827054\du}{9.593593\du}}
\pgfpathlineto{\pgfpoint{27.846404\du}{9.621705\du}}
\pgfpathlineto{\pgfpoint{27.861008\du}{9.649817\du}}
\pgfpathlineto{\pgfpoint{27.871230\du}{9.678295\du}}
\pgfpathlineto{\pgfpoint{27.878532\du}{9.708233\du}}
\pgfpathlineto{\pgfpoint{27.879993\du}{9.737441\du}}
\pgfusepath{fill}
\pgfsetbuttcap
\pgfsetmiterjoin
\pgfsetdash{}{0pt}
\definecolor{dialinecolor}{rgb}{0.678431, 0.839216, 0.905882}
\pgfsetfillcolor{dialinecolor}
\pgfpathmoveto{\pgfpoint{26.200183\du}{10.324881\du}}
\pgfpathlineto{\pgfpoint{26.200183\du}{10.324881\du}}
\pgfpathlineto{\pgfpoint{26.243629\du}{10.324881\du}}
\pgfpathlineto{\pgfpoint{26.287076\du}{10.324151\du}}
\pgfpathlineto{\pgfpoint{26.330157\du}{10.323056\du}}
\pgfpathlineto{\pgfpoint{26.372143\du}{10.321961\du}}
\pgfpathlineto{\pgfpoint{26.414859\du}{10.320135\du}}
\pgfpathlineto{\pgfpoint{26.456480\du}{10.318310\du}}
\pgfpathlineto{\pgfpoint{26.498101\du}{10.316119\du}}
\pgfpathlineto{\pgfpoint{26.539723\du}{10.313198\du}}
\pgfpathlineto{\pgfpoint{26.580248\du}{10.310277\du}}
\pgfpathlineto{\pgfpoint{26.621139\du}{10.307357\du}}
\pgfpathlineto{\pgfpoint{26.660935\du}{10.303341\du}}
\pgfpathlineto{\pgfpoint{26.701095\du}{10.299325\du}}
\pgfpathlineto{\pgfpoint{26.739796\du}{10.294578\du}}
\pgfpathlineto{\pgfpoint{26.779226\du}{10.289832\du}}
\pgfpathlineto{\pgfpoint{26.817196\du}{10.285086\du}}
\pgfpathlineto{\pgfpoint{26.856261\du}{10.279974\du}}
\pgfpathlineto{\pgfpoint{26.893501\du}{10.274133\du}}
\pgfpathlineto{\pgfpoint{26.930376\du}{10.268291\du}}
\pgfpathlineto{\pgfpoint{26.966886\du}{10.261720\du}}
\pgfpathlineto{\pgfpoint{27.003395\du}{10.255148\du}}
\pgfpathlineto{\pgfpoint{27.038810\du}{10.248211\du}}
\pgfpathlineto{\pgfpoint{27.073494\du}{10.241274\du}}
\pgfpathlineto{\pgfpoint{27.108178\du}{10.234337\du}}
\pgfpathlineto{\pgfpoint{27.142132\du}{10.226670\du}}
\pgfpathlineto{\pgfpoint{27.175721\du}{10.218273\du}}
\pgfpathlineto{\pgfpoint{27.208580\du}{10.210241\du}}
\pgfpathlineto{\pgfpoint{27.240343\du}{10.201844\du}}
\pgfpathlineto{\pgfpoint{27.271742\du}{10.192716\du}}
\pgfpathlineto{\pgfpoint{27.287076\du}{10.188700\du}}
\pgfpathlineto{\pgfpoint{27.302410\du}{10.183954\du}}
\pgfpathlineto{\pgfpoint{27.318474\du}{10.179208\du}}
\pgfpathlineto{\pgfpoint{27.333078\du}{10.174461\du}}
\pgfpathlineto{\pgfpoint{27.347317\du}{10.169715\du}}
\pgfpathlineto{\pgfpoint{27.362286\du}{10.165334\du}}
\pgfpathlineto{\pgfpoint{27.377254\du}{10.160588\du}}
\pgfpathlineto{\pgfpoint{27.391128\du}{10.155111\du}}
\pgfpathlineto{\pgfpoint{27.405367\du}{10.150365\du}}
\pgfpathlineto{\pgfpoint{27.419241\du}{10.145254\du}}
\pgfpathlineto{\pgfpoint{27.433479\du}{10.140507\du}}
\pgfpathlineto{\pgfpoint{27.446623\du}{10.135396\du}}
\pgfpathlineto{\pgfpoint{27.460862\du}{10.129920\du}}
\pgfpathlineto{\pgfpoint{27.474005\du}{10.124808\du}}
\pgfpathlineto{\pgfpoint{27.487149\du}{10.119697\du}}
\pgfpathlineto{\pgfpoint{27.500657\du}{10.114221\du}}
\pgfpathlineto{\pgfpoint{27.513436\du}{10.109109\du}}
\pgfpathlineto{\pgfpoint{27.525484\du}{10.103633\du}}
\pgfpathlineto{\pgfpoint{27.538262\du}{10.097791\du}}
\pgfpathlineto{\pgfpoint{27.550675\du}{10.091950\du}}
\pgfpathlineto{\pgfpoint{27.563089\du}{10.086838\du}}
\pgfpathlineto{\pgfpoint{27.574407\du}{10.080997\du}}
\pgfpathlineto{\pgfpoint{27.586455\du}{10.075155\du}}
\pgfpathlineto{\pgfpoint{27.597408\du}{10.069314\du}}
\pgfpathlineto{\pgfpoint{27.609091\du}{10.063837\du}}
\pgfpathlineto{\pgfpoint{27.620409\du}{10.057996\du}}
\pgfpathlineto{\pgfpoint{27.631727\du}{10.051789\du}}
\pgfpathlineto{\pgfpoint{27.642315\du}{10.045947\du}}
\pgfpathlineto{\pgfpoint{27.652172\du}{10.040106\du}}
\pgfpathlineto{\pgfpoint{27.662760\du}{10.034264\du}}
\pgfpathlineto{\pgfpoint{27.672983\du}{10.027693\du}}
\pgfpathlineto{\pgfpoint{27.682840\du}{10.021121\du}}
\pgfpathlineto{\pgfpoint{27.692698\du}{10.015279\du}}
\pgfpathlineto{\pgfpoint{27.701825\du}{10.009073\du}}
\pgfpathlineto{\pgfpoint{27.710953\du}{10.002501\du}}
\pgfpathlineto{\pgfpoint{27.720445\du}{9.995929\du}}
\pgfpathlineto{\pgfpoint{27.729208\du}{9.989723\du}}
\pgfpathlineto{\pgfpoint{27.738335\du}{9.983151\du}}
\pgfpathlineto{\pgfpoint{27.746367\du}{9.976579\du}}
\pgfpathlineto{\pgfpoint{27.755130\du}{9.970372\du}}
\pgfpathlineto{\pgfpoint{27.762432\du}{9.963801\du}}
\pgfpathlineto{\pgfpoint{27.770464\du}{9.956864\du}}
\pgfpathlineto{\pgfpoint{27.778496\du}{9.950292\du}}
\pgfpathlineto{\pgfpoint{27.785068\du}{9.943355\du}}
\pgfpathlineto{\pgfpoint{27.792369\du}{9.936783\du}}
\pgfpathlineto{\pgfpoint{27.798941\du}{9.929847\du}}
\pgfpathlineto{\pgfpoint{27.805878\du}{9.922910\du}}
\pgfpathlineto{\pgfpoint{27.812450\du}{9.916338\du}}
\pgfpathlineto{\pgfpoint{27.818656\du}{9.909401\du}}
\pgfpathlineto{\pgfpoint{27.824498\du}{9.902464\du}}
\pgfpathlineto{\pgfpoint{27.829974\du}{9.895528\du}}
\pgfpathlineto{\pgfpoint{27.835816\du}{9.887861\du}}
\pgfpathlineto{\pgfpoint{27.840562\du}{9.880924\du}}
\pgfpathlineto{\pgfpoint{27.846039\du}{9.873622\du}}
\pgfpathlineto{\pgfpoint{27.850785\du}{9.866685\du}}
\pgfpathlineto{\pgfpoint{27.855166\du}{9.859018\du}}
\pgfpathlineto{\pgfpoint{27.859182\du}{9.852081\du}}
\pgfpathlineto{\pgfpoint{27.863198\du}{9.844414\du}}
\pgfpathlineto{\pgfpoint{27.866484\du}{9.836747\du}}
\pgfpathlineto{\pgfpoint{27.870500\du}{9.829445\du}}
\pgfpathlineto{\pgfpoint{27.873786\du}{9.822143\du}}
\pgfpathlineto{\pgfpoint{27.876342\du}{9.814841\du}}
\pgfpathlineto{\pgfpoint{27.879263\du}{9.807174\du}}
\pgfpathlineto{\pgfpoint{27.881088\du}{9.799507\du}}
\pgfpathlineto{\pgfpoint{27.884009\du}{9.791840\du}}
\pgfpathlineto{\pgfpoint{27.885104\du}{9.784173\du}}
\pgfpathlineto{\pgfpoint{27.887295\du}{9.776506\du}}
\pgfpathlineto{\pgfpoint{27.888390\du}{9.769204\du}}
\pgfpathlineto{\pgfpoint{27.889120\du}{9.760807\du}}
\pgfpathlineto{\pgfpoint{27.889850\du}{9.753140\du}}
\pgfpathlineto{\pgfpoint{27.890581\du}{9.745473\du}}
\pgfpathlineto{\pgfpoint{27.890581\du}{9.737441\du}}
\pgfpathlineto{\pgfpoint{27.870500\du}{9.737441\du}}
\pgfpathlineto{\pgfpoint{27.869770\du}{9.744378\du}}
\pgfpathlineto{\pgfpoint{27.869770\du}{9.752045\du}}
\pgfpathlineto{\pgfpoint{27.869405\du}{9.758981\du}}
\pgfpathlineto{\pgfpoint{27.867579\du}{9.765918\du}}
\pgfpathlineto{\pgfpoint{27.866484\du}{9.773220\du}}
\pgfpathlineto{\pgfpoint{27.865754\du}{9.779427\du}}
\pgfpathlineto{\pgfpoint{27.863563\du}{9.786729\du}}
\pgfpathlineto{\pgfpoint{27.862103\du}{9.793666\du}}
\pgfpathlineto{\pgfpoint{27.859912\du}{9.800602\du}}
\pgfpathlineto{\pgfpoint{27.857357\du}{9.807539\du}}
\pgfpathlineto{\pgfpoint{27.854436\du}{9.814841\du}}
\pgfpathlineto{\pgfpoint{27.851880\du}{9.821048\du}}
\pgfpathlineto{\pgfpoint{27.847864\du}{9.827985\du}}
\pgfpathlineto{\pgfpoint{27.844943\du}{9.835287\du}}
\pgfpathlineto{\pgfpoint{27.840927\du}{9.842223\du}}
\pgfpathlineto{\pgfpoint{27.838007\du}{9.848430\du}}
\pgfpathlineto{\pgfpoint{27.833625\du}{9.855732\du}}
\pgfpathlineto{\pgfpoint{27.829244\du}{9.861939\du}}
\pgfpathlineto{\pgfpoint{27.824498\du}{9.869241\du}}
\pgfpathlineto{\pgfpoint{27.819387\du}{9.875447\du}}
\pgfpathlineto{\pgfpoint{27.814640\du}{9.882384\du}}
\pgfpathlineto{\pgfpoint{27.808434\du}{9.888956\du}}
\pgfpathlineto{\pgfpoint{27.802957\du}{9.895528\du}}
\pgfpathlineto{\pgfpoint{27.797846\du}{9.902464\du}}
\pgfpathlineto{\pgfpoint{27.791274\du}{9.909036\du}}
\pgfpathlineto{\pgfpoint{27.784337\du}{9.915243\du}}
\pgfpathlineto{\pgfpoint{27.778496\du}{9.921815\du}}
\pgfpathlineto{\pgfpoint{27.771194\du}{9.928751\du}}
\pgfpathlineto{\pgfpoint{27.764622\du}{9.935323\du}}
\pgfpathlineto{\pgfpoint{27.756955\du}{9.941530\du}}
\pgfpathlineto{\pgfpoint{27.749653\du}{9.948101\du}}
\pgfpathlineto{\pgfpoint{27.741621\du}{9.953943\du}}
\pgfpathlineto{\pgfpoint{27.733954\du}{9.960515\du}}
\pgfpathlineto{\pgfpoint{27.725192\du}{9.966721\du}}
\pgfpathlineto{\pgfpoint{27.716794\du}{9.973293\du}}
\pgfpathlineto{\pgfpoint{27.708032\du}{9.979500\du}}
\pgfpathlineto{\pgfpoint{27.700000\du}{9.985341\du}}
\pgfpathlineto{\pgfpoint{27.691238\du}{9.991913\du}}
\pgfpathlineto{\pgfpoint{27.681745\du}{9.997755\du}}
\pgfpathlineto{\pgfpoint{27.671522\du}{10.004326\du}}
\pgfpathlineto{\pgfpoint{27.662395\du}{10.010168\du}}
\pgfpathlineto{\pgfpoint{27.652172\du}{10.016009\du}}
\pgfpathlineto{\pgfpoint{27.642315\du}{10.021851\du}}
\pgfpathlineto{\pgfpoint{27.632457\du}{10.027693\du}}
\pgfpathlineto{\pgfpoint{27.621504\du}{10.034264\du}}
\pgfpathlineto{\pgfpoint{27.610551\du}{10.039376\du}}
\pgfpathlineto{\pgfpoint{27.600329\du}{10.045217\du}}
\pgfpathlineto{\pgfpoint{27.588280\du}{10.051059\du}}
\pgfpathlineto{\pgfpoint{27.577693\du}{10.056900\du}}
\pgfpathlineto{\pgfpoint{27.565644\du}{10.062742\du}}
\pgfpathlineto{\pgfpoint{27.554326\du}{10.067853\du}}
\pgfpathlineto{\pgfpoint{27.541913\du}{10.073330\du}}
\pgfpathlineto{\pgfpoint{27.529865\du}{10.079171\du}}
\pgfpathlineto{\pgfpoint{27.517452\du}{10.084283\du}}
\pgfpathlineto{\pgfpoint{27.505403\du}{10.089759\du}}
\pgfpathlineto{\pgfpoint{27.492260\du}{10.094870\du}}
\pgfpathlineto{\pgfpoint{27.479482\du}{10.100347\du}}
\pgfpathlineto{\pgfpoint{27.466703\du}{10.106188\du}}
\pgfpathlineto{\pgfpoint{27.453560\du}{10.110570\du}}
\pgfpathlineto{\pgfpoint{27.440051\du}{10.116046\du}}
\pgfpathlineto{\pgfpoint{27.426908\du}{10.121157\du}}
\pgfpathlineto{\pgfpoint{27.412669\du}{10.125904\du}}
\pgfpathlineto{\pgfpoint{27.398795\du}{10.131380\du}}
\pgfpathlineto{\pgfpoint{27.384556\du}{10.135761\du}}
\pgfpathlineto{\pgfpoint{27.369953\du}{10.140507\du}}
\pgfpathlineto{\pgfpoint{27.356079\du}{10.145984\du}}
\pgfpathlineto{\pgfpoint{27.341475\du}{10.150365\du}}
\pgfpathlineto{\pgfpoint{27.327236\du}{10.155111\du}}
\pgfpathlineto{\pgfpoint{27.311537\du}{10.159858\du}}
\pgfpathlineto{\pgfpoint{27.297298\du}{10.163874\du}}
\pgfpathlineto{\pgfpoint{27.281599\du}{10.168620\du}}
\pgfpathlineto{\pgfpoint{27.266630\du}{10.173366\du}}
\pgfpathlineto{\pgfpoint{27.235232\du}{10.181398\du}}
\pgfpathlineto{\pgfpoint{27.203103\du}{10.190161\du}}
\pgfpathlineto{\pgfpoint{27.170975\du}{10.198558\du}}
\pgfpathlineto{\pgfpoint{27.137751\du}{10.206225\du}}
\pgfpathlineto{\pgfpoint{27.103432\du}{10.213892\du}}
\pgfpathlineto{\pgfpoint{27.069478\du}{10.221194\du}}
\pgfpathlineto{\pgfpoint{27.034794\du}{10.228496\du}}
\pgfpathlineto{\pgfpoint{26.999014\du}{10.235433\du}}
\pgfpathlineto{\pgfpoint{26.963600\du}{10.241639\du}}
\pgfpathlineto{\pgfpoint{26.926725\du}{10.247481\du}}
\pgfpathlineto{\pgfpoint{26.890215\du}{10.253687\du}}
\pgfpathlineto{\pgfpoint{26.852976\du}{10.259529\du}}
\pgfpathlineto{\pgfpoint{26.815371\du}{10.264640\du}}
\pgfpathlineto{\pgfpoint{26.776670\du}{10.269752\du}}
\pgfpathlineto{\pgfpoint{26.737970\du}{10.274498\du}}
\pgfpathlineto{\pgfpoint{26.698540\du}{10.278514\du}}
\pgfpathlineto{\pgfpoint{26.659474\du}{10.282895\du}}
\pgfpathlineto{\pgfpoint{26.619314\du}{10.286181\du}}
\pgfpathlineto{\pgfpoint{26.579153\du}{10.289832\du}}
\pgfpathlineto{\pgfpoint{26.537897\du}{10.292753\du}}
\pgfpathlineto{\pgfpoint{26.497006\du}{10.295674\du}}
\pgfpathlineto{\pgfpoint{26.455750\du}{10.297864\du}}
\pgfpathlineto{\pgfpoint{26.414129\du}{10.300420\du}}
\pgfpathlineto{\pgfpoint{26.371413\du}{10.301515\du}}
\pgfpathlineto{\pgfpoint{26.328697\du}{10.303341\du}}
\pgfpathlineto{\pgfpoint{26.286710\du}{10.303706\du}}
\pgfpathlineto{\pgfpoint{26.243264\du}{10.304436\du}}
\pgfpathlineto{\pgfpoint{26.200183\du}{10.304436\du}}
\pgfpathlineto{\pgfpoint{26.200183\du}{10.304436\du}}
\pgfpathlineto{\pgfpoint{26.200183\du}{10.304436\du}}
\pgfpathlineto{\pgfpoint{26.199452\du}{10.305166\du}}
\pgfpathlineto{\pgfpoint{26.197627\du}{10.305166\du}}
\pgfpathlineto{\pgfpoint{26.196532\du}{10.305166\du}}
\pgfpathlineto{\pgfpoint{26.195801\du}{10.305531\du}}
\pgfpathlineto{\pgfpoint{26.195436\du}{10.306261\du}}
\pgfpathlineto{\pgfpoint{26.193976\du}{10.306261\du}}
\pgfpathlineto{\pgfpoint{26.193246\du}{10.307357\du}}
\pgfpathlineto{\pgfpoint{26.192516\du}{10.308087\du}}
\pgfpathlineto{\pgfpoint{26.191420\du}{10.309182\du}}
\pgfpathlineto{\pgfpoint{26.190690\du}{10.311008\du}}
\pgfpathlineto{\pgfpoint{26.190690\du}{10.313198\du}}
\pgfpathlineto{\pgfpoint{26.189960\du}{10.315024\du}}
\pgfpathlineto{\pgfpoint{26.190690\du}{10.316849\du}}
\pgfpathlineto{\pgfpoint{26.190690\du}{10.318310\du}}
\pgfpathlineto{\pgfpoint{26.191420\du}{10.320135\du}}
\pgfpathlineto{\pgfpoint{26.192516\du}{10.321961\du}}
\pgfpathlineto{\pgfpoint{26.193246\du}{10.322691\du}}
\pgfpathlineto{\pgfpoint{26.193976\du}{10.323056\du}}
\pgfpathlineto{\pgfpoint{26.195436\du}{10.323786\du}}
\pgfpathlineto{\pgfpoint{26.195801\du}{10.324151\du}}
\pgfpathlineto{\pgfpoint{26.196532\du}{10.324881\du}}
\pgfpathlineto{\pgfpoint{26.197627\du}{10.324881\du}}
\pgfpathlineto{\pgfpoint{26.199452\du}{10.324881\du}}
\pgfpathlineto{\pgfpoint{26.200183\du}{10.324881\du}}
\pgfusepath{fill}
\pgfsetbuttcap
\pgfsetmiterjoin
\pgfsetdash{}{0pt}
\definecolor{dialinecolor}{rgb}{0.678431, 0.839216, 0.905882}
\pgfsetfillcolor{dialinecolor}
\pgfpathmoveto{\pgfpoint{24.509419\du}{9.737441\du}}
\pgfpathlineto{\pgfpoint{24.509419\du}{9.737441\du}}
\pgfpathlineto{\pgfpoint{24.509419\du}{9.745473\du}}
\pgfpathlineto{\pgfpoint{24.509785\du}{9.753140\du}}
\pgfpathlineto{\pgfpoint{24.510515\du}{9.760807\du}}
\pgfpathlineto{\pgfpoint{24.511610\du}{9.769204\du}}
\pgfpathlineto{\pgfpoint{24.512705\du}{9.776506\du}}
\pgfpathlineto{\pgfpoint{24.514531\du}{9.784173\du}}
\pgfpathlineto{\pgfpoint{24.516356\du}{9.791840\du}}
\pgfpathlineto{\pgfpoint{24.518547\du}{9.799507\du}}
\pgfpathlineto{\pgfpoint{24.520737\du}{9.807174\du}}
\pgfpathlineto{\pgfpoint{24.523293\du}{9.814841\du}}
\pgfpathlineto{\pgfpoint{24.526214\du}{9.822143\du}}
\pgfpathlineto{\pgfpoint{24.529865\du}{9.829445\du}}
\pgfpathlineto{\pgfpoint{24.533151\du}{9.836747\du}}
\pgfpathlineto{\pgfpoint{24.536802\du}{9.844414\du}}
\pgfpathlineto{\pgfpoint{24.541183\du}{9.852081\du}}
\pgfpathlineto{\pgfpoint{24.544834\du}{9.859018\du}}
\pgfpathlineto{\pgfpoint{24.549945\du}{9.866685\du}}
\pgfpathlineto{\pgfpoint{24.553961\du}{9.873622\du}}
\pgfpathlineto{\pgfpoint{24.559438\du}{9.880924\du}}
\pgfpathlineto{\pgfpoint{24.564184\du}{9.887861\du}}
\pgfpathlineto{\pgfpoint{24.569660\du}{9.895528\du}}
\pgfpathlineto{\pgfpoint{24.575502\du}{9.902464\du}}
\pgfpathlineto{\pgfpoint{24.581344\du}{9.909401\du}}
\pgfpathlineto{\pgfpoint{24.587185\du}{9.916338\du}}
\pgfpathlineto{\pgfpoint{24.594122\du}{9.922910\du}}
\pgfpathlineto{\pgfpoint{24.600694\du}{9.929847\du}}
\pgfpathlineto{\pgfpoint{24.607631\du}{9.936783\du}}
\pgfpathlineto{\pgfpoint{24.614567\du}{9.943355\du}}
\pgfpathlineto{\pgfpoint{24.621504\du}{9.950292\du}}
\pgfpathlineto{\pgfpoint{24.630267\du}{9.956864\du}}
\pgfpathlineto{\pgfpoint{24.637203\du}{9.963801\du}}
\pgfpathlineto{\pgfpoint{24.644870\du}{9.970372\du}}
\pgfpathlineto{\pgfpoint{24.653633\du}{9.976579\du}}
\pgfpathlineto{\pgfpoint{24.662030\du}{9.983151\du}}
\pgfpathlineto{\pgfpoint{24.670792\du}{9.989723\du}}
\pgfpathlineto{\pgfpoint{24.679189\du}{9.995929\du}}
\pgfpathlineto{\pgfpoint{24.689047\du}{10.002501\du}}
\pgfpathlineto{\pgfpoint{24.698175\du}{10.009073\du}}
\pgfpathlineto{\pgfpoint{24.707667\du}{10.015279\du}}
\pgfpathlineto{\pgfpoint{24.717160\du}{10.021121\du}}
\pgfpathlineto{\pgfpoint{24.727017\du}{10.027693\du}}
\pgfpathlineto{\pgfpoint{24.737240\du}{10.034264\du}}
\pgfpathlineto{\pgfpoint{24.747463\du}{10.040106\du}}
\pgfpathlineto{\pgfpoint{24.758050\du}{10.045947\du}}
\pgfpathlineto{\pgfpoint{24.768273\du}{10.051789\du}}
\pgfpathlineto{\pgfpoint{24.779226\du}{10.057996\du}}
\pgfpathlineto{\pgfpoint{24.790909\du}{10.063837\du}}
\pgfpathlineto{\pgfpoint{24.802227\du}{10.069314\du}}
\pgfpathlineto{\pgfpoint{24.813910\du}{10.075155\du}}
\pgfpathlineto{\pgfpoint{24.825228\du}{10.080997\du}}
\pgfpathlineto{\pgfpoint{24.836911\du}{10.086838\du}}
\pgfpathlineto{\pgfpoint{24.849325\du}{10.091950\du}}
\pgfpathlineto{\pgfpoint{24.861373\du}{10.097791\du}}
\pgfpathlineto{\pgfpoint{24.874516\du}{10.103633\du}}
\pgfpathlineto{\pgfpoint{24.886564\du}{10.109109\du}}
\pgfpathlineto{\pgfpoint{24.899343\du}{10.114221\du}}
\pgfpathlineto{\pgfpoint{24.913217\du}{10.119697\du}}
\pgfpathlineto{\pgfpoint{24.925630\du}{10.124808\du}}
\pgfpathlineto{\pgfpoint{24.938773\du}{10.129920\du}}
\pgfpathlineto{\pgfpoint{24.953012\du}{10.135396\du}}
\pgfpathlineto{\pgfpoint{24.966156\du}{10.140507\du}}
\pgfpathlineto{\pgfpoint{24.980394\du}{10.145254\du}}
\pgfpathlineto{\pgfpoint{24.994268\du}{10.150365\du}}
\pgfpathlineto{\pgfpoint{25.008872\du}{10.155111\du}}
\pgfpathlineto{\pgfpoint{25.022746\du}{10.160588\du}}
\pgfpathlineto{\pgfpoint{25.038445\du}{10.165334\du}}
\pgfpathlineto{\pgfpoint{25.052318\du}{10.169715\du}}
\pgfpathlineto{\pgfpoint{25.066922\du}{10.174461\du}}
\pgfpathlineto{\pgfpoint{25.082256\du}{10.179208\du}}
\pgfpathlineto{\pgfpoint{25.097955\du}{10.183954\du}}
\pgfpathlineto{\pgfpoint{25.112924\du}{10.188700\du}}
\pgfpathlineto{\pgfpoint{25.128624\du}{10.192716\du}}
\pgfpathlineto{\pgfpoint{25.160387\du}{10.201844\du}}
\pgfpathlineto{\pgfpoint{25.192150\du}{10.210241\du}}
\pgfpathlineto{\pgfpoint{25.225374\du}{10.218273\du}}
\pgfpathlineto{\pgfpoint{25.257868\du}{10.226670\du}}
\pgfpathlineto{\pgfpoint{25.292552\du}{10.234337\du}}
\pgfpathlineto{\pgfpoint{25.326871\du}{10.241274\du}}
\pgfpathlineto{\pgfpoint{25.361555\du}{10.248211\du}}
\pgfpathlineto{\pgfpoint{25.397335\du}{10.255148\du}}
\pgfpathlineto{\pgfpoint{25.433479\du}{10.261720\du}}
\pgfpathlineto{\pgfpoint{25.469989\du}{10.268291\du}}
\pgfpathlineto{\pgfpoint{25.507229\du}{10.274133\du}}
\pgfpathlineto{\pgfpoint{25.544834\du}{10.279974\du}}
\pgfpathlineto{\pgfpoint{25.582804\du}{10.285086\du}}
\pgfpathlineto{\pgfpoint{25.621504\du}{10.289832\du}}
\pgfpathlineto{\pgfpoint{25.660204\du}{10.294578\du}}
\pgfpathlineto{\pgfpoint{25.699270\du}{10.299325\du}}
\pgfpathlineto{\pgfpoint{25.739430\du}{10.303341\du}}
\pgfpathlineto{\pgfpoint{25.779226\du}{10.307357\du}}
\pgfpathlineto{\pgfpoint{25.820117\du}{10.310277\du}}
\pgfpathlineto{\pgfpoint{25.861008\du}{10.313198\du}}
\pgfpathlineto{\pgfpoint{25.902264\du}{10.316119\du}}
\pgfpathlineto{\pgfpoint{25.944250\du}{10.318310\du}}
\pgfpathlineto{\pgfpoint{25.985871\du}{10.320135\du}}
\pgfpathlineto{\pgfpoint{26.028222\du}{10.321961\du}}
\pgfpathlineto{\pgfpoint{26.070208\du}{10.323056\du}}
\pgfpathlineto{\pgfpoint{26.113655\du}{10.324151\du}}
\pgfpathlineto{\pgfpoint{26.156371\du}{10.324881\du}}
\pgfpathlineto{\pgfpoint{26.200183\du}{10.324881\du}}
\pgfpathlineto{\pgfpoint{26.200183\du}{10.304436\du}}
\pgfpathlineto{\pgfpoint{26.157466\du}{10.304436\du}}
\pgfpathlineto{\pgfpoint{26.114020\du}{10.303706\du}}
\pgfpathlineto{\pgfpoint{26.071668\du}{10.303341\du}}
\pgfpathlineto{\pgfpoint{26.028952\du}{10.301515\du}}
\pgfpathlineto{\pgfpoint{25.986601\du}{10.300420\du}}
\pgfpathlineto{\pgfpoint{25.944615\du}{10.297864\du}}
\pgfpathlineto{\pgfpoint{25.903724\du}{10.295674\du}}
\pgfpathlineto{\pgfpoint{25.862468\du}{10.292753\du}}
\pgfpathlineto{\pgfpoint{25.821577\du}{10.289832\du}}
\pgfpathlineto{\pgfpoint{25.781417\du}{10.286181\du}}
\pgfpathlineto{\pgfpoint{25.741621\du}{10.282895\du}}
\pgfpathlineto{\pgfpoint{25.702191\du}{10.278514\du}}
\pgfpathlineto{\pgfpoint{25.662760\du}{10.274498\du}}
\pgfpathlineto{\pgfpoint{25.623695\du}{10.269752\du}}
\pgfpathlineto{\pgfpoint{25.585360\du}{10.264640\du}}
\pgfpathlineto{\pgfpoint{25.547755\du}{10.259529\du}}
\pgfpathlineto{\pgfpoint{25.510515\du}{10.253687\du}}
\pgfpathlineto{\pgfpoint{25.474005\du}{10.247481\du}}
\pgfpathlineto{\pgfpoint{25.436765\du}{10.241639\du}}
\pgfpathlineto{\pgfpoint{25.401716\du}{10.235433\du}}
\pgfpathlineto{\pgfpoint{25.365571\du}{10.228496\du}}
\pgfpathlineto{\pgfpoint{25.330887\du}{10.221194\du}}
\pgfpathlineto{\pgfpoint{25.296568\du}{10.213892\du}}
\pgfpathlineto{\pgfpoint{25.262614\du}{10.206225\du}}
\pgfpathlineto{\pgfpoint{25.229390\du}{10.198558\du}}
\pgfpathlineto{\pgfpoint{25.197992\du}{10.190161\du}}
\pgfpathlineto{\pgfpoint{25.165498\du}{10.181398\du}}
\pgfpathlineto{\pgfpoint{25.134465\du}{10.173366\du}}
\pgfpathlineto{\pgfpoint{25.118766\du}{10.168620\du}}
\pgfpathlineto{\pgfpoint{25.103067\du}{10.163874\du}}
\pgfpathlineto{\pgfpoint{25.088463\du}{10.159858\du}}
\pgfpathlineto{\pgfpoint{25.073494\du}{10.155111\du}}
\pgfpathlineto{\pgfpoint{25.058160\du}{10.150365\du}}
\pgfpathlineto{\pgfpoint{25.043921\du}{10.145984\du}}
\pgfpathlineto{\pgfpoint{25.029682\du}{10.140507\du}}
\pgfpathlineto{\pgfpoint{25.015444\du}{10.135761\du}}
\pgfpathlineto{\pgfpoint{25.001205\du}{10.131380\du}}
\pgfpathlineto{\pgfpoint{24.987331\du}{10.125904\du}}
\pgfpathlineto{\pgfpoint{24.973457\du}{10.121157\du}}
\pgfpathlineto{\pgfpoint{24.959949\du}{10.116046\du}}
\pgfpathlineto{\pgfpoint{24.946440\du}{10.110570\du}}
\pgfpathlineto{\pgfpoint{24.933662\du}{10.106188\du}}
\pgfpathlineto{\pgfpoint{24.920153\du}{10.100347\du}}
\pgfpathlineto{\pgfpoint{24.907375\du}{10.094870\du}}
\pgfpathlineto{\pgfpoint{24.894962\du}{10.089759\du}}
\pgfpathlineto{\pgfpoint{24.881818\du}{10.084283\du}}
\pgfpathlineto{\pgfpoint{24.869770\du}{10.079171\du}}
\pgfpathlineto{\pgfpoint{24.858452\du}{10.073330\du}}
\pgfpathlineto{\pgfpoint{24.845674\du}{10.067853\du}}
\pgfpathlineto{\pgfpoint{24.833991\du}{10.062742\du}}
\pgfpathlineto{\pgfpoint{24.822307\du}{10.056900\du}}
\pgfpathlineto{\pgfpoint{24.811355\du}{10.051059\du}}
\pgfpathlineto{\pgfpoint{24.799671\du}{10.045217\du}}
\pgfpathlineto{\pgfpoint{24.789814\du}{10.039376\du}}
\pgfpathlineto{\pgfpoint{24.778496\du}{10.034264\du}}
\pgfpathlineto{\pgfpoint{24.767908\du}{10.027693\du}}
\pgfpathlineto{\pgfpoint{24.758050\du}{10.021851\du}}
\pgfpathlineto{\pgfpoint{24.747463\du}{10.016009\du}}
\pgfpathlineto{\pgfpoint{24.737605\du}{10.010168\du}}
\pgfpathlineto{\pgfpoint{24.728112\du}{10.004326\du}}
\pgfpathlineto{\pgfpoint{24.718255\du}{9.997755\du}}
\pgfpathlineto{\pgfpoint{24.709127\du}{9.991913\du}}
\pgfpathlineto{\pgfpoint{24.700730\du}{9.985341\du}}
\pgfpathlineto{\pgfpoint{24.691968\du}{9.979500\du}}
\pgfpathlineto{\pgfpoint{24.682840\du}{9.973293\du}}
\pgfpathlineto{\pgfpoint{24.674078\du}{9.966721\du}}
\pgfpathlineto{\pgfpoint{24.665681\du}{9.960515\du}}
\pgfpathlineto{\pgfpoint{24.658379\du}{9.953943\du}}
\pgfpathlineto{\pgfpoint{24.650347\du}{9.948101\du}}
\pgfpathlineto{\pgfpoint{24.642680\du}{9.941530\du}}
\pgfpathlineto{\pgfpoint{24.635743\du}{9.935323\du}}
\pgfpathlineto{\pgfpoint{24.628441\du}{9.928751\du}}
\pgfpathlineto{\pgfpoint{24.621504\du}{9.921815\du}}
\pgfpathlineto{\pgfpoint{24.615298\du}{9.915243\du}}
\pgfpathlineto{\pgfpoint{24.608726\du}{9.909036\du}}
\pgfpathlineto{\pgfpoint{24.602884\du}{9.902464\du}}
\pgfpathlineto{\pgfpoint{24.596678\du}{9.895528\du}}
\pgfpathlineto{\pgfpoint{24.591566\du}{9.888956\du}}
\pgfpathlineto{\pgfpoint{24.585360\du}{9.882384\du}}
\pgfpathlineto{\pgfpoint{24.580978\du}{9.875447\du}}
\pgfpathlineto{\pgfpoint{24.575502\du}{9.869241\du}}
\pgfpathlineto{\pgfpoint{24.571121\du}{9.861939\du}}
\pgfpathlineto{\pgfpoint{24.566740\du}{9.855732\du}}
\pgfpathlineto{\pgfpoint{24.562359\du}{9.848430\du}}
\pgfpathlineto{\pgfpoint{24.559073\du}{9.842223\du}}
\pgfpathlineto{\pgfpoint{24.555057\du}{9.835287\du}}
\pgfpathlineto{\pgfpoint{24.551041\du}{9.827985\du}}
\pgfpathlineto{\pgfpoint{24.548120\du}{9.821048\du}}
\pgfpathlineto{\pgfpoint{24.545564\du}{9.814841\du}}
\pgfpathlineto{\pgfpoint{24.542278\du}{9.807539\du}}
\pgfpathlineto{\pgfpoint{24.540088\du}{9.800602\du}}
\pgfpathlineto{\pgfpoint{24.537897\du}{9.793666\du}}
\pgfpathlineto{\pgfpoint{24.536437\du}{9.786729\du}}
\pgfpathlineto{\pgfpoint{24.534246\du}{9.779427\du}}
\pgfpathlineto{\pgfpoint{24.533151\du}{9.773220\du}}
\pgfpathlineto{\pgfpoint{24.532055\du}{9.765918\du}}
\pgfpathlineto{\pgfpoint{24.530595\du}{9.758981\du}}
\pgfpathlineto{\pgfpoint{24.530230\du}{9.752045\du}}
\pgfpathlineto{\pgfpoint{24.530230\du}{9.744378\du}}
\pgfpathlineto{\pgfpoint{24.529865\du}{9.737441\du}}
\pgfpathlineto{\pgfpoint{24.529865\du}{9.737441\du}}
\pgfpathlineto{\pgfpoint{24.529865\du}{9.737441\du}}
\pgfpathlineto{\pgfpoint{24.529865\du}{9.736345\du}}
\pgfpathlineto{\pgfpoint{24.529865\du}{9.735250\du}}
\pgfpathlineto{\pgfpoint{24.529500\du}{9.733790\du}}
\pgfpathlineto{\pgfpoint{24.529500\du}{9.733425\du}}
\pgfpathlineto{\pgfpoint{24.528405\du}{9.732329\du}}
\pgfpathlineto{\pgfpoint{24.528039\du}{9.730869\du}}
\pgfpathlineto{\pgfpoint{24.527674\du}{9.730504\du}}
\pgfpathlineto{\pgfpoint{24.526579\du}{9.729774\du}}
\pgfpathlineto{\pgfpoint{24.525119\du}{9.728678\du}}
\pgfpathlineto{\pgfpoint{24.523293\du}{9.727948\du}}
\pgfpathlineto{\pgfpoint{24.521468\du}{9.727583\du}}
\pgfpathlineto{\pgfpoint{24.519642\du}{9.727583\du}}
\pgfpathlineto{\pgfpoint{24.517452\du}{9.727583\du}}
\pgfpathlineto{\pgfpoint{24.515991\du}{9.727948\du}}
\pgfpathlineto{\pgfpoint{24.513801\du}{9.728678\du}}
\pgfpathlineto{\pgfpoint{24.511975\du}{9.729774\du}}
\pgfpathlineto{\pgfpoint{24.511610\du}{9.730504\du}}
\pgfpathlineto{\pgfpoint{24.511245\du}{9.730869\du}}
\pgfpathlineto{\pgfpoint{24.510515\du}{9.732329\du}}
\pgfpathlineto{\pgfpoint{24.509785\du}{9.733425\du}}
\pgfpathlineto{\pgfpoint{24.509785\du}{9.733790\du}}
\pgfpathlineto{\pgfpoint{24.509419\du}{9.735250\du}}
\pgfpathlineto{\pgfpoint{24.509419\du}{9.736345\du}}
\pgfpathlineto{\pgfpoint{24.509419\du}{9.737441\du}}
\pgfusepath{fill}
\pgfsetbuttcap
\pgfsetmiterjoin
\pgfsetdash{}{0pt}
\definecolor{dialinecolor}{rgb}{0.678431, 0.839216, 0.905882}
\pgfsetfillcolor{dialinecolor}
\pgfpathmoveto{\pgfpoint{26.200183\du}{9.150000\du}}
\pgfpathlineto{\pgfpoint{26.200183\du}{9.150000\du}}
\pgfpathlineto{\pgfpoint{26.156371\du}{9.150000\du}}
\pgfpathlineto{\pgfpoint{26.113655\du}{9.150730\du}}
\pgfpathlineto{\pgfpoint{26.070208\du}{9.151825\du}}
\pgfpathlineto{\pgfpoint{26.028222\du}{9.152921\du}}
\pgfpathlineto{\pgfpoint{25.985871\du}{9.154746\du}}
\pgfpathlineto{\pgfpoint{25.944250\du}{9.156937\du}}
\pgfpathlineto{\pgfpoint{25.902264\du}{9.159493\du}}
\pgfpathlineto{\pgfpoint{25.861008\du}{9.161683\du}}
\pgfpathlineto{\pgfpoint{25.820117\du}{9.164604\du}}
\pgfpathlineto{\pgfpoint{25.779226\du}{9.168255\du}}
\pgfpathlineto{\pgfpoint{25.739430\du}{9.171906\du}}
\pgfpathlineto{\pgfpoint{25.699270\du}{9.175922\du}}
\pgfpathlineto{\pgfpoint{25.660204\du}{9.180303\du}}
\pgfpathlineto{\pgfpoint{25.621504\du}{9.185049\du}}
\pgfpathlineto{\pgfpoint{25.582804\du}{9.190161\du}}
\pgfpathlineto{\pgfpoint{25.544834\du}{9.195637\du}}
\pgfpathlineto{\pgfpoint{25.507229\du}{9.200748\du}}
\pgfpathlineto{\pgfpoint{25.469989\du}{9.207320\du}}
\pgfpathlineto{\pgfpoint{25.433479\du}{9.213162\du}}
\pgfpathlineto{\pgfpoint{25.397335\du}{9.219733\du}}
\pgfpathlineto{\pgfpoint{25.361555\du}{9.226670\du}}
\pgfpathlineto{\pgfpoint{25.326871\du}{9.233607\du}}
\pgfpathlineto{\pgfpoint{25.292552\du}{9.241274\du}}
\pgfpathlineto{\pgfpoint{25.257868\du}{9.248941\du}}
\pgfpathlineto{\pgfpoint{25.225374\du}{9.256973\du}}
\pgfpathlineto{\pgfpoint{25.192150\du}{9.265371\du}}
\pgfpathlineto{\pgfpoint{25.160387\du}{9.273403\du}}
\pgfpathlineto{\pgfpoint{25.128624\du}{9.282165\du}}
\pgfpathlineto{\pgfpoint{25.097955\du}{9.291658\du}}
\pgfpathlineto{\pgfpoint{25.066922\du}{9.300420\du}}
\pgfpathlineto{\pgfpoint{25.052318\du}{9.305166\du}}
\pgfpathlineto{\pgfpoint{25.038445\du}{9.310277\du}}
\pgfpathlineto{\pgfpoint{25.022746\du}{9.315024\du}}
\pgfpathlineto{\pgfpoint{25.008872\du}{9.319770\du}}
\pgfpathlineto{\pgfpoint{24.994268\du}{9.324881\du}}
\pgfpathlineto{\pgfpoint{24.980394\du}{9.329628\du}}
\pgfpathlineto{\pgfpoint{24.966156\du}{9.334739\du}}
\pgfpathlineto{\pgfpoint{24.953012\du}{9.339485\du}}
\pgfpathlineto{\pgfpoint{24.938773\du}{9.344962\du}}
\pgfpathlineto{\pgfpoint{24.925630\du}{9.350073\du}}
\pgfpathlineto{\pgfpoint{24.913217\du}{9.355184\du}}
\pgfpathlineto{\pgfpoint{24.899343\du}{9.361026\du}}
\pgfpathlineto{\pgfpoint{24.886564\du}{9.366502\du}}
\pgfpathlineto{\pgfpoint{24.874516\du}{9.371614\du}}
\pgfpathlineto{\pgfpoint{24.861373\du}{9.377455\du}}
\pgfpathlineto{\pgfpoint{24.849325\du}{9.382932\du}}
\pgfpathlineto{\pgfpoint{24.836911\du}{9.388773\du}}
\pgfpathlineto{\pgfpoint{24.825228\du}{9.393885\du}}
\pgfpathlineto{\pgfpoint{24.813910\du}{9.399726\du}}
\pgfpathlineto{\pgfpoint{24.802227\du}{9.405568\du}}
\pgfpathlineto{\pgfpoint{24.790909\du}{9.411409\du}}
\pgfpathlineto{\pgfpoint{24.779226\du}{9.417251\du}}
\pgfpathlineto{\pgfpoint{24.768273\du}{9.423092\du}}
\pgfpathlineto{\pgfpoint{24.758050\du}{9.429664\du}}
\pgfpathlineto{\pgfpoint{24.747463\du}{9.435506\du}}
\pgfpathlineto{\pgfpoint{24.737240\du}{9.441347\du}}
\pgfpathlineto{\pgfpoint{24.727017\du}{9.447919\du}}
\pgfpathlineto{\pgfpoint{24.717160\du}{9.453760\du}}
\pgfpathlineto{\pgfpoint{24.707667\du}{9.459967\du}}
\pgfpathlineto{\pgfpoint{24.698175\du}{9.466539\du}}
\pgfpathlineto{\pgfpoint{24.689047\du}{9.473111\du}}
\pgfpathlineto{\pgfpoint{24.679189\du}{9.478952\du}}
\pgfpathlineto{\pgfpoint{24.670792\du}{9.485159\du}}
\pgfpathlineto{\pgfpoint{24.662030\du}{9.491731\du}}
\pgfpathlineto{\pgfpoint{24.653633\du}{9.497937\du}}
\pgfpathlineto{\pgfpoint{24.644870\du}{9.505239\du}}
\pgfpathlineto{\pgfpoint{24.637203\du}{9.511446\du}}
\pgfpathlineto{\pgfpoint{24.630267\du}{9.518018\du}}
\pgfpathlineto{\pgfpoint{24.621504\du}{9.524954\du}}
\pgfpathlineto{\pgfpoint{24.614567\du}{9.531526\du}}
\pgfpathlineto{\pgfpoint{24.607631\du}{9.538463\du}}
\pgfpathlineto{\pgfpoint{24.600694\du}{9.545400\du}}
\pgfpathlineto{\pgfpoint{24.594122\du}{9.551972\du}}
\pgfpathlineto{\pgfpoint{24.587185\du}{9.558908\du}}
\pgfpathlineto{\pgfpoint{24.581344\du}{9.565845\du}}
\pgfpathlineto{\pgfpoint{24.575502\du}{9.573147\du}}
\pgfpathlineto{\pgfpoint{24.569660\du}{9.580084\du}}
\pgfpathlineto{\pgfpoint{24.564184\du}{9.587021\du}}
\pgfpathlineto{\pgfpoint{24.559438\du}{9.593958\du}}
\pgfpathlineto{\pgfpoint{24.553961\du}{9.601625\du}}
\pgfpathlineto{\pgfpoint{24.549945\du}{9.608562\du}}
\pgfpathlineto{\pgfpoint{24.544834\du}{9.615863\du}}
\pgfpathlineto{\pgfpoint{24.541183\du}{9.623165\du}}
\pgfpathlineto{\pgfpoint{24.536802\du}{9.630467\du}}
\pgfpathlineto{\pgfpoint{24.533151\du}{9.638134\du}}
\pgfpathlineto{\pgfpoint{24.529865\du}{9.645436\du}}
\pgfpathlineto{\pgfpoint{24.526214\du}{9.653103\du}}
\pgfpathlineto{\pgfpoint{24.523293\du}{9.660770\du}}
\pgfpathlineto{\pgfpoint{24.520737\du}{9.667707\du}}
\pgfpathlineto{\pgfpoint{24.518547\du}{9.675374\du}}
\pgfpathlineto{\pgfpoint{24.516356\du}{9.683041\du}}
\pgfpathlineto{\pgfpoint{24.514531\du}{9.691073\du}}
\pgfpathlineto{\pgfpoint{24.512705\du}{9.698740\du}}
\pgfpathlineto{\pgfpoint{24.511610\du}{9.706407\du}}
\pgfpathlineto{\pgfpoint{24.510515\du}{9.714074\du}}
\pgfpathlineto{\pgfpoint{24.509785\du}{9.722107\du}}
\pgfpathlineto{\pgfpoint{24.509419\du}{9.729774\du}}
\pgfpathlineto{\pgfpoint{24.509419\du}{9.737441\du}}
\pgfpathlineto{\pgfpoint{24.529865\du}{9.737441\du}}
\pgfpathlineto{\pgfpoint{24.530230\du}{9.730504\du}}
\pgfpathlineto{\pgfpoint{24.530230\du}{9.723567\du}}
\pgfpathlineto{\pgfpoint{24.530595\du}{9.716265\du}}
\pgfpathlineto{\pgfpoint{24.532055\du}{9.709328\du}}
\pgfpathlineto{\pgfpoint{24.533151\du}{9.702391\du}}
\pgfpathlineto{\pgfpoint{24.534246\du}{9.695455\du}}
\pgfpathlineto{\pgfpoint{24.536437\du}{9.688153\du}}
\pgfpathlineto{\pgfpoint{24.537897\du}{9.681946\du}}
\pgfpathlineto{\pgfpoint{24.540088\du}{9.674644\du}}
\pgfpathlineto{\pgfpoint{24.542278\du}{9.667707\du}}
\pgfpathlineto{\pgfpoint{24.545564\du}{9.660770\du}}
\pgfpathlineto{\pgfpoint{24.548120\du}{9.653834\du}}
\pgfpathlineto{\pgfpoint{24.551041\du}{9.646897\du}}
\pgfpathlineto{\pgfpoint{24.555057\du}{9.640325\du}}
\pgfpathlineto{\pgfpoint{24.559073\du}{9.633388\du}}
\pgfpathlineto{\pgfpoint{24.562359\du}{9.626816\du}}
\pgfpathlineto{\pgfpoint{24.566375\du}{9.619880\du}}
\pgfpathlineto{\pgfpoint{24.571121\du}{9.612943\du}}
\pgfpathlineto{\pgfpoint{24.575502\du}{9.606371\du}}
\pgfpathlineto{\pgfpoint{24.580978\du}{9.599799\du}}
\pgfpathlineto{\pgfpoint{24.585360\du}{9.592862\du}}
\pgfpathlineto{\pgfpoint{24.591566\du}{9.586291\du}}
\pgfpathlineto{\pgfpoint{24.596678\du}{9.579354\du}}
\pgfpathlineto{\pgfpoint{24.602884\du}{9.573147\du}}
\pgfpathlineto{\pgfpoint{24.608726\du}{9.566575\du}}
\pgfpathlineto{\pgfpoint{24.615298\du}{9.559639\du}}
\pgfpathlineto{\pgfpoint{24.621504\du}{9.553067\du}}
\pgfpathlineto{\pgfpoint{24.628441\du}{9.546860\du}}
\pgfpathlineto{\pgfpoint{24.635743\du}{9.540288\du}}
\pgfpathlineto{\pgfpoint{24.642680\du}{9.533717\du}}
\pgfpathlineto{\pgfpoint{24.650347\du}{9.527510\du}}
\pgfpathlineto{\pgfpoint{24.658379\du}{9.520938\du}}
\pgfpathlineto{\pgfpoint{24.665681\du}{9.514367\du}}
\pgfpathlineto{\pgfpoint{24.674078\du}{9.508160\du}}
\pgfpathlineto{\pgfpoint{24.682840\du}{9.502318\du}}
\pgfpathlineto{\pgfpoint{24.691968\du}{9.495382\du}}
\pgfpathlineto{\pgfpoint{24.700730\du}{9.489905\du}}
\pgfpathlineto{\pgfpoint{24.709127\du}{9.483333\du}}
\pgfpathlineto{\pgfpoint{24.718255\du}{9.477127\du}}
\pgfpathlineto{\pgfpoint{24.728112\du}{9.471285\du}}
\pgfpathlineto{\pgfpoint{24.737605\du}{9.465444\du}}
\pgfpathlineto{\pgfpoint{24.747463\du}{9.458872\du}}
\pgfpathlineto{\pgfpoint{24.758050\du}{9.453030\du}}
\pgfpathlineto{\pgfpoint{24.767908\du}{9.447189\du}}
\pgfpathlineto{\pgfpoint{24.778496\du}{9.441347\du}}
\pgfpathlineto{\pgfpoint{24.789814\du}{9.435506\du}}
\pgfpathlineto{\pgfpoint{24.799671\du}{9.429664\du}}
\pgfpathlineto{\pgfpoint{24.811355\du}{9.423823\du}}
\pgfpathlineto{\pgfpoint{24.822307\du}{9.418711\du}}
\pgfpathlineto{\pgfpoint{24.833991\du}{9.412505\du}}
\pgfpathlineto{\pgfpoint{24.845674\du}{9.406663\du}}
\pgfpathlineto{\pgfpoint{24.858452\du}{9.401552\du}}
\pgfpathlineto{\pgfpoint{24.869770\du}{9.396440\du}}
\pgfpathlineto{\pgfpoint{24.881818\du}{9.390599\du}}
\pgfpathlineto{\pgfpoint{24.894962\du}{9.385122\du}}
\pgfpathlineto{\pgfpoint{24.907375\du}{9.380011\du}}
\pgfpathlineto{\pgfpoint{24.920153\du}{9.374535\du}}
\pgfpathlineto{\pgfpoint{24.933662\du}{9.369423\du}}
\pgfpathlineto{\pgfpoint{24.946440\du}{9.363947\du}}
\pgfpathlineto{\pgfpoint{24.959949\du}{9.358470\du}}
\pgfpathlineto{\pgfpoint{24.973457\du}{9.354089\du}}
\pgfpathlineto{\pgfpoint{24.987331\du}{9.348978\du}}
\pgfpathlineto{\pgfpoint{25.001205\du}{9.344231\du}}
\pgfpathlineto{\pgfpoint{25.015444\du}{9.339120\du}}
\pgfpathlineto{\pgfpoint{25.029682\du}{9.334374\du}}
\pgfpathlineto{\pgfpoint{25.043921\du}{9.329628\du}}
\pgfpathlineto{\pgfpoint{25.058160\du}{9.324881\du}}
\pgfpathlineto{\pgfpoint{25.073494\du}{9.320135\du}}
\pgfpathlineto{\pgfpoint{25.103067\du}{9.311008\du}}
\pgfpathlineto{\pgfpoint{25.134465\du}{9.302245\du}}
\pgfpathlineto{\pgfpoint{25.165498\du}{9.293483\du}}
\pgfpathlineto{\pgfpoint{25.197992\du}{9.285086\du}}
\pgfpathlineto{\pgfpoint{25.229390\du}{9.277054\du}}
\pgfpathlineto{\pgfpoint{25.262614\du}{9.268656\du}}
\pgfpathlineto{\pgfpoint{25.296568\du}{9.260989\du}}
\pgfpathlineto{\pgfpoint{25.330887\du}{9.254053\du}}
\pgfpathlineto{\pgfpoint{25.365571\du}{9.247116\du}}
\pgfpathlineto{\pgfpoint{25.401716\du}{9.239814\du}}
\pgfpathlineto{\pgfpoint{25.436765\du}{9.233607\du}}
\pgfpathlineto{\pgfpoint{25.474005\du}{9.226670\du}}
\pgfpathlineto{\pgfpoint{25.510515\du}{9.221194\du}}
\pgfpathlineto{\pgfpoint{25.547755\du}{9.215352\du}}
\pgfpathlineto{\pgfpoint{25.585360\du}{9.210241\du}}
\pgfpathlineto{\pgfpoint{25.623695\du}{9.205495\du}}
\pgfpathlineto{\pgfpoint{25.662760\du}{9.200748\du}}
\pgfpathlineto{\pgfpoint{25.702191\du}{9.196367\du}}
\pgfpathlineto{\pgfpoint{25.741621\du}{9.192716\du}}
\pgfpathlineto{\pgfpoint{25.781417\du}{9.188700\du}}
\pgfpathlineto{\pgfpoint{25.821577\du}{9.185779\du}}
\pgfpathlineto{\pgfpoint{25.862468\du}{9.182129\du}}
\pgfpathlineto{\pgfpoint{25.903724\du}{9.179208\du}}
\pgfpathlineto{\pgfpoint{25.944615\du}{9.177017\du}}
\pgfpathlineto{\pgfpoint{25.986601\du}{9.175192\du}}
\pgfpathlineto{\pgfpoint{26.028952\du}{9.173366\du}}
\pgfpathlineto{\pgfpoint{26.071668\du}{9.171906\du}}
\pgfpathlineto{\pgfpoint{26.114020\du}{9.171541\du}}
\pgfpathlineto{\pgfpoint{26.157466\du}{9.171176\du}}
\pgfpathlineto{\pgfpoint{26.200183\du}{9.170445\du}}
\pgfpathlineto{\pgfpoint{26.200183\du}{9.170445\du}}
\pgfpathlineto{\pgfpoint{26.200183\du}{9.170445\du}}
\pgfpathlineto{\pgfpoint{26.201278\du}{9.170445\du}}
\pgfpathlineto{\pgfpoint{26.202373\du}{9.170445\du}}
\pgfpathlineto{\pgfpoint{26.203834\du}{9.169715\du}}
\pgfpathlineto{\pgfpoint{26.204929\du}{9.169715\du}}
\pgfpathlineto{\pgfpoint{26.205294\du}{9.169350\du}}
\pgfpathlineto{\pgfpoint{26.206389\du}{9.168620\du}}
\pgfpathlineto{\pgfpoint{26.207119\du}{9.168255\du}}
\pgfpathlineto{\pgfpoint{26.208215\du}{9.167525\du}}
\pgfpathlineto{\pgfpoint{26.209310\du}{9.165699\du}}
\pgfpathlineto{\pgfpoint{26.210040\du}{9.163874\du}}
\pgfpathlineto{\pgfpoint{26.210040\du}{9.162413\du}}
\pgfpathlineto{\pgfpoint{26.210770\du}{9.160588\du}}
\pgfpathlineto{\pgfpoint{26.210040\du}{9.158762\du}}
\pgfpathlineto{\pgfpoint{26.210040\du}{9.156572\du}}
\pgfpathlineto{\pgfpoint{26.209310\du}{9.154746\du}}
\pgfpathlineto{\pgfpoint{26.208215\du}{9.152921\du}}
\pgfpathlineto{\pgfpoint{26.207119\du}{9.152191\du}}
\pgfpathlineto{\pgfpoint{26.206389\du}{9.151825\du}}
\pgfpathlineto{\pgfpoint{26.205294\du}{9.151095\du}}
\pgfpathlineto{\pgfpoint{26.204929\du}{9.150730\du}}
\pgfpathlineto{\pgfpoint{26.203834\du}{9.150730\du}}
\pgfpathlineto{\pgfpoint{26.202373\du}{9.150000\du}}
\pgfpathlineto{\pgfpoint{26.201278\du}{9.150000\du}}
\pgfpathlineto{\pgfpoint{26.200183\du}{9.150000\du}}
\pgfusepath{fill}
\pgfsetbuttcap
\pgfsetmiterjoin
\pgfsetdash{}{0pt}
\definecolor{dialinecolor}{rgb}{0.678431, 0.839216, 0.905882}
\pgfsetfillcolor{dialinecolor}
\pgfpathmoveto{\pgfpoint{27.890581\du}{9.737441\du}}
\pgfpathlineto{\pgfpoint{27.890581\du}{9.729774\du}}
\pgfpathlineto{\pgfpoint{27.889850\du}{9.722107\du}}
\pgfpathlineto{\pgfpoint{27.889120\du}{9.714074\du}}
\pgfpathlineto{\pgfpoint{27.888390\du}{9.706407\du}}
\pgfpathlineto{\pgfpoint{27.887295\du}{9.698740\du}}
\pgfpathlineto{\pgfpoint{27.885104\du}{9.691073\du}}
\pgfpathlineto{\pgfpoint{27.884009\du}{9.683041\du}}
\pgfpathlineto{\pgfpoint{27.881088\du}{9.675374\du}}
\pgfpathlineto{\pgfpoint{27.879263\du}{9.667707\du}}
\pgfpathlineto{\pgfpoint{27.876342\du}{9.660770\du}}
\pgfpathlineto{\pgfpoint{27.873786\du}{9.653103\du}}
\pgfpathlineto{\pgfpoint{27.870500\du}{9.645436\du}}
\pgfpathlineto{\pgfpoint{27.866484\du}{9.638134\du}}
\pgfpathlineto{\pgfpoint{27.863198\du}{9.630467\du}}
\pgfpathlineto{\pgfpoint{27.859182\du}{9.623165\du}}
\pgfpathlineto{\pgfpoint{27.855166\du}{9.615863\du}}
\pgfpathlineto{\pgfpoint{27.850785\du}{9.608562\du}}
\pgfpathlineto{\pgfpoint{27.846039\du}{9.601625\du}}
\pgfpathlineto{\pgfpoint{27.840562\du}{9.593958\du}}
\pgfpathlineto{\pgfpoint{27.835816\du}{9.587021\du}}
\pgfpathlineto{\pgfpoint{27.829974\du}{9.580084\du}}
\pgfpathlineto{\pgfpoint{27.824498\du}{9.573147\du}}
\pgfpathlineto{\pgfpoint{27.818656\du}{9.565845\du}}
\pgfpathlineto{\pgfpoint{27.812450\du}{9.558908\du}}
\pgfpathlineto{\pgfpoint{27.805878\du}{9.551972\du}}
\pgfpathlineto{\pgfpoint{27.798941\du}{9.545400\du}}
\pgfpathlineto{\pgfpoint{27.792369\du}{9.538463\du}}
\pgfpathlineto{\pgfpoint{27.785068\du}{9.531526\du}}
\pgfpathlineto{\pgfpoint{27.778496\du}{9.524954\du}}
\pgfpathlineto{\pgfpoint{27.770464\du}{9.518018\du}}
\pgfpathlineto{\pgfpoint{27.762432\du}{9.511446\du}}
\pgfpathlineto{\pgfpoint{27.755130\du}{9.505239\du}}
\pgfpathlineto{\pgfpoint{27.746367\du}{9.497937\du}}
\pgfpathlineto{\pgfpoint{27.738335\du}{9.491731\du}}
\pgfpathlineto{\pgfpoint{27.729208\du}{9.485159\du}}
\pgfpathlineto{\pgfpoint{27.720445\du}{9.478952\du}}
\pgfpathlineto{\pgfpoint{27.710953\du}{9.473111\du}}
\pgfpathlineto{\pgfpoint{27.701825\du}{9.466539\du}}
\pgfpathlineto{\pgfpoint{27.692698\du}{9.459967\du}}
\pgfpathlineto{\pgfpoint{27.682840\du}{9.453760\du}}
\pgfpathlineto{\pgfpoint{27.672983\du}{9.447919\du}}
\pgfpathlineto{\pgfpoint{27.662760\du}{9.441347\du}}
\pgfpathlineto{\pgfpoint{27.652172\du}{9.435506\du}}
\pgfpathlineto{\pgfpoint{27.642315\du}{9.429664\du}}
\pgfpathlineto{\pgfpoint{27.631727\du}{9.423092\du}}
\pgfpathlineto{\pgfpoint{27.620409\du}{9.417251\du}}
\pgfpathlineto{\pgfpoint{27.609091\du}{9.411409\du}}
\pgfpathlineto{\pgfpoint{27.597408\du}{9.405568\du}}
\pgfpathlineto{\pgfpoint{27.586455\du}{9.399726\du}}
\pgfpathlineto{\pgfpoint{27.574407\du}{9.393885\du}}
\pgfpathlineto{\pgfpoint{27.563089\du}{9.388773\du}}
\pgfpathlineto{\pgfpoint{27.550675\du}{9.382932\du}}
\pgfpathlineto{\pgfpoint{27.538262\du}{9.377455\du}}
\pgfpathlineto{\pgfpoint{27.525484\du}{9.371614\du}}
\pgfpathlineto{\pgfpoint{27.513436\du}{9.366502\du}}
\pgfpathlineto{\pgfpoint{27.500657\du}{9.361026\du}}
\pgfpathlineto{\pgfpoint{27.487149\du}{9.355184\du}}
\pgfpathlineto{\pgfpoint{27.474005\du}{9.350073\du}}
\pgfpathlineto{\pgfpoint{27.460862\du}{9.344962\du}}
\pgfpathlineto{\pgfpoint{27.446623\du}{9.339485\du}}
\pgfpathlineto{\pgfpoint{27.433479\du}{9.334739\du}}
\pgfpathlineto{\pgfpoint{27.419241\du}{9.329628\du}}
\pgfpathlineto{\pgfpoint{27.405367\du}{9.324881\du}}
\pgfpathlineto{\pgfpoint{27.391128\du}{9.319770\du}}
\pgfpathlineto{\pgfpoint{27.377254\du}{9.315024\du}}
\pgfpathlineto{\pgfpoint{27.362286\du}{9.310277\du}}
\pgfpathlineto{\pgfpoint{27.347317\du}{9.305166\du}}
\pgfpathlineto{\pgfpoint{27.333078\du}{9.300420\du}}
\pgfpathlineto{\pgfpoint{27.302410\du}{9.291658\du}}
\pgfpathlineto{\pgfpoint{27.271742\du}{9.282165\du}}
\pgfpathlineto{\pgfpoint{27.240343\du}{9.273403\du}}
\pgfpathlineto{\pgfpoint{27.208580\du}{9.265371\du}}
\pgfpathlineto{\pgfpoint{27.175721\du}{9.256973\du}}
\pgfpathlineto{\pgfpoint{27.142132\du}{9.248941\du}}
\pgfpathlineto{\pgfpoint{27.108178\du}{9.241274\du}}
\pgfpathlineto{\pgfpoint{27.073494\du}{9.233607\du}}
\pgfpathlineto{\pgfpoint{27.038810\du}{9.226670\du}}
\pgfpathlineto{\pgfpoint{27.003395\du}{9.219733\du}}
\pgfpathlineto{\pgfpoint{26.966886\du}{9.213162\du}}
\pgfpathlineto{\pgfpoint{26.930376\du}{9.207320\du}}
\pgfpathlineto{\pgfpoint{26.893501\du}{9.200748\du}}
\pgfpathlineto{\pgfpoint{26.856261\du}{9.195637\du}}
\pgfpathlineto{\pgfpoint{26.817196\du}{9.190161\du}}
\pgfpathlineto{\pgfpoint{26.779226\du}{9.185049\du}}
\pgfpathlineto{\pgfpoint{26.739796\du}{9.180303\du}}
\pgfpathlineto{\pgfpoint{26.701095\du}{9.175922\du}}
\pgfpathlineto{\pgfpoint{26.660935\du}{9.171906\du}}
\pgfpathlineto{\pgfpoint{26.621139\du}{9.168255\du}}
\pgfpathlineto{\pgfpoint{26.580248\du}{9.164604\du}}
\pgfpathlineto{\pgfpoint{26.539723\du}{9.161683\du}}
\pgfpathlineto{\pgfpoint{26.498101\du}{9.159493\du}}
\pgfpathlineto{\pgfpoint{26.456480\du}{9.156937\du}}
\pgfpathlineto{\pgfpoint{26.414859\du}{9.154746\du}}
\pgfpathlineto{\pgfpoint{26.372143\du}{9.152921\du}}
\pgfpathlineto{\pgfpoint{26.330157\du}{9.151825\du}}
\pgfpathlineto{\pgfpoint{26.287076\du}{9.150730\du}}
\pgfpathlineto{\pgfpoint{26.243629\du}{9.150000\du}}
\pgfpathlineto{\pgfpoint{26.200183\du}{9.150000\du}}
\pgfpathlineto{\pgfpoint{26.200183\du}{9.170445\du}}
\pgfpathlineto{\pgfpoint{26.243264\du}{9.171176\du}}
\pgfpathlineto{\pgfpoint{26.286710\du}{9.171541\du}}
\pgfpathlineto{\pgfpoint{26.328697\du}{9.171906\du}}
\pgfpathlineto{\pgfpoint{26.371413\du}{9.173366\du}}
\pgfpathlineto{\pgfpoint{26.414129\du}{9.175192\du}}
\pgfpathlineto{\pgfpoint{26.455750\du}{9.177017\du}}
\pgfpathlineto{\pgfpoint{26.497006\du}{9.179208\du}}
\pgfpathlineto{\pgfpoint{26.537897\du}{9.182129\du}}
\pgfpathlineto{\pgfpoint{26.579153\du}{9.185779\du}}
\pgfpathlineto{\pgfpoint{26.619314\du}{9.188700\du}}
\pgfpathlineto{\pgfpoint{26.659474\du}{9.192716\du}}
\pgfpathlineto{\pgfpoint{26.698540\du}{9.196367\du}}
\pgfpathlineto{\pgfpoint{26.737970\du}{9.200748\du}}
\pgfpathlineto{\pgfpoint{26.776670\du}{9.205495\du}}
\pgfpathlineto{\pgfpoint{26.815371\du}{9.210241\du}}
\pgfpathlineto{\pgfpoint{26.852976\du}{9.215352\du}}
\pgfpathlineto{\pgfpoint{26.890215\du}{9.221194\du}}
\pgfpathlineto{\pgfpoint{26.926725\du}{9.226670\du}}
\pgfpathlineto{\pgfpoint{26.963600\du}{9.233607\du}}
\pgfpathlineto{\pgfpoint{26.999014\du}{9.239814\du}}
\pgfpathlineto{\pgfpoint{27.034794\du}{9.247116\du}}
\pgfpathlineto{\pgfpoint{27.069478\du}{9.254053\du}}
\pgfpathlineto{\pgfpoint{27.103432\du}{9.260989\du}}
\pgfpathlineto{\pgfpoint{27.137751\du}{9.268656\du}}
\pgfpathlineto{\pgfpoint{27.170975\du}{9.277054\du}}
\pgfpathlineto{\pgfpoint{27.203103\du}{9.285086\du}}
\pgfpathlineto{\pgfpoint{27.235232\du}{9.293483\du}}
\pgfpathlineto{\pgfpoint{27.266630\du}{9.302245\du}}
\pgfpathlineto{\pgfpoint{27.297298\du}{9.311008\du}}
\pgfpathlineto{\pgfpoint{27.327236\du}{9.320135\du}}
\pgfpathlineto{\pgfpoint{27.341475\du}{9.324881\du}}
\pgfpathlineto{\pgfpoint{27.356079\du}{9.329628\du}}
\pgfpathlineto{\pgfpoint{27.369953\du}{9.334374\du}}
\pgfpathlineto{\pgfpoint{27.384556\du}{9.339120\du}}
\pgfpathlineto{\pgfpoint{27.398795\du}{9.344231\du}}
\pgfpathlineto{\pgfpoint{27.412669\du}{9.348978\du}}
\pgfpathlineto{\pgfpoint{27.426908\du}{9.354089\du}}
\pgfpathlineto{\pgfpoint{27.440051\du}{9.358470\du}}
\pgfpathlineto{\pgfpoint{27.453560\du}{9.363947\du}}
\pgfpathlineto{\pgfpoint{27.466703\du}{9.369423\du}}
\pgfpathlineto{\pgfpoint{27.479482\du}{9.374535\du}}
\pgfpathlineto{\pgfpoint{27.492260\du}{9.380011\du}}
\pgfpathlineto{\pgfpoint{27.505403\du}{9.385122\du}}
\pgfpathlineto{\pgfpoint{27.517452\du}{9.390599\du}}
\pgfpathlineto{\pgfpoint{27.529865\du}{9.396440\du}}
\pgfpathlineto{\pgfpoint{27.541913\du}{9.401552\du}}
\pgfpathlineto{\pgfpoint{27.554326\du}{9.406663\du}}
\pgfpathlineto{\pgfpoint{27.565644\du}{9.412505\du}}
\pgfpathlineto{\pgfpoint{27.577693\du}{9.418711\du}}
\pgfpathlineto{\pgfpoint{27.588280\du}{9.423823\du}}
\pgfpathlineto{\pgfpoint{27.600329\du}{9.429664\du}}
\pgfpathlineto{\pgfpoint{27.610551\du}{9.435506\du}}
\pgfpathlineto{\pgfpoint{27.621504\du}{9.441347\du}}
\pgfpathlineto{\pgfpoint{27.632457\du}{9.447189\du}}
\pgfpathlineto{\pgfpoint{27.642315\du}{9.453030\du}}
\pgfpathlineto{\pgfpoint{27.652172\du}{9.458872\du}}
\pgfpathlineto{\pgfpoint{27.662395\du}{9.465444\du}}
\pgfpathlineto{\pgfpoint{27.671522\du}{9.471285\du}}
\pgfpathlineto{\pgfpoint{27.681745\du}{9.477127\du}}
\pgfpathlineto{\pgfpoint{27.691238\du}{9.483333\du}}
\pgfpathlineto{\pgfpoint{27.700000\du}{9.489905\du}}
\pgfpathlineto{\pgfpoint{27.708032\du}{9.495382\du}}
\pgfpathlineto{\pgfpoint{27.716794\du}{9.502318\du}}
\pgfpathlineto{\pgfpoint{27.725192\du}{9.508160\du}}
\pgfpathlineto{\pgfpoint{27.733954\du}{9.514367\du}}
\pgfpathlineto{\pgfpoint{27.741621\du}{9.520938\du}}
\pgfpathlineto{\pgfpoint{27.749653\du}{9.527510\du}}
\pgfpathlineto{\pgfpoint{27.756955\du}{9.533717\du}}
\pgfpathlineto{\pgfpoint{27.764622\du}{9.540288\du}}
\pgfpathlineto{\pgfpoint{27.771194\du}{9.546860\du}}
\pgfpathlineto{\pgfpoint{27.778496\du}{9.553067\du}}
\pgfpathlineto{\pgfpoint{27.784337\du}{9.559639\du}}
\pgfpathlineto{\pgfpoint{27.791274\du}{9.566575\du}}
\pgfpathlineto{\pgfpoint{27.797846\du}{9.573147\du}}
\pgfpathlineto{\pgfpoint{27.802957\du}{9.579354\du}}
\pgfpathlineto{\pgfpoint{27.808434\du}{9.586291\du}}
\pgfpathlineto{\pgfpoint{27.814640\du}{9.592862\du}}
\pgfpathlineto{\pgfpoint{27.819387\du}{9.599799\du}}
\pgfpathlineto{\pgfpoint{27.824498\du}{9.606371\du}}
\pgfpathlineto{\pgfpoint{27.829244\du}{9.612943\du}}
\pgfpathlineto{\pgfpoint{27.833625\du}{9.619880\du}}
\pgfpathlineto{\pgfpoint{27.838007\du}{9.626816\du}}
\pgfpathlineto{\pgfpoint{27.840927\du}{9.633388\du}}
\pgfpathlineto{\pgfpoint{27.844943\du}{9.640325\du}}
\pgfpathlineto{\pgfpoint{27.847864\du}{9.646897\du}}
\pgfpathlineto{\pgfpoint{27.851880\du}{9.653834\du}}
\pgfpathlineto{\pgfpoint{27.854436\du}{9.660770\du}}
\pgfpathlineto{\pgfpoint{27.857357\du}{9.667707\du}}
\pgfpathlineto{\pgfpoint{27.859912\du}{9.674644\du}}
\pgfpathlineto{\pgfpoint{27.862103\du}{9.681946\du}}
\pgfpathlineto{\pgfpoint{27.863563\du}{9.688153\du}}
\pgfpathlineto{\pgfpoint{27.865754\du}{9.695455\du}}
\pgfpathlineto{\pgfpoint{27.866484\du}{9.702391\du}}
\pgfpathlineto{\pgfpoint{27.867579\du}{9.709328\du}}
\pgfpathlineto{\pgfpoint{27.869405\du}{9.716265\du}}
\pgfpathlineto{\pgfpoint{27.869770\du}{9.723567\du}}
\pgfpathlineto{\pgfpoint{27.869770\du}{9.730504\du}}
\pgfpathlineto{\pgfpoint{27.870500\du}{9.737441\du}}
\pgfpathlineto{\pgfpoint{27.890581\du}{9.737441\du}}
\pgfusepath{fill}
\pgfsetbuttcap
\pgfsetmiterjoin
\pgfsetdash{}{0pt}
\definecolor{dialinecolor}{rgb}{0.074510, 0.082353, 0.086275}
\pgfsetfillcolor{dialinecolor}
\pgfpathmoveto{\pgfpoint{26.243264\du}{9.610022\du}}
\pgfpathlineto{\pgfpoint{26.491165\du}{9.692534\du}}
\pgfpathlineto{\pgfpoint{27.076415\du}{9.458142\du}}
\pgfpathlineto{\pgfpoint{27.349142\du}{9.525685\du}}
\pgfpathlineto{\pgfpoint{27.205294\du}{9.317214\du}}
\pgfpathlineto{\pgfpoint{26.501022\du}{9.317214\du}}
\pgfpathlineto{\pgfpoint{26.795290\du}{9.389869\du}}
\pgfpathlineto{\pgfpoint{26.243264\du}{9.610022\du}}
\pgfusepath{fill}
\pgfsetbuttcap
\pgfsetmiterjoin
\pgfsetdash{}{0pt}
\definecolor{dialinecolor}{rgb}{0.074510, 0.082353, 0.086275}
\pgfsetfillcolor{dialinecolor}
\pgfpathmoveto{\pgfpoint{26.141402\du}{9.848065\du}}
\pgfpathlineto{\pgfpoint{25.893501\du}{9.765918\du}}
\pgfpathlineto{\pgfpoint{25.308251\du}{9.999580\du}}
\pgfpathlineto{\pgfpoint{25.035159\du}{9.932767\du}}
\pgfpathlineto{\pgfpoint{25.179007\du}{10.140507\du}}
\pgfpathlineto{\pgfpoint{25.884374\du}{10.140507\du}}
\pgfpathlineto{\pgfpoint{25.589376\du}{10.068583\du}}
\pgfpathlineto{\pgfpoint{26.141402\du}{9.848065\du}}
\pgfusepath{fill}
\pgfsetbuttcap
\pgfsetmiterjoin
\pgfsetdash{}{0pt}
\definecolor{dialinecolor}{rgb}{0.074510, 0.082353, 0.086275}
\pgfsetfillcolor{dialinecolor}
\pgfpathmoveto{\pgfpoint{25.095400\du}{9.389138\du}}
\pgfpathlineto{\pgfpoint{25.342935\du}{9.307357\du}}
\pgfpathlineto{\pgfpoint{25.928185\du}{9.540654\du}}
\pgfpathlineto{\pgfpoint{26.201278\du}{9.474206\du}}
\pgfpathlineto{\pgfpoint{26.057430\du}{9.681946\du}}
\pgfpathlineto{\pgfpoint{25.352428\du}{9.681946\du}}
\pgfpathlineto{\pgfpoint{25.647426\du}{9.610022\du}}
\pgfpathlineto{\pgfpoint{25.095400\du}{9.389138\du}}
\pgfusepath{fill}
\pgfsetbuttcap
\pgfsetmiterjoin
\pgfsetdash{}{0pt}
\definecolor{dialinecolor}{rgb}{0.074510, 0.082353, 0.086275}
\pgfsetfillcolor{dialinecolor}
\pgfpathmoveto{\pgfpoint{27.313728\du}{10.084283\du}}
\pgfpathlineto{\pgfpoint{27.066192\du}{10.166429\du}}
\pgfpathlineto{\pgfpoint{26.480942\du}{9.932767\du}}
\pgfpathlineto{\pgfpoint{26.207119\du}{9.999580\du}}
\pgfpathlineto{\pgfpoint{26.351698\du}{9.791840\du}}
\pgfpathlineto{\pgfpoint{27.057065\du}{9.791840\du}}
\pgfpathlineto{\pgfpoint{26.761701\du}{9.863764\du}}
\pgfpathlineto{\pgfpoint{27.313728\du}{10.084283\du}}
\pgfusepath{fill}
\pgfsetbuttcap
\pgfsetmiterjoin
\pgfsetdash{}{0pt}
\definecolor{dialinecolor}{rgb}{1.000000, 1.000000, 1.000000}
\pgfsetfillcolor{dialinecolor}
\pgfpathmoveto{\pgfpoint{26.264074\du}{9.630467\du}}
\pgfpathlineto{\pgfpoint{26.511610\du}{9.712979\du}}
\pgfpathlineto{\pgfpoint{27.096860\du}{9.478952\du}}
\pgfpathlineto{\pgfpoint{27.369222\du}{9.546130\du}}
\pgfpathlineto{\pgfpoint{27.226470\du}{9.337660\du}}
\pgfpathlineto{\pgfpoint{26.521103\du}{9.337660\du}}
\pgfpathlineto{\pgfpoint{26.816101\du}{9.410314\du}}
\pgfpathlineto{\pgfpoint{26.264074\du}{9.630467\du}}
\pgfusepath{fill}
\pgfsetbuttcap
\pgfsetmiterjoin
\pgfsetdash{}{0pt}
\definecolor{dialinecolor}{rgb}{1.000000, 1.000000, 1.000000}
\pgfsetfillcolor{dialinecolor}
\pgfpathmoveto{\pgfpoint{26.162212\du}{9.869241\du}}
\pgfpathlineto{\pgfpoint{25.913582\du}{9.786729\du}}
\pgfpathlineto{\pgfpoint{25.328697\du}{10.020756\du}}
\pgfpathlineto{\pgfpoint{25.055239\du}{9.953213\du}}
\pgfpathlineto{\pgfpoint{25.200183\du}{10.161683\du}}
\pgfpathlineto{\pgfpoint{25.904819\du}{10.161683\du}}
\pgfpathlineto{\pgfpoint{25.610186\du}{10.089029\du}}
\pgfpathlineto{\pgfpoint{26.162212\du}{9.869241\du}}
\pgfusepath{fill}
\pgfsetbuttcap
\pgfsetmiterjoin
\pgfsetdash{}{0pt}
\definecolor{dialinecolor}{rgb}{1.000000, 1.000000, 1.000000}
\pgfsetfillcolor{dialinecolor}
\pgfpathmoveto{\pgfpoint{25.115845\du}{9.409584\du}}
\pgfpathlineto{\pgfpoint{25.363381\du}{9.327802\du}}
\pgfpathlineto{\pgfpoint{25.948996\du}{9.561829\du}}
\pgfpathlineto{\pgfpoint{26.222088\du}{9.495016\du}}
\pgfpathlineto{\pgfpoint{26.077145\du}{9.702391\du}}
\pgfpathlineto{\pgfpoint{25.372873\du}{9.702391\du}}
\pgfpathlineto{\pgfpoint{25.667506\du}{9.630467\du}}
\pgfpathlineto{\pgfpoint{25.115845\du}{9.409584\du}}
\pgfusepath{fill}
\pgfsetbuttcap
\pgfsetmiterjoin
\pgfsetdash{}{0pt}
\definecolor{dialinecolor}{rgb}{1.000000, 1.000000, 1.000000}
\pgfsetfillcolor{dialinecolor}
\pgfpathmoveto{\pgfpoint{27.333808\du}{10.104728\du}}
\pgfpathlineto{\pgfpoint{27.086272\du}{10.186875\du}}
\pgfpathlineto{\pgfpoint{26.501387\du}{9.953213\du}}
\pgfpathlineto{\pgfpoint{26.227930\du}{10.020026\du}}
\pgfpathlineto{\pgfpoint{26.372143\du}{9.812286\du}}
\pgfpathlineto{\pgfpoint{27.077145\du}{9.812286\du}}
\pgfpathlineto{\pgfpoint{26.782877\du}{9.884210\du}}
\pgfpathlineto{\pgfpoint{27.333808\du}{10.104728\du}}
\pgfusepath{fill}
\pgfsetbuttcap
\pgfsetmiterjoin
\pgfsetdash{}{0pt}
\definecolor{dialinecolor}{rgb}{0.678431, 0.839216, 0.905882}
\pgfsetfillcolor{dialinecolor}
\pgfpathmoveto{\pgfpoint{24.529865\du}{9.748028\du}}
\pgfpathlineto{\pgfpoint{24.529865\du}{9.737441\du}}
\pgfpathlineto{\pgfpoint{24.509419\du}{9.737441\du}}
\pgfpathlineto{\pgfpoint{24.509419\du}{9.748028\du}}
\pgfpathlineto{\pgfpoint{24.529865\du}{9.748028\du}}
\pgfusepath{fill}
\pgfsetbuttcap
\pgfsetmiterjoin
\pgfsetdash{}{0pt}
\definecolor{dialinecolor}{rgb}{0.678431, 0.839216, 0.905882}
\pgfsetfillcolor{dialinecolor}
\pgfpathmoveto{\pgfpoint{24.529865\du}{10.577528\du}}
\pgfpathlineto{\pgfpoint{24.529865\du}{9.748028\du}}
\pgfpathlineto{\pgfpoint{24.509419\du}{9.748028\du}}
\pgfpathlineto{\pgfpoint{24.509419\du}{10.577528\du}}
\pgfpathlineto{\pgfpoint{24.529865\du}{10.577528\du}}
\pgfusepath{fill}
\pgfsetbuttcap
\pgfsetmiterjoin
\pgfsetdash{}{0pt}
\definecolor{dialinecolor}{rgb}{0.678431, 0.839216, 0.905882}
\pgfsetfillcolor{dialinecolor}
\pgfpathmoveto{\pgfpoint{24.509419\du}{10.577528\du}}
\pgfpathlineto{\pgfpoint{24.509419\du}{10.588116\du}}
\pgfpathlineto{\pgfpoint{24.529865\du}{10.588116\du}}
\pgfpathlineto{\pgfpoint{24.529865\du}{10.577528\du}}
\pgfpathlineto{\pgfpoint{24.509419\du}{10.577528\du}}
\pgfusepath{fill}
\pgfsetbuttcap
\pgfsetmiterjoin
\pgfsetdash{}{0pt}
\definecolor{dialinecolor}{rgb}{0.678431, 0.839216, 0.905882}
\pgfsetfillcolor{dialinecolor}
\pgfpathmoveto{\pgfpoint{27.890581\du}{9.748028\du}}
\pgfpathlineto{\pgfpoint{27.890581\du}{9.737441\du}}
\pgfpathlineto{\pgfpoint{27.870500\du}{9.737441\du}}
\pgfpathlineto{\pgfpoint{27.870500\du}{9.748028\du}}
\pgfpathlineto{\pgfpoint{27.890581\du}{9.748028\du}}
\pgfusepath{fill}
\pgfsetbuttcap
\pgfsetmiterjoin
\pgfsetdash{}{0pt}
\definecolor{dialinecolor}{rgb}{0.678431, 0.839216, 0.905882}
\pgfsetfillcolor{dialinecolor}
\pgfpathmoveto{\pgfpoint{27.890581\du}{10.577528\du}}
\pgfpathlineto{\pgfpoint{27.890581\du}{9.748028\du}}
\pgfpathlineto{\pgfpoint{27.870500\du}{9.748028\du}}
\pgfpathlineto{\pgfpoint{27.870500\du}{10.577528\du}}
\pgfpathlineto{\pgfpoint{27.890581\du}{10.577528\du}}
\pgfusepath{fill}
\pgfsetbuttcap
\pgfsetmiterjoin
\pgfsetdash{}{0pt}
\definecolor{dialinecolor}{rgb}{0.678431, 0.839216, 0.905882}
\pgfsetfillcolor{dialinecolor}
\pgfpathmoveto{\pgfpoint{27.870500\du}{10.577528\du}}
\pgfpathlineto{\pgfpoint{27.870500\du}{10.588116\du}}
\pgfpathlineto{\pgfpoint{27.890581\du}{10.588116\du}}
\pgfpathlineto{\pgfpoint{27.890581\du}{10.577528\du}}
\pgfpathlineto{\pgfpoint{27.870500\du}{10.577528\du}}
\pgfusepath{fill}
\pgfsetbuttcap
\pgfsetmiterjoin
\pgfsetdash{}{0pt}
\definecolor{dialinecolor}{rgb}{0.027451, 0.372549, 0.529412}
\pgfsetfillcolor{dialinecolor}
\pgfpathmoveto{\pgfpoint{26.816831\du}{10.726853\du}}
\pgfpathlineto{\pgfpoint{26.816466\du}{10.744378\du}}
\pgfpathlineto{\pgfpoint{26.813910\du}{10.761537\du}}
\pgfpathlineto{\pgfpoint{26.810259\du}{10.777601\du}}
\pgfpathlineto{\pgfpoint{26.804418\du}{10.794761\du}}
\pgfpathlineto{\pgfpoint{26.797846\du}{10.810460\du}}
\pgfpathlineto{\pgfpoint{26.789449\du}{10.826524\du}}
\pgfpathlineto{\pgfpoint{26.779956\du}{10.842223\du}}
\pgfpathlineto{\pgfpoint{26.768273\du}{10.857192\du}}
\pgfpathlineto{\pgfpoint{26.756590\du}{10.872161\du}}
\pgfpathlineto{\pgfpoint{26.743081\du}{10.886765\du}}
\pgfpathlineto{\pgfpoint{26.728112\du}{10.900639\du}}
\pgfpathlineto{\pgfpoint{26.712048\du}{10.914878\du}}
\pgfpathlineto{\pgfpoint{26.694524\du}{10.927656\du}}
\pgfpathlineto{\pgfpoint{26.676269\du}{10.940434\du}}
\pgfpathlineto{\pgfpoint{26.657284\du}{10.952848\du}}
\pgfpathlineto{\pgfpoint{26.637203\du}{10.964531\du}}
\pgfpathlineto{\pgfpoint{26.615663\du}{10.975119\du}}
\pgfpathlineto{\pgfpoint{26.593027\du}{10.986072\du}}
\pgfpathlineto{\pgfpoint{26.570026\du}{10.995929\du}}
\pgfpathlineto{\pgfpoint{26.546294\du}{11.005422\du}}
\pgfpathlineto{\pgfpoint{26.521103\du}{11.013454\du}}
\pgfpathlineto{\pgfpoint{26.495546\du}{11.021851\du}}
\pgfpathlineto{\pgfpoint{26.468894\du}{11.029518\du}}
\pgfpathlineto{\pgfpoint{26.442242\du}{11.036455\du}}
\pgfpathlineto{\pgfpoint{26.414129\du}{11.042296\du}}
\pgfpathlineto{\pgfpoint{26.385652\du}{11.047408\du}}
\pgfpathlineto{\pgfpoint{26.356079\du}{11.051789\du}}
\pgfpathlineto{\pgfpoint{26.325776\du}{11.055805\du}}
\pgfpathlineto{\pgfpoint{26.295838\du}{11.058726\du}}
\pgfpathlineto{\pgfpoint{26.265170\du}{11.060916\du}}
\pgfpathlineto{\pgfpoint{26.233771\du}{11.062012\du}}
\pgfpathlineto{\pgfpoint{26.202373\du}{11.062742\du}}
\pgfpathlineto{\pgfpoint{26.171340\du}{11.062012\du}}
\pgfpathlineto{\pgfpoint{26.139942\du}{11.060916\du}}
\pgfpathlineto{\pgfpoint{26.108908\du}{11.058726\du}}
\pgfpathlineto{\pgfpoint{26.078605\du}{11.055805\du}}
\pgfpathlineto{\pgfpoint{26.049032\du}{11.051789\du}}
\pgfpathlineto{\pgfpoint{26.019460\du}{11.047408\du}}
\pgfpathlineto{\pgfpoint{25.990982\du}{11.042296\du}}
\pgfpathlineto{\pgfpoint{25.963235\du}{11.036455\du}}
\pgfpathlineto{\pgfpoint{25.935853\du}{11.029518\du}}
\pgfpathlineto{\pgfpoint{25.909566\du}{11.021851\du}}
\pgfpathlineto{\pgfpoint{25.884374\du}{11.013454\du}}
\pgfpathlineto{\pgfpoint{25.858817\du}{11.005422\du}}
\pgfpathlineto{\pgfpoint{25.834721\du}{10.995929\du}}
\pgfpathlineto{\pgfpoint{25.812085\du}{10.986072\du}}
\pgfpathlineto{\pgfpoint{25.789449\du}{10.975119\du}}
\pgfpathlineto{\pgfpoint{25.768638\du}{10.964531\du}}
\pgfpathlineto{\pgfpoint{25.748193\du}{10.952848\du}}
\pgfpathlineto{\pgfpoint{25.727747\du}{10.940434\du}}
\pgfpathlineto{\pgfpoint{25.709858\du}{10.927656\du}}
\pgfpathlineto{\pgfpoint{25.693428\du}{10.914878\du}}
\pgfpathlineto{\pgfpoint{25.676634\du}{10.900639\du}}
\pgfpathlineto{\pgfpoint{25.662030\du}{10.886765\du}}
\pgfpathlineto{\pgfpoint{25.648886\du}{10.872161\du}}
\pgfpathlineto{\pgfpoint{25.636473\du}{10.857192\du}}
\pgfpathlineto{\pgfpoint{25.625520\du}{10.842223\du}}
\pgfpathlineto{\pgfpoint{25.616028\du}{10.826524\du}}
\pgfpathlineto{\pgfpoint{25.607265\du}{10.810460\du}}
\pgfpathlineto{\pgfpoint{25.600329\du}{10.794761\du}}
\pgfpathlineto{\pgfpoint{25.595582\du}{10.777601\du}}
\pgfpathlineto{\pgfpoint{25.591201\du}{10.761537\du}}
\pgfpathlineto{\pgfpoint{25.588645\du}{10.744378\du}}
\pgfpathlineto{\pgfpoint{25.587550\du}{10.726853\du}}
\pgfpathlineto{\pgfpoint{25.588645\du}{10.709328\du}}
\pgfpathlineto{\pgfpoint{25.591201\du}{10.692169\du}}
\pgfpathlineto{\pgfpoint{25.595582\du}{10.676104\du}}
\pgfpathlineto{\pgfpoint{25.600329\du}{10.658945\du}}
\pgfpathlineto{\pgfpoint{25.607265\du}{10.643246\du}}
\pgfpathlineto{\pgfpoint{25.616028\du}{10.627547\du}}
\pgfpathlineto{\pgfpoint{25.625520\du}{10.611482\du}}
\pgfpathlineto{\pgfpoint{25.636473\du}{10.596513\du}}
\pgfpathlineto{\pgfpoint{25.648886\du}{10.581909\du}}
\pgfpathlineto{\pgfpoint{25.662030\du}{10.567306\du}}
\pgfpathlineto{\pgfpoint{25.676634\du}{10.553067\du}}
\pgfpathlineto{\pgfpoint{25.693428\du}{10.539193\du}}
\pgfpathlineto{\pgfpoint{25.709858\du}{10.526050\du}}
\pgfpathlineto{\pgfpoint{25.727747\du}{10.514001\du}}
\pgfpathlineto{\pgfpoint{25.748193\du}{10.501588\du}}
\pgfpathlineto{\pgfpoint{25.768638\du}{10.489905\du}}
\pgfpathlineto{\pgfpoint{25.789449\du}{10.478952\du}}
\pgfpathlineto{\pgfpoint{25.812085\du}{10.468364\du}}
\pgfpathlineto{\pgfpoint{25.834721\du}{10.457777\du}}
\pgfpathlineto{\pgfpoint{25.858817\du}{10.449014\du}}
\pgfpathlineto{\pgfpoint{25.884374\du}{10.440252\du}}
\pgfpathlineto{\pgfpoint{25.909566\du}{10.431855\du}}
\pgfpathlineto{\pgfpoint{25.935853\du}{10.424188\du}}
\pgfpathlineto{\pgfpoint{25.963235\du}{10.417981\du}}
\pgfpathlineto{\pgfpoint{25.990982\du}{10.411409\du}}
\pgfpathlineto{\pgfpoint{26.019460\du}{10.406298\du}}
\pgfpathlineto{\pgfpoint{26.049032\du}{10.402282\du}}
\pgfpathlineto{\pgfpoint{26.078605\du}{10.397901\du}}
\pgfpathlineto{\pgfpoint{26.108908\du}{10.394980\du}}
\pgfpathlineto{\pgfpoint{26.139942\du}{10.393520\du}}
\pgfpathlineto{\pgfpoint{26.171340\du}{10.391694\du}}
\pgfpathlineto{\pgfpoint{26.202373\du}{10.391694\du}}
\pgfpathlineto{\pgfpoint{26.233771\du}{10.391694\du}}
\pgfpathlineto{\pgfpoint{26.265170\du}{10.393520\du}}
\pgfpathlineto{\pgfpoint{26.295838\du}{10.394980\du}}
\pgfpathlineto{\pgfpoint{26.325776\du}{10.397901\du}}
\pgfpathlineto{\pgfpoint{26.356079\du}{10.402282\du}}
\pgfpathlineto{\pgfpoint{26.385652\du}{10.406298\du}}
\pgfpathlineto{\pgfpoint{26.414129\du}{10.411409\du}}
\pgfpathlineto{\pgfpoint{26.442242\du}{10.417981\du}}
\pgfpathlineto{\pgfpoint{26.468894\du}{10.424188\du}}
\pgfpathlineto{\pgfpoint{26.495546\du}{10.431855\du}}
\pgfpathlineto{\pgfpoint{26.521103\du}{10.440252\du}}
\pgfpathlineto{\pgfpoint{26.546294\du}{10.449014\du}}
\pgfpathlineto{\pgfpoint{26.570026\du}{10.457777\du}}
\pgfpathlineto{\pgfpoint{26.593027\du}{10.468364\du}}
\pgfpathlineto{\pgfpoint{26.615663\du}{10.478952\du}}
\pgfpathlineto{\pgfpoint{26.637203\du}{10.489905\du}}
\pgfpathlineto{\pgfpoint{26.657284\du}{10.501588\du}}
\pgfpathlineto{\pgfpoint{26.676269\du}{10.514001\du}}
\pgfpathlineto{\pgfpoint{26.694524\du}{10.526050\du}}
\pgfpathlineto{\pgfpoint{26.712048\du}{10.539193\du}}
\pgfpathlineto{\pgfpoint{26.728112\du}{10.553067\du}}
\pgfpathlineto{\pgfpoint{26.743081\du}{10.567306\du}}
\pgfpathlineto{\pgfpoint{26.756590\du}{10.581909\du}}
\pgfpathlineto{\pgfpoint{26.768273\du}{10.596513\du}}
\pgfpathlineto{\pgfpoint{26.779956\du}{10.611482\du}}
\pgfpathlineto{\pgfpoint{26.789449\du}{10.627547\du}}
\pgfpathlineto{\pgfpoint{26.797846\du}{10.643246\du}}
\pgfpathlineto{\pgfpoint{26.804418\du}{10.658945\du}}
\pgfpathlineto{\pgfpoint{26.810259\du}{10.676104\du}}
\pgfpathlineto{\pgfpoint{26.813910\du}{10.692169\du}}
\pgfpathlineto{\pgfpoint{26.816466\du}{10.709328\du}}
\pgfpathlineto{\pgfpoint{26.816831\du}{10.726853\du}}
\pgfusepath{fill}
\pgfsetbuttcap
\pgfsetmiterjoin
\pgfsetdash{}{0pt}
\definecolor{dialinecolor}{rgb}{0.678431, 0.839216, 0.905882}
\pgfsetfillcolor{dialinecolor}
\pgfpathmoveto{\pgfpoint{26.202373\du}{11.072599\du}}
\pgfpathlineto{\pgfpoint{26.202373\du}{11.072599\du}}
\pgfpathlineto{\pgfpoint{26.218437\du}{11.072599\du}}
\pgfpathlineto{\pgfpoint{26.234502\du}{11.072234\du}}
\pgfpathlineto{\pgfpoint{26.250566\du}{11.071504\du}}
\pgfpathlineto{\pgfpoint{26.265900\du}{11.070774\du}}
\pgfpathlineto{\pgfpoint{26.281599\du}{11.069679\du}}
\pgfpathlineto{\pgfpoint{26.296933\du}{11.068583\du}}
\pgfpathlineto{\pgfpoint{26.311902\du}{11.067488\du}}
\pgfpathlineto{\pgfpoint{26.327966\du}{11.065663\du}}
\pgfpathlineto{\pgfpoint{26.342205\du}{11.063837\du}}
\pgfpathlineto{\pgfpoint{26.357174\du}{11.062012\du}}
\pgfpathlineto{\pgfpoint{26.372143\du}{11.059821\du}}
\pgfpathlineto{\pgfpoint{26.387477\du}{11.057631\du}}
\pgfpathlineto{\pgfpoint{26.401716\du}{11.054710\du}}
\pgfpathlineto{\pgfpoint{26.415590\du}{11.052154\du}}
\pgfpathlineto{\pgfpoint{26.430194\du}{11.049233\du}}
\pgfpathlineto{\pgfpoint{26.444067\du}{11.046313\du}}
\pgfpathlineto{\pgfpoint{26.457576\du}{11.042662\du}}
\pgfpathlineto{\pgfpoint{26.471449\du}{11.039376\du}}
\pgfpathlineto{\pgfpoint{26.484958\du}{11.035725\du}}
\pgfpathlineto{\pgfpoint{26.498101\du}{11.031709\du}}
\pgfpathlineto{\pgfpoint{26.511245\du}{11.027693\du}}
\pgfpathlineto{\pgfpoint{26.524388\du}{11.023677\du}}
\pgfpathlineto{\pgfpoint{26.537167\du}{11.018930\du}}
\pgfpathlineto{\pgfpoint{26.549945\du}{11.014914\du}}
\pgfpathlineto{\pgfpoint{26.561628\du}{11.010168\du}}
\pgfpathlineto{\pgfpoint{26.574407\du}{11.005422\du}}
\pgfpathlineto{\pgfpoint{26.585725\du}{10.999945\du}}
\pgfpathlineto{\pgfpoint{26.597408\du}{10.994834\du}}
\pgfpathlineto{\pgfpoint{26.608726\du}{10.989723\du}}
\pgfpathlineto{\pgfpoint{26.620044\du}{10.984246\du}}
\pgfpathlineto{\pgfpoint{26.630997\du}{10.979135\du}}
\pgfpathlineto{\pgfpoint{26.642315\du}{10.973293\du}}
\pgfpathlineto{\pgfpoint{26.652172\du}{10.967452\du}}
\pgfpathlineto{\pgfpoint{26.662395\du}{10.960880\du}}
\pgfpathlineto{\pgfpoint{26.672253\du}{10.955038\du}}
\pgfpathlineto{\pgfpoint{26.682475\du}{10.948467\du}}
\pgfpathlineto{\pgfpoint{26.691968\du}{10.942260\du}}
\pgfpathlineto{\pgfpoint{26.701095\du}{10.935688\du}}
\pgfpathlineto{\pgfpoint{26.709858\du}{10.929482\du}}
\pgfpathlineto{\pgfpoint{26.718255\du}{10.922180\du}}
\pgfpathlineto{\pgfpoint{26.726652\du}{10.915243\du}}
\pgfpathlineto{\pgfpoint{26.734684\du}{10.908306\du}}
\pgfpathlineto{\pgfpoint{26.742716\du}{10.901369\du}}
\pgfpathlineto{\pgfpoint{26.749653\du}{10.894067\du}}
\pgfpathlineto{\pgfpoint{26.757320\du}{10.886765\du}}
\pgfpathlineto{\pgfpoint{26.763892\du}{10.879098\du}}
\pgfpathlineto{\pgfpoint{26.770464\du}{10.871431\du}}
\pgfpathlineto{\pgfpoint{26.776670\du}{10.863764\du}}
\pgfpathlineto{\pgfpoint{26.782877\du}{10.855732\du}}
\pgfpathlineto{\pgfpoint{26.788719\du}{10.848065\du}}
\pgfpathlineto{\pgfpoint{26.793465\du}{10.839668\du}}
\pgfpathlineto{\pgfpoint{26.798211\du}{10.831636\du}}
\pgfpathlineto{\pgfpoint{26.802957\du}{10.823238\du}}
\pgfpathlineto{\pgfpoint{26.807338\du}{10.815206\du}}
\pgfpathlineto{\pgfpoint{26.810624\du}{10.806444\du}}
\pgfpathlineto{\pgfpoint{26.814640\du}{10.798412\du}}
\pgfpathlineto{\pgfpoint{26.817196\du}{10.789650\du}}
\pgfpathlineto{\pgfpoint{26.820117\du}{10.780887\du}}
\pgfpathlineto{\pgfpoint{26.821942\du}{10.771760\du}}
\pgfpathlineto{\pgfpoint{26.824133\du}{10.762997\du}}
\pgfpathlineto{\pgfpoint{26.825593\du}{10.754235\du}}
\pgfpathlineto{\pgfpoint{26.826689\du}{10.745108\du}}
\pgfpathlineto{\pgfpoint{26.827419\du}{10.736345\du}}
\pgfpathlineto{\pgfpoint{26.827419\du}{10.726853\du}}
\pgfpathlineto{\pgfpoint{26.807338\du}{10.726853\du}}
\pgfpathlineto{\pgfpoint{26.806608\du}{10.734885\du}}
\pgfpathlineto{\pgfpoint{26.806243\du}{10.743282\du}}
\pgfpathlineto{\pgfpoint{26.805513\du}{10.751314\du}}
\pgfpathlineto{\pgfpoint{26.804053\du}{10.758981\du}}
\pgfpathlineto{\pgfpoint{26.802592\du}{10.767379\du}}
\pgfpathlineto{\pgfpoint{26.800037\du}{10.775411\du}}
\pgfpathlineto{\pgfpoint{26.797846\du}{10.783078\du}}
\pgfpathlineto{\pgfpoint{26.794560\du}{10.790745\du}}
\pgfpathlineto{\pgfpoint{26.792004\du}{10.798412\du}}
\pgfpathlineto{\pgfpoint{26.788719\du}{10.806444\du}}
\pgfpathlineto{\pgfpoint{26.784702\du}{10.814111\du}}
\pgfpathlineto{\pgfpoint{26.779956\du}{10.821778\du}}
\pgfpathlineto{\pgfpoint{26.775940\du}{10.829445\du}}
\pgfpathlineto{\pgfpoint{26.770829\du}{10.836382\du}}
\pgfpathlineto{\pgfpoint{26.766083\du}{10.844049\du}}
\pgfpathlineto{\pgfpoint{26.760606\du}{10.850986\du}}
\pgfpathlineto{\pgfpoint{26.755495\du}{10.858653\du}}
\pgfpathlineto{\pgfpoint{26.748558\du}{10.865590\du}}
\pgfpathlineto{\pgfpoint{26.742716\du}{10.872526\du}}
\pgfpathlineto{\pgfpoint{26.735414\du}{10.879463\du}}
\pgfpathlineto{\pgfpoint{26.728478\du}{10.886765\du}}
\pgfpathlineto{\pgfpoint{26.721176\du}{10.892972\du}}
\pgfpathlineto{\pgfpoint{26.714239\du}{10.899909\du}}
\pgfpathlineto{\pgfpoint{26.705842\du}{10.906480\du}}
\pgfpathlineto{\pgfpoint{26.697809\du}{10.913052\du}}
\pgfpathlineto{\pgfpoint{26.688682\du}{10.919259\du}}
\pgfpathlineto{\pgfpoint{26.679920\du}{10.925100\du}}
\pgfpathlineto{\pgfpoint{26.670427\du}{10.931672\du}}
\pgfpathlineto{\pgfpoint{26.661300\du}{10.938244\du}}
\pgfpathlineto{\pgfpoint{26.652172\du}{10.943355\du}}
\pgfpathlineto{\pgfpoint{26.642315\du}{10.949197\du}}
\pgfpathlineto{\pgfpoint{26.632457\du}{10.955038\du}}
\pgfpathlineto{\pgfpoint{26.621869\du}{10.960515\du}}
\pgfpathlineto{\pgfpoint{26.611281\du}{10.966356\du}}
\pgfpathlineto{\pgfpoint{26.600694\du}{10.970737\du}}
\pgfpathlineto{\pgfpoint{26.589011\du}{10.976579\du}}
\pgfpathlineto{\pgfpoint{26.577693\du}{10.981325\du}}
\pgfpathlineto{\pgfpoint{26.566375\du}{10.986072\du}}
\pgfpathlineto{\pgfpoint{26.554691\du}{10.990818\du}}
\pgfpathlineto{\pgfpoint{26.542643\du}{10.995564\du}}
\pgfpathlineto{\pgfpoint{26.530230\du}{10.999580\du}}
\pgfpathlineto{\pgfpoint{26.518547\du}{11.004326\du}}
\pgfpathlineto{\pgfpoint{26.505769\du}{11.008342\du}}
\pgfpathlineto{\pgfpoint{26.492260\du}{11.011993\du}}
\pgfpathlineto{\pgfpoint{26.479482\du}{11.016009\du}}
\pgfpathlineto{\pgfpoint{26.466338\du}{11.019295\du}}
\pgfpathlineto{\pgfpoint{26.452829\du}{11.022946\du}}
\pgfpathlineto{\pgfpoint{26.439321\du}{11.025867\du}}
\pgfpathlineto{\pgfpoint{26.425447\du}{11.029518\du}}
\pgfpathlineto{\pgfpoint{26.411574\du}{11.031709\du}}
\pgfpathlineto{\pgfpoint{26.397700\du}{11.034629\du}}
\pgfpathlineto{\pgfpoint{26.383461\du}{11.036820\du}}
\pgfpathlineto{\pgfpoint{26.369222\du}{11.039376\du}}
\pgfpathlineto{\pgfpoint{26.354984\du}{11.041566\du}}
\pgfpathlineto{\pgfpoint{26.340380\du}{11.043392\du}}
\pgfpathlineto{\pgfpoint{26.325411\du}{11.045217\du}}
\pgfpathlineto{\pgfpoint{26.310077\du}{11.047043\du}}
\pgfpathlineto{\pgfpoint{26.295473\du}{11.048138\du}}
\pgfpathlineto{\pgfpoint{26.279774\du}{11.049233\du}}
\pgfpathlineto{\pgfpoint{26.264805\du}{11.050329\du}}
\pgfpathlineto{\pgfpoint{26.249471\du}{11.051059\du}}
\pgfpathlineto{\pgfpoint{26.233771\du}{11.051789\du}}
\pgfpathlineto{\pgfpoint{26.218437\du}{11.052154\du}}
\pgfpathlineto{\pgfpoint{26.202373\du}{11.052154\du}}
\pgfpathlineto{\pgfpoint{26.202373\du}{11.052154\du}}
\pgfpathlineto{\pgfpoint{26.202373\du}{11.052154\du}}
\pgfpathlineto{\pgfpoint{26.201278\du}{11.052154\du}}
\pgfpathlineto{\pgfpoint{26.200183\du}{11.052154\du}}
\pgfpathlineto{\pgfpoint{26.199452\du}{11.052884\du}}
\pgfpathlineto{\pgfpoint{26.197627\du}{11.052884\du}}
\pgfpathlineto{\pgfpoint{26.197262\du}{11.053249\du}}
\pgfpathlineto{\pgfpoint{26.196166\du}{11.053980\du}}
\pgfpathlineto{\pgfpoint{26.195801\du}{11.054710\du}}
\pgfpathlineto{\pgfpoint{26.195436\du}{11.055075\du}}
\pgfpathlineto{\pgfpoint{26.193976\du}{11.056900\du}}
\pgfpathlineto{\pgfpoint{26.192516\du}{11.058726\du}}
\pgfpathlineto{\pgfpoint{26.192516\du}{11.060551\du}}
\pgfpathlineto{\pgfpoint{26.192150\du}{11.062742\du}}
\pgfpathlineto{\pgfpoint{26.192516\du}{11.064567\du}}
\pgfpathlineto{\pgfpoint{26.192516\du}{11.066393\du}}
\pgfpathlineto{\pgfpoint{26.193976\du}{11.067853\du}}
\pgfpathlineto{\pgfpoint{26.195436\du}{11.069314\du}}
\pgfpathlineto{\pgfpoint{26.195801\du}{11.070409\du}}
\pgfpathlineto{\pgfpoint{26.196166\du}{11.070774\du}}
\pgfpathlineto{\pgfpoint{26.197262\du}{11.071504\du}}
\pgfpathlineto{\pgfpoint{26.197627\du}{11.071504\du}}
\pgfpathlineto{\pgfpoint{26.199452\du}{11.072234\du}}
\pgfpathlineto{\pgfpoint{26.200183\du}{11.072599\du}}
\pgfpathlineto{\pgfpoint{26.201278\du}{11.072599\du}}
\pgfpathlineto{\pgfpoint{26.202373\du}{11.072599\du}}
\pgfusepath{fill}
\pgfsetbuttcap
\pgfsetmiterjoin
\pgfsetdash{}{0pt}
\definecolor{dialinecolor}{rgb}{0.678431, 0.839216, 0.905882}
\pgfsetfillcolor{dialinecolor}
\pgfpathmoveto{\pgfpoint{25.577693\du}{10.726853\du}}
\pgfpathlineto{\pgfpoint{25.577693\du}{10.726853\du}}
\pgfpathlineto{\pgfpoint{25.577693\du}{10.735615\du}}
\pgfpathlineto{\pgfpoint{25.578423\du}{10.745108\du}}
\pgfpathlineto{\pgfpoint{25.579518\du}{10.754235\du}}
\pgfpathlineto{\pgfpoint{25.581709\du}{10.762997\du}}
\pgfpathlineto{\pgfpoint{25.582804\du}{10.771760\du}}
\pgfpathlineto{\pgfpoint{25.585360\du}{10.780887\du}}
\pgfpathlineto{\pgfpoint{25.587550\du}{10.789650\du}}
\pgfpathlineto{\pgfpoint{25.590836\du}{10.798412\du}}
\pgfpathlineto{\pgfpoint{25.594122\du}{10.806444\du}}
\pgfpathlineto{\pgfpoint{25.598138\du}{10.815206\du}}
\pgfpathlineto{\pgfpoint{25.602519\du}{10.823238\du}}
\pgfpathlineto{\pgfpoint{25.606900\du}{10.831636\du}}
\pgfpathlineto{\pgfpoint{25.611647\du}{10.839668\du}}
\pgfpathlineto{\pgfpoint{25.616758\du}{10.848065\du}}
\pgfpathlineto{\pgfpoint{25.622965\du}{10.855732\du}}
\pgfpathlineto{\pgfpoint{25.627711\du}{10.863764\du}}
\pgfpathlineto{\pgfpoint{25.634283\du}{10.871431\du}}
\pgfpathlineto{\pgfpoint{25.641219\du}{10.879098\du}}
\pgfpathlineto{\pgfpoint{25.647791\du}{10.886765\du}}
\pgfpathlineto{\pgfpoint{25.654728\du}{10.894067\du}}
\pgfpathlineto{\pgfpoint{25.662030\du}{10.901369\du}}
\pgfpathlineto{\pgfpoint{25.670792\du}{10.908306\du}}
\pgfpathlineto{\pgfpoint{25.677729\du}{10.915243\du}}
\pgfpathlineto{\pgfpoint{25.686857\du}{10.922180\du}}
\pgfpathlineto{\pgfpoint{25.694889\du}{10.929482\du}}
\pgfpathlineto{\pgfpoint{25.704016\du}{10.935688\du}}
\pgfpathlineto{\pgfpoint{25.713874\du}{10.942260\du}}
\pgfpathlineto{\pgfpoint{25.723001\du}{10.948467\du}}
\pgfpathlineto{\pgfpoint{25.732494\du}{10.955038\du}}
\pgfpathlineto{\pgfpoint{25.742351\du}{10.960880\du}}
\pgfpathlineto{\pgfpoint{25.752574\du}{10.967452\du}}
\pgfpathlineto{\pgfpoint{25.762797\du}{10.973293\du}}
\pgfpathlineto{\pgfpoint{25.773750\du}{10.979135\du}}
\pgfpathlineto{\pgfpoint{25.785068\du}{10.984246\du}}
\pgfpathlineto{\pgfpoint{25.796386\du}{10.989723\du}}
\pgfpathlineto{\pgfpoint{25.807704\du}{10.994834\du}}
\pgfpathlineto{\pgfpoint{25.819752\du}{10.999945\du}}
\pgfpathlineto{\pgfpoint{25.831070\du}{11.005422\du}}
\pgfpathlineto{\pgfpoint{25.843483\du}{11.010168\du}}
\pgfpathlineto{\pgfpoint{25.855531\du}{11.014914\du}}
\pgfpathlineto{\pgfpoint{25.867579\du}{11.018930\du}}
\pgfpathlineto{\pgfpoint{25.880723\du}{11.023677\du}}
\pgfpathlineto{\pgfpoint{25.893501\du}{11.027693\du}}
\pgfpathlineto{\pgfpoint{25.907010\du}{11.031709\du}}
\pgfpathlineto{\pgfpoint{25.920518\du}{11.035725\du}}
\pgfpathlineto{\pgfpoint{25.933297\du}{11.039376\du}}
\pgfpathlineto{\pgfpoint{25.946805\du}{11.042662\du}}
\pgfpathlineto{\pgfpoint{25.960679\du}{11.046313\du}}
\pgfpathlineto{\pgfpoint{25.975283\du}{11.049233\du}}
\pgfpathlineto{\pgfpoint{25.989887\du}{11.052154\du}}
\pgfpathlineto{\pgfpoint{26.003760\du}{11.054710\du}}
\pgfpathlineto{\pgfpoint{26.017999\du}{11.057631\du}}
\pgfpathlineto{\pgfpoint{26.032968\du}{11.059821\du}}
\pgfpathlineto{\pgfpoint{26.047937\du}{11.062012\du}}
\pgfpathlineto{\pgfpoint{26.062541\du}{11.063837\du}}
\pgfpathlineto{\pgfpoint{26.077145\du}{11.065663\du}}
\pgfpathlineto{\pgfpoint{26.092844\du}{11.067488\du}}
\pgfpathlineto{\pgfpoint{26.108543\du}{11.068583\du}}
\pgfpathlineto{\pgfpoint{26.123147\du}{11.069679\du}}
\pgfpathlineto{\pgfpoint{26.139211\du}{11.070774\du}}
\pgfpathlineto{\pgfpoint{26.154545\du}{11.071504\du}}
\pgfpathlineto{\pgfpoint{26.170610\du}{11.072234\du}}
\pgfpathlineto{\pgfpoint{26.186674\du}{11.072599\du}}
\pgfpathlineto{\pgfpoint{26.202373\du}{11.072599\du}}
\pgfpathlineto{\pgfpoint{26.202373\du}{11.052154\du}}
\pgfpathlineto{\pgfpoint{26.186674\du}{11.052154\du}}
\pgfpathlineto{\pgfpoint{26.171340\du}{11.051789\du}}
\pgfpathlineto{\pgfpoint{26.155276\du}{11.051059\du}}
\pgfpathlineto{\pgfpoint{26.140672\du}{11.050329\du}}
\pgfpathlineto{\pgfpoint{26.124973\du}{11.049233\du}}
\pgfpathlineto{\pgfpoint{26.110004\du}{11.048138\du}}
\pgfpathlineto{\pgfpoint{26.095035\du}{11.047043\du}}
\pgfpathlineto{\pgfpoint{26.080066\du}{11.045217\du}}
\pgfpathlineto{\pgfpoint{26.065097\du}{11.043392\du}}
\pgfpathlineto{\pgfpoint{26.050493\du}{11.041566\du}}
\pgfpathlineto{\pgfpoint{26.035889\du}{11.039376\du}}
\pgfpathlineto{\pgfpoint{26.021650\du}{11.036820\du}}
\pgfpathlineto{\pgfpoint{26.007777\du}{11.034629\du}}
\pgfpathlineto{\pgfpoint{25.993903\du}{11.031709\du}}
\pgfpathlineto{\pgfpoint{25.979664\du}{11.029518\du}}
\pgfpathlineto{\pgfpoint{25.965790\du}{11.025867\du}}
\pgfpathlineto{\pgfpoint{25.952282\du}{11.022946\du}}
\pgfpathlineto{\pgfpoint{25.939138\du}{11.019295\du}}
\pgfpathlineto{\pgfpoint{25.925630\du}{11.016009\du}}
\pgfpathlineto{\pgfpoint{25.912851\du}{11.011993\du}}
\pgfpathlineto{\pgfpoint{25.899708\du}{11.008342\du}}
\pgfpathlineto{\pgfpoint{25.886930\du}{11.004326\du}}
\pgfpathlineto{\pgfpoint{25.874881\du}{10.999580\du}}
\pgfpathlineto{\pgfpoint{25.862468\du}{10.995564\du}}
\pgfpathlineto{\pgfpoint{25.850055\du}{10.990818\du}}
\pgfpathlineto{\pgfpoint{25.838737\du}{10.986072\du}}
\pgfpathlineto{\pgfpoint{25.827054\du}{10.981325\du}}
\pgfpathlineto{\pgfpoint{25.816101\du}{10.976579\du}}
\pgfpathlineto{\pgfpoint{25.804783\du}{10.970737\du}}
\pgfpathlineto{\pgfpoint{25.794195\du}{10.966356\du}}
\pgfpathlineto{\pgfpoint{25.783242\du}{10.960515\du}}
\pgfpathlineto{\pgfpoint{25.773019\du}{10.955038\du}}
\pgfpathlineto{\pgfpoint{25.762797\du}{10.949197\du}}
\pgfpathlineto{\pgfpoint{25.752939\du}{10.943355\du}}
\pgfpathlineto{\pgfpoint{25.743447\du}{10.938244\du}}
\pgfpathlineto{\pgfpoint{25.733589\du}{10.931672\du}}
\pgfpathlineto{\pgfpoint{25.725192\du}{10.925100\du}}
\pgfpathlineto{\pgfpoint{25.716064\du}{10.919259\du}}
\pgfpathlineto{\pgfpoint{25.707667\du}{10.913052\du}}
\pgfpathlineto{\pgfpoint{25.699270\du}{10.906480\du}}
\pgfpathlineto{\pgfpoint{25.691238\du}{10.899909\du}}
\pgfpathlineto{\pgfpoint{25.683936\du}{10.892972\du}}
\pgfpathlineto{\pgfpoint{25.676269\du}{10.886765\du}}
\pgfpathlineto{\pgfpoint{25.669697\du}{10.879463\du}}
\pgfpathlineto{\pgfpoint{25.662760\du}{10.872526\du}}
\pgfpathlineto{\pgfpoint{25.656553\du}{10.865590\du}}
\pgfpathlineto{\pgfpoint{25.650347\du}{10.858653\du}}
\pgfpathlineto{\pgfpoint{25.644140\du}{10.850986\du}}
\pgfpathlineto{\pgfpoint{25.639029\du}{10.844049\du}}
\pgfpathlineto{\pgfpoint{25.633917\du}{10.836382\du}}
\pgfpathlineto{\pgfpoint{25.629171\du}{10.829445\du}}
\pgfpathlineto{\pgfpoint{25.624790\du}{10.821778\du}}
\pgfpathlineto{\pgfpoint{25.620409\du}{10.814111\du}}
\pgfpathlineto{\pgfpoint{25.616758\du}{10.806444\du}}
\pgfpathlineto{\pgfpoint{25.613472\du}{10.798412\du}}
\pgfpathlineto{\pgfpoint{25.610186\du}{10.790745\du}}
\pgfpathlineto{\pgfpoint{25.607265\du}{10.783078\du}}
\pgfpathlineto{\pgfpoint{25.604710\du}{10.775411\du}}
\pgfpathlineto{\pgfpoint{25.602884\du}{10.767379\du}}
\pgfpathlineto{\pgfpoint{25.601059\du}{10.758981\du}}
\pgfpathlineto{\pgfpoint{25.599963\du}{10.751314\du}}
\pgfpathlineto{\pgfpoint{25.598503\du}{10.743282\du}}
\pgfpathlineto{\pgfpoint{25.598138\du}{10.734885\du}}
\pgfpathlineto{\pgfpoint{25.598138\du}{10.726853\du}}
\pgfpathlineto{\pgfpoint{25.598138\du}{10.726853\du}}
\pgfpathlineto{\pgfpoint{25.598138\du}{10.726853\du}}
\pgfpathlineto{\pgfpoint{25.598138\du}{10.725758\du}}
\pgfpathlineto{\pgfpoint{25.598138\du}{10.724662\du}}
\pgfpathlineto{\pgfpoint{25.597773\du}{10.723202\du}}
\pgfpathlineto{\pgfpoint{25.597773\du}{10.722107\du}}
\pgfpathlineto{\pgfpoint{25.597408\du}{10.721742\du}}
\pgfpathlineto{\pgfpoint{25.596313\du}{10.720281\du}}
\pgfpathlineto{\pgfpoint{25.595582\du}{10.719916\du}}
\pgfpathlineto{\pgfpoint{25.595582\du}{10.719186\du}}
\pgfpathlineto{\pgfpoint{25.593392\du}{10.718091\du}}
\pgfpathlineto{\pgfpoint{25.591566\du}{10.717360\du}}
\pgfpathlineto{\pgfpoint{25.589741\du}{10.716995\du}}
\pgfpathlineto{\pgfpoint{25.587550\du}{10.716265\du}}
\pgfpathlineto{\pgfpoint{25.586090\du}{10.716995\du}}
\pgfpathlineto{\pgfpoint{25.584264\du}{10.717360\du}}
\pgfpathlineto{\pgfpoint{25.582439\du}{10.718091\du}}
\pgfpathlineto{\pgfpoint{25.580613\du}{10.719186\du}}
\pgfpathlineto{\pgfpoint{25.579883\du}{10.719916\du}}
\pgfpathlineto{\pgfpoint{25.579518\du}{10.720281\du}}
\pgfpathlineto{\pgfpoint{25.579153\du}{10.721742\du}}
\pgfpathlineto{\pgfpoint{25.578423\du}{10.722107\du}}
\pgfpathlineto{\pgfpoint{25.578423\du}{10.723202\du}}
\pgfpathlineto{\pgfpoint{25.577693\du}{10.724662\du}}
\pgfpathlineto{\pgfpoint{25.577693\du}{10.725758\du}}
\pgfpathlineto{\pgfpoint{25.577693\du}{10.726853\du}}
\pgfusepath{fill}
\pgfsetbuttcap
\pgfsetmiterjoin
\pgfsetdash{}{0pt}
\definecolor{dialinecolor}{rgb}{0.678431, 0.839216, 0.905882}
\pgfsetfillcolor{dialinecolor}
\pgfpathmoveto{\pgfpoint{26.202373\du}{10.381106\du}}
\pgfpathlineto{\pgfpoint{26.202373\du}{10.381106\du}}
\pgfpathlineto{\pgfpoint{26.186674\du}{10.381106\du}}
\pgfpathlineto{\pgfpoint{26.170610\du}{10.381471\du}}
\pgfpathlineto{\pgfpoint{26.154545\du}{10.382202\du}}
\pgfpathlineto{\pgfpoint{26.139211\du}{10.382932\du}}
\pgfpathlineto{\pgfpoint{26.123147\du}{10.384027\du}}
\pgfpathlineto{\pgfpoint{26.108543\du}{10.385122\du}}
\pgfpathlineto{\pgfpoint{26.092844\du}{10.386218\du}}
\pgfpathlineto{\pgfpoint{26.077145\du}{10.388043\du}}
\pgfpathlineto{\pgfpoint{26.062541\du}{10.389869\du}}
\pgfpathlineto{\pgfpoint{26.047937\du}{10.391694\du}}
\pgfpathlineto{\pgfpoint{26.032968\du}{10.393885\du}}
\pgfpathlineto{\pgfpoint{26.017999\du}{10.396440\du}}
\pgfpathlineto{\pgfpoint{26.003760\du}{10.399361\du}}
\pgfpathlineto{\pgfpoint{25.989887\du}{10.401552\du}}
\pgfpathlineto{\pgfpoint{25.975283\du}{10.404472\du}}
\pgfpathlineto{\pgfpoint{25.960679\du}{10.407393\du}}
\pgfpathlineto{\pgfpoint{25.946805\du}{10.411044\du}}
\pgfpathlineto{\pgfpoint{25.933297\du}{10.414330\du}}
\pgfpathlineto{\pgfpoint{25.920518\du}{10.418346\du}}
\pgfpathlineto{\pgfpoint{25.907010\du}{10.421997\du}}
\pgfpathlineto{\pgfpoint{25.893501\du}{10.426013\du}}
\pgfpathlineto{\pgfpoint{25.880723\du}{10.430394\du}}
\pgfpathlineto{\pgfpoint{25.867579\du}{10.434775\du}}
\pgfpathlineto{\pgfpoint{25.855531\du}{10.439157\du}}
\pgfpathlineto{\pgfpoint{25.843483\du}{10.443538\du}}
\pgfpathlineto{\pgfpoint{25.831070\du}{10.449014\du}}
\pgfpathlineto{\pgfpoint{25.819752\du}{10.453760\du}}
\pgfpathlineto{\pgfpoint{25.807704\du}{10.458872\du}}
\pgfpathlineto{\pgfpoint{25.796386\du}{10.463983\du}}
\pgfpathlineto{\pgfpoint{25.785068\du}{10.469460\du}}
\pgfpathlineto{\pgfpoint{25.773750\du}{10.475301\du}}
\pgfpathlineto{\pgfpoint{25.762797\du}{10.480413\du}}
\pgfpathlineto{\pgfpoint{25.752574\du}{10.486254\du}}
\pgfpathlineto{\pgfpoint{25.742351\du}{10.492826\du}}
\pgfpathlineto{\pgfpoint{25.732494\du}{10.498667\du}}
\pgfpathlineto{\pgfpoint{25.723001\du}{10.505239\du}}
\pgfpathlineto{\pgfpoint{25.713874\du}{10.511446\du}}
\pgfpathlineto{\pgfpoint{25.704016\du}{10.518018\du}}
\pgfpathlineto{\pgfpoint{25.694889\du}{10.524589\du}}
\pgfpathlineto{\pgfpoint{25.686857\du}{10.531526\du}}
\pgfpathlineto{\pgfpoint{25.677729\du}{10.538463\du}}
\pgfpathlineto{\pgfpoint{25.670792\du}{10.545400\du}}
\pgfpathlineto{\pgfpoint{25.662030\du}{10.552337\du}}
\pgfpathlineto{\pgfpoint{25.654728\du}{10.559639\du}}
\pgfpathlineto{\pgfpoint{25.647791\du}{10.567306\du}}
\pgfpathlineto{\pgfpoint{25.641219\du}{10.574608\du}}
\pgfpathlineto{\pgfpoint{25.634283\du}{10.582275\du}}
\pgfpathlineto{\pgfpoint{25.627711\du}{10.589942\du}}
\pgfpathlineto{\pgfpoint{25.622965\du}{10.597974\du}}
\pgfpathlineto{\pgfpoint{25.616758\du}{10.605641\du}}
\pgfpathlineto{\pgfpoint{25.611647\du}{10.614038\du}}
\pgfpathlineto{\pgfpoint{25.606900\du}{10.622070\du}}
\pgfpathlineto{\pgfpoint{25.602519\du}{10.630467\du}}
\pgfpathlineto{\pgfpoint{25.598138\du}{10.638499\du}}
\pgfpathlineto{\pgfpoint{25.594122\du}{10.647262\du}}
\pgfpathlineto{\pgfpoint{25.590836\du}{10.655659\du}}
\pgfpathlineto{\pgfpoint{25.587550\du}{10.664421\du}}
\pgfpathlineto{\pgfpoint{25.585360\du}{10.673184\du}}
\pgfpathlineto{\pgfpoint{25.582804\du}{10.681946\du}}
\pgfpathlineto{\pgfpoint{25.581709\du}{10.690708\du}}
\pgfpathlineto{\pgfpoint{25.579518\du}{10.699471\du}}
\pgfpathlineto{\pgfpoint{25.578423\du}{10.708598\du}}
\pgfpathlineto{\pgfpoint{25.577693\du}{10.718091\du}}
\pgfpathlineto{\pgfpoint{25.577693\du}{10.726853\du}}
\pgfpathlineto{\pgfpoint{25.598138\du}{10.726853\du}}
\pgfpathlineto{\pgfpoint{25.598138\du}{10.718821\du}}
\pgfpathlineto{\pgfpoint{25.598503\du}{10.710424\du}}
\pgfpathlineto{\pgfpoint{25.599963\du}{10.702391\du}}
\pgfpathlineto{\pgfpoint{25.601059\du}{10.694724\du}}
\pgfpathlineto{\pgfpoint{25.602884\du}{10.686327\du}}
\pgfpathlineto{\pgfpoint{25.604710\du}{10.678295\du}}
\pgfpathlineto{\pgfpoint{25.607265\du}{10.670628\du}}
\pgfpathlineto{\pgfpoint{25.610186\du}{10.662961\du}}
\pgfpathlineto{\pgfpoint{25.613472\du}{10.655659\du}}
\pgfpathlineto{\pgfpoint{25.616758\du}{10.647262\du}}
\pgfpathlineto{\pgfpoint{25.620409\du}{10.639595\du}}
\pgfpathlineto{\pgfpoint{25.624790\du}{10.631928\du}}
\pgfpathlineto{\pgfpoint{25.629171\du}{10.624991\du}}
\pgfpathlineto{\pgfpoint{25.633917\du}{10.617324\du}}
\pgfpathlineto{\pgfpoint{25.639029\du}{10.610022\du}}
\pgfpathlineto{\pgfpoint{25.644140\du}{10.602720\du}}
\pgfpathlineto{\pgfpoint{25.650347\du}{10.595053\du}}
\pgfpathlineto{\pgfpoint{25.656553\du}{10.588116\du}}
\pgfpathlineto{\pgfpoint{25.662760\du}{10.581179\du}}
\pgfpathlineto{\pgfpoint{25.669697\du}{10.574242\du}}
\pgfpathlineto{\pgfpoint{25.676269\du}{10.567671\du}}
\pgfpathlineto{\pgfpoint{25.683936\du}{10.560734\du}}
\pgfpathlineto{\pgfpoint{25.691238\du}{10.553797\du}}
\pgfpathlineto{\pgfpoint{25.699270\du}{10.547225\du}}
\pgfpathlineto{\pgfpoint{25.707667\du}{10.540654\du}}
\pgfpathlineto{\pgfpoint{25.716064\du}{10.534447\du}}
\pgfpathlineto{\pgfpoint{25.725192\du}{10.528605\du}}
\pgfpathlineto{\pgfpoint{25.733589\du}{10.522034\du}}
\pgfpathlineto{\pgfpoint{25.743447\du}{10.516192\du}}
\pgfpathlineto{\pgfpoint{25.752939\du}{10.510350\du}}
\pgfpathlineto{\pgfpoint{25.762797\du}{10.504509\du}}
\pgfpathlineto{\pgfpoint{25.773019\du}{10.498667\du}}
\pgfpathlineto{\pgfpoint{25.783242\du}{10.493556\du}}
\pgfpathlineto{\pgfpoint{25.794195\du}{10.487714\du}}
\pgfpathlineto{\pgfpoint{25.804783\du}{10.482968\du}}
\pgfpathlineto{\pgfpoint{25.816101\du}{10.477492\du}}
\pgfpathlineto{\pgfpoint{25.827054\du}{10.472380\du}}
\pgfpathlineto{\pgfpoint{25.838737\du}{10.467634\du}}
\pgfpathlineto{\pgfpoint{25.850055\du}{10.462888\du}}
\pgfpathlineto{\pgfpoint{25.862468\du}{10.458142\du}}
\pgfpathlineto{\pgfpoint{25.874881\du}{10.454126\du}}
\pgfpathlineto{\pgfpoint{25.886930\du}{10.449379\du}}
\pgfpathlineto{\pgfpoint{25.899708\du}{10.445363\du}}
\pgfpathlineto{\pgfpoint{25.912851\du}{10.442077\du}}
\pgfpathlineto{\pgfpoint{25.925630\du}{10.437696\du}}
\pgfpathlineto{\pgfpoint{25.939138\du}{10.434410\du}}
\pgfpathlineto{\pgfpoint{25.952282\du}{10.430759\du}}
\pgfpathlineto{\pgfpoint{25.965790\du}{10.427839\du}}
\pgfpathlineto{\pgfpoint{25.979664\du}{10.424918\du}}
\pgfpathlineto{\pgfpoint{25.993903\du}{10.421997\du}}
\pgfpathlineto{\pgfpoint{26.007777\du}{10.419076\du}}
\pgfpathlineto{\pgfpoint{26.021650\du}{10.416886\du}}
\pgfpathlineto{\pgfpoint{26.035889\du}{10.414330\du}}
\pgfpathlineto{\pgfpoint{26.050493\du}{10.412139\du}}
\pgfpathlineto{\pgfpoint{26.065097\du}{10.410314\du}}
\pgfpathlineto{\pgfpoint{26.080066\du}{10.408488\du}}
\pgfpathlineto{\pgfpoint{26.095035\du}{10.406663\du}}
\pgfpathlineto{\pgfpoint{26.110004\du}{10.405568\du}}
\pgfpathlineto{\pgfpoint{26.124973\du}{10.404472\du}}
\pgfpathlineto{\pgfpoint{26.140672\du}{10.403377\du}}
\pgfpathlineto{\pgfpoint{26.155276\du}{10.402647\du}}
\pgfpathlineto{\pgfpoint{26.171340\du}{10.402282\du}}
\pgfpathlineto{\pgfpoint{26.186674\du}{10.401552\du}}
\pgfpathlineto{\pgfpoint{26.202373\du}{10.401552\du}}
\pgfpathlineto{\pgfpoint{26.202373\du}{10.401552\du}}
\pgfpathlineto{\pgfpoint{26.202373\du}{10.401552\du}}
\pgfpathlineto{\pgfpoint{26.203834\du}{10.401552\du}}
\pgfpathlineto{\pgfpoint{26.204929\du}{10.401552\du}}
\pgfpathlineto{\pgfpoint{26.206024\du}{10.401552\du}}
\pgfpathlineto{\pgfpoint{26.207119\du}{10.400821\du}}
\pgfpathlineto{\pgfpoint{26.208215\du}{10.400456\du}}
\pgfpathlineto{\pgfpoint{26.209310\du}{10.399726\du}}
\pgfpathlineto{\pgfpoint{26.209310\du}{10.399361\du}}
\pgfpathlineto{\pgfpoint{26.210040\du}{10.398631\du}}
\pgfpathlineto{\pgfpoint{26.211135\du}{10.396805\du}}
\pgfpathlineto{\pgfpoint{26.212231\du}{10.394980\du}}
\pgfpathlineto{\pgfpoint{26.212596\du}{10.393520\du}}
\pgfpathlineto{\pgfpoint{26.212596\du}{10.391694\du}}
\pgfpathlineto{\pgfpoint{26.212596\du}{10.389138\du}}
\pgfpathlineto{\pgfpoint{26.212231\du}{10.387678\du}}
\pgfpathlineto{\pgfpoint{26.211135\du}{10.385853\du}}
\pgfpathlineto{\pgfpoint{26.210040\du}{10.384757\du}}
\pgfpathlineto{\pgfpoint{26.209310\du}{10.383297\du}}
\pgfpathlineto{\pgfpoint{26.209310\du}{10.382932\du}}
\pgfpathlineto{\pgfpoint{26.208215\du}{10.382202\du}}
\pgfpathlineto{\pgfpoint{26.207119\du}{10.382202\du}}
\pgfpathlineto{\pgfpoint{26.206024\du}{10.381471\du}}
\pgfpathlineto{\pgfpoint{26.204929\du}{10.381471\du}}
\pgfpathlineto{\pgfpoint{26.203834\du}{10.381106\du}}
\pgfpathlineto{\pgfpoint{26.202373\du}{10.381106\du}}
\pgfusepath{fill}
\pgfsetbuttcap
\pgfsetmiterjoin
\pgfsetdash{}{0pt}
\definecolor{dialinecolor}{rgb}{0.678431, 0.839216, 0.905882}
\pgfsetfillcolor{dialinecolor}
\pgfpathmoveto{\pgfpoint{26.827419\du}{10.726853\du}}
\pgfpathlineto{\pgfpoint{26.827419\du}{10.717360\du}}
\pgfpathlineto{\pgfpoint{26.826689\du}{10.708598\du}}
\pgfpathlineto{\pgfpoint{26.825593\du}{10.699471\du}}
\pgfpathlineto{\pgfpoint{26.824133\du}{10.690708\du}}
\pgfpathlineto{\pgfpoint{26.821942\du}{10.681946\du}}
\pgfpathlineto{\pgfpoint{26.820117\du}{10.673184\du}}
\pgfpathlineto{\pgfpoint{26.817196\du}{10.664421\du}}
\pgfpathlineto{\pgfpoint{26.814640\du}{10.655659\du}}
\pgfpathlineto{\pgfpoint{26.810624\du}{10.647262\du}}
\pgfpathlineto{\pgfpoint{26.807338\du}{10.638499\du}}
\pgfpathlineto{\pgfpoint{26.802957\du}{10.630467\du}}
\pgfpathlineto{\pgfpoint{26.798211\du}{10.622070\du}}
\pgfpathlineto{\pgfpoint{26.793465\du}{10.614038\du}}
\pgfpathlineto{\pgfpoint{26.788719\du}{10.605641\du}}
\pgfpathlineto{\pgfpoint{26.782877\du}{10.597974\du}}
\pgfpathlineto{\pgfpoint{26.776670\du}{10.589942\du}}
\pgfpathlineto{\pgfpoint{26.770464\du}{10.582275\du}}
\pgfpathlineto{\pgfpoint{26.763892\du}{10.574608\du}}
\pgfpathlineto{\pgfpoint{26.757320\du}{10.567306\du}}
\pgfpathlineto{\pgfpoint{26.749653\du}{10.559639\du}}
\pgfpathlineto{\pgfpoint{26.742716\du}{10.552337\du}}
\pgfpathlineto{\pgfpoint{26.734684\du}{10.545400\du}}
\pgfpathlineto{\pgfpoint{26.726652\du}{10.538463\du}}
\pgfpathlineto{\pgfpoint{26.718255\du}{10.531526\du}}
\pgfpathlineto{\pgfpoint{26.709858\du}{10.524589\du}}
\pgfpathlineto{\pgfpoint{26.701095\du}{10.518018\du}}
\pgfpathlineto{\pgfpoint{26.691968\du}{10.511446\du}}
\pgfpathlineto{\pgfpoint{26.682475\du}{10.505239\du}}
\pgfpathlineto{\pgfpoint{26.672253\du}{10.498667\du}}
\pgfpathlineto{\pgfpoint{26.662395\du}{10.492826\du}}
\pgfpathlineto{\pgfpoint{26.652172\du}{10.486254\du}}
\pgfpathlineto{\pgfpoint{26.642315\du}{10.480413\du}}
\pgfpathlineto{\pgfpoint{26.630997\du}{10.475301\du}}
\pgfpathlineto{\pgfpoint{26.620044\du}{10.469460\du}}
\pgfpathlineto{\pgfpoint{26.608726\du}{10.463983\du}}
\pgfpathlineto{\pgfpoint{26.597408\du}{10.458872\du}}
\pgfpathlineto{\pgfpoint{26.585725\du}{10.453760\du}}
\pgfpathlineto{\pgfpoint{26.574407\du}{10.449014\du}}
\pgfpathlineto{\pgfpoint{26.561628\du}{10.443538\du}}
\pgfpathlineto{\pgfpoint{26.549945\du}{10.439157\du}}
\pgfpathlineto{\pgfpoint{26.537167\du}{10.434775\du}}
\pgfpathlineto{\pgfpoint{26.524388\du}{10.430394\du}}
\pgfpathlineto{\pgfpoint{26.511245\du}{10.426013\du}}
\pgfpathlineto{\pgfpoint{26.498101\du}{10.421997\du}}
\pgfpathlineto{\pgfpoint{26.484958\du}{10.418346\du}}
\pgfpathlineto{\pgfpoint{26.471449\du}{10.414330\du}}
\pgfpathlineto{\pgfpoint{26.457576\du}{10.411044\du}}
\pgfpathlineto{\pgfpoint{26.444067\du}{10.407393\du}}
\pgfpathlineto{\pgfpoint{26.430194\du}{10.404472\du}}
\pgfpathlineto{\pgfpoint{26.415590\du}{10.401552\du}}
\pgfpathlineto{\pgfpoint{26.401716\du}{10.399361\du}}
\pgfpathlineto{\pgfpoint{26.387477\du}{10.396440\du}}
\pgfpathlineto{\pgfpoint{26.372143\du}{10.393885\du}}
\pgfpathlineto{\pgfpoint{26.357174\du}{10.391694\du}}
\pgfpathlineto{\pgfpoint{26.342205\du}{10.389869\du}}
\pgfpathlineto{\pgfpoint{26.327966\du}{10.388043\du}}
\pgfpathlineto{\pgfpoint{26.311902\du}{10.386218\du}}
\pgfpathlineto{\pgfpoint{26.296933\du}{10.385122\du}}
\pgfpathlineto{\pgfpoint{26.281599\du}{10.384027\du}}
\pgfpathlineto{\pgfpoint{26.265900\du}{10.382932\du}}
\pgfpathlineto{\pgfpoint{26.250566\du}{10.382202\du}}
\pgfpathlineto{\pgfpoint{26.234502\du}{10.381471\du}}
\pgfpathlineto{\pgfpoint{26.218437\du}{10.381106\du}}
\pgfpathlineto{\pgfpoint{26.202373\du}{10.381106\du}}
\pgfpathlineto{\pgfpoint{26.202373\du}{10.401552\du}}
\pgfpathlineto{\pgfpoint{26.218437\du}{10.401552\du}}
\pgfpathlineto{\pgfpoint{26.233771\du}{10.402282\du}}
\pgfpathlineto{\pgfpoint{26.249471\du}{10.402647\du}}
\pgfpathlineto{\pgfpoint{26.264805\du}{10.403377\du}}
\pgfpathlineto{\pgfpoint{26.279774\du}{10.404472\du}}
\pgfpathlineto{\pgfpoint{26.295473\du}{10.405568\du}}
\pgfpathlineto{\pgfpoint{26.310077\du}{10.406663\du}}
\pgfpathlineto{\pgfpoint{26.325411\du}{10.408488\du}}
\pgfpathlineto{\pgfpoint{26.340380\du}{10.410314\du}}
\pgfpathlineto{\pgfpoint{26.354984\du}{10.412139\du}}
\pgfpathlineto{\pgfpoint{26.369222\du}{10.414330\du}}
\pgfpathlineto{\pgfpoint{26.383461\du}{10.416886\du}}
\pgfpathlineto{\pgfpoint{26.397700\du}{10.419076\du}}
\pgfpathlineto{\pgfpoint{26.411574\du}{10.421997\du}}
\pgfpathlineto{\pgfpoint{26.425447\du}{10.424918\du}}
\pgfpathlineto{\pgfpoint{26.439321\du}{10.427839\du}}
\pgfpathlineto{\pgfpoint{26.452829\du}{10.430759\du}}
\pgfpathlineto{\pgfpoint{26.466338\du}{10.434410\du}}
\pgfpathlineto{\pgfpoint{26.479482\du}{10.437696\du}}
\pgfpathlineto{\pgfpoint{26.492260\du}{10.442077\du}}
\pgfpathlineto{\pgfpoint{26.505769\du}{10.445363\du}}
\pgfpathlineto{\pgfpoint{26.518547\du}{10.449379\du}}
\pgfpathlineto{\pgfpoint{26.530230\du}{10.454126\du}}
\pgfpathlineto{\pgfpoint{26.542643\du}{10.458142\du}}
\pgfpathlineto{\pgfpoint{26.554691\du}{10.462888\du}}
\pgfpathlineto{\pgfpoint{26.566375\du}{10.467634\du}}
\pgfpathlineto{\pgfpoint{26.577693\du}{10.472380\du}}
\pgfpathlineto{\pgfpoint{26.589011\du}{10.477492\du}}
\pgfpathlineto{\pgfpoint{26.600694\du}{10.482968\du}}
\pgfpathlineto{\pgfpoint{26.611281\du}{10.487714\du}}
\pgfpathlineto{\pgfpoint{26.621869\du}{10.493556\du}}
\pgfpathlineto{\pgfpoint{26.632457\du}{10.498667\du}}
\pgfpathlineto{\pgfpoint{26.642315\du}{10.504509\du}}
\pgfpathlineto{\pgfpoint{26.652172\du}{10.510350\du}}
\pgfpathlineto{\pgfpoint{26.661300\du}{10.516192\du}}
\pgfpathlineto{\pgfpoint{26.670427\du}{10.522034\du}}
\pgfpathlineto{\pgfpoint{26.679920\du}{10.528605\du}}
\pgfpathlineto{\pgfpoint{26.688682\du}{10.534447\du}}
\pgfpathlineto{\pgfpoint{26.697809\du}{10.540654\du}}
\pgfpathlineto{\pgfpoint{26.705842\du}{10.547225\du}}
\pgfpathlineto{\pgfpoint{26.714239\du}{10.553797\du}}
\pgfpathlineto{\pgfpoint{26.721176\du}{10.560734\du}}
\pgfpathlineto{\pgfpoint{26.728478\du}{10.567671\du}}
\pgfpathlineto{\pgfpoint{26.735414\du}{10.574242\du}}
\pgfpathlineto{\pgfpoint{26.742716\du}{10.581179\du}}
\pgfpathlineto{\pgfpoint{26.748558\du}{10.588116\du}}
\pgfpathlineto{\pgfpoint{26.755495\du}{10.595053\du}}
\pgfpathlineto{\pgfpoint{26.760606\du}{10.602720\du}}
\pgfpathlineto{\pgfpoint{26.766083\du}{10.610022\du}}
\pgfpathlineto{\pgfpoint{26.770829\du}{10.617324\du}}
\pgfpathlineto{\pgfpoint{26.775940\du}{10.624991\du}}
\pgfpathlineto{\pgfpoint{26.779956\du}{10.631928\du}}
\pgfpathlineto{\pgfpoint{26.784702\du}{10.639595\du}}
\pgfpathlineto{\pgfpoint{26.788719\du}{10.647262\du}}
\pgfpathlineto{\pgfpoint{26.792004\du}{10.655659\du}}
\pgfpathlineto{\pgfpoint{26.794560\du}{10.662961\du}}
\pgfpathlineto{\pgfpoint{26.797846\du}{10.670628\du}}
\pgfpathlineto{\pgfpoint{26.800037\du}{10.678295\du}}
\pgfpathlineto{\pgfpoint{26.802592\du}{10.686327\du}}
\pgfpathlineto{\pgfpoint{26.804053\du}{10.694724\du}}
\pgfpathlineto{\pgfpoint{26.805513\du}{10.702391\du}}
\pgfpathlineto{\pgfpoint{26.806243\du}{10.710424\du}}
\pgfpathlineto{\pgfpoint{26.806608\du}{10.718821\du}}
\pgfpathlineto{\pgfpoint{26.807338\du}{10.726853\du}}
\pgfpathlineto{\pgfpoint{26.827419\du}{10.726853\du}}
\pgfusepath{fill}
\pgfsetbuttcap
\pgfsetmiterjoin
\pgfsetdash{}{0pt}
\definecolor{dialinecolor}{rgb}{0.074510, 0.082353, 0.086275}
\pgfsetfillcolor{dialinecolor}
\pgfpathmoveto{\pgfpoint{25.881453\du}{10.821048\du}}
\pgfpathlineto{\pgfpoint{26.108908\du}{10.592862\du}}
\pgfpathlineto{\pgfpoint{26.049032\du}{10.531891\du}}
\pgfpathlineto{\pgfpoint{26.229390\du}{10.531891\du}}
\pgfpathlineto{\pgfpoint{26.229390\du}{10.720281\du}}
\pgfpathlineto{\pgfpoint{26.169149\du}{10.660040\du}}
\pgfpathlineto{\pgfpoint{25.948996\du}{10.881289\du}}
\pgfpathlineto{\pgfpoint{25.881453\du}{10.821048\du}}
\pgfusepath{fill}
\pgfsetbuttcap
\pgfsetmiterjoin
\pgfsetdash{}{0pt}
\definecolor{dialinecolor}{rgb}{0.074510, 0.082353, 0.086275}
\pgfsetfillcolor{dialinecolor}
\pgfpathmoveto{\pgfpoint{26.149434\du}{10.941530\du}}
\pgfpathlineto{\pgfpoint{26.376524\du}{10.713344\du}}
\pgfpathlineto{\pgfpoint{26.315918\du}{10.653103\du}}
\pgfpathlineto{\pgfpoint{26.497006\du}{10.653103\du}}
\pgfpathlineto{\pgfpoint{26.497006\du}{10.841128\du}}
\pgfpathlineto{\pgfpoint{26.436400\du}{10.780887\du}}
\pgfpathlineto{\pgfpoint{26.215882\du}{11.001771\du}}
\pgfpathlineto{\pgfpoint{26.149434\du}{10.941530\du}}
\pgfusepath{fill}
\pgfsetbuttcap
\pgfsetmiterjoin
\pgfsetdash{}{0pt}
\definecolor{dialinecolor}{rgb}{1.000000, 1.000000, 1.000000}
\pgfsetfillcolor{dialinecolor}
\pgfpathmoveto{\pgfpoint{25.868310\du}{10.807539\du}}
\pgfpathlineto{\pgfpoint{26.095400\du}{10.579354\du}}
\pgfpathlineto{\pgfpoint{26.035889\du}{10.519113\du}}
\pgfpathlineto{\pgfpoint{26.215882\du}{10.519113\du}}
\pgfpathlineto{\pgfpoint{26.215882\du}{10.707138\du}}
\pgfpathlineto{\pgfpoint{26.156006\du}{10.646166\du}}
\pgfpathlineto{\pgfpoint{25.935487\du}{10.867780\du}}
\pgfpathlineto{\pgfpoint{25.868310\du}{10.807539\du}}
\pgfusepath{fill}
\pgfsetbuttcap
\pgfsetmiterjoin
\pgfsetdash{}{0pt}
\definecolor{dialinecolor}{rgb}{1.000000, 1.000000, 1.000000}
\pgfsetfillcolor{dialinecolor}
\pgfpathmoveto{\pgfpoint{26.135926\du}{10.928021\du}}
\pgfpathlineto{\pgfpoint{26.363016\du}{10.699836\du}}
\pgfpathlineto{\pgfpoint{26.302775\du}{10.639595\du}}
\pgfpathlineto{\pgfpoint{26.483133\du}{10.639595\du}}
\pgfpathlineto{\pgfpoint{26.483133\du}{10.827620\du}}
\pgfpathlineto{\pgfpoint{26.423622\du}{10.767379\du}}
\pgfpathlineto{\pgfpoint{26.202373\du}{10.988262\du}}
\pgfpathlineto{\pgfpoint{26.135926\du}{10.928021\du}}
\pgfusepath{fill}
\pgfsetlinewidth{0.000000\du}
\pgfsetdash{}{0pt}
\pgfsetdash{}{0pt}
\pgfsetbuttcap
\pgfsetmiterjoin
\pgfsetlinewidth{0.000000\du}
\pgfsetbuttcap
\pgfsetmiterjoin
\pgfsetdash{}{0pt}
\definecolor{dialinecolor}{rgb}{0.717647, 0.717647, 0.615686}
\pgfsetfillcolor{dialinecolor}
\pgfpathmoveto{\pgfpoint{34.599416\du}{7.674024\du}}
\pgfpathlineto{\pgfpoint{34.599416\du}{11.050000\du}}
\pgfpathlineto{\pgfpoint{36.410357\du}{11.050000\du}}
\pgfpathlineto{\pgfpoint{36.410357\du}{7.674024\du}}
\pgfpathlineto{\pgfpoint{34.599416\du}{7.674024\du}}
\pgfusepath{fill}
\pgfsetbuttcap
\pgfsetmiterjoin
\pgfsetdash{}{0pt}
\definecolor{dialinecolor}{rgb}{0.286275, 0.286275, 0.211765}
\pgfsetstrokecolor{dialinecolor}
\pgfpathmoveto{\pgfpoint{34.599416\du}{7.674024\du}}
\pgfpathlineto{\pgfpoint{34.599416\du}{11.050000\du}}
\pgfpathlineto{\pgfpoint{36.410357\du}{11.050000\du}}
\pgfpathlineto{\pgfpoint{36.410357\du}{7.674024\du}}
\pgfpathlineto{\pgfpoint{34.599416\du}{7.674024\du}}
\pgfusepath{stroke}
\pgfsetbuttcap
\pgfsetmiterjoin
\pgfsetdash{}{0pt}
\definecolor{dialinecolor}{rgb}{0.000000, 0.000000, 0.000000}
\pgfsetfillcolor{dialinecolor}
\pgfpathmoveto{\pgfpoint{35.180823\du}{8.054322\du}}
\pgfpathlineto{\pgfpoint{35.180823\du}{8.538987\du}}
\pgfpathlineto{\pgfpoint{35.888997\du}{8.538987\du}}
\pgfpathlineto{\pgfpoint{35.888997\du}{8.054322\du}}
\pgfpathlineto{\pgfpoint{35.180823\du}{8.054322\du}}
\pgfusepath{fill}
\pgfsetbuttcap
\pgfsetmiterjoin
\pgfsetdash{}{0pt}
\definecolor{dialinecolor}{rgb}{0.000000, 0.000000, 0.000000}
\pgfsetfillcolor{dialinecolor}
\pgfpathmoveto{\pgfpoint{35.097425\du}{8.886878\du}}
\pgfpathlineto{\pgfpoint{35.097425\du}{8.880206\du}}
\pgfpathlineto{\pgfpoint{35.096948\du}{8.874964\du}}
\pgfpathlineto{\pgfpoint{35.096471\du}{8.869245\du}}
\pgfpathlineto{\pgfpoint{35.095042\du}{8.863527\du}}
\pgfpathlineto{\pgfpoint{35.094089\du}{8.858284\du}}
\pgfpathlineto{\pgfpoint{35.092182\du}{8.852089\du}}
\pgfpathlineto{\pgfpoint{35.090753\du}{8.846847\du}}
\pgfpathlineto{\pgfpoint{35.087893\du}{8.841128\du}}
\pgfpathlineto{\pgfpoint{35.085511\du}{8.835409\du}}
\pgfpathlineto{\pgfpoint{35.083128\du}{8.830644\du}}
\pgfpathlineto{\pgfpoint{35.080268\du}{8.825878\du}}
\pgfpathlineto{\pgfpoint{35.077409\du}{8.820636\du}}
\pgfpathlineto{\pgfpoint{35.073596\du}{8.816347\du}}
\pgfpathlineto{\pgfpoint{35.070260\du}{8.811581\du}}
\pgfpathlineto{\pgfpoint{35.065971\du}{8.807292\du}}
\pgfpathlineto{\pgfpoint{35.061682\du}{8.803003\du}}
\pgfpathlineto{\pgfpoint{35.057870\du}{8.799667\du}}
\pgfpathlineto{\pgfpoint{35.053581\du}{8.795378\du}}
\pgfpathlineto{\pgfpoint{35.048815\du}{8.792042\du}}
\pgfpathlineto{\pgfpoint{35.044049\du}{8.788706\du}}
\pgfpathlineto{\pgfpoint{35.038331\du}{8.785847\du}}
\pgfpathlineto{\pgfpoint{35.034042\du}{8.783464\du}}
\pgfpathlineto{\pgfpoint{35.028323\du}{8.781081\du}}
\pgfpathlineto{\pgfpoint{35.023081\du}{8.778698\du}}
\pgfpathlineto{\pgfpoint{35.016885\du}{8.776792\du}}
\pgfpathlineto{\pgfpoint{35.011643\du}{8.774886\du}}
\pgfpathlineto{\pgfpoint{35.006401\du}{8.773933\du}}
\pgfpathlineto{\pgfpoint{35.000206\du}{8.772503\du}}
\pgfpathlineto{\pgfpoint{34.994487\du}{8.772026\du}}
\pgfpathlineto{\pgfpoint{34.989245\du}{8.771550\du}}
\pgfpathlineto{\pgfpoint{34.983526\du}{8.771550\du}}
\pgfpathlineto{\pgfpoint{34.983526\du}{8.771550\du}}
\pgfpathlineto{\pgfpoint{34.977331\du}{8.771550\du}}
\pgfpathlineto{\pgfpoint{34.971612\du}{8.772026\du}}
\pgfpathlineto{\pgfpoint{34.965893\du}{8.772503\du}}
\pgfpathlineto{\pgfpoint{34.959698\du}{8.773933\du}}
\pgfpathlineto{\pgfpoint{34.954456\du}{8.774886\du}}
\pgfpathlineto{\pgfpoint{34.949213\du}{8.776792\du}}
\pgfpathlineto{\pgfpoint{34.943018\du}{8.778698\du}}
\pgfpathlineto{\pgfpoint{34.937776\du}{8.781081\du}}
\pgfpathlineto{\pgfpoint{34.933010\du}{8.783464\du}}
\pgfpathlineto{\pgfpoint{34.927768\du}{8.785847\du}}
\pgfpathlineto{\pgfpoint{34.922049\du}{8.788706\du}}
\pgfpathlineto{\pgfpoint{34.917760\du}{8.792042\du}}
\pgfpathlineto{\pgfpoint{34.912518\du}{8.795378\du}}
\pgfpathlineto{\pgfpoint{34.909182\du}{8.799667\du}}
\pgfpathlineto{\pgfpoint{34.904416\du}{8.803003\du}}
\pgfpathlineto{\pgfpoint{34.900127\du}{8.807292\du}}
\pgfpathlineto{\pgfpoint{34.895838\du}{8.811581\du}}
\pgfpathlineto{\pgfpoint{34.892502\du}{8.816347\du}}
\pgfpathlineto{\pgfpoint{34.888690\du}{8.820636\du}}
\pgfpathlineto{\pgfpoint{34.885830\du}{8.825878\du}}
\pgfpathlineto{\pgfpoint{34.882971\du}{8.830644\du}}
\pgfpathlineto{\pgfpoint{34.880588\du}{8.835409\du}}
\pgfpathlineto{\pgfpoint{34.878205\du}{8.841128\du}}
\pgfpathlineto{\pgfpoint{34.876299\du}{8.846847\du}}
\pgfpathlineto{\pgfpoint{34.873916\du}{8.852089\du}}
\pgfpathlineto{\pgfpoint{34.872010\du}{8.858284\du}}
\pgfpathlineto{\pgfpoint{34.871057\du}{8.863527\du}}
\pgfpathlineto{\pgfpoint{34.870104\du}{8.869245\du}}
\pgfpathlineto{\pgfpoint{34.869151\du}{8.874964\du}}
\pgfpathlineto{\pgfpoint{34.868674\du}{8.880206\du}}
\pgfpathlineto{\pgfpoint{34.868674\du}{8.886878\du}}
\pgfpathlineto{\pgfpoint{34.868674\du}{8.886878\du}}
\pgfpathlineto{\pgfpoint{34.868674\du}{8.892120\du}}
\pgfpathlineto{\pgfpoint{34.869151\du}{8.898792\du}}
\pgfpathlineto{\pgfpoint{34.870104\du}{8.904034\du}}
\pgfpathlineto{\pgfpoint{34.871057\du}{8.910230\du}}
\pgfpathlineto{\pgfpoint{34.872010\du}{8.915472\du}}
\pgfpathlineto{\pgfpoint{34.873916\du}{8.921667\du}}
\pgfpathlineto{\pgfpoint{34.876299\du}{8.926910\du}}
\pgfpathlineto{\pgfpoint{34.878205\du}{8.932152\du}}
\pgfpathlineto{\pgfpoint{34.880588\du}{8.937870\du}}
\pgfpathlineto{\pgfpoint{34.882971\du}{8.942636\du}}
\pgfpathlineto{\pgfpoint{34.885830\du}{8.947878\du}}
\pgfpathlineto{\pgfpoint{34.888690\du}{8.952644\du}}
\pgfpathlineto{\pgfpoint{34.892502\du}{8.957410\du}}
\pgfpathlineto{\pgfpoint{34.895838\du}{8.962175\du}}
\pgfpathlineto{\pgfpoint{34.900127\du}{8.966464\du}}
\pgfpathlineto{\pgfpoint{34.904416\du}{8.970753\du}}
\pgfpathlineto{\pgfpoint{34.909182\du}{8.974089\du}}
\pgfpathlineto{\pgfpoint{34.912518\du}{8.978378\du}}
\pgfpathlineto{\pgfpoint{34.917760\du}{8.981238\du}}
\pgfpathlineto{\pgfpoint{34.922049\du}{8.984574\du}}
\pgfpathlineto{\pgfpoint{34.927768\du}{8.987433\du}}
\pgfpathlineto{\pgfpoint{34.933010\du}{8.990292\du}}
\pgfpathlineto{\pgfpoint{34.937776\du}{8.992675\du}}
\pgfpathlineto{\pgfpoint{34.943018\du}{8.995058\du}}
\pgfpathlineto{\pgfpoint{34.949213\du}{8.996488\du}}
\pgfpathlineto{\pgfpoint{34.954456\du}{8.998871\du}}
\pgfpathlineto{\pgfpoint{34.959698\du}{8.999347\du}}
\pgfpathlineto{\pgfpoint{34.965893\du}{9.001253\du}}
\pgfpathlineto{\pgfpoint{34.971612\du}{9.001730\du}}
\pgfpathlineto{\pgfpoint{34.977331\du}{9.002207\du}}
\pgfpathlineto{\pgfpoint{34.983526\du}{9.002207\du}}
\pgfpathlineto{\pgfpoint{34.983526\du}{9.002207\du}}
\pgfpathlineto{\pgfpoint{34.989245\du}{9.002207\du}}
\pgfpathlineto{\pgfpoint{34.994487\du}{9.001730\du}}
\pgfpathlineto{\pgfpoint{35.000206\du}{9.001253\du}}
\pgfpathlineto{\pgfpoint{35.006401\du}{8.999347\du}}
\pgfpathlineto{\pgfpoint{35.011643\du}{8.998871\du}}
\pgfpathlineto{\pgfpoint{35.016885\du}{8.996488\du}}
\pgfpathlineto{\pgfpoint{35.023081\du}{8.995058\du}}
\pgfpathlineto{\pgfpoint{35.028323\du}{8.992675\du}}
\pgfpathlineto{\pgfpoint{35.034042\du}{8.990292\du}}
\pgfpathlineto{\pgfpoint{35.038331\du}{8.987433\du}}
\pgfpathlineto{\pgfpoint{35.044049\du}{8.984574\du}}
\pgfpathlineto{\pgfpoint{35.048815\du}{8.981238\du}}
\pgfpathlineto{\pgfpoint{35.053581\du}{8.978378\du}}
\pgfpathlineto{\pgfpoint{35.057870\du}{8.974089\du}}
\pgfpathlineto{\pgfpoint{35.061682\du}{8.970753\du}}
\pgfpathlineto{\pgfpoint{35.065971\du}{8.966464\du}}
\pgfpathlineto{\pgfpoint{35.070260\du}{8.962175\du}}
\pgfpathlineto{\pgfpoint{35.073596\du}{8.957410\du}}
\pgfpathlineto{\pgfpoint{35.077409\du}{8.952644\du}}
\pgfpathlineto{\pgfpoint{35.080268\du}{8.947878\du}}
\pgfpathlineto{\pgfpoint{35.083128\du}{8.942636\du}}
\pgfpathlineto{\pgfpoint{35.085511\du}{8.937870\du}}
\pgfpathlineto{\pgfpoint{35.087893\du}{8.932152\du}}
\pgfpathlineto{\pgfpoint{35.090753\du}{8.926910\du}}
\pgfpathlineto{\pgfpoint{35.092182\du}{8.921667\du}}
\pgfpathlineto{\pgfpoint{35.094089\du}{8.915472\du}}
\pgfpathlineto{\pgfpoint{35.095042\du}{8.910230\du}}
\pgfpathlineto{\pgfpoint{35.096471\du}{8.904034\du}}
\pgfpathlineto{\pgfpoint{35.096948\du}{8.898792\du}}
\pgfpathlineto{\pgfpoint{35.097425\du}{8.892120\du}}
\pgfpathlineto{\pgfpoint{35.097425\du}{8.886878\du}}
\pgfusepath{fill}
\pgfsetbuttcap
\pgfsetmiterjoin
\pgfsetdash{}{0pt}
\definecolor{dialinecolor}{rgb}{0.000000, 0.000000, 0.000000}
\pgfsetfillcolor{dialinecolor}
\pgfpathmoveto{\pgfpoint{36.138239\du}{8.886878\du}}
\pgfpathlineto{\pgfpoint{36.138239\du}{8.880206\du}}
\pgfpathlineto{\pgfpoint{36.137763\du}{8.874964\du}}
\pgfpathlineto{\pgfpoint{36.136810\du}{8.869245\du}}
\pgfpathlineto{\pgfpoint{36.136333\du}{8.863527\du}}
\pgfpathlineto{\pgfpoint{36.134427\du}{8.858284\du}}
\pgfpathlineto{\pgfpoint{36.133474\du}{8.852089\du}}
\pgfpathlineto{\pgfpoint{36.131091\du}{8.846847\du}}
\pgfpathlineto{\pgfpoint{36.129185\du}{8.841128\du}}
\pgfpathlineto{\pgfpoint{36.126325\du}{8.835409\du}}
\pgfpathlineto{\pgfpoint{36.124419\du}{8.830644\du}}
\pgfpathlineto{\pgfpoint{36.121560\du}{8.825878\du}}
\pgfpathlineto{\pgfpoint{36.118224\du}{8.820636\du}}
\pgfpathlineto{\pgfpoint{36.114411\du}{8.816347\du}}
\pgfpathlineto{\pgfpoint{36.110599\du}{8.811581\du}}
\pgfpathlineto{\pgfpoint{36.106786\du}{8.807292\du}}
\pgfpathlineto{\pgfpoint{36.103450\du}{8.803003\du}}
\pgfpathlineto{\pgfpoint{36.098208\du}{8.799667\du}}
\pgfpathlineto{\pgfpoint{36.094395\du}{8.795378\du}}
\pgfpathlineto{\pgfpoint{36.089153\du}{8.792042\du}}
\pgfpathlineto{\pgfpoint{36.084388\du}{8.788706\du}}
\pgfpathlineto{\pgfpoint{36.079622\du}{8.785847\du}}
\pgfpathlineto{\pgfpoint{36.074380\du}{8.783464\du}}
\pgfpathlineto{\pgfpoint{36.069138\du}{8.781081\du}}
\pgfpathlineto{\pgfpoint{36.063419\du}{8.778698\du}}
\pgfpathlineto{\pgfpoint{36.058653\du}{8.776792\du}}
\pgfpathlineto{\pgfpoint{36.052934\du}{8.774886\du}}
\pgfpathlineto{\pgfpoint{36.046739\du}{8.773933\du}}
\pgfpathlineto{\pgfpoint{36.041497\du}{8.772503\du}}
\pgfpathlineto{\pgfpoint{36.035778\du}{8.772026\du}}
\pgfpathlineto{\pgfpoint{36.030059\du}{8.771550\du}}
\pgfpathlineto{\pgfpoint{36.024341\du}{8.771550\du}}
\pgfpathlineto{\pgfpoint{36.024341\du}{8.771550\du}}
\pgfpathlineto{\pgfpoint{36.018145\du}{8.771550\du}}
\pgfpathlineto{\pgfpoint{36.012903\du}{8.772026\du}}
\pgfpathlineto{\pgfpoint{36.007184\du}{8.772503\du}}
\pgfpathlineto{\pgfpoint{36.001942\du}{8.773933\du}}
\pgfpathlineto{\pgfpoint{35.996223\du}{8.774886\du}}
\pgfpathlineto{\pgfpoint{35.990505\du}{8.776792\du}}
\pgfpathlineto{\pgfpoint{35.984786\du}{8.778698\du}}
\pgfpathlineto{\pgfpoint{35.979544\du}{8.781081\du}}
\pgfpathlineto{\pgfpoint{35.974301\du}{8.783464\du}}
\pgfpathlineto{\pgfpoint{35.969536\du}{8.785847\du}}
\pgfpathlineto{\pgfpoint{35.963817\du}{8.788706\du}}
\pgfpathlineto{\pgfpoint{35.959051\du}{8.792042\du}}
\pgfpathlineto{\pgfpoint{35.954762\du}{8.795378\du}}
\pgfpathlineto{\pgfpoint{35.950473\du}{8.799667\du}}
\pgfpathlineto{\pgfpoint{35.945708\du}{8.803003\du}}
\pgfpathlineto{\pgfpoint{35.942372\du}{8.807292\du}}
\pgfpathlineto{\pgfpoint{35.937606\du}{8.811581\du}}
\pgfpathlineto{\pgfpoint{35.934270\du}{8.816347\du}}
\pgfpathlineto{\pgfpoint{35.930934\du}{8.820636\du}}
\pgfpathlineto{\pgfpoint{35.927598\du}{8.825878\du}}
\pgfpathlineto{\pgfpoint{35.924739\du}{8.830644\du}}
\pgfpathlineto{\pgfpoint{35.922356\du}{8.835409\du}}
\pgfpathlineto{\pgfpoint{35.919497\du}{8.841128\du}}
\pgfpathlineto{\pgfpoint{35.917590\du}{8.846847\du}}
\pgfpathlineto{\pgfpoint{35.915684\du}{8.852089\du}}
\pgfpathlineto{\pgfpoint{35.913778\du}{8.858284\du}}
\pgfpathlineto{\pgfpoint{35.912825\du}{8.863527\du}}
\pgfpathlineto{\pgfpoint{35.911395\du}{8.869245\du}}
\pgfpathlineto{\pgfpoint{35.910919\du}{8.874964\du}}
\pgfpathlineto{\pgfpoint{35.910442\du}{8.880206\du}}
\pgfpathlineto{\pgfpoint{35.910442\du}{8.886878\du}}
\pgfpathlineto{\pgfpoint{35.910442\du}{8.886878\du}}
\pgfpathlineto{\pgfpoint{35.910442\du}{8.892120\du}}
\pgfpathlineto{\pgfpoint{35.910919\du}{8.898792\du}}
\pgfpathlineto{\pgfpoint{35.911395\du}{8.904034\du}}
\pgfpathlineto{\pgfpoint{35.912825\du}{8.910230\du}}
\pgfpathlineto{\pgfpoint{35.913778\du}{8.915472\du}}
\pgfpathlineto{\pgfpoint{35.915684\du}{8.921667\du}}
\pgfpathlineto{\pgfpoint{35.917590\du}{8.926910\du}}
\pgfpathlineto{\pgfpoint{35.919497\du}{8.932152\du}}
\pgfpathlineto{\pgfpoint{35.922356\du}{8.937870\du}}
\pgfpathlineto{\pgfpoint{35.924739\du}{8.942636\du}}
\pgfpathlineto{\pgfpoint{35.927598\du}{8.947878\du}}
\pgfpathlineto{\pgfpoint{35.930934\du}{8.952644\du}}
\pgfpathlineto{\pgfpoint{35.934270\du}{8.957410\du}}
\pgfpathlineto{\pgfpoint{35.937606\du}{8.962175\du}}
\pgfpathlineto{\pgfpoint{35.942372\du}{8.966464\du}}
\pgfpathlineto{\pgfpoint{35.945708\du}{8.970753\du}}
\pgfpathlineto{\pgfpoint{35.950473\du}{8.974089\du}}
\pgfpathlineto{\pgfpoint{35.954762\du}{8.978378\du}}
\pgfpathlineto{\pgfpoint{35.959051\du}{8.981238\du}}
\pgfpathlineto{\pgfpoint{35.963817\du}{8.984574\du}}
\pgfpathlineto{\pgfpoint{35.969536\du}{8.987433\du}}
\pgfpathlineto{\pgfpoint{35.974301\du}{8.990292\du}}
\pgfpathlineto{\pgfpoint{35.979544\du}{8.992675\du}}
\pgfpathlineto{\pgfpoint{35.984786\du}{8.995058\du}}
\pgfpathlineto{\pgfpoint{35.990505\du}{8.996488\du}}
\pgfpathlineto{\pgfpoint{35.996223\du}{8.998871\du}}
\pgfpathlineto{\pgfpoint{36.001942\du}{8.999347\du}}
\pgfpathlineto{\pgfpoint{36.007184\du}{9.001253\du}}
\pgfpathlineto{\pgfpoint{36.012903\du}{9.001730\du}}
\pgfpathlineto{\pgfpoint{36.018145\du}{9.002207\du}}
\pgfpathlineto{\pgfpoint{36.024341\du}{9.002207\du}}
\pgfpathlineto{\pgfpoint{36.024341\du}{9.002207\du}}
\pgfpathlineto{\pgfpoint{36.030059\du}{9.002207\du}}
\pgfpathlineto{\pgfpoint{36.035778\du}{9.001730\du}}
\pgfpathlineto{\pgfpoint{36.041497\du}{9.001253\du}}
\pgfpathlineto{\pgfpoint{36.046739\du}{8.999347\du}}
\pgfpathlineto{\pgfpoint{36.052934\du}{8.998871\du}}
\pgfpathlineto{\pgfpoint{36.058653\du}{8.996488\du}}
\pgfpathlineto{\pgfpoint{36.063419\du}{8.995058\du}}
\pgfpathlineto{\pgfpoint{36.069138\du}{8.992675\du}}
\pgfpathlineto{\pgfpoint{36.074380\du}{8.990292\du}}
\pgfpathlineto{\pgfpoint{36.079622\du}{8.987433\du}}
\pgfpathlineto{\pgfpoint{36.084388\du}{8.984574\du}}
\pgfpathlineto{\pgfpoint{36.089153\du}{8.981238\du}}
\pgfpathlineto{\pgfpoint{36.094395\du}{8.978378\du}}
\pgfpathlineto{\pgfpoint{36.098208\du}{8.974089\du}}
\pgfpathlineto{\pgfpoint{36.103450\du}{8.970753\du}}
\pgfpathlineto{\pgfpoint{36.106786\du}{8.966464\du}}
\pgfpathlineto{\pgfpoint{36.110599\du}{8.962175\du}}
\pgfpathlineto{\pgfpoint{36.114411\du}{8.957410\du}}
\pgfpathlineto{\pgfpoint{36.118224\du}{8.952644\du}}
\pgfpathlineto{\pgfpoint{36.121560\du}{8.947878\du}}
\pgfpathlineto{\pgfpoint{36.124419\du}{8.942636\du}}
\pgfpathlineto{\pgfpoint{36.126325\du}{8.937870\du}}
\pgfpathlineto{\pgfpoint{36.129185\du}{8.932152\du}}
\pgfpathlineto{\pgfpoint{36.131091\du}{8.926910\du}}
\pgfpathlineto{\pgfpoint{36.133474\du}{8.921667\du}}
\pgfpathlineto{\pgfpoint{36.134427\du}{8.915472\du}}
\pgfpathlineto{\pgfpoint{36.136333\du}{8.910230\du}}
\pgfpathlineto{\pgfpoint{36.136810\du}{8.904034\du}}
\pgfpathlineto{\pgfpoint{36.137763\du}{8.898792\du}}
\pgfpathlineto{\pgfpoint{36.138239\du}{8.892120\du}}
\pgfpathlineto{\pgfpoint{36.138239\du}{8.886878\du}}
\pgfusepath{fill}
\pgfsetlinewidth{0.000000\du}
\pgfsetbuttcap
\pgfsetmiterjoin
\pgfsetdash{}{0pt}
\definecolor{dialinecolor}{rgb}{0.000000, 0.000000, 0.000000}
\pgfsetstrokecolor{dialinecolor}
\pgfpathmoveto{\pgfpoint{34.994010\du}{8.877347\du}}
\pgfpathlineto{\pgfpoint{35.991934\du}{8.877347\du}}
\pgfusepath{stroke}
\pgfsetlinewidth{0.000000\du}
\pgfsetbuttcap
\pgfsetmiterjoin
\pgfsetdash{}{0pt}
\definecolor{dialinecolor}{rgb}{1.000000, 1.000000, 1.000000}
\pgfsetfillcolor{dialinecolor}
\pgfpathmoveto{\pgfpoint{35.097425\du}{9.309113\du}}
\pgfpathlineto{\pgfpoint{35.097425\du}{9.303395\du}}
\pgfpathlineto{\pgfpoint{35.096948\du}{9.297199\du}}
\pgfpathlineto{\pgfpoint{35.096471\du}{9.291957\du}}
\pgfpathlineto{\pgfpoint{35.095042\du}{9.286238\du}}
\pgfpathlineto{\pgfpoint{35.094089\du}{9.280520\du}}
\pgfpathlineto{\pgfpoint{35.092182\du}{9.274801\du}}
\pgfpathlineto{\pgfpoint{35.090753\du}{9.269082\du}}
\pgfpathlineto{\pgfpoint{35.087893\du}{9.263840\du}}
\pgfpathlineto{\pgfpoint{35.085511\du}{9.258598\du}}
\pgfpathlineto{\pgfpoint{35.083128\du}{9.253356\du}}
\pgfpathlineto{\pgfpoint{35.080268\du}{9.248590\du}}
\pgfpathlineto{\pgfpoint{35.077409\du}{9.243348\du}}
\pgfpathlineto{\pgfpoint{35.073596\du}{9.238582\du}}
\pgfpathlineto{\pgfpoint{35.070260\du}{9.233816\du}}
\pgfpathlineto{\pgfpoint{35.065971\du}{9.230004\du}}
\pgfpathlineto{\pgfpoint{35.061682\du}{9.225715\du}}
\pgfpathlineto{\pgfpoint{35.057870\du}{9.221902\du}}
\pgfpathlineto{\pgfpoint{35.053581\du}{9.217613\du}}
\pgfpathlineto{\pgfpoint{35.048815\du}{9.214754\du}}
\pgfpathlineto{\pgfpoint{35.044049\du}{9.210941\du}}
\pgfpathlineto{\pgfpoint{35.038331\du}{9.208082\du}}
\pgfpathlineto{\pgfpoint{35.034042\du}{9.205699\du}}
\pgfpathlineto{\pgfpoint{35.028323\du}{9.203316\du}}
\pgfpathlineto{\pgfpoint{35.023081\du}{9.200934\du}}
\pgfpathlineto{\pgfpoint{35.016885\du}{9.199027\du}}
\pgfpathlineto{\pgfpoint{35.011643\du}{9.197598\du}}
\pgfpathlineto{\pgfpoint{35.006401\du}{9.196168\du}}
\pgfpathlineto{\pgfpoint{35.000206\du}{9.195215\du}}
\pgfpathlineto{\pgfpoint{34.994487\du}{9.194262\du}}
\pgfpathlineto{\pgfpoint{34.989245\du}{9.193785\du}}
\pgfpathlineto{\pgfpoint{34.983526\du}{9.193785\du}}
\pgfpathlineto{\pgfpoint{34.983526\du}{9.193785\du}}
\pgfpathlineto{\pgfpoint{34.977331\du}{9.193785\du}}
\pgfpathlineto{\pgfpoint{34.971612\du}{9.194262\du}}
\pgfpathlineto{\pgfpoint{34.965893\du}{9.195215\du}}
\pgfpathlineto{\pgfpoint{34.959698\du}{9.196168\du}}
\pgfpathlineto{\pgfpoint{34.954456\du}{9.197598\du}}
\pgfpathlineto{\pgfpoint{34.949213\du}{9.199027\du}}
\pgfpathlineto{\pgfpoint{34.943018\du}{9.200934\du}}
\pgfpathlineto{\pgfpoint{34.937776\du}{9.203316\du}}
\pgfpathlineto{\pgfpoint{34.933010\du}{9.205699\du}}
\pgfpathlineto{\pgfpoint{34.927768\du}{9.208082\du}}
\pgfpathlineto{\pgfpoint{34.922049\du}{9.210941\du}}
\pgfpathlineto{\pgfpoint{34.917760\du}{9.214754\du}}
\pgfpathlineto{\pgfpoint{34.912518\du}{9.217613\du}}
\pgfpathlineto{\pgfpoint{34.909182\du}{9.221902\du}}
\pgfpathlineto{\pgfpoint{34.904416\du}{9.225715\du}}
\pgfpathlineto{\pgfpoint{34.900127\du}{9.230004\du}}
\pgfpathlineto{\pgfpoint{34.895838\du}{9.233816\du}}
\pgfpathlineto{\pgfpoint{34.892502\du}{9.238582\du}}
\pgfpathlineto{\pgfpoint{34.888690\du}{9.243348\du}}
\pgfpathlineto{\pgfpoint{34.885830\du}{9.248590\du}}
\pgfpathlineto{\pgfpoint{34.882971\du}{9.253356\du}}
\pgfpathlineto{\pgfpoint{34.880588\du}{9.258598\du}}
\pgfpathlineto{\pgfpoint{34.878205\du}{9.263840\du}}
\pgfpathlineto{\pgfpoint{34.876299\du}{9.269082\du}}
\pgfpathlineto{\pgfpoint{34.873916\du}{9.274801\du}}
\pgfpathlineto{\pgfpoint{34.872010\du}{9.280520\du}}
\pgfpathlineto{\pgfpoint{34.871057\du}{9.286238\du}}
\pgfpathlineto{\pgfpoint{34.870104\du}{9.291957\du}}
\pgfpathlineto{\pgfpoint{34.869151\du}{9.297199\du}}
\pgfpathlineto{\pgfpoint{34.868674\du}{9.303395\du}}
\pgfpathlineto{\pgfpoint{34.868674\du}{9.309113\du}}
\pgfpathlineto{\pgfpoint{34.868674\du}{9.309113\du}}
\pgfpathlineto{\pgfpoint{34.868674\du}{9.315309\du}}
\pgfpathlineto{\pgfpoint{34.869151\du}{9.321028\du}}
\pgfpathlineto{\pgfpoint{34.870104\du}{9.326746\du}}
\pgfpathlineto{\pgfpoint{34.871057\du}{9.332942\du}}
\pgfpathlineto{\pgfpoint{34.872010\du}{9.338184\du}}
\pgfpathlineto{\pgfpoint{34.873916\du}{9.343903\du}}
\pgfpathlineto{\pgfpoint{34.876299\du}{9.349621\du}}
\pgfpathlineto{\pgfpoint{34.878205\du}{9.354864\du}}
\pgfpathlineto{\pgfpoint{34.880588\du}{9.360106\du}}
\pgfpathlineto{\pgfpoint{34.882971\du}{9.365825\du}}
\pgfpathlineto{\pgfpoint{34.885830\du}{9.370590\du}}
\pgfpathlineto{\pgfpoint{34.888690\du}{9.375356\du}}
\pgfpathlineto{\pgfpoint{34.892502\du}{9.380121\du}}
\pgfpathlineto{\pgfpoint{34.895838\du}{9.384887\du}}
\pgfpathlineto{\pgfpoint{34.900127\du}{9.389176\du}}
\pgfpathlineto{\pgfpoint{34.904416\du}{9.392512\du}}
\pgfpathlineto{\pgfpoint{34.909182\du}{9.396801\du}}
\pgfpathlineto{\pgfpoint{34.912518\du}{9.401090\du}}
\pgfpathlineto{\pgfpoint{34.917760\du}{9.403950\du}}
\pgfpathlineto{\pgfpoint{34.922049\du}{9.407762\du}}
\pgfpathlineto{\pgfpoint{34.927768\du}{9.410621\du}}
\pgfpathlineto{\pgfpoint{34.933010\du}{9.413004\du}}
\pgfpathlineto{\pgfpoint{34.937776\du}{9.415387\du}}
\pgfpathlineto{\pgfpoint{34.943018\du}{9.417770\du}}
\pgfpathlineto{\pgfpoint{34.949213\du}{9.419676\du}}
\pgfpathlineto{\pgfpoint{34.954456\du}{9.421106\du}}
\pgfpathlineto{\pgfpoint{34.959698\du}{9.422536\du}}
\pgfpathlineto{\pgfpoint{34.965893\du}{9.423965\du}}
\pgfpathlineto{\pgfpoint{34.971612\du}{9.424442\du}}
\pgfpathlineto{\pgfpoint{34.977331\du}{9.424918\du}}
\pgfpathlineto{\pgfpoint{34.983526\du}{9.424918\du}}
\pgfpathlineto{\pgfpoint{34.983526\du}{9.424918\du}}
\pgfpathlineto{\pgfpoint{34.989245\du}{9.424918\du}}
\pgfpathlineto{\pgfpoint{34.994487\du}{9.424442\du}}
\pgfpathlineto{\pgfpoint{35.000206\du}{9.423965\du}}
\pgfpathlineto{\pgfpoint{35.006401\du}{9.422536\du}}
\pgfpathlineto{\pgfpoint{35.011643\du}{9.421106\du}}
\pgfpathlineto{\pgfpoint{35.016885\du}{9.419676\du}}
\pgfpathlineto{\pgfpoint{35.023081\du}{9.417770\du}}
\pgfpathlineto{\pgfpoint{35.028323\du}{9.415387\du}}
\pgfpathlineto{\pgfpoint{35.034042\du}{9.413004\du}}
\pgfpathlineto{\pgfpoint{35.038331\du}{9.410621\du}}
\pgfpathlineto{\pgfpoint{35.044049\du}{9.407762\du}}
\pgfpathlineto{\pgfpoint{35.048815\du}{9.403950\du}}
\pgfpathlineto{\pgfpoint{35.053581\du}{9.401090\du}}
\pgfpathlineto{\pgfpoint{35.057870\du}{9.396801\du}}
\pgfpathlineto{\pgfpoint{35.061682\du}{9.392512\du}}
\pgfpathlineto{\pgfpoint{35.065971\du}{9.389176\du}}
\pgfpathlineto{\pgfpoint{35.070260\du}{9.384887\du}}
\pgfpathlineto{\pgfpoint{35.073596\du}{9.380121\du}}
\pgfpathlineto{\pgfpoint{35.077409\du}{9.375356\du}}
\pgfpathlineto{\pgfpoint{35.080268\du}{9.370590\du}}
\pgfpathlineto{\pgfpoint{35.083128\du}{9.365825\du}}
\pgfpathlineto{\pgfpoint{35.085511\du}{9.360106\du}}
\pgfpathlineto{\pgfpoint{35.087893\du}{9.354864\du}}
\pgfpathlineto{\pgfpoint{35.090753\du}{9.349621\du}}
\pgfpathlineto{\pgfpoint{35.092182\du}{9.343903\du}}
\pgfpathlineto{\pgfpoint{35.094089\du}{9.338184\du}}
\pgfpathlineto{\pgfpoint{35.095042\du}{9.332942\du}}
\pgfpathlineto{\pgfpoint{35.096471\du}{9.326746\du}}
\pgfpathlineto{\pgfpoint{35.096948\du}{9.321028\du}}
\pgfpathlineto{\pgfpoint{35.097425\du}{9.315309\du}}
\pgfpathlineto{\pgfpoint{35.097425\du}{9.309113\du}}
\pgfusepath{fill}
\pgfsetbuttcap
\pgfsetmiterjoin
\pgfsetdash{}{0pt}
\definecolor{dialinecolor}{rgb}{1.000000, 1.000000, 1.000000}
\pgfsetfillcolor{dialinecolor}
\pgfpathmoveto{\pgfpoint{36.138239\du}{9.309113\du}}
\pgfpathlineto{\pgfpoint{36.138239\du}{9.303395\du}}
\pgfpathlineto{\pgfpoint{36.137763\du}{9.297199\du}}
\pgfpathlineto{\pgfpoint{36.136810\du}{9.291957\du}}
\pgfpathlineto{\pgfpoint{36.136333\du}{9.286238\du}}
\pgfpathlineto{\pgfpoint{36.134427\du}{9.280520\du}}
\pgfpathlineto{\pgfpoint{36.133474\du}{9.274801\du}}
\pgfpathlineto{\pgfpoint{36.131091\du}{9.269082\du}}
\pgfpathlineto{\pgfpoint{36.129185\du}{9.263840\du}}
\pgfpathlineto{\pgfpoint{36.126325\du}{9.258598\du}}
\pgfpathlineto{\pgfpoint{36.124419\du}{9.253356\du}}
\pgfpathlineto{\pgfpoint{36.121560\du}{9.248590\du}}
\pgfpathlineto{\pgfpoint{36.118224\du}{9.243348\du}}
\pgfpathlineto{\pgfpoint{36.114411\du}{9.238582\du}}
\pgfpathlineto{\pgfpoint{36.110599\du}{9.233816\du}}
\pgfpathlineto{\pgfpoint{36.106786\du}{9.230004\du}}
\pgfpathlineto{\pgfpoint{36.103450\du}{9.225715\du}}
\pgfpathlineto{\pgfpoint{36.098208\du}{9.221902\du}}
\pgfpathlineto{\pgfpoint{36.094395\du}{9.217613\du}}
\pgfpathlineto{\pgfpoint{36.089153\du}{9.214754\du}}
\pgfpathlineto{\pgfpoint{36.084388\du}{9.210941\du}}
\pgfpathlineto{\pgfpoint{36.079622\du}{9.208082\du}}
\pgfpathlineto{\pgfpoint{36.074380\du}{9.205699\du}}
\pgfpathlineto{\pgfpoint{36.069138\du}{9.203316\du}}
\pgfpathlineto{\pgfpoint{36.063419\du}{9.200934\du}}
\pgfpathlineto{\pgfpoint{36.058653\du}{9.199027\du}}
\pgfpathlineto{\pgfpoint{36.052934\du}{9.197598\du}}
\pgfpathlineto{\pgfpoint{36.046739\du}{9.196168\du}}
\pgfpathlineto{\pgfpoint{36.041497\du}{9.195215\du}}
\pgfpathlineto{\pgfpoint{36.035778\du}{9.194262\du}}
\pgfpathlineto{\pgfpoint{36.030059\du}{9.193785\du}}
\pgfpathlineto{\pgfpoint{36.024341\du}{9.193785\du}}
\pgfpathlineto{\pgfpoint{36.024341\du}{9.193785\du}}
\pgfpathlineto{\pgfpoint{36.018145\du}{9.193785\du}}
\pgfpathlineto{\pgfpoint{36.012903\du}{9.194262\du}}
\pgfpathlineto{\pgfpoint{36.007184\du}{9.195215\du}}
\pgfpathlineto{\pgfpoint{36.001942\du}{9.196168\du}}
\pgfpathlineto{\pgfpoint{35.996223\du}{9.197598\du}}
\pgfpathlineto{\pgfpoint{35.990505\du}{9.199027\du}}
\pgfpathlineto{\pgfpoint{35.984786\du}{9.200934\du}}
\pgfpathlineto{\pgfpoint{35.979544\du}{9.203316\du}}
\pgfpathlineto{\pgfpoint{35.974301\du}{9.205699\du}}
\pgfpathlineto{\pgfpoint{35.969536\du}{9.208082\du}}
\pgfpathlineto{\pgfpoint{35.963817\du}{9.210941\du}}
\pgfpathlineto{\pgfpoint{35.959051\du}{9.214754\du}}
\pgfpathlineto{\pgfpoint{35.954762\du}{9.217613\du}}
\pgfpathlineto{\pgfpoint{35.950473\du}{9.221902\du}}
\pgfpathlineto{\pgfpoint{35.945708\du}{9.225715\du}}
\pgfpathlineto{\pgfpoint{35.942372\du}{9.230004\du}}
\pgfpathlineto{\pgfpoint{35.937606\du}{9.233816\du}}
\pgfpathlineto{\pgfpoint{35.934270\du}{9.238582\du}}
\pgfpathlineto{\pgfpoint{35.930934\du}{9.243348\du}}
\pgfpathlineto{\pgfpoint{35.927598\du}{9.248590\du}}
\pgfpathlineto{\pgfpoint{35.924739\du}{9.253356\du}}
\pgfpathlineto{\pgfpoint{35.922356\du}{9.258598\du}}
\pgfpathlineto{\pgfpoint{35.919497\du}{9.263840\du}}
\pgfpathlineto{\pgfpoint{35.917590\du}{9.269082\du}}
\pgfpathlineto{\pgfpoint{35.915684\du}{9.274801\du}}
\pgfpathlineto{\pgfpoint{35.913778\du}{9.280520\du}}
\pgfpathlineto{\pgfpoint{35.912825\du}{9.286238\du}}
\pgfpathlineto{\pgfpoint{35.911395\du}{9.291957\du}}
\pgfpathlineto{\pgfpoint{35.910919\du}{9.297199\du}}
\pgfpathlineto{\pgfpoint{35.910442\du}{9.303395\du}}
\pgfpathlineto{\pgfpoint{35.910442\du}{9.309113\du}}
\pgfpathlineto{\pgfpoint{35.910442\du}{9.309113\du}}
\pgfpathlineto{\pgfpoint{35.910442\du}{9.315309\du}}
\pgfpathlineto{\pgfpoint{35.910919\du}{9.321028\du}}
\pgfpathlineto{\pgfpoint{35.911395\du}{9.326746\du}}
\pgfpathlineto{\pgfpoint{35.912825\du}{9.332942\du}}
\pgfpathlineto{\pgfpoint{35.913778\du}{9.338184\du}}
\pgfpathlineto{\pgfpoint{35.915684\du}{9.343903\du}}
\pgfpathlineto{\pgfpoint{35.917590\du}{9.349621\du}}
\pgfpathlineto{\pgfpoint{35.919497\du}{9.354864\du}}
\pgfpathlineto{\pgfpoint{35.922356\du}{9.360106\du}}
\pgfpathlineto{\pgfpoint{35.924739\du}{9.365825\du}}
\pgfpathlineto{\pgfpoint{35.927598\du}{9.370590\du}}
\pgfpathlineto{\pgfpoint{35.930934\du}{9.375356\du}}
\pgfpathlineto{\pgfpoint{35.934270\du}{9.380121\du}}
\pgfpathlineto{\pgfpoint{35.937606\du}{9.384887\du}}
\pgfpathlineto{\pgfpoint{35.942372\du}{9.389176\du}}
\pgfpathlineto{\pgfpoint{35.945708\du}{9.392512\du}}
\pgfpathlineto{\pgfpoint{35.950473\du}{9.396801\du}}
\pgfpathlineto{\pgfpoint{35.954762\du}{9.401090\du}}
\pgfpathlineto{\pgfpoint{35.959051\du}{9.403950\du}}
\pgfpathlineto{\pgfpoint{35.963817\du}{9.407762\du}}
\pgfpathlineto{\pgfpoint{35.969536\du}{9.410621\du}}
\pgfpathlineto{\pgfpoint{35.974301\du}{9.413004\du}}
\pgfpathlineto{\pgfpoint{35.979544\du}{9.415387\du}}
\pgfpathlineto{\pgfpoint{35.984786\du}{9.417770\du}}
\pgfpathlineto{\pgfpoint{35.990505\du}{9.419676\du}}
\pgfpathlineto{\pgfpoint{35.996223\du}{9.421106\du}}
\pgfpathlineto{\pgfpoint{36.001942\du}{9.422536\du}}
\pgfpathlineto{\pgfpoint{36.007184\du}{9.423965\du}}
\pgfpathlineto{\pgfpoint{36.012903\du}{9.424442\du}}
\pgfpathlineto{\pgfpoint{36.018145\du}{9.424918\du}}
\pgfpathlineto{\pgfpoint{36.024341\du}{9.424918\du}}
\pgfpathlineto{\pgfpoint{36.024341\du}{9.424918\du}}
\pgfpathlineto{\pgfpoint{36.030059\du}{9.424918\du}}
\pgfpathlineto{\pgfpoint{36.035778\du}{9.424442\du}}
\pgfpathlineto{\pgfpoint{36.041497\du}{9.423965\du}}
\pgfpathlineto{\pgfpoint{36.046739\du}{9.422536\du}}
\pgfpathlineto{\pgfpoint{36.052934\du}{9.421106\du}}
\pgfpathlineto{\pgfpoint{36.058653\du}{9.419676\du}}
\pgfpathlineto{\pgfpoint{36.063419\du}{9.417770\du}}
\pgfpathlineto{\pgfpoint{36.069138\du}{9.415387\du}}
\pgfpathlineto{\pgfpoint{36.074380\du}{9.413004\du}}
\pgfpathlineto{\pgfpoint{36.079622\du}{9.410621\du}}
\pgfpathlineto{\pgfpoint{36.084388\du}{9.407762\du}}
\pgfpathlineto{\pgfpoint{36.089153\du}{9.403950\du}}
\pgfpathlineto{\pgfpoint{36.094395\du}{9.401090\du}}
\pgfpathlineto{\pgfpoint{36.098208\du}{9.396801\du}}
\pgfpathlineto{\pgfpoint{36.103450\du}{9.392512\du}}
\pgfpathlineto{\pgfpoint{36.106786\du}{9.389176\du}}
\pgfpathlineto{\pgfpoint{36.110599\du}{9.384887\du}}
\pgfpathlineto{\pgfpoint{36.114411\du}{9.380121\du}}
\pgfpathlineto{\pgfpoint{36.118224\du}{9.375356\du}}
\pgfpathlineto{\pgfpoint{36.121560\du}{9.370590\du}}
\pgfpathlineto{\pgfpoint{36.124419\du}{9.365825\du}}
\pgfpathlineto{\pgfpoint{36.126325\du}{9.360106\du}}
\pgfpathlineto{\pgfpoint{36.129185\du}{9.354864\du}}
\pgfpathlineto{\pgfpoint{36.131091\du}{9.349621\du}}
\pgfpathlineto{\pgfpoint{36.133474\du}{9.343903\du}}
\pgfpathlineto{\pgfpoint{36.134427\du}{9.338184\du}}
\pgfpathlineto{\pgfpoint{36.136333\du}{9.332942\du}}
\pgfpathlineto{\pgfpoint{36.136810\du}{9.326746\du}}
\pgfpathlineto{\pgfpoint{36.137763\du}{9.321028\du}}
\pgfpathlineto{\pgfpoint{36.138239\du}{9.315309\du}}
\pgfpathlineto{\pgfpoint{36.138239\du}{9.309113\du}}
\pgfusepath{fill}
\pgfsetlinewidth{0.000000\du}
\pgfsetbuttcap
\pgfsetmiterjoin
\pgfsetdash{}{0pt}
\definecolor{dialinecolor}{rgb}{0.000000, 0.000000, 0.000000}
\pgfsetstrokecolor{dialinecolor}
\pgfpathmoveto{\pgfpoint{34.994010\du}{9.299106\du}}
\pgfpathlineto{\pgfpoint{35.991934\du}{9.299106\du}}
\pgfusepath{stroke}
\pgfsetlinewidth{0.000000\du}
\pgfsetbuttcap
\pgfsetmiterjoin
\pgfsetdash{}{0pt}
\definecolor{dialinecolor}{rgb}{1.000000, 1.000000, 1.000000}
\pgfsetfillcolor{dialinecolor}
\pgfpathmoveto{\pgfpoint{35.097425\du}{9.731349\du}}
\pgfpathlineto{\pgfpoint{35.097425\du}{9.725630\du}}
\pgfpathlineto{\pgfpoint{35.096948\du}{9.719435\du}}
\pgfpathlineto{\pgfpoint{35.096471\du}{9.714192\du}}
\pgfpathlineto{\pgfpoint{35.095042\du}{9.707997\du}}
\pgfpathlineto{\pgfpoint{35.094089\du}{9.702278\du}}
\pgfpathlineto{\pgfpoint{35.092182\du}{9.696560\du}}
\pgfpathlineto{\pgfpoint{35.090753\du}{9.691317\du}}
\pgfpathlineto{\pgfpoint{35.087893\du}{9.686075\du}}
\pgfpathlineto{\pgfpoint{35.085511\du}{9.680356\du}}
\pgfpathlineto{\pgfpoint{35.083128\du}{9.675114\du}}
\pgfpathlineto{\pgfpoint{35.080268\du}{9.670349\du}}
\pgfpathlineto{\pgfpoint{35.077409\du}{9.665583\du}}
\pgfpathlineto{\pgfpoint{35.073596\du}{9.660817\du}}
\pgfpathlineto{\pgfpoint{35.070260\du}{9.656052\du}}
\pgfpathlineto{\pgfpoint{35.065971\du}{9.651763\du}}
\pgfpathlineto{\pgfpoint{35.061682\du}{9.647474\du}}
\pgfpathlineto{\pgfpoint{35.057870\du}{9.644138\du}}
\pgfpathlineto{\pgfpoint{35.053581\du}{9.639849\du}}
\pgfpathlineto{\pgfpoint{35.048815\du}{9.636989\du}}
\pgfpathlineto{\pgfpoint{35.044049\du}{9.633177\du}}
\pgfpathlineto{\pgfpoint{35.038331\du}{9.630317\du}}
\pgfpathlineto{\pgfpoint{35.034042\du}{9.627934\du}}
\pgfpathlineto{\pgfpoint{35.028323\du}{9.625552\du}}
\pgfpathlineto{\pgfpoint{35.023081\du}{9.623169\du}}
\pgfpathlineto{\pgfpoint{35.016885\du}{9.621263\du}}
\pgfpathlineto{\pgfpoint{35.011643\du}{9.618880\du}}
\pgfpathlineto{\pgfpoint{35.006401\du}{9.618403\du}}
\pgfpathlineto{\pgfpoint{35.000206\du}{9.616973\du}}
\pgfpathlineto{\pgfpoint{34.994487\du}{9.616020\du}}
\pgfpathlineto{\pgfpoint{34.989245\du}{9.616020\du}}
\pgfpathlineto{\pgfpoint{34.983526\du}{9.616020\du}}
\pgfpathlineto{\pgfpoint{34.983526\du}{9.616020\du}}
\pgfpathlineto{\pgfpoint{34.977331\du}{9.616020\du}}
\pgfpathlineto{\pgfpoint{34.971612\du}{9.616020\du}}
\pgfpathlineto{\pgfpoint{34.965893\du}{9.616973\du}}
\pgfpathlineto{\pgfpoint{34.959698\du}{9.618403\du}}
\pgfpathlineto{\pgfpoint{34.954456\du}{9.618880\du}}
\pgfpathlineto{\pgfpoint{34.949213\du}{9.621263\du}}
\pgfpathlineto{\pgfpoint{34.943018\du}{9.623169\du}}
\pgfpathlineto{\pgfpoint{34.937776\du}{9.625552\du}}
\pgfpathlineto{\pgfpoint{34.933010\du}{9.627934\du}}
\pgfpathlineto{\pgfpoint{34.927768\du}{9.630317\du}}
\pgfpathlineto{\pgfpoint{34.922049\du}{9.633177\du}}
\pgfpathlineto{\pgfpoint{34.917760\du}{9.636989\du}}
\pgfpathlineto{\pgfpoint{34.912518\du}{9.639849\du}}
\pgfpathlineto{\pgfpoint{34.909182\du}{9.644138\du}}
\pgfpathlineto{\pgfpoint{34.904416\du}{9.647474\du}}
\pgfpathlineto{\pgfpoint{34.900127\du}{9.651763\du}}
\pgfpathlineto{\pgfpoint{34.895838\du}{9.656052\du}}
\pgfpathlineto{\pgfpoint{34.892502\du}{9.660817\du}}
\pgfpathlineto{\pgfpoint{34.888690\du}{9.665583\du}}
\pgfpathlineto{\pgfpoint{34.885830\du}{9.670349\du}}
\pgfpathlineto{\pgfpoint{34.882971\du}{9.675114\du}}
\pgfpathlineto{\pgfpoint{34.880588\du}{9.680356\du}}
\pgfpathlineto{\pgfpoint{34.878205\du}{9.686075\du}}
\pgfpathlineto{\pgfpoint{34.876299\du}{9.691317\du}}
\pgfpathlineto{\pgfpoint{34.873916\du}{9.696560\du}}
\pgfpathlineto{\pgfpoint{34.872010\du}{9.702278\du}}
\pgfpathlineto{\pgfpoint{34.871057\du}{9.707997\du}}
\pgfpathlineto{\pgfpoint{34.870104\du}{9.714192\du}}
\pgfpathlineto{\pgfpoint{34.869151\du}{9.719435\du}}
\pgfpathlineto{\pgfpoint{34.868674\du}{9.725630\du}}
\pgfpathlineto{\pgfpoint{34.868674\du}{9.731349\du}}
\pgfpathlineto{\pgfpoint{34.868674\du}{9.731349\du}}
\pgfpathlineto{\pgfpoint{34.868674\du}{9.737544\du}}
\pgfpathlineto{\pgfpoint{34.869151\du}{9.743263\du}}
\pgfpathlineto{\pgfpoint{34.870104\du}{9.748982\du}}
\pgfpathlineto{\pgfpoint{34.871057\du}{9.754700\du}}
\pgfpathlineto{\pgfpoint{34.872010\du}{9.759943\du}}
\pgfpathlineto{\pgfpoint{34.873916\du}{9.766138\du}}
\pgfpathlineto{\pgfpoint{34.876299\du}{9.770904\du}}
\pgfpathlineto{\pgfpoint{34.878205\du}{9.777099\du}}
\pgfpathlineto{\pgfpoint{34.880588\du}{9.782341\du}}
\pgfpathlineto{\pgfpoint{34.882971\du}{9.787583\du}}
\pgfpathlineto{\pgfpoint{34.885830\du}{9.792349\du}}
\pgfpathlineto{\pgfpoint{34.888690\du}{9.797115\du}}
\pgfpathlineto{\pgfpoint{34.892502\du}{9.801880\du}}
\pgfpathlineto{\pgfpoint{34.895838\du}{9.806646\du}}
\pgfpathlineto{\pgfpoint{34.900127\du}{9.810935\du}}
\pgfpathlineto{\pgfpoint{34.904416\du}{9.815224\du}}
\pgfpathlineto{\pgfpoint{34.909182\du}{9.818560\du}}
\pgfpathlineto{\pgfpoint{34.912518\du}{9.822849\du}}
\pgfpathlineto{\pgfpoint{34.917760\du}{9.826185\du}}
\pgfpathlineto{\pgfpoint{34.922049\du}{9.829521\du}}
\pgfpathlineto{\pgfpoint{34.927768\du}{9.832380\du}}
\pgfpathlineto{\pgfpoint{34.933010\du}{9.834763\du}}
\pgfpathlineto{\pgfpoint{34.937776\du}{9.837146\du}}
\pgfpathlineto{\pgfpoint{34.943018\du}{9.839529\du}}
\pgfpathlineto{\pgfpoint{34.949213\du}{9.841435\du}}
\pgfpathlineto{\pgfpoint{34.954456\du}{9.843341\du}}
\pgfpathlineto{\pgfpoint{34.959698\du}{9.844294\du}}
\pgfpathlineto{\pgfpoint{34.965893\du}{9.845247\du}}
\pgfpathlineto{\pgfpoint{34.971612\du}{9.846201\du}}
\pgfpathlineto{\pgfpoint{34.977331\du}{9.846677\du}}
\pgfpathlineto{\pgfpoint{34.983526\du}{9.846677\du}}
\pgfpathlineto{\pgfpoint{34.983526\du}{9.846677\du}}
\pgfpathlineto{\pgfpoint{34.989245\du}{9.846677\du}}
\pgfpathlineto{\pgfpoint{34.994487\du}{9.846201\du}}
\pgfpathlineto{\pgfpoint{35.000206\du}{9.845247\du}}
\pgfpathlineto{\pgfpoint{35.006401\du}{9.844294\du}}
\pgfpathlineto{\pgfpoint{35.011643\du}{9.843341\du}}
\pgfpathlineto{\pgfpoint{35.016885\du}{9.841435\du}}
\pgfpathlineto{\pgfpoint{35.023081\du}{9.839529\du}}
\pgfpathlineto{\pgfpoint{35.028323\du}{9.837146\du}}
\pgfpathlineto{\pgfpoint{35.034042\du}{9.834763\du}}
\pgfpathlineto{\pgfpoint{35.038331\du}{9.832380\du}}
\pgfpathlineto{\pgfpoint{35.044049\du}{9.829521\du}}
\pgfpathlineto{\pgfpoint{35.048815\du}{9.826185\du}}
\pgfpathlineto{\pgfpoint{35.053581\du}{9.822849\du}}
\pgfpathlineto{\pgfpoint{35.057870\du}{9.818560\du}}
\pgfpathlineto{\pgfpoint{35.061682\du}{9.815224\du}}
\pgfpathlineto{\pgfpoint{35.065971\du}{9.810935\du}}
\pgfpathlineto{\pgfpoint{35.070260\du}{9.806646\du}}
\pgfpathlineto{\pgfpoint{35.073596\du}{9.801880\du}}
\pgfpathlineto{\pgfpoint{35.077409\du}{9.797115\du}}
\pgfpathlineto{\pgfpoint{35.080268\du}{9.792349\du}}
\pgfpathlineto{\pgfpoint{35.083128\du}{9.787583\du}}
\pgfpathlineto{\pgfpoint{35.085511\du}{9.782341\du}}
\pgfpathlineto{\pgfpoint{35.087893\du}{9.777099\du}}
\pgfpathlineto{\pgfpoint{35.090753\du}{9.770904\du}}
\pgfpathlineto{\pgfpoint{35.092182\du}{9.766138\du}}
\pgfpathlineto{\pgfpoint{35.094089\du}{9.759943\du}}
\pgfpathlineto{\pgfpoint{35.095042\du}{9.754700\du}}
\pgfpathlineto{\pgfpoint{35.096471\du}{9.748982\du}}
\pgfpathlineto{\pgfpoint{35.096948\du}{9.743263\du}}
\pgfpathlineto{\pgfpoint{35.097425\du}{9.737544\du}}
\pgfpathlineto{\pgfpoint{35.097425\du}{9.731349\du}}
\pgfusepath{fill}
\pgfsetbuttcap
\pgfsetmiterjoin
\pgfsetdash{}{0pt}
\definecolor{dialinecolor}{rgb}{1.000000, 1.000000, 1.000000}
\pgfsetfillcolor{dialinecolor}
\pgfpathmoveto{\pgfpoint{36.138239\du}{9.731349\du}}
\pgfpathlineto{\pgfpoint{36.138239\du}{9.725630\du}}
\pgfpathlineto{\pgfpoint{36.137763\du}{9.719435\du}}
\pgfpathlineto{\pgfpoint{36.136810\du}{9.714192\du}}
\pgfpathlineto{\pgfpoint{36.136333\du}{9.707997\du}}
\pgfpathlineto{\pgfpoint{36.134427\du}{9.702278\du}}
\pgfpathlineto{\pgfpoint{36.133474\du}{9.696560\du}}
\pgfpathlineto{\pgfpoint{36.131091\du}{9.691317\du}}
\pgfpathlineto{\pgfpoint{36.129185\du}{9.686075\du}}
\pgfpathlineto{\pgfpoint{36.126325\du}{9.680356\du}}
\pgfpathlineto{\pgfpoint{36.124419\du}{9.675114\du}}
\pgfpathlineto{\pgfpoint{36.121560\du}{9.670349\du}}
\pgfpathlineto{\pgfpoint{36.118224\du}{9.665583\du}}
\pgfpathlineto{\pgfpoint{36.114411\du}{9.660817\du}}
\pgfpathlineto{\pgfpoint{36.110599\du}{9.656052\du}}
\pgfpathlineto{\pgfpoint{36.106786\du}{9.651763\du}}
\pgfpathlineto{\pgfpoint{36.103450\du}{9.647474\du}}
\pgfpathlineto{\pgfpoint{36.098208\du}{9.644138\du}}
\pgfpathlineto{\pgfpoint{36.094395\du}{9.639849\du}}
\pgfpathlineto{\pgfpoint{36.089153\du}{9.636989\du}}
\pgfpathlineto{\pgfpoint{36.084388\du}{9.633177\du}}
\pgfpathlineto{\pgfpoint{36.079622\du}{9.630317\du}}
\pgfpathlineto{\pgfpoint{36.074380\du}{9.627934\du}}
\pgfpathlineto{\pgfpoint{36.069138\du}{9.625552\du}}
\pgfpathlineto{\pgfpoint{36.063419\du}{9.623169\du}}
\pgfpathlineto{\pgfpoint{36.058653\du}{9.621263\du}}
\pgfpathlineto{\pgfpoint{36.052934\du}{9.618880\du}}
\pgfpathlineto{\pgfpoint{36.046739\du}{9.618403\du}}
\pgfpathlineto{\pgfpoint{36.041497\du}{9.616973\du}}
\pgfpathlineto{\pgfpoint{36.035778\du}{9.616020\du}}
\pgfpathlineto{\pgfpoint{36.030059\du}{9.616020\du}}
\pgfpathlineto{\pgfpoint{36.024341\du}{9.616020\du}}
\pgfpathlineto{\pgfpoint{36.024341\du}{9.616020\du}}
\pgfpathlineto{\pgfpoint{36.018145\du}{9.616020\du}}
\pgfpathlineto{\pgfpoint{36.012903\du}{9.616020\du}}
\pgfpathlineto{\pgfpoint{36.007184\du}{9.616973\du}}
\pgfpathlineto{\pgfpoint{36.001942\du}{9.618403\du}}
\pgfpathlineto{\pgfpoint{35.996223\du}{9.618880\du}}
\pgfpathlineto{\pgfpoint{35.990505\du}{9.621263\du}}
\pgfpathlineto{\pgfpoint{35.984786\du}{9.623169\du}}
\pgfpathlineto{\pgfpoint{35.979544\du}{9.625552\du}}
\pgfpathlineto{\pgfpoint{35.974301\du}{9.627934\du}}
\pgfpathlineto{\pgfpoint{35.969536\du}{9.630317\du}}
\pgfpathlineto{\pgfpoint{35.963817\du}{9.633177\du}}
\pgfpathlineto{\pgfpoint{35.959051\du}{9.636989\du}}
\pgfpathlineto{\pgfpoint{35.954762\du}{9.639849\du}}
\pgfpathlineto{\pgfpoint{35.950473\du}{9.644138\du}}
\pgfpathlineto{\pgfpoint{35.945708\du}{9.647474\du}}
\pgfpathlineto{\pgfpoint{35.942372\du}{9.651763\du}}
\pgfpathlineto{\pgfpoint{35.937606\du}{9.656052\du}}
\pgfpathlineto{\pgfpoint{35.934270\du}{9.660817\du}}
\pgfpathlineto{\pgfpoint{35.930934\du}{9.665583\du}}
\pgfpathlineto{\pgfpoint{35.927598\du}{9.670349\du}}
\pgfpathlineto{\pgfpoint{35.924739\du}{9.675114\du}}
\pgfpathlineto{\pgfpoint{35.922356\du}{9.680356\du}}
\pgfpathlineto{\pgfpoint{35.919497\du}{9.686075\du}}
\pgfpathlineto{\pgfpoint{35.917590\du}{9.691317\du}}
\pgfpathlineto{\pgfpoint{35.915684\du}{9.696560\du}}
\pgfpathlineto{\pgfpoint{35.913778\du}{9.702278\du}}
\pgfpathlineto{\pgfpoint{35.912825\du}{9.707997\du}}
\pgfpathlineto{\pgfpoint{35.911395\du}{9.714192\du}}
\pgfpathlineto{\pgfpoint{35.910919\du}{9.719435\du}}
\pgfpathlineto{\pgfpoint{35.910442\du}{9.725630\du}}
\pgfpathlineto{\pgfpoint{35.910442\du}{9.731349\du}}
\pgfpathlineto{\pgfpoint{35.910442\du}{9.731349\du}}
\pgfpathlineto{\pgfpoint{35.910442\du}{9.737544\du}}
\pgfpathlineto{\pgfpoint{35.910919\du}{9.743263\du}}
\pgfpathlineto{\pgfpoint{35.911395\du}{9.748982\du}}
\pgfpathlineto{\pgfpoint{35.912825\du}{9.754700\du}}
\pgfpathlineto{\pgfpoint{35.913778\du}{9.759943\du}}
\pgfpathlineto{\pgfpoint{35.915684\du}{9.766138\du}}
\pgfpathlineto{\pgfpoint{35.917590\du}{9.770904\du}}
\pgfpathlineto{\pgfpoint{35.919497\du}{9.777099\du}}
\pgfpathlineto{\pgfpoint{35.922356\du}{9.782341\du}}
\pgfpathlineto{\pgfpoint{35.924739\du}{9.787583\du}}
\pgfpathlineto{\pgfpoint{35.927598\du}{9.792349\du}}
\pgfpathlineto{\pgfpoint{35.930934\du}{9.797115\du}}
\pgfpathlineto{\pgfpoint{35.934270\du}{9.801880\du}}
\pgfpathlineto{\pgfpoint{35.937606\du}{9.806646\du}}
\pgfpathlineto{\pgfpoint{35.942372\du}{9.810935\du}}
\pgfpathlineto{\pgfpoint{35.945708\du}{9.815224\du}}
\pgfpathlineto{\pgfpoint{35.950473\du}{9.818560\du}}
\pgfpathlineto{\pgfpoint{35.954762\du}{9.822849\du}}
\pgfpathlineto{\pgfpoint{35.959051\du}{9.826185\du}}
\pgfpathlineto{\pgfpoint{35.963817\du}{9.829521\du}}
\pgfpathlineto{\pgfpoint{35.969536\du}{9.832380\du}}
\pgfpathlineto{\pgfpoint{35.974301\du}{9.834763\du}}
\pgfpathlineto{\pgfpoint{35.979544\du}{9.837146\du}}
\pgfpathlineto{\pgfpoint{35.984786\du}{9.839529\du}}
\pgfpathlineto{\pgfpoint{35.990505\du}{9.841435\du}}
\pgfpathlineto{\pgfpoint{35.996223\du}{9.843341\du}}
\pgfpathlineto{\pgfpoint{36.001942\du}{9.844294\du}}
\pgfpathlineto{\pgfpoint{36.007184\du}{9.845247\du}}
\pgfpathlineto{\pgfpoint{36.012903\du}{9.846201\du}}
\pgfpathlineto{\pgfpoint{36.018145\du}{9.846677\du}}
\pgfpathlineto{\pgfpoint{36.024341\du}{9.846677\du}}
\pgfpathlineto{\pgfpoint{36.024341\du}{9.846677\du}}
\pgfpathlineto{\pgfpoint{36.030059\du}{9.846677\du}}
\pgfpathlineto{\pgfpoint{36.035778\du}{9.846201\du}}
\pgfpathlineto{\pgfpoint{36.041497\du}{9.845247\du}}
\pgfpathlineto{\pgfpoint{36.046739\du}{9.844294\du}}
\pgfpathlineto{\pgfpoint{36.052934\du}{9.843341\du}}
\pgfpathlineto{\pgfpoint{36.058653\du}{9.841435\du}}
\pgfpathlineto{\pgfpoint{36.063419\du}{9.839529\du}}
\pgfpathlineto{\pgfpoint{36.069138\du}{9.837146\du}}
\pgfpathlineto{\pgfpoint{36.074380\du}{9.834763\du}}
\pgfpathlineto{\pgfpoint{36.079622\du}{9.832380\du}}
\pgfpathlineto{\pgfpoint{36.084388\du}{9.829521\du}}
\pgfpathlineto{\pgfpoint{36.089153\du}{9.826185\du}}
\pgfpathlineto{\pgfpoint{36.094395\du}{9.822849\du}}
\pgfpathlineto{\pgfpoint{36.098208\du}{9.818560\du}}
\pgfpathlineto{\pgfpoint{36.103450\du}{9.815224\du}}
\pgfpathlineto{\pgfpoint{36.106786\du}{9.810935\du}}
\pgfpathlineto{\pgfpoint{36.110599\du}{9.806646\du}}
\pgfpathlineto{\pgfpoint{36.114411\du}{9.801880\du}}
\pgfpathlineto{\pgfpoint{36.118224\du}{9.797115\du}}
\pgfpathlineto{\pgfpoint{36.121560\du}{9.792349\du}}
\pgfpathlineto{\pgfpoint{36.124419\du}{9.787583\du}}
\pgfpathlineto{\pgfpoint{36.126325\du}{9.782341\du}}
\pgfpathlineto{\pgfpoint{36.129185\du}{9.777099\du}}
\pgfpathlineto{\pgfpoint{36.131091\du}{9.770904\du}}
\pgfpathlineto{\pgfpoint{36.133474\du}{9.766138\du}}
\pgfpathlineto{\pgfpoint{36.134427\du}{9.759943\du}}
\pgfpathlineto{\pgfpoint{36.136333\du}{9.754700\du}}
\pgfpathlineto{\pgfpoint{36.136810\du}{9.748982\du}}
\pgfpathlineto{\pgfpoint{36.137763\du}{9.743263\du}}
\pgfpathlineto{\pgfpoint{36.138239\du}{9.737544\du}}
\pgfpathlineto{\pgfpoint{36.138239\du}{9.731349\du}}
\pgfusepath{fill}
\pgfsetlinewidth{0.000000\du}
\pgfsetbuttcap
\pgfsetmiterjoin
\pgfsetdash{}{0pt}
\definecolor{dialinecolor}{rgb}{0.000000, 0.000000, 0.000000}
\pgfsetstrokecolor{dialinecolor}
\pgfpathmoveto{\pgfpoint{34.994010\du}{9.721341\du}}
\pgfpathlineto{\pgfpoint{35.991934\du}{9.721341\du}}
\pgfusepath{stroke}
\pgfsetlinewidth{0.000000\du}
\pgfsetbuttcap
\pgfsetmiterjoin
\pgfsetdash{}{0pt}
\definecolor{dialinecolor}{rgb}{1.000000, 1.000000, 1.000000}
\pgfsetfillcolor{dialinecolor}
\pgfpathmoveto{\pgfpoint{35.097425\du}{10.153584\du}}
\pgfpathlineto{\pgfpoint{35.097425\du}{10.147389\du}}
\pgfpathlineto{\pgfpoint{35.096948\du}{10.141193\du}}
\pgfpathlineto{\pgfpoint{35.096471\du}{10.135951\du}}
\pgfpathlineto{\pgfpoint{35.095042\du}{10.130232\du}}
\pgfpathlineto{\pgfpoint{35.094089\du}{10.124514\du}}
\pgfpathlineto{\pgfpoint{35.092182\du}{10.118795\du}}
\pgfpathlineto{\pgfpoint{35.090753\du}{10.113076\du}}
\pgfpathlineto{\pgfpoint{35.087893\du}{10.107357\du}}
\pgfpathlineto{\pgfpoint{35.085511\du}{10.102592\du}}
\pgfpathlineto{\pgfpoint{35.083128\du}{10.096873\du}}
\pgfpathlineto{\pgfpoint{35.080268\du}{10.092107\du}}
\pgfpathlineto{\pgfpoint{35.077409\du}{10.087342\du}}
\pgfpathlineto{\pgfpoint{35.073596\du}{10.082576\du}}
\pgfpathlineto{\pgfpoint{35.070260\du}{10.077810\du}}
\pgfpathlineto{\pgfpoint{35.065971\du}{10.073998\du}}
\pgfpathlineto{\pgfpoint{35.061682\du}{10.069709\du}}
\pgfpathlineto{\pgfpoint{35.057870\du}{10.065896\du}}
\pgfpathlineto{\pgfpoint{35.053581\du}{10.061607\du}}
\pgfpathlineto{\pgfpoint{35.048815\du}{10.058748\du}}
\pgfpathlineto{\pgfpoint{35.044049\du}{10.055412\du}}
\pgfpathlineto{\pgfpoint{35.038331\du}{10.052076\du}}
\pgfpathlineto{\pgfpoint{35.034042\du}{10.049693\du}}
\pgfpathlineto{\pgfpoint{35.028323\du}{10.047310\du}}
\pgfpathlineto{\pgfpoint{35.023081\du}{10.044928\du}}
\pgfpathlineto{\pgfpoint{35.016885\du}{10.043021\du}}
\pgfpathlineto{\pgfpoint{35.011643\du}{10.041592\du}}
\pgfpathlineto{\pgfpoint{35.006401\du}{10.040162\du}}
\pgfpathlineto{\pgfpoint{35.000206\du}{10.039209\du}}
\pgfpathlineto{\pgfpoint{34.994487\du}{10.038256\du}}
\pgfpathlineto{\pgfpoint{34.989245\du}{10.037779\du}}
\pgfpathlineto{\pgfpoint{34.983526\du}{10.037779\du}}
\pgfpathlineto{\pgfpoint{34.983526\du}{10.037779\du}}
\pgfpathlineto{\pgfpoint{34.977331\du}{10.037779\du}}
\pgfpathlineto{\pgfpoint{34.971612\du}{10.038256\du}}
\pgfpathlineto{\pgfpoint{34.965893\du}{10.039209\du}}
\pgfpathlineto{\pgfpoint{34.959698\du}{10.040162\du}}
\pgfpathlineto{\pgfpoint{34.954456\du}{10.041592\du}}
\pgfpathlineto{\pgfpoint{34.949213\du}{10.043021\du}}
\pgfpathlineto{\pgfpoint{34.943018\du}{10.044928\du}}
\pgfpathlineto{\pgfpoint{34.937776\du}{10.047310\du}}
\pgfpathlineto{\pgfpoint{34.933010\du}{10.049693\du}}
\pgfpathlineto{\pgfpoint{34.927768\du}{10.052076\du}}
\pgfpathlineto{\pgfpoint{34.922049\du}{10.055412\du}}
\pgfpathlineto{\pgfpoint{34.917760\du}{10.058748\du}}
\pgfpathlineto{\pgfpoint{34.912518\du}{10.061607\du}}
\pgfpathlineto{\pgfpoint{34.909182\du}{10.065896\du}}
\pgfpathlineto{\pgfpoint{34.904416\du}{10.069709\du}}
\pgfpathlineto{\pgfpoint{34.900127\du}{10.073998\du}}
\pgfpathlineto{\pgfpoint{34.895838\du}{10.077810\du}}
\pgfpathlineto{\pgfpoint{34.892502\du}{10.082576\du}}
\pgfpathlineto{\pgfpoint{34.888690\du}{10.087342\du}}
\pgfpathlineto{\pgfpoint{34.885830\du}{10.092107\du}}
\pgfpathlineto{\pgfpoint{34.882971\du}{10.096873\du}}
\pgfpathlineto{\pgfpoint{34.880588\du}{10.102592\du}}
\pgfpathlineto{\pgfpoint{34.878205\du}{10.107357\du}}
\pgfpathlineto{\pgfpoint{34.876299\du}{10.113076\du}}
\pgfpathlineto{\pgfpoint{34.873916\du}{10.118795\du}}
\pgfpathlineto{\pgfpoint{34.872010\du}{10.124514\du}}
\pgfpathlineto{\pgfpoint{34.871057\du}{10.130232\du}}
\pgfpathlineto{\pgfpoint{34.870104\du}{10.135951\du}}
\pgfpathlineto{\pgfpoint{34.869151\du}{10.141193\du}}
\pgfpathlineto{\pgfpoint{34.868674\du}{10.147389\du}}
\pgfpathlineto{\pgfpoint{34.868674\du}{10.153584\du}}
\pgfpathlineto{\pgfpoint{34.868674\du}{10.153584\du}}
\pgfpathlineto{\pgfpoint{34.868674\du}{10.159303\du}}
\pgfpathlineto{\pgfpoint{34.869151\du}{10.165498\du}}
\pgfpathlineto{\pgfpoint{34.870104\du}{10.170740\du}}
\pgfpathlineto{\pgfpoint{34.871057\du}{10.176936\du}}
\pgfpathlineto{\pgfpoint{34.872010\du}{10.182178\du}}
\pgfpathlineto{\pgfpoint{34.873916\du}{10.187897\du}}
\pgfpathlineto{\pgfpoint{34.876299\du}{10.193615\du}}
\pgfpathlineto{\pgfpoint{34.878205\du}{10.198858\du}}
\pgfpathlineto{\pgfpoint{34.880588\du}{10.204100\du}}
\pgfpathlineto{\pgfpoint{34.882971\du}{10.209819\du}}
\pgfpathlineto{\pgfpoint{34.885830\du}{10.214584\du}}
\pgfpathlineto{\pgfpoint{34.888690\du}{10.219350\du}}
\pgfpathlineto{\pgfpoint{34.892502\du}{10.223639\du}}
\pgfpathlineto{\pgfpoint{34.895838\du}{10.228881\du}}
\pgfpathlineto{\pgfpoint{34.900127\du}{10.233170\du}}
\pgfpathlineto{\pgfpoint{34.904416\du}{10.236983\du}}
\pgfpathlineto{\pgfpoint{34.909182\du}{10.240795\du}}
\pgfpathlineto{\pgfpoint{34.912518\du}{10.245084\du}}
\pgfpathlineto{\pgfpoint{34.917760\du}{10.247467\du}}
\pgfpathlineto{\pgfpoint{34.922049\du}{10.251756\du}}
\pgfpathlineto{\pgfpoint{34.927768\du}{10.254615\du}}
\pgfpathlineto{\pgfpoint{34.933010\du}{10.256998\du}}
\pgfpathlineto{\pgfpoint{34.937776\du}{10.259381\du}}
\pgfpathlineto{\pgfpoint{34.943018\du}{10.261764\du}}
\pgfpathlineto{\pgfpoint{34.949213\du}{10.263670\du}}
\pgfpathlineto{\pgfpoint{34.954456\du}{10.265100\du}}
\pgfpathlineto{\pgfpoint{34.959698\du}{10.266530\du}}
\pgfpathlineto{\pgfpoint{34.965893\du}{10.267483\du}}
\pgfpathlineto{\pgfpoint{34.971612\du}{10.268436\du}}
\pgfpathlineto{\pgfpoint{34.977331\du}{10.268912\du}}
\pgfpathlineto{\pgfpoint{34.983526\du}{10.268912\du}}
\pgfpathlineto{\pgfpoint{34.983526\du}{10.268912\du}}
\pgfpathlineto{\pgfpoint{34.989245\du}{10.268912\du}}
\pgfpathlineto{\pgfpoint{34.994487\du}{10.268436\du}}
\pgfpathlineto{\pgfpoint{35.000206\du}{10.267483\du}}
\pgfpathlineto{\pgfpoint{35.006401\du}{10.266530\du}}
\pgfpathlineto{\pgfpoint{35.011643\du}{10.265100\du}}
\pgfpathlineto{\pgfpoint{35.016885\du}{10.263670\du}}
\pgfpathlineto{\pgfpoint{35.023081\du}{10.261764\du}}
\pgfpathlineto{\pgfpoint{35.028323\du}{10.259381\du}}
\pgfpathlineto{\pgfpoint{35.034042\du}{10.256998\du}}
\pgfpathlineto{\pgfpoint{35.038331\du}{10.254615\du}}
\pgfpathlineto{\pgfpoint{35.044049\du}{10.251756\du}}
\pgfpathlineto{\pgfpoint{35.048815\du}{10.247467\du}}
\pgfpathlineto{\pgfpoint{35.053581\du}{10.245084\du}}
\pgfpathlineto{\pgfpoint{35.057870\du}{10.240795\du}}
\pgfpathlineto{\pgfpoint{35.061682\du}{10.236983\du}}
\pgfpathlineto{\pgfpoint{35.065971\du}{10.233170\du}}
\pgfpathlineto{\pgfpoint{35.070260\du}{10.228881\du}}
\pgfpathlineto{\pgfpoint{35.073596\du}{10.223639\du}}
\pgfpathlineto{\pgfpoint{35.077409\du}{10.219350\du}}
\pgfpathlineto{\pgfpoint{35.080268\du}{10.214584\du}}
\pgfpathlineto{\pgfpoint{35.083128\du}{10.209819\du}}
\pgfpathlineto{\pgfpoint{35.085511\du}{10.204100\du}}
\pgfpathlineto{\pgfpoint{35.087893\du}{10.198858\du}}
\pgfpathlineto{\pgfpoint{35.090753\du}{10.193615\du}}
\pgfpathlineto{\pgfpoint{35.092182\du}{10.187897\du}}
\pgfpathlineto{\pgfpoint{35.094089\du}{10.182178\du}}
\pgfpathlineto{\pgfpoint{35.095042\du}{10.176936\du}}
\pgfpathlineto{\pgfpoint{35.096471\du}{10.170740\du}}
\pgfpathlineto{\pgfpoint{35.096948\du}{10.165498\du}}
\pgfpathlineto{\pgfpoint{35.097425\du}{10.159303\du}}
\pgfpathlineto{\pgfpoint{35.097425\du}{10.153584\du}}
\pgfusepath{fill}
\pgfsetbuttcap
\pgfsetmiterjoin
\pgfsetdash{}{0pt}
\definecolor{dialinecolor}{rgb}{1.000000, 1.000000, 1.000000}
\pgfsetfillcolor{dialinecolor}
\pgfpathmoveto{\pgfpoint{36.138239\du}{10.153584\du}}
\pgfpathlineto{\pgfpoint{36.138239\du}{10.147389\du}}
\pgfpathlineto{\pgfpoint{36.137763\du}{10.141193\du}}
\pgfpathlineto{\pgfpoint{36.136810\du}{10.135951\du}}
\pgfpathlineto{\pgfpoint{36.136333\du}{10.130232\du}}
\pgfpathlineto{\pgfpoint{36.134427\du}{10.124514\du}}
\pgfpathlineto{\pgfpoint{36.133474\du}{10.118795\du}}
\pgfpathlineto{\pgfpoint{36.131091\du}{10.113076\du}}
\pgfpathlineto{\pgfpoint{36.129185\du}{10.107357\du}}
\pgfpathlineto{\pgfpoint{36.126325\du}{10.102592\du}}
\pgfpathlineto{\pgfpoint{36.124419\du}{10.096873\du}}
\pgfpathlineto{\pgfpoint{36.121560\du}{10.092107\du}}
\pgfpathlineto{\pgfpoint{36.118224\du}{10.087342\du}}
\pgfpathlineto{\pgfpoint{36.114411\du}{10.082576\du}}
\pgfpathlineto{\pgfpoint{36.110599\du}{10.077810\du}}
\pgfpathlineto{\pgfpoint{36.106786\du}{10.073998\du}}
\pgfpathlineto{\pgfpoint{36.103450\du}{10.069709\du}}
\pgfpathlineto{\pgfpoint{36.098208\du}{10.065896\du}}
\pgfpathlineto{\pgfpoint{36.094395\du}{10.061607\du}}
\pgfpathlineto{\pgfpoint{36.089153\du}{10.058748\du}}
\pgfpathlineto{\pgfpoint{36.084388\du}{10.055412\du}}
\pgfpathlineto{\pgfpoint{36.079622\du}{10.052076\du}}
\pgfpathlineto{\pgfpoint{36.074380\du}{10.049693\du}}
\pgfpathlineto{\pgfpoint{36.069138\du}{10.047310\du}}
\pgfpathlineto{\pgfpoint{36.063419\du}{10.044928\du}}
\pgfpathlineto{\pgfpoint{36.058653\du}{10.043021\du}}
\pgfpathlineto{\pgfpoint{36.052934\du}{10.041592\du}}
\pgfpathlineto{\pgfpoint{36.046739\du}{10.040162\du}}
\pgfpathlineto{\pgfpoint{36.041497\du}{10.039209\du}}
\pgfpathlineto{\pgfpoint{36.035778\du}{10.038256\du}}
\pgfpathlineto{\pgfpoint{36.030059\du}{10.037779\du}}
\pgfpathlineto{\pgfpoint{36.024341\du}{10.037779\du}}
\pgfpathlineto{\pgfpoint{36.024341\du}{10.037779\du}}
\pgfpathlineto{\pgfpoint{36.018145\du}{10.037779\du}}
\pgfpathlineto{\pgfpoint{36.012903\du}{10.038256\du}}
\pgfpathlineto{\pgfpoint{36.007184\du}{10.039209\du}}
\pgfpathlineto{\pgfpoint{36.001942\du}{10.040162\du}}
\pgfpathlineto{\pgfpoint{35.996223\du}{10.041592\du}}
\pgfpathlineto{\pgfpoint{35.990505\du}{10.043021\du}}
\pgfpathlineto{\pgfpoint{35.984786\du}{10.044928\du}}
\pgfpathlineto{\pgfpoint{35.979544\du}{10.047310\du}}
\pgfpathlineto{\pgfpoint{35.974301\du}{10.049693\du}}
\pgfpathlineto{\pgfpoint{35.969536\du}{10.052076\du}}
\pgfpathlineto{\pgfpoint{35.963817\du}{10.055412\du}}
\pgfpathlineto{\pgfpoint{35.959051\du}{10.058748\du}}
\pgfpathlineto{\pgfpoint{35.954762\du}{10.061607\du}}
\pgfpathlineto{\pgfpoint{35.950473\du}{10.065896\du}}
\pgfpathlineto{\pgfpoint{35.945708\du}{10.069709\du}}
\pgfpathlineto{\pgfpoint{35.942372\du}{10.073998\du}}
\pgfpathlineto{\pgfpoint{35.937606\du}{10.077810\du}}
\pgfpathlineto{\pgfpoint{35.934270\du}{10.082576\du}}
\pgfpathlineto{\pgfpoint{35.930934\du}{10.087342\du}}
\pgfpathlineto{\pgfpoint{35.927598\du}{10.092107\du}}
\pgfpathlineto{\pgfpoint{35.924739\du}{10.096873\du}}
\pgfpathlineto{\pgfpoint{35.922356\du}{10.102592\du}}
\pgfpathlineto{\pgfpoint{35.919497\du}{10.107357\du}}
\pgfpathlineto{\pgfpoint{35.917590\du}{10.113076\du}}
\pgfpathlineto{\pgfpoint{35.915684\du}{10.118795\du}}
\pgfpathlineto{\pgfpoint{35.913778\du}{10.124514\du}}
\pgfpathlineto{\pgfpoint{35.912825\du}{10.130232\du}}
\pgfpathlineto{\pgfpoint{35.911395\du}{10.135951\du}}
\pgfpathlineto{\pgfpoint{35.910919\du}{10.141193\du}}
\pgfpathlineto{\pgfpoint{35.910442\du}{10.147389\du}}
\pgfpathlineto{\pgfpoint{35.910442\du}{10.153584\du}}
\pgfpathlineto{\pgfpoint{35.910442\du}{10.153584\du}}
\pgfpathlineto{\pgfpoint{35.910442\du}{10.159303\du}}
\pgfpathlineto{\pgfpoint{35.910919\du}{10.165498\du}}
\pgfpathlineto{\pgfpoint{35.911395\du}{10.170740\du}}
\pgfpathlineto{\pgfpoint{35.912825\du}{10.176936\du}}
\pgfpathlineto{\pgfpoint{35.913778\du}{10.182178\du}}
\pgfpathlineto{\pgfpoint{35.915684\du}{10.187897\du}}
\pgfpathlineto{\pgfpoint{35.917590\du}{10.193615\du}}
\pgfpathlineto{\pgfpoint{35.919497\du}{10.198858\du}}
\pgfpathlineto{\pgfpoint{35.922356\du}{10.204100\du}}
\pgfpathlineto{\pgfpoint{35.924739\du}{10.209819\du}}
\pgfpathlineto{\pgfpoint{35.927598\du}{10.214584\du}}
\pgfpathlineto{\pgfpoint{35.930934\du}{10.219350\du}}
\pgfpathlineto{\pgfpoint{35.934270\du}{10.223639\du}}
\pgfpathlineto{\pgfpoint{35.937606\du}{10.228881\du}}
\pgfpathlineto{\pgfpoint{35.942372\du}{10.233170\du}}
\pgfpathlineto{\pgfpoint{35.945708\du}{10.236983\du}}
\pgfpathlineto{\pgfpoint{35.950473\du}{10.240795\du}}
\pgfpathlineto{\pgfpoint{35.954762\du}{10.245084\du}}
\pgfpathlineto{\pgfpoint{35.959051\du}{10.247467\du}}
\pgfpathlineto{\pgfpoint{35.963817\du}{10.251756\du}}
\pgfpathlineto{\pgfpoint{35.969536\du}{10.254615\du}}
\pgfpathlineto{\pgfpoint{35.974301\du}{10.256998\du}}
\pgfpathlineto{\pgfpoint{35.979544\du}{10.259381\du}}
\pgfpathlineto{\pgfpoint{35.984786\du}{10.261764\du}}
\pgfpathlineto{\pgfpoint{35.990505\du}{10.263670\du}}
\pgfpathlineto{\pgfpoint{35.996223\du}{10.265100\du}}
\pgfpathlineto{\pgfpoint{36.001942\du}{10.266530\du}}
\pgfpathlineto{\pgfpoint{36.007184\du}{10.267483\du}}
\pgfpathlineto{\pgfpoint{36.012903\du}{10.268436\du}}
\pgfpathlineto{\pgfpoint{36.018145\du}{10.268912\du}}
\pgfpathlineto{\pgfpoint{36.024341\du}{10.268912\du}}
\pgfpathlineto{\pgfpoint{36.024341\du}{10.268912\du}}
\pgfpathlineto{\pgfpoint{36.030059\du}{10.268912\du}}
\pgfpathlineto{\pgfpoint{36.035778\du}{10.268436\du}}
\pgfpathlineto{\pgfpoint{36.041497\du}{10.267483\du}}
\pgfpathlineto{\pgfpoint{36.046739\du}{10.266530\du}}
\pgfpathlineto{\pgfpoint{36.052934\du}{10.265100\du}}
\pgfpathlineto{\pgfpoint{36.058653\du}{10.263670\du}}
\pgfpathlineto{\pgfpoint{36.063419\du}{10.261764\du}}
\pgfpathlineto{\pgfpoint{36.069138\du}{10.259381\du}}
\pgfpathlineto{\pgfpoint{36.074380\du}{10.256998\du}}
\pgfpathlineto{\pgfpoint{36.079622\du}{10.254615\du}}
\pgfpathlineto{\pgfpoint{36.084388\du}{10.251756\du}}
\pgfpathlineto{\pgfpoint{36.089153\du}{10.247467\du}}
\pgfpathlineto{\pgfpoint{36.094395\du}{10.245084\du}}
\pgfpathlineto{\pgfpoint{36.098208\du}{10.240795\du}}
\pgfpathlineto{\pgfpoint{36.103450\du}{10.236983\du}}
\pgfpathlineto{\pgfpoint{36.106786\du}{10.233170\du}}
\pgfpathlineto{\pgfpoint{36.110599\du}{10.228881\du}}
\pgfpathlineto{\pgfpoint{36.114411\du}{10.223639\du}}
\pgfpathlineto{\pgfpoint{36.118224\du}{10.219350\du}}
\pgfpathlineto{\pgfpoint{36.121560\du}{10.214584\du}}
\pgfpathlineto{\pgfpoint{36.124419\du}{10.209819\du}}
\pgfpathlineto{\pgfpoint{36.126325\du}{10.204100\du}}
\pgfpathlineto{\pgfpoint{36.129185\du}{10.198858\du}}
\pgfpathlineto{\pgfpoint{36.131091\du}{10.193615\du}}
\pgfpathlineto{\pgfpoint{36.133474\du}{10.187897\du}}
\pgfpathlineto{\pgfpoint{36.134427\du}{10.182178\du}}
\pgfpathlineto{\pgfpoint{36.136333\du}{10.176936\du}}
\pgfpathlineto{\pgfpoint{36.136810\du}{10.170740\du}}
\pgfpathlineto{\pgfpoint{36.137763\du}{10.165498\du}}
\pgfpathlineto{\pgfpoint{36.138239\du}{10.159303\du}}
\pgfpathlineto{\pgfpoint{36.138239\du}{10.153584\du}}
\pgfusepath{fill}
\pgfsetlinewidth{0.000000\du}
\pgfsetbuttcap
\pgfsetmiterjoin
\pgfsetdash{}{0pt}
\definecolor{dialinecolor}{rgb}{0.000000, 0.000000, 0.000000}
\pgfsetstrokecolor{dialinecolor}
\pgfpathmoveto{\pgfpoint{34.994010\du}{10.142623\du}}
\pgfpathlineto{\pgfpoint{35.991934\du}{10.142623\du}}
\pgfusepath{stroke}
\pgfsetlinewidth{0.000000\du}
\pgfsetbuttcap
\pgfsetmiterjoin
\pgfsetdash{}{0pt}
\definecolor{dialinecolor}{rgb}{1.000000, 1.000000, 1.000000}
\pgfsetfillcolor{dialinecolor}
\pgfpathmoveto{\pgfpoint{35.097425\du}{10.574866\du}}
\pgfpathlineto{\pgfpoint{35.097425\du}{10.568671\du}}
\pgfpathlineto{\pgfpoint{35.096948\du}{10.562952\du}}
\pgfpathlineto{\pgfpoint{35.096471\du}{10.557233\du}}
\pgfpathlineto{\pgfpoint{35.095042\du}{10.551515\du}}
\pgfpathlineto{\pgfpoint{35.094089\du}{10.545319\du}}
\pgfpathlineto{\pgfpoint{35.092182\du}{10.540077\du}}
\pgfpathlineto{\pgfpoint{35.090753\du}{10.534835\du}}
\pgfpathlineto{\pgfpoint{35.087893\du}{10.528640\du}}
\pgfpathlineto{\pgfpoint{35.085511\du}{10.523874\du}}
\pgfpathlineto{\pgfpoint{35.083128\du}{10.518632\du}}
\pgfpathlineto{\pgfpoint{35.080268\du}{10.512436\du}}
\pgfpathlineto{\pgfpoint{35.077409\du}{10.508147\du}}
\pgfpathlineto{\pgfpoint{35.073596\du}{10.503382\du}}
\pgfpathlineto{\pgfpoint{35.070260\du}{10.499093\du}}
\pgfpathlineto{\pgfpoint{35.065971\du}{10.494327\du}}
\pgfpathlineto{\pgfpoint{35.061682\du}{10.490514\du}}
\pgfpathlineto{\pgfpoint{35.057870\du}{10.486702\du}}
\pgfpathlineto{\pgfpoint{35.053581\du}{10.483366\du}}
\pgfpathlineto{\pgfpoint{35.048815\du}{10.479553\du}}
\pgfpathlineto{\pgfpoint{35.044049\du}{10.476218\du}}
\pgfpathlineto{\pgfpoint{35.038331\du}{10.472882\du}}
\pgfpathlineto{\pgfpoint{35.034042\du}{10.470499\du}}
\pgfpathlineto{\pgfpoint{35.028323\du}{10.468116\du}}
\pgfpathlineto{\pgfpoint{35.023081\du}{10.465733\du}}
\pgfpathlineto{\pgfpoint{35.016885\du}{10.463827\du}}
\pgfpathlineto{\pgfpoint{35.011643\du}{10.461921\du}}
\pgfpathlineto{\pgfpoint{35.006401\du}{10.460967\du}}
\pgfpathlineto{\pgfpoint{35.000206\du}{10.460014\du}}
\pgfpathlineto{\pgfpoint{34.994487\du}{10.459061\du}}
\pgfpathlineto{\pgfpoint{34.989245\du}{10.458585\du}}
\pgfpathlineto{\pgfpoint{34.983526\du}{10.458585\du}}
\pgfpathlineto{\pgfpoint{34.983526\du}{10.458585\du}}
\pgfpathlineto{\pgfpoint{34.977331\du}{10.458585\du}}
\pgfpathlineto{\pgfpoint{34.971612\du}{10.459061\du}}
\pgfpathlineto{\pgfpoint{34.965893\du}{10.460014\du}}
\pgfpathlineto{\pgfpoint{34.959698\du}{10.460967\du}}
\pgfpathlineto{\pgfpoint{34.954456\du}{10.461921\du}}
\pgfpathlineto{\pgfpoint{34.949213\du}{10.463827\du}}
\pgfpathlineto{\pgfpoint{34.943018\du}{10.465733\du}}
\pgfpathlineto{\pgfpoint{34.937776\du}{10.468116\du}}
\pgfpathlineto{\pgfpoint{34.933010\du}{10.470499\du}}
\pgfpathlineto{\pgfpoint{34.927768\du}{10.472882\du}}
\pgfpathlineto{\pgfpoint{34.922049\du}{10.476218\du}}
\pgfpathlineto{\pgfpoint{34.917760\du}{10.479553\du}}
\pgfpathlineto{\pgfpoint{34.912518\du}{10.483366\du}}
\pgfpathlineto{\pgfpoint{34.909182\du}{10.486702\du}}
\pgfpathlineto{\pgfpoint{34.904416\du}{10.490514\du}}
\pgfpathlineto{\pgfpoint{34.900127\du}{10.494327\du}}
\pgfpathlineto{\pgfpoint{34.895838\du}{10.499093\du}}
\pgfpathlineto{\pgfpoint{34.892502\du}{10.503382\du}}
\pgfpathlineto{\pgfpoint{34.888690\du}{10.508147\du}}
\pgfpathlineto{\pgfpoint{34.885830\du}{10.512436\du}}
\pgfpathlineto{\pgfpoint{34.882971\du}{10.518632\du}}
\pgfpathlineto{\pgfpoint{34.880588\du}{10.523874\du}}
\pgfpathlineto{\pgfpoint{34.878205\du}{10.528640\du}}
\pgfpathlineto{\pgfpoint{34.876299\du}{10.534835\du}}
\pgfpathlineto{\pgfpoint{34.873916\du}{10.540077\du}}
\pgfpathlineto{\pgfpoint{34.872010\du}{10.545319\du}}
\pgfpathlineto{\pgfpoint{34.871057\du}{10.551515\du}}
\pgfpathlineto{\pgfpoint{34.870104\du}{10.557233\du}}
\pgfpathlineto{\pgfpoint{34.869151\du}{10.562952\du}}
\pgfpathlineto{\pgfpoint{34.868674\du}{10.568671\du}}
\pgfpathlineto{\pgfpoint{34.868674\du}{10.574866\du}}
\pgfpathlineto{\pgfpoint{34.868674\du}{10.574866\du}}
\pgfpathlineto{\pgfpoint{34.868674\du}{10.580585\du}}
\pgfpathlineto{\pgfpoint{34.869151\du}{10.586780\du}}
\pgfpathlineto{\pgfpoint{34.870104\du}{10.592022\du}}
\pgfpathlineto{\pgfpoint{34.871057\du}{10.598218\du}}
\pgfpathlineto{\pgfpoint{34.872010\du}{10.603937\du}}
\pgfpathlineto{\pgfpoint{34.873916\du}{10.609655\du}}
\pgfpathlineto{\pgfpoint{34.876299\du}{10.614898\du}}
\pgfpathlineto{\pgfpoint{34.878205\du}{10.620616\du}}
\pgfpathlineto{\pgfpoint{34.880588\du}{10.625858\du}}
\pgfpathlineto{\pgfpoint{34.882971\du}{10.631101\du}}
\pgfpathlineto{\pgfpoint{34.885830\du}{10.636343\du}}
\pgfpathlineto{\pgfpoint{34.888690\du}{10.640632\du}}
\pgfpathlineto{\pgfpoint{34.892502\du}{10.645874\du}}
\pgfpathlineto{\pgfpoint{34.895838\du}{10.650163\du}}
\pgfpathlineto{\pgfpoint{34.900127\du}{10.654929\du}}
\pgfpathlineto{\pgfpoint{34.904416\du}{10.659218\du}}
\pgfpathlineto{\pgfpoint{34.909182\du}{10.662554\du}}
\pgfpathlineto{\pgfpoint{34.912518\du}{10.666366\du}}
\pgfpathlineto{\pgfpoint{34.917760\du}{10.669702\du}}
\pgfpathlineto{\pgfpoint{34.922049\du}{10.673515\du}}
\pgfpathlineto{\pgfpoint{34.927768\du}{10.676374\du}}
\pgfpathlineto{\pgfpoint{34.933010\du}{10.678757\du}}
\pgfpathlineto{\pgfpoint{34.937776\du}{10.681140\du}}
\pgfpathlineto{\pgfpoint{34.943018\du}{10.683523\du}}
\pgfpathlineto{\pgfpoint{34.949213\du}{10.685429\du}}
\pgfpathlineto{\pgfpoint{34.954456\du}{10.687335\du}}
\pgfpathlineto{\pgfpoint{34.959698\du}{10.688288\du}}
\pgfpathlineto{\pgfpoint{34.965893\du}{10.689718\du}}
\pgfpathlineto{\pgfpoint{34.971612\du}{10.690195\du}}
\pgfpathlineto{\pgfpoint{34.977331\du}{10.690671\du}}
\pgfpathlineto{\pgfpoint{34.983526\du}{10.690671\du}}
\pgfpathlineto{\pgfpoint{34.983526\du}{10.690671\du}}
\pgfpathlineto{\pgfpoint{34.989245\du}{10.690671\du}}
\pgfpathlineto{\pgfpoint{34.994487\du}{10.690195\du}}
\pgfpathlineto{\pgfpoint{35.000206\du}{10.689718\du}}
\pgfpathlineto{\pgfpoint{35.006401\du}{10.688288\du}}
\pgfpathlineto{\pgfpoint{35.011643\du}{10.687335\du}}
\pgfpathlineto{\pgfpoint{35.016885\du}{10.685429\du}}
\pgfpathlineto{\pgfpoint{35.023081\du}{10.683523\du}}
\pgfpathlineto{\pgfpoint{35.028323\du}{10.681140\du}}
\pgfpathlineto{\pgfpoint{35.034042\du}{10.678757\du}}
\pgfpathlineto{\pgfpoint{35.038331\du}{10.676374\du}}
\pgfpathlineto{\pgfpoint{35.044049\du}{10.673515\du}}
\pgfpathlineto{\pgfpoint{35.048815\du}{10.669702\du}}
\pgfpathlineto{\pgfpoint{35.053581\du}{10.666366\du}}
\pgfpathlineto{\pgfpoint{35.057870\du}{10.662554\du}}
\pgfpathlineto{\pgfpoint{35.061682\du}{10.659218\du}}
\pgfpathlineto{\pgfpoint{35.065971\du}{10.654929\du}}
\pgfpathlineto{\pgfpoint{35.070260\du}{10.650163\du}}
\pgfpathlineto{\pgfpoint{35.073596\du}{10.645874\du}}
\pgfpathlineto{\pgfpoint{35.077409\du}{10.640632\du}}
\pgfpathlineto{\pgfpoint{35.080268\du}{10.636343\du}}
\pgfpathlineto{\pgfpoint{35.083128\du}{10.631101\du}}
\pgfpathlineto{\pgfpoint{35.085511\du}{10.625858\du}}
\pgfpathlineto{\pgfpoint{35.087893\du}{10.620616\du}}
\pgfpathlineto{\pgfpoint{35.090753\du}{10.614898\du}}
\pgfpathlineto{\pgfpoint{35.092182\du}{10.609655\du}}
\pgfpathlineto{\pgfpoint{35.094089\du}{10.603937\du}}
\pgfpathlineto{\pgfpoint{35.095042\du}{10.598218\du}}
\pgfpathlineto{\pgfpoint{35.096471\du}{10.592022\du}}
\pgfpathlineto{\pgfpoint{35.096948\du}{10.586780\du}}
\pgfpathlineto{\pgfpoint{35.097425\du}{10.580585\du}}
\pgfpathlineto{\pgfpoint{35.097425\du}{10.574866\du}}
\pgfusepath{fill}
\pgfsetbuttcap
\pgfsetmiterjoin
\pgfsetdash{}{0pt}
\definecolor{dialinecolor}{rgb}{1.000000, 1.000000, 1.000000}
\pgfsetfillcolor{dialinecolor}
\pgfpathmoveto{\pgfpoint{36.138239\du}{10.574866\du}}
\pgfpathlineto{\pgfpoint{36.138239\du}{10.568671\du}}
\pgfpathlineto{\pgfpoint{36.137763\du}{10.562952\du}}
\pgfpathlineto{\pgfpoint{36.136810\du}{10.557233\du}}
\pgfpathlineto{\pgfpoint{36.136333\du}{10.551515\du}}
\pgfpathlineto{\pgfpoint{36.134427\du}{10.545319\du}}
\pgfpathlineto{\pgfpoint{36.133474\du}{10.540077\du}}
\pgfpathlineto{\pgfpoint{36.131091\du}{10.534835\du}}
\pgfpathlineto{\pgfpoint{36.129185\du}{10.528640\du}}
\pgfpathlineto{\pgfpoint{36.126325\du}{10.523874\du}}
\pgfpathlineto{\pgfpoint{36.124419\du}{10.518632\du}}
\pgfpathlineto{\pgfpoint{36.121560\du}{10.512436\du}}
\pgfpathlineto{\pgfpoint{36.118224\du}{10.508147\du}}
\pgfpathlineto{\pgfpoint{36.114411\du}{10.503382\du}}
\pgfpathlineto{\pgfpoint{36.110599\du}{10.499093\du}}
\pgfpathlineto{\pgfpoint{36.106786\du}{10.494327\du}}
\pgfpathlineto{\pgfpoint{36.103450\du}{10.490514\du}}
\pgfpathlineto{\pgfpoint{36.098208\du}{10.486702\du}}
\pgfpathlineto{\pgfpoint{36.094395\du}{10.483366\du}}
\pgfpathlineto{\pgfpoint{36.089153\du}{10.479553\du}}
\pgfpathlineto{\pgfpoint{36.084388\du}{10.476218\du}}
\pgfpathlineto{\pgfpoint{36.079622\du}{10.472882\du}}
\pgfpathlineto{\pgfpoint{36.074380\du}{10.470499\du}}
\pgfpathlineto{\pgfpoint{36.069138\du}{10.468116\du}}
\pgfpathlineto{\pgfpoint{36.063419\du}{10.465733\du}}
\pgfpathlineto{\pgfpoint{36.058653\du}{10.463827\du}}
\pgfpathlineto{\pgfpoint{36.052934\du}{10.461921\du}}
\pgfpathlineto{\pgfpoint{36.046739\du}{10.460967\du}}
\pgfpathlineto{\pgfpoint{36.041497\du}{10.460014\du}}
\pgfpathlineto{\pgfpoint{36.035778\du}{10.459061\du}}
\pgfpathlineto{\pgfpoint{36.030059\du}{10.458585\du}}
\pgfpathlineto{\pgfpoint{36.024341\du}{10.458585\du}}
\pgfpathlineto{\pgfpoint{36.024341\du}{10.458585\du}}
\pgfpathlineto{\pgfpoint{36.018145\du}{10.458585\du}}
\pgfpathlineto{\pgfpoint{36.012903\du}{10.459061\du}}
\pgfpathlineto{\pgfpoint{36.007184\du}{10.460014\du}}
\pgfpathlineto{\pgfpoint{36.001942\du}{10.460967\du}}
\pgfpathlineto{\pgfpoint{35.996223\du}{10.461921\du}}
\pgfpathlineto{\pgfpoint{35.990505\du}{10.463827\du}}
\pgfpathlineto{\pgfpoint{35.984786\du}{10.465733\du}}
\pgfpathlineto{\pgfpoint{35.979544\du}{10.468116\du}}
\pgfpathlineto{\pgfpoint{35.974301\du}{10.470499\du}}
\pgfpathlineto{\pgfpoint{35.969536\du}{10.472882\du}}
\pgfpathlineto{\pgfpoint{35.963817\du}{10.476218\du}}
\pgfpathlineto{\pgfpoint{35.959051\du}{10.479553\du}}
\pgfpathlineto{\pgfpoint{35.954762\du}{10.483366\du}}
\pgfpathlineto{\pgfpoint{35.950473\du}{10.486702\du}}
\pgfpathlineto{\pgfpoint{35.945708\du}{10.490514\du}}
\pgfpathlineto{\pgfpoint{35.942372\du}{10.494327\du}}
\pgfpathlineto{\pgfpoint{35.937606\du}{10.499093\du}}
\pgfpathlineto{\pgfpoint{35.934270\du}{10.503382\du}}
\pgfpathlineto{\pgfpoint{35.930934\du}{10.508147\du}}
\pgfpathlineto{\pgfpoint{35.927598\du}{10.512436\du}}
\pgfpathlineto{\pgfpoint{35.924739\du}{10.518632\du}}
\pgfpathlineto{\pgfpoint{35.922356\du}{10.523874\du}}
\pgfpathlineto{\pgfpoint{35.919497\du}{10.528640\du}}
\pgfpathlineto{\pgfpoint{35.917590\du}{10.534835\du}}
\pgfpathlineto{\pgfpoint{35.915684\du}{10.540077\du}}
\pgfpathlineto{\pgfpoint{35.913778\du}{10.545319\du}}
\pgfpathlineto{\pgfpoint{35.912825\du}{10.551515\du}}
\pgfpathlineto{\pgfpoint{35.911395\du}{10.557233\du}}
\pgfpathlineto{\pgfpoint{35.910919\du}{10.562952\du}}
\pgfpathlineto{\pgfpoint{35.910442\du}{10.568671\du}}
\pgfpathlineto{\pgfpoint{35.910442\du}{10.574866\du}}
\pgfpathlineto{\pgfpoint{35.910442\du}{10.574866\du}}
\pgfpathlineto{\pgfpoint{35.910442\du}{10.580585\du}}
\pgfpathlineto{\pgfpoint{35.910919\du}{10.586780\du}}
\pgfpathlineto{\pgfpoint{35.911395\du}{10.592022\du}}
\pgfpathlineto{\pgfpoint{35.912825\du}{10.598218\du}}
\pgfpathlineto{\pgfpoint{35.913778\du}{10.603937\du}}
\pgfpathlineto{\pgfpoint{35.915684\du}{10.609655\du}}
\pgfpathlineto{\pgfpoint{35.917590\du}{10.614898\du}}
\pgfpathlineto{\pgfpoint{35.919497\du}{10.620616\du}}
\pgfpathlineto{\pgfpoint{35.922356\du}{10.625858\du}}
\pgfpathlineto{\pgfpoint{35.924739\du}{10.631101\du}}
\pgfpathlineto{\pgfpoint{35.927598\du}{10.636343\du}}
\pgfpathlineto{\pgfpoint{35.930934\du}{10.640632\du}}
\pgfpathlineto{\pgfpoint{35.934270\du}{10.645874\du}}
\pgfpathlineto{\pgfpoint{35.937606\du}{10.650163\du}}
\pgfpathlineto{\pgfpoint{35.942372\du}{10.654929\du}}
\pgfpathlineto{\pgfpoint{35.945708\du}{10.659218\du}}
\pgfpathlineto{\pgfpoint{35.950473\du}{10.662554\du}}
\pgfpathlineto{\pgfpoint{35.954762\du}{10.666366\du}}
\pgfpathlineto{\pgfpoint{35.959051\du}{10.669702\du}}
\pgfpathlineto{\pgfpoint{35.963817\du}{10.673515\du}}
\pgfpathlineto{\pgfpoint{35.969536\du}{10.676374\du}}
\pgfpathlineto{\pgfpoint{35.974301\du}{10.678757\du}}
\pgfpathlineto{\pgfpoint{35.979544\du}{10.681140\du}}
\pgfpathlineto{\pgfpoint{35.984786\du}{10.683523\du}}
\pgfpathlineto{\pgfpoint{35.990505\du}{10.685429\du}}
\pgfpathlineto{\pgfpoint{35.996223\du}{10.687335\du}}
\pgfpathlineto{\pgfpoint{36.001942\du}{10.688288\du}}
\pgfpathlineto{\pgfpoint{36.007184\du}{10.689718\du}}
\pgfpathlineto{\pgfpoint{36.012903\du}{10.690195\du}}
\pgfpathlineto{\pgfpoint{36.018145\du}{10.690671\du}}
\pgfpathlineto{\pgfpoint{36.024341\du}{10.690671\du}}
\pgfpathlineto{\pgfpoint{36.024341\du}{10.690671\du}}
\pgfpathlineto{\pgfpoint{36.030059\du}{10.690671\du}}
\pgfpathlineto{\pgfpoint{36.035778\du}{10.690195\du}}
\pgfpathlineto{\pgfpoint{36.041497\du}{10.689718\du}}
\pgfpathlineto{\pgfpoint{36.046739\du}{10.688288\du}}
\pgfpathlineto{\pgfpoint{36.052934\du}{10.687335\du}}
\pgfpathlineto{\pgfpoint{36.058653\du}{10.685429\du}}
\pgfpathlineto{\pgfpoint{36.063419\du}{10.683523\du}}
\pgfpathlineto{\pgfpoint{36.069138\du}{10.681140\du}}
\pgfpathlineto{\pgfpoint{36.074380\du}{10.678757\du}}
\pgfpathlineto{\pgfpoint{36.079622\du}{10.676374\du}}
\pgfpathlineto{\pgfpoint{36.084388\du}{10.673515\du}}
\pgfpathlineto{\pgfpoint{36.089153\du}{10.669702\du}}
\pgfpathlineto{\pgfpoint{36.094395\du}{10.666366\du}}
\pgfpathlineto{\pgfpoint{36.098208\du}{10.662554\du}}
\pgfpathlineto{\pgfpoint{36.103450\du}{10.659218\du}}
\pgfpathlineto{\pgfpoint{36.106786\du}{10.654929\du}}
\pgfpathlineto{\pgfpoint{36.110599\du}{10.650163\du}}
\pgfpathlineto{\pgfpoint{36.114411\du}{10.645874\du}}
\pgfpathlineto{\pgfpoint{36.118224\du}{10.640632\du}}
\pgfpathlineto{\pgfpoint{36.121560\du}{10.636343\du}}
\pgfpathlineto{\pgfpoint{36.124419\du}{10.631101\du}}
\pgfpathlineto{\pgfpoint{36.126325\du}{10.625858\du}}
\pgfpathlineto{\pgfpoint{36.129185\du}{10.620616\du}}
\pgfpathlineto{\pgfpoint{36.131091\du}{10.614898\du}}
\pgfpathlineto{\pgfpoint{36.133474\du}{10.609655\du}}
\pgfpathlineto{\pgfpoint{36.134427\du}{10.603937\du}}
\pgfpathlineto{\pgfpoint{36.136333\du}{10.598218\du}}
\pgfpathlineto{\pgfpoint{36.136810\du}{10.592022\du}}
\pgfpathlineto{\pgfpoint{36.137763\du}{10.586780\du}}
\pgfpathlineto{\pgfpoint{36.138239\du}{10.580585\du}}
\pgfpathlineto{\pgfpoint{36.138239\du}{10.574866\du}}
\pgfusepath{fill}
\pgfsetlinewidth{0.000000\du}
\pgfsetbuttcap
\pgfsetmiterjoin
\pgfsetdash{}{0pt}
\definecolor{dialinecolor}{rgb}{0.000000, 0.000000, 0.000000}
\pgfsetstrokecolor{dialinecolor}
\pgfpathmoveto{\pgfpoint{34.994010\du}{10.564382\du}}
\pgfpathlineto{\pgfpoint{35.991934\du}{10.564382\du}}
\pgfusepath{stroke}
\pgfsetlinewidth{0.000000\du}
\pgfsetbuttcap
\pgfsetmiterjoin
\pgfsetdash{}{0pt}
\definecolor{dialinecolor}{rgb}{1.000000, 1.000000, 1.000000}
\pgfsetfillcolor{dialinecolor}
\pgfpathmoveto{\pgfpoint{35.202745\du}{8.075291\du}}
\pgfpathlineto{\pgfpoint{35.202745\du}{8.560432\du}}
\pgfpathlineto{\pgfpoint{35.910442\du}{8.560432\du}}
\pgfpathlineto{\pgfpoint{35.910442\du}{8.075291\du}}
\pgfpathlineto{\pgfpoint{35.202745\du}{8.075291\du}}
\pgfusepath{fill}
\pgfsetbuttcap
\pgfsetmiterjoin
\pgfsetdash{}{0pt}
\definecolor{dialinecolor}{rgb}{1.000000, 1.000000, 1.000000}
\pgfsetfillcolor{dialinecolor}
\pgfpathmoveto{\pgfpoint{35.118393\du}{8.908324\du}}
\pgfpathlineto{\pgfpoint{35.118393\du}{8.902605\du}}
\pgfpathlineto{\pgfpoint{35.117917\du}{8.896409\du}}
\pgfpathlineto{\pgfpoint{35.117440\du}{8.891167\du}}
\pgfpathlineto{\pgfpoint{35.116011\du}{8.884972\du}}
\pgfpathlineto{\pgfpoint{35.115057\du}{8.879730\du}}
\pgfpathlineto{\pgfpoint{35.113151\du}{8.873534\du}}
\pgfpathlineto{\pgfpoint{35.111722\du}{8.868292\du}}
\pgfpathlineto{\pgfpoint{35.109339\du}{8.863050\du}}
\pgfpathlineto{\pgfpoint{35.106479\du}{8.857331\du}}
\pgfpathlineto{\pgfpoint{35.104096\du}{8.852089\du}}
\pgfpathlineto{\pgfpoint{35.101237\du}{8.847323\du}}
\pgfpathlineto{\pgfpoint{35.098378\du}{8.842558\du}}
\pgfpathlineto{\pgfpoint{35.094565\du}{8.837792\du}}
\pgfpathlineto{\pgfpoint{35.091229\du}{8.832550\du}}
\pgfpathlineto{\pgfpoint{35.086940\du}{8.828737\du}}
\pgfpathlineto{\pgfpoint{35.083128\du}{8.824448\du}}
\pgfpathlineto{\pgfpoint{35.078839\du}{8.820636\du}}
\pgfpathlineto{\pgfpoint{35.074073\du}{8.816823\du}}
\pgfpathlineto{\pgfpoint{35.069784\du}{8.813964\du}}
\pgfpathlineto{\pgfpoint{35.064542\du}{8.810151\du}}
\pgfpathlineto{\pgfpoint{35.059776\du}{8.807292\du}}
\pgfpathlineto{\pgfpoint{35.055010\du}{8.804909\du}}
\pgfpathlineto{\pgfpoint{35.049292\du}{8.802526\du}}
\pgfpathlineto{\pgfpoint{35.044049\du}{8.799667\du}}
\pgfpathlineto{\pgfpoint{35.038331\du}{8.798237\du}}
\pgfpathlineto{\pgfpoint{35.033089\du}{8.796331\du}}
\pgfpathlineto{\pgfpoint{35.027370\du}{8.795378\du}}
\pgfpathlineto{\pgfpoint{35.022128\du}{8.793948\du}}
\pgfpathlineto{\pgfpoint{35.016409\du}{8.793472\du}}
\pgfpathlineto{\pgfpoint{35.010690\du}{8.792995\du}}
\pgfpathlineto{\pgfpoint{35.004495\du}{8.792995\du}}
\pgfpathlineto{\pgfpoint{35.004495\du}{8.792995\du}}
\pgfpathlineto{\pgfpoint{34.998776\du}{8.792995\du}}
\pgfpathlineto{\pgfpoint{34.993057\du}{8.793472\du}}
\pgfpathlineto{\pgfpoint{34.987338\du}{8.793948\du}}
\pgfpathlineto{\pgfpoint{34.982096\du}{8.795378\du}}
\pgfpathlineto{\pgfpoint{34.975901\du}{8.796331\du}}
\pgfpathlineto{\pgfpoint{34.970659\du}{8.798237\du}}
\pgfpathlineto{\pgfpoint{34.965417\du}{8.799667\du}}
\pgfpathlineto{\pgfpoint{34.959698\du}{8.802526\du}}
\pgfpathlineto{\pgfpoint{34.954456\du}{8.804909\du}}
\pgfpathlineto{\pgfpoint{34.949213\du}{8.807292\du}}
\pgfpathlineto{\pgfpoint{34.944924\du}{8.810151\du}}
\pgfpathlineto{\pgfpoint{34.939682\du}{8.813964\du}}
\pgfpathlineto{\pgfpoint{34.934916\du}{8.816823\du}}
\pgfpathlineto{\pgfpoint{34.930627\du}{8.820636\du}}
\pgfpathlineto{\pgfpoint{34.925862\du}{8.824448\du}}
\pgfpathlineto{\pgfpoint{34.922049\du}{8.828737\du}}
\pgfpathlineto{\pgfpoint{34.918237\du}{8.832550\du}}
\pgfpathlineto{\pgfpoint{34.914901\du}{8.837792\du}}
\pgfpathlineto{\pgfpoint{34.911088\du}{8.842558\du}}
\pgfpathlineto{\pgfpoint{34.907752\du}{8.847323\du}}
\pgfpathlineto{\pgfpoint{34.904893\du}{8.852089\du}}
\pgfpathlineto{\pgfpoint{34.902987\du}{8.857331\du}}
\pgfpathlineto{\pgfpoint{34.900127\du}{8.863050\du}}
\pgfpathlineto{\pgfpoint{34.897744\du}{8.868292\du}}
\pgfpathlineto{\pgfpoint{34.895838\du}{8.873534\du}}
\pgfpathlineto{\pgfpoint{34.894409\du}{8.879730\du}}
\pgfpathlineto{\pgfpoint{34.892979\du}{8.884972\du}}
\pgfpathlineto{\pgfpoint{34.892026\du}{8.891167\du}}
\pgfpathlineto{\pgfpoint{34.891073\du}{8.896409\du}}
\pgfpathlineto{\pgfpoint{34.891073\du}{8.902605\du}}
\pgfpathlineto{\pgfpoint{34.891073\du}{8.908324\du}}
\pgfpathlineto{\pgfpoint{34.891073\du}{8.908324\du}}
\pgfpathlineto{\pgfpoint{34.891073\du}{8.914519\du}}
\pgfpathlineto{\pgfpoint{34.891073\du}{8.920238\du}}
\pgfpathlineto{\pgfpoint{34.892026\du}{8.925956\du}}
\pgfpathlineto{\pgfpoint{34.892979\du}{8.931675\du}}
\pgfpathlineto{\pgfpoint{34.894409\du}{8.936917\du}}
\pgfpathlineto{\pgfpoint{34.895838\du}{8.942636\du}}
\pgfpathlineto{\pgfpoint{34.897744\du}{8.948355\du}}
\pgfpathlineto{\pgfpoint{34.900127\du}{8.954074\du}}
\pgfpathlineto{\pgfpoint{34.902987\du}{8.959316\du}}
\pgfpathlineto{\pgfpoint{34.904893\du}{8.964558\du}}
\pgfpathlineto{\pgfpoint{34.907752\du}{8.969324\du}}
\pgfpathlineto{\pgfpoint{34.911088\du}{8.974089\du}}
\pgfpathlineto{\pgfpoint{34.914901\du}{8.978378\du}}
\pgfpathlineto{\pgfpoint{34.918237\du}{8.983621\du}}
\pgfpathlineto{\pgfpoint{34.922049\du}{8.987433\du}}
\pgfpathlineto{\pgfpoint{34.925862\du}{8.992199\du}}
\pgfpathlineto{\pgfpoint{34.930627\du}{8.995535\du}}
\pgfpathlineto{\pgfpoint{34.934916\du}{8.999347\du}}
\pgfpathlineto{\pgfpoint{34.939682\du}{9.003160\du}}
\pgfpathlineto{\pgfpoint{34.944924\du}{9.006496\du}}
\pgfpathlineto{\pgfpoint{34.949213\du}{9.009355\du}}
\pgfpathlineto{\pgfpoint{34.954456\du}{9.011261\du}}
\pgfpathlineto{\pgfpoint{34.959698\du}{9.014121\du}}
\pgfpathlineto{\pgfpoint{34.965417\du}{9.016503\du}}
\pgfpathlineto{\pgfpoint{34.970659\du}{9.018410\du}}
\pgfpathlineto{\pgfpoint{34.975901\du}{9.020316\du}}
\pgfpathlineto{\pgfpoint{34.982096\du}{9.021746\du}}
\pgfpathlineto{\pgfpoint{34.987338\du}{9.022699\du}}
\pgfpathlineto{\pgfpoint{34.993057\du}{9.023175\du}}
\pgfpathlineto{\pgfpoint{34.998776\du}{9.024128\du}}
\pgfpathlineto{\pgfpoint{35.004495\du}{9.024128\du}}
\pgfpathlineto{\pgfpoint{35.004495\du}{9.024128\du}}
\pgfpathlineto{\pgfpoint{35.010690\du}{9.024128\du}}
\pgfpathlineto{\pgfpoint{35.016409\du}{9.023175\du}}
\pgfpathlineto{\pgfpoint{35.022128\du}{9.022699\du}}
\pgfpathlineto{\pgfpoint{35.027370\du}{9.021746\du}}
\pgfpathlineto{\pgfpoint{35.033089\du}{9.020316\du}}
\pgfpathlineto{\pgfpoint{35.038331\du}{9.018410\du}}
\pgfpathlineto{\pgfpoint{35.044049\du}{9.016503\du}}
\pgfpathlineto{\pgfpoint{35.049292\du}{9.014121\du}}
\pgfpathlineto{\pgfpoint{35.055010\du}{9.011261\du}}
\pgfpathlineto{\pgfpoint{35.059776\du}{9.009355\du}}
\pgfpathlineto{\pgfpoint{35.064542\du}{9.006496\du}}
\pgfpathlineto{\pgfpoint{35.069784\du}{9.003160\du}}
\pgfpathlineto{\pgfpoint{35.074073\du}{8.999347\du}}
\pgfpathlineto{\pgfpoint{35.078839\du}{8.995535\du}}
\pgfpathlineto{\pgfpoint{35.083128\du}{8.992199\du}}
\pgfpathlineto{\pgfpoint{35.086940\du}{8.987433\du}}
\pgfpathlineto{\pgfpoint{35.091229\du}{8.983621\du}}
\pgfpathlineto{\pgfpoint{35.094565\du}{8.978378\du}}
\pgfpathlineto{\pgfpoint{35.098378\du}{8.974089\du}}
\pgfpathlineto{\pgfpoint{35.101237\du}{8.969324\du}}
\pgfpathlineto{\pgfpoint{35.104096\du}{8.964558\du}}
\pgfpathlineto{\pgfpoint{35.106479\du}{8.959316\du}}
\pgfpathlineto{\pgfpoint{35.109339\du}{8.954074\du}}
\pgfpathlineto{\pgfpoint{35.111722\du}{8.948355\du}}
\pgfpathlineto{\pgfpoint{35.113151\du}{8.942636\du}}
\pgfpathlineto{\pgfpoint{35.115057\du}{8.936917\du}}
\pgfpathlineto{\pgfpoint{35.116011\du}{8.931675\du}}
\pgfpathlineto{\pgfpoint{35.117440\du}{8.925956\du}}
\pgfpathlineto{\pgfpoint{35.117917\du}{8.920238\du}}
\pgfpathlineto{\pgfpoint{35.118393\du}{8.914519\du}}
\pgfpathlineto{\pgfpoint{35.118393\du}{8.908324\du}}
\pgfusepath{fill}
\pgfsetbuttcap
\pgfsetmiterjoin
\pgfsetdash{}{0pt}
\definecolor{dialinecolor}{rgb}{1.000000, 1.000000, 1.000000}
\pgfsetfillcolor{dialinecolor}
\pgfpathmoveto{\pgfpoint{36.159208\du}{8.908324\du}}
\pgfpathlineto{\pgfpoint{36.159208\du}{8.902605\du}}
\pgfpathlineto{\pgfpoint{36.158255\du}{8.896409\du}}
\pgfpathlineto{\pgfpoint{36.157778\du}{8.891167\du}}
\pgfpathlineto{\pgfpoint{36.157302\du}{8.884972\du}}
\pgfpathlineto{\pgfpoint{36.155396\du}{8.879730\du}}
\pgfpathlineto{\pgfpoint{36.153966\du}{8.873534\du}}
\pgfpathlineto{\pgfpoint{36.152060\du}{8.868292\du}}
\pgfpathlineto{\pgfpoint{36.149677\du}{8.863050\du}}
\pgfpathlineto{\pgfpoint{36.147294\du}{8.857331\du}}
\pgfpathlineto{\pgfpoint{36.145388\du}{8.852089\du}}
\pgfpathlineto{\pgfpoint{36.142052\du}{8.847323\du}}
\pgfpathlineto{\pgfpoint{36.139192\du}{8.842558\du}}
\pgfpathlineto{\pgfpoint{36.135380\du}{8.837792\du}}
\pgfpathlineto{\pgfpoint{36.131567\du}{8.832550\du}}
\pgfpathlineto{\pgfpoint{36.127278\du}{8.828737\du}}
\pgfpathlineto{\pgfpoint{36.123466\du}{8.824448\du}}
\pgfpathlineto{\pgfpoint{36.119177\du}{8.820636\du}}
\pgfpathlineto{\pgfpoint{36.115364\du}{8.816823\du}}
\pgfpathlineto{\pgfpoint{36.110122\du}{8.813964\du}}
\pgfpathlineto{\pgfpoint{36.105356\du}{8.810151\du}}
\pgfpathlineto{\pgfpoint{36.100591\du}{8.807292\du}}
\pgfpathlineto{\pgfpoint{36.095349\du}{8.804909\du}}
\pgfpathlineto{\pgfpoint{36.090106\du}{8.802526\du}}
\pgfpathlineto{\pgfpoint{36.084388\du}{8.799667\du}}
\pgfpathlineto{\pgfpoint{36.078669\du}{8.798237\du}}
\pgfpathlineto{\pgfpoint{36.073427\du}{8.796331\du}}
\pgfpathlineto{\pgfpoint{36.067708\du}{8.795378\du}}
\pgfpathlineto{\pgfpoint{36.061989\du}{8.793948\du}}
\pgfpathlineto{\pgfpoint{36.055794\du}{8.793472\du}}
\pgfpathlineto{\pgfpoint{36.050552\du}{8.792995\du}}
\pgfpathlineto{\pgfpoint{36.044356\du}{8.792995\du}}
\pgfpathlineto{\pgfpoint{36.044356\du}{8.792995\du}}
\pgfpathlineto{\pgfpoint{36.038638\du}{8.792995\du}}
\pgfpathlineto{\pgfpoint{36.033395\du}{8.793472\du}}
\pgfpathlineto{\pgfpoint{36.027200\du}{8.793948\du}}
\pgfpathlineto{\pgfpoint{36.021481\du}{8.795378\du}}
\pgfpathlineto{\pgfpoint{36.015763\du}{8.796331\du}}
\pgfpathlineto{\pgfpoint{36.010997\du}{8.798237\du}}
\pgfpathlineto{\pgfpoint{36.005278\du}{8.799667\du}}
\pgfpathlineto{\pgfpoint{35.999083\du}{8.802526\du}}
\pgfpathlineto{\pgfpoint{35.993841\du}{8.804909\du}}
\pgfpathlineto{\pgfpoint{35.989075\du}{8.807292\du}}
\pgfpathlineto{\pgfpoint{35.984309\du}{8.810151\du}}
\pgfpathlineto{\pgfpoint{35.979067\du}{8.813964\du}}
\pgfpathlineto{\pgfpoint{35.974301\du}{8.816823\du}}
\pgfpathlineto{\pgfpoint{35.970012\du}{8.820636\du}}
\pgfpathlineto{\pgfpoint{35.966200\du}{8.824448\du}}
\pgfpathlineto{\pgfpoint{35.961434\du}{8.828737\du}}
\pgfpathlineto{\pgfpoint{35.957622\du}{8.832550\du}}
\pgfpathlineto{\pgfpoint{35.954286\du}{8.837792\du}}
\pgfpathlineto{\pgfpoint{35.950473\du}{8.842558\du}}
\pgfpathlineto{\pgfpoint{35.947137\du}{8.847323\du}}
\pgfpathlineto{\pgfpoint{35.944278\du}{8.852089\du}}
\pgfpathlineto{\pgfpoint{35.942372\du}{8.857331\du}}
\pgfpathlineto{\pgfpoint{35.939512\du}{8.863050\du}}
\pgfpathlineto{\pgfpoint{35.937130\du}{8.868292\du}}
\pgfpathlineto{\pgfpoint{35.935223\du}{8.873534\du}}
\pgfpathlineto{\pgfpoint{35.933794\du}{8.879730\du}}
\pgfpathlineto{\pgfpoint{35.932364\du}{8.884972\du}}
\pgfpathlineto{\pgfpoint{35.931411\du}{8.891167\du}}
\pgfpathlineto{\pgfpoint{35.930934\du}{8.896409\du}}
\pgfpathlineto{\pgfpoint{35.930458\du}{8.902605\du}}
\pgfpathlineto{\pgfpoint{35.930458\du}{8.908324\du}}
\pgfpathlineto{\pgfpoint{35.930458\du}{8.908324\du}}
\pgfpathlineto{\pgfpoint{35.930458\du}{8.914519\du}}
\pgfpathlineto{\pgfpoint{35.930934\du}{8.920238\du}}
\pgfpathlineto{\pgfpoint{35.931411\du}{8.925956\du}}
\pgfpathlineto{\pgfpoint{35.932364\du}{8.931675\du}}
\pgfpathlineto{\pgfpoint{35.933794\du}{8.936917\du}}
\pgfpathlineto{\pgfpoint{35.935223\du}{8.942636\du}}
\pgfpathlineto{\pgfpoint{35.937130\du}{8.948355\du}}
\pgfpathlineto{\pgfpoint{35.939512\du}{8.954074\du}}
\pgfpathlineto{\pgfpoint{35.942372\du}{8.959316\du}}
\pgfpathlineto{\pgfpoint{35.944278\du}{8.964558\du}}
\pgfpathlineto{\pgfpoint{35.947137\du}{8.969324\du}}
\pgfpathlineto{\pgfpoint{35.950473\du}{8.974089\du}}
\pgfpathlineto{\pgfpoint{35.954286\du}{8.978378\du}}
\pgfpathlineto{\pgfpoint{35.957622\du}{8.983621\du}}
\pgfpathlineto{\pgfpoint{35.961434\du}{8.987433\du}}
\pgfpathlineto{\pgfpoint{35.966200\du}{8.992199\du}}
\pgfpathlineto{\pgfpoint{35.970012\du}{8.995535\du}}
\pgfpathlineto{\pgfpoint{35.974301\du}{8.999347\du}}
\pgfpathlineto{\pgfpoint{35.979067\du}{9.003160\du}}
\pgfpathlineto{\pgfpoint{35.984309\du}{9.006496\du}}
\pgfpathlineto{\pgfpoint{35.989075\du}{9.009355\du}}
\pgfpathlineto{\pgfpoint{35.993841\du}{9.011261\du}}
\pgfpathlineto{\pgfpoint{35.999083\du}{9.014121\du}}
\pgfpathlineto{\pgfpoint{36.005278\du}{9.016503\du}}
\pgfpathlineto{\pgfpoint{36.010997\du}{9.018410\du}}
\pgfpathlineto{\pgfpoint{36.015763\du}{9.020316\du}}
\pgfpathlineto{\pgfpoint{36.021481\du}{9.021746\du}}
\pgfpathlineto{\pgfpoint{36.027200\du}{9.022699\du}}
\pgfpathlineto{\pgfpoint{36.033395\du}{9.023175\du}}
\pgfpathlineto{\pgfpoint{36.038638\du}{9.024128\du}}
\pgfpathlineto{\pgfpoint{36.044356\du}{9.024128\du}}
\pgfpathlineto{\pgfpoint{36.044356\du}{9.024128\du}}
\pgfpathlineto{\pgfpoint{36.050552\du}{9.024128\du}}
\pgfpathlineto{\pgfpoint{36.055794\du}{9.023175\du}}
\pgfpathlineto{\pgfpoint{36.061989\du}{9.022699\du}}
\pgfpathlineto{\pgfpoint{36.067708\du}{9.021746\du}}
\pgfpathlineto{\pgfpoint{36.073427\du}{9.020316\du}}
\pgfpathlineto{\pgfpoint{36.078669\du}{9.018410\du}}
\pgfpathlineto{\pgfpoint{36.084388\du}{9.016503\du}}
\pgfpathlineto{\pgfpoint{36.090106\du}{9.014121\du}}
\pgfpathlineto{\pgfpoint{36.095349\du}{9.011261\du}}
\pgfpathlineto{\pgfpoint{36.100591\du}{9.009355\du}}
\pgfpathlineto{\pgfpoint{36.105356\du}{9.006496\du}}
\pgfpathlineto{\pgfpoint{36.110122\du}{9.003160\du}}
\pgfpathlineto{\pgfpoint{36.115364\du}{8.999347\du}}
\pgfpathlineto{\pgfpoint{36.119177\du}{8.995535\du}}
\pgfpathlineto{\pgfpoint{36.123466\du}{8.992199\du}}
\pgfpathlineto{\pgfpoint{36.127278\du}{8.987433\du}}
\pgfpathlineto{\pgfpoint{36.131567\du}{8.983621\du}}
\pgfpathlineto{\pgfpoint{36.135380\du}{8.978378\du}}
\pgfpathlineto{\pgfpoint{36.139192\du}{8.974089\du}}
\pgfpathlineto{\pgfpoint{36.142052\du}{8.969324\du}}
\pgfpathlineto{\pgfpoint{36.145388\du}{8.964558\du}}
\pgfpathlineto{\pgfpoint{36.147294\du}{8.959316\du}}
\pgfpathlineto{\pgfpoint{36.149677\du}{8.954074\du}}
\pgfpathlineto{\pgfpoint{36.152060\du}{8.948355\du}}
\pgfpathlineto{\pgfpoint{36.153966\du}{8.942636\du}}
\pgfpathlineto{\pgfpoint{36.155396\du}{8.936917\du}}
\pgfpathlineto{\pgfpoint{36.157302\du}{8.931675\du}}
\pgfpathlineto{\pgfpoint{36.157778\du}{8.925956\du}}
\pgfpathlineto{\pgfpoint{36.158255\du}{8.920238\du}}
\pgfpathlineto{\pgfpoint{36.159208\du}{8.914519\du}}
\pgfpathlineto{\pgfpoint{36.159208\du}{8.908324\du}}
\pgfusepath{fill}
\pgfsetlinewidth{0.000000\du}
\pgfsetbuttcap
\pgfsetmiterjoin
\pgfsetdash{}{0pt}
\definecolor{dialinecolor}{rgb}{1.000000, 1.000000, 1.000000}
\pgfsetstrokecolor{dialinecolor}
\pgfpathmoveto{\pgfpoint{35.015456\du}{8.898316\du}}
\pgfpathlineto{\pgfpoint{36.013856\du}{8.898316\du}}
\pgfusepath{stroke}
\pgfsetlinewidth{0.000000\du}
\pgfsetbuttcap
\pgfsetmiterjoin
\pgfsetdash{}{0pt}
\definecolor{dialinecolor}{rgb}{1.000000, 1.000000, 1.000000}
\pgfsetfillcolor{dialinecolor}
\pgfpathmoveto{\pgfpoint{35.118393\du}{9.330082\du}}
\pgfpathlineto{\pgfpoint{35.118393\du}{9.323887\du}}
\pgfpathlineto{\pgfpoint{35.117917\du}{9.318168\du}}
\pgfpathlineto{\pgfpoint{35.117440\du}{9.312926\du}}
\pgfpathlineto{\pgfpoint{35.116011\du}{9.307207\du}}
\pgfpathlineto{\pgfpoint{35.115057\du}{9.301488\du}}
\pgfpathlineto{\pgfpoint{35.113151\du}{9.295770\du}}
\pgfpathlineto{\pgfpoint{35.111722\du}{9.290051\du}}
\pgfpathlineto{\pgfpoint{35.109339\du}{9.284809\du}}
\pgfpathlineto{\pgfpoint{35.106479\du}{9.279567\du}}
\pgfpathlineto{\pgfpoint{35.104096\du}{9.274324\du}}
\pgfpathlineto{\pgfpoint{35.101237\du}{9.269082\du}}
\pgfpathlineto{\pgfpoint{35.098378\du}{9.264316\du}}
\pgfpathlineto{\pgfpoint{35.094565\du}{9.259551\du}}
\pgfpathlineto{\pgfpoint{35.091229\du}{9.254785\du}}
\pgfpathlineto{\pgfpoint{35.086940\du}{9.250973\du}}
\pgfpathlineto{\pgfpoint{35.083128\du}{9.246684\du}}
\pgfpathlineto{\pgfpoint{35.078839\du}{9.242871\du}}
\pgfpathlineto{\pgfpoint{35.074073\du}{9.238582\du}}
\pgfpathlineto{\pgfpoint{35.069784\du}{9.235723\du}}
\pgfpathlineto{\pgfpoint{35.064542\du}{9.232387\du}}
\pgfpathlineto{\pgfpoint{35.059776\du}{9.228574\du}}
\pgfpathlineto{\pgfpoint{35.055010\du}{9.226668\du}}
\pgfpathlineto{\pgfpoint{35.049292\du}{9.224285\du}}
\pgfpathlineto{\pgfpoint{35.044049\du}{9.221902\du}}
\pgfpathlineto{\pgfpoint{35.038331\du}{9.219996\du}}
\pgfpathlineto{\pgfpoint{35.033089\du}{9.218566\du}}
\pgfpathlineto{\pgfpoint{35.027370\du}{9.216660\du}}
\pgfpathlineto{\pgfpoint{35.022128\du}{9.216184\du}}
\pgfpathlineto{\pgfpoint{35.016409\du}{9.215230\du}}
\pgfpathlineto{\pgfpoint{35.010690\du}{9.214754\du}}
\pgfpathlineto{\pgfpoint{35.004495\du}{9.214754\du}}
\pgfpathlineto{\pgfpoint{35.004495\du}{9.214754\du}}
\pgfpathlineto{\pgfpoint{34.998776\du}{9.214754\du}}
\pgfpathlineto{\pgfpoint{34.993057\du}{9.215230\du}}
\pgfpathlineto{\pgfpoint{34.987338\du}{9.216184\du}}
\pgfpathlineto{\pgfpoint{34.982096\du}{9.216660\du}}
\pgfpathlineto{\pgfpoint{34.975901\du}{9.218566\du}}
\pgfpathlineto{\pgfpoint{34.970659\du}{9.219996\du}}
\pgfpathlineto{\pgfpoint{34.965417\du}{9.221902\du}}
\pgfpathlineto{\pgfpoint{34.959698\du}{9.224285\du}}
\pgfpathlineto{\pgfpoint{34.954456\du}{9.226668\du}}
\pgfpathlineto{\pgfpoint{34.949213\du}{9.228574\du}}
\pgfpathlineto{\pgfpoint{34.944924\du}{9.232387\du}}
\pgfpathlineto{\pgfpoint{34.939682\du}{9.235723\du}}
\pgfpathlineto{\pgfpoint{34.934916\du}{9.238582\du}}
\pgfpathlineto{\pgfpoint{34.930627\du}{9.242871\du}}
\pgfpathlineto{\pgfpoint{34.925862\du}{9.246684\du}}
\pgfpathlineto{\pgfpoint{34.922049\du}{9.250973\du}}
\pgfpathlineto{\pgfpoint{34.918237\du}{9.254785\du}}
\pgfpathlineto{\pgfpoint{34.914901\du}{9.259551\du}}
\pgfpathlineto{\pgfpoint{34.911088\du}{9.264316\du}}
\pgfpathlineto{\pgfpoint{34.907752\du}{9.269082\du}}
\pgfpathlineto{\pgfpoint{34.904893\du}{9.274324\du}}
\pgfpathlineto{\pgfpoint{34.902987\du}{9.279567\du}}
\pgfpathlineto{\pgfpoint{34.900127\du}{9.284809\du}}
\pgfpathlineto{\pgfpoint{34.897744\du}{9.290051\du}}
\pgfpathlineto{\pgfpoint{34.895838\du}{9.295770\du}}
\pgfpathlineto{\pgfpoint{34.894409\du}{9.301488\du}}
\pgfpathlineto{\pgfpoint{34.892979\du}{9.307207\du}}
\pgfpathlineto{\pgfpoint{34.892026\du}{9.312926\du}}
\pgfpathlineto{\pgfpoint{34.891073\du}{9.318168\du}}
\pgfpathlineto{\pgfpoint{34.891073\du}{9.323887\du}}
\pgfpathlineto{\pgfpoint{34.891073\du}{9.330082\du}}
\pgfpathlineto{\pgfpoint{34.891073\du}{9.330082\du}}
\pgfpathlineto{\pgfpoint{34.891073\du}{9.335801\du}}
\pgfpathlineto{\pgfpoint{34.891073\du}{9.341996\du}}
\pgfpathlineto{\pgfpoint{34.892026\du}{9.347715\du}}
\pgfpathlineto{\pgfpoint{34.892979\du}{9.353910\du}}
\pgfpathlineto{\pgfpoint{34.894409\du}{9.359153\du}}
\pgfpathlineto{\pgfpoint{34.895838\du}{9.364871\du}}
\pgfpathlineto{\pgfpoint{34.897744\du}{9.370590\du}}
\pgfpathlineto{\pgfpoint{34.900127\du}{9.375832\du}}
\pgfpathlineto{\pgfpoint{34.902987\du}{9.380598\du}}
\pgfpathlineto{\pgfpoint{34.904893\du}{9.386793\du}}
\pgfpathlineto{\pgfpoint{34.907752\du}{9.391559\du}}
\pgfpathlineto{\pgfpoint{34.911088\du}{9.396325\du}}
\pgfpathlineto{\pgfpoint{34.914901\du}{9.401090\du}}
\pgfpathlineto{\pgfpoint{34.918237\du}{9.405856\du}}
\pgfpathlineto{\pgfpoint{34.922049\du}{9.410145\du}}
\pgfpathlineto{\pgfpoint{34.925862\du}{9.413481\du}}
\pgfpathlineto{\pgfpoint{34.930627\du}{9.417770\du}}
\pgfpathlineto{\pgfpoint{34.934916\du}{9.422059\du}}
\pgfpathlineto{\pgfpoint{34.939682\du}{9.424918\du}}
\pgfpathlineto{\pgfpoint{34.944924\du}{9.428254\du}}
\pgfpathlineto{\pgfpoint{34.949213\du}{9.431590\du}}
\pgfpathlineto{\pgfpoint{34.954456\du}{9.433973\du}}
\pgfpathlineto{\pgfpoint{34.959698\du}{9.436356\du}}
\pgfpathlineto{\pgfpoint{34.965417\du}{9.438739\du}}
\pgfpathlineto{\pgfpoint{34.970659\du}{9.440168\du}}
\pgfpathlineto{\pgfpoint{34.975901\du}{9.442075\du}}
\pgfpathlineto{\pgfpoint{34.982096\du}{9.443504\du}}
\pgfpathlineto{\pgfpoint{34.987338\du}{9.444458\du}}
\pgfpathlineto{\pgfpoint{34.993057\du}{9.445411\du}}
\pgfpathlineto{\pgfpoint{34.998776\du}{9.445887\du}}
\pgfpathlineto{\pgfpoint{35.004495\du}{9.445887\du}}
\pgfpathlineto{\pgfpoint{35.004495\du}{9.445887\du}}
\pgfpathlineto{\pgfpoint{35.010690\du}{9.445887\du}}
\pgfpathlineto{\pgfpoint{35.016409\du}{9.445411\du}}
\pgfpathlineto{\pgfpoint{35.022128\du}{9.444458\du}}
\pgfpathlineto{\pgfpoint{35.027370\du}{9.443504\du}}
\pgfpathlineto{\pgfpoint{35.033089\du}{9.442075\du}}
\pgfpathlineto{\pgfpoint{35.038331\du}{9.440168\du}}
\pgfpathlineto{\pgfpoint{35.044049\du}{9.438739\du}}
\pgfpathlineto{\pgfpoint{35.049292\du}{9.436356\du}}
\pgfpathlineto{\pgfpoint{35.055010\du}{9.433973\du}}
\pgfpathlineto{\pgfpoint{35.059776\du}{9.431590\du}}
\pgfpathlineto{\pgfpoint{35.064542\du}{9.428254\du}}
\pgfpathlineto{\pgfpoint{35.069784\du}{9.424918\du}}
\pgfpathlineto{\pgfpoint{35.074073\du}{9.422059\du}}
\pgfpathlineto{\pgfpoint{35.078839\du}{9.417770\du}}
\pgfpathlineto{\pgfpoint{35.083128\du}{9.413481\du}}
\pgfpathlineto{\pgfpoint{35.086940\du}{9.410145\du}}
\pgfpathlineto{\pgfpoint{35.091229\du}{9.405856\du}}
\pgfpathlineto{\pgfpoint{35.094565\du}{9.401090\du}}
\pgfpathlineto{\pgfpoint{35.098378\du}{9.396325\du}}
\pgfpathlineto{\pgfpoint{35.101237\du}{9.391559\du}}
\pgfpathlineto{\pgfpoint{35.104096\du}{9.386793\du}}
\pgfpathlineto{\pgfpoint{35.106479\du}{9.380598\du}}
\pgfpathlineto{\pgfpoint{35.109339\du}{9.375832\du}}
\pgfpathlineto{\pgfpoint{35.111722\du}{9.370590\du}}
\pgfpathlineto{\pgfpoint{35.113151\du}{9.364871\du}}
\pgfpathlineto{\pgfpoint{35.115057\du}{9.359153\du}}
\pgfpathlineto{\pgfpoint{35.116011\du}{9.353910\du}}
\pgfpathlineto{\pgfpoint{35.117440\du}{9.347715\du}}
\pgfpathlineto{\pgfpoint{35.117917\du}{9.341996\du}}
\pgfpathlineto{\pgfpoint{35.118393\du}{9.335801\du}}
\pgfpathlineto{\pgfpoint{35.118393\du}{9.330082\du}}
\pgfusepath{fill}
\pgfsetbuttcap
\pgfsetmiterjoin
\pgfsetdash{}{0pt}
\definecolor{dialinecolor}{rgb}{1.000000, 1.000000, 1.000000}
\pgfsetfillcolor{dialinecolor}
\pgfpathmoveto{\pgfpoint{36.159208\du}{9.330082\du}}
\pgfpathlineto{\pgfpoint{36.159208\du}{9.323887\du}}
\pgfpathlineto{\pgfpoint{36.158255\du}{9.318168\du}}
\pgfpathlineto{\pgfpoint{36.157778\du}{9.312926\du}}
\pgfpathlineto{\pgfpoint{36.157302\du}{9.307207\du}}
\pgfpathlineto{\pgfpoint{36.155396\du}{9.301488\du}}
\pgfpathlineto{\pgfpoint{36.153966\du}{9.295770\du}}
\pgfpathlineto{\pgfpoint{36.152060\du}{9.290051\du}}
\pgfpathlineto{\pgfpoint{36.149677\du}{9.284809\du}}
\pgfpathlineto{\pgfpoint{36.147294\du}{9.279567\du}}
\pgfpathlineto{\pgfpoint{36.145388\du}{9.274324\du}}
\pgfpathlineto{\pgfpoint{36.142052\du}{9.269082\du}}
\pgfpathlineto{\pgfpoint{36.139192\du}{9.264316\du}}
\pgfpathlineto{\pgfpoint{36.135380\du}{9.259551\du}}
\pgfpathlineto{\pgfpoint{36.131567\du}{9.254785\du}}
\pgfpathlineto{\pgfpoint{36.127278\du}{9.250973\du}}
\pgfpathlineto{\pgfpoint{36.123466\du}{9.246684\du}}
\pgfpathlineto{\pgfpoint{36.119177\du}{9.242871\du}}
\pgfpathlineto{\pgfpoint{36.115364\du}{9.238582\du}}
\pgfpathlineto{\pgfpoint{36.110122\du}{9.235723\du}}
\pgfpathlineto{\pgfpoint{36.105356\du}{9.232387\du}}
\pgfpathlineto{\pgfpoint{36.100591\du}{9.228574\du}}
\pgfpathlineto{\pgfpoint{36.095349\du}{9.226668\du}}
\pgfpathlineto{\pgfpoint{36.090106\du}{9.224285\du}}
\pgfpathlineto{\pgfpoint{36.084388\du}{9.221902\du}}
\pgfpathlineto{\pgfpoint{36.078669\du}{9.219996\du}}
\pgfpathlineto{\pgfpoint{36.073427\du}{9.218566\du}}
\pgfpathlineto{\pgfpoint{36.067708\du}{9.216660\du}}
\pgfpathlineto{\pgfpoint{36.061989\du}{9.216184\du}}
\pgfpathlineto{\pgfpoint{36.055794\du}{9.215230\du}}
\pgfpathlineto{\pgfpoint{36.050552\du}{9.214754\du}}
\pgfpathlineto{\pgfpoint{36.044356\du}{9.214754\du}}
\pgfpathlineto{\pgfpoint{36.044356\du}{9.214754\du}}
\pgfpathlineto{\pgfpoint{36.038638\du}{9.214754\du}}
\pgfpathlineto{\pgfpoint{36.033395\du}{9.215230\du}}
\pgfpathlineto{\pgfpoint{36.027200\du}{9.216184\du}}
\pgfpathlineto{\pgfpoint{36.021481\du}{9.216660\du}}
\pgfpathlineto{\pgfpoint{36.015763\du}{9.218566\du}}
\pgfpathlineto{\pgfpoint{36.010997\du}{9.219996\du}}
\pgfpathlineto{\pgfpoint{36.005278\du}{9.221902\du}}
\pgfpathlineto{\pgfpoint{35.999083\du}{9.224285\du}}
\pgfpathlineto{\pgfpoint{35.993841\du}{9.226668\du}}
\pgfpathlineto{\pgfpoint{35.989075\du}{9.228574\du}}
\pgfpathlineto{\pgfpoint{35.984309\du}{9.232387\du}}
\pgfpathlineto{\pgfpoint{35.979067\du}{9.235723\du}}
\pgfpathlineto{\pgfpoint{35.974301\du}{9.238582\du}}
\pgfpathlineto{\pgfpoint{35.970012\du}{9.242871\du}}
\pgfpathlineto{\pgfpoint{35.966200\du}{9.246684\du}}
\pgfpathlineto{\pgfpoint{35.961434\du}{9.250973\du}}
\pgfpathlineto{\pgfpoint{35.957622\du}{9.254785\du}}
\pgfpathlineto{\pgfpoint{35.954286\du}{9.259551\du}}
\pgfpathlineto{\pgfpoint{35.950473\du}{9.264316\du}}
\pgfpathlineto{\pgfpoint{35.947137\du}{9.269082\du}}
\pgfpathlineto{\pgfpoint{35.944278\du}{9.274324\du}}
\pgfpathlineto{\pgfpoint{35.942372\du}{9.279567\du}}
\pgfpathlineto{\pgfpoint{35.939512\du}{9.284809\du}}
\pgfpathlineto{\pgfpoint{35.937130\du}{9.290051\du}}
\pgfpathlineto{\pgfpoint{35.935223\du}{9.295770\du}}
\pgfpathlineto{\pgfpoint{35.933794\du}{9.301488\du}}
\pgfpathlineto{\pgfpoint{35.932364\du}{9.307207\du}}
\pgfpathlineto{\pgfpoint{35.931411\du}{9.312926\du}}
\pgfpathlineto{\pgfpoint{35.930934\du}{9.318168\du}}
\pgfpathlineto{\pgfpoint{35.930458\du}{9.323887\du}}
\pgfpathlineto{\pgfpoint{35.930458\du}{9.330082\du}}
\pgfpathlineto{\pgfpoint{35.930458\du}{9.330082\du}}
\pgfpathlineto{\pgfpoint{35.930458\du}{9.335801\du}}
\pgfpathlineto{\pgfpoint{35.930934\du}{9.341996\du}}
\pgfpathlineto{\pgfpoint{35.931411\du}{9.347715\du}}
\pgfpathlineto{\pgfpoint{35.932364\du}{9.353910\du}}
\pgfpathlineto{\pgfpoint{35.933794\du}{9.359153\du}}
\pgfpathlineto{\pgfpoint{35.935223\du}{9.364871\du}}
\pgfpathlineto{\pgfpoint{35.937130\du}{9.370590\du}}
\pgfpathlineto{\pgfpoint{35.939512\du}{9.375832\du}}
\pgfpathlineto{\pgfpoint{35.942372\du}{9.380598\du}}
\pgfpathlineto{\pgfpoint{35.944278\du}{9.386793\du}}
\pgfpathlineto{\pgfpoint{35.947137\du}{9.391559\du}}
\pgfpathlineto{\pgfpoint{35.950473\du}{9.396325\du}}
\pgfpathlineto{\pgfpoint{35.954286\du}{9.401090\du}}
\pgfpathlineto{\pgfpoint{35.957622\du}{9.405856\du}}
\pgfpathlineto{\pgfpoint{35.961434\du}{9.410145\du}}
\pgfpathlineto{\pgfpoint{35.966200\du}{9.413481\du}}
\pgfpathlineto{\pgfpoint{35.970012\du}{9.417770\du}}
\pgfpathlineto{\pgfpoint{35.974301\du}{9.422059\du}}
\pgfpathlineto{\pgfpoint{35.979067\du}{9.424918\du}}
\pgfpathlineto{\pgfpoint{35.984309\du}{9.428254\du}}
\pgfpathlineto{\pgfpoint{35.989075\du}{9.431590\du}}
\pgfpathlineto{\pgfpoint{35.993841\du}{9.433973\du}}
\pgfpathlineto{\pgfpoint{35.999083\du}{9.436356\du}}
\pgfpathlineto{\pgfpoint{36.005278\du}{9.438739\du}}
\pgfpathlineto{\pgfpoint{36.010997\du}{9.440168\du}}
\pgfpathlineto{\pgfpoint{36.015763\du}{9.442075\du}}
\pgfpathlineto{\pgfpoint{36.021481\du}{9.443504\du}}
\pgfpathlineto{\pgfpoint{36.027200\du}{9.444458\du}}
\pgfpathlineto{\pgfpoint{36.033395\du}{9.445411\du}}
\pgfpathlineto{\pgfpoint{36.038638\du}{9.445887\du}}
\pgfpathlineto{\pgfpoint{36.044356\du}{9.445887\du}}
\pgfpathlineto{\pgfpoint{36.044356\du}{9.445887\du}}
\pgfpathlineto{\pgfpoint{36.050552\du}{9.445887\du}}
\pgfpathlineto{\pgfpoint{36.055794\du}{9.445411\du}}
\pgfpathlineto{\pgfpoint{36.061989\du}{9.444458\du}}
\pgfpathlineto{\pgfpoint{36.067708\du}{9.443504\du}}
\pgfpathlineto{\pgfpoint{36.073427\du}{9.442075\du}}
\pgfpathlineto{\pgfpoint{36.078669\du}{9.440168\du}}
\pgfpathlineto{\pgfpoint{36.084388\du}{9.438739\du}}
\pgfpathlineto{\pgfpoint{36.090106\du}{9.436356\du}}
\pgfpathlineto{\pgfpoint{36.095349\du}{9.433973\du}}
\pgfpathlineto{\pgfpoint{36.100591\du}{9.431590\du}}
\pgfpathlineto{\pgfpoint{36.105356\du}{9.428254\du}}
\pgfpathlineto{\pgfpoint{36.110122\du}{9.424918\du}}
\pgfpathlineto{\pgfpoint{36.115364\du}{9.422059\du}}
\pgfpathlineto{\pgfpoint{36.119177\du}{9.417770\du}}
\pgfpathlineto{\pgfpoint{36.123466\du}{9.413481\du}}
\pgfpathlineto{\pgfpoint{36.127278\du}{9.410145\du}}
\pgfpathlineto{\pgfpoint{36.131567\du}{9.405856\du}}
\pgfpathlineto{\pgfpoint{36.135380\du}{9.401090\du}}
\pgfpathlineto{\pgfpoint{36.139192\du}{9.396325\du}}
\pgfpathlineto{\pgfpoint{36.142052\du}{9.391559\du}}
\pgfpathlineto{\pgfpoint{36.145388\du}{9.386793\du}}
\pgfpathlineto{\pgfpoint{36.147294\du}{9.380598\du}}
\pgfpathlineto{\pgfpoint{36.149677\du}{9.375832\du}}
\pgfpathlineto{\pgfpoint{36.152060\du}{9.370590\du}}
\pgfpathlineto{\pgfpoint{36.153966\du}{9.364871\du}}
\pgfpathlineto{\pgfpoint{36.155396\du}{9.359153\du}}
\pgfpathlineto{\pgfpoint{36.157302\du}{9.353910\du}}
\pgfpathlineto{\pgfpoint{36.157778\du}{9.347715\du}}
\pgfpathlineto{\pgfpoint{36.158255\du}{9.341996\du}}
\pgfpathlineto{\pgfpoint{36.159208\du}{9.335801\du}}
\pgfpathlineto{\pgfpoint{36.159208\du}{9.330082\du}}
\pgfusepath{fill}
\pgfsetlinewidth{0.000000\du}
\pgfsetbuttcap
\pgfsetmiterjoin
\pgfsetdash{}{0pt}
\definecolor{dialinecolor}{rgb}{1.000000, 1.000000, 1.000000}
\pgfsetstrokecolor{dialinecolor}
\pgfpathmoveto{\pgfpoint{35.015456\du}{9.320074\du}}
\pgfpathlineto{\pgfpoint{36.013856\du}{9.320074\du}}
\pgfusepath{stroke}
\pgfsetlinewidth{0.000000\du}
\pgfsetbuttcap
\pgfsetmiterjoin
\pgfsetdash{}{0pt}
\definecolor{dialinecolor}{rgb}{1.000000, 1.000000, 1.000000}
\pgfsetfillcolor{dialinecolor}
\pgfpathmoveto{\pgfpoint{35.118393\du}{9.751841\du}}
\pgfpathlineto{\pgfpoint{35.118393\du}{9.745646\du}}
\pgfpathlineto{\pgfpoint{35.117917\du}{9.739927\du}}
\pgfpathlineto{\pgfpoint{35.117440\du}{9.734208\du}}
\pgfpathlineto{\pgfpoint{35.116011\du}{9.728489\du}}
\pgfpathlineto{\pgfpoint{35.115057\du}{9.722294\du}}
\pgfpathlineto{\pgfpoint{35.113151\du}{9.717052\du}}
\pgfpathlineto{\pgfpoint{35.111722\du}{9.711333\du}}
\pgfpathlineto{\pgfpoint{35.109339\du}{9.705614\du}}
\pgfpathlineto{\pgfpoint{35.106479\du}{9.700849\du}}
\pgfpathlineto{\pgfpoint{35.104096\du}{9.695606\du}}
\pgfpathlineto{\pgfpoint{35.101237\du}{9.689888\du}}
\pgfpathlineto{\pgfpoint{35.098378\du}{9.685122\du}}
\pgfpathlineto{\pgfpoint{35.094565\du}{9.680356\du}}
\pgfpathlineto{\pgfpoint{35.091229\du}{9.676544\du}}
\pgfpathlineto{\pgfpoint{35.086940\du}{9.671302\du}}
\pgfpathlineto{\pgfpoint{35.083128\du}{9.667489\du}}
\pgfpathlineto{\pgfpoint{35.078839\du}{9.663677\du}}
\pgfpathlineto{\pgfpoint{35.074073\du}{9.660341\du}}
\pgfpathlineto{\pgfpoint{35.069784\du}{9.656528\du}}
\pgfpathlineto{\pgfpoint{35.064542\du}{9.653192\du}}
\pgfpathlineto{\pgfpoint{35.059776\du}{9.649856\du}}
\pgfpathlineto{\pgfpoint{35.055010\du}{9.647474\du}}
\pgfpathlineto{\pgfpoint{35.049292\du}{9.645091\du}}
\pgfpathlineto{\pgfpoint{35.044049\du}{9.642708\du}}
\pgfpathlineto{\pgfpoint{35.038331\du}{9.640802\du}}
\pgfpathlineto{\pgfpoint{35.033089\du}{9.639372\du}}
\pgfpathlineto{\pgfpoint{35.027370\du}{9.637942\du}}
\pgfpathlineto{\pgfpoint{35.022128\du}{9.636989\du}}
\pgfpathlineto{\pgfpoint{35.016409\du}{9.636036\du}}
\pgfpathlineto{\pgfpoint{35.010690\du}{9.635559\du}}
\pgfpathlineto{\pgfpoint{35.004495\du}{9.635559\du}}
\pgfpathlineto{\pgfpoint{35.004495\du}{9.635559\du}}
\pgfpathlineto{\pgfpoint{34.998776\du}{9.635559\du}}
\pgfpathlineto{\pgfpoint{34.993057\du}{9.636036\du}}
\pgfpathlineto{\pgfpoint{34.987338\du}{9.636989\du}}
\pgfpathlineto{\pgfpoint{34.982096\du}{9.637942\du}}
\pgfpathlineto{\pgfpoint{34.975901\du}{9.639372\du}}
\pgfpathlineto{\pgfpoint{34.970659\du}{9.640802\du}}
\pgfpathlineto{\pgfpoint{34.965417\du}{9.642708\du}}
\pgfpathlineto{\pgfpoint{34.959698\du}{9.645091\du}}
\pgfpathlineto{\pgfpoint{34.954456\du}{9.647474\du}}
\pgfpathlineto{\pgfpoint{34.949213\du}{9.649856\du}}
\pgfpathlineto{\pgfpoint{34.944924\du}{9.653192\du}}
\pgfpathlineto{\pgfpoint{34.939682\du}{9.656528\du}}
\pgfpathlineto{\pgfpoint{34.934916\du}{9.660341\du}}
\pgfpathlineto{\pgfpoint{34.930627\du}{9.663677\du}}
\pgfpathlineto{\pgfpoint{34.925862\du}{9.667489\du}}
\pgfpathlineto{\pgfpoint{34.922049\du}{9.671302\du}}
\pgfpathlineto{\pgfpoint{34.918237\du}{9.676544\du}}
\pgfpathlineto{\pgfpoint{34.914901\du}{9.680356\du}}
\pgfpathlineto{\pgfpoint{34.911088\du}{9.685122\du}}
\pgfpathlineto{\pgfpoint{34.907752\du}{9.689888\du}}
\pgfpathlineto{\pgfpoint{34.904893\du}{9.695606\du}}
\pgfpathlineto{\pgfpoint{34.902987\du}{9.700849\du}}
\pgfpathlineto{\pgfpoint{34.900127\du}{9.705614\du}}
\pgfpathlineto{\pgfpoint{34.897744\du}{9.711333\du}}
\pgfpathlineto{\pgfpoint{34.895838\du}{9.717052\du}}
\pgfpathlineto{\pgfpoint{34.894409\du}{9.722294\du}}
\pgfpathlineto{\pgfpoint{34.892979\du}{9.728489\du}}
\pgfpathlineto{\pgfpoint{34.892026\du}{9.734208\du}}
\pgfpathlineto{\pgfpoint{34.891073\du}{9.739927\du}}
\pgfpathlineto{\pgfpoint{34.891073\du}{9.745646\du}}
\pgfpathlineto{\pgfpoint{34.891073\du}{9.751841\du}}
\pgfpathlineto{\pgfpoint{34.891073\du}{9.751841\du}}
\pgfpathlineto{\pgfpoint{34.891073\du}{9.757560\du}}
\pgfpathlineto{\pgfpoint{34.891073\du}{9.763755\du}}
\pgfpathlineto{\pgfpoint{34.892026\du}{9.768997\du}}
\pgfpathlineto{\pgfpoint{34.892979\du}{9.775193\du}}
\pgfpathlineto{\pgfpoint{34.894409\du}{9.780911\du}}
\pgfpathlineto{\pgfpoint{34.895838\du}{9.786630\du}}
\pgfpathlineto{\pgfpoint{34.897744\du}{9.791872\du}}
\pgfpathlineto{\pgfpoint{34.900127\du}{9.797591\du}}
\pgfpathlineto{\pgfpoint{34.902987\du}{9.802833\du}}
\pgfpathlineto{\pgfpoint{34.904893\du}{9.808075\du}}
\pgfpathlineto{\pgfpoint{34.907752\du}{9.813318\du}}
\pgfpathlineto{\pgfpoint{34.911088\du}{9.818083\du}}
\pgfpathlineto{\pgfpoint{34.914901\du}{9.822849\du}}
\pgfpathlineto{\pgfpoint{34.918237\du}{9.827138\du}}
\pgfpathlineto{\pgfpoint{34.922049\du}{9.831904\du}}
\pgfpathlineto{\pgfpoint{34.925862\du}{9.836193\du}}
\pgfpathlineto{\pgfpoint{34.930627\du}{9.839529\du}}
\pgfpathlineto{\pgfpoint{34.934916\du}{9.843341\du}}
\pgfpathlineto{\pgfpoint{34.939682\du}{9.846677\du}}
\pgfpathlineto{\pgfpoint{34.944924\du}{9.850490\du}}
\pgfpathlineto{\pgfpoint{34.949213\du}{9.853349\du}}
\pgfpathlineto{\pgfpoint{34.954456\du}{9.855732\du}}
\pgfpathlineto{\pgfpoint{34.959698\du}{9.858115\du}}
\pgfpathlineto{\pgfpoint{34.965417\du}{9.860497\du}}
\pgfpathlineto{\pgfpoint{34.970659\du}{9.862404\du}}
\pgfpathlineto{\pgfpoint{34.975901\du}{9.864310\du}}
\pgfpathlineto{\pgfpoint{34.982096\du}{9.865263\du}}
\pgfpathlineto{\pgfpoint{34.987338\du}{9.866216\du}}
\pgfpathlineto{\pgfpoint{34.993057\du}{9.867169\du}}
\pgfpathlineto{\pgfpoint{34.998776\du}{9.867646\du}}
\pgfpathlineto{\pgfpoint{35.004495\du}{9.867646\du}}
\pgfpathlineto{\pgfpoint{35.004495\du}{9.867646\du}}
\pgfpathlineto{\pgfpoint{35.010690\du}{9.867646\du}}
\pgfpathlineto{\pgfpoint{35.016409\du}{9.867169\du}}
\pgfpathlineto{\pgfpoint{35.022128\du}{9.866216\du}}
\pgfpathlineto{\pgfpoint{35.027370\du}{9.865263\du}}
\pgfpathlineto{\pgfpoint{35.033089\du}{9.864310\du}}
\pgfpathlineto{\pgfpoint{35.038331\du}{9.862404\du}}
\pgfpathlineto{\pgfpoint{35.044049\du}{9.860497\du}}
\pgfpathlineto{\pgfpoint{35.049292\du}{9.858115\du}}
\pgfpathlineto{\pgfpoint{35.055010\du}{9.855732\du}}
\pgfpathlineto{\pgfpoint{35.059776\du}{9.853349\du}}
\pgfpathlineto{\pgfpoint{35.064542\du}{9.850490\du}}
\pgfpathlineto{\pgfpoint{35.069784\du}{9.846677\du}}
\pgfpathlineto{\pgfpoint{35.074073\du}{9.843341\du}}
\pgfpathlineto{\pgfpoint{35.078839\du}{9.839529\du}}
\pgfpathlineto{\pgfpoint{35.083128\du}{9.836193\du}}
\pgfpathlineto{\pgfpoint{35.086940\du}{9.831904\du}}
\pgfpathlineto{\pgfpoint{35.091229\du}{9.827138\du}}
\pgfpathlineto{\pgfpoint{35.094565\du}{9.822849\du}}
\pgfpathlineto{\pgfpoint{35.098378\du}{9.818083\du}}
\pgfpathlineto{\pgfpoint{35.101237\du}{9.813318\du}}
\pgfpathlineto{\pgfpoint{35.104096\du}{9.808075\du}}
\pgfpathlineto{\pgfpoint{35.106479\du}{9.802833\du}}
\pgfpathlineto{\pgfpoint{35.109339\du}{9.797591\du}}
\pgfpathlineto{\pgfpoint{35.111722\du}{9.791872\du}}
\pgfpathlineto{\pgfpoint{35.113151\du}{9.786630\du}}
\pgfpathlineto{\pgfpoint{35.115057\du}{9.780911\du}}
\pgfpathlineto{\pgfpoint{35.116011\du}{9.775193\du}}
\pgfpathlineto{\pgfpoint{35.117440\du}{9.768997\du}}
\pgfpathlineto{\pgfpoint{35.117917\du}{9.763755\du}}
\pgfpathlineto{\pgfpoint{35.118393\du}{9.757560\du}}
\pgfpathlineto{\pgfpoint{35.118393\du}{9.751841\du}}
\pgfusepath{fill}
\pgfsetbuttcap
\pgfsetmiterjoin
\pgfsetdash{}{0pt}
\definecolor{dialinecolor}{rgb}{1.000000, 1.000000, 1.000000}
\pgfsetfillcolor{dialinecolor}
\pgfpathmoveto{\pgfpoint{36.159208\du}{9.751841\du}}
\pgfpathlineto{\pgfpoint{36.159208\du}{9.745646\du}}
\pgfpathlineto{\pgfpoint{36.158255\du}{9.739927\du}}
\pgfpathlineto{\pgfpoint{36.157778\du}{9.734208\du}}
\pgfpathlineto{\pgfpoint{36.157302\du}{9.728489\du}}
\pgfpathlineto{\pgfpoint{36.155396\du}{9.722294\du}}
\pgfpathlineto{\pgfpoint{36.153966\du}{9.717052\du}}
\pgfpathlineto{\pgfpoint{36.152060\du}{9.711333\du}}
\pgfpathlineto{\pgfpoint{36.149677\du}{9.705614\du}}
\pgfpathlineto{\pgfpoint{36.147294\du}{9.700849\du}}
\pgfpathlineto{\pgfpoint{36.145388\du}{9.695606\du}}
\pgfpathlineto{\pgfpoint{36.142052\du}{9.689888\du}}
\pgfpathlineto{\pgfpoint{36.139192\du}{9.685122\du}}
\pgfpathlineto{\pgfpoint{36.135380\du}{9.680356\du}}
\pgfpathlineto{\pgfpoint{36.131567\du}{9.676544\du}}
\pgfpathlineto{\pgfpoint{36.127278\du}{9.671302\du}}
\pgfpathlineto{\pgfpoint{36.123466\du}{9.667489\du}}
\pgfpathlineto{\pgfpoint{36.119177\du}{9.663677\du}}
\pgfpathlineto{\pgfpoint{36.115364\du}{9.660341\du}}
\pgfpathlineto{\pgfpoint{36.110122\du}{9.656528\du}}
\pgfpathlineto{\pgfpoint{36.105356\du}{9.653192\du}}
\pgfpathlineto{\pgfpoint{36.100591\du}{9.649856\du}}
\pgfpathlineto{\pgfpoint{36.095349\du}{9.647474\du}}
\pgfpathlineto{\pgfpoint{36.090106\du}{9.645091\du}}
\pgfpathlineto{\pgfpoint{36.084388\du}{9.642708\du}}
\pgfpathlineto{\pgfpoint{36.078669\du}{9.640802\du}}
\pgfpathlineto{\pgfpoint{36.073427\du}{9.639372\du}}
\pgfpathlineto{\pgfpoint{36.067708\du}{9.637942\du}}
\pgfpathlineto{\pgfpoint{36.061989\du}{9.636989\du}}
\pgfpathlineto{\pgfpoint{36.055794\du}{9.636036\du}}
\pgfpathlineto{\pgfpoint{36.050552\du}{9.635559\du}}
\pgfpathlineto{\pgfpoint{36.044356\du}{9.635559\du}}
\pgfpathlineto{\pgfpoint{36.044356\du}{9.635559\du}}
\pgfpathlineto{\pgfpoint{36.038638\du}{9.635559\du}}
\pgfpathlineto{\pgfpoint{36.033395\du}{9.636036\du}}
\pgfpathlineto{\pgfpoint{36.027200\du}{9.636989\du}}
\pgfpathlineto{\pgfpoint{36.021481\du}{9.637942\du}}
\pgfpathlineto{\pgfpoint{36.015763\du}{9.639372\du}}
\pgfpathlineto{\pgfpoint{36.010997\du}{9.640802\du}}
\pgfpathlineto{\pgfpoint{36.005278\du}{9.642708\du}}
\pgfpathlineto{\pgfpoint{35.999083\du}{9.645091\du}}
\pgfpathlineto{\pgfpoint{35.993841\du}{9.647474\du}}
\pgfpathlineto{\pgfpoint{35.989075\du}{9.649856\du}}
\pgfpathlineto{\pgfpoint{35.984309\du}{9.653192\du}}
\pgfpathlineto{\pgfpoint{35.979067\du}{9.656528\du}}
\pgfpathlineto{\pgfpoint{35.974301\du}{9.660341\du}}
\pgfpathlineto{\pgfpoint{35.970012\du}{9.663677\du}}
\pgfpathlineto{\pgfpoint{35.966200\du}{9.667489\du}}
\pgfpathlineto{\pgfpoint{35.961434\du}{9.671302\du}}
\pgfpathlineto{\pgfpoint{35.957622\du}{9.676544\du}}
\pgfpathlineto{\pgfpoint{35.954286\du}{9.680356\du}}
\pgfpathlineto{\pgfpoint{35.950473\du}{9.685122\du}}
\pgfpathlineto{\pgfpoint{35.947137\du}{9.689888\du}}
\pgfpathlineto{\pgfpoint{35.944278\du}{9.695606\du}}
\pgfpathlineto{\pgfpoint{35.942372\du}{9.700849\du}}
\pgfpathlineto{\pgfpoint{35.939512\du}{9.705614\du}}
\pgfpathlineto{\pgfpoint{35.937130\du}{9.711333\du}}
\pgfpathlineto{\pgfpoint{35.935223\du}{9.717052\du}}
\pgfpathlineto{\pgfpoint{35.933794\du}{9.722294\du}}
\pgfpathlineto{\pgfpoint{35.932364\du}{9.728489\du}}
\pgfpathlineto{\pgfpoint{35.931411\du}{9.734208\du}}
\pgfpathlineto{\pgfpoint{35.930934\du}{9.739927\du}}
\pgfpathlineto{\pgfpoint{35.930458\du}{9.745646\du}}
\pgfpathlineto{\pgfpoint{35.930458\du}{9.751841\du}}
\pgfpathlineto{\pgfpoint{35.930458\du}{9.751841\du}}
\pgfpathlineto{\pgfpoint{35.930458\du}{9.757560\du}}
\pgfpathlineto{\pgfpoint{35.930934\du}{9.763755\du}}
\pgfpathlineto{\pgfpoint{35.931411\du}{9.768997\du}}
\pgfpathlineto{\pgfpoint{35.932364\du}{9.775193\du}}
\pgfpathlineto{\pgfpoint{35.933794\du}{9.780911\du}}
\pgfpathlineto{\pgfpoint{35.935223\du}{9.786630\du}}
\pgfpathlineto{\pgfpoint{35.937130\du}{9.791872\du}}
\pgfpathlineto{\pgfpoint{35.939512\du}{9.797591\du}}
\pgfpathlineto{\pgfpoint{35.942372\du}{9.802833\du}}
\pgfpathlineto{\pgfpoint{35.944278\du}{9.808075\du}}
\pgfpathlineto{\pgfpoint{35.947137\du}{9.813318\du}}
\pgfpathlineto{\pgfpoint{35.950473\du}{9.818083\du}}
\pgfpathlineto{\pgfpoint{35.954286\du}{9.822849\du}}
\pgfpathlineto{\pgfpoint{35.957622\du}{9.827138\du}}
\pgfpathlineto{\pgfpoint{35.961434\du}{9.831904\du}}
\pgfpathlineto{\pgfpoint{35.966200\du}{9.836193\du}}
\pgfpathlineto{\pgfpoint{35.970012\du}{9.839529\du}}
\pgfpathlineto{\pgfpoint{35.974301\du}{9.843341\du}}
\pgfpathlineto{\pgfpoint{35.979067\du}{9.846677\du}}
\pgfpathlineto{\pgfpoint{35.984309\du}{9.850490\du}}
\pgfpathlineto{\pgfpoint{35.989075\du}{9.853349\du}}
\pgfpathlineto{\pgfpoint{35.993841\du}{9.855732\du}}
\pgfpathlineto{\pgfpoint{35.999083\du}{9.858115\du}}
\pgfpathlineto{\pgfpoint{36.005278\du}{9.860497\du}}
\pgfpathlineto{\pgfpoint{36.010997\du}{9.862404\du}}
\pgfpathlineto{\pgfpoint{36.015763\du}{9.864310\du}}
\pgfpathlineto{\pgfpoint{36.021481\du}{9.865263\du}}
\pgfpathlineto{\pgfpoint{36.027200\du}{9.866216\du}}
\pgfpathlineto{\pgfpoint{36.033395\du}{9.867169\du}}
\pgfpathlineto{\pgfpoint{36.038638\du}{9.867646\du}}
\pgfpathlineto{\pgfpoint{36.044356\du}{9.867646\du}}
\pgfpathlineto{\pgfpoint{36.044356\du}{9.867646\du}}
\pgfpathlineto{\pgfpoint{36.050552\du}{9.867646\du}}
\pgfpathlineto{\pgfpoint{36.055794\du}{9.867169\du}}
\pgfpathlineto{\pgfpoint{36.061989\du}{9.866216\du}}
\pgfpathlineto{\pgfpoint{36.067708\du}{9.865263\du}}
\pgfpathlineto{\pgfpoint{36.073427\du}{9.864310\du}}
\pgfpathlineto{\pgfpoint{36.078669\du}{9.862404\du}}
\pgfpathlineto{\pgfpoint{36.084388\du}{9.860497\du}}
\pgfpathlineto{\pgfpoint{36.090106\du}{9.858115\du}}
\pgfpathlineto{\pgfpoint{36.095349\du}{9.855732\du}}
\pgfpathlineto{\pgfpoint{36.100591\du}{9.853349\du}}
\pgfpathlineto{\pgfpoint{36.105356\du}{9.850490\du}}
\pgfpathlineto{\pgfpoint{36.110122\du}{9.846677\du}}
\pgfpathlineto{\pgfpoint{36.115364\du}{9.843341\du}}
\pgfpathlineto{\pgfpoint{36.119177\du}{9.839529\du}}
\pgfpathlineto{\pgfpoint{36.123466\du}{9.836193\du}}
\pgfpathlineto{\pgfpoint{36.127278\du}{9.831904\du}}
\pgfpathlineto{\pgfpoint{36.131567\du}{9.827138\du}}
\pgfpathlineto{\pgfpoint{36.135380\du}{9.822849\du}}
\pgfpathlineto{\pgfpoint{36.139192\du}{9.818083\du}}
\pgfpathlineto{\pgfpoint{36.142052\du}{9.813318\du}}
\pgfpathlineto{\pgfpoint{36.145388\du}{9.808075\du}}
\pgfpathlineto{\pgfpoint{36.147294\du}{9.802833\du}}
\pgfpathlineto{\pgfpoint{36.149677\du}{9.797591\du}}
\pgfpathlineto{\pgfpoint{36.152060\du}{9.791872\du}}
\pgfpathlineto{\pgfpoint{36.153966\du}{9.786630\du}}
\pgfpathlineto{\pgfpoint{36.155396\du}{9.780911\du}}
\pgfpathlineto{\pgfpoint{36.157302\du}{9.775193\du}}
\pgfpathlineto{\pgfpoint{36.157778\du}{9.768997\du}}
\pgfpathlineto{\pgfpoint{36.158255\du}{9.763755\du}}
\pgfpathlineto{\pgfpoint{36.159208\du}{9.757560\du}}
\pgfpathlineto{\pgfpoint{36.159208\du}{9.751841\du}}
\pgfusepath{fill}
\pgfsetlinewidth{0.000000\du}
\pgfsetbuttcap
\pgfsetmiterjoin
\pgfsetdash{}{0pt}
\definecolor{dialinecolor}{rgb}{1.000000, 1.000000, 1.000000}
\pgfsetstrokecolor{dialinecolor}
\pgfpathmoveto{\pgfpoint{35.015456\du}{9.742786\du}}
\pgfpathlineto{\pgfpoint{36.013856\du}{9.742786\du}}
\pgfusepath{stroke}
\pgfsetlinewidth{0.000000\du}
\pgfsetbuttcap
\pgfsetmiterjoin
\pgfsetdash{}{0pt}
\definecolor{dialinecolor}{rgb}{1.000000, 1.000000, 1.000000}
\pgfsetfillcolor{dialinecolor}
\pgfpathmoveto{\pgfpoint{35.118393\du}{10.173600\du}}
\pgfpathlineto{\pgfpoint{35.118393\du}{10.167881\du}}
\pgfpathlineto{\pgfpoint{35.117917\du}{10.161686\du}}
\pgfpathlineto{\pgfpoint{35.117440\du}{10.156443\du}}
\pgfpathlineto{\pgfpoint{35.116011\du}{10.150248\du}}
\pgfpathlineto{\pgfpoint{35.115057\du}{10.145006\du}}
\pgfpathlineto{\pgfpoint{35.113151\du}{10.138811\du}}
\pgfpathlineto{\pgfpoint{35.111722\du}{10.133568\du}}
\pgfpathlineto{\pgfpoint{35.109339\du}{10.128326\du}}
\pgfpathlineto{\pgfpoint{35.106479\du}{10.122607\du}}
\pgfpathlineto{\pgfpoint{35.104096\du}{10.117365\du}}
\pgfpathlineto{\pgfpoint{35.101237\du}{10.112600\du}}
\pgfpathlineto{\pgfpoint{35.098378\du}{10.107357\du}}
\pgfpathlineto{\pgfpoint{35.094565\du}{10.103068\du}}
\pgfpathlineto{\pgfpoint{35.091229\du}{10.098303\du}}
\pgfpathlineto{\pgfpoint{35.086940\du}{10.094014\du}}
\pgfpathlineto{\pgfpoint{35.083128\du}{10.089725\du}}
\pgfpathlineto{\pgfpoint{35.078839\du}{10.086389\du}}
\pgfpathlineto{\pgfpoint{35.074073\du}{10.082099\du}}
\pgfpathlineto{\pgfpoint{35.069784\du}{10.079240\du}}
\pgfpathlineto{\pgfpoint{35.064542\du}{10.075428\du}}
\pgfpathlineto{\pgfpoint{35.059776\du}{10.072568\du}}
\pgfpathlineto{\pgfpoint{35.055010\du}{10.070185\du}}
\pgfpathlineto{\pgfpoint{35.049292\du}{10.067803\du}}
\pgfpathlineto{\pgfpoint{35.044049\du}{10.065420\du}}
\pgfpathlineto{\pgfpoint{35.038331\du}{10.063514\du}}
\pgfpathlineto{\pgfpoint{35.033089\du}{10.061607\du}}
\pgfpathlineto{\pgfpoint{35.027370\du}{10.060654\du}}
\pgfpathlineto{\pgfpoint{35.022128\du}{10.059224\du}}
\pgfpathlineto{\pgfpoint{35.016409\du}{10.058748\du}}
\pgfpathlineto{\pgfpoint{35.010690\du}{10.058271\du}}
\pgfpathlineto{\pgfpoint{35.004495\du}{10.058271\du}}
\pgfpathlineto{\pgfpoint{35.004495\du}{10.058271\du}}
\pgfpathlineto{\pgfpoint{34.998776\du}{10.058271\du}}
\pgfpathlineto{\pgfpoint{34.993057\du}{10.058748\du}}
\pgfpathlineto{\pgfpoint{34.987338\du}{10.059224\du}}
\pgfpathlineto{\pgfpoint{34.982096\du}{10.060654\du}}
\pgfpathlineto{\pgfpoint{34.975901\du}{10.061607\du}}
\pgfpathlineto{\pgfpoint{34.970659\du}{10.063514\du}}
\pgfpathlineto{\pgfpoint{34.965417\du}{10.065420\du}}
\pgfpathlineto{\pgfpoint{34.959698\du}{10.067803\du}}
\pgfpathlineto{\pgfpoint{34.954456\du}{10.070185\du}}
\pgfpathlineto{\pgfpoint{34.949213\du}{10.072568\du}}
\pgfpathlineto{\pgfpoint{34.944924\du}{10.075428\du}}
\pgfpathlineto{\pgfpoint{34.939682\du}{10.079240\du}}
\pgfpathlineto{\pgfpoint{34.934916\du}{10.082099\du}}
\pgfpathlineto{\pgfpoint{34.930627\du}{10.086389\du}}
\pgfpathlineto{\pgfpoint{34.925862\du}{10.089725\du}}
\pgfpathlineto{\pgfpoint{34.922049\du}{10.094014\du}}
\pgfpathlineto{\pgfpoint{34.918237\du}{10.098303\du}}
\pgfpathlineto{\pgfpoint{34.914901\du}{10.103068\du}}
\pgfpathlineto{\pgfpoint{34.911088\du}{10.107357\du}}
\pgfpathlineto{\pgfpoint{34.907752\du}{10.112600\du}}
\pgfpathlineto{\pgfpoint{34.904893\du}{10.117365\du}}
\pgfpathlineto{\pgfpoint{34.902987\du}{10.122607\du}}
\pgfpathlineto{\pgfpoint{34.900127\du}{10.128326\du}}
\pgfpathlineto{\pgfpoint{34.897744\du}{10.133568\du}}
\pgfpathlineto{\pgfpoint{34.895838\du}{10.138811\du}}
\pgfpathlineto{\pgfpoint{34.894409\du}{10.145006\du}}
\pgfpathlineto{\pgfpoint{34.892979\du}{10.150248\du}}
\pgfpathlineto{\pgfpoint{34.892026\du}{10.156443\du}}
\pgfpathlineto{\pgfpoint{34.891073\du}{10.161686\du}}
\pgfpathlineto{\pgfpoint{34.891073\du}{10.167881\du}}
\pgfpathlineto{\pgfpoint{34.891073\du}{10.173600\du}}
\pgfpathlineto{\pgfpoint{34.891073\du}{10.173600\du}}
\pgfpathlineto{\pgfpoint{34.891073\du}{10.179795\du}}
\pgfpathlineto{\pgfpoint{34.891073\du}{10.185514\du}}
\pgfpathlineto{\pgfpoint{34.892026\du}{10.190756\du}}
\pgfpathlineto{\pgfpoint{34.892979\du}{10.196951\du}}
\pgfpathlineto{\pgfpoint{34.894409\du}{10.202670\du}}
\pgfpathlineto{\pgfpoint{34.895838\du}{10.208389\du}}
\pgfpathlineto{\pgfpoint{34.897744\du}{10.213631\du}}
\pgfpathlineto{\pgfpoint{34.900127\du}{10.219350\du}}
\pgfpathlineto{\pgfpoint{34.902987\du}{10.224592\du}}
\pgfpathlineto{\pgfpoint{34.904893\du}{10.229834\du}}
\pgfpathlineto{\pgfpoint{34.907752\du}{10.234600\du}}
\pgfpathlineto{\pgfpoint{34.911088\du}{10.239365\du}}
\pgfpathlineto{\pgfpoint{34.914901\du}{10.244131\du}}
\pgfpathlineto{\pgfpoint{34.918237\du}{10.249373\du}}
\pgfpathlineto{\pgfpoint{34.922049\du}{10.253186\du}}
\pgfpathlineto{\pgfpoint{34.925862\du}{10.257475\du}}
\pgfpathlineto{\pgfpoint{34.930627\du}{10.261287\du}}
\pgfpathlineto{\pgfpoint{34.934916\du}{10.265100\du}}
\pgfpathlineto{\pgfpoint{34.939682\du}{10.268436\du}}
\pgfpathlineto{\pgfpoint{34.944924\du}{10.271295\du}}
\pgfpathlineto{\pgfpoint{34.949213\du}{10.275108\du}}
\pgfpathlineto{\pgfpoint{34.954456\du}{10.277491\du}}
\pgfpathlineto{\pgfpoint{34.959698\du}{10.279873\du}}
\pgfpathlineto{\pgfpoint{34.965417\du}{10.282256\du}}
\pgfpathlineto{\pgfpoint{34.970659\du}{10.283209\du}}
\pgfpathlineto{\pgfpoint{34.975901\du}{10.285592\du}}
\pgfpathlineto{\pgfpoint{34.982096\du}{10.287022\du}}
\pgfpathlineto{\pgfpoint{34.987338\du}{10.287975\du}}
\pgfpathlineto{\pgfpoint{34.993057\du}{10.288451\du}}
\pgfpathlineto{\pgfpoint{34.998776\du}{10.289405\du}}
\pgfpathlineto{\pgfpoint{35.004495\du}{10.289405\du}}
\pgfpathlineto{\pgfpoint{35.004495\du}{10.289405\du}}
\pgfpathlineto{\pgfpoint{35.010690\du}{10.289405\du}}
\pgfpathlineto{\pgfpoint{35.016409\du}{10.288451\du}}
\pgfpathlineto{\pgfpoint{35.022128\du}{10.287975\du}}
\pgfpathlineto{\pgfpoint{35.027370\du}{10.287022\du}}
\pgfpathlineto{\pgfpoint{35.033089\du}{10.285592\du}}
\pgfpathlineto{\pgfpoint{35.038331\du}{10.283209\du}}
\pgfpathlineto{\pgfpoint{35.044049\du}{10.282256\du}}
\pgfpathlineto{\pgfpoint{35.049292\du}{10.279873\du}}
\pgfpathlineto{\pgfpoint{35.055010\du}{10.277491\du}}
\pgfpathlineto{\pgfpoint{35.059776\du}{10.275108\du}}
\pgfpathlineto{\pgfpoint{35.064542\du}{10.271295\du}}
\pgfpathlineto{\pgfpoint{35.069784\du}{10.268436\du}}
\pgfpathlineto{\pgfpoint{35.074073\du}{10.265100\du}}
\pgfpathlineto{\pgfpoint{35.078839\du}{10.261287\du}}
\pgfpathlineto{\pgfpoint{35.083128\du}{10.257475\du}}
\pgfpathlineto{\pgfpoint{35.086940\du}{10.253186\du}}
\pgfpathlineto{\pgfpoint{35.091229\du}{10.249373\du}}
\pgfpathlineto{\pgfpoint{35.094565\du}{10.244131\du}}
\pgfpathlineto{\pgfpoint{35.098378\du}{10.239365\du}}
\pgfpathlineto{\pgfpoint{35.101237\du}{10.234600\du}}
\pgfpathlineto{\pgfpoint{35.104096\du}{10.229834\du}}
\pgfpathlineto{\pgfpoint{35.106479\du}{10.224592\du}}
\pgfpathlineto{\pgfpoint{35.109339\du}{10.219350\du}}
\pgfpathlineto{\pgfpoint{35.111722\du}{10.213631\du}}
\pgfpathlineto{\pgfpoint{35.113151\du}{10.208389\du}}
\pgfpathlineto{\pgfpoint{35.115057\du}{10.202670\du}}
\pgfpathlineto{\pgfpoint{35.116011\du}{10.196951\du}}
\pgfpathlineto{\pgfpoint{35.117440\du}{10.190756\du}}
\pgfpathlineto{\pgfpoint{35.117917\du}{10.185514\du}}
\pgfpathlineto{\pgfpoint{35.118393\du}{10.179795\du}}
\pgfpathlineto{\pgfpoint{35.118393\du}{10.173600\du}}
\pgfusepath{fill}
\pgfsetbuttcap
\pgfsetmiterjoin
\pgfsetdash{}{0pt}
\definecolor{dialinecolor}{rgb}{1.000000, 1.000000, 1.000000}
\pgfsetfillcolor{dialinecolor}
\pgfpathmoveto{\pgfpoint{36.159208\du}{10.173600\du}}
\pgfpathlineto{\pgfpoint{36.159208\du}{10.167881\du}}
\pgfpathlineto{\pgfpoint{36.158255\du}{10.161686\du}}
\pgfpathlineto{\pgfpoint{36.157778\du}{10.156443\du}}
\pgfpathlineto{\pgfpoint{36.157302\du}{10.150248\du}}
\pgfpathlineto{\pgfpoint{36.155396\du}{10.145006\du}}
\pgfpathlineto{\pgfpoint{36.153966\du}{10.138811\du}}
\pgfpathlineto{\pgfpoint{36.152060\du}{10.133568\du}}
\pgfpathlineto{\pgfpoint{36.149677\du}{10.128326\du}}
\pgfpathlineto{\pgfpoint{36.147294\du}{10.122607\du}}
\pgfpathlineto{\pgfpoint{36.145388\du}{10.117365\du}}
\pgfpathlineto{\pgfpoint{36.142052\du}{10.112600\du}}
\pgfpathlineto{\pgfpoint{36.139192\du}{10.107357\du}}
\pgfpathlineto{\pgfpoint{36.135380\du}{10.103068\du}}
\pgfpathlineto{\pgfpoint{36.131567\du}{10.098303\du}}
\pgfpathlineto{\pgfpoint{36.127278\du}{10.094014\du}}
\pgfpathlineto{\pgfpoint{36.123466\du}{10.089725\du}}
\pgfpathlineto{\pgfpoint{36.119177\du}{10.086389\du}}
\pgfpathlineto{\pgfpoint{36.115364\du}{10.082099\du}}
\pgfpathlineto{\pgfpoint{36.110122\du}{10.079240\du}}
\pgfpathlineto{\pgfpoint{36.105356\du}{10.075428\du}}
\pgfpathlineto{\pgfpoint{36.100591\du}{10.072568\du}}
\pgfpathlineto{\pgfpoint{36.095349\du}{10.070185\du}}
\pgfpathlineto{\pgfpoint{36.090106\du}{10.067803\du}}
\pgfpathlineto{\pgfpoint{36.084388\du}{10.065420\du}}
\pgfpathlineto{\pgfpoint{36.078669\du}{10.063514\du}}
\pgfpathlineto{\pgfpoint{36.073427\du}{10.061607\du}}
\pgfpathlineto{\pgfpoint{36.067708\du}{10.060654\du}}
\pgfpathlineto{\pgfpoint{36.061989\du}{10.059224\du}}
\pgfpathlineto{\pgfpoint{36.055794\du}{10.058748\du}}
\pgfpathlineto{\pgfpoint{36.050552\du}{10.058271\du}}
\pgfpathlineto{\pgfpoint{36.044356\du}{10.058271\du}}
\pgfpathlineto{\pgfpoint{36.044356\du}{10.058271\du}}
\pgfpathlineto{\pgfpoint{36.038638\du}{10.058271\du}}
\pgfpathlineto{\pgfpoint{36.033395\du}{10.058748\du}}
\pgfpathlineto{\pgfpoint{36.027200\du}{10.059224\du}}
\pgfpathlineto{\pgfpoint{36.021481\du}{10.060654\du}}
\pgfpathlineto{\pgfpoint{36.015763\du}{10.061607\du}}
\pgfpathlineto{\pgfpoint{36.010997\du}{10.063514\du}}
\pgfpathlineto{\pgfpoint{36.005278\du}{10.065420\du}}
\pgfpathlineto{\pgfpoint{35.999083\du}{10.067803\du}}
\pgfpathlineto{\pgfpoint{35.993841\du}{10.070185\du}}
\pgfpathlineto{\pgfpoint{35.989075\du}{10.072568\du}}
\pgfpathlineto{\pgfpoint{35.984309\du}{10.075428\du}}
\pgfpathlineto{\pgfpoint{35.979067\du}{10.079240\du}}
\pgfpathlineto{\pgfpoint{35.974301\du}{10.082099\du}}
\pgfpathlineto{\pgfpoint{35.970012\du}{10.086389\du}}
\pgfpathlineto{\pgfpoint{35.966200\du}{10.089725\du}}
\pgfpathlineto{\pgfpoint{35.961434\du}{10.094014\du}}
\pgfpathlineto{\pgfpoint{35.957622\du}{10.098303\du}}
\pgfpathlineto{\pgfpoint{35.954286\du}{10.103068\du}}
\pgfpathlineto{\pgfpoint{35.950473\du}{10.107357\du}}
\pgfpathlineto{\pgfpoint{35.947137\du}{10.112600\du}}
\pgfpathlineto{\pgfpoint{35.944278\du}{10.117365\du}}
\pgfpathlineto{\pgfpoint{35.942372\du}{10.122607\du}}
\pgfpathlineto{\pgfpoint{35.939512\du}{10.128326\du}}
\pgfpathlineto{\pgfpoint{35.937130\du}{10.133568\du}}
\pgfpathlineto{\pgfpoint{35.935223\du}{10.138811\du}}
\pgfpathlineto{\pgfpoint{35.933794\du}{10.145006\du}}
\pgfpathlineto{\pgfpoint{35.932364\du}{10.150248\du}}
\pgfpathlineto{\pgfpoint{35.931411\du}{10.156443\du}}
\pgfpathlineto{\pgfpoint{35.930934\du}{10.161686\du}}
\pgfpathlineto{\pgfpoint{35.930458\du}{10.167881\du}}
\pgfpathlineto{\pgfpoint{35.930458\du}{10.173600\du}}
\pgfpathlineto{\pgfpoint{35.930458\du}{10.173600\du}}
\pgfpathlineto{\pgfpoint{35.930458\du}{10.179795\du}}
\pgfpathlineto{\pgfpoint{35.930934\du}{10.185514\du}}
\pgfpathlineto{\pgfpoint{35.931411\du}{10.190756\du}}
\pgfpathlineto{\pgfpoint{35.932364\du}{10.196951\du}}
\pgfpathlineto{\pgfpoint{35.933794\du}{10.202670\du}}
\pgfpathlineto{\pgfpoint{35.935223\du}{10.208389\du}}
\pgfpathlineto{\pgfpoint{35.937130\du}{10.213631\du}}
\pgfpathlineto{\pgfpoint{35.939512\du}{10.219350\du}}
\pgfpathlineto{\pgfpoint{35.942372\du}{10.224592\du}}
\pgfpathlineto{\pgfpoint{35.944278\du}{10.229834\du}}
\pgfpathlineto{\pgfpoint{35.947137\du}{10.234600\du}}
\pgfpathlineto{\pgfpoint{35.950473\du}{10.239365\du}}
\pgfpathlineto{\pgfpoint{35.954286\du}{10.244131\du}}
\pgfpathlineto{\pgfpoint{35.957622\du}{10.249373\du}}
\pgfpathlineto{\pgfpoint{35.961434\du}{10.253186\du}}
\pgfpathlineto{\pgfpoint{35.966200\du}{10.257475\du}}
\pgfpathlineto{\pgfpoint{35.970012\du}{10.261287\du}}
\pgfpathlineto{\pgfpoint{35.974301\du}{10.265100\du}}
\pgfpathlineto{\pgfpoint{35.979067\du}{10.268436\du}}
\pgfpathlineto{\pgfpoint{35.984309\du}{10.271295\du}}
\pgfpathlineto{\pgfpoint{35.989075\du}{10.275108\du}}
\pgfpathlineto{\pgfpoint{35.993841\du}{10.277491\du}}
\pgfpathlineto{\pgfpoint{35.999083\du}{10.279873\du}}
\pgfpathlineto{\pgfpoint{36.005278\du}{10.282256\du}}
\pgfpathlineto{\pgfpoint{36.010997\du}{10.283209\du}}
\pgfpathlineto{\pgfpoint{36.015763\du}{10.285592\du}}
\pgfpathlineto{\pgfpoint{36.021481\du}{10.287022\du}}
\pgfpathlineto{\pgfpoint{36.027200\du}{10.287975\du}}
\pgfpathlineto{\pgfpoint{36.033395\du}{10.288451\du}}
\pgfpathlineto{\pgfpoint{36.038638\du}{10.289405\du}}
\pgfpathlineto{\pgfpoint{36.044356\du}{10.289405\du}}
\pgfpathlineto{\pgfpoint{36.044356\du}{10.289405\du}}
\pgfpathlineto{\pgfpoint{36.050552\du}{10.289405\du}}
\pgfpathlineto{\pgfpoint{36.055794\du}{10.288451\du}}
\pgfpathlineto{\pgfpoint{36.061989\du}{10.287975\du}}
\pgfpathlineto{\pgfpoint{36.067708\du}{10.287022\du}}
\pgfpathlineto{\pgfpoint{36.073427\du}{10.285592\du}}
\pgfpathlineto{\pgfpoint{36.078669\du}{10.283209\du}}
\pgfpathlineto{\pgfpoint{36.084388\du}{10.282256\du}}
\pgfpathlineto{\pgfpoint{36.090106\du}{10.279873\du}}
\pgfpathlineto{\pgfpoint{36.095349\du}{10.277491\du}}
\pgfpathlineto{\pgfpoint{36.100591\du}{10.275108\du}}
\pgfpathlineto{\pgfpoint{36.105356\du}{10.271295\du}}
\pgfpathlineto{\pgfpoint{36.110122\du}{10.268436\du}}
\pgfpathlineto{\pgfpoint{36.115364\du}{10.265100\du}}
\pgfpathlineto{\pgfpoint{36.119177\du}{10.261287\du}}
\pgfpathlineto{\pgfpoint{36.123466\du}{10.257475\du}}
\pgfpathlineto{\pgfpoint{36.127278\du}{10.253186\du}}
\pgfpathlineto{\pgfpoint{36.131567\du}{10.249373\du}}
\pgfpathlineto{\pgfpoint{36.135380\du}{10.244131\du}}
\pgfpathlineto{\pgfpoint{36.139192\du}{10.239365\du}}
\pgfpathlineto{\pgfpoint{36.142052\du}{10.234600\du}}
\pgfpathlineto{\pgfpoint{36.145388\du}{10.229834\du}}
\pgfpathlineto{\pgfpoint{36.147294\du}{10.224592\du}}
\pgfpathlineto{\pgfpoint{36.149677\du}{10.219350\du}}
\pgfpathlineto{\pgfpoint{36.152060\du}{10.213631\du}}
\pgfpathlineto{\pgfpoint{36.153966\du}{10.208389\du}}
\pgfpathlineto{\pgfpoint{36.155396\du}{10.202670\du}}
\pgfpathlineto{\pgfpoint{36.157302\du}{10.196951\du}}
\pgfpathlineto{\pgfpoint{36.157778\du}{10.190756\du}}
\pgfpathlineto{\pgfpoint{36.158255\du}{10.185514\du}}
\pgfpathlineto{\pgfpoint{36.159208\du}{10.179795\du}}
\pgfpathlineto{\pgfpoint{36.159208\du}{10.173600\du}}
\pgfusepath{fill}
\pgfsetlinewidth{0.000000\du}
\pgfsetbuttcap
\pgfsetmiterjoin
\pgfsetdash{}{0pt}
\definecolor{dialinecolor}{rgb}{1.000000, 1.000000, 1.000000}
\pgfsetstrokecolor{dialinecolor}
\pgfpathmoveto{\pgfpoint{35.015456\du}{10.164068\du}}
\pgfpathlineto{\pgfpoint{36.013856\du}{10.164068\du}}
\pgfusepath{stroke}
\pgfsetlinewidth{0.000000\du}
\pgfsetbuttcap
\pgfsetmiterjoin
\pgfsetdash{}{0pt}
\definecolor{dialinecolor}{rgb}{1.000000, 1.000000, 1.000000}
\pgfsetfillcolor{dialinecolor}
\pgfpathmoveto{\pgfpoint{35.118393\du}{10.595835\du}}
\pgfpathlineto{\pgfpoint{35.118393\du}{10.590116\du}}
\pgfpathlineto{\pgfpoint{35.117917\du}{10.583921\du}}
\pgfpathlineto{\pgfpoint{35.117440\du}{10.579155\du}}
\pgfpathlineto{\pgfpoint{35.116011\du}{10.572960\du}}
\pgfpathlineto{\pgfpoint{35.115057\du}{10.566765\du}}
\pgfpathlineto{\pgfpoint{35.113151\du}{10.561522\du}}
\pgfpathlineto{\pgfpoint{35.111722\du}{10.556280\du}}
\pgfpathlineto{\pgfpoint{35.109339\du}{10.550085\du}}
\pgfpathlineto{\pgfpoint{35.106479\du}{10.545319\du}}
\pgfpathlineto{\pgfpoint{35.104096\du}{10.540077\du}}
\pgfpathlineto{\pgfpoint{35.101237\du}{10.534835\du}}
\pgfpathlineto{\pgfpoint{35.098378\du}{10.530069\du}}
\pgfpathlineto{\pgfpoint{35.094565\du}{10.525304\du}}
\pgfpathlineto{\pgfpoint{35.091229\du}{10.521014\du}}
\pgfpathlineto{\pgfpoint{35.086940\du}{10.516249\du}}
\pgfpathlineto{\pgfpoint{35.083128\du}{10.511960\du}}
\pgfpathlineto{\pgfpoint{35.078839\du}{10.508147\du}}
\pgfpathlineto{\pgfpoint{35.074073\du}{10.504811\du}}
\pgfpathlineto{\pgfpoint{35.069784\du}{10.500522\du}}
\pgfpathlineto{\pgfpoint{35.064542\du}{10.497663\du}}
\pgfpathlineto{\pgfpoint{35.059776\du}{10.494327\du}}
\pgfpathlineto{\pgfpoint{35.055010\du}{10.491944\du}}
\pgfpathlineto{\pgfpoint{35.049292\du}{10.489561\du}}
\pgfpathlineto{\pgfpoint{35.044049\du}{10.487178\du}}
\pgfpathlineto{\pgfpoint{35.038331\du}{10.485749\du}}
\pgfpathlineto{\pgfpoint{35.033089\du}{10.483843\du}}
\pgfpathlineto{\pgfpoint{35.027370\du}{10.482413\du}}
\pgfpathlineto{\pgfpoint{35.022128\du}{10.481460\du}}
\pgfpathlineto{\pgfpoint{35.016409\du}{10.480983\du}}
\pgfpathlineto{\pgfpoint{35.010690\du}{10.480030\du}}
\pgfpathlineto{\pgfpoint{35.004495\du}{10.480030\du}}
\pgfpathlineto{\pgfpoint{35.004495\du}{10.480030\du}}
\pgfpathlineto{\pgfpoint{34.998776\du}{10.480030\du}}
\pgfpathlineto{\pgfpoint{34.993057\du}{10.480983\du}}
\pgfpathlineto{\pgfpoint{34.987338\du}{10.481460\du}}
\pgfpathlineto{\pgfpoint{34.982096\du}{10.482413\du}}
\pgfpathlineto{\pgfpoint{34.975901\du}{10.483843\du}}
\pgfpathlineto{\pgfpoint{34.970659\du}{10.485749\du}}
\pgfpathlineto{\pgfpoint{34.965417\du}{10.487178\du}}
\pgfpathlineto{\pgfpoint{34.959698\du}{10.489561\du}}
\pgfpathlineto{\pgfpoint{34.954456\du}{10.491944\du}}
\pgfpathlineto{\pgfpoint{34.949213\du}{10.494327\du}}
\pgfpathlineto{\pgfpoint{34.944924\du}{10.497663\du}}
\pgfpathlineto{\pgfpoint{34.939682\du}{10.500522\du}}
\pgfpathlineto{\pgfpoint{34.934916\du}{10.504811\du}}
\pgfpathlineto{\pgfpoint{34.930627\du}{10.508147\du}}
\pgfpathlineto{\pgfpoint{34.925862\du}{10.511960\du}}
\pgfpathlineto{\pgfpoint{34.922049\du}{10.516249\du}}
\pgfpathlineto{\pgfpoint{34.918237\du}{10.521014\du}}
\pgfpathlineto{\pgfpoint{34.914901\du}{10.525304\du}}
\pgfpathlineto{\pgfpoint{34.911088\du}{10.530069\du}}
\pgfpathlineto{\pgfpoint{34.907752\du}{10.534835\du}}
\pgfpathlineto{\pgfpoint{34.904893\du}{10.540077\du}}
\pgfpathlineto{\pgfpoint{34.902987\du}{10.545319\du}}
\pgfpathlineto{\pgfpoint{34.900127\du}{10.550085\du}}
\pgfpathlineto{\pgfpoint{34.897744\du}{10.556280\du}}
\pgfpathlineto{\pgfpoint{34.895838\du}{10.561522\du}}
\pgfpathlineto{\pgfpoint{34.894409\du}{10.566765\du}}
\pgfpathlineto{\pgfpoint{34.892979\du}{10.572960\du}}
\pgfpathlineto{\pgfpoint{34.892026\du}{10.579155\du}}
\pgfpathlineto{\pgfpoint{34.891073\du}{10.583921\du}}
\pgfpathlineto{\pgfpoint{34.891073\du}{10.590116\du}}
\pgfpathlineto{\pgfpoint{34.891073\du}{10.595835\du}}
\pgfpathlineto{\pgfpoint{34.891073\du}{10.595835\du}}
\pgfpathlineto{\pgfpoint{34.891073\du}{10.602507\du}}
\pgfpathlineto{\pgfpoint{34.891073\du}{10.607749\du}}
\pgfpathlineto{\pgfpoint{34.892026\du}{10.613468\du}}
\pgfpathlineto{\pgfpoint{34.892979\du}{10.619663\du}}
\pgfpathlineto{\pgfpoint{34.894409\du}{10.625858\du}}
\pgfpathlineto{\pgfpoint{34.895838\du}{10.631101\du}}
\pgfpathlineto{\pgfpoint{34.897744\du}{10.636343\du}}
\pgfpathlineto{\pgfpoint{34.900127\du}{10.642538\du}}
\pgfpathlineto{\pgfpoint{34.902987\du}{10.647304\du}}
\pgfpathlineto{\pgfpoint{34.904893\du}{10.652546\du}}
\pgfpathlineto{\pgfpoint{34.907752\du}{10.657788\du}}
\pgfpathlineto{\pgfpoint{34.911088\du}{10.662554\du}}
\pgfpathlineto{\pgfpoint{34.914901\du}{10.667319\du}}
\pgfpathlineto{\pgfpoint{34.918237\du}{10.671609\du}}
\pgfpathlineto{\pgfpoint{34.922049\du}{10.676374\du}}
\pgfpathlineto{\pgfpoint{34.925862\du}{10.680663\du}}
\pgfpathlineto{\pgfpoint{34.930627\du}{10.684476\du}}
\pgfpathlineto{\pgfpoint{34.934916\du}{10.687812\du}}
\pgfpathlineto{\pgfpoint{34.939682\du}{10.691148\du}}
\pgfpathlineto{\pgfpoint{34.944924\du}{10.694960\du}}
\pgfpathlineto{\pgfpoint{34.949213\du}{10.697820\du}}
\pgfpathlineto{\pgfpoint{34.954456\du}{10.700202\du}}
\pgfpathlineto{\pgfpoint{34.959698\du}{10.703062\du}}
\pgfpathlineto{\pgfpoint{34.965417\du}{10.705445\du}}
\pgfpathlineto{\pgfpoint{34.970659\du}{10.706874\du}}
\pgfpathlineto{\pgfpoint{34.975901\du}{10.708781\du}}
\pgfpathlineto{\pgfpoint{34.982096\du}{10.710210\du}}
\pgfpathlineto{\pgfpoint{34.987338\du}{10.711163\du}}
\pgfpathlineto{\pgfpoint{34.993057\du}{10.711640\du}}
\pgfpathlineto{\pgfpoint{34.998776\du}{10.712116\du}}
\pgfpathlineto{\pgfpoint{35.004495\du}{10.712116\du}}
\pgfpathlineto{\pgfpoint{35.004495\du}{10.712116\du}}
\pgfpathlineto{\pgfpoint{35.010690\du}{10.712116\du}}
\pgfpathlineto{\pgfpoint{35.016409\du}{10.711640\du}}
\pgfpathlineto{\pgfpoint{35.022128\du}{10.711163\du}}
\pgfpathlineto{\pgfpoint{35.027370\du}{10.710210\du}}
\pgfpathlineto{\pgfpoint{35.033089\du}{10.708781\du}}
\pgfpathlineto{\pgfpoint{35.038331\du}{10.706874\du}}
\pgfpathlineto{\pgfpoint{35.044049\du}{10.705445\du}}
\pgfpathlineto{\pgfpoint{35.049292\du}{10.703062\du}}
\pgfpathlineto{\pgfpoint{35.055010\du}{10.700202\du}}
\pgfpathlineto{\pgfpoint{35.059776\du}{10.697820\du}}
\pgfpathlineto{\pgfpoint{35.064542\du}{10.694960\du}}
\pgfpathlineto{\pgfpoint{35.069784\du}{10.691148\du}}
\pgfpathlineto{\pgfpoint{35.074073\du}{10.687812\du}}
\pgfpathlineto{\pgfpoint{35.078839\du}{10.684476\du}}
\pgfpathlineto{\pgfpoint{35.083128\du}{10.680663\du}}
\pgfpathlineto{\pgfpoint{35.086940\du}{10.676374\du}}
\pgfpathlineto{\pgfpoint{35.091229\du}{10.671609\du}}
\pgfpathlineto{\pgfpoint{35.094565\du}{10.667319\du}}
\pgfpathlineto{\pgfpoint{35.098378\du}{10.662554\du}}
\pgfpathlineto{\pgfpoint{35.101237\du}{10.657788\du}}
\pgfpathlineto{\pgfpoint{35.104096\du}{10.652546\du}}
\pgfpathlineto{\pgfpoint{35.106479\du}{10.647304\du}}
\pgfpathlineto{\pgfpoint{35.109339\du}{10.642538\du}}
\pgfpathlineto{\pgfpoint{35.111722\du}{10.636343\du}}
\pgfpathlineto{\pgfpoint{35.113151\du}{10.631101\du}}
\pgfpathlineto{\pgfpoint{35.115057\du}{10.625858\du}}
\pgfpathlineto{\pgfpoint{35.116011\du}{10.619663\du}}
\pgfpathlineto{\pgfpoint{35.117440\du}{10.613468\du}}
\pgfpathlineto{\pgfpoint{35.117917\du}{10.607749\du}}
\pgfpathlineto{\pgfpoint{35.118393\du}{10.602507\du}}
\pgfpathlineto{\pgfpoint{35.118393\du}{10.595835\du}}
\pgfusepath{fill}
\pgfsetbuttcap
\pgfsetmiterjoin
\pgfsetdash{}{0pt}
\definecolor{dialinecolor}{rgb}{1.000000, 1.000000, 1.000000}
\pgfsetfillcolor{dialinecolor}
\pgfpathmoveto{\pgfpoint{36.159208\du}{10.595835\du}}
\pgfpathlineto{\pgfpoint{36.159208\du}{10.590116\du}}
\pgfpathlineto{\pgfpoint{36.158255\du}{10.583921\du}}
\pgfpathlineto{\pgfpoint{36.157778\du}{10.579155\du}}
\pgfpathlineto{\pgfpoint{36.157302\du}{10.572960\du}}
\pgfpathlineto{\pgfpoint{36.155396\du}{10.566765\du}}
\pgfpathlineto{\pgfpoint{36.153966\du}{10.561522\du}}
\pgfpathlineto{\pgfpoint{36.152060\du}{10.556280\du}}
\pgfpathlineto{\pgfpoint{36.149677\du}{10.550085\du}}
\pgfpathlineto{\pgfpoint{36.147294\du}{10.545319\du}}
\pgfpathlineto{\pgfpoint{36.145388\du}{10.540077\du}}
\pgfpathlineto{\pgfpoint{36.142052\du}{10.534835\du}}
\pgfpathlineto{\pgfpoint{36.139192\du}{10.530069\du}}
\pgfpathlineto{\pgfpoint{36.135380\du}{10.525304\du}}
\pgfpathlineto{\pgfpoint{36.131567\du}{10.521014\du}}
\pgfpathlineto{\pgfpoint{36.127278\du}{10.516249\du}}
\pgfpathlineto{\pgfpoint{36.123466\du}{10.511960\du}}
\pgfpathlineto{\pgfpoint{36.119177\du}{10.508147\du}}
\pgfpathlineto{\pgfpoint{36.115364\du}{10.504811\du}}
\pgfpathlineto{\pgfpoint{36.110122\du}{10.500522\du}}
\pgfpathlineto{\pgfpoint{36.105356\du}{10.497663\du}}
\pgfpathlineto{\pgfpoint{36.100591\du}{10.494327\du}}
\pgfpathlineto{\pgfpoint{36.095349\du}{10.491944\du}}
\pgfpathlineto{\pgfpoint{36.090106\du}{10.489561\du}}
\pgfpathlineto{\pgfpoint{36.084388\du}{10.487178\du}}
\pgfpathlineto{\pgfpoint{36.078669\du}{10.485749\du}}
\pgfpathlineto{\pgfpoint{36.073427\du}{10.483843\du}}
\pgfpathlineto{\pgfpoint{36.067708\du}{10.482413\du}}
\pgfpathlineto{\pgfpoint{36.061989\du}{10.481460\du}}
\pgfpathlineto{\pgfpoint{36.055794\du}{10.480983\du}}
\pgfpathlineto{\pgfpoint{36.050552\du}{10.480030\du}}
\pgfpathlineto{\pgfpoint{36.044356\du}{10.480030\du}}
\pgfpathlineto{\pgfpoint{36.044356\du}{10.480030\du}}
\pgfpathlineto{\pgfpoint{36.038638\du}{10.480030\du}}
\pgfpathlineto{\pgfpoint{36.033395\du}{10.480983\du}}
\pgfpathlineto{\pgfpoint{36.027200\du}{10.481460\du}}
\pgfpathlineto{\pgfpoint{36.021481\du}{10.482413\du}}
\pgfpathlineto{\pgfpoint{36.015763\du}{10.483843\du}}
\pgfpathlineto{\pgfpoint{36.010997\du}{10.485749\du}}
\pgfpathlineto{\pgfpoint{36.005278\du}{10.487178\du}}
\pgfpathlineto{\pgfpoint{35.999083\du}{10.489561\du}}
\pgfpathlineto{\pgfpoint{35.993841\du}{10.491944\du}}
\pgfpathlineto{\pgfpoint{35.989075\du}{10.494327\du}}
\pgfpathlineto{\pgfpoint{35.984309\du}{10.497663\du}}
\pgfpathlineto{\pgfpoint{35.979067\du}{10.500522\du}}
\pgfpathlineto{\pgfpoint{35.974301\du}{10.504811\du}}
\pgfpathlineto{\pgfpoint{35.970012\du}{10.508147\du}}
\pgfpathlineto{\pgfpoint{35.966200\du}{10.511960\du}}
\pgfpathlineto{\pgfpoint{35.961434\du}{10.516249\du}}
\pgfpathlineto{\pgfpoint{35.957622\du}{10.521014\du}}
\pgfpathlineto{\pgfpoint{35.954286\du}{10.525304\du}}
\pgfpathlineto{\pgfpoint{35.950473\du}{10.530069\du}}
\pgfpathlineto{\pgfpoint{35.947137\du}{10.534835\du}}
\pgfpathlineto{\pgfpoint{35.944278\du}{10.540077\du}}
\pgfpathlineto{\pgfpoint{35.942372\du}{10.545319\du}}
\pgfpathlineto{\pgfpoint{35.939512\du}{10.550085\du}}
\pgfpathlineto{\pgfpoint{35.937130\du}{10.556280\du}}
\pgfpathlineto{\pgfpoint{35.935223\du}{10.561522\du}}
\pgfpathlineto{\pgfpoint{35.933794\du}{10.566765\du}}
\pgfpathlineto{\pgfpoint{35.932364\du}{10.572960\du}}
\pgfpathlineto{\pgfpoint{35.931411\du}{10.579155\du}}
\pgfpathlineto{\pgfpoint{35.930934\du}{10.583921\du}}
\pgfpathlineto{\pgfpoint{35.930458\du}{10.590116\du}}
\pgfpathlineto{\pgfpoint{35.930458\du}{10.595835\du}}
\pgfpathlineto{\pgfpoint{35.930458\du}{10.595835\du}}
\pgfpathlineto{\pgfpoint{35.930458\du}{10.602507\du}}
\pgfpathlineto{\pgfpoint{35.930934\du}{10.607749\du}}
\pgfpathlineto{\pgfpoint{35.931411\du}{10.613468\du}}
\pgfpathlineto{\pgfpoint{35.932364\du}{10.619663\du}}
\pgfpathlineto{\pgfpoint{35.933794\du}{10.625858\du}}
\pgfpathlineto{\pgfpoint{35.935223\du}{10.631101\du}}
\pgfpathlineto{\pgfpoint{35.937130\du}{10.636343\du}}
\pgfpathlineto{\pgfpoint{35.939512\du}{10.642538\du}}
\pgfpathlineto{\pgfpoint{35.942372\du}{10.647304\du}}
\pgfpathlineto{\pgfpoint{35.944278\du}{10.652546\du}}
\pgfpathlineto{\pgfpoint{35.947137\du}{10.657788\du}}
\pgfpathlineto{\pgfpoint{35.950473\du}{10.662554\du}}
\pgfpathlineto{\pgfpoint{35.954286\du}{10.667319\du}}
\pgfpathlineto{\pgfpoint{35.957622\du}{10.671609\du}}
\pgfpathlineto{\pgfpoint{35.961434\du}{10.676374\du}}
\pgfpathlineto{\pgfpoint{35.966200\du}{10.680663\du}}
\pgfpathlineto{\pgfpoint{35.970012\du}{10.684476\du}}
\pgfpathlineto{\pgfpoint{35.974301\du}{10.687812\du}}
\pgfpathlineto{\pgfpoint{35.979067\du}{10.691148\du}}
\pgfpathlineto{\pgfpoint{35.984309\du}{10.694960\du}}
\pgfpathlineto{\pgfpoint{35.989075\du}{10.697820\du}}
\pgfpathlineto{\pgfpoint{35.993841\du}{10.700202\du}}
\pgfpathlineto{\pgfpoint{35.999083\du}{10.703062\du}}
\pgfpathlineto{\pgfpoint{36.005278\du}{10.705445\du}}
\pgfpathlineto{\pgfpoint{36.010997\du}{10.706874\du}}
\pgfpathlineto{\pgfpoint{36.015763\du}{10.708781\du}}
\pgfpathlineto{\pgfpoint{36.021481\du}{10.710210\du}}
\pgfpathlineto{\pgfpoint{36.027200\du}{10.711163\du}}
\pgfpathlineto{\pgfpoint{36.033395\du}{10.711640\du}}
\pgfpathlineto{\pgfpoint{36.038638\du}{10.712116\du}}
\pgfpathlineto{\pgfpoint{36.044356\du}{10.712116\du}}
\pgfpathlineto{\pgfpoint{36.044356\du}{10.712116\du}}
\pgfpathlineto{\pgfpoint{36.050552\du}{10.712116\du}}
\pgfpathlineto{\pgfpoint{36.055794\du}{10.711640\du}}
\pgfpathlineto{\pgfpoint{36.061989\du}{10.711163\du}}
\pgfpathlineto{\pgfpoint{36.067708\du}{10.710210\du}}
\pgfpathlineto{\pgfpoint{36.073427\du}{10.708781\du}}
\pgfpathlineto{\pgfpoint{36.078669\du}{10.706874\du}}
\pgfpathlineto{\pgfpoint{36.084388\du}{10.705445\du}}
\pgfpathlineto{\pgfpoint{36.090106\du}{10.703062\du}}
\pgfpathlineto{\pgfpoint{36.095349\du}{10.700202\du}}
\pgfpathlineto{\pgfpoint{36.100591\du}{10.697820\du}}
\pgfpathlineto{\pgfpoint{36.105356\du}{10.694960\du}}
\pgfpathlineto{\pgfpoint{36.110122\du}{10.691148\du}}
\pgfpathlineto{\pgfpoint{36.115364\du}{10.687812\du}}
\pgfpathlineto{\pgfpoint{36.119177\du}{10.684476\du}}
\pgfpathlineto{\pgfpoint{36.123466\du}{10.680663\du}}
\pgfpathlineto{\pgfpoint{36.127278\du}{10.676374\du}}
\pgfpathlineto{\pgfpoint{36.131567\du}{10.671609\du}}
\pgfpathlineto{\pgfpoint{36.135380\du}{10.667319\du}}
\pgfpathlineto{\pgfpoint{36.139192\du}{10.662554\du}}
\pgfpathlineto{\pgfpoint{36.142052\du}{10.657788\du}}
\pgfpathlineto{\pgfpoint{36.145388\du}{10.652546\du}}
\pgfpathlineto{\pgfpoint{36.147294\du}{10.647304\du}}
\pgfpathlineto{\pgfpoint{36.149677\du}{10.642538\du}}
\pgfpathlineto{\pgfpoint{36.152060\du}{10.636343\du}}
\pgfpathlineto{\pgfpoint{36.153966\du}{10.631101\du}}
\pgfpathlineto{\pgfpoint{36.155396\du}{10.625858\du}}
\pgfpathlineto{\pgfpoint{36.157302\du}{10.619663\du}}
\pgfpathlineto{\pgfpoint{36.157778\du}{10.613468\du}}
\pgfpathlineto{\pgfpoint{36.158255\du}{10.607749\du}}
\pgfpathlineto{\pgfpoint{36.159208\du}{10.602507\du}}
\pgfpathlineto{\pgfpoint{36.159208\du}{10.595835\du}}
\pgfusepath{fill}
\pgfsetlinewidth{0.000000\du}
\pgfsetbuttcap
\pgfsetmiterjoin
\pgfsetdash{}{0pt}
\definecolor{dialinecolor}{rgb}{1.000000, 1.000000, 1.000000}
\pgfsetstrokecolor{dialinecolor}
\pgfpathmoveto{\pgfpoint{35.015456\du}{10.586304\du}}
\pgfpathlineto{\pgfpoint{36.013856\du}{10.586304\du}}
\pgfusepath{stroke}
\pgfsetlinewidth{0.000000\du}
\pgfsetbuttcap
\pgfsetmiterjoin
\pgfsetdash{}{0pt}
\definecolor{dialinecolor}{rgb}{0.788235, 0.788235, 0.713726}
\pgfsetfillcolor{dialinecolor}
\pgfpathmoveto{\pgfpoint{34.599416\du}{7.674024\du}}
\pgfpathlineto{\pgfpoint{34.891073\du}{7.400000\du}}
\pgfpathlineto{\pgfpoint{36.700584\du}{7.400000\du}}
\pgfpathlineto{\pgfpoint{36.410357\du}{7.674024\du}}
\pgfpathlineto{\pgfpoint{34.599416\du}{7.674024\du}}
\pgfusepath{fill}
\pgfsetbuttcap
\pgfsetmiterjoin
\pgfsetdash{}{0pt}
\definecolor{dialinecolor}{rgb}{0.286275, 0.286275, 0.211765}
\pgfsetstrokecolor{dialinecolor}
\pgfpathmoveto{\pgfpoint{34.599416\du}{7.674024\du}}
\pgfpathlineto{\pgfpoint{34.891073\du}{7.400000\du}}
\pgfpathlineto{\pgfpoint{36.700584\du}{7.400000\du}}
\pgfpathlineto{\pgfpoint{36.410357\du}{7.674024\du}}
\pgfpathlineto{\pgfpoint{34.599416\du}{7.674024\du}}
\pgfusepath{stroke}
\pgfsetbuttcap
\pgfsetmiterjoin
\pgfsetdash{}{0pt}
\definecolor{dialinecolor}{rgb}{0.478431, 0.478431, 0.352941}
\pgfsetfillcolor{dialinecolor}
\pgfpathmoveto{\pgfpoint{36.410357\du}{11.050000\du}}
\pgfpathlineto{\pgfpoint{36.700584\du}{10.755007\du}}
\pgfpathlineto{\pgfpoint{36.700584\du}{7.400000\du}}
\pgfpathlineto{\pgfpoint{36.410357\du}{7.674024\du}}
\pgfpathlineto{\pgfpoint{36.410357\du}{11.050000\du}}
\pgfusepath{fill}
\pgfsetbuttcap
\pgfsetmiterjoin
\pgfsetdash{}{0pt}
\definecolor{dialinecolor}{rgb}{0.286275, 0.286275, 0.211765}
\pgfsetstrokecolor{dialinecolor}
\pgfpathmoveto{\pgfpoint{36.410357\du}{11.050000\du}}
\pgfpathlineto{\pgfpoint{36.700584\du}{10.755007\du}}
\pgfpathlineto{\pgfpoint{36.700584\du}{7.400000\du}}
\pgfpathlineto{\pgfpoint{36.410357\du}{7.674024\du}}
\pgfpathlineto{\pgfpoint{36.410357\du}{11.050000\du}}
\pgfusepath{stroke}
% setfont left to latex
\definecolor{dialinecolor}{rgb}{0.000000, 0.000000, 0.000000}
\pgfsetstrokecolor{dialinecolor}
\node[anchor=west] at (16.000000\du,5.650000\du){X};
% setfont left to latex
\definecolor{dialinecolor}{rgb}{0.000000, 0.000000, 0.000000}
\pgfsetstrokecolor{dialinecolor}
\node[anchor=west] at (25.335000\du,5.712500\du){Y};
% setfont left to latex
\definecolor{dialinecolor}{rgb}{0.000000, 0.000000, 0.000000}
\pgfsetstrokecolor{dialinecolor}
\node[anchor=west] at (34.970000\du,5.822500\du){D};
% setfont left to latex
\definecolor{dialinecolor}{rgb}{0.000000, 0.000000, 0.000000}
\pgfsetstrokecolor{dialinecolor}
\node[anchor=west] at (7.205000\du,5.532500\du){P};
\end{tikzpicture}

}
	\caption{Illustration des sauts d'un traceroute avec leurs informations}
	\label{fig:traceroute}
\end{figure}

\subsection{Paramètres de l'algorithme de  détection}

\begin{itemize}
		\item  Objectif : suivre l'évolution du délais d'un lien au cours du temps en suivant son RTT différentiel par période du temps (\textit{timeWindow}).
	\item Entrées : date de début de l'analyse \textit{start}, date de la fin de l'analyse \textit{end}, lien à analyser (\textit{link}) et la fenêtre de l'analyse (\textit{timeWindow}).
	\item Sorties : les dates $d_i$ pendant lesquelles des anomalies ont été détectées.
\end{itemize}

Soient $ d_1 $,$  d_2 $, ..., $ d_N $ les périodes entre \textit{start} et \textit{end} où

\begin{center}
	 $  d_{i+1} $ - $  d_{i} $ = $  d_{j+1} $ - $  d_{j} $ = \textit{step} 
	 
	 \textit{step} est la durée d'une période en secondes, $3600$ pour fenêtre d'une heure.
\end{center}
 
 pour tout $ i $ et $ j $ dans$  [1,N] $
\subsection{Processus de  détection des anomalies }\label{steps-rtt-analysis}

Le processus de  détection des anomalies passe par plusieurs étapes. Ces étapes sont reprises dans  la Figure \ref{fig:process-rttanalysis_tex}. D'abord on  trie les traceroutes à analyser par période $d_i$ (étape 1). Ensuite, on prépare les traceroutes en   appliquant sur les traceroutes d'une période $d_i$ un nombre d'opérations (étapes entre 2 et 6). A la fin de la préparation des traceroutes de toutes les périodes, on ne considère que les données qui concernent le lien à suivre. Ce sont les données qui servent à la comparaison  les  délais d'un lien avec les valeurs de référence (étape 7). Les étapes de détection sont les suivantes:

\paragraph{1. Trie des traceroutes } par période $d_i$. En effet, chaque période $d_i$ est associée à un ensemble de traceroutes ayant été effectués entre $d_i$ et $d_i + step - 1$. 




Les opérations  (2 à 6) concernent  les traceroutes par tout $d_i$.  

\paragraph{2. Vérification de la validité de chaque traceroute} 
Les traceroutes sont filtrés en prenant en considération  les points suivants:
\begin{itemize}
	\item élimination des traceroutes échoués complètement;
	\item élimination des signaux contenant une adresse IP privée;
	\item élimination des signaux qui ne contiennent pas un RTT ou  qui contiennent un RTT négatif;
	\item  élimination des signaux échoués.
\end{itemize}

Il existe deux sortes d'échecs dans un traceroute : échec complet et échec partiel. Dans le premier,   la sonde ne réussit pas à atteindre la destination. Par conséquent, la liste des sauts est vide. Toutefois, dans le deuxième cas, l'échec peut concerner un ou plusieurs saut, ou bien il peut concerner un, deux ou trois signaux d'un saut.


\paragraph{3. Calcul de la médiane des RTTs par saut} Pour tout saut d'un traceroute,  on calcule la médiane des RTTs par adresse IP. Soit le saut $h =\{s_i \}$\footnote{La notation $\{ a\}$ désigne un ensemble d'éléments de type $a$.} où $s$ est un  signal. La médiane des RTTs d'un saut $h$ est  :

mediane\_rtt ($h$) =  $\{median(\{s_i.rtt_{i, j}\})\}$\footnote{La notation $a.b$ désigne la valeur de l'attribut $b$ de l'objet $a$, ainsi $\{a.b\}$ est l'ensemble des $b$ obtenus à partir d'un ensemble d'éléments de type $a$ .}  pour tout signal $s$ ayant la même adresse IP. Autrement dit, le nouveau saut du traceroute est reconstruit en regroupant les signaux par adresse IP ($ from_{i, j} $) et ensuite en calculant la médiane de leurs RTTs ($rtt_{i,j}$). 




\paragraph{4. Inférence des liens topologiques par traceroute.} Un lien topologique est formé par chaque paire de routeurs consécutifs dans un traceroute. De manière générale, la Figure \ref{fig:link-inference} illustre la constitution des liens possibles  dans un traceroute. Soient  RAi, avec $i \in {1,2, ...,N}$,  l'ensemble de routeurs pour le saut A et RBj, avec $j \in \{1,2, ..., M\}$, l'ensemble  de routeurs pour le saut B, avec N et M deux entiers. Les liens  construits sont ceux partant de tout RAi vers tout RBj, où A et B sont deux sauts consécutifs. A l'issue de cette étape, pour tout traceroute, on obtient la liste des liens possibles tout en reprenant des informations générales de la requête traceroute.
\begin{figure}[H]
	\centering
	\captionsetup{justification=centering}
	%\includegraphics[width=0.5\linewidth]{illustrations/link-inference}
	% Graphic for TeX using PGF
% Title: /home/hayat/RipeAtlasTraceroutesAnalysis/report/illustrations/link-inference.dia
% Creator: Dia v0.97+git
% CreationDate: Thu Nov 29 20:58:09 2018
% For: hayat
% \usepackage{tikz}
% The following commands are not supported in PSTricks at present
% We define them conditionally, so when they are implemented,
% this pgf file will use them.
\ifx\du\undefined
  \newlength{\du}
\fi
\setlength{\du}{15\unitlength}
\begin{tikzpicture}[even odd rule]
\pgftransformxscale{1.000000}
\pgftransformyscale{-1.000000}
\definecolor{dialinecolor}{rgb}{0.000000, 0.000000, 0.000000}
\pgfsetstrokecolor{dialinecolor}
\pgfsetstrokeopacity{1.000000}
\definecolor{diafillcolor}{rgb}{1.000000, 1.000000, 1.000000}
\pgfsetfillcolor{diafillcolor}
\pgfsetfillopacity{1.000000}
\pgfsetlinewidth{0.100000\du}
\pgfsetdash{}{0pt}
\pgfsetmiterjoin
\definecolor{diafillcolor}{rgb}{1.000000, 1.000000, 1.000000}
\pgfsetfillcolor{diafillcolor}
\pgfsetfillopacity{1.000000}
\pgfpathellipse{\pgfpoint{27.430614\du}{13.298442\du}}{\pgfpoint{1.349214\du}{0\du}}{\pgfpoint{0\du}{1.217042\du}}
\pgfusepath{fill}
\definecolor{dialinecolor}{rgb}{0.000000, 0.000000, 0.000000}
\pgfsetstrokecolor{dialinecolor}
\pgfsetstrokeopacity{1.000000}
\pgfpathellipse{\pgfpoint{27.430614\du}{13.298442\du}}{\pgfpoint{1.349214\du}{0\du}}{\pgfpoint{0\du}{1.217042\du}}
\pgfusepath{stroke}
% setfont left to latex
\definecolor{dialinecolor}{rgb}{0.000000, 0.000000, 0.000000}
\pgfsetstrokecolor{dialinecolor}
\pgfsetstrokeopacity{1.000000}
\definecolor{diafillcolor}{rgb}{0.000000, 0.000000, 0.000000}
\pgfsetfillcolor{diafillcolor}
\pgfsetfillopacity{1.000000}
\node[anchor=base,inner sep=0pt, outer sep=0pt,color=dialinecolor] at (27.430614\du,13.493442\du){$R_{a,1}$};
\pgfsetlinewidth{0.100000\du}
\pgfsetdash{}{0pt}
\pgfsetmiterjoin
\definecolor{diafillcolor}{rgb}{1.000000, 1.000000, 1.000000}
\pgfsetfillcolor{diafillcolor}
\pgfsetfillopacity{1.000000}
\pgfpathellipse{\pgfpoint{27.475614\du}{17.238442\du}}{\pgfpoint{1.349214\du}{0\du}}{\pgfpoint{0\du}{1.217042\du}}
\pgfusepath{fill}
\definecolor{dialinecolor}{rgb}{0.000000, 0.000000, 0.000000}
\pgfsetstrokecolor{dialinecolor}
\pgfsetstrokeopacity{1.000000}
\pgfpathellipse{\pgfpoint{27.475614\du}{17.238442\du}}{\pgfpoint{1.349214\du}{0\du}}{\pgfpoint{0\du}{1.217042\du}}
\pgfusepath{stroke}
% setfont left to latex
\definecolor{dialinecolor}{rgb}{0.000000, 0.000000, 0.000000}
\pgfsetstrokecolor{dialinecolor}
\pgfsetstrokeopacity{1.000000}
\definecolor{diafillcolor}{rgb}{0.000000, 0.000000, 0.000000}
\pgfsetfillcolor{diafillcolor}
\pgfsetfillopacity{1.000000}
\node[anchor=base,inner sep=0pt, outer sep=0pt,color=dialinecolor] at (27.475614\du,17.433442\du){$R_{a,2}$};
\pgfsetlinewidth{0.100000\du}
\pgfsetdash{}{0pt}
\pgfsetmiterjoin
\definecolor{diafillcolor}{rgb}{1.000000, 1.000000, 1.000000}
\pgfsetfillcolor{diafillcolor}
\pgfsetfillopacity{1.000000}
\pgfpathellipse{\pgfpoint{27.570578\du}{23.578496\du}}{\pgfpoint{1.376878\du}{0\du}}{\pgfpoint{0\du}{1.241996\du}}
\pgfusepath{fill}
\definecolor{dialinecolor}{rgb}{0.000000, 0.000000, 0.000000}
\pgfsetstrokecolor{dialinecolor}
\pgfsetstrokeopacity{1.000000}
\pgfpathellipse{\pgfpoint{27.570578\du}{23.578496\du}}{\pgfpoint{1.376878\du}{0\du}}{\pgfpoint{0\du}{1.241996\du}}
\pgfusepath{stroke}
% setfont left to latex
\definecolor{dialinecolor}{rgb}{0.000000, 0.000000, 0.000000}
\pgfsetstrokecolor{dialinecolor}
\pgfsetstrokeopacity{1.000000}
\definecolor{diafillcolor}{rgb}{0.000000, 0.000000, 0.000000}
\pgfsetfillcolor{diafillcolor}
\pgfsetfillopacity{1.000000}
\node[anchor=base,inner sep=0pt, outer sep=0pt,color=dialinecolor] at (27.570578\du,23.773496\du){$R_{a,N}$};
\pgfsetlinewidth{0.300000\du}
\pgfsetdash{{\pgflinewidth}{0.200000\du}}{0cm}
\pgfsetbuttcap
{
\definecolor{diafillcolor}{rgb}{0.000000, 0.000000, 0.000000}
\pgfsetfillcolor{diafillcolor}
\pgfsetfillopacity{1.000000}
% was here!!!
\definecolor{dialinecolor}{rgb}{0.000000, 0.000000, 0.000000}
\pgfsetstrokecolor{dialinecolor}
\pgfsetstrokeopacity{1.000000}
\draw (27.540000\du,18.980000\du)--(27.540000\du,21.480000\du);
}
\pgfsetlinewidth{0.100000\du}
\pgfsetdash{}{0pt}
\pgfsetmiterjoin
\definecolor{diafillcolor}{rgb}{1.000000, 1.000000, 1.000000}
\pgfsetfillcolor{diafillcolor}
\pgfsetfillopacity{1.000000}
\pgfpathellipse{\pgfpoint{37.980536\du}{13.298484\du}}{\pgfpoint{1.355136\du}{0\du}}{\pgfpoint{0\du}{1.222384\du}}
\pgfusepath{fill}
\definecolor{dialinecolor}{rgb}{0.000000, 0.000000, 0.000000}
\pgfsetstrokecolor{dialinecolor}
\pgfsetstrokeopacity{1.000000}
\pgfpathellipse{\pgfpoint{37.980536\du}{13.298484\du}}{\pgfpoint{1.355136\du}{0\du}}{\pgfpoint{0\du}{1.222384\du}}
\pgfusepath{stroke}
% setfont left to latex
\definecolor{dialinecolor}{rgb}{0.000000, 0.000000, 0.000000}
\pgfsetstrokecolor{dialinecolor}
\pgfsetstrokeopacity{1.000000}
\definecolor{diafillcolor}{rgb}{0.000000, 0.000000, 0.000000}
\pgfsetfillcolor{diafillcolor}
\pgfsetfillopacity{1.000000}
\node[anchor=base,inner sep=0pt, outer sep=0pt,color=dialinecolor] at (37.980536\du,13.493484\du){$R_{b,1}$};
\pgfsetlinewidth{0.100000\du}
\pgfsetdash{}{0pt}
\pgfsetmiterjoin
\definecolor{diafillcolor}{rgb}{1.000000, 1.000000, 1.000000}
\pgfsetfillcolor{diafillcolor}
\pgfsetfillopacity{1.000000}
\pgfpathellipse{\pgfpoint{37.925536\du}{17.238484\du}}{\pgfpoint{1.355136\du}{0\du}}{\pgfpoint{0\du}{1.222384\du}}
\pgfusepath{fill}
\definecolor{dialinecolor}{rgb}{0.000000, 0.000000, 0.000000}
\pgfsetstrokecolor{dialinecolor}
\pgfsetstrokeopacity{1.000000}
\pgfpathellipse{\pgfpoint{37.925536\du}{17.238484\du}}{\pgfpoint{1.355136\du}{0\du}}{\pgfpoint{0\du}{1.222384\du}}
\pgfusepath{stroke}
% setfont left to latex
\definecolor{dialinecolor}{rgb}{0.000000, 0.000000, 0.000000}
\pgfsetstrokecolor{dialinecolor}
\pgfsetstrokeopacity{1.000000}
\definecolor{diafillcolor}{rgb}{0.000000, 0.000000, 0.000000}
\pgfsetfillcolor{diafillcolor}
\pgfsetfillopacity{1.000000}
\node[anchor=base,inner sep=0pt, outer sep=0pt,color=dialinecolor] at (37.925536\du,17.433484\du){$R_{b,2}$};
\pgfsetlinewidth{0.100000\du}
\pgfsetdash{}{0pt}
\pgfsetmiterjoin
\definecolor{diafillcolor}{rgb}{1.000000, 1.000000, 1.000000}
\pgfsetfillcolor{diafillcolor}
\pgfsetfillopacity{1.000000}
\pgfpathellipse{\pgfpoint{38.170599\du}{23.478445\du}}{\pgfpoint{1.411299\du}{0\du}}{\pgfpoint{0\du}{1.273045\du}}
\pgfusepath{fill}
\definecolor{dialinecolor}{rgb}{0.000000, 0.000000, 0.000000}
\pgfsetstrokecolor{dialinecolor}
\pgfsetstrokeopacity{1.000000}
\pgfpathellipse{\pgfpoint{38.170599\du}{23.478445\du}}{\pgfpoint{1.411299\du}{0\du}}{\pgfpoint{0\du}{1.273045\du}}
\pgfusepath{stroke}
% setfont left to latex
\definecolor{dialinecolor}{rgb}{0.000000, 0.000000, 0.000000}
\pgfsetstrokecolor{dialinecolor}
\pgfsetstrokeopacity{1.000000}
\definecolor{diafillcolor}{rgb}{0.000000, 0.000000, 0.000000}
\pgfsetfillcolor{diafillcolor}
\pgfsetfillopacity{1.000000}
\node[anchor=base,inner sep=0pt, outer sep=0pt,color=dialinecolor] at (38.170599\du,23.673445\du){$R_{b,M}$};
\pgfsetlinewidth{0.300000\du}
\pgfsetdash{{\pgflinewidth}{0.200000\du}}{0cm}
\pgfsetbuttcap
{
\definecolor{diafillcolor}{rgb}{0.000000, 0.000000, 0.000000}
\pgfsetfillcolor{diafillcolor}
\pgfsetfillopacity{1.000000}
% was here!!!
\definecolor{dialinecolor}{rgb}{0.000000, 0.000000, 0.000000}
\pgfsetstrokecolor{dialinecolor}
\pgfsetstrokeopacity{1.000000}
\draw (37.985000\du,18.970000\du)--(37.985000\du,21.470000\du);
}
\pgfsetlinewidth{0.050000\du}
\pgfsetdash{}{0pt}
\pgfsetbuttcap
{
\definecolor{diafillcolor}{rgb}{0.000000, 1.000000, 0.000000}
\pgfsetfillcolor{diafillcolor}
\pgfsetfillopacity{1.000000}
% was here!!!
\pgfsetarrowsstart{stealth}
\definecolor{dialinecolor}{rgb}{0.000000, 1.000000, 0.000000}
\pgfsetstrokecolor{dialinecolor}
\pgfsetstrokeopacity{1.000000}
\draw (28.384600\du,12.437900\du)--(37.022310\du,12.434128\du);
}
\pgfsetlinewidth{0.050000\du}
\pgfsetdash{}{0pt}
\pgfsetbuttcap
{
\definecolor{diafillcolor}{rgb}{1.000000, 0.000000, 0.000000}
\pgfsetfillcolor{diafillcolor}
\pgfsetfillopacity{1.000000}
% was here!!!
\pgfsetarrowsstart{stealth}
\definecolor{dialinecolor}{rgb}{1.000000, 0.000000, 0.000000}
\pgfsetstrokecolor{dialinecolor}
\pgfsetstrokeopacity{1.000000}
\draw (28.842600\du,24.053700\du)--(37.630518\du,24.654584\du);
}
\pgfsetlinewidth{0.100000\du}
\pgfsetdash{{\pgflinewidth}{0.200000\du}}{0cm}
\pgfsetmiterjoin
\pgfsetbuttcap
{\pgfsetcornersarced{\pgfpoint{0.000000\du}{0.000000\du}}\definecolor{dialinecolor}{rgb}{0.000000, 0.000000, 0.000000}
\pgfsetstrokecolor{dialinecolor}
\pgfsetstrokeopacity{1.000000}
\draw (25.638764\du,11.830038\du)--(25.638764\du,25.156536\du)--(29.721269\du,25.156536\du)--(29.721269\du,11.830038\du)--cycle;
}\pgfsetlinewidth{0.100000\du}
\pgfsetdash{{\pgflinewidth}{0.200000\du}}{0cm}
\pgfsetmiterjoin
\pgfsetbuttcap
{\pgfsetcornersarced{\pgfpoint{0.000000\du}{0.000000\du}}\definecolor{dialinecolor}{rgb}{0.000000, 0.000000, 0.000000}
\pgfsetstrokecolor{dialinecolor}
\pgfsetstrokeopacity{1.000000}
\draw (36.096688\du,11.857262\du)--(36.096688\du,25.128574\du)--(39.871607\du,25.128574\du)--(39.871607\du,11.857262\du)--cycle;
}% setfont left to latex
\definecolor{dialinecolor}{rgb}{0.000000, 0.000000, 0.000000}
\pgfsetstrokecolor{dialinecolor}
\pgfsetstrokeopacity{1.000000}
\definecolor{diafillcolor}{rgb}{0.000000, 0.000000, 0.000000}
\pgfsetfillcolor{diafillcolor}
\pgfsetfillopacity{1.000000}
\node[anchor=base west,inner sep=0pt,outer sep=0pt,color=dialinecolor] at (26.470600\du,11.464400\du){Saut a};
% setfont left to latex
\definecolor{dialinecolor}{rgb}{0.000000, 0.000000, 0.000000}
\pgfsetstrokecolor{dialinecolor}
\pgfsetstrokeopacity{1.000000}
\definecolor{diafillcolor}{rgb}{0.000000, 0.000000, 0.000000}
\pgfsetfillcolor{diafillcolor}
\pgfsetfillopacity{1.000000}
\node[anchor=base west,inner sep=0pt,outer sep=0pt,color=dialinecolor] at (36.898700\du,11.464100\du){Saut b};
\pgfsetlinewidth{0.050000\du}
\pgfsetdash{}{0pt}
\pgfsetbuttcap
{
\definecolor{diafillcolor}{rgb}{0.000000, 0.000000, 1.000000}
\pgfsetfillcolor{diafillcolor}
\pgfsetfillopacity{1.000000}
% was here!!!
\pgfsetarrowsstart{stealth}
\definecolor{dialinecolor}{rgb}{0.000000, 0.000000, 1.000000}
\pgfsetstrokecolor{dialinecolor}
\pgfsetstrokeopacity{1.000000}
\draw (28.779828\du,13.298442\du)--(37.406948\du,16.109148\du);
}
\pgfsetlinewidth{0.050000\du}
\pgfsetdash{}{0pt}
\pgfsetbuttcap
{
\definecolor{diafillcolor}{rgb}{0.000000, 1.000000, 0.000000}
\pgfsetfillcolor{diafillcolor}
\pgfsetfillopacity{1.000000}
% was here!!!
\pgfsetarrowsstart{stealth}
\definecolor{dialinecolor}{rgb}{0.000000, 1.000000, 0.000000}
\pgfsetstrokecolor{dialinecolor}
\pgfsetstrokeopacity{1.000000}
\draw (28.722125\du,16.772700\du)--(36.728554\du,12.830698\du);
}
\pgfsetlinewidth{0.050000\du}
\pgfsetdash{}{0pt}
\pgfsetbuttcap
{
\definecolor{diafillcolor}{rgb}{0.000000, 1.000000, 0.000000}
\pgfsetfillcolor{diafillcolor}
\pgfsetfillopacity{1.000000}
% was here!!!
\pgfsetarrowsstart{stealth}
\definecolor{dialinecolor}{rgb}{0.000000, 1.000000, 0.000000}
\pgfsetstrokecolor{dialinecolor}
\pgfsetstrokeopacity{1.000000}
\draw (28.602774\du,20.095348\du)--(36.625400\du,13.298484\du);
}
\pgfsetlinewidth{0.050000\du}
\pgfsetdash{}{0pt}
\pgfsetbuttcap
{
\definecolor{diafillcolor}{rgb}{0.000000, 1.000000, 0.000000}
\pgfsetfillcolor{diafillcolor}
\pgfsetfillopacity{1.000000}
% was here!!!
\pgfsetarrowsstart{stealth}
\definecolor{dialinecolor}{rgb}{0.000000, 1.000000, 0.000000}
\pgfsetstrokecolor{dialinecolor}
\pgfsetstrokeopacity{1.000000}
\draw (28.097487\du,22.431041\du)--(36.728554\du,13.766270\du);
}
\pgfsetlinewidth{0.050000\du}
\pgfsetdash{}{0pt}
\pgfsetbuttcap
{
\definecolor{diafillcolor}{rgb}{0.000000, 0.000000, 1.000000}
\pgfsetfillcolor{diafillcolor}
\pgfsetfillopacity{1.000000}
% was here!!!
\pgfsetarrowsstart{stealth}
\definecolor{dialinecolor}{rgb}{0.000000, 0.000000, 1.000000}
\pgfsetstrokecolor{dialinecolor}
\pgfsetstrokeopacity{1.000000}
\draw (28.824828\du,17.238442\du)--(36.967310\du,16.374128\du);
}
\pgfsetlinewidth{0.050000\du}
\pgfsetdash{}{0pt}
\pgfsetbuttcap
{
\definecolor{diafillcolor}{rgb}{0.000000, 0.000000, 1.000000}
\pgfsetfillcolor{diafillcolor}
\pgfsetfillopacity{1.000000}
% was here!!!
\pgfsetarrowsstart{stealth}
\definecolor{dialinecolor}{rgb}{0.000000, 0.000000, 1.000000}
\pgfsetstrokecolor{dialinecolor}
\pgfsetstrokeopacity{1.000000}
\draw (28.798511\du,20.486821\du)--(36.673554\du,16.770698\du);
}
\pgfsetlinewidth{0.050000\du}
\pgfsetdash{}{0pt}
\pgfsetbuttcap
{
\definecolor{diafillcolor}{rgb}{0.000000, 0.000000, 1.000000}
\pgfsetfillcolor{diafillcolor}
\pgfsetfillopacity{1.000000}
% was here!!!
\pgfsetarrowsstart{stealth}
\definecolor{dialinecolor}{rgb}{0.000000, 0.000000, 1.000000}
\pgfsetstrokecolor{dialinecolor}
\pgfsetstrokeopacity{1.000000}
\draw (28.544178\du,22.700272\du)--(36.516123\du,17.355037\du);
}
\pgfsetlinewidth{0.050000\du}
\pgfsetdash{}{0pt}
\pgfsetbuttcap
{
\definecolor{diafillcolor}{rgb}{1.000000, 0.000000, 0.000000}
\pgfsetfillcolor{diafillcolor}
\pgfsetfillopacity{1.000000}
% was here!!!
\pgfsetarrowsstart{stealth}
\definecolor{dialinecolor}{rgb}{1.000000, 0.000000, 0.000000}
\pgfsetstrokecolor{dialinecolor}
\pgfsetstrokeopacity{1.000000}
\draw (28.938323\du,20.850332\du)--(36.866729\du,23.965618\du);
}
\pgfsetlinewidth{0.050000\du}
\pgfsetdash{}{0pt}
\pgfsetbuttcap
{
\definecolor{diafillcolor}{rgb}{1.000000, 0.000000, 0.000000}
\pgfsetfillcolor{diafillcolor}
\pgfsetfillopacity{1.000000}
% was here!!!
\pgfsetarrowsstart{stealth}
\definecolor{dialinecolor}{rgb}{1.000000, 0.000000, 0.000000}
\pgfsetstrokecolor{dialinecolor}
\pgfsetstrokeopacity{1.000000}
\draw (28.722125\du,17.704183\du)--(36.712688\du,23.381227\du);
}
\pgfsetlinewidth{0.050000\du}
\pgfsetdash{}{0pt}
\pgfsetbuttcap
{
\definecolor{diafillcolor}{rgb}{1.000000, 0.000000, 0.000000}
\pgfsetfillcolor{diafillcolor}
\pgfsetfillopacity{1.000000}
% was here!!!
\pgfsetarrowsstart{stealth}
\definecolor{dialinecolor}{rgb}{1.000000, 0.000000, 0.000000}
\pgfsetstrokecolor{dialinecolor}
\pgfsetstrokeopacity{1.000000}
\draw (28.677125\du,13.764183\du)--(36.866729\du,22.991272\du);
}
\end{tikzpicture}

	\caption{Inférence des liens possibles entre les routeurs des deux sauts consécutifs RAi et RBj}
	\label{fig:link-inference}
\end{figure}
\paragraph{5. Caractérisation des liens} avec leurs RTTs différentiels. A cette étape, on calcule le RTT différentiel d'un lien en calculant la différence entre les RTTs des deux routeurs du lien en question. En plus du RTT différentiel, on note aussi la sonde Atlas ayant effectué la requête traceroute où le lien a été identifié. 

\paragraph{6. Fusion des informations d'un lien} Etant donné qu'un lien (IP1, IP2) peut être identifié plusieurs fois pendant une même période $d_i$ d'une part, et le lien (IP2, IP1) est similaire\footnote{La similarité est mesurée par le RTT différentiel.} au lien  (IP1, IP2) d'autre part, la fusion permet de construire une nouvelle distribution des RTTs différentiels caractérisant le lien (IP1, IP2) qui reprend les RTTs différentiels du lien(IP1, IP2) ainsi que ceux du lien (IP2, IP1).


A la fin de l'étape 6, tous les traceroutes sont préparés. A présent, l'objectif est d'identifier les dates pendant lesquelles des anomalies ont été détectées. Pour ce faire, l'idée du travail de référence est de conserver, pour un lien donné, une référence du RTT différentiel médian.  Cette référence est d'abord comparée avec la médiane courante du RTT différentiel,  puis,  mise à jour,   tout au long de la période d'analyse.

  
  \paragraph{7. Calcul de la médiane des RTTs différentiels et   l'intervalle de confiance courant} du lien analysé. Pour un lien donné, on calcule la médiane des RTTs différentiels d'une $d_i$, ensuite on calcule les deux bornes de l'intervalle de confiance pour $d_i$.
  
 
  
  
  \paragraph{8. Mise à jour de la médiane et de l'intervalle de  référence du lien analysé.} La médiane des RTTs différentiels de référence sont d'abords comparés avec ceux de la période $d_i$ courante. Ensuite, ces références sont mises à jour pour prendre en compte ces nouvelles valeurs. A l'issue de cette comparaison, la liste des dates des anomalies est mise à jour.
  

 \paragraph{9. Comparaison des intervalles de confiance} 

La comparaison de l'état courant du lien avec celui de référence est effectuée en analysant le chevauchement d'intervalles de confiance  courant et de référence. Le délai d'un lien est jugé anormal si son intervalle de confiance courant est inclus dans l'intervalle de confiance de référence. C'est le cas 1 dans la Figure 	\ref{fig:intervals-comparaison},  \textit{referenceLow} et \textit{referenceHight} sont les bornes de l'intervalle de confiance de référence.  \textit{currentLow} et \textit{currentHight} sont les bornes de l'intervalle de confiance courant.

D'après le travail de référence, on distingue quatre cas possibles  illustrés dans la Figure 	\ref{fig:intervals-comparaison}:

\textbf{Cas 1 :} le délai du lien est normal.

\textbf{Cas 2 :} le délai du lien est anormal.

\textbf{Cas 3 :} le délai du lien est anormal.

\textbf{Cas 4 :} le délai du lien est anormal.

Dans le cas où le délai est jugé anormal, on introduit ce qu'on appelle la \textit{déviation}. Cette métrique caractérise l'anomalie détectée. Elle est calculée différemment dans le cas où le délai est anormal.

\begin{figure}[h]
	\centering
	\captionsetup{justification=centering}
	\resizebox{\textwidth}{!}{
		% Graphic for TeX using PGF
% Title: /home/hayat/RipeAtlasTraceroutesAnalysis/report/illustrations/intervals-comparaison.dia
% Creator: Dia v0.97+git
% CreationDate: Tue Dec  4 13:30:20 2018
% For: hayat
% \usepackage{tikz}
% The following commands are not supported in PSTricks at present
% We define them conditionally, so when they are implemented,
% this pgf file will use them.
\ifx\du\undefined
  \newlength{\du}
\fi
\setlength{\du}{15\unitlength}
\begin{tikzpicture}[even odd rule]
\pgftransformxscale{1.000000}
\pgftransformyscale{-1.000000}
\definecolor{dialinecolor}{rgb}{0.000000, 0.000000, 0.000000}
\pgfsetstrokecolor{dialinecolor}
\pgfsetstrokeopacity{1.000000}
\definecolor{diafillcolor}{rgb}{1.000000, 1.000000, 1.000000}
\pgfsetfillcolor{diafillcolor}
\pgfsetfillopacity{1.000000}
\pgfsetlinewidth{0.100000\du}
\pgfsetdash{}{0pt}
\pgfsetbuttcap
{
\definecolor{diafillcolor}{rgb}{0.662745, 0.709804, 0.674510}
\pgfsetfillcolor{diafillcolor}
\pgfsetfillopacity{1.000000}
% was here!!!
}
\definecolor{dialinecolor}{rgb}{0.662745, 0.709804, 0.674510}
\pgfsetstrokecolor{dialinecolor}
\pgfsetstrokeopacity{1.000000}
\draw (14.150000\du,0.400000\du)--(27.000000\du,0.400000\du);
\pgfsetlinewidth{0.100000\du}
\pgfsetdash{}{0pt}
\pgfsetmiterjoin
\pgfsetbuttcap
\definecolor{diafillcolor}{rgb}{0.662745, 0.709804, 0.674510}
\pgfsetfillcolor{diafillcolor}
\pgfsetfillopacity{1.000000}
\fill (13.900000\du,0.550000\du)--(13.900000\du,0.250000\du)--(14.200000\du,0.250000\du)--(14.200000\du,0.550000\du)--cycle;
\definecolor{dialinecolor}{rgb}{0.662745, 0.709804, 0.674510}
\pgfsetstrokecolor{dialinecolor}
\pgfsetstrokeopacity{1.000000}
\draw (14.050000\du,0.700000\du)--(14.050000\du,0.100000\du);
\pgfsetlinewidth{0.100000\du}
\pgfsetdash{}{0pt}
\pgfsetmiterjoin
\pgfsetbuttcap
\definecolor{diafillcolor}{rgb}{0.662745, 0.709804, 0.674510}
\pgfsetfillcolor{diafillcolor}
\pgfsetfillopacity{1.000000}
\fill (27.250000\du,0.250000\du)--(27.250000\du,0.550000\du)--(26.950000\du,0.550000\du)--(26.950000\du,0.250000\du)--cycle;
\definecolor{dialinecolor}{rgb}{0.662745, 0.709804, 0.674510}
\pgfsetstrokecolor{dialinecolor}
\pgfsetstrokeopacity{1.000000}
\draw (27.100000\du,0.100000\du)--(27.100000\du,0.700000\du);
\pgfsetlinewidth{0.100000\du}
\pgfsetdash{}{0pt}
\pgfsetbuttcap
{
\definecolor{diafillcolor}{rgb}{0.000000, 0.000000, 0.000000}
\pgfsetfillcolor{diafillcolor}
\pgfsetfillopacity{1.000000}
% was here!!!
}
\definecolor{dialinecolor}{rgb}{0.000000, 0.000000, 0.000000}
\pgfsetstrokecolor{dialinecolor}
\pgfsetstrokeopacity{1.000000}
\draw (16.150000\du,2.450000\du)--(24.800000\du,2.450000\du);
\pgfsetlinewidth{0.100000\du}
\pgfsetdash{}{0pt}
\pgfsetmiterjoin
\pgfsetbuttcap
\definecolor{diafillcolor}{rgb}{0.000000, 0.000000, 0.000000}
\pgfsetfillcolor{diafillcolor}
\pgfsetfillopacity{1.000000}
\fill (15.900000\du,2.600000\du)--(15.900000\du,2.300000\du)--(16.200000\du,2.300000\du)--(16.200000\du,2.600000\du)--cycle;
\definecolor{dialinecolor}{rgb}{0.000000, 0.000000, 0.000000}
\pgfsetstrokecolor{dialinecolor}
\pgfsetstrokeopacity{1.000000}
\draw (16.050000\du,2.750000\du)--(16.050000\du,2.150000\du);
\pgfsetlinewidth{0.100000\du}
\pgfsetdash{}{0pt}
\pgfsetmiterjoin
\pgfsetbuttcap
\definecolor{diafillcolor}{rgb}{0.000000, 0.000000, 0.000000}
\pgfsetfillcolor{diafillcolor}
\pgfsetfillopacity{1.000000}
\fill (25.050000\du,2.300000\du)--(25.050000\du,2.600000\du)--(24.750000\du,2.600000\du)--(24.750000\du,2.300000\du)--cycle;
\definecolor{dialinecolor}{rgb}{0.000000, 0.000000, 0.000000}
\pgfsetstrokecolor{dialinecolor}
\pgfsetstrokeopacity{1.000000}
\draw (24.900000\du,2.150000\du)--(24.900000\du,2.750000\du);
\pgfsetlinewidth{0.100000\du}
\pgfsetdash{}{0pt}
\pgfsetbuttcap
{
\definecolor{diafillcolor}{rgb}{0.662745, 0.709804, 0.674510}
\pgfsetfillcolor{diafillcolor}
\pgfsetfillopacity{1.000000}
% was here!!!
}
\definecolor{dialinecolor}{rgb}{0.662745, 0.709804, 0.674510}
\pgfsetstrokecolor{dialinecolor}
\pgfsetstrokeopacity{1.000000}
\draw (14.008246\du,11.548114\du)--(26.858246\du,11.548114\du);
\pgfsetlinewidth{0.100000\du}
\pgfsetdash{}{0pt}
\pgfsetmiterjoin
\pgfsetbuttcap
\definecolor{diafillcolor}{rgb}{0.662745, 0.709804, 0.674510}
\pgfsetfillcolor{diafillcolor}
\pgfsetfillopacity{1.000000}
\fill (13.758246\du,11.698114\du)--(13.758246\du,11.398114\du)--(14.058246\du,11.398114\du)--(14.058246\du,11.698114\du)--cycle;
\definecolor{dialinecolor}{rgb}{0.662745, 0.709804, 0.674510}
\pgfsetstrokecolor{dialinecolor}
\pgfsetstrokeopacity{1.000000}
\draw (13.908246\du,11.848114\du)--(13.908246\du,11.248114\du);
\pgfsetlinewidth{0.100000\du}
\pgfsetdash{}{0pt}
\pgfsetmiterjoin
\pgfsetbuttcap
\definecolor{diafillcolor}{rgb}{0.662745, 0.709804, 0.674510}
\pgfsetfillcolor{diafillcolor}
\pgfsetfillopacity{1.000000}
\fill (27.108246\du,11.398114\du)--(27.108246\du,11.698114\du)--(26.808246\du,11.698114\du)--(26.808246\du,11.398114\du)--cycle;
\definecolor{dialinecolor}{rgb}{0.662745, 0.709804, 0.674510}
\pgfsetstrokecolor{dialinecolor}
\pgfsetstrokeopacity{1.000000}
\draw (26.958246\du,11.248114\du)--(26.958246\du,11.848114\du);
\pgfsetlinewidth{0.100000\du}
\pgfsetdash{}{0pt}
\pgfsetbuttcap
{
\definecolor{diafillcolor}{rgb}{0.000000, 0.000000, 0.000000}
\pgfsetfillcolor{diafillcolor}
\pgfsetfillopacity{1.000000}
% was here!!!
}
\definecolor{dialinecolor}{rgb}{0.000000, 0.000000, 0.000000}
\pgfsetstrokecolor{dialinecolor}
\pgfsetstrokeopacity{1.000000}
\draw (23.108246\du,13.598114\du)--(31.758246\du,13.598114\du);
\pgfsetlinewidth{0.100000\du}
\pgfsetdash{}{0pt}
\pgfsetmiterjoin
\pgfsetbuttcap
\definecolor{diafillcolor}{rgb}{0.000000, 0.000000, 0.000000}
\pgfsetfillcolor{diafillcolor}
\pgfsetfillopacity{1.000000}
\fill (22.858246\du,13.748114\du)--(22.858246\du,13.448114\du)--(23.158246\du,13.448114\du)--(23.158246\du,13.748114\du)--cycle;
\definecolor{dialinecolor}{rgb}{0.000000, 0.000000, 0.000000}
\pgfsetstrokecolor{dialinecolor}
\pgfsetstrokeopacity{1.000000}
\draw (23.008246\du,13.898114\du)--(23.008246\du,13.298114\du);
\pgfsetlinewidth{0.100000\du}
\pgfsetdash{}{0pt}
\pgfsetmiterjoin
\pgfsetbuttcap
\definecolor{diafillcolor}{rgb}{0.000000, 0.000000, 0.000000}
\pgfsetfillcolor{diafillcolor}
\pgfsetfillopacity{1.000000}
\fill (32.008246\du,13.448114\du)--(32.008246\du,13.748114\du)--(31.708246\du,13.748114\du)--(31.708246\du,13.448114\du)--cycle;
\definecolor{dialinecolor}{rgb}{0.000000, 0.000000, 0.000000}
\pgfsetstrokecolor{dialinecolor}
\pgfsetstrokeopacity{1.000000}
\draw (31.858246\du,13.298114\du)--(31.858246\du,13.898114\du);
\pgfsetlinewidth{0.100000\du}
\pgfsetdash{}{0pt}
\pgfsetbuttcap
{
\definecolor{diafillcolor}{rgb}{0.662745, 0.709804, 0.674510}
\pgfsetfillcolor{diafillcolor}
\pgfsetfillopacity{1.000000}
% was here!!!
}
\definecolor{dialinecolor}{rgb}{0.662745, 0.709804, 0.674510}
\pgfsetstrokecolor{dialinecolor}
\pgfsetstrokeopacity{1.000000}
\draw (14.108246\du,17.048114\du)--(26.958246\du,17.048114\du);
\pgfsetlinewidth{0.100000\du}
\pgfsetdash{}{0pt}
\pgfsetmiterjoin
\pgfsetbuttcap
\definecolor{diafillcolor}{rgb}{0.662745, 0.709804, 0.674510}
\pgfsetfillcolor{diafillcolor}
\pgfsetfillopacity{1.000000}
\fill (13.858246\du,17.198114\du)--(13.858246\du,16.898114\du)--(14.158246\du,16.898114\du)--(14.158246\du,17.198114\du)--cycle;
\definecolor{dialinecolor}{rgb}{0.662745, 0.709804, 0.674510}
\pgfsetstrokecolor{dialinecolor}
\pgfsetstrokeopacity{1.000000}
\draw (14.008246\du,17.348114\du)--(14.008246\du,16.748114\du);
\pgfsetlinewidth{0.100000\du}
\pgfsetdash{}{0pt}
\pgfsetmiterjoin
\pgfsetbuttcap
\definecolor{diafillcolor}{rgb}{0.662745, 0.709804, 0.674510}
\pgfsetfillcolor{diafillcolor}
\pgfsetfillopacity{1.000000}
\fill (27.208246\du,16.898114\du)--(27.208246\du,17.198114\du)--(26.908246\du,17.198114\du)--(26.908246\du,16.898114\du)--cycle;
\definecolor{dialinecolor}{rgb}{0.662745, 0.709804, 0.674510}
\pgfsetstrokecolor{dialinecolor}
\pgfsetstrokeopacity{1.000000}
\draw (27.058246\du,16.748114\du)--(27.058246\du,17.348114\du);
\pgfsetlinewidth{0.100000\du}
\pgfsetdash{}{0pt}
\pgfsetbuttcap
{
\definecolor{diafillcolor}{rgb}{0.000000, 0.000000, 0.000000}
\pgfsetfillcolor{diafillcolor}
\pgfsetfillopacity{1.000000}
% was here!!!
}
\definecolor{dialinecolor}{rgb}{0.000000, 0.000000, 0.000000}
\pgfsetstrokecolor{dialinecolor}
\pgfsetstrokeopacity{1.000000}
\draw (9.308246\du,19.048114\du)--(17.958246\du,19.048114\du);
\pgfsetlinewidth{0.100000\du}
\pgfsetdash{}{0pt}
\pgfsetmiterjoin
\pgfsetbuttcap
\definecolor{diafillcolor}{rgb}{0.000000, 0.000000, 0.000000}
\pgfsetfillcolor{diafillcolor}
\pgfsetfillopacity{1.000000}
\fill (9.058246\du,19.198114\du)--(9.058246\du,18.898114\du)--(9.358246\du,18.898114\du)--(9.358246\du,19.198114\du)--cycle;
\definecolor{dialinecolor}{rgb}{0.000000, 0.000000, 0.000000}
\pgfsetstrokecolor{dialinecolor}
\pgfsetstrokeopacity{1.000000}
\draw (9.208246\du,19.348114\du)--(9.208246\du,18.748114\du);
\pgfsetlinewidth{0.100000\du}
\pgfsetdash{}{0pt}
\pgfsetmiterjoin
\pgfsetbuttcap
\definecolor{diafillcolor}{rgb}{0.000000, 0.000000, 0.000000}
\pgfsetfillcolor{diafillcolor}
\pgfsetfillopacity{1.000000}
\fill (18.208246\du,18.898114\du)--(18.208246\du,19.198114\du)--(17.908246\du,19.198114\du)--(17.908246\du,18.898114\du)--cycle;
\definecolor{dialinecolor}{rgb}{0.000000, 0.000000, 0.000000}
\pgfsetstrokecolor{dialinecolor}
\pgfsetstrokeopacity{1.000000}
\draw (18.058246\du,18.748114\du)--(18.058246\du,19.348114\du);
\pgfsetlinewidth{0.100000\du}
\pgfsetdash{{\pgflinewidth}{0.200000\du}}{0cm}
\pgfsetbuttcap
{
\definecolor{diafillcolor}{rgb}{0.000000, 0.000000, 0.000000}
\pgfsetfillcolor{diafillcolor}
\pgfsetfillopacity{1.000000}
% was here!!!
\pgfsetarrowsend{stealth}
\definecolor{dialinecolor}{rgb}{0.000000, 0.000000, 0.000000}
\pgfsetstrokecolor{dialinecolor}
\pgfsetstrokeopacity{1.000000}
\draw (5.050000\du,21.000000\du)--(38.800000\du,20.950000\du);
}
\pgfsetlinewidth{0.100000\du}
\pgfsetdash{}{0pt}
\pgfsetbuttcap
{
\definecolor{diafillcolor}{rgb}{0.662745, 0.709804, 0.674510}
\pgfsetfillcolor{diafillcolor}
\pgfsetfillopacity{1.000000}
% was here!!!
}
\definecolor{dialinecolor}{rgb}{0.662745, 0.709804, 0.674510}
\pgfsetstrokecolor{dialinecolor}
\pgfsetstrokeopacity{1.000000}
\draw (13.958246\du,6.048114\du)--(26.808246\du,6.048114\du);
\pgfsetlinewidth{0.100000\du}
\pgfsetdash{}{0pt}
\pgfsetmiterjoin
\pgfsetbuttcap
\definecolor{diafillcolor}{rgb}{0.662745, 0.709804, 0.674510}
\pgfsetfillcolor{diafillcolor}
\pgfsetfillopacity{1.000000}
\fill (13.708246\du,6.198114\du)--(13.708246\du,5.898114\du)--(14.008246\du,5.898114\du)--(14.008246\du,6.198114\du)--cycle;
\definecolor{dialinecolor}{rgb}{0.662745, 0.709804, 0.674510}
\pgfsetstrokecolor{dialinecolor}
\pgfsetstrokeopacity{1.000000}
\draw (13.858246\du,6.348114\du)--(13.858246\du,5.748114\du);
\pgfsetlinewidth{0.100000\du}
\pgfsetdash{}{0pt}
\pgfsetmiterjoin
\pgfsetbuttcap
\definecolor{diafillcolor}{rgb}{0.662745, 0.709804, 0.674510}
\pgfsetfillcolor{diafillcolor}
\pgfsetfillopacity{1.000000}
\fill (27.058246\du,5.898114\du)--(27.058246\du,6.198114\du)--(26.758246\du,6.198114\du)--(26.758246\du,5.898114\du)--cycle;
\definecolor{dialinecolor}{rgb}{0.662745, 0.709804, 0.674510}
\pgfsetstrokecolor{dialinecolor}
\pgfsetstrokeopacity{1.000000}
\draw (26.908246\du,5.748114\du)--(26.908246\du,6.348114\du);
\pgfsetlinewidth{0.100000\du}
\pgfsetdash{}{0pt}
\pgfsetbuttcap
{
\definecolor{diafillcolor}{rgb}{0.000000, 0.000000, 0.000000}
\pgfsetfillcolor{diafillcolor}
\pgfsetfillopacity{1.000000}
% was here!!!
}
\definecolor{dialinecolor}{rgb}{0.000000, 0.000000, 0.000000}
\pgfsetstrokecolor{dialinecolor}
\pgfsetstrokeopacity{1.000000}
\draw (10.049999\du,8.049461\du)--(32.750001\du,8.000539\du);
\pgfsetlinewidth{0.100000\du}
\pgfsetdash{}{0pt}
\pgfsetmiterjoin
\pgfsetbuttcap
\definecolor{diafillcolor}{rgb}{0.000000, 0.000000, 0.000000}
\pgfsetfillcolor{diafillcolor}
\pgfsetfillopacity{1.000000}
\fill (9.800323\du,8.200000\du)--(9.799677\du,7.900000\du)--(10.099676\du,7.899354\du)--(10.100323\du,8.199353\du)--cycle;
\definecolor{dialinecolor}{rgb}{0.000000, 0.000000, 0.000000}
\pgfsetstrokecolor{dialinecolor}
\pgfsetstrokeopacity{1.000000}
\draw (9.950646\du,8.349676\du)--(9.949353\du,7.749677\du);
\pgfsetlinewidth{0.100000\du}
\pgfsetdash{}{0pt}
\pgfsetmiterjoin
\pgfsetbuttcap
\definecolor{diafillcolor}{rgb}{0.000000, 0.000000, 0.000000}
\pgfsetfillcolor{diafillcolor}
\pgfsetfillopacity{1.000000}
\fill (32.999677\du,7.850000\du)--(33.000323\du,8.150000\du)--(32.700324\du,8.150646\du)--(32.699677\du,7.850647\du)--cycle;
\definecolor{dialinecolor}{rgb}{0.000000, 0.000000, 0.000000}
\pgfsetstrokecolor{dialinecolor}
\pgfsetstrokeopacity{1.000000}
\draw (32.849354\du,7.700324\du)--(32.850647\du,8.300323\du);
\pgfsetlinewidth{0.050000\du}
\pgfsetdash{}{0pt}
\pgfsetmiterjoin
\pgfsetbuttcap
{\pgfsetcornersarced{\pgfpoint{0.000000\du}{0.000000\du}}\definecolor{dialinecolor}{rgb}{0.000000, 0.000000, 0.000000}
\pgfsetstrokecolor{dialinecolor}
\pgfsetstrokeopacity{1.000000}
\draw (6.900000\du,4.500000\du)--(6.900000\du,8.950000\du)--(36.950000\du,8.950000\du)--(36.950000\du,4.500000\du)--cycle;
}% setfont left to latex
\definecolor{dialinecolor}{rgb}{0.000000, 0.000000, 0.000000}
\pgfsetstrokecolor{dialinecolor}
\pgfsetstrokeopacity{1.000000}
\definecolor{diafillcolor}{rgb}{0.000000, 0.000000, 0.000000}
\pgfsetfillcolor{diafillcolor}
\pgfsetfillopacity{1.000000}
\node[anchor=base west,inner sep=0pt,outer sep=0pt,color=dialinecolor] at (7.950000\du,7.550000\du){currentLow};
% setfont left to latex
\definecolor{dialinecolor}{rgb}{0.000000, 0.000000, 0.000000}
\pgfsetstrokecolor{dialinecolor}
\pgfsetstrokeopacity{1.000000}
\definecolor{diafillcolor}{rgb}{0.000000, 0.000000, 0.000000}
\pgfsetfillcolor{diafillcolor}
\pgfsetfillopacity{1.000000}
\node[anchor=base west,inner sep=0pt,outer sep=0pt,color=dialinecolor] at (30.800000\du,7.500000\du){currentHight};
% setfont left to latex
\definecolor{dialinecolor}{rgb}{0.282353, 0.313726, 0.290196}
\pgfsetstrokecolor{dialinecolor}
\pgfsetstrokeopacity{1.000000}
\definecolor{diafillcolor}{rgb}{0.282353, 0.313726, 0.290196}
\pgfsetfillcolor{diafillcolor}
\pgfsetfillopacity{1.000000}
\node[anchor=base west,inner sep=0pt,outer sep=0pt,color=dialinecolor] at (11.450000\du,5.550000\du){referenceLow};
% setfont left to latex
\definecolor{dialinecolor}{rgb}{0.282353, 0.313726, 0.290196}
\pgfsetstrokecolor{dialinecolor}
\pgfsetstrokeopacity{1.000000}
\definecolor{diafillcolor}{rgb}{0.282353, 0.313726, 0.290196}
\pgfsetfillcolor{diafillcolor}
\pgfsetfillopacity{1.000000}
\node[anchor=base west,inner sep=0pt,outer sep=0pt,color=dialinecolor] at (23.750000\du,5.550000\du){referenceHight};
% setfont left to latex
\definecolor{dialinecolor}{rgb}{0.282353, 0.313726, 0.290196}
\pgfsetstrokecolor{dialinecolor}
\pgfsetstrokeopacity{1.000000}
\definecolor{diafillcolor}{rgb}{0.282353, 0.313726, 0.290196}
\pgfsetfillcolor{diafillcolor}
\pgfsetfillopacity{1.000000}
\node[anchor=base west,inner sep=0pt,outer sep=0pt,color=dialinecolor] at (24.545000\du,-0.297500\du){referenceHight};
% setfont left to latex
\definecolor{dialinecolor}{rgb}{0.282353, 0.313726, 0.290196}
\pgfsetstrokecolor{dialinecolor}
\pgfsetstrokeopacity{1.000000}
\definecolor{diafillcolor}{rgb}{0.282353, 0.313726, 0.290196}
\pgfsetfillcolor{diafillcolor}
\pgfsetfillopacity{1.000000}
\node[anchor=base west,inner sep=0pt,outer sep=0pt,color=dialinecolor] at (24.390000\du,10.792500\du){referenceHight};
% setfont left to latex
\definecolor{dialinecolor}{rgb}{0.282353, 0.313726, 0.290196}
\pgfsetstrokecolor{dialinecolor}
\pgfsetstrokeopacity{1.000000}
\definecolor{diafillcolor}{rgb}{0.282353, 0.313726, 0.290196}
\pgfsetfillcolor{diafillcolor}
\pgfsetfillopacity{1.000000}
\node[anchor=base west,inner sep=0pt,outer sep=0pt,color=dialinecolor] at (24.485000\du,16.282500\du){referenceHight};
% setfont left to latex
\definecolor{dialinecolor}{rgb}{0.282353, 0.313726, 0.290196}
\pgfsetstrokecolor{dialinecolor}
\pgfsetstrokeopacity{1.000000}
\definecolor{diafillcolor}{rgb}{0.282353, 0.313726, 0.290196}
\pgfsetfillcolor{diafillcolor}
\pgfsetfillopacity{1.000000}
\node[anchor=base west,inner sep=0pt,outer sep=0pt,color=dialinecolor] at (12.045000\du,-0.247500\du){referenceLow};
% setfont left to latex
\definecolor{dialinecolor}{rgb}{0.282353, 0.313726, 0.290196}
\pgfsetstrokecolor{dialinecolor}
\pgfsetstrokeopacity{1.000000}
\definecolor{diafillcolor}{rgb}{0.282353, 0.313726, 0.290196}
\pgfsetfillcolor{diafillcolor}
\pgfsetfillopacity{1.000000}
\node[anchor=base west,inner sep=0pt,outer sep=0pt,color=dialinecolor] at (11.340000\du,10.892500\du){referenceLow};
% setfont left to latex
\definecolor{dialinecolor}{rgb}{0.282353, 0.313726, 0.290196}
\pgfsetstrokecolor{dialinecolor}
\pgfsetstrokeopacity{1.000000}
\definecolor{diafillcolor}{rgb}{0.282353, 0.313726, 0.290196}
\pgfsetfillcolor{diafillcolor}
\pgfsetfillopacity{1.000000}
\node[anchor=base west,inner sep=0pt,outer sep=0pt,color=dialinecolor] at (11.535000\du,16.532500\du){referenceLow};
% setfont left to latex
\definecolor{dialinecolor}{rgb}{0.000000, 0.000000, 0.000000}
\pgfsetstrokecolor{dialinecolor}
\pgfsetstrokeopacity{1.000000}
\definecolor{diafillcolor}{rgb}{0.000000, 0.000000, 0.000000}
\pgfsetfillcolor{diafillcolor}
\pgfsetfillopacity{1.000000}
\node[anchor=base west,inner sep=0pt,outer sep=0pt,color=dialinecolor] at (13.845000\du,2.002500\du){currentLow};
% setfont left to latex
\definecolor{dialinecolor}{rgb}{0.000000, 0.000000, 0.000000}
\pgfsetstrokecolor{dialinecolor}
\pgfsetstrokeopacity{1.000000}
\definecolor{diafillcolor}{rgb}{0.000000, 0.000000, 0.000000}
\pgfsetfillcolor{diafillcolor}
\pgfsetfillopacity{1.000000}
\node[anchor=base west,inner sep=0pt,outer sep=0pt,color=dialinecolor] at (22.745000\du,2.002500\du){currentHight};
\pgfsetlinewidth{0.050000\du}
\pgfsetdash{}{0pt}
\pgfsetmiterjoin
\pgfsetbuttcap
{\pgfsetcornersarced{\pgfpoint{0.000000\du}{0.000000\du}}\definecolor{dialinecolor}{rgb}{0.000000, 0.000000, 0.000000}
\pgfsetstrokecolor{dialinecolor}
\pgfsetstrokeopacity{1.000000}
\draw (7.000000\du,-1.085000\du)--(7.000000\du,3.365000\du)--(37.000000\du,3.365000\du)--(37.000000\du,-1.085000\du)--cycle;
}% setfont left to latex
\definecolor{dialinecolor}{rgb}{0.000000, 0.000000, 0.000000}
\pgfsetstrokecolor{dialinecolor}
\pgfsetstrokeopacity{1.000000}
\definecolor{diafillcolor}{rgb}{0.000000, 0.000000, 0.000000}
\pgfsetfillcolor{diafillcolor}
\pgfsetfillopacity{1.000000}
\node[anchor=base west,inner sep=0pt,outer sep=0pt,color=dialinecolor] at (20.895000\du,13.002500\du){currentLow};
% setfont left to latex
\definecolor{dialinecolor}{rgb}{0.000000, 0.000000, 0.000000}
\pgfsetstrokecolor{dialinecolor}
\pgfsetstrokeopacity{1.000000}
\definecolor{diafillcolor}{rgb}{0.000000, 0.000000, 0.000000}
\pgfsetfillcolor{diafillcolor}
\pgfsetfillopacity{1.000000}
\node[anchor=base west,inner sep=0pt,outer sep=0pt,color=dialinecolor] at (7.140000\du,18.292500\du){currentLow};
% setfont left to latex
\definecolor{dialinecolor}{rgb}{0.000000, 0.000000, 0.000000}
\pgfsetstrokecolor{dialinecolor}
\pgfsetstrokeopacity{1.000000}
\definecolor{diafillcolor}{rgb}{0.000000, 0.000000, 0.000000}
\pgfsetfillcolor{diafillcolor}
\pgfsetfillopacity{1.000000}
\node[anchor=base west,inner sep=0pt,outer sep=0pt,color=dialinecolor] at (29.695000\du,13.002500\du){currentHight};
% setfont left to latex
\definecolor{dialinecolor}{rgb}{0.000000, 0.000000, 0.000000}
\pgfsetstrokecolor{dialinecolor}
\pgfsetstrokeopacity{1.000000}
\definecolor{diafillcolor}{rgb}{0.000000, 0.000000, 0.000000}
\pgfsetfillcolor{diafillcolor}
\pgfsetfillopacity{1.000000}
\node[anchor=base west,inner sep=0pt,outer sep=0pt,color=dialinecolor] at (15.840000\du,18.442500\du){currentHight};
\pgfsetlinewidth{0.050000\du}
\pgfsetdash{}{0pt}
\pgfsetmiterjoin
\pgfsetbuttcap
{\pgfsetcornersarced{\pgfpoint{0.000000\du}{0.000000\du}}\definecolor{dialinecolor}{rgb}{0.000000, 0.000000, 0.000000}
\pgfsetstrokecolor{dialinecolor}
\pgfsetstrokeopacity{1.000000}
\draw (7.000000\du,10.015000\du)--(7.000000\du,14.465000\du)--(37.050000\du,14.465000\du)--(37.050000\du,10.015000\du)--cycle;
}\pgfsetlinewidth{0.050000\du}
\pgfsetdash{}{0pt}
\pgfsetmiterjoin
\pgfsetbuttcap
{\pgfsetcornersarced{\pgfpoint{0.000000\du}{0.000000\du}}\definecolor{dialinecolor}{rgb}{0.000000, 0.000000, 0.000000}
\pgfsetstrokecolor{dialinecolor}
\pgfsetstrokeopacity{1.000000}
\draw (6.950000\du,15.515000\du)--(6.950000\du,19.965000\du)--(37.050000\du,19.965000\du)--(37.050000\du,15.515000\du)--cycle;
}% setfont left to latex
\definecolor{dialinecolor}{rgb}{0.000000, 0.000000, 0.000000}
\pgfsetstrokecolor{dialinecolor}
\pgfsetstrokeopacity{1.000000}
\definecolor{diafillcolor}{rgb}{0.000000, 0.000000, 0.000000}
\pgfsetfillcolor{diafillcolor}
\pgfsetfillopacity{1.000000}
\node[anchor=base west,inner sep=0pt,outer sep=0pt,color=dialinecolor] at (3.950000\du,1.000000\du){Cas 1};
% setfont left to latex
\definecolor{dialinecolor}{rgb}{0.000000, 0.000000, 0.000000}
\pgfsetstrokecolor{dialinecolor}
\pgfsetstrokeopacity{1.000000}
\definecolor{diafillcolor}{rgb}{0.000000, 0.000000, 0.000000}
\pgfsetfillcolor{diafillcolor}
\pgfsetfillopacity{1.000000}
\node[anchor=base west,inner sep=0pt,outer sep=0pt,color=dialinecolor] at (4.050000\du,7.000000\du){Cas 2};
% setfont left to latex
\definecolor{dialinecolor}{rgb}{0.000000, 0.000000, 0.000000}
\pgfsetstrokecolor{dialinecolor}
\pgfsetstrokeopacity{1.000000}
\definecolor{diafillcolor}{rgb}{0.000000, 0.000000, 0.000000}
\pgfsetfillcolor{diafillcolor}
\pgfsetfillopacity{1.000000}
\node[anchor=base west,inner sep=0pt,outer sep=0pt,color=dialinecolor] at (4.050000\du,12.000000\du){Cas 3};
% setfont left to latex
\definecolor{dialinecolor}{rgb}{0.000000, 0.000000, 0.000000}
\pgfsetstrokecolor{dialinecolor}
\pgfsetstrokeopacity{1.000000}
\definecolor{diafillcolor}{rgb}{0.000000, 0.000000, 0.000000}
\pgfsetfillcolor{diafillcolor}
\pgfsetfillopacity{1.000000}
\node[anchor=base west,inner sep=0pt,outer sep=0pt,color=dialinecolor] at (4.050000\du,18.100000\du){Cas 4};
\end{tikzpicture}

	}
	\caption{La comparaison des deux intervalles de confiance : courant et référence }
	\label{fig:intervals-comparaison}
\end{figure}

\subsection{Vue globale des étapes de la détection des anomalies à travers l'évolution des RTTs différentiels}
La figure 	\ref{fig:process-rttanalysis_tex} présente la succession des étapes de la détection des anomalies dans les délais d'un lien donné. 

\begin{figure}[h]
	\centering
	\resizebox{\textwidth}{\textheight}{
		% Graphic for TeX using PGF
% Title: /home/hayat/RipeAtlasTraceroutesAnalysis/dia/process-rttanalysis.dia
% Creator: Dia v0.97+git
% CreationDate: Thu Nov 29 01:56:23 2018
% For: hayat
% \usepackage{tikz}
% The following commands are not supported in PSTricks at present
% We define them conditionally, so when they are implemented,
% this pgf file will use them.
\ifx\du\undefined
  \newlength{\du}
\fi
\setlength{\du}{15\unitlength}
\begin{tikzpicture}[even odd rule]
\pgftransformxscale{1.000000}
\pgftransformyscale{-1.000000}
\definecolor{dialinecolor}{rgb}{0.000000, 0.000000, 0.000000}
\pgfsetstrokecolor{dialinecolor}
\pgfsetstrokeopacity{1.000000}
\definecolor{diafillcolor}{rgb}{1.000000, 1.000000, 1.000000}
\pgfsetfillcolor{diafillcolor}
\pgfsetfillopacity{1.000000}
\pgfsetlinewidth{0.100000\du}
\pgfsetdash{}{0pt}
\pgfsetmiterjoin
{\pgfsetcornersarced{\pgfpoint{0.000000\du}{0.000000\du}}\definecolor{diafillcolor}{rgb}{1.000000, 1.000000, 1.000000}
\pgfsetfillcolor{diafillcolor}
\pgfsetfillopacity{1.000000}
\fill (8.138750\du,-5.900000\du)--(8.138750\du,-4.000000\du)--(21.861250\du,-4.000000\du)--(21.861250\du,-5.900000\du)--cycle;
}{\pgfsetcornersarced{\pgfpoint{0.000000\du}{0.000000\du}}\definecolor{dialinecolor}{rgb}{0.000000, 0.000000, 0.000000}
\pgfsetstrokecolor{dialinecolor}
\pgfsetstrokeopacity{1.000000}
\draw (8.138750\du,-5.900000\du)--(8.138750\du,-4.000000\du)--(21.861250\du,-4.000000\du)--(21.861250\du,-5.900000\du)--cycle;
}% setfont left to latex
\definecolor{dialinecolor}{rgb}{0.000000, 0.000000, 0.000000}
\pgfsetstrokecolor{dialinecolor}
\pgfsetstrokeopacity{1.000000}
\definecolor{diafillcolor}{rgb}{0.000000, 0.000000, 0.000000}
\pgfsetfillcolor{diafillcolor}
\pgfsetfillopacity{1.000000}
\node[anchor=base,inner sep=0pt, outer sep=0pt,color=dialinecolor] at (15.000000\du,-4.755000\du){\ensuremath{[}traceroute:\{from:"",type:"", result:""\}\ensuremath{]}};
\pgfsetlinewidth{0.100000\du}
\pgfsetdash{}{0pt}
\pgfsetbuttcap
{
\definecolor{diafillcolor}{rgb}{0.000000, 0.000000, 0.000000}
\pgfsetfillcolor{diafillcolor}
\pgfsetfillopacity{1.000000}
% was here!!!
\pgfsetarrowsend{stealth}
\definecolor{dialinecolor}{rgb}{0.000000, 0.000000, 0.000000}
\pgfsetstrokecolor{dialinecolor}
\pgfsetstrokeopacity{1.000000}
\draw (20.060900\du,1.256430\du)--(15.048400\du,4.142720\du);
}
\pgfsetlinewidth{0.100000\du}
\pgfsetdash{}{0pt}
\pgfsetmiterjoin
{\pgfsetcornersarced{\pgfpoint{0.000000\du}{0.000000\du}}\definecolor{diafillcolor}{rgb}{1.000000, 1.000000, 1.000000}
\pgfsetfillcolor{diafillcolor}
\pgfsetfillopacity{1.000000}
\fill (4.175940\du,4.142080\du)--(4.175940\du,6.042080\du)--(9.163440\du,6.042080\du)--(9.163440\du,4.142080\du)--cycle;
}{\pgfsetcornersarced{\pgfpoint{0.000000\du}{0.000000\du}}\definecolor{dialinecolor}{rgb}{0.000000, 0.000000, 0.000000}
\pgfsetstrokecolor{dialinecolor}
\pgfsetstrokeopacity{1.000000}
\draw (4.175940\du,4.142080\du)--(4.175940\du,6.042080\du)--(9.163440\du,6.042080\du)--(9.163440\du,4.142080\du)--cycle;
}% setfont left to latex
\definecolor{dialinecolor}{rgb}{0.000000, 0.000000, 0.000000}
\pgfsetstrokecolor{dialinecolor}
\pgfsetstrokeopacity{1.000000}
\definecolor{diafillcolor}{rgb}{0.000000, 0.000000, 0.000000}
\pgfsetfillcolor{diafillcolor}
\pgfsetfillopacity{1.000000}
\node[anchor=base,inner sep=0pt, outer sep=0pt,color=dialinecolor] at (6.669690\du,5.287080\du){\ensuremath{[}Traceroute\ensuremath{]}};
% setfont left to latex
\definecolor{dialinecolor}{rgb}{0.000000, 0.000000, 0.000000}
\pgfsetstrokecolor{dialinecolor}
\pgfsetstrokeopacity{1.000000}
\definecolor{diafillcolor}{rgb}{0.000000, 0.000000, 0.000000}
\pgfsetfillcolor{diafillcolor}
\pgfsetfillopacity{1.000000}
\node[anchor=base west,inner sep=0pt,outer sep=0pt,color=dialinecolor] at (16.400000\du,9.562040\du){};
\pgfsetlinewidth{0.100000\du}
\pgfsetdash{}{0pt}
\pgfsetbuttcap
\pgfsetmiterjoin
\pgfsetlinewidth{0.100000\du}
\pgfsetbuttcap
\pgfsetmiterjoin
\pgfsetdash{}{0pt}
\definecolor{diafillcolor}{rgb}{1.000000, 1.000000, 1.000000}
\pgfsetfillcolor{diafillcolor}
\pgfsetfillopacity{1.000000}
\definecolor{dialinecolor}{rgb}{0.000000, 0.000000, 0.000000}
\pgfsetstrokecolor{dialinecolor}
\pgfsetstrokeopacity{1.000000}
\pgfpathmoveto{\pgfpoint{24.362725\du}{6.967570\du}}
\pgfpathlineto{\pgfpoint{38.555225\du}{6.967570\du}}
\pgfpathcurveto{\pgfpoint{40.514801\du}{6.967570\du}}{\pgfpoint{42.103350\du}{7.213813\du}}{\pgfpoint{42.103350\du}{7.517570\du}}
\pgfpathcurveto{\pgfpoint{42.103350\du}{7.821327\du}}{\pgfpoint{40.514801\du}{8.067570\du}}{\pgfpoint{38.555225\du}{8.067570\du}}
\pgfpathlineto{\pgfpoint{24.362725\du}{8.067570\du}}
\pgfpathcurveto{\pgfpoint{22.403149\du}{8.067570\du}}{\pgfpoint{20.814600\du}{7.821327\du}}{\pgfpoint{20.814600\du}{7.517570\du}}
\pgfpathcurveto{\pgfpoint{20.814600\du}{7.213813\du}}{\pgfpoint{22.403149\du}{6.967570\du}}{\pgfpoint{24.362725\du}{6.967570\du}}
\pgfpathclose
\pgfusepath{fill,stroke}
% setfont left to latex
\definecolor{dialinecolor}{rgb}{0.000000, 0.000000, 0.000000}
\pgfsetstrokecolor{dialinecolor}
\pgfsetstrokeopacity{1.000000}
\definecolor{diafillcolor}{rgb}{0.000000, 0.000000, 0.000000}
\pgfsetfillcolor{diafillcolor}
\pgfsetfillopacity{1.000000}
\node[anchor=base,inner sep=0pt, outer sep=0pt,color=dialinecolor] at (31.458975\du,7.717570\du){2.Elimination des traceroutes échoués, etc.};
\pgfsetlinewidth{0.100000\du}
\pgfsetdash{}{0pt}
\pgfsetmiterjoin
{\pgfsetcornersarced{\pgfpoint{0.000000\du}{0.000000\du}}\definecolor{diafillcolor}{rgb}{1.000000, 1.000000, 1.000000}
\pgfsetfillcolor{diafillcolor}
\pgfsetfillopacity{1.000000}
\fill (10.748200\du,8.560140\du)--(10.748200\du,10.460140\du)--(19.318200\du,10.460140\du)--(19.318200\du,8.560140\du)--cycle;
}{\pgfsetcornersarced{\pgfpoint{0.000000\du}{0.000000\du}}\definecolor{dialinecolor}{rgb}{0.000000, 0.000000, 0.000000}
\pgfsetstrokecolor{dialinecolor}
\pgfsetstrokeopacity{1.000000}
\draw (10.748200\du,8.560140\du)--(10.748200\du,10.460140\du)--(19.318200\du,10.460140\du)--(19.318200\du,8.560140\du)--cycle;
}% setfont left to latex
\definecolor{dialinecolor}{rgb}{0.000000, 0.000000, 0.000000}
\pgfsetstrokecolor{dialinecolor}
\pgfsetstrokeopacity{1.000000}
\definecolor{diafillcolor}{rgb}{0.000000, 0.000000, 0.000000}
\pgfsetfillcolor{diafillcolor}
\pgfsetfillopacity{1.000000}
\node[anchor=base,inner sep=0pt, outer sep=0pt,color=dialinecolor] at (15.033200\du,9.705140\du){\ensuremath{[}Traceroute\ensuremath{]}};
\pgfsetlinewidth{0.100000\du}
\pgfsetdash{}{0pt}
\pgfsetbuttcap
{
\definecolor{diafillcolor}{rgb}{0.000000, 0.000000, 0.000000}
\pgfsetfillcolor{diafillcolor}
\pgfsetfillopacity{1.000000}
% was here!!!
\pgfsetarrowsend{stealth}
\definecolor{dialinecolor}{rgb}{0.000000, 0.000000, 0.000000}
\pgfsetstrokecolor{dialinecolor}
\pgfsetstrokeopacity{1.000000}
\draw (15.063300\du,6.216320\du)--(15.033200\du,8.560140\du);
}
\pgfsetlinewidth{0.100000\du}
\pgfsetdash{}{0pt}
\pgfsetmiterjoin
{\pgfsetcornersarced{\pgfpoint{0.000000\du}{0.000000\du}}\definecolor{diafillcolor}{rgb}{1.000000, 1.000000, 1.000000}
\pgfsetfillcolor{diafillcolor}
\pgfsetfillopacity{1.000000}
\fill (11.463400\du,17.484800\du)--(11.463400\du,19.384800\du)--(18.483400\du,19.384800\du)--(18.483400\du,17.484800\du)--cycle;
}{\pgfsetcornersarced{\pgfpoint{0.000000\du}{0.000000\du}}\definecolor{dialinecolor}{rgb}{0.000000, 0.000000, 0.000000}
\pgfsetstrokecolor{dialinecolor}
\pgfsetstrokeopacity{1.000000}
\draw (11.463400\du,17.484800\du)--(11.463400\du,19.384800\du)--(18.483400\du,19.384800\du)--(18.483400\du,17.484800\du)--cycle;
}% setfont left to latex
\definecolor{dialinecolor}{rgb}{0.000000, 0.000000, 0.000000}
\pgfsetstrokecolor{dialinecolor}
\pgfsetstrokeopacity{1.000000}
\definecolor{diafillcolor}{rgb}{0.000000, 0.000000, 0.000000}
\pgfsetfillcolor{diafillcolor}
\pgfsetfillopacity{1.000000}
\node[anchor=base,inner sep=0pt, outer sep=0pt,color=dialinecolor] at (14.973400\du,18.629800\du){\ensuremath{[}LinksTraceroute\ensuremath{]}};
\pgfsetlinewidth{0.100000\du}
\pgfsetdash{}{0pt}
\pgfsetbuttcap
{
\definecolor{diafillcolor}{rgb}{0.000000, 0.000000, 0.000000}
\pgfsetfillcolor{diafillcolor}
\pgfsetfillopacity{1.000000}
% was here!!!
\pgfsetarrowsend{stealth}
\definecolor{dialinecolor}{rgb}{0.000000, 0.000000, 0.000000}
\pgfsetstrokecolor{dialinecolor}
\pgfsetstrokeopacity{1.000000}
\draw (14.954000\du,14.821200\du)--(14.973400\du,17.484800\du);
}
\pgfsetlinewidth{0.100000\du}
\pgfsetdash{}{0pt}
\pgfsetbuttcap
{
\definecolor{diafillcolor}{rgb}{0.000000, 0.000000, 0.000000}
\pgfsetfillcolor{diafillcolor}
\pgfsetfillopacity{1.000000}
% was here!!!
\pgfsetarrowsend{stealth}
\definecolor{dialinecolor}{rgb}{0.000000, 0.000000, 0.000000}
\pgfsetstrokecolor{dialinecolor}
\pgfsetstrokeopacity{1.000000}
\draw (15.033200\du,10.460100\du)--(14.954000\du,12.921200\du);
}
\pgfsetlinewidth{0.100000\du}
\pgfsetdash{}{0pt}
\pgfsetmiterjoin
{\pgfsetcornersarced{\pgfpoint{0.000000\du}{0.000000\du}}\definecolor{diafillcolor}{rgb}{1.000000, 1.000000, 1.000000}
\pgfsetfillcolor{diafillcolor}
\pgfsetfillopacity{1.000000}
\fill (10.244000\du,12.921200\du)--(10.244000\du,14.821200\du)--(19.664000\du,14.821200\du)--(19.664000\du,12.921200\du)--cycle;
}{\pgfsetcornersarced{\pgfpoint{0.000000\du}{0.000000\du}}\definecolor{dialinecolor}{rgb}{0.000000, 0.000000, 0.000000}
\pgfsetstrokecolor{dialinecolor}
\pgfsetstrokeopacity{1.000000}
\draw (10.244000\du,12.921200\du)--(10.244000\du,14.821200\du)--(19.664000\du,14.821200\du)--(19.664000\du,12.921200\du)--cycle;
}% setfont left to latex
\definecolor{dialinecolor}{rgb}{0.000000, 0.000000, 0.000000}
\pgfsetstrokecolor{dialinecolor}
\pgfsetstrokeopacity{1.000000}
\definecolor{diafillcolor}{rgb}{0.000000, 0.000000, 0.000000}
\pgfsetfillcolor{diafillcolor}
\pgfsetfillopacity{1.000000}
\node[anchor=base,inner sep=0pt, outer sep=0pt,color=dialinecolor] at (14.954000\du,14.066200\du){\ensuremath{[}MedianByHopTraceroute\ensuremath{]}};
\pgfsetlinewidth{0.100000\du}
\pgfsetdash{}{0pt}
\pgfsetbuttcap
\pgfsetmiterjoin
\pgfsetlinewidth{0.100000\du}
\pgfsetbuttcap
\pgfsetmiterjoin
\pgfsetdash{}{0pt}
\definecolor{diafillcolor}{rgb}{1.000000, 1.000000, 1.000000}
\pgfsetfillcolor{diafillcolor}
\pgfsetfillopacity{1.000000}
\definecolor{dialinecolor}{rgb}{0.000000, 0.000000, 0.000000}
\pgfsetstrokecolor{dialinecolor}
\pgfsetstrokeopacity{1.000000}
\pgfpathmoveto{\pgfpoint{21.475800\du}{11.116500\du}}
\pgfpathlineto{\pgfpoint{31.295800\du}{11.116500\du}}
\pgfpathcurveto{\pgfpoint{32.651660\du}{11.116500\du}}{\pgfpoint{33.750800\du}{11.362743\du}}{\pgfpoint{33.750800\du}{11.666500\du}}
\pgfpathcurveto{\pgfpoint{33.750800\du}{11.970257\du}}{\pgfpoint{32.651660\du}{12.216500\du}}{\pgfpoint{31.295800\du}{12.216500\du}}
\pgfpathlineto{\pgfpoint{21.475800\du}{12.216500\du}}
\pgfpathcurveto{\pgfpoint{20.119940\du}{12.216500\du}}{\pgfpoint{19.020800\du}{11.970257\du}}{\pgfpoint{19.020800\du}{11.666500\du}}
\pgfpathcurveto{\pgfpoint{19.020800\du}{11.362743\du}}{\pgfpoint{20.119940\du}{11.116500\du}}{\pgfpoint{21.475800\du}{11.116500\du}}
\pgfpathclose
\pgfusepath{fill,stroke}
% setfont left to latex
\definecolor{dialinecolor}{rgb}{0.000000, 0.000000, 0.000000}
\pgfsetstrokecolor{dialinecolor}
\pgfsetstrokeopacity{1.000000}
\definecolor{diafillcolor}{rgb}{0.000000, 0.000000, 0.000000}
\pgfsetfillcolor{diafillcolor}
\pgfsetfillopacity{1.000000}
\node[anchor=base,inner sep=0pt, outer sep=0pt,color=dialinecolor] at (26.385800\du,11.866500\du){3. calcul de mediane par saut};
\pgfsetlinewidth{0.100000\du}
\pgfsetdash{}{0pt}
\pgfsetbuttcap
\pgfsetmiterjoin
\pgfsetlinewidth{0.100000\du}
\pgfsetbuttcap
\pgfsetmiterjoin
\pgfsetdash{}{0pt}
\definecolor{diafillcolor}{rgb}{1.000000, 1.000000, 1.000000}
\pgfsetfillcolor{diafillcolor}
\pgfsetfillopacity{1.000000}
\definecolor{dialinecolor}{rgb}{0.000000, 0.000000, 0.000000}
\pgfsetstrokecolor{dialinecolor}
\pgfsetstrokeopacity{1.000000}
\pgfpathmoveto{\pgfpoint{20.956000\du}{15.486500\du}}
\pgfpathlineto{\pgfpoint{28.296000\du}{15.486500\du}}
\pgfpathcurveto{\pgfpoint{29.309443\du}{15.486500\du}}{\pgfpoint{30.131000\du}{15.732743\du}}{\pgfpoint{30.131000\du}{16.036500\du}}
\pgfpathcurveto{\pgfpoint{30.131000\du}{16.340257\du}}{\pgfpoint{29.309443\du}{16.586500\du}}{\pgfpoint{28.296000\du}{16.586500\du}}
\pgfpathlineto{\pgfpoint{20.956000\du}{16.586500\du}}
\pgfpathcurveto{\pgfpoint{19.942557\du}{16.586500\du}}{\pgfpoint{19.121000\du}{16.340257\du}}{\pgfpoint{19.121000\du}{16.036500\du}}
\pgfpathcurveto{\pgfpoint{19.121000\du}{15.732743\du}}{\pgfpoint{19.942557\du}{15.486500\du}}{\pgfpoint{20.956000\du}{15.486500\du}}
\pgfpathclose
\pgfusepath{fill,stroke}
% setfont left to latex
\definecolor{dialinecolor}{rgb}{0.000000, 0.000000, 0.000000}
\pgfsetstrokecolor{dialinecolor}
\pgfsetstrokeopacity{1.000000}
\definecolor{diafillcolor}{rgb}{0.000000, 0.000000, 0.000000}
\pgfsetfillcolor{diafillcolor}
\pgfsetfillopacity{1.000000}
\node[anchor=base,inner sep=0pt, outer sep=0pt,color=dialinecolor] at (24.626000\du,16.236500\du){4. inférence des liens };
\pgfsetlinewidth{0.100000\du}
\pgfsetdash{}{0pt}
\pgfsetmiterjoin
{\pgfsetcornersarced{\pgfpoint{0.000000\du}{0.000000\du}}\definecolor{diafillcolor}{rgb}{1.000000, 1.000000, 1.000000}
\pgfsetfillcolor{diafillcolor}
\pgfsetfillopacity{1.000000}
\fill (13.005900\du,21.418700\du)--(13.005900\du,23.318700\du)--(17.040900\du,23.318700\du)--(17.040900\du,21.418700\du)--cycle;
}{\pgfsetcornersarced{\pgfpoint{0.000000\du}{0.000000\du}}\definecolor{dialinecolor}{rgb}{0.000000, 0.000000, 0.000000}
\pgfsetstrokecolor{dialinecolor}
\pgfsetstrokeopacity{1.000000}
\draw (13.005900\du,21.418700\du)--(13.005900\du,23.318700\du)--(17.040900\du,23.318700\du)--(17.040900\du,21.418700\du)--cycle;
}% setfont left to latex
\definecolor{dialinecolor}{rgb}{0.000000, 0.000000, 0.000000}
\pgfsetstrokecolor{dialinecolor}
\pgfsetstrokeopacity{1.000000}
\definecolor{diafillcolor}{rgb}{0.000000, 0.000000, 0.000000}
\pgfsetfillcolor{diafillcolor}
\pgfsetfillopacity{1.000000}
\node[anchor=base,inner sep=0pt, outer sep=0pt,color=dialinecolor] at (15.023400\du,22.563700\du){\ensuremath{[}DiffRTT\ensuremath{]}};
\pgfsetlinewidth{0.100000\du}
\pgfsetdash{}{0pt}
\pgfsetbuttcap
{
\definecolor{diafillcolor}{rgb}{0.000000, 0.000000, 0.000000}
\pgfsetfillcolor{diafillcolor}
\pgfsetfillopacity{1.000000}
% was here!!!
\pgfsetarrowsend{stealth}
\definecolor{dialinecolor}{rgb}{0.000000, 0.000000, 0.000000}
\pgfsetstrokecolor{dialinecolor}
\pgfsetstrokeopacity{1.000000}
\draw (14.973400\du,19.384800\du)--(15.006640\du,21.368482\du);
}
\pgfsetlinewidth{0.100000\du}
\pgfsetdash{}{0pt}
\pgfsetbuttcap
\pgfsetmiterjoin
\pgfsetlinewidth{0.100000\du}
\pgfsetbuttcap
\pgfsetmiterjoin
\pgfsetdash{}{0pt}
\definecolor{diafillcolor}{rgb}{1.000000, 1.000000, 1.000000}
\pgfsetfillcolor{diafillcolor}
\pgfsetfillopacity{1.000000}
\definecolor{dialinecolor}{rgb}{0.000000, 0.000000, 0.000000}
\pgfsetstrokecolor{dialinecolor}
\pgfsetstrokeopacity{1.000000}
\pgfpathmoveto{\pgfpoint{22.489775\du}{20.027600\du}}
\pgfpathlineto{\pgfpoint{38.917275\du}{20.027600\du}}
\pgfpathcurveto{\pgfpoint{41.185440\du}{20.027600\du}}{\pgfpoint{43.024150\du}{20.273843\du}}{\pgfpoint{43.024150\du}{20.577600\du}}
\pgfpathcurveto{\pgfpoint{43.024150\du}{20.881357\du}}{\pgfpoint{41.185440\du}{21.127600\du}}{\pgfpoint{38.917275\du}{21.127600\du}}
\pgfpathlineto{\pgfpoint{22.489775\du}{21.127600\du}}
\pgfpathcurveto{\pgfpoint{20.221610\du}{21.127600\du}}{\pgfpoint{18.382900\du}{20.881357\du}}{\pgfpoint{18.382900\du}{20.577600\du}}
\pgfpathcurveto{\pgfpoint{18.382900\du}{20.273843\du}}{\pgfpoint{20.221610\du}{20.027600\du}}{\pgfpoint{22.489775\du}{20.027600\du}}
\pgfpathclose
\pgfusepath{fill,stroke}
% setfont left to latex
\definecolor{dialinecolor}{rgb}{0.000000, 0.000000, 0.000000}
\pgfsetstrokecolor{dialinecolor}
\pgfsetstrokeopacity{1.000000}
\definecolor{diafillcolor}{rgb}{0.000000, 0.000000, 0.000000}
\pgfsetfillcolor{diafillcolor}
\pgfsetfillopacity{1.000000}
\node[anchor=base,inner sep=0pt, outer sep=0pt,color=dialinecolor] at (30.703525\du,20.777600\du){5. Caractérisation de chaque lien en objet DiffRTT };
\pgfsetlinewidth{0.100000\du}
\pgfsetdash{}{0pt}
\pgfsetbuttcap
{
\definecolor{diafillcolor}{rgb}{0.000000, 0.000000, 0.000000}
\pgfsetfillcolor{diafillcolor}
\pgfsetfillopacity{1.000000}
% was here!!!
\pgfsetarrowsend{stealth}
\definecolor{dialinecolor}{rgb}{0.000000, 0.000000, 0.000000}
\pgfsetstrokecolor{dialinecolor}
\pgfsetstrokeopacity{1.000000}
\draw (15.035872\du,23.368063\du)--(15.069300\du,26.046500\du);
}
\pgfsetlinewidth{0.100000\du}
\pgfsetdash{}{0pt}
\pgfsetmiterjoin
{\pgfsetcornersarced{\pgfpoint{0.000000\du}{0.000000\du}}\definecolor{diafillcolor}{rgb}{1.000000, 1.000000, 1.000000}
\pgfsetfillcolor{diafillcolor}
\pgfsetfillopacity{1.000000}
\fill (10.881300\du,-0.650000\du)--(10.881300\du,1.250000\du)--(19.218800\du,1.250000\du)--(19.218800\du,-0.650000\du)--cycle;
}{\pgfsetcornersarced{\pgfpoint{0.000000\du}{0.000000\du}}\definecolor{dialinecolor}{rgb}{0.000000, 0.000000, 0.000000}
\pgfsetstrokecolor{dialinecolor}
\pgfsetstrokeopacity{1.000000}
\draw (10.881300\du,-0.650000\du)--(10.881300\du,1.250000\du)--(19.218800\du,1.250000\du)--(19.218800\du,-0.650000\du)--cycle;
}% setfont left to latex
\definecolor{dialinecolor}{rgb}{0.000000, 0.000000, 0.000000}
\pgfsetstrokecolor{dialinecolor}
\pgfsetstrokeopacity{1.000000}
\definecolor{diafillcolor}{rgb}{0.000000, 0.000000, 0.000000}
\pgfsetfillcolor{diafillcolor}
\pgfsetfillopacity{1.000000}
\node[anchor=base,inner sep=0pt, outer sep=0pt,color=dialinecolor] at (15.050050\du,0.495000\du){\ensuremath{[}TraceroutesPerPeriod\ensuremath{]}};
\pgfsetlinewidth{0.100000\du}
\pgfsetdash{}{0pt}
\pgfsetbuttcap
{
\definecolor{diafillcolor}{rgb}{0.000000, 0.000000, 0.000000}
\pgfsetfillcolor{diafillcolor}
\pgfsetfillopacity{1.000000}
% was here!!!
\pgfsetarrowsend{stealth}
\definecolor{dialinecolor}{rgb}{0.000000, 0.000000, 0.000000}
\pgfsetstrokecolor{dialinecolor}
\pgfsetstrokeopacity{1.000000}
\draw (15.000000\du,-4.000000\du)--(15.050100\du,-0.650000\du);
}
\pgfsetlinewidth{0.100000\du}
\pgfsetdash{}{0pt}
\pgfsetbuttcap
\pgfsetmiterjoin
\pgfsetlinewidth{0.100000\du}
\pgfsetbuttcap
\pgfsetmiterjoin
\pgfsetdash{}{0pt}
\definecolor{diafillcolor}{rgb}{1.000000, 1.000000, 1.000000}
\pgfsetfillcolor{diafillcolor}
\pgfsetfillopacity{1.000000}
\definecolor{dialinecolor}{rgb}{0.000000, 0.000000, 0.000000}
\pgfsetstrokecolor{dialinecolor}
\pgfsetstrokeopacity{1.000000}
\pgfpathmoveto{\pgfpoint{23.806850\du}{-2.900000\du}}
\pgfpathlineto{\pgfpoint{36.731850\du}{-2.900000\du}}
\pgfpathcurveto{\pgfpoint{38.516421\du}{-2.900000\du}}{\pgfpoint{39.963100\du}{-2.620178\du}}{\pgfpoint{39.963100\du}{-2.275000\du}}
\pgfpathcurveto{\pgfpoint{39.963100\du}{-1.929822\du}}{\pgfpoint{38.516421\du}{-1.650000\du}}{\pgfpoint{36.731850\du}{-1.650000\du}}
\pgfpathlineto{\pgfpoint{23.806850\du}{-1.650000\du}}
\pgfpathcurveto{\pgfpoint{22.022279\du}{-1.650000\du}}{\pgfpoint{20.575600\du}{-1.929822\du}}{\pgfpoint{20.575600\du}{-2.275000\du}}
\pgfpathcurveto{\pgfpoint{20.575600\du}{-2.620178\du}}{\pgfpoint{22.022279\du}{-2.900000\du}}{\pgfpoint{23.806850\du}{-2.900000\du}}
\pgfpathclose
\pgfusepath{fill,stroke}
% setfont left to latex
\definecolor{dialinecolor}{rgb}{0.000000, 0.000000, 0.000000}
\pgfsetstrokecolor{dialinecolor}
\pgfsetstrokeopacity{1.000000}
\definecolor{diafillcolor}{rgb}{0.000000, 0.000000, 0.000000}
\pgfsetfillcolor{diafillcolor}
\pgfsetfillopacity{1.000000}
\node[anchor=base,inner sep=0pt, outer sep=0pt,color=dialinecolor] at (30.269350\du,-2.075000\du){1. Trier les traceroutes par timeWindow};
% setfont left to latex
\definecolor{dialinecolor}{rgb}{0.000000, 0.000000, 0.000000}
\pgfsetstrokecolor{dialinecolor}
\pgfsetstrokeopacity{1.000000}
\definecolor{diafillcolor}{rgb}{0.000000, 0.000000, 0.000000}
\pgfsetfillcolor{diafillcolor}
\pgfsetfillopacity{1.000000}
\node[anchor=base west,inner sep=0pt,outer sep=0pt,color=dialinecolor] at (21.150000\du,0.700000\du){ Pour chaque instance du TraceroutesPerPeriod appliquer 2. à 6.};
\pgfsetlinewidth{0.100000\du}
\pgfsetdash{}{0pt}
\pgfsetmiterjoin
{\pgfsetcornersarced{\pgfpoint{0.000000\du}{0.000000\du}}\definecolor{diafillcolor}{rgb}{1.000000, 1.000000, 1.000000}
\pgfsetfillcolor{diafillcolor}
\pgfsetfillopacity{1.000000}
\fill (11.031200\du,4.142720\du)--(11.031200\du,6.042720\du)--(19.065613\du,6.042720\du)--(19.065613\du,4.142720\du)--cycle;
}{\pgfsetcornersarced{\pgfpoint{0.000000\du}{0.000000\du}}\definecolor{dialinecolor}{rgb}{0.000000, 0.000000, 0.000000}
\pgfsetstrokecolor{dialinecolor}
\pgfsetstrokeopacity{1.000000}
\draw (11.031200\du,4.142720\du)--(11.031200\du,6.042720\du)--(19.065613\du,6.042720\du)--(19.065613\du,4.142720\du)--cycle;
}% setfont left to latex
\definecolor{dialinecolor}{rgb}{0.000000, 0.000000, 0.000000}
\pgfsetstrokecolor{dialinecolor}
\pgfsetstrokeopacity{1.000000}
\definecolor{diafillcolor}{rgb}{0.000000, 0.000000, 0.000000}
\pgfsetfillcolor{diafillcolor}
\pgfsetfillopacity{1.000000}
\node[anchor=base,inner sep=0pt, outer sep=0pt,color=dialinecolor] at (15.048406\du,5.287720\du){TraceroutesPerPeriod};
\pgfsetlinewidth{0.100000\du}
\pgfsetdash{}{0pt}
\pgfsetmiterjoin
{\pgfsetcornersarced{\pgfpoint{0.000000\du}{0.000000\du}}\definecolor{diafillcolor}{rgb}{1.000000, 1.000000, 1.000000}
\pgfsetfillcolor{diafillcolor}
\pgfsetfillopacity{1.000000}
\fill (12.053100\du,26.046500\du)--(12.053100\du,27.946500\du)--(18.085600\du,27.946500\du)--(18.085600\du,26.046500\du)--cycle;
}{\pgfsetcornersarced{\pgfpoint{0.000000\du}{0.000000\du}}\definecolor{dialinecolor}{rgb}{0.000000, 0.000000, 0.000000}
\pgfsetstrokecolor{dialinecolor}
\pgfsetstrokeopacity{1.000000}
\draw (12.053100\du,26.046500\du)--(12.053100\du,27.946500\du)--(18.085600\du,27.946500\du)--(18.085600\du,26.046500\du)--cycle;
}% setfont left to latex
\definecolor{dialinecolor}{rgb}{0.000000, 0.000000, 0.000000}
\pgfsetstrokecolor{dialinecolor}
\pgfsetstrokeopacity{1.000000}
\definecolor{diafillcolor}{rgb}{0.000000, 0.000000, 0.000000}
\pgfsetfillcolor{diafillcolor}
\pgfsetfillopacity{1.000000}
\node[anchor=base,inner sep=0pt, outer sep=0pt,color=dialinecolor] at (15.069350\du,27.191500\du){\ensuremath{[}DiffRTT\ensuremath{]}};
\pgfsetlinewidth{0.100000\du}
\pgfsetdash{}{0pt}
\pgfsetbuttcap
\pgfsetmiterjoin
\pgfsetlinewidth{0.100000\du}
\pgfsetbuttcap
\pgfsetmiterjoin
\pgfsetdash{}{0pt}
\definecolor{diafillcolor}{rgb}{1.000000, 1.000000, 1.000000}
\pgfsetfillcolor{diafillcolor}
\pgfsetfillopacity{1.000000}
\definecolor{dialinecolor}{rgb}{0.000000, 0.000000, 0.000000}
\pgfsetstrokecolor{dialinecolor}
\pgfsetstrokeopacity{1.000000}
\pgfpathmoveto{\pgfpoint{21.153600\du}{24.632500\du}}
\pgfpathlineto{\pgfpoint{35.633600\du}{24.632500\du}}
\pgfpathcurveto{\pgfpoint{37.632872\du}{24.632500\du}}{\pgfpoint{39.253600\du}{24.878743\du}}{\pgfpoint{39.253600\du}{25.182500\du}}
\pgfpathcurveto{\pgfpoint{39.253600\du}{25.486257\du}}{\pgfpoint{37.632872\du}{25.732500\du}}{\pgfpoint{35.633600\du}{25.732500\du}}
\pgfpathlineto{\pgfpoint{21.153600\du}{25.732500\du}}
\pgfpathcurveto{\pgfpoint{19.154328\du}{25.732500\du}}{\pgfpoint{17.533600\du}{25.486257\du}}{\pgfpoint{17.533600\du}{25.182500\du}}
\pgfpathcurveto{\pgfpoint{17.533600\du}{24.878743\du}}{\pgfpoint{19.154328\du}{24.632500\du}}{\pgfpoint{21.153600\du}{24.632500\du}}
\pgfpathclose
\pgfusepath{fill,stroke}
% setfont left to latex
\definecolor{dialinecolor}{rgb}{0.000000, 0.000000, 0.000000}
\pgfsetstrokecolor{dialinecolor}
\pgfsetstrokeopacity{1.000000}
\definecolor{diafillcolor}{rgb}{0.000000, 0.000000, 0.000000}
\pgfsetfillcolor{diafillcolor}
\pgfsetfillopacity{1.000000}
\node[anchor=base,inner sep=0pt, outer sep=0pt,color=dialinecolor] at (28.393600\du,25.382500\du){6.1. ordonner les adresses IP de chaque lien};
\pgfsetlinewidth{0.100000\du}
\pgfsetdash{}{0pt}
\pgfsetmiterjoin
{\pgfsetcornersarced{\pgfpoint{0.000000\du}{0.000000\du}}\definecolor{diafillcolor}{rgb}{1.000000, 1.000000, 1.000000}
\pgfsetfillcolor{diafillcolor}
\pgfsetfillopacity{1.000000}
\fill (11.729100\du,30.612100\du)--(11.729100\du,32.512100\du)--(18.496600\du,32.512100\du)--(18.496600\du,30.612100\du)--cycle;
}{\pgfsetcornersarced{\pgfpoint{0.000000\du}{0.000000\du}}\definecolor{dialinecolor}{rgb}{0.000000, 0.000000, 0.000000}
\pgfsetstrokecolor{dialinecolor}
\pgfsetstrokeopacity{1.000000}
\draw (11.729100\du,30.612100\du)--(11.729100\du,32.512100\du)--(18.496600\du,32.512100\du)--(18.496600\du,30.612100\du)--cycle;
}% setfont left to latex
\definecolor{dialinecolor}{rgb}{0.000000, 0.000000, 0.000000}
\pgfsetstrokecolor{dialinecolor}
\pgfsetstrokeopacity{1.000000}
\definecolor{diafillcolor}{rgb}{0.000000, 0.000000, 0.000000}
\pgfsetfillcolor{diafillcolor}
\pgfsetfillopacity{1.000000}
\node[anchor=base,inner sep=0pt, outer sep=0pt,color=dialinecolor] at (15.112850\du,31.757100\du){(LinkIPs, \ensuremath{[}DiffRtt\ensuremath{]})};
\pgfsetlinewidth{0.100000\du}
\pgfsetdash{}{0pt}
\pgfsetbuttcap
\pgfsetmiterjoin
\pgfsetlinewidth{0.100000\du}
\pgfsetbuttcap
\pgfsetmiterjoin
\pgfsetdash{}{0pt}
\definecolor{diafillcolor}{rgb}{1.000000, 1.000000, 1.000000}
\pgfsetfillcolor{diafillcolor}
\pgfsetfillopacity{1.000000}
\definecolor{dialinecolor}{rgb}{0.000000, 0.000000, 0.000000}
\pgfsetstrokecolor{dialinecolor}
\pgfsetstrokeopacity{1.000000}
\pgfpathmoveto{\pgfpoint{19.696075\du}{33.034600\du}}
\pgfpathlineto{\pgfpoint{31.713575\du}{33.034600\du}}
\pgfpathcurveto{\pgfpoint{33.372846\du}{33.034600\du}}{\pgfpoint{34.717950\du}{33.280843\du}}{\pgfpoint{34.717950\du}{33.584600\du}}
\pgfpathcurveto{\pgfpoint{34.717950\du}{33.888357\du}}{\pgfpoint{33.372846\du}{34.134600\du}}{\pgfpoint{31.713575\du}{34.134600\du}}
\pgfpathlineto{\pgfpoint{19.696075\du}{34.134600\du}}
\pgfpathcurveto{\pgfpoint{18.036804\du}{34.134600\du}}{\pgfpoint{16.691700\du}{33.888357\du}}{\pgfpoint{16.691700\du}{33.584600\du}}
\pgfpathcurveto{\pgfpoint{16.691700\du}{33.280843\du}}{\pgfpoint{18.036804\du}{33.034600\du}}{\pgfpoint{19.696075\du}{33.034600\du}}
\pgfpathclose
\pgfusepath{fill,stroke}
% setfont left to latex
\definecolor{dialinecolor}{rgb}{0.000000, 0.000000, 0.000000}
\pgfsetstrokecolor{dialinecolor}
\pgfsetstrokeopacity{1.000000}
\definecolor{diafillcolor}{rgb}{0.000000, 0.000000, 0.000000}
\pgfsetfillcolor{diafillcolor}
\pgfsetfillopacity{1.000000}
\node[anchor=base,inner sep=0pt, outer sep=0pt,color=dialinecolor] at (25.704825\du,33.784600\du){6.3. organisation les dates des liens };
\pgfsetlinewidth{0.100000\du}
\pgfsetdash{}{0pt}
\pgfsetbuttcap
{
\definecolor{diafillcolor}{rgb}{0.000000, 0.000000, 0.000000}
\pgfsetfillcolor{diafillcolor}
\pgfsetfillopacity{1.000000}
% was here!!!
\pgfsetarrowsend{stealth}
\definecolor{dialinecolor}{rgb}{0.000000, 0.000000, 0.000000}
\pgfsetstrokecolor{dialinecolor}
\pgfsetstrokeopacity{1.000000}
\draw (15.069300\du,27.946500\du)--(15.112800\du,30.612100\du);
}
\pgfsetlinewidth{0.100000\du}
\pgfsetdash{}{0pt}
\pgfsetmiterjoin
{\pgfsetcornersarced{\pgfpoint{0.000000\du}{0.000000\du}}\definecolor{diafillcolor}{rgb}{1.000000, 1.000000, 1.000000}
\pgfsetfillcolor{diafillcolor}
\pgfsetfillopacity{1.000000}
\fill (12.209300\du,35.321700\du)--(12.209300\du,37.221700\du)--(18.084300\du,37.221700\du)--(18.084300\du,35.321700\du)--cycle;
}{\pgfsetcornersarced{\pgfpoint{0.000000\du}{0.000000\du}}\definecolor{dialinecolor}{rgb}{0.000000, 0.000000, 0.000000}
\pgfsetstrokecolor{dialinecolor}
\pgfsetstrokeopacity{1.000000}
\draw (12.209300\du,35.321700\du)--(12.209300\du,37.221700\du)--(18.084300\du,37.221700\du)--(18.084300\du,35.321700\du)--cycle;
}% setfont left to latex
\definecolor{dialinecolor}{rgb}{0.000000, 0.000000, 0.000000}
\pgfsetstrokecolor{dialinecolor}
\pgfsetstrokeopacity{1.000000}
\definecolor{diafillcolor}{rgb}{0.000000, 0.000000, 0.000000}
\pgfsetfillcolor{diafillcolor}
\pgfsetfillopacity{1.000000}
\node[anchor=base,inner sep=0pt, outer sep=0pt,color=dialinecolor] at (15.146800\du,36.466700\du){\ensuremath{[}DiffRTTPeriod\ensuremath{]}};
\pgfsetlinewidth{0.100000\du}
\pgfsetdash{}{0pt}
\pgfsetbuttcap
{
\definecolor{diafillcolor}{rgb}{0.000000, 0.000000, 0.000000}
\pgfsetfillcolor{diafillcolor}
\pgfsetfillopacity{1.000000}
% was here!!!
\pgfsetarrowsend{stealth}
\definecolor{dialinecolor}{rgb}{0.000000, 0.000000, 0.000000}
\pgfsetstrokecolor{dialinecolor}
\pgfsetstrokeopacity{1.000000}
\draw (15.112800\du,32.512100\du)--(15.146800\du,35.321700\du);
}
\pgfsetlinewidth{0.100000\du}
\pgfsetdash{}{0pt}
\pgfsetbuttcap
\pgfsetmiterjoin
\pgfsetlinewidth{0.100000\du}
\pgfsetbuttcap
\pgfsetmiterjoin
\pgfsetdash{}{0pt}
\definecolor{diafillcolor}{rgb}{1.000000, 1.000000, 1.000000}
\pgfsetfillcolor{diafillcolor}
\pgfsetfillopacity{1.000000}
\definecolor{dialinecolor}{rgb}{0.000000, 0.000000, 0.000000}
\pgfsetstrokecolor{dialinecolor}
\pgfsetstrokeopacity{1.000000}
\pgfpathmoveto{\pgfpoint{20.708450\du}{28.664600\du}}
\pgfpathlineto{\pgfpoint{33.403450\du}{28.664600\du}}
\pgfpathcurveto{\pgfpoint{35.156265\du}{28.664600\du}}{\pgfpoint{36.577200\du}{28.910843\du}}{\pgfpoint{36.577200\du}{29.214600\du}}
\pgfpathcurveto{\pgfpoint{36.577200\du}{29.518357\du}}{\pgfpoint{35.156265\du}{29.764600\du}}{\pgfpoint{33.403450\du}{29.764600\du}}
\pgfpathlineto{\pgfpoint{20.708450\du}{29.764600\du}}
\pgfpathcurveto{\pgfpoint{18.955635\du}{29.764600\du}}{\pgfpoint{17.534700\du}{29.518357\du}}{\pgfpoint{17.534700\du}{29.214600\du}}
\pgfpathcurveto{\pgfpoint{17.534700\du}{28.910843\du}}{\pgfpoint{18.955635\du}{28.664600\du}}{\pgfpoint{20.708450\du}{28.664600\du}}
\pgfpathclose
\pgfusepath{fill,stroke}
% setfont left to latex
\definecolor{dialinecolor}{rgb}{0.000000, 0.000000, 0.000000}
\pgfsetstrokecolor{dialinecolor}
\pgfsetstrokeopacity{1.000000}
\definecolor{diafillcolor}{rgb}{0.000000, 0.000000, 0.000000}
\pgfsetfillcolor{diafillcolor}
\pgfsetfillopacity{1.000000}
\node[anchor=base,inner sep=0pt, outer sep=0pt,color=dialinecolor] at (27.055950\du,29.414600\du){6.2. aggregation des données par lien };
\pgfsetlinewidth{0.100000\du}
\pgfsetdash{}{0pt}
\pgfsetmiterjoin
{\pgfsetcornersarced{\pgfpoint{0.000000\du}{0.000000\du}}\definecolor{diafillcolor}{rgb}{1.000000, 1.000000, 1.000000}
\pgfsetfillcolor{diafillcolor}
\pgfsetfillopacity{1.000000}
\fill (11.630600\du,39.879700\du)--(11.630600\du,41.779700\du)--(18.663100\du,41.779700\du)--(18.663100\du,39.879700\du)--cycle;
}{\pgfsetcornersarced{\pgfpoint{0.000000\du}{0.000000\du}}\definecolor{dialinecolor}{rgb}{0.000000, 0.000000, 0.000000}
\pgfsetstrokecolor{dialinecolor}
\pgfsetstrokeopacity{1.000000}
\draw (11.630600\du,39.879700\du)--(11.630600\du,41.779700\du)--(18.663100\du,41.779700\du)--(18.663100\du,39.879700\du)--cycle;
}% setfont left to latex
\definecolor{dialinecolor}{rgb}{0.000000, 0.000000, 0.000000}
\pgfsetstrokecolor{dialinecolor}
\pgfsetstrokeopacity{1.000000}
\definecolor{diafillcolor}{rgb}{0.000000, 0.000000, 0.000000}
\pgfsetfillcolor{diafillcolor}
\pgfsetfillopacity{1.000000}
\node[anchor=base,inner sep=0pt, outer sep=0pt,color=dialinecolor] at (15.146850\du,41.024700\du){current : LinkState};
\pgfsetlinewidth{0.100000\du}
\pgfsetdash{}{0pt}
\pgfsetmiterjoin
{\pgfsetcornersarced{\pgfpoint{0.000000\du}{0.000000\du}}\definecolor{diafillcolor}{rgb}{1.000000, 1.000000, 1.000000}
\pgfsetfillcolor{diafillcolor}
\pgfsetfillopacity{1.000000}
\fill (11.275000\du,44.391900\du)--(11.275000\du,46.291900\du)--(19.057500\du,46.291900\du)--(19.057500\du,44.391900\du)--cycle;
}{\pgfsetcornersarced{\pgfpoint{0.000000\du}{0.000000\du}}\definecolor{dialinecolor}{rgb}{0.000000, 0.000000, 0.000000}
\pgfsetstrokecolor{dialinecolor}
\pgfsetstrokeopacity{1.000000}
\draw (11.275000\du,44.391900\du)--(11.275000\du,46.291900\du)--(19.057500\du,46.291900\du)--(19.057500\du,44.391900\du)--cycle;
}% setfont left to latex
\definecolor{dialinecolor}{rgb}{0.000000, 0.000000, 0.000000}
\pgfsetstrokecolor{dialinecolor}
\pgfsetstrokeopacity{1.000000}
\definecolor{diafillcolor}{rgb}{0.000000, 0.000000, 0.000000}
\pgfsetfillcolor{diafillcolor}
\pgfsetfillopacity{1.000000}
\node[anchor=base,inner sep=0pt, outer sep=0pt,color=dialinecolor] at (15.166250\du,45.536900\du){reference : LinkState};
\pgfsetlinewidth{0.100000\du}
\pgfsetdash{}{0pt}
\pgfsetbuttcap
\pgfsetmiterjoin
\pgfsetlinewidth{0.100000\du}
\pgfsetbuttcap
\pgfsetmiterjoin
\pgfsetdash{}{0pt}
\definecolor{diafillcolor}{rgb}{1.000000, 1.000000, 1.000000}
\pgfsetfillcolor{diafillcolor}
\pgfsetfillopacity{1.000000}
\definecolor{dialinecolor}{rgb}{0.000000, 0.000000, 0.000000}
\pgfsetstrokecolor{dialinecolor}
\pgfsetstrokeopacity{1.000000}
\pgfpathmoveto{\pgfpoint{22.357225\du}{38.015300\du}}
\pgfpathlineto{\pgfpoint{40.139725\du}{38.015300\du}}
\pgfpathcurveto{\pgfpoint{42.594977\du}{38.015300\du}}{\pgfpoint{44.585350\du}{38.261543\du}}{\pgfpoint{44.585350\du}{38.565300\du}}
\pgfpathcurveto{\pgfpoint{44.585350\du}{38.869057\du}}{\pgfpoint{42.594977\du}{39.115300\du}}{\pgfpoint{40.139725\du}{39.115300\du}}
\pgfpathlineto{\pgfpoint{22.357225\du}{39.115300\du}}
\pgfpathcurveto{\pgfpoint{19.901973\du}{39.115300\du}}{\pgfpoint{17.911600\du}{38.869057\du}}{\pgfpoint{17.911600\du}{38.565300\du}}
\pgfpathcurveto{\pgfpoint{17.911600\du}{38.261543\du}}{\pgfpoint{19.901973\du}{38.015300\du}}{\pgfpoint{22.357225\du}{38.015300\du}}
\pgfpathclose
\pgfusepath{fill,stroke}
% setfont left to latex
\definecolor{dialinecolor}{rgb}{0.000000, 0.000000, 0.000000}
\pgfsetstrokecolor{dialinecolor}
\pgfsetstrokeopacity{1.000000}
\definecolor{diafillcolor}{rgb}{0.000000, 0.000000, 0.000000}
\pgfsetfillcolor{diafillcolor}
\pgfsetfillopacity{1.000000}
\node[anchor=base,inner sep=0pt, outer sep=0pt,color=dialinecolor] at (31.248475\du,38.765300\du){7. pour chaque période calculer l'état courant du lien};
\pgfsetlinewidth{0.100000\du}
\pgfsetdash{}{0pt}
\pgfsetbuttcap
{
\definecolor{diafillcolor}{rgb}{0.000000, 0.000000, 0.000000}
\pgfsetfillcolor{diafillcolor}
\pgfsetfillopacity{1.000000}
% was here!!!
\pgfsetarrowsend{stealth}
\definecolor{dialinecolor}{rgb}{0.000000, 0.000000, 0.000000}
\pgfsetstrokecolor{dialinecolor}
\pgfsetstrokeopacity{1.000000}
\draw (15.146800\du,37.221700\du)--(15.146900\du,39.879700\du);
}
\pgfsetlinewidth{0.100000\du}
\pgfsetdash{}{0pt}
\pgfsetbuttcap
{
\definecolor{diafillcolor}{rgb}{0.000000, 0.000000, 0.000000}
\pgfsetfillcolor{diafillcolor}
\pgfsetfillopacity{1.000000}
% was here!!!
\pgfsetarrowsend{stealth}
\definecolor{dialinecolor}{rgb}{0.000000, 0.000000, 0.000000}
\pgfsetstrokecolor{dialinecolor}
\pgfsetstrokeopacity{1.000000}
\draw (15.146900\du,41.779700\du)--(15.166200\du,44.391900\du);
}
\pgfsetlinewidth{0.100000\du}
\pgfsetdash{}{0pt}
\pgfsetbuttcap
\pgfsetmiterjoin
\pgfsetlinewidth{0.100000\du}
\pgfsetbuttcap
\pgfsetmiterjoin
\pgfsetdash{}{0pt}
\definecolor{diafillcolor}{rgb}{1.000000, 1.000000, 1.000000}
\pgfsetfillcolor{diafillcolor}
\pgfsetfillopacity{1.000000}
\definecolor{dialinecolor}{rgb}{0.000000, 0.000000, 0.000000}
\pgfsetstrokecolor{dialinecolor}
\pgfsetstrokeopacity{1.000000}
\pgfpathmoveto{\pgfpoint{23.185050\du}{42.538700\du}}
\pgfpathlineto{\pgfpoint{43.610050\du}{42.538700\du}}
\pgfpathcurveto{\pgfpoint{46.430155\du}{42.538700\du}}{\pgfpoint{48.716300\du}{42.784943\du}}{\pgfpoint{48.716300\du}{43.088700\du}}
\pgfpathcurveto{\pgfpoint{48.716300\du}{43.392457\du}}{\pgfpoint{46.430155\du}{43.638700\du}}{\pgfpoint{43.610050\du}{43.638700\du}}
\pgfpathlineto{\pgfpoint{23.185050\du}{43.638700\du}}
\pgfpathcurveto{\pgfpoint{20.364945\du}{43.638700\du}}{\pgfpoint{18.078800\du}{43.392457\du}}{\pgfpoint{18.078800\du}{43.088700\du}}
\pgfpathcurveto{\pgfpoint{18.078800\du}{42.784943\du}}{\pgfpoint{20.364945\du}{42.538700\du}}{\pgfpoint{23.185050\du}{42.538700\du}}
\pgfpathclose
\pgfusepath{fill,stroke}
% setfont left to latex
\definecolor{dialinecolor}{rgb}{0.000000, 0.000000, 0.000000}
\pgfsetstrokecolor{dialinecolor}
\pgfsetstrokeopacity{1.000000}
\definecolor{diafillcolor}{rgb}{0.000000, 0.000000, 0.000000}
\pgfsetfillcolor{diafillcolor}
\pgfsetfillopacity{1.000000}
\node[anchor=base,inner sep=0pt, outer sep=0pt,color=dialinecolor] at (33.397550\du,43.288700\du){8. comparaison des états du lien et mise à jour de la référence};
\pgfsetlinewidth{0.100000\du}
\pgfsetdash{}{0pt}
\pgfsetmiterjoin
{\pgfsetcornersarced{\pgfpoint{0.000000\du}{0.000000\du}}\definecolor{diafillcolor}{rgb}{1.000000, 1.000000, 1.000000}
\pgfsetfillcolor{diafillcolor}
\pgfsetfillopacity{1.000000}
\fill (13.357800\du,47.900000\du)--(13.357800\du,49.800000\du)--(17.127800\du,49.800000\du)--(17.127800\du,47.900000\du)--cycle;
}{\pgfsetcornersarced{\pgfpoint{0.000000\du}{0.000000\du}}\definecolor{dialinecolor}{rgb}{0.000000, 0.000000, 0.000000}
\pgfsetstrokecolor{dialinecolor}
\pgfsetstrokeopacity{1.000000}
\draw (13.357800\du,47.900000\du)--(13.357800\du,49.800000\du)--(17.127800\du,49.800000\du)--(17.127800\du,47.900000\du)--cycle;
}% setfont left to latex
\definecolor{dialinecolor}{rgb}{0.000000, 0.000000, 0.000000}
\pgfsetstrokecolor{dialinecolor}
\pgfsetstrokeopacity{1.000000}
\definecolor{diafillcolor}{rgb}{0.000000, 0.000000, 0.000000}
\pgfsetfillcolor{diafillcolor}
\pgfsetfillopacity{1.000000}
\node[anchor=base,inner sep=0pt, outer sep=0pt,color=dialinecolor] at (15.242800\du,49.045000\du){\ensuremath{[}alarm\ensuremath{]}};
\pgfsetlinewidth{0.100000\du}
\pgfsetdash{}{0pt}
\pgfsetbuttcap
{
\definecolor{diafillcolor}{rgb}{0.000000, 0.000000, 0.000000}
\pgfsetfillcolor{diafillcolor}
\pgfsetfillopacity{1.000000}
% was here!!!
\pgfsetarrowsend{stealth}
\definecolor{dialinecolor}{rgb}{0.000000, 0.000000, 0.000000}
\pgfsetstrokecolor{dialinecolor}
\pgfsetstrokeopacity{1.000000}
\draw (15.196171\du,46.341782\du)--(15.242800\du,47.900000\du);
}
\pgfsetlinewidth{0.100000\du}
\pgfsetdash{}{0pt}
\pgfsetbuttcap
\pgfsetmiterjoin
\pgfsetlinewidth{0.100000\du}
\pgfsetbuttcap
\pgfsetmiterjoin
\pgfsetdash{}{0pt}
\definecolor{diafillcolor}{rgb}{1.000000, 1.000000, 1.000000}
\pgfsetfillcolor{diafillcolor}
\pgfsetfillopacity{1.000000}
\definecolor{dialinecolor}{rgb}{0.000000, 0.000000, 0.000000}
\pgfsetstrokecolor{dialinecolor}
\pgfsetstrokeopacity{1.000000}
\pgfpathmoveto{\pgfpoint{20.591200\du}{46.464400\du}}
\pgfpathlineto{\pgfpoint{29.241200\du}{46.464400\du}}
\pgfpathcurveto{\pgfpoint{30.435516\du}{46.464400\du}}{\pgfpoint{31.403700\du}{46.710643\du}}{\pgfpoint{31.403700\du}{47.014400\du}}
\pgfpathcurveto{\pgfpoint{31.403700\du}{47.318157\du}}{\pgfpoint{30.435516\du}{47.564400\du}}{\pgfpoint{29.241200\du}{47.564400\du}}
\pgfpathlineto{\pgfpoint{20.591200\du}{47.564400\du}}
\pgfpathcurveto{\pgfpoint{19.396884\du}{47.564400\du}}{\pgfpoint{18.428700\du}{47.318157\du}}{\pgfpoint{18.428700\du}{47.014400\du}}
\pgfpathcurveto{\pgfpoint{18.428700\du}{46.710643\du}}{\pgfpoint{19.396884\du}{46.464400\du}}{\pgfpoint{20.591200\du}{46.464400\du}}
\pgfpathclose
\pgfusepath{fill,stroke}
% setfont left to latex
\definecolor{dialinecolor}{rgb}{0.000000, 0.000000, 0.000000}
\pgfsetstrokecolor{dialinecolor}
\pgfsetstrokeopacity{1.000000}
\definecolor{diafillcolor}{rgb}{0.000000, 0.000000, 0.000000}
\pgfsetfillcolor{diafillcolor}
\pgfsetfillopacity{1.000000}
\node[anchor=base,inner sep=0pt, outer sep=0pt,color=dialinecolor] at (24.916200\du,47.214400\du){9. détection des alarmes};
% setfont left to latex
\definecolor{dialinecolor}{rgb}{0.000000, 0.000000, 0.000000}
\pgfsetstrokecolor{dialinecolor}
\pgfsetstrokeopacity{1.000000}
\definecolor{diafillcolor}{rgb}{0.000000, 0.000000, 0.000000}
\pgfsetfillcolor{diafillcolor}
\pgfsetfillopacity{1.000000}
\node[anchor=base west,inner sep=0pt,outer sep=0pt,color=dialinecolor] at (19.550000\du,2.533740\du){Chargement des traceroutes de la période concernée};
\pgfsetlinewidth{0.100000\du}
\pgfsetdash{}{0pt}
\pgfsetbuttcap
{
\definecolor{diafillcolor}{rgb}{0.000000, 0.000000, 0.000000}
\pgfsetfillcolor{diafillcolor}
\pgfsetfillopacity{1.000000}
% was here!!!
\pgfsetarrowsend{to}
\definecolor{dialinecolor}{rgb}{0.000000, 0.000000, 0.000000}
\pgfsetstrokecolor{dialinecolor}
\pgfsetstrokeopacity{1.000000}
\pgfpathmoveto{\pgfpoint{19.218821\du}{0.775034\du}}
\pgfpatharc{508}{212}{0.914891\du and 0.914891\du}
\pgfusepath{stroke}
}
% setfont left to latex
\definecolor{dialinecolor}{rgb}{0.000000, 0.000000, 0.000000}
\pgfsetstrokecolor{dialinecolor}
\pgfsetstrokeopacity{1.000000}
\definecolor{diafillcolor}{rgb}{0.000000, 0.000000, 0.000000}
\pgfsetfillcolor{diafillcolor}
\pgfsetfillopacity{1.000000}
\node[anchor=base west,inner sep=0pt,outer sep=0pt,color=dialinecolor] at (20.365600\du,4.846460\du){};
% setfont left to latex
\definecolor{dialinecolor}{rgb}{0.000000, 0.000000, 0.000000}
\pgfsetstrokecolor{dialinecolor}
\pgfsetstrokeopacity{1.000000}
\definecolor{diafillcolor}{rgb}{0.000000, 0.000000, 0.000000}
\pgfsetfillcolor{diafillcolor}
\pgfsetfillopacity{1.000000}
\node[anchor=base west,inner sep=0pt,outer sep=0pt,color=dialinecolor] at (23.265600\du,4.646460\du){Illustration des opérations sur une instance de TraceroutesPerPeriod };
% setfont left to latex
\definecolor{dialinecolor}{rgb}{0.000000, 0.000000, 0.000000}
\pgfsetstrokecolor{dialinecolor}
\pgfsetstrokeopacity{1.000000}
\definecolor{diafillcolor}{rgb}{0.000000, 0.000000, 0.000000}
\pgfsetfillcolor{diafillcolor}
\pgfsetfillopacity{1.000000}
\node[anchor=base west,inner sep=0pt,outer sep=0pt,color=dialinecolor] at (9.738010\du,5.397040\du){=};
\end{tikzpicture}

	}
	\caption{Le processus de la détection des anomalies dans les délais des liens}
	\label{fig:process-rttanalysis_tex}
\end{figure}



\section{La caractérisation des anomalies dans les délais d'un lien}
La section \ref{rttevolution} décrit la détection des anomalies dans les liens à travers l'évolutions des RTTs différentiels d'un lien tout au long d'une période donnée. Toutefois, il existe une autre exploitation des traceroutes pour la détection des anomalies similaire à celle présentée précédemment, la différence se voit au niveau l'importance d'un lien. Ainsi, on analyse pas tout liens, plutôt les liens ayant certaine diversité en matière d'ASs ayant identifié ce lien, le nombre de sondes ayant identifié ce lien, etc. 

\chapter{Introduction au Big Data} \label{chap:big-data-intro}
	\section{Introduction}
		Big Data est un terme associé aux données massives, rapidement générées, ayant une grande diversité que les  outils traditionnels sont incapables de gérer ces données. En Big Data,  tout type de données peut être utilisé, et ce en vue de livrer la bonne information à la bonne personne et au bon moment dans le but d'aider à prendre les bonnes décisions. Dans ce chapitre, nous  présentons brièvement un des processus d'analyse de données. Ensuite, nous décrivons quelques concepts impliqués dans un processus d'analyse des données massives. Enfin, nous parcourons un ensemble de technologies utilisées dans l'analyse des données à grande échelle.
			
	\section{Processus d'analyse de données massives}
	%668
Les étapes d'un processus d'analyse de données, à grande échelle ou non, sont différentes selon le processus adopté.  Ce dernier peut impliquer plusieurs besoins et concepts. En particulier, on note le besoin du stockage de données massives, du traitement de données massives et le besoin de la visualisation des résultats des traitements appliqués. Le choix du traitement à appliquer sur les données est dirigé par les objectifs de  l'analyse menée. Par exemple, une analyse peut envisager la création d'un système de recommandations dans les sites Internet, la conception d'un outil de prédiction basé sur les données historiées, un système de suivi en temps réel, etc.  Ces applications appuient sur des algorithmes basés sur les mathématiques, en particulier les mathématiques statistiques et probabilistes. Afin d'assurer l'efficacité de l'analyse de données massives, la coopération de plusieurs ressources accélère considérablement les étapes de l'analyse, c'est ce que l'informatique distribuée a apporté.
\paragraph{Exemple d'un  processus d'analyse de données} \label{sec:process-data-analysis}~

Le processus d'analyse de données passe généralement par des étapes classiques. A ces étapes peuvent s'ajouter d'autres étapes intermédiaires ou supplémentaires. 
Pour précision, non seulement  ces étapes peuvent s'appliquer dans le cas  où les données sont volumineuses,  mais aussi dans le cas des données en provenance des outils traditionnels comme les bases de données relationnelles. IBM Knowledge Center \footnote{Source : \url{https://www.ibm.com/support/knowledgecenter/fr/SSEPGG_9.5.0/com.ibm.im.easy.doc/c_dm_process.html}, consultée le $06/08/2018$.} a résumé le processus de l'analyse de données dans les étapes suivantes :
définition de problème, exploration de données, préparation de données, modélisation, évaluation et enfin l'étape de déploiement.


\textbf{Définition de problème} C'est le point d'entrée vers tout projet d'analyse de données. Les objectifs doivent être clairs. A la fin de cette étape, on connait les objectifs de l'analyse, mais pas forcément comment cette analyse va être menée en terme de technologies utilisées,  les méthodes statistiques, etc. \par

\textbf{Exploration de données } Ce que caractérise cette étape, c'est la découverte de la nature  des données, les sources des données, la qualité des données, etc. \par

\textbf{Préparation de données} \label{par:step-preparedata} C'est durant cette phase que se passent les opérations d'ETL (Extract-Transform-Load), ce sont les opérations de transformation, de nettoyage et de chargement des données vers les entrepôts de données (voir ETL dans la section \ref{sec:etl}). Durant cette étape,  ce ne sont que  des tentatives  faites pour créer le schéma de données.  Dans certains cas, cela amène  à chercher des sources complémentaires de données ou bien à appliquer de nouvelles transformations  sur les données. \par
%Tel qu'un  schéma comporte les différentes tables, les attributs pertinents et autres. 

\textbf{Modélisation} A l'issue de cette étape, un modèle de qualité est créé. Ce modèle  doit  répondre aux objectifs précédemment établis dans l'étape de la définition du problème. Cette étape implique des principes en statistiques, probabilités, \textit{machine learning}, etc. \par

\textbf{Evaluation} C'est à cette étape qu'on valide le modèle créé à l'étape de  Modélisation. L'évaluation et la validation du modèle prennent en compte les objectifs prédéfinis; si le modèle répond aux attentes précédemment exprimées. \par

\textbf{Déploiement}  C'est l'utilisation  des outils existants ou  la mise en place d'une solution permettant d'exploiter les résultats obtenus lors de l'application du modèle créé sur les données disponibles. Par exemple, la  visualisation des résultats de l'analyse  sous forme d'un tableau d'indicateurs sur une page Web. \par
\section{Quelques concepts associés au Big Data}
Etant donné que le domaine du Big Data implique plusieurs concepts, nous présentons une liste de concepts  non exhaustive. 
\subsection{Définition du Big Data : Volume, Vélocité, Variété et Véracité}
		IBM définit le Big Data suivant les quatre dimensions suivants : volume, variété, vélocité et véracité. 
		\paragraph{Volume de données}~
		La quantité de données manipulées par les outils traditionnels de la gestion des données est de l'ordre de gigaoctets (GO) et de téraoctets (TO). Toutefois, le Big Data est mesuré en pétaoctets (PO), exaoctets (EO), voire plus. Une des premières applications du Big Data est la recherche dans Word-Wide Web (WWW). Selon l'étude (\cite{6567202}, $2013$) de l'International Data Corporation (IDC), le volume de données  va atteindre $40$ zettaoctets\footnote{ $ 1$ ZO = $1,000,000,000,000$ GO. } par  entreprise en $2020$. Dans un rapport récent (\cite{IDC-data}, $ 2018 $), 
		IDC prédit que le volume de la  datasphère  planétaire, c'est-à-dire le volume de nouvelles données créées et répliquées chaque année, appelée  Global Datasphere en anglais,  passera de $  33 $ ZO en $2018$ à $175$ ZO d'ici $ 2025 $.
		
		\paragraph{Vélocité de données} ~
		le Big Data est généré par des milliards
		d'appareils. Les données générées sont communiquées avec la vitesse de la lumière via l'Internet. L'augmentation de la vitesse de l'Internet est une  des raisons ayant contribué à l'augmentation de la vitesse de la génération de données.	
		%la vitesse décrit la fréquence de la génération et du partage des données. Autrement dit, les données sont générées rapidement et doivent être traitées rapidement pour extraire les informations  pertinentes.
		Par exemple, Wallmart (international discount retail chain) génère environ   $2.5$ pétaoctets de données chaque  heure via les transactions de ses consommateurs\footnote{Source : \url{https://www.bernardmarr.com/default.asp?contentID=690}, consultée le $ 30/06/2018 $.}.
		\paragraph{Variété de données} \label{variete-data}
		
		Le Big Data inclut toutes les formes de données, des fonctions diversifiés des données et des sources variées des données.
		
		Le premier aspect de la variété  des données massives est la \textbf{forme} de celles-ci. Les données manipulées incluent du texte, des graphes, des cartes, des vidéos, des photos, etc.
		
		Le deuxième aspect de la variété des données massives concerne les \textbf{fonctions} assurées par ces données. Des données sont issues des conversations humaines, d'autres des transactions des consommateurs, ou bien des données archivées, etc.
		
		Les \textbf{sources} du Big Data est le troisième aspect de  variété. Des données sont en provenance des téléphones mobiles, des tablettes ou des ordinateurs portables, des fichiers journaux, des réseaux de capteurs, etc. Les sources du Big Data peuvent être classées en trois grandes catégories : communications \textbf{\textit{human to human}} comme les conversations échangées dans les réseaux sociaux, communications \textbf{\textit{ human to machine}} comme l'accès des utilisateurs aux données dans le Web et enfin communications \textbf{\textit{machine to machine}} comme les données issues de la communication entre les capteurs dans un réseau de capteurs.
		%au-delà des données des transactions habituelles des consommateurs, les données manipulées sont en provenance de sources différentes et de formats différents. Par exemple: des vidéos, des documents, des commentaires, des données géospatiales, des journaux, etc. Les grands ensembles de données sont constitués de données structurées et non structurées, publiques ou privées, locales ou distantes, partagées ou confidentielles, complètes ou incomplètes, etc.
		\paragraph{Véracité de données}
		
		La véracité concerne la crédibilité et la qualité de données. La mauvaise qualité de données est due à de nombreuses raisons, telles que les pannes techniques comme le dysfonctionnement des appareils comme les capteurs, les erreurs humaines,  etc. De plus, les données peuvent être intentionnellement erronées pour des raisons de concurrence ou des raisons stratégiques.
	
		\subsection{L'architecture standard du Big Data}
		
		L'architecture standard du Big Data présentée dans ce qui suit est celle proposée dans  \cite{anil-big-data}. Cette architecture  est composée des couches suivantes :  \textit{Data sources}, \textit{Data Ingest}, \textit{Batch processing}, \textit{Stream Processing}, \textit{Data organizing}, \textit{Infrastructure}, \textit{Distributed File System} et enfin la couche \textit{Data consuption}. 
		%Pour  chaque couche, il existe au moins une technologie.
		 La Figure \ref{fig:bigdata-architecture} reprend l'organisation des différentes couches de cette architecture.
		 		\begin{figure}[h]
		 	\captionsetup{justification = centering}
		 	\centering
		 	\includegraphics[width=\linewidth]{illustrations/bigdata-architecture}
		 	\caption{Architecture standard du Big Data. Source :  \cite{anil-big-data}}
		 	\label{fig:bigdata-architecture}
		 \end{figure}
		
		Dans un premier temps, les données sont   accueillies   via la couche  \textit{ingest system} par diverses sources de données. Ensuite, les données sont traitées selon deux modes :  \textit{stream processing } et  \textit{batch processing}.  Les résultats de ce traitement peuvent être envoyés vers les bases de données NoSQL (voir la section \ref{sec:nosql}) pour une utilisation ultérieure, ou bien être utilisés  comme entrées pour d'autres applications.  Une solution Big Data comprend typiquement  ces  couches logiques. Chaque  couche peut être représentée par une ou plusieurs technologies disponibles. Reprenons chaque couche logique:
		
		
		%\begin{description}
			\paragraph{Data sources layer} Le choix des sources de donnés pour une application donnée dépend des objectifs qui dirigent l'analyse en question. Les sources avec leurs différents aspects sont détaillées dans la section \ref{variete-data}.
		\paragraph{Data Ingest layer} Cette couche permet de récupérer les données depuis les différentes sources de données. Les données sont accueillies à travers des points d'entrées multiples. Ces points  sont capables de recevoir  ces données ayant une vélocité variable ainsi qu'une quantité aussi variable.  Après avoir traversé  la couche \textit{Data Ingest}, les données sont envoyées au \textit{batch processing system}, au \textit{stream processing system}, ou bien à un système de stockage particulier.
		\paragraph{Batch processing layer} Les données reçues sur cette couche sont celles en provenance du \textit{Data Ingest} ou bien d'une des bases de données NoSQL. Ces données sont ensuite traitées, par exemple, en utilisant les techniques de la programmation parallèle en vue de fournir les résultats souhaités dans un temps raisonnable. La présente couche doit avoir connaissance des sources de données, des types de données, des algorithmes qui vont travailler sur ces données et enfin des résultats souhaités. Les résultats des traitements peuvent être utilisés par une des applications ou bien être sauvegardés dans une des bases de données adaptées.\par
		\paragraph{Stream Processing layer} Cette couche approvisionne les données directement d'une des entrées du \textit{Data Ingest layer}; c'est ce qui différencie cette couche de la couche Batch processing layer. En revanche, \textit{Stream Processing} est similaire à la couche   \textit{Batch processing} en matière  des techniques de la programmation parallèle utilisées ainsi que la nécessité d'avoir les détails sur les sources des données, les types de données et les résultats souhaités.\par
		\paragraph{Data organizing layer} Le rôle de cette couche est d'organiser les données en provenance de la  couche  \textit{Stream Processing} et de la couche \textit{Batch processing}. Cette couche est représentée par les bases de données NoSQL. 
		 %afin de faciliter l'accès à ces dernières. Ce sont les  données obtenues  de la part .
		\paragraph{Infrastructure layer} Cette composante est responsable de la gestion des ressources de stockage, des ressources du calcul et de la gestion de la communication. Les fonctionnalités de cette couche sont typiquement fournies à travers le cloud computing.
		\paragraph{Distributed File System layer} Cette couche permet de stocker une grande quantité de données, de sorte que ces données soient rapidement et facilement accessibles à toutes les couches qui forment un système  Big Data. C'est ce qu'assure, par exemple, Hadoop Distributed File System (HDFS).
		\paragraph{Data consumption layer} Cette dernière couche utilise les résultats obtenus par les couches de l'analyse. Les résultats fournis peuvent être exprimés avec des rapports, des tableaux d'indicateurs, des visualisations, un moteur de recommandation ou tout autre format.
	\subsection{Les bases de données NoSQL (Not Only SQL) } \label{sec:nosql}
	\paragraph{Introduction } ~
	
    Au cours de ces dernières années, on constate une révolution dans le stockage de données non structurées ayant une taille importante.  De plus,  les objets à sauvegarder sont complexes; ils sont issus de sources hétérogènes.  Cette complexité a mis en question les performances des bases de données relationnelles. 
	
	Le terme NoSQL est apparu pour la première fois en $ 1998 $. Carlo Strozzi \cite{CarloStrozziNosql} a parlé des bases de données relationnelles qui n'utilisent pas le SQL comme langage d'interrogation des tables. Des années plus tard, des solutions  open source basées sur ce concept ont vu le jour. 
	
	Les bases de données relationnelles sont conçues pour gérer les données structurées et sont optimisées pour offrir la précision et la cohérence de données. De plus, elles sont utilisées par de nombreuses entreprises pour plusieurs raisons comme   la facilité d'utilisation, la disponibilité de plusieurs produits et développeurs, etc. Ces dernières années, avec l'augmentation exponentielle de la quantité de données générées par certaines entreprises, ces dernières ont constaté l'insuffisance des Systèmes de Gestion de Bases de Données Relationnelles (SGBDR) pour répondre à leurs besoins.
	
    %	\paragraph{ Les besoins auxquels répondent NoSQL}  ~
	Les bases de données NoSQL sont conçues pour gérer des  volumes de données importants. Le flux ainsi que la structure de  ces données sont imprévisibles. C'est pourquoi les bases de données relationnelles ne sont pas convenables. L'idée  des bases de données NoSQL, c'est d'abord assurer la capacité de stocker des données à grande échelle dont la  quantité  évolue rapidement, voire exponentiellement.  En deuxième lieu, les données stockées  doivent être interrogées  avec efficacité. Les données stockées dans  les bases de données NoSQL n'obéissent pas à un modèle prédéfini comme c'est le cas pour les bases de données relationnelles. Cette flexibilité est une des caractéristiques des bases de données NoSQL.
	\paragraph{Types de base de données NoSQL} \label{sec:nosql-database}  ~
	
	Il existe quatre catégories distinctes de bases de données NoSQL. Chaque catégorie répond  à des besoins particuliers. On distingue les bases de données clé-valeur, document, graphe et colonne.
	\subparagraph {Clé-valeur} Une base de données de type clé-valeur repose sur le paradigme clé-valeur; chaque donnée, que ce soit un nombre, du texte ou tout autre type est associé à une clé unique. Cette clé est le seul moyen d'accéder aux données stockées.
	Dans les bases de données NoSQL de type clé-valeur, les enregistrements  n'adhèrent pas à une structure prédéfinie. Par exemple, on peut avoir le premier enregistrement de type entier et le deuxième enregistrement de type texte. Cela assure une forte évolutivité grâce à l'absence d'une structure ou de typage. La Figure \ref{fig:key-value-nosql} montre un exemple de données stockées sous forme de paires clé-valeur.
	\begin{figure}[h]
		\captionsetup{justification=centering}
		\centering
		\resizebox{.4\textwidth}{!}{
			\input{illustrations/key-value-nosql-4.tex}
	    }
		\caption{Illustration d'une base de données NoSQL de type clé-valeur}
		\label{fig:key-value-nosql}
	\end{figure}

	\subparagraph{Graphe} Dans une base de données de type graphe, les données stockées sont les n\oe{}uds, les liens et les propriétés sur les n\oe{}uds et sur les liens. Un exemple  de base de données NoSQL de type graphe est le réseau social; chaque entité représente une personne et les relations entre ces personnes peuvent prendre plusieurs formes. Un autre exemple de données stockées  dans une base de données orientée graphe est donné dans la Figure  	\ref{fig:graphe-nosql}. Avec cette représentation, par exemple, on peut chercher les membres du groupe \textit{Chess}.  
\begin{figure}[h]
	\centering
	%	\resizebox{\textwidth}{!}{
	\includegraphics[width=.9\linewidth]{illustrations/GraphDatabase_PropertyGraph.png}		
	%   }
	\caption{Illustration d'une base de données NoSQL de type graphe. Source : \url{https://en.wikipedia.org/wiki/Graph_database}, consultée le $10/05/2019$}
	\label{fig:graphe-nosql}
\end{figure}
		\subparagraph{Document} Une base de données NoSQL de type document permet de stocker les données en reposant sur le paradigme clé-valeur. Toutefois, les valeur stockées sont complexes, il s'agit de documents de type JSON, XML, etc. L'accès aux données d'un enregistrement peut se faire de manière hiérarchique. La possibilité de stocker des objets complexes et hétérogènes  est un des points forts des bases de données NoSQL de type  document. Un exemple est fourni dans la Figure \ref{fig:document-nosql}. Une des différences majeures entre les bases de données clé-valeur et celles de type document c'est que pour les premières, l'indexation est au niveau  des clés seulement, tandis que pour les deuxièmes, l'indexation est au niveau clé et valeur. En pratique, pour les bases de données clé-valeur, les données sont récupérées en se basant sur la clé et pour les autres, la récupération d'un enregistrement peut être basée sur la clé et sur la valeur étant donné que la valeur est semi-structuré (valeur de type JSON, XML, etc.)
		\begin{figure}[h]
			\centering
				\resizebox{\textwidth}{!}{
			% Graphic for TeX using PGF
% Title: /home/hayat/RipeAtlasTraceroutesAnalysis/2019/Rapport_Mai/illustrations/document-nosql.dia
% Creator: Dia v0.97+git
% CreationDate: Tue May  7 11:56:16 2019
% For: hayat
% \usepackage{tikz}
% The following commands are not supported in PSTricks at present
% We define them conditionally, so when they are implemented,
% this pgf file will use them.
\ifx\du\undefined
  \newlength{\du}
\fi
\setlength{\du}{15\unitlength}
\begin{tikzpicture}[even odd rule]
\pgftransformxscale{1.000000}
\pgftransformyscale{-1.000000}
\definecolor{dialinecolor}{rgb}{0.000000, 0.000000, 0.000000}
\pgfsetstrokecolor{dialinecolor}
\pgfsetstrokeopacity{1.000000}
\definecolor{diafillcolor}{rgb}{1.000000, 1.000000, 1.000000}
\pgfsetfillcolor{diafillcolor}
\pgfsetfillopacity{1.000000}
\pgfsetlinewidth{0.000000\du}
\pgfsetdash{}{0pt}
\pgfsetmiterjoin
\definecolor{diafillcolor}{rgb}{0.960784, 0.960784, 0.960784}
\pgfsetfillcolor{diafillcolor}
\pgfsetfillopacity{1.000000}
\pgfpathellipse{\pgfpoint{7.395991\du}{3.752005\du}}{\pgfpoint{1.899011\du}{0\du}}{\pgfpoint{0\du}{1.012005\du}}
\pgfusepath{fill}
\definecolor{dialinecolor}{rgb}{0.000000, 0.000000, 0.000000}
\pgfsetstrokecolor{dialinecolor}
\pgfsetstrokeopacity{1.000000}
\pgfpathellipse{\pgfpoint{7.395991\du}{3.752005\du}}{\pgfpoint{1.899011\du}{0\du}}{\pgfpoint{0\du}{1.012005\du}}
\pgfusepath{stroke}
% setfont left to latex
\definecolor{dialinecolor}{rgb}{0.000000, 0.000000, 0.000000}
\pgfsetstrokecolor{dialinecolor}
\pgfsetstrokeopacity{1.000000}
\definecolor{diafillcolor}{rgb}{0.000000, 0.000000, 0.000000}
\pgfsetfillcolor{diafillcolor}
\pgfsetfillopacity{1.000000}
\node[anchor=base,inner sep=0pt, outer sep=0pt,color=dialinecolor] at (7.395991\du,3.947005\du){ta};
\pgfsetlinewidth{0.000000\du}
\pgfsetdash{}{0pt}
\pgfsetmiterjoin
\definecolor{diafillcolor}{rgb}{0.960784, 0.960784, 0.960784}
\pgfsetfillcolor{diafillcolor}
\pgfsetfillopacity{1.000000}
\pgfpathellipse{\pgfpoint{7.469011\du}{6.436015\du}}{\pgfpoint{1.899011\du}{0\du}}{\pgfpoint{0\du}{1.012005\du}}
\pgfusepath{fill}
\definecolor{dialinecolor}{rgb}{0.000000, 0.000000, 0.000000}
\pgfsetstrokecolor{dialinecolor}
\pgfsetstrokeopacity{1.000000}
\pgfpathellipse{\pgfpoint{7.469011\du}{6.436015\du}}{\pgfpoint{1.899011\du}{0\du}}{\pgfpoint{0\du}{1.012005\du}}
\pgfusepath{stroke}
% setfont left to latex
\definecolor{dialinecolor}{rgb}{0.000000, 0.000000, 0.000000}
\pgfsetstrokecolor{dialinecolor}
\pgfsetstrokeopacity{1.000000}
\definecolor{diafillcolor}{rgb}{0.000000, 0.000000, 0.000000}
\pgfsetfillcolor{diafillcolor}
\pgfsetfillopacity{1.000000}
\node[anchor=base,inner sep=0pt, outer sep=0pt,color=dialinecolor] at (7.469011\du,6.631015\du){tb};
\pgfsetlinewidth{0.000000\du}
\pgfsetdash{}{0pt}
\pgfsetmiterjoin
\definecolor{diafillcolor}{rgb}{0.960784, 0.960784, 0.960784}
\pgfsetfillcolor{diafillcolor}
\pgfsetfillopacity{1.000000}
\pgfpathellipse{\pgfpoint{7.394011\du}{9.196015\du}}{\pgfpoint{1.899011\du}{0\du}}{\pgfpoint{0\du}{1.012005\du}}
\pgfusepath{fill}
\definecolor{dialinecolor}{rgb}{0.000000, 0.000000, 0.000000}
\pgfsetstrokecolor{dialinecolor}
\pgfsetstrokeopacity{1.000000}
\pgfpathellipse{\pgfpoint{7.394011\du}{9.196015\du}}{\pgfpoint{1.899011\du}{0\du}}{\pgfpoint{0\du}{1.012005\du}}
\pgfusepath{stroke}
% setfont left to latex
\definecolor{dialinecolor}{rgb}{0.000000, 0.000000, 0.000000}
\pgfsetstrokecolor{dialinecolor}
\pgfsetstrokeopacity{1.000000}
\definecolor{diafillcolor}{rgb}{0.000000, 0.000000, 0.000000}
\pgfsetfillcolor{diafillcolor}
\pgfsetfillopacity{1.000000}
\node[anchor=base,inner sep=0pt, outer sep=0pt,color=dialinecolor] at (7.394011\du,9.391015\du){tc};
\pgfsetlinewidth{0.000000\du}
\pgfsetdash{}{0pt}
\pgfsetmiterjoin
{\pgfsetcornersarced{\pgfpoint{0.000000\du}{0.000000\du}}\definecolor{diafillcolor}{rgb}{0.960784, 0.960784, 0.862745}
\pgfsetfillcolor{diafillcolor}
\pgfsetfillopacity{1.000000}
\fill (12.651200\du,2.864010\du)--(12.651200\du,4.664010\du)--(49.000000\du,4.664010\du)--(49.000000\du,2.864010\du)--cycle;
}{\pgfsetcornersarced{\pgfpoint{0.000000\du}{0.000000\du}}\definecolor{dialinecolor}{rgb}{0.000000, 0.000000, 0.000000}
\pgfsetstrokecolor{dialinecolor}
\pgfsetstrokeopacity{1.000000}
\draw (12.651200\du,2.864010\du)--(12.651200\du,4.664010\du)--(49.000000\du,4.664010\du)--(49.000000\du,2.864010\du)--cycle;
}% setfont left to latex
\definecolor{dialinecolor}{rgb}{0.000000, 0.000000, 0.000000}
\pgfsetstrokecolor{dialinecolor}
\pgfsetstrokeopacity{1.000000}
\definecolor{diafillcolor}{rgb}{0.000000, 0.000000, 0.000000}
\pgfsetfillcolor{diafillcolor}
\pgfsetfillopacity{1.000000}
\node[anchor=base,inner sep=0pt, outer sep=0pt,color=dialinecolor] at (30.825600\du,3.959010\du){\{"timestamp":1427847501,"type":"traceroute"\}};
\pgfsetlinewidth{0.000000\du}
\pgfsetdash{}{0pt}
\pgfsetmiterjoin
{\pgfsetcornersarced{\pgfpoint{0.000000\du}{0.000000\du}}\definecolor{diafillcolor}{rgb}{0.960784, 0.960784, 0.862745}
\pgfsetfillcolor{diafillcolor}
\pgfsetfillopacity{1.000000}
\fill (12.570000\du,5.574010\du)--(12.570000\du,7.374010\du)--(49.000000\du,7.374010\du)--(49.000000\du,5.574010\du)--cycle;
}{\pgfsetcornersarced{\pgfpoint{0.000000\du}{0.000000\du}}\definecolor{dialinecolor}{rgb}{0.000000, 0.000000, 0.000000}
\pgfsetstrokecolor{dialinecolor}
\pgfsetstrokeopacity{1.000000}
\draw (12.570000\du,5.574010\du)--(12.570000\du,7.374010\du)--(49.000000\du,7.374010\du)--(49.000000\du,5.574010\du)--cycle;
}% setfont left to latex
\definecolor{dialinecolor}{rgb}{0.000000, 0.000000, 0.000000}
\pgfsetstrokecolor{dialinecolor}
\pgfsetstrokeopacity{1.000000}
\definecolor{diafillcolor}{rgb}{0.000000, 0.000000, 0.000000}
\pgfsetfillcolor{diafillcolor}
\pgfsetfillopacity{1.000000}
\node[anchor=base,inner sep=0pt, outer sep=0pt,color=dialinecolor] at (30.785000\du,6.669010\du){\{"af":6,"dst\_addr":"2001:7fd::1","timestamp":1427847501,"type":"traceroute"\}};
\pgfsetlinewidth{0.000000\du}
\pgfsetdash{}{0pt}
\pgfsetmiterjoin
{\pgfsetcornersarced{\pgfpoint{0.000000\du}{0.000000\du}}\definecolor{diafillcolor}{rgb}{0.960784, 0.960784, 0.862745}
\pgfsetfillcolor{diafillcolor}
\pgfsetfillopacity{1.000000}
\fill (12.620000\du,8.324010\du)--(12.620000\du,10.124010\du)--(48.995000\du,10.124010\du)--(48.995000\du,8.324010\du)--cycle;
}{\pgfsetcornersarced{\pgfpoint{0.000000\du}{0.000000\du}}\definecolor{dialinecolor}{rgb}{0.000000, 0.000000, 0.000000}
\pgfsetstrokecolor{dialinecolor}
\pgfsetstrokeopacity{1.000000}
\draw (12.620000\du,8.324010\du)--(12.620000\du,10.124010\du)--(48.995000\du,10.124010\du)--(48.995000\du,8.324010\du)--cycle;
}% setfont left to latex
\definecolor{dialinecolor}{rgb}{0.000000, 0.000000, 0.000000}
\pgfsetstrokecolor{dialinecolor}
\pgfsetstrokeopacity{1.000000}
\definecolor{diafillcolor}{rgb}{0.000000, 0.000000, 0.000000}
\pgfsetfillcolor{diafillcolor}
\pgfsetfillopacity{1.000000}
\node[anchor=base,inner sep=0pt, outer sep=0pt,color=dialinecolor] at (30.807500\du,9.419010\du){\{"af":6, "result":\ensuremath{[}\{"hop":1,"result":\ensuremath{[}\{"from":"2a01:240:fec5::ff","rtt":0.477,"size":88,"ttl":64\}\ensuremath{]}\}\ensuremath{]}\}};
\pgfsetlinewidth{0.100000\du}
\pgfsetdash{}{0pt}
\pgfsetbuttcap
{
\definecolor{diafillcolor}{rgb}{0.000000, 0.000000, 0.000000}
\pgfsetfillcolor{diafillcolor}
\pgfsetfillopacity{1.000000}
% was here!!!
\pgfsetarrowsend{to}
\definecolor{dialinecolor}{rgb}{0.000000, 0.000000, 0.000000}
\pgfsetstrokecolor{dialinecolor}
\pgfsetstrokeopacity{1.000000}
\draw (9.295000\du,3.752010\du)--(12.651200\du,3.764010\du);
}
\pgfsetlinewidth{0.100000\du}
\pgfsetdash{}{0pt}
\pgfsetbuttcap
{
\definecolor{diafillcolor}{rgb}{0.000000, 0.000000, 0.000000}
\pgfsetfillcolor{diafillcolor}
\pgfsetfillopacity{1.000000}
% was here!!!
\pgfsetarrowsend{to}
\definecolor{dialinecolor}{rgb}{0.000000, 0.000000, 0.000000}
\pgfsetstrokecolor{dialinecolor}
\pgfsetstrokeopacity{1.000000}
\draw (9.368020\du,6.436010\du)--(12.570000\du,6.474010\du);
}
\pgfsetlinewidth{0.100000\du}
\pgfsetdash{}{0pt}
\pgfsetbuttcap
{
\definecolor{diafillcolor}{rgb}{0.000000, 0.000000, 0.000000}
\pgfsetfillcolor{diafillcolor}
\pgfsetfillopacity{1.000000}
% was here!!!
\pgfsetarrowsend{to}
\definecolor{dialinecolor}{rgb}{0.000000, 0.000000, 0.000000}
\pgfsetstrokecolor{dialinecolor}
\pgfsetstrokeopacity{1.000000}
\draw (9.293020\du,9.196010\du)--(12.620000\du,9.224010\du);
}
% setfont left to latex
\definecolor{dialinecolor}{rgb}{0.000000, 0.000000, 0.000000}
\pgfsetstrokecolor{dialinecolor}
\pgfsetstrokeopacity{1.000000}
\definecolor{diafillcolor}{rgb}{0.000000, 0.000000, 0.000000}
\pgfsetfillcolor{diafillcolor}
\pgfsetfillopacity{1.000000}
\node[anchor=base west,inner sep=0pt,outer sep=0pt,color=dialinecolor] at (9.850000\du,14.350000\du){};
\end{tikzpicture}

		}
			\caption{Illustration d'une base de données NoSQL de type document}
			\label{fig:document-nosql}
		\end{figure}
		\subparagraph{Colonnes} Dans les bases de données traditionnelles, les données sont stockées sur des lignes. Dans le cas d'une base NoSQL orientée colonne, les données sont stockées par colonne. L'interrogation de ce type de bases travaille sur une colonne particulière sans devoir passer par les autres colonnes comme dans les bases de données relationnelles classiques. Une base de données de type colonne est adaptée pour les requêtes analytiques comme les requêtes d'agrégation (moyennes, maximum, etc). La Figure \ref{fig:comomn-nosql} illustre la différence entre le stockage dans une base de données relationnelle et une base de données orientée colonnes\footnote{Cette illustration  est basée sur une figure disponible sur \url{https://www.illustradata.com/bases-nosql-orientees-colonnes-quest-cest}, consulté le $02/05/2019$.}. 
		Les bases de données NoSQL orientées colonnes sont conçues pour pouvoir ajouter facilement de nouveaux colonnes, jusqu'à des millions de colonnes. De plus, le coût du stockage de \textit{null} vaut $ 0 $.
		
	\begin{figure}[h]
		\centering
		\includegraphics[width=\linewidth]{illustrations/colomn-db-2.png}
		\caption{Illustration d'une base de données NoSQL de type colonne}
		\label{fig:comomn-nosql}
	\end{figure}


Les bases de données NoSQL sont conçues pour répondre à des besoins spécifiques. Elles n'ont pas été créées pour remplacer les bases de données relationnelles. Il existe plusieurs implémentations des quatre types des bases de données NoSQL. Chaque implémentation favorise un ou plusieurs  éléments suivants : la disponibilité des données, la cohérence des données et la tolérance au partitionnement.  C'est ce qu'explique le théorème CAP.	
		\paragraph{Big Data et le théorème  CAP} \label{par:cap-theorem}:
		
		Dans le but d'assurer un traitement rapide de données à grande échelle, ces dernières sont réparties sur un cluster de machines  (appelées aussi  n\oe{}uds). Le théorème CAP annonce que dans le cadre d'un système distribué où le stockage de données est  réparti sur plusieurs machines,
		%(ou n\oe{}uds) (voir \ref{sec:distruted-camput}),  
		une base de données ne peut pas garantir les trois attributs suivants : \textit{Consistency}, \textit{Availability} et \textit{Partition Tolerence}  en même temps. \par
			\textbf{Consistency (ou intégrité)} Chaque donnée a un seul état visible depuis l'extérieur. Par exemple, les différents serveurs hébergeant la base de données voient tous les mêmes données. C'est pourquoi une lecture faite après une écriture doit renvoyer la donnée précédemment écrite. \par
			\textbf{Availability (ou disponibilité)} Une base de données doit toujours fournir une réponse à une requête d'un client.\par
			%En cas d'une panne technique sur un des serveurs qui hébergent  la base de données, il faut s'assurer de desservir  des clients.  Généralement, cela est assuré à travers la réplication des données
			 \textbf{Partition tolerance (ou  tolérance au partitionnement) } Une coupure du réseau entre deux n\oe{}uds ou l'indisponibilité d'un de ces n\oe{}uds ne devrait pas affecter le bon fonctionnement du système. Tout de même,  ce dernier doit répondre à la demande d'un client. 

		
		%conclusion
		Les trois attributs du théorème CAP s'opposent entre eux. On distingue les trois scénarios possibles:
		
		\begin{itemize}
			\item [--] Le couple \textbf{CA} : les SGBDR adoptent les deux attributs C et A, qui sont une forte cohérence et disponibilité. Cependant, l'attribut partitionnement réseau n'est pas toujours pris en compte.
			\item [--] Le couple \textbf{CP} : les implémentations du C et du P assurent la tolérance aux pannes en distribuant les données sur plusieurs serveurs. Malgré cette réplication, ces implémentations assurent la cohérence des données même en présence de mises à jour concurrentielles.
			\item [--] Le couple \textbf{AP} : les implémentations du A et du  P assurent un temps de réponse rapide et une réplication de données. Cependant, les mises à jour étant asynchrones, la garantie que la version d'une donnée soit bonne, ne peut pas être assurée.
			
		\end{itemize}
		
		La Figure \ref{fig:cap} présente des implémentations des différents types de bases de données NoSQL pour chaque couple CA, CP et AP.
		
		\begin{figure}[h]
			\centering
			\captionsetup{justification=centering}
			\includegraphics[width=1\linewidth]{illustrations/cap}
			\caption{Bases de données NoSQL suivant le théorème de CAP }
			\label{fig:cap}
			\source{\url{https://www.researchgate.net/figure/CAP-theorem-concept-5-II-WHY-YOU-NEED-NOSQL-The-first-reason-to-use-NoSQL-is-because_fig2_323309389}, consultée le $05/08/2018$.}
		\end{figure}
		
		
		
		Le choix d'une base de données relationnelle ou NoSQL dépend des besoins des entreprises. En terme de tendances, la Figure \ref{fig:ranking-db} reprend un classement des SGBDs au $1$ août $ 2018 $. La suite de la liste ainsi que  la méthode qui dirige ce classement sont    disponibles sur le site  Web \textit{DB-Engines Ranking}\footnote{URL : \url{https://db-engines.com/},  consulté le $01/08/2018$.}. Parmi les critères du classement, on trouve le nombre de références du SGBD sur les sites Internet. 
		
		%Ce nombre de référence est quantifiable à partir du  nombre lui-même de résultats obtenus des différents moteurs de recherche comme Google, Bing, etc. 
		
		\begin{figure}[h]
			\centering
			\captionsetup{justification=centering}
			\includegraphics[width=1\linewidth]{illustrations/ranking-db}
			\caption{Un classement des SGBDs sur \textit{DB-Engines Ranking} du $1$ août $2018$ }
			\label{fig:ranking-db}
			\source{\url{https://db-engines.com/en/ranking}, consultée le $01/08/2018$.}
		\end{figure}
		
		\subsection{Extraction, Transformation, Loading (ETL)} \label{sec:etl}
		Dans une même organisation, il est possible d'avoir plusieurs sources de données~: des bases de données relationnelles, des bases de données NoSQL, des fichiers de données de type Excel, etc. 
		Dans le cas où une analyse de données devrait impliquer des données en provenance de  sources  de données hétérogènes, il est nécessaire de faire appel aux opérations ETL :
		
		\textbf{Extraction} :  la diversité des sources de données implique des formats de données différents. Généralement, ce sont des données en provenance des bases de données relationnelles, des fichiers plats, des bases de données non relationnelles,  etc. Le but de la phase d'extraction est de convertir les données en un seul format approprié à l'étape de la transformation. Cette phase  vérifie si les données respectent  une structure attendue. Si ce n'est pas le cas, les données peuvent être totalement ou partiellement rejetées.
		
		\textbf{Transformation} : cette étape  s'applique  sur les données extraites. Elle comprend  une série de règles ou de fonctions. Ces derniers sont appliqués sur les données avant de les envoyer vers la cible. Certaines sources de données nécessitent  peu de transformations, voire aucune. Dans d'autres cas, un ou plusieurs  types de transformation sont  nécessaires. Quelques exemples de transformations  sont    listées ci-dessous :
		\begin{itemize}
			\item sélection d'un nombre  de colonnes à charger parmi plusieurs colonnes;
			\item adaptation des codes. Par exemple,  quand le système de stockage source utilise 1 pour dénoter un homme et l'entrepôt de données cible  utilise "H"; 
			\item conversion des devises, c'est le cas par exemple où les salaires sont exprimés en une devise différente de l'euro;
			\item élimination des doublons.
		\end{itemize}
		
		\textbf{Loading} : l'objectif de cette étape est de charger les données transformées vers l'entrepôt de données. Le chargement des données dépend des besoins de l'organisation. Dans certains cas, on prévoit le remplacement des informations existantes par des informations cumulatives plus récentes. Dans d'autres cas, les nouvelles informations sont ajoutées dans la suite de celles existantes. 
		
		En ce qui concerne la fréquence  des opérations ETL, elles peuvent être planifiées de manière horaire, quotidienne, hebdomadaire, mensuelle ou autre.
		
		 %ETL est un système de chargement de données depuis les différentes sources de données jusqu'à l'entrepôt de données. Ce système  s'occupe de faire passer les données par un ensemble de traitements pour les nettoyer,  les contextualiser et enfin les charger. Les tâches ETL prennent énormément du temps dans un projet d'analyse de données. Il est important d'assurer  la qualité de données et d'éviter les fausses données ou les données inutiles. 
		 
		
		
		
		\subsection{Schema on Write VS Schema on Read} \label{sec:schema-read-write}
		
		Lors du chargement des données depuis leurs sources de stockage, on distingue deux approches : \textit{ Schema on Write} et \textit{Schema on Read}.
		% L'approche \textit{Schema on Read } est celle utilisée par l'outil Amazon Athena présenté dans la section \ref{par:allservices}.
		
		Dans la première, il faut définir les colonnes, le format de données, les types, etc. La lecture des données est rapide et moins coûteuse étant donné l'effort entrepris pour définir la structure. C'est le cas des bases de données relationnelles.
		
		Dans la deuxième, les données sont chargées telles qu'elles sont, sans transformations ou changements. L'interprétation de ces données se fait lors de la lecture, et cela dépend des besoins pour lesquels les données sont analysées. Ainsi, les mêmes données peuvent être lues de différentes manières. Par exemple, l'action  de lire les données  d'une colonne, qu'elles soient de type entier ou bien chaîne de caractère d'un fichier CSV est la même, c'est le type de la donnée qui diffère. C'est l'approche utilisée par Amazon Athena (voir la section \ref{aws:athena}). 
		
\paragraph{Exemple illustratif}~

 Afin de montrer la différence entre les deux approches, nous comparons Apache Spark (présenté en détail dans la section \ref{apache-spark}) avec une base de données relationnelle (SQL Server).   Les étapes suivantes concernent  SQL Server:
%\begin{tcolorbox}{SQL RDBMS : exemple de SQL Server}
\begin{enumerate}
	\item Créer la table Traceroutes :
\begin{lstlisting}[language = sql, basicstyle=\small]
CREATE TABLE Traceroutes(
	id INT,
	dst_name VARCHAR(30),
	...
)
\end{lstlisting}
		
	\item  Charger les données depuis le fichier \textit{traceroute-2019-04-08T0000.json} vers la table Traceroutes, chaque ligne du fichier doit correspondre à la  structure  créée à l'étape $ 1 $:
\begin{lstlisting}[language = sql, basicstyle=\small]
BULK INSERT  Traceroutes
FROM 'c:\traceroutes\traceroute-2019-04-08T0000.csv'
WITH ROWTERMINATOR = '\n'
\end{lstlisting}
	
	\item Interroger les données : à l'issue de l'étape $2$,  les données sont chargées et prêtes à l'interrogation: 
\begin{lstlisting}[language = sql, basicstyle=\small]
SELECT dst_name FROM  Traceroutes
\end{lstlisting}
	\end{enumerate}

Pour les bases de données relationnelles, qui adoptent l'approche \textit{Schema on Write}, on ne peut pas ajouter des données avant de créer le schéma dirigeant ces dernières. De plus, la création du schéma nécessite la compréhension  exhaustive des données. 
Car en cas d'un changement du contenu du fichier de données, en terme de structure, la table créée doit être supprimée, mise à jour, ensuite il faut recharger à nouveau les données. L'implication de la mise à jour du schéma peut être coûteuse en terme de temps  dans le cas  d'un  volume de donnée  important, de l'ordre de plusieurs centaines de téraoctets. Dans certains cas, une mise à jour des relations avec d'autres tables est requise.

En ce qui concerne l'approche \textit{Schema on Read}, nous prenons l'exemple d'Apache Spark présenté en détail dans la section \ref{apache-spark}. En utilisant cette technologie, on crée un schéma selon les besoins de l'analyse de ces données.
% et non pas pour que la structure des données soit exactement celle à créer, sauf si cela fait partie des besoins. 
%Par exemple, on peut créer une table dont les colonnes correspondent  seulement à la moitié des colonnes possibles. Un autre exemple concerne l'utilisation  des conditions; on crée une table et durant le chargement des données, on ne charge que celles vérifiant une condition donnée. Pour AWS Athena et les autres technologies qui se basent sur  cette approche, les données sont chargées au moment de l'utilisation et avec plus de flexibilité.
Pour illustrer cette approche, l'annexe \ref{exemple-traceroute} présente une réponse d'une requête traceroute. Pour les besoins de l'outil de détection (voir le chapitre \ref{chap:algorith-detection}), plusieurs attributs ne sont pas pertinents. Ainsi, il est inutile de les charger lors de l'analyse.   En Spark, les données sont lues en associant les attributs d'une classe aux attributs de la réponse traceroute; c'est la classe Traceroute décrite dans le Listing \ref{lst:case-class-Traceroute}. En résultat, seules les données utiles qui sont chargées en mémoire pour qu'elles soient traitées.

La meilleure approche dépend des besoins de l'analyse. La première approche est meilleure en performances, en revanche, la deuxième est tolérante aux erreurs  humaines.

		\subsection{L'informatique distribuée et l'analyse de données massives} \label{sec:distruted-camput}
		Il existe deux stratégies pour appliquer des traitements sur un grand ensemble de données: 
		
		
		\begin{itemize}
			\item[--] Par distribution des traitements (\textit{scaling} des traitements)~: les traitements sont distribués sur un nombre de n\oe{}uds important. De ce fait, les données sont amenées jusqu'à ces n\oe{}uds.
			
			\item[--] Par distribution des données (\textit{scaling} des données)~: les données sont distribuées sur un nombre important de n\oe{}uds. Par ailleurs cela permet  de stocker un maximum de données. Il s'agit d'amener les traitements aux machines sur lesquelles les données sont stockées. Du fait que le stockage de données est réparti sur plusieurs machines, il est possible de traiter des données très volumineuses en un temps optimal. La première mise en \oe{}uvre de cette approche est le schéma MapReduce (voir la section  \ref{mapreducesection}). 
		\end{itemize}
\subsection{MapReduce} \label{mapreducesection}
%https://softwareengineering.stackexchange.com/questions/220605/why-big-data-needs-to-be-functional

MapReduce est un modèle de programmation proposé par Google. Il est conçu pour   le traitement distribué de grands ensembles de données   sur un cluster de machines. Un programme MapReduce est composé de deux phases principales   \textit{Map} et  \textit{Reduce}.

MapReduce divise le travail en petites parties, chacune pouvant être effectuée en parallèle sur le cluster de machines. Autrement dit, le  problème est divisé en un grand nombre de problèmes plus petits, chaque petit problème est traité pour donner des résultats individuelles. Ces résultats sont ensuite traités pour donner un résultat final. 
%Hadoop MapReduce est évolutif et peut également être utilisé sur de nombreux ordinateurs. 
%\begin{tcolorbox}
%	\textit{ MapReduce est un patron d'architecture de développement informatique, inventé par Google, dans lequel sont effectués des %calculs parallèles, et souvent distribués, de données potentiellement très volumineuses, typiquement supérieures en taille à 1 %téraoctet} \footnote{Source : \url{https://fr.wikipedia.org/wiki/MapReduce}, consultée le $20/12/2018$.}. 
%\end{tcolorbox}
La Figure 	\ref{fig:2-figure1-1-map-reduce-workflow} représente une vue d'ensemble du modèle de programmation MapReduce. 

\begin{figure}[h]
	\centering
	\includegraphics[width=\linewidth]{illustrations/2-Figure1-1-map-reduce-workflow}
	\caption{Vue d'ensemble du modèle MapReduce. Source : \cite{6427487}}
	\label{fig:2-figure1-1-map-reduce-workflow}
\end{figure}


Le modèle de données de base dans MapReduce  est la paire clé/valeur. Pendant la phase de \textit{Map}, la fonction  associée  au Map est exécutée sur chaque paire clé/valeur (K1,V1) des données d'entrée (\textit{Data Split}), ce que produit   une liste des paires clé/valeur (List(K2,V2)) intermédiaires pour chaque clé/valeur (K1,V1). Il existe une autre phase entre  \textit{Map} et de \textit{Reduce}, c'est la phase de \textit{Shffling}. Durant cette dernière,  les paires clé/valeur (List(K2,V2))  intermédiaire sont triées et regroupées par la clé. Il en résulte un ensemble de paires clé/valeur (K2, List(V2)) où chaque paire contient toutes les valeurs associées à une clé particulière (K2). Ces paires clé/valeur sont ensuite partitionnées et regroupées sur la clé, puis passée à la phase Reduce au cours de laquelle la fonction de réduction est appliquée individuellement à chaque clé et aux valeurs associées pour cette clé (K2,List(V2)). La phase \textit{Reduce} peut produire une valeur nulle ou une sortie (List(K3,V3)).

\section{Parcours de quelques technologies du Big Data}
La liste des technologies  du Big Data est en expansion continue pour répondre au mieux aux besoins de l'analyse de données massives. C'est pourquoi nous allons parcourir une liste non exhaustive des technologies liées au Big Data. En particulier, ce sont les technologies expérimentées pour analyser les traceroutes en provenance du projet Atlas.

MongoDB est la base de données NoSQL utilisée dans l'implémentation du travail de référence \cite{InternetHealthReport}. Amazon DynamoDB est la première technologie Big Data que nous avons utilisé pour réécrire l'outil de détection, ensuite, nous avons évalué la combinaison de trois services Web d'Amazon S3, Glue et Athena. Enfin, nous avons reproduit l'outil de détection en Spark/Scala. Pour Amazon Elastic MapReduce, nous l'avons utilisé pour exécuter l'application Spark/Scala dans un cluster de machines. Nous présentons aussi l'écosystème Hadoop, étant donné qu'il fait partie des principales plateformes du Big Data.
	
\subsection{MongoDB} \label{subsubsection:mongodb}
%	\paragraph{Introduction à la base de données MongoDB} \label{subsubsection:mongodb}~
MongoDB\footnote{URL : \url{https://www.mongodb.com/}, consulté le $02/08/2018$.} est une base de données  NoSQL de type Document\footnote{Une base de données NoSQL de type document est décrite dans la section \ref{sec:nosql-database}.}.  MongoDB est classé parmi les  SGBDs adoptant le couple CP (Consistency et Partition Tolerance) dans le théorème  CAP\footnote{Le théorème  CAP est décrit dans la section \ref{par:cap-theorem}}. Une base de données créées dans MongoDB est un ensemble de collections. Une collection dans MongoDB est équivalente à une table dans un SGBDR.
	
En $ 2016 $, MongoDB devient disponible en mode cloud sous le nom  MongoDB Atlas \footnote{URL : \url{https://www.mongodb.com/cloud/atlas}, consulté le $ 02/08/2018 $.}.  Il est distribué à travers les trois fournisseurs du cloud: Amazon Web Services (AWS), Google Cloud Platform et Microsoft Azure.  En terme de tarifs, plusieurs formules sont proposées \footnote{Source : \url{https://www.mongodb.com/cloud/atlas/pricing}, consultée le $ 02/08/2018 $.}, y inclut l'offre gratuite pour expérimenter MongoDB Atlas.  Les frais d'utilisation du service MongoDB Atlas dépendent du stockage, de la  RAM allouée et des options choisies. Les documents sont stockés dans MongoDB sous format BSON.
	
\begin{tcolorbox}
	BSON (ou Binary JSON) est un format utilisé pour stocker et transférer les données dans la base de données MongoDB. BSON facilite la représentation des structures de données simples et des tableaux associatifs\footnote{Source : \url{https://fr.wikipedia.org/wiki/BSON}, consultée le $ 02/08/2018 $.}.
\end{tcolorbox}
	
	\subsection{Amazon DynamoDB}\label{aws:dynmo}~
	
	% \paragraph{Amazon DynamoDB :}\label{aws:dynmo}~
	
	Amazon DynamoDB\footnote{URL : \url{https://aws.amazon.com/fr/dynamodb/}, consulté le $02/05/2018$.} est une base de données NoSQL de type clé-valeur distribuée, gérée par les services d'Amazon. Elle est capable de stocker un volume important de données limité par la capacité de l'infrastructure d'AWS. Amazon DynamoDB   est un service  simple et facile à utiliser,  il ne nécessite aucune configuration préalable. 
	
	Amazon DynamoDB  est une base de données évolutive permettant à l'utilisateur final de passer à l'échelle facilement et rapidement. Cette technologie offre des performances constantes à une échelle essentiellement infinie, limitée uniquement par la taille physique du cloud AWS. Amazon DynamoDB est flexible. Aucun schéma n'est requis pour stocker les données. Les frais d'utilisation de ce service dépendent de trois éléments\footnote{Source : \url{https://aws.amazon.com/fr/dynamodb/pricing/}, consultée $02/05/2018$.}:
	\begin{itemize}
		\item[--] la quantité de données stockées : DynamoDB est facturé par Go d'espace disque utilisé ($ 0,250 $ USD par Go par mois);
		\item[--] la capacité en lecture par seconde ($ 0,470 $ USD par unité de capacité d'écriture par mois);
		\item[--]  la capacité en écriture par seconde ($ 0,090 $ USD par unité de capacité de lecture par mois);
	\end{itemize}

\subsection{Amazon S3, Amazon Glue et Amazon Athena }

La combinaison d'Amazon S3, Amazon Glue et Amazon Athena permet de créer un environnement Big Data capable d'assurer respectivement le stockage de données, le chargement de données  et l'interrogation de données. 

%\paragraph{Introduction aux services Amazon S3, Amazon Athena et Amazon Glue }

\paragraph{Amazon S3}
\footnote{URL : \url{https://aws.amazon.com/fr/s3/}, consulté le $06/07/2018$.} est un service de stockage d'objets dans le cloud. Il est conçu pour stocker et  récupérer toute quantité de données. Il peut assurer $ 99,999999999 $ \% de durabilité. La sécurité  et l'accès aux données sont assurés. Il existe plusieurs classes de stockage qui répondent aux différents besoins. 

Dans Amazon S3, le fichier à stocker et considéré comme objet. Un objet est référencé par une clé qui reprend d'abord le chemin vers un pseudo répertoire suivi par  le nom de l'objet.  Le terme pseudo répertoire est utilisé car, en réalité, Amazon S3 ne stocke pas les objets dans des dossiers comme le cas d'un système d'exploitation. Chaque objet appartient à un compartiment, un compartiment appartient aussi à une des régions d'Amazon et le nom d'un compartiment est unique. Prenons l'exemple d'un  compartiment \textit{foo}, contenant deux objets ayant respectivement les clés \textit{A/b/c/i.txt} et \textit{A/b/d/k.txt}, dans ce cas, ces deux objets ne partagent que le même compartiment.
%Les fichiers des données sont organisés dans ce qu'on appelle un compartiment, c'est une simulation de dossier dans un système d'exploitation. A l'intérieur d'un compartiment, il est possible de créer des compartiments imbriqués. C'est une simulation d'arborescence de dossiers car physiquement cet arborescence n'existe pas. En ce qui concerne les frais du service AWS S3, . 
Le Tableau   	\ref{tab:pricing-s3-standard} décrit les tarifs de la formule standard relative à un mois.
\begin{table}[H]
	\centering
	\captionsetup{justification=centering}
	\begin{tabular}{l c }
		\textbf{Région} & UE (Irlande) \\ \hline
		\textbf{Première tranche de $ 50 $ To} &	$ 0,023 $ USD par Go\\ \hline
		\textbf{$ 450 $ To suivants} &	$ 0,022 $ USD par Go \\ \hline
		\textbf{Plus de $ 500 $ To} &	$ 0,021 $ USD par Go\\ \hline
	\end{tabular}
	\caption{Les tarifs du AWS S3 (formule Stockage standard S3)}
	\label{tab:pricing-s3-standard}
	\source{\url{https://aws.amazon.com/fr/s3/pricing/}, consultée le $05/08/2018$.}
\end{table}


\paragraph{Amazon  Glue} \label{aws:glue}
\footnote{URL : \url{https://aws.amazon.com/fr/glue/}, consulté le $06/07/2018$.} est un service d'extraction, de transformation et de chargement. L'objectif de ce service est de découvrir les données, les transformer et les rendre accessibles à la recherche et à l'interrogation.  Amazon Glue  est utile pour la construction des entrepôts de données; il découvre les métadonnées relatives aux magasins de données et les rend accessibles dans un catalogue central. En prenant en entrée les données  présentes dans un compartiment dans Amazon S3, Amazon Glue découvre le schéma de ces données. Il dispose de plusieurs classificateurs intégrés pour la découverte des données. Par exemple un classificateur pour trouver le schéma  de données en format JSON, XML, etc. Si les classificateurs intégrés ne répondent pas aux besoins particuliers, il est possible de créer des classificateurs personnalisés. 

Les frais de ce service dépendent du temps écoulé lors de l'analyse des données par les robots d'analyse durant la découverte du schéma. A ces frais s'ajoutent les frais du catalogue de données qui va être peuplé par les résultats fournis par les robots de l'analyse. Par exemple, on paye $ 0,44 $ USD par heure par DPU\footnote{DPU : unité de traitement des données.}, il est facturé à la seconde avec un minimum de $ 10 $ minutes par robot d'analyse exécuté. Plus de détails sont disponibles sur le site Web d'Amazon Glue\footnote{Source : \url{https://aws.amazon.com/fr/glue/pricing/}, consultée le $05/08/2018$.}.

\paragraph{Amazon Athena}\label{aws:athena}\footnote{URL : \url{https://aws.amazon.com/fr/athena/}, consulté le $06/07/2018$.} est un service de requêtes  interactif. Il permet d'interroger les données présentes dans Amazon S3 avec des requêtes SQL plus avancées. Le service Amazon Athena est considéré comme \textit{serverless}. Amazon Athena utilise l'approche \textbf{\textit{schema-on-read}} (voir la section \ref{sec:schema-read-write}) afin de projeter le schéma donné en entrée sur les données au moment de l'exécution de la requête SQL demandée. Le schéma sur lequel les données peuvent être projetées peut être créé manuellement ou bien utiliser le catalogue créé dans Amazon Glue.
Le service Amazon Athena est facturé suivant la quantité de données analysée. Précisément, $ 5 $ USD par To de données analysées.


\begin{tcolorbox}
	Une \textbf{\textit{requête est	interactive}} si on peut  obtenir immédiatement une réponse à la requête.  Dans le cas échéant, les résultats sont obtenus  dans le cadre d'un code source pour un des langages de programmation, typiquement à travers une API.
\end{tcolorbox}

\begin{tcolorbox}
	\textbf{\textit{Serverless}} peut être décomposé en \textit{server} et \textit{less}. Un outil est \textit{serverless} quand l'utilisateur final de cet outil peut l'utiliser sans se soucier de toute configuration ou gestion des serveurs derrière ce service. D'après Amazon\footnote{URL : \url{https://aws.amazon.com/serverless/}, consultée le  $05/08/2018$.},  \textit{Serverless} est l'architecture native du cloud.
\end{tcolorbox}

L'exécution des requêtes SQL est effectuée par le moteur de requêtes SQL Presto. Pour les instruction DDL (Data Definition Language), elles sont effectuées par  \textit{Hive Data Definition Language} \footnote{URL : \url{https://cwiki.apache.org/confluence/display/Hive/LanguageManual+DDL}, consulté le $05/08/2018$.}. Les requêtes DDL incluent la création, la suppression et la mise à jour de la structure de la table dans le cas d'une base de données relationnelles, d'une collection, d'une vue, etc. 

\begin{tcolorbox}
	\textbf{Hive Data definition language} (DLL) est un sous-ensemble de déclarations qui décrivent la structure de données dans Apache Hive.  Principalement, ce sont les instruction de création, suppression et de mise à jour de la structure des objets comme les bases de données, les tables, les vues et autres.
\end{tcolorbox}

\begin{tcolorbox}
	\textbf{\textit{Presto\footnote{URL : \url{http://prestodb.io/}, consulté le $01/08/2018$.} }} est un moteur de requêtes SQL open source destiné au Big Data. Il permet d'exécuter des requêtes analytiques interactives sur des données de taille importante; jusqu'à des pétaoctets de données.
	Presto interroge les données où elles sont hébergées. Ceci inclut les bases de données relationnelles, Amazon S3 et autres dépôts propriétaires. De plus, une même requête SQL peut combiner plusieurs sources de données. C'est intéressant pour les organisations ayant plusieurs sources de données.% Il fournit les résultats en quelques secondes, voire quelques minutes.  Il supporte les types de données complexes comme les objets JSON, les tableaux d'éléments, etc. 
\end{tcolorbox} 


\subsection{Apache Hadoop}

Hadoop est un framework open source conçu pour garantir le stockage et le traitement des données massives. Ceci est effectué  en utilisant des machines  qui collaborent  au sein d'un cluster. Hadoop regroupe plusieurs modules.  On présente brièvement les trois modules principaux :   \par 
\textbf{Hadoop HDFS}  (Hadoop Distributed File System) est un système de fichier distribué permettant de stocker les fichiers volumineux dans un cluster Hadoop  tout en  offrant une haute disponibilité, fiabilité et tolérance aux pannes.\par
\textbf{Hadoop YARN} (Yet Another Resource Negotiator) est la composante  de Hadoop permettant de  gérer les ressources dans un cluster Hadoop. Ceci inclut l'allocation des ressources aux différentes applications.  YARN  garantit les différents traitements d'être exécutés sur les données stockées dans HDFS.\par

\textbf{Hadoop MapReduce} (voir la section \ref{mapreducesection}).

 %Apache Spark est présenté dans la section \ref{apache-spark}.



% est un modèle de programmation faisant partie de l'écosystème Hadoop. 



% conçu pour créer des applications  qui traitent des données massives en parallèle et sur un cluster de machines. Ce framework apporte la facilité de gérer le traitement de gros volumes de données en distribuant le travail sur un ensemble de machine, en gérant le balancement des charges, en prenant en compte la synchronisation entre les éléments du cluster, etc. Ces tâches sont effectuées automatiquement.



\subsection{Apache Spark } \label{apache-spark}

%\subsection{Introduction à Apache Sprk}

Apache Spark\footnote{URL : \url{https://spark.apache.org/}, consulté le $14/12/2018$.} 
est un framework de calcul distribué. C'est un ensemble de composantes conçues pour assurer la rapidité, la facilité d'utilisation ainsi que la flexibilité dans l'analyse des données à grande échelle. Plusieurs APIs sont disponibles pour interagir avec Spark et 
 appliquer les transformations sur les données à analyser. 

\paragraph{Core Concepts et architecture de Spark}

\subparagraph{Spark Cluster et Resource Management System}

Spark est un système distribué conçu pour traiter les données massives rapidement et avec efficacité. Ce système est déployé sur un ensemble de machines, qu'on appelle Spark \textit{cluster}. La taille du cluster en nombre de machines est variable. Il est possible d'avoir un cluster avec peu de machines mais aussi un cluster avec des milliers de machines. En vue de gérer efficacement les machines d'un cluster, les entreprises recourent à un système de gestion de ressources tel que Apache YARN\footnote{Description dans \url{https://hadoop.apache.org/docs/current/hadoop-yarn/hadoop-yarn-site/YARN.html}, consulté le $09/12/2018$.} ou Apache Mesos\footnote{URL : \url{https://mesos.apache.org/}, consulté le $09/12/2018$.}. Les deux composantes les plus importantes dans un système de gestion de ressources sont : le \textit{cluster manager} et le \textit{worker}.

Le \textit{cluster manager} a une vue globale de l'emplacement des \textit{workers}; la mémoire qu'ils ont et le nombre de c\oe{}urs CPU dont chaque \textit{worker} dispose. Le rôle du \textit{cluster manager} est d'orchestrer le travail en le désignant à chaque \textit{worker}. Tandis que le rôle d'un \textit{worker} est de fournir les informations utiles pour le \textit{cluster manager} ainsi que la réalisation du travail y assigné. La Figure \ref{fig:cluster-overview} montre l'interaction entre une application Spark, le cluster manager et les \textit{workers}.


\begin{figure}[h]
	\centering
	\captionsetup{justification= centering}
	\includegraphics[width=0.7\linewidth]{illustrations/cluster-overview.jpg}
	\caption{ Interaction entre une application Spark et le cluster manager. Source : \cite{eginning-Apache-Spark-2-cluster-overwiew}}
	\label{fig:cluster-overview}
\end{figure}




\subparagraph{Application Spark} \label{sparkpresentationsection}
Une application Spark consiste en deux parties. La première partie concerne la logique décrivant les traitements à appliquer sur les données.  Cette logique est décrite en utilisant les APIs\footnote{API en Java, Scala, Python ou R.} disponibles. La deuxième partie est appelée le \textit{driver}, c'est le coordinateur principal d'une application Spark. Le driver interagit avec le cluster manager afin de trouver les machines sur lesquelles le traitement de données doit être réalisé. Ainsi, pour chacune de ces machines, le driver Spark lance le processus \textit{executor} en passant par le cluster manager. Un autre rôle du  driver Spark est de gérer et de distribuer les tâches Spark en provenance de l'application Spark sur chaque executor. Pour précision , dans la Figure \ref{fig:cluster-overview}, la classe \textit{SparkSession} est le point d'entrée vers une application Spark.

\subparagraph{Spark driver et executor}

Chaque Spark \textit{executor} est alloué exclusivement à une application Spark spécifique et la durée de vie d'un \textit{executor} est celle de l'application Spark. 

Spark utilise l'architecture master-slave. Spark \textit{driver} est le master et Spark \textit{executor} est le slave. De ce fait, une application Spark n'a qu'un seul Spark driver et plusieurs Spark \textit{executors}. Chaque Spark \textit{executor} s'occupe d'un traitement  sur une partie  des données à analyser. De cette manière,  Spark est capable de traiter  les données de façon parallèle. 
%La Figure \ref{fig:small-cluster-3} illustre un exemple d'un cluster. Ce dernier est composé de trois \textit{executors}.
%\begin{figure}[H]
%	\centering
%	\includegraphics[width=0.7\linewidth]{illustrations/small-cluster-3}
%	\caption{Un exemple d'un cluster formé de trois executors. Source : \cite{eginning-Apache-Spark-2-cluster-example}}
%	\label{fig:small-cluster-3}
%\end{figure}


\subparagraph{Spark Unified Stack} \label{Spark Uniffied-Stack}Spark offre ce qu'on appelle Spark \textit{Stack}. C'est un ensemble de composantes construites au dessus de la composante Spark Core.  Ces composantes sont conçues pour répondre à des besoins spécifiques :
\begin{itemize}
	\item Spark SQL  est conçu effectuer des traitements sur des données structurées. Les traitements s'effectuent en se basant sur des requêtes SQL;
	\item Spark Streaming est utilisé pour les traitements    en temps réel des données en flux;
	\item  GraphX est destiné au traitement de graphes. Des fonctionnalités sont offertes afin d'analyser les graphes;
	\item MLib est conçu pour le machine learning. Ceci inclut la disponibilité des différents algorithmes et utilitaire du machine learning. Par exemple, les algorithmes de clustering, la classification, etc;
	\item SparkR est consacré aux traitements liés au machine learning en utilisant  R.
\end{itemize}
\subparagraph{Spark Core} est la base du moteur Spark pour le traitement distribué de données.  On distingue deux parties formant Spark Core. Premièrement, la partie concernant l'infrastructure distribuée du calcul. Cette dernière est responsable de la distribution, de la coordination et de la planification des tâches  sur les différentes machines formant le cluster. De plus, cette partie gère l'échec d'un traitement donné et le transfert de données entre les machines. Le deuxième élément formant Spark Core est appelé RDD (Resilient Distributed Dataset). Un RDD est une collection partitionnée d'objets, tolérante aux pannes et en lecture seule. 
%La Figure \ref{fig:unified-stack}  présente les différentes entités du Spark Unified Stack avec Spark Core.
%\begin{figure}[H]
%	\centering
%	\captionsetup{justification=centering}
%	\includegraphics[width=0.7\linewidth]{illustrations/unified-stack}
%	\caption{ Spark Unified Stack. Source : \cite{eginning-Apache-Spark-2-unified-stack}}
%	\label{fig:unified-stack}
%\end{figure}

\paragraph{Resilient Distributed Datasets}\label{rdd-presentation} ~


Spark dispose d'une abstraction notée Resilient Distributed Datasets (ou RDDs). Un RDD est une collection d'objets immuable. Ces objets sont répartis sur les n\oe{}uds du cluster afin d'être traités en parallèle. Un RDD peut être conservé  pour une éventuelle réutilisation. On distingue deux manières de persistance. La conservation du RDD dans la mémoire (in-memory) ce que garantit l'amélioration des performances. En outre, un RDD peut être aussi conservé dans un disque. 

Les RDDs supportent deux types d'opérations sur les objets stockés : les transformations et les actions. Une  transformation appliquée sur un RDD crée un nouveau RDD, par exemple la transformation \textit{filter} retourne un RDD ayant vérifié la condition donnée en entrée.  Pour les actions, une action appliquée sur un RDD retourne une seule valeur, par exemple l'action \textit{count} compte le nombre d'objets d'un RDD.



%Dans la Figure 	\ref{fig:globalviewrdd}, Input data représente  les données à analyser en utilisant Spark. Ces données sont récupérées depuis des sources extérieures vers Spark. Ce dernier crée un RDD basé sur ces données. Un RDD est représenté par le rectangle en orange, les morceaux en orange dans le rectangle représentent les partitions d'un RDD. 
%On peut enchaîner plusieurs transformations sur un RDD. Comme une transformation est à la base  \textit{lazy}, les partitions ne sont partagées sur les n\oe{}uds du cluster qu'à la suite de l'appel d'une action. Une fois qu'une partition est localisée sur un n\oe{}ud donné, les transformations ainsi que les actions peuvent s'enchaîner.


 La Figure 	\ref{fig:globalviewrdd} illustre un flux de données avec l'utilisation de Spark. Dans cette figure, \textit{input data} représente les données qu'on souhaite analyser. Ces dernières sont en provenance de sources de stockage externes. Le framework Spark crée le RDD représenté par le premier rectangle orange à gauche. Ce dernier  comprend  des petits rectangles chacun représente une partition du RDD.  Les transformations  peuvent être enchaînées sur le RDD créé sans être exécutées. Les partitions seront envoyées à travers les n\oe{}ds dés que le \textit{driver} appelle une action sur ce RDD. Un n\oe{}ud est une machine avec des ressources de stockage (\textit{disk}), de calcul (\textit{CPU}), etc. Enfin, le reste des opérations peuvent s'enchaîner  sur chaque n\oe{}ud où se trouvent les données.

En cas de perte de partition pour une raison ou une autre, Spark est capable de reproduire automatiquement la partition en question. Cette fonctionnalité est assurée via le DAG (Direct Acyclic Graph). Dans ce graphe, Spark enregistre toutes les opérations appliquées sur un RDD.

\begin{figure}[h]
	\centering
	\captionsetup{justification= centering}
	\includegraphics[width=0.7\linewidth]{illustrations/global_view_rdd}
	\caption{Exemple d'un flux de données avec Spark}
	\label{fig:globalviewrdd}
	\source{ \url{https://www.duchess-france.org/starting-with-spark-in-practice/}, consultée le $15/12/2018$.}
\end{figure}

%Resilient Distributed Dataset (RDD) allows Spark to transparently store data on the memory, and send to disk only what’s important or needed. As a result, a lot of time that is spent on the disc read and write is saved.

%Le RDD (Resilient Distributed Dataset) permet à Spark de stocker des données de manière transparente dans la mémoire et d’envoyer sur disque uniquement ce qui est important ou nécessaire. En conséquence, une grande partie du temps consacré à la lecture et à l'écriture du disque est enregistrée.[!]

\paragraph{Lazy evaluation} \label{lazy-evaluation}
Une transformation en Spark est considérée comme \textit{lazy}. Ceci implique que lorsqu'on exécute des fonctions de transformation en Spark, celles-ci ne sont pas exécutées de suite. Par contre, elles sont enregistrées dans le graphe (DAG).  Les transformations sont exécutées une fois le \textit{driver} invoque un appel à une fonction de type action. \textit{Lazy} ou l'évaluation paresseuse est un mécanisme permettant d'éviter le chargement de données depuis la source tant que ceci n'est pas nécessaire. Par conséquent, cela peut améliorer considérablement les performances.


 




%\paragraph{Les APIs de Spark}

%Le framework Spark a été écrit en Scala.  Il existe plusieurs APIs pour utiliser les fonctionnalités fournies par ce framework : Scala, Python, Java et R. 
%Il n'est pas évident de choisir entre ces quatre langages, car ce choix dépend du contexte de l'analyse. 


%Nous allons comparer ces langages dans le contexte du Big Data.



\paragraph{Les APIs du Spark}~

Spark dispose des APIs standards permettant aux développeurs  de créer une application Spark. Il existe une API en Scala, Python, Java et R. 

\paragraph{Lancement d'une application avec de Spark} \label{spark-master-modes}~

%Afin de pouvoir exécuter une application Spark, il faut l'installer. 
On distingue deux modes d'exécution d'une application  Spark : cluster et local. En  mode local, l'application  peut être exécuté au sein d'une machine locale. Par exemple,  avec un seul \textit{worker} thread (aucun parallélisme),  en précisant $K$ \textit{worker} sur $K$ threads, idéalement $K$ est le nombre des  c\oe{}urs de la machine sur laquelle Spark est installé, etc. 
 Pour le mode cluster, on distingue plusieurs types de clusters. Ces derniers  varient suivant le gestionnaire de ressources qui assure la communication entre les différentes entités du cluster selon le principe master-slave. Par exemple Hadoop YARN\footnote{URL : \url{https://hadoop.apache.org/docs/current/hadoop-yarn/hadoop-yarn-site/YARN.html}, consulté  le $14/05/2019$ } et Apache Mesos\footnote{URL : \url{https://mesos.apache.org/}, consulté le $14/05/2019$.}. La liste détaillée des alternatives pour chaque mode est donnée dans le site Web de Spark\footnote{URL : \url{https://spark.apache.org/docs/latest/submitting-applications.html\#master-urls}, consulté le $14/05/2019$.}. 
 
Il est possible d'installer Spark dans un cluster Amazon EMR (Elastic MapReduce)\footnote{URL  : \url{https://aws.amazon.com/fr/emr/}, consulté le $08/05/2019$.}. Par défaut, ce cluster utilise YARN comme gestionnaire de ressources.

\paragraph{Apache Spark VS Apache Hadoop}

Hadoop MapReduce est considéré dans \cite{Global-Journals} comme étant moins rapide en le comparant au framework Apache Spark.  Une étude \cite{article-comparaison-spark-hadoop} comparative entre Hadoop MapReduce et Spark a montré que Spark est $ 3 $ à $ 4 $ fois rapide que Hadoop MapReduce.
 Ceci est dû au fait que Spark traite les données suivant l'approche in-memory. Tandis que  les traitements basés sur le framework Hadoop sont basés sur des lectures/écritures sur le disque.  
%http://csjournals.com/IJCSC/PDF7-2/13.%20JP.pdf
\subsection{Amazon Elastic MapReduce} \label{emr-aws-presentation}

Amazon Elastic MapReduce\footnote{URL : \url{https://docs.aws.amazon.com/fr_fr/emr/}, onsulté le $10/05/2019$.} (EMR) est  un service Web proposé par Amazon  destiné aux traitements des données massives en utilisant Apache Hadoop, Apache Spark et autres.
Amazon EMR distribue le traitement de données à travers un cluster  de machines virtuelles redimensionnable. Ces machines sont des instances de type Amazon EC2\footnote{URL : \url{https://aws.amazon.com/ec2/}, consulté le $14/05/2019$.}.
Il existe plusieurs variantes d'instances EC2. Les instances sont conçues pour répondre 
aux besoins classés par AWS dans les catégories suivantes : usage général, calcul optimisé, calcul accéléré et stockage optimisé.
Elles se varient suivant leurs caractéristiques techniques comme la  RAM disponible, le nombre de CPUs, le nombre de c\oe{}urs physiques, etc. Les coûts d'utilisation des instances EC2 dépendent de ces caractéristiques,  de la région où se trouvent ces instances et du temps d'utilisation. Le coût d'utilisation des instances EC2 indépendamment d'EMR est différent du cas où une instance est utilisée pour former un cluster EMR.

Par défaut, Amazon EMR utilise YARN pour gérer les ressources. On distingue trois types de n\oe{}uds dans Amazon EMR. 
\begin{itemize}
	\item N\oe{}ud principal (\textit{Master node}) : il gère le cluster.
	\item N\oe{}ud de noyau (\textit{Core node}): exécute les tâches et stocke les données dans le système de fichier distribué Hadoop HDFS sur le cluster.
	\item N\oe{}ud de  tâche (\textit{Task node}) : il exécute les tâches et ne stocke pas les données dans HDFS. Les n\oe{}uds de tâches sont facultatifs.
\end{itemize}

Les coûts appliqués  à l'utilisation du service Amazon EMR dépendent des caractéristiques techniques des entités du cluster, la région d'Amazon choisie du cluster et enfin le temps d'utilisation du cluster. Il est possible d'estimer les frais d'utilisation d'Amazon EMR via le calculateur disponible sur le site Web d'Amazon\footnote{URL : \url{https://calculator.s3.amazonaws.com/index.html\#s=EMR}, consultée le $14/05/2019$.}.


\section{Conclusion}

Dans ce chapitre,  nous avons décrit brièvement  quelques technologies du Big Data, car la liste de toutes les technologies est très longue. Afin de découvrir ces technologies en pratique, nous allons aborder dans le chapitre \ref{chap:application-on-traceroutes} l'utilisation de ces technologies dans le cas de l'analyse des délais d'un lien décrite dans la chapitre \ref{chap:algorith-detection}.    










\chapter{Etude comparative des technologies Big Data}

\section{MongoDB}


\section{DynamoDB}

\section{AWS Athena + S3}

\section{Apache Spark }
\chapter*{Conclusion}
\addcontentsline{toc}{chapter}{Conclusion}
%Rappel de la problématique.

L'objectif du présent travail consiste à évaluer un sous-ensemble de technologies Big Data sur des données à grande échelle en provenance du dépôt d'Atlas. Etant donné que l'évaluation des technologies Big Data peut prendre plusieurs formes, la présente évaluation envisage 
 la mise en place de la technologie  et   le calcul du temps d'exécution obtenu en analysant différents échantillons de données avec cette technologie.
 
 Le modèle choisi pour l'évaluation des technologies Big Data se base sur des réponses aux requêtes de type traceroute effectués par des sondes.  Le volume et la vélocité  avec lesquels les traceroutes  sont générés dépassent la capacité des outils traditionnels comme les bases de données relationnelles. La gestion des données à grande échelle  fait partie des objectifs du Big Data.
%L'évaluation de la convenance d'une technologie Big Data nécessite de disposer de données assez suffisantes pour couvrir  le plus des  cas possibles. 
%Nous avons choisi des données dans le domaine des réseaux informatiques,  disponibles sur le dépôt de RIPE Atlas,  d'où l'intérêt  de bien détailler ce projet.



%Que peut-on déduire du travail de recherche ?

%Résultats de recherche et réponses aux questions de recherche.
%% reponse à la question 1 de recherche : pourquoi passer au big data


%données   ayant un volume important.  Une dizaine de gigaoctets de traceroutes sont générés quotidiennement. De plus, les données quotidiennes augmentent d'un jour au un autre.
%ont démontré  les limites des outils traditionnels et leur incapacité de  manipuler les données à grande échelle.

 
Après avoir manipulé de nombreuses traceroutes, nous avons appris que les technologies Big Data doivent garantir plusieurs tâches dans un processus d'analyse de donnés. En particulier, la collecte de données, le stockage de données, le traitement de ces dernières, voire d'autres.

 %sont capables de gérer les données à grande échelle. 
%Les technologies Big Data sont capables de manipuler les données à grande échelle
 En terme de stockage,   le service Amazon S3 est dédié au stockage des fichiers de données de grande taille. Les données stockées dans ces fichiers sont accessibles à travers le service Amazon Athena. 
  Une autre catégorie de stockage est  les bases de données NoSQL, dont l'exemple  du service Amazon DynamoDB. Ce dernier   garantit le stockage de toute quantité de données. Un autre exemple est la base   de données NoSQL   MongoDB Atlas. Du fait qu'elle est de type document, elle peut stocker les réponses de requête traceroute sauvegardées en format JSON.  
  
  
  En plus du stockage de données massives, la récupération de ces données pour  les traiter  rapidement est un autre défi du Big Data.
  C'est pourquoi, l'utilisation d'une technologie permettant de distribuer ces traitements permet d'améliorer considérablement une analyse en terme de temps. C'est l'objectif  du framework Spark, tel qu'il garantit la  distribution des traitements des données sur un \textit{cluster} de machines.
  
   Chaque technologie Big Data apporte ses propres concepts. En effet, pour un traitement donné, des solutions sont proposées en adéquation avec ces concepts. Une base de données MongoDB manipule des collections, des documents, etc. Amazon Athena adopte le principe de \textit{schema on read}, l'organisation des fichiers de données à travers le partitionnement, etc. La mise en place d'une application basée sur Spark nécessite  la traduction des traitements pour qu'elles soient effectués le plus possible  de manière distribuée. De plus, il  faut construire les machines du \textit{cluster} tout en prenant en compte le coût, le temps et les volumes de données manipulés.
    
   %demande des efforts pour choisir les entités du cluster et  configurer ce dernier tout en prenant en compte le volume de données traitées, la nature des opérations, les caractéristiques techniques du cluster, etc.  Tous ces éléments ont un effet sur les frais d'utilisation de la présente technologie.
  
  
  % gérer différents types de projets vu les modules riches qu'il inclut. Comme les projets où on manipule des données semi-structurées, des données structurées, etc. Des projets relatifs au machine learning.  Ce framework est capable de distribuer les traitements sur cluster de machine et les données sont manipulées en mémoire.
 

%la réutilisation d'un travail existant
%Le travail de référence sur lequel ce travail est  
%Le travail de référence 
 %La réécriture complète ou partielle d'un travail autre travail peut 
%Les implémentations présentées dans ce travail utilise le principe de détection des anomalies créé par Fontugne et al. \cite{DBLP:journals/corr/FontugneAPB16}. La réécriture exacte n'est pas toujours optimale, car certains choix ont été faits selon la technologie avec laquelle est implémentée. 


%l'utilisation des technologies Big Data
%Suite à l'application  du sous-ensemble de technologies sur l'outil de détection, on distingue deux défis. Le premier est relatif à l'utilisation d'un modèle existant. Tandis que le deuxième défi est lié à la mise en place d'une technologie Big Data.


L'adoption d'une technologie Big Data en particulier pour traiter des données  doit prendre en compte plusieurs entrées. Nous citons quelques-unes selon l'évaluation réalisée dans ce mémoire. D'abord, il faut préciser la fréquence de l'analyse : si l'analyse est effectuée une seule fois ou d'une manière périodique. La fréquence de la génération des données à analyser.   Le temps admissible pour analyser un volume de données, au-delà de ce temps, l'analyse est considérée non optimale. 
%Ce temps dépend à la fois des données analysées et les choix techniques comme les caractéristiques techniques d'un cluster EMR. De plus, la nature de données manipulées est importante. 
%Ceci  permet de choisir la technologie la plus adaptée. Car 
La nature des données traitées, car certaines technologies sont conçues  et optimisées pour traiter un type de donnée en particulier, c'est l'exemple des bases de données NoSQL. La nature des traitements à appliquer sur les données. Les frais d'utilisation d'une technologie est un autre facteur dans le choix de cette dernière. 


%perspectives


%La disponibilité des outils informatiques permettant de stocker et de traiter des données à grande échelle, avec efficacité,  est important. Toutefois,  le modèle qui dirige l'ensemble de données est aussi de même degré d'importance dans un processus d'analyse de données, comme le cas  du modèle créé par Fontugne et al. \cite{DBLP:journals/corr/FontugneAPB16} pour la détection des anomalies.  Nous avons compris le fonctionnement ainsi que le paramétrage du modèle. Comme continuité du présent travail, on note quelques possibilités. 
Nous avons validé les résultats obtenus   avec  l'implémentation de référence avec ceux obtenus avec nos deux implémentations : les services d'Amazon S3 et Athena et  Spark/Scala. Notre évaluation des technologies Big Data a examiné la mise en place de ces dernières ainsi que  le temps d'exécution obtenu sur différents échantillons de données.  Comme  suite de ce travail, il est intéressant d'évaluer la précision de l'outil de détection  en choisissant des données ciblées.
Du fait que RIPE Atlas dispose du mode Streaming dans lequel les données peuvent être récupérées en temps réel, il est intéressant d'évaluer l'intégration de l'extension \textit{Spark streaming} pour analyser des données  d'Atlas en temps réel.  

 %soutil de détection 
%Par exemple, il est possible de réévaluer la précision  de ce modèle avec de nouvelles données, varier les paramètres du modèle de la détection comme la méthode adoptée au calcul des intervalles de confiance, etc. 


 































\appendix

\chapter{Amazon Athena } \label{athena-appendix}

\section{Création de la table traceroutes } \label{creer-table-traceroute}
La création d'une table reprend plusieurs parties :
\begin{itemize}
	\item les colonnes de la table avec le type correspondant (int, string, array pour définir une liste, struct pour définir un objet );
	\item LOCATION : c'est l'endroit où les données sont stockées dans Amazon S3, il faut préciser le chemin vers le compartiment de données;
	\item ROW FORMAT SERDE : elle définit la manière dont chaque ligne d'un fichier de données est sérialisée/désérialisée par Amazon Athena;
	\item PARTITIONED BY : elle définit la manière dont les données sont organisées dans le compartiment de données; 
	\item WITH serdeproperties : elle définit les options de la sérialisation/désérialisation.
\end{itemize}
\begin{lstlisting}[language=SQL, basicstyle=\footnotesize,]
CREATE EXTERNAL TABLE traceroutes(
	af int,
	bundle int,
	dst_addr string,
	dst_name string,
	fw int,
	endtime int,
	`from` string,
	group_id int,
	lts int,
	msm_id int,
	msm_name string,
	paris_id int,
	prb_id int,
	proto string,
	size int,
	src_addr string,
	`timestamp` int,
	ttr float,
	type string,
    result array< struct< hop:int,error:string, result:array<
        struct<x:string, err:string, `from`:string, ittl:int, edst:string, late:int, mtu:int, rtt:float, size:int, ttl:int , flags:string, dstoptsize:int, hbhoptsize:int, icmpext:
        	struct<version:int, rfc4884:int, obj:array< 
        		struct<class:int, type:int, mpls:array<struct< exp:int, label:int, s:int, ttl:int>>>>>>>>> 
)
PARTITIONED BY (
	af_ string,
	type_ string,
	msm string ,
	year string,
	month string,
	day string,
	hour string
) 
ROW FORMAT SERDE 'org.openx.data.jsonserde.JsonSerDe'
WITH serdeproperties ('paths'='af,bundle,dst_addr, dst_name,fw, endtime, from, lts, msm_id, paris_id, prb_id, proto, size, src_addr, timestamp, type,fw, msm_name' ) 
LOCATION 's3://ripeatlasdata/traceroute/source=api/'
\end{lstlisting}

\paragraph{Partitionnement de données dans un compartiment AWS S3} \label{subsubsection:partitionnement}~

Le partitionnement  de données présentes dans un compartiment Amazon S3 permet de limiter la quantité de données à analyser par une requête Amazon Athena. Le partitionnement améliore  les performances d'Amazon Athena. D'une part, on obtient une réponse rapidement, d'autre part, on réduit les coûts engendrés  suite à l'utilisation du service car on est facturé selon la quantité de données analysées.  

Les partitions créées joue un rôle similaire à celui d'un colonne durant l'interrogation d'une table dans Athena. 

Prenons un exemple, nous avons des traceroutes ayant comme adressage IP la version  4 et d'autres traceroutes ont l'adressage IPv6 :

\begin{lstlisting}
s3://ripeatlasdata/traceroute/
				type=4/
				type=6/
\end{lstlisting}

Sans l'utilisation du partitionnement et si on souhaite récupérer que les traceroutes ayant comme adressage IPv4, toutes les données (type = 4 et type = 6) sont analysées. Toutefois, en partitionnant les données suivant le type d'adressage, seuls les fichiers dans le dossier type = 4 qui sont analysés. Par conséquent, le partitionnement permet de réduire les coûts d'utilisation du service Amazon Athena, surtout si la quantité de données est très importante. 



\include{annexe_machine}
%\chapter{Notions et concepts}

\paragraph{In-memory}

\newpage
\listoffigures
\newpage
\listoftables
\newpage
\lstlistoflistings
\newpage
%\chapter{Application de quelques technologies Big Data sur l'analyse des  traceroutes} \label{chap:application-on-traceroutes}


\section{Introduction}

Ce chapitre reprend un ensemble de   technologies destinées  à la manipulation des données massives. Ce sont les technologies que nous avons expérimenté pour analyser des traceroutes disponibles dans le dépôt de RIPE Atlas. Précisément, ce sont les traceroutes permettant de tracer l'évolution du délai d'un lien comme c'est détaillé dans le chapitre \ref{chap:big-data-intro}.
% Nous allons présenter l'objectif de chaque technologie, ses avantages, ses inconvénients et ses limitations dans le cas de la présente analyse.
%Le présent chapitre reprend l'application de quelques technologies du Big Data manipulées en vue d'analyser le délai des liens. 
%On ne peut pas comparer ces technologies entre elles car elles ne se trouvent pas dans la même catégorie; quelques technologies n'assurent que le stockage, une autre technologie gère l'analyse ainsi que le stockage.  En revanche,
Les technologies que nous présentons  couvrent les besoins d'une ou de plusieurs étapes d'un processus d'analyse de données.
% (voir un exemple d'un processus d'analyse de données dans la section \ref{sec:process-data-analysis}). 
 
 %L'évaluation des performances  des technologies choisies est faite sur une machine ayant les caractéristiques reprises dans le tableau[!].

\section{Critères d'évaluation des technologies  Big Data}
Les critères d'évaluation d'une technologie Big Data  varient  suivant son objectif : stockage, calcul, etc.  En générale, la liste des critères que l'on peut considérer dans la comparaison des technologies Big Data est très longue.  Les critères sur lesquels nous  évaluons  les différentes technologies  Big Data expérimentées sont les suivants:
\begin{itemize}
	\item facilité de la mise en route et de la configuration de l'environnement de la technologie;
	\item flexibilité liée à la définition du  schéma de  données présentes dans les fichiers;
	\item temps d'exécution nécessaire pour fournir les résultats finaux d'une analyse de traceroutes;
	\item évolutivité de l'environnement Big Data mis en place pour des nouvelles données et de nouveaux besoins.
\end{itemize}

Dans la présente évaluation de quelques technologies Big Data, nous n'avons pas pris en compte d'autres critères. Car nous ne pouvons pas les évaluer. Par exemple, l'utilisation du  Big Data engendre des coûts  liés aux ressources nécessaires au stockage de données massives ainsi qu'au traitement de ces dernières. Nous avons donné des indications théoriques concernant les frais d'utilisation de deux technologies dédiées au stockage de données massives : Amazon S3 (voir le Tableau \ref{tab:pricing-s3-standard}) et MongoDB Atlas dont les frais d'utilisation  dépendent de plusieurs paramètres\footnote{Une estimation est possible suivant le fournisseur de cloud, elle est disponible  sur \url{https://www.mongodb.com/cloud/atlas/pricing}, consulté le $25/12/2018$.}. 

\section{Caractéristiques de l'environnement de test} \label{machine-openvz-caracteritics}

\paragraph{La machine de test} L'évaluation des technologies Big Data choisies sur un échantillon de traceroutes a été réalisé sur un conteneur de type OpenVZ ayant les caractéristiques suivantes :  système Debian GNU/Linux 7.11 (wheezy),  32768 MB de  RAM, CPU MHz $ 2294.331 $.

%Le Tableau \ref{tab:test-machine} présente les caractéristique de la machine sur laquelle nous avons effectué les différents tests. 
%\begin{table}[H]
%	\begin{tabular}{cc}
%		Type& OpenVZ container\\
%		RAM (MB)& 32768 \\
%		CPU & 64 (The logical CPU number of a CPU as used by the Linux kernel) \\
%	\end{tabular}
%	\caption{Caractéristiques de la machine de test}
%	\label{tab:test-machine}
%\end{table}
\begin{tcolorbox}
	Il existe différentes catégories de virtualisation. \textbf{OpenVZ} s'inscrit dans la catégorie Isolateur. Un isolateur est un logiciel permettant d'isoler l'exécution des applications dans des contextes ou zones d'exécution. Un conteneur OpenVZ  adopte un partitionnement logique au niveau des ressources systèmes : processus, réseau et système de fichier\footnote{Source : \url{http://cesar.resinfo.org/IMG/pdf/jtsiars-openvz_1_.pdf}, consultée le $29/12/2018$.}.
\end{tcolorbox}

Les différents tests effectués, présentés dans le présent chapitre, ont été effectués  au sein de cette machine. Nous notons qu'un seul test est lancé à un moment donné dans la machine.
\paragraph{Paramètres de l'analyse} Pour les paramètres de détection, nous avons utilisé les valeurs suivantes : \textit{timeWindow} est de $ 3600 $ secondes, l'intervalle de confiance est de $0,05$, \textit{alpha} est de $ 0,01 $ et \textit{minSeen} est de $3$. Les dates de début et de fin  varient suivant les traceroutes analysés.



\section{Application 1 : MongoDB} \label{mongodb-impleme}
%\paragraph{ Application sur  MongoDB}~

%Les données relatives aux mesures traceroutes peuvent être récupérées de différentes manières. Par exemple,  les traceroutes à destination des instances du serveur DNS K-root. En ce qui concerne le travail de référence, les traceroutes sont récupérés à la fois par type d'adressage : IPv4 et IPv6 en se basant sur  l'identifiant  de la mesure : $ 5001 $, $ 6006 $, etc et par date.  Ainsi les

\paragraph{Evaluation des critères de sélection}~

MongoDB est la technologie Big Data utilisée par  Fontugne et al.  dans l'implémentation de l'outil de détection \cite{InternetHealthReport}. Dans MongoDB, les traceroutes sont organisés  dans des collections.  Chaque collection stocke les traceroutes effectués lors de la journée $YYYY\_MM\_DD$ et en adressage $V$. Par convention,  $V$  vaut $6$ s'il s'agit de l'adressage IPv6 et est vide.  La nomenclature  des collections permet de ne récupérer que les traceroutes concernés par l'analyse lancée. Le nom d'une collection est structuré comme suit: 	$tracerouteV\_YYYY\_MM\_DD$.
 

 
MongoDB est une technologie conçue pour assurer  le stockage de données dans un processus d'analyse de données. Nous avons utilisé la version locale de MongoDB, la quantité de données que nous pouvons stocker ainsi que le traitement appliqué sur les données récupérées dépendent principalement des ressources de la machine dans laquelle MongoDB est installé.
 
MongoDB est flexible en terme de définition du schéma de données; aucun schéma n'est requis.   Par exemple, dans certains cas, les traceroutes planifiés ne réussissent pas à atteindre une destination, dans ce cas, la structure de ces  traceroutes est différente  de la structure des traceroutes réussis. Les deux types de traceroutes sont stockés sans contrainte.

Les données stockées dans une collection MongoDB peuvent être manipulées en mode lecture et en mode écriture. Dans le premier, on cherche à lire des données en provenance de différentes sources. C'est le plus répandu dans les projets Big Data. Pour le deuxième mode, on peut mettre à jour un enregistrement dans une collection MongoDB. Ceci est moins fréquent dans les projets Big Data.

%Généralement l'analyse de données à grande échelle se limite qu'au mode lecture de données pour en tirer les connaissances. 
%MongoDB est adapté  aux projets visant la lecture de données massives mais aussi aux projets où on envisage la mise à jour d'un objet dans une collection (modification ou suppression). 
 
 MongoDB est évolutif; en cas de   mise à jour de la structure de nouveaux  objets traceroutes par Atlas,  cela n'affecte pas les données précédemment  stockées  dans MongoDB.
 
 %Malgré la convenance de MongoDB aux données non structurées et massives, l'utilisation de telle base de données, en version locale, nécessite l'ajustement de la machine locale où MongoDB tourne. 



%\paragraph{Les limitations de MongoDB}

%L'implémentation proposée de l'outil de détection utilise la version locale de la base de données MongoDB pour le stockage des données.  La quantité de données dont MongoDB peut stocker dépend de l'espace mémoire de stockage disponible dans la machine dans laquelle MongoDB est installé. De plus, les performances d'une détection lancée concernant une période donnée dépendent de la RAM de la machine en question. Pour conclure, l'utilisation de la version locale d MongoDB pour analyser les traceroutes à travers l'outil de détection dépend typiquement de la machine locale.
%\paragraph{Les performances de MongoDB}
\paragraph{Performances de la base de données MongoDB dans l'analyse des délais }~


Nous mesurons le temps écoulé durant l'analyse des traceroutes, stockés dans une base de données MongoDB,  en vue de détecter les anomalies dans le délai des liens. C'est le temps nécessaire à l'accomplissement des étapes de la phase I, de la phase II et de l'écriture des résultats dans un fichier. Chaque ligne de ce dernier  décrit un lien comme l'exemple donné dans le Listing \ref{resultLink}.  

Dans le Tableau \ref{tab:mongotiming-timing}, nous varions l'ensemble de traceroutes. Pour Chaque période, nous mesurons le temps nécessaire pour analyser les traceroutes capturés durant cette période pour plusieurs reprises. L'analyse de chaque période est fait $5$ fois, ce qu'on appelle ici des essais : Essai 1, Essai 2, etc.  Les traceroutes analysés sont ceux à destination des instances du f.root-servers.net\footnote{Voir les détails de la mesure 5004 sur \url{https://atlas.ripe.net/measurements/5004/}, consultée le $12/12/2018$}.

%en terme de stockage pour mesurer le temps nécessaire pour avoir  l'évolution de tous les liens présents dans  les traceroutes analysés, nous ne présentons que les premier trois essais. 


\begin{table}[h]
	\captionsetup{justification=centering}
	\resizebox{1\textwidth}{!}{
	\begin{tabular}{cccccccc}
\textbf{Période}&\textbf{Taille (bytes)}&\textbf{Essai 1 (s)}&\textbf{Essai 2 (s)} &\textbf{Essai 3 (s)}&\textbf{Essai 4 (s)}&\textbf{Essai 5 (s)}&\textbf{Médiane (s)} \\ \hline
$ 07/02/18 - 07/02/18 $&$  1,028,343,572 $&$  $&$  $&$  $ &&&\\ \hline 
$ 07/02/2018 - 08/02/2018 $&$  $&$  $&$  $&$  $ &&&\\ \hline 
$ 07/02/2018 - 09/02/2018 $&$  $&$  $&$  $&$  $ &&&\\ \hline 
$ 07/02/2018 - 10/02/2018 $&$  $&$  $&$  $&$  $&&&\\ \hline 
$ 07/02/2018 - 11/02/2018 $&$  $&$  $&$  $&$  $&&&  \\ \hline 
	\end{tabular}
}
\caption{Les temps d'exécution d'analyse de traceroutes en fonction de la taille de données avec MongoDB}
\label{tab:mongotiming-timing}
\end{table}

Nous reprenons les informations du Tableau \ref{tab:mongotiming-timing} dans la Figure \ref{fig:mongodbtiming}. L'axe  des abscisses représente la taille des fichiers contenant les traceroutes  analysés, appelée  $q$. L'axe  des ordonnées  représente le temps nécessaire à l'analyse d'une quantité  de traceroutes. Nous agrégeons les temps des différents essais et nous calculons  leur valeur minimale, maximale et la médiane.
Pour précision, le temps calculé est la différence entre l'instant  qui précède le lancement de l'analyse et l'instant qui suit la fin de l'analyse.

\begin{figure}[h]
	\centering
	\captionsetup{justification=centering}
	\includegraphics[width=0.7\linewidth]{illustrations/mongoDBtiming_0}
	\caption{}
	\label{fig:mongodbtiming}
\end{figure}

%Tandis que la  Figure \ref{fig:mongodbtiming}	 reprend la médiane  des temps d'exécution de la détection en fonction de  la taille de données analysées en utilisant MongoDB. 
%La Figure 	 \ref{fig:moustachemongodb} illustre la variation de la distribution des temps d'exécution. La Figure 	 \ref{fig:moustachemongodb} ainsi que la Figure \ref{fig:mongodbtiming} ont été obtenues en considérant les temps d'exécution de $10$ essais pour chaque taille de données.




%\begin{figure}[h]
%	\centering
%		\captionsetup{justification=centering}
%	\includegraphics[width=0.7\linewidth]{illustrations/moustacheMongodb_0}
%	\caption{Les temps d'exécution en fonction de la taille de traceroutes analysés (MongoDB)}
%	\label{fig:moustachemongodb}
%\end{figure}

%La variation des temps d'exécution en fonction du nombre de traceroutes 


%Nous évaluons les temps d'exécution lors de l'analyse des délais des liens présents dans un ensemble de traceroutes, en utilisant MongoDB comme technologie de stockage de données massives.   

%On distingue deux types de variations : la taille de l'ensemble de données en terme de stockage et la taille en nombre de traceroutes présents dans l'ensemble de données.

%\begin{table}[h]
%	\captionsetup{justification=centering}
%	\resizebox{1\textwidth}{!}{
%		\begin{tabular}{cccccccc}
%			\textbf{Période}&\textbf{Taille (bytes)}&\textbf{Essai 1 (s)}&\textbf{Essai 2 (s)} %&\textbf{Essai 3 (s)}&\textbf{Essai 4 (s)}&\textbf{Essai 5 (s)}&\textbf{Médiane (s)} \\ %\hline
%			$ 07/02/18 - 07/02/18 $&$  1,028,343,572 $&$ 752,03 $&$ 750 $&$ 755,24 $ &&&\\ \hline 
%			$ 07/02/2018 - 08/02/2018 $&$ 2 $&$ 1499,28 $&$ 1504,98 $&$ 1503,62 $ &&&\\ \hline 
%			$ 07/02/2018 - 09/02/2018 $&$ 3 $&$ 2275,89 $&$2265,96 $&$ 2284,98 $ &&&\\ \hline 
%			$ 07/02/2018 - 10/02/2018 $&$ 4 $&$ 3035,19 $&$ 3043,21 $&$ 3057,45 $&&&\\ \hline 
%			$ 07/02/2018 - 11/02/2018 $&$ 5 $&$ 3871 $&$ 3889,57 $&$ 3894,5 $&&&  \\ \hline 
%		\end{tabular}
%	}
\section{Application 2 : Amazon DynamoDB}


%\paragraph{Application sur les traceroutes}~


L'élasticité est une des caractéristiques attirantes des services web d'Amazon. En particulier, c'est le cas d'Amazon DynamoDB. Ainsi, une implémentation basée sur Amazon DynamoDB  n'a pas à se soucier de la capacité  de stockage de données si la quantité de données évolue rapidement. 

 Amazon DynamoDB  n'assure que le stockage de données dans un processus d'analyse de données. La récupération et le traitement  des données stockées nécessitent l'ajustement des ressources de la machine qui reçoivent ces données, pareillement à MongoDB. La différence se situe à l'évolutivité implicite du stockage de données, qui ne se limite que par la capacité de stockage physique d'AWS. Tandis qu'une installation locale de MongoDB est liée aux ressources de la machine hébergeant ce dernier.  Nous n'avons pas expérimenté Amazon DynamoDB pour analyser les traceroutes, étant donné que notre évaluation des temps d'exécution est effectuée sur une machine locale, nous aurons les mêmes remarques que dans le cas de MongoDB en ce qui concerne l'ajustement des ressources de la machine qui reçoive les données.  
 
 A titre indicatif, une heure de tous les traceroutes effectués par toutes les sondes Atlas, concernant tous les identifiants de mesure,  fait une taille moyenne de  $620$ MB en format compressé, ce que représente une quantité d'environ $9$ GB en format texte.
%Toutefois,  au moment de de la récupération et de la manipulation de ces données, il faut ajuster les ressources pour pouvoir récupérer et traiter une quantité importante de données.

\section{Application 3 : Amazon S3, Amazon Glue  et Amazon Athena }

%\paragraph{Application sur les traceroutes}~
\paragraph{Vue générale}~

Nous avons combiné les trois services d'Amazon (Amazon S3, Amazon Glue  et Amazon Athena)  afin de créer un environnement d'analyse de données massives. 
Un des scénarios possibles mettant en pratique ensemble ces trois services est illustré dans la Figure
\ref{fig:gluecrawler}\footnote{Amazon Redshift  est un entrepôt de données et  Amazon Quicksight  est un service cloud d'informatique décisionnelle.}. Nous détaillons chaque services dans les sections suivantes.

Afin d'utiliser Amazon Athena pour l'interrogations des traceroutes stockés dans des fichiers, nous avons besoin d'abord de stocker les fichiers dans Amazon S3. De plus, nous avons besoin de créer un  schéma de données. Il s'agit de créer une table comme les tables dans un SGBDR. Chaque enregistrement dans cette table correspond à une ligne dans les fichiers de données censés être lus par cette table. Il existe deux manières pour créer une table dans Athena : en utilisant Amazon Glue ou création manuelle. \textit{traceroutes\_api} désigne le nom de la table reprenant tous les traceroutes.
%Une vue globale du  processus de l'analyse  est illustré dans la Figure  

\begin{figure}[h]
	\centering
	\captionsetup{justification=centering}
	\includegraphics[width=1\linewidth]{illustrations/glue_crawler}
	\caption{Une combinaison des services web d'Amazon : Amazon S3, Amazon Glue, Amazon Athena, Amazon Quicksight  et Amazon Redshift}
	\label{fig:gluecrawler}
	\source{\url{https://docs.aws.amazon.com/fr_fr/athena/latest/ug/glue-best-practices.html}, consultée le $16/05/2018$.}
\end{figure}


\paragraph{Création de la table traceroutes avec Amazon Glue}~

Nous avons lancé    la détection automatique du schéma, avec Amazon Glue, d'un ensemble de  traceroutes enregistrés dans un fichier faisant une taille de $500$ MB. Toutefois, la détection a échoué. Autrement dit, Amazon Glue n'a pas pu inférer le schéma d'une seule table capable de lire tout traceroute dans ce fichier.  L'échec de l'inférence est dû au fait que le fichier contient des traceroutes différents en terme de structure, car la structure dépend de la version du firmware de la sonde ayant effectué le traceroute. Les différentes versions du firmware  pour chaque type de mesure sont détaillées dans le site Web d'Atlas\footnote{\url{https://atlas.ripe.net/docs/data_struct/}, consultée le $16/01/2018$.}.
 %L'origine de cette différence  est le fait que ces traceroutes ont été effectués par des sondes ayant un firmware différent. Car le contenu des résultats d'une requête traceroute  et son organisation dans un objet JSON dépend partiellement du firmware de la sonde. 
 
\paragraph{Création manuelle de la table \textit{traceroutes\_api}}~

Nous avons créé la structure de la table \textit{traceroutes\_api}  manuellement en se basant sur la structure détaillée d'une réponse traceroute pour chaque version du firmware. Les différentes structures de réponses d'une requête traceroute n'a posé aucun problème dans la création manuelle de la table. Dans notre cas, la réussite de la création manuelle est due au fait que les attributs dont l'outil de détection a besoin sont présents dans toutes les versions du firmware d'une part. D'autre part, Amazon Athena est flexible en ce qui concerne l'association entre un objet JSON et un enregistrement dans une table. Autrement dit, si un attribut existe dans l'objet JSON, la colonne correspondante prend sa valeur et vide dans le cas échéant.  

\paragraph{Partitionnement des données stockés dans Amazon S3}~

Nous avons   pris en compte  le partitionnement de données dans un compartiment S3 dans la création de la table \textit{traceroutes\_api}.  L'utilisation du partitionnement est optionnel. 
Le partitionnement  de données présentes dans un compartiment Amazon S3 permet de limiter la quantité de données à analyser par une requête Amazon Athena. Le partitionnement améliore  les performances d'Amazon Athena. D'une part, la requête s'exécute rapidement. D'autre part, le partitionnement réduit les coûts engendrés  suite à l'utilisation d'Amazon Athena, car ce dernier est facturé selon la quantité de données analysées. En pratique,  une partition créée joue un rôle similaire à celui d'une colonne durant l'interrogation d'une table dans Athena. 

Prenons un exemple illustrant l'apport du partitionnement. Nous avons des traceroutes effectué en adressage IP la version  $ 4 $ et $ 6 $.


\textit{af\_} désigne le type d'adressage : \textit{af\_} vaut $4$ en cas d'adressage IPv4 et $6$ en cas d'adressage IPv6. Sans l'utilisation du partitionnement et si on ne souhaite récupérer que  les traceroutes ayant comme adressage IPv4, tous les traceroutes présents dans le compartiment S3 (appelé \textit{s3://ripeatlasdata}), dédié au stockage des traceroutes récupérés depuis le dépôt d'Atlas, sont évalués\footnote{L'évaluation du type  d'adressage est effectué selon la valeur de l'attribut \textit{af} d'un traceroute, il vaut \textit{4} ou \textit{6}.}.

Toutefois, en partitionnant les données suivant par exemple le type d'adressage, seuls les fichiers dans la partition\footnote{Partition dans le sens d'Amazon Athena.} af\_ = 4 qui sont analysés. Par conséquent, le partitionnement permet de réduire les coûts d'utilisation du service Amazon Athena, surtout si la quantité de données est très importante. 


Les partitions   créées sont illustrées  dans la Figure 	\ref{fig:partitionnement-athenaa}. Nous détaillons les différentes partitions dans le Tableau \ref{tab:partition-description}. Nous donnons le nom de la partition dans la première colonne, quelques valeurs de chaque partition dans la deuxième colonne et la troisième colonne de ce tableau reprend une description de la partition.

\begin{figure}[H]
	\centering
	\captionsetup{justification=centering}
	\includegraphics[width=1\linewidth]{illustrations/partitionnement-athena}
	\caption{L'organisation des traceroutes dans le compartiment Amazon S3 \textit{s3://ripeatlasdata}}
	\label{fig:partitionnement-athenaa}
\end{figure}

\begin{table}
		\centering
		\captionsetup{justification=centering}
\begin{tabular}{|c|c|l|}
	\hline 
\textbf{partition}	& \textbf{Valeurs} & \multicolumn{1}{c|}{\textbf{Commentaires} }\\ 
	\hline 
 source& api & Les  traceroutes récupérés depuis le dépôt  d'Atlas via l'API \\ 
	\cline{2-3}
	 &typeanddate& Les  traceroutes récupérés depuis la page Web\\
	\hline 
	af\_& 4  & Les traceroutes en adressage IPv4 \\ 
	\cline{2-3} &6& Les traceroutes en adressage IPv6\\	\hline 
	type& builtin  & Les traceroutes en provenances des mesures intégrées \\ 
	\cline{2-3} 
 &anchor& Les traceroutes à destinations des ancres\\ \hline
	msm& 5001 & Les traceroutes ayant msm\_id = 5001 \\ 
	\cline{2-3}  &5004& Les traceroutes ayant msm\_id = 5001 \\
	\hline 
	year& 2016 & Les traceroutes effectués en $2016$ \\ 
	\hline 
	month& 10 & Les traceroutes effectués en octobre \\ 
	\hline 
	day& 1 & Les traceroutes effectué le premier du mois \\ 
	\hline 
\end{tabular}
\caption{Exemple des partitions créées dans un compartiment Amazon S3} 
\label{tab:partition-description}
\end{table}

 Les   partitions \textit{af\_} et \textit{type\_} sont nommée de cette manière, au lieu de \textit{af} et \textit{type} car  la table \textit{traceroutes\_api} contient des colonnes avec ces noms et comme les partitions agissent comme des colonnes  lors de l'évaluation d'une requêtes avec Amazon Athena, les noms de ces partitions ont été adaptés.

Par exemple, les traceroutes qui se trouvent dans  le fichier

 \textit{2016-10-01 00:00:00\_msmId5001.json.gz} sont analysés par toute requête Athena  
impliquant les partitions d'une des manières suivantes : 
(source = api) ou (source = api et af\_ = 4) ou (source = api et af\_ = 4 et type = builtin) ou (source = api et af\_ = 4 et type = builtin et msm = 5001) ou 
(source = api et af\_ = 4 et type = builtin et msm = 5001 et year = 2016) ou (source = api et af\_ = 4 et type = builtin et msm = 5001 et year = 2016 et month = 10) ou
 (source = api et af\_ = 4 et type = builtin et msm = 5001 et year = 2016 et month = 10 et day = 1).

\paragraph{Interrogation des données Avec Amazon Athena}~

Une fois les fichiers de données  synchronisés vers le compartiment AWS S3 et le schéma  de données  créé, on passe à l'interrogation de données en utilisant les requêtes SQL basées sur Presto.  
Nous donnons un exemple d'une requête Athena dans la section \ref{sql-athena-request} de l'annexe \ref{athena-appendix}.

\paragraph{Intégration d'Amazon Athena dans l'outil de détection}~ \label{integration-aws-possibilite-une}
 

Pour intégrer Amazon Athena dans l'outil de détection \cite{InternetHealthReport}, on distingue deux possibilités. La première possibilité n'utilise Athena que pour récupérer  les traceroutes stockés dans Amazon S3 vers  la machine locale. Ensuite, cette dernière poursuit les traitements  décrits dans la phase I et II. Dans ce cas, nous ne profitons pas  des performances d'Amazon Athena vu que les traitements complexes sont effectués dans la machine locale.
%vérifiés en terme de validité; c'est l'objectif des étapes $1$ et $2$ dans le processus de la création de l'évolution des RTTs différentiels des liens (voir la section \ref{steps-rtt-analysis}).  
 %Les traitements qui suivent (étapes  à partir de $3$) sont effectués dans la machine locale. Dans ce cas, l'utilisation des technologies  Big Data est limité qu'au niveau stockage de données massives. 

Tandis que  la deuxième possibilité vise la maximisation des traitements des deux phases I et II au sein de l'infrastructure  d'Athena. De ce fait, la machine locale n'a qu'à recevoir les derniers résultats de la détection, voire les résultats finaux. Pour cette deuxième possibilité, les données doivent être manipulées de sorte à maximiser,  	au niveau d'Amazon Athena,  les traitements relatifs à toutes les étapes des deux phases I et II. 

Pour la deuxième possibilité, le défi est de trouver la requête ou bien l'ensemble de requêtes SQL à exécuter sur Athena en vue d'avoir l'évolution du RTT différentiel des liens. 
Vu la complexité des  étapes des phases I et II, on ne peut pas trouver une seule requête SQL assurant toutes ces étapes à la fois. Supposons qu'il existe une requête SQL capable de trouver les liens possibles avec leurs RTTs différentiels : à l'étape $ 4 $ dans \ref{steps-rtt-analysis}, on construit la distribution des RTTs différentiels pour tout lien $l$ identifié dans les traceroutes de la période $d_k$. Cette distribution est mise à jour à chaque fois $l$ est identifié dans un des traceroutes  de la période $d_k$. 

Soient  $T_k$ = \{$t_{k, j}$\}  l'ensemble de traceroutes effectués durant $d_k$, avec $j \in [1, R_k]$ et $R_k$ est le nombre de traceroutes effectués durant $d_k$. Nous décrivons le parcours des traceroutes d'une période $d_k$ brièvement dans le pseudo-code \ref{alo-inference-link}. Nous n'avons pas donné  tous les détails, car l'objectif est d'évaluer la convenance d'Athena au traitement souhaité.
\begin{algorithm}[H]
\begin{algorithmic}[1]
	 \ForAll{ $t_{k, j}$ $\in$ $T_k$} \
	  \State $links$ $\leftarrow$ getLinksFromTraceroute($t_{k, j}$)
	  	 \ForAll{$l$ $\in$ $links$}
	  	 		\State updateLinkRttDistribution($l$) \label{update-link}
	  	 \EndFor
	 \EndFor
\end{algorithmic}
\caption{Une partie de l'étape $4$ du processus de la détection des anomalies des délais }
\label{alo-inference-link}
\end{algorithm}

Avec : 
\begin{itemize}
	\item \textit{getLinksFromTraceroute($t_{k, j}$)} énumère tous les liens possibles dans le traceroute $t_{k, j}$.

    \item \textit{updateLinkRttDistribution($l$)} ajoute le RTT différentiel calculé du lien $l$ à la distribution des RTTs différentiels courante de ce lien pour la période $d_k$.
\end{itemize}


Le service Athena est conçu pour la lecture de données, toute mise à jour de données n'est pas possible avec ce service. C'est pourquoi la distribution des RTTs différentiels de chaque  lien identifié doit être sauvegardée dans un endroit accessible en lecture et en écriture, par exemple dans un compartiment AWS S3. Que ce soit un fichier reprenant la distribution des RTTs différentiels  par un seul lien ou bien un fichier pour tous les liens,   à la ligne  \ref{update-link} du pseudo-code \ref{alo-inference-link}, un fichier doit être lu et mise à jour avec de nouvelle valeur. Pour une période $d_k$ d'une heure, le nombre de traceroutes est de l'ordre de milliers. Chaque traceroute $t_{k,j}$ peut inclure $L_{k,j}$ liens. Dans ce cas, le nombre total, d'une période $d_k$, de mise à jour de la distribution des RTTs différentiels est    $ \sum_{m=1}^{R_k}  L_{k,m}$. $ L_{k,j} $ dépend du nombre de saut du  traceroute $t_{k,j}$.

 %de l'ordre $R_k$\texttimes$L$ de fois.  Cette estimation est à titre indicatif, de plus elle ne concerne que l'étape $4$, le nombre de lectures et/ou d'écritures dépend des requêtes SQL créées pour les autres étapes. 

En plus du nombre de lectures et d'écritures, relatives à la phase I, que nous venons de décrire, à la phase II, la détection des anomalies s'effectue en  comparant les intervalles de confiances : un intervalle de confiance courant du lien avec celui de référence. Cette comparaison révèle deux contraintes. La première contrainte concerne  la fonction permettant de calculer les deux bornes de l'intervalle de confiance de Wilson ne fait pas partie des fonctions disponibles sur Amazon Athena. D'autre part, Amazon Athena ne permet pas la création des fonctions personnalisées pour répondre à des besoins non couverts par Amazon Athena. La deuxième contrainte concerne la mise à jour de l'intervalle de confiance de référence qui doit être faite à chaque nouvelle période.


\paragraph{Evaluation des critères pour Amazon S3 et Amazon Athena  }~

Afin d'utiliser le service Amazon Athena à moindre coût, il est conseillé d'utiliser le partitionnement, car moins de frais sont appliqués. Si un partitionnement particulier est adopté, la création du schéma de données est basé sur ce partitionnement ainsi que les requêtes SQL destinés à l'interrogation de la table de données.

En ce qui concerne l'évolutivité d'une application basée sur ces deux services d'Amazon, on note que toute mise à jour de la structure de données des objets traceroutes peut affecter l'entièreté de la configuration initiale. A savoir, l'organisation des fichiers de données via le partitionnement, le schéma de données et les requêtes SQL.

 Quant à la flexibilité du schéma de données, le service  Amazon Athena est tolérant au données manquantes. Etant donné que la structure d'un objet traceroute dépend de la version du firmware de la sonde, nous avons créé trois schémas de tables. La première table  modélise tout objet traceroute de  version $5$, la deuxième modélise tout objet traceroute de version $6$ et enfin la troisième table modélise ceux ayant la version $7$. En expérimentant différentes requêtes, nous avons conclu  que Amazon Athena a pu récupérer les données de la version récente ($7$) via le schéma de la version $5$ malgré que la version $7$ a plus d'attributs par rapport à la version $5$.
 
%Avec une autre technologie qui travaille en mémoire, les résultats sont données plus rapidement. 

\paragraph{Performances des services Amazon S3 et Athena dans l'analyse des délais }~ \label{aws-perforsm}

Nous avons utilisé Amazon Athena et Amazon S3 pour analyser les traceroutes et détecter les anomalies des délais. Nous précisons que nous avons évalué la première possibilité décrite dans la section \ref{integration-aws-possibilite-une}. Nous avons bénéficié de la  possibilité  de lancer des requêtes destinées à Amazon Athena à travers l'API REST, précisément en Python. Et comme l'implémentation proposée par les auteurs du travail de référence est écrite en Python, nous avons adapté cette implémentation de sorte  de récupérer les traceroutes depuis   depuis Amazon S3 via Amazon Athena au   lieu de le faire depuis la base de données locale MongoDB.  

Etant donné que nous avons utilisé le partitionnement de données, une analyse des délais nécessite d'autres paramètres à ajuster en plus de ceux relatifs à la détection. Ce sont les paramètres permettant de sélectionner les traceroutes présents sur Amazon S3. Du fait que le partitionnement de données (voir une partie de l'arborescence dans la Figure 	\ref{fig:partitionnement-athena}) est réalisé sur base du type de traceroute (\textit{builtin} ou \textit{anchor}) et de l'identifiant de la mesure ($ 5004 $, $6001$, etc) qui a enregistré un traceroute, nous devons personnaliser  la requête visant la   récupération des traceroutes  depuis Amazon S3 pour qu'elle prennent en compte aussi les partitions.

Le Tableau \ref{tab:athena-data} contient les temps d'exécution suivant la taille de l'ensemble de données donné en entrée de la détection.

\begin{table}[H]
	\centering
	\captionsetup{justification=centering}
\begin{tabular}{c c c}
	\textbf{Période} & \textbf{Taille(GB)} & \textbf{Temps (secondes) } \\ 	\hline 
$ 07/02/18 - 07/02/18 $	&$ 1 $&	$ 1898.31 $ \\ 	\hline 
$ 07/02/2018 - 08/02/2018 $	&$ 2 $&	$ 3533.6562171 $ \\ 	\hline 
$ 07/02/2018 - 09/02/2018 $&	$ 3 $&	$ 5284.91494989  $ \\ 	\hline 
$ 07/02/2018 - 10/02/2018 $	&$ 4 $&$ 	7228.88 $  \\ 	\hline 
$ 07/02/2018 - 11/02/2018 $	&$ 5 $& $ 8984.873281	 $ \\ 	\hline 
\end{tabular} 
\caption{Les temps d'exécution par taille de l'ensemble de données (Amazon Athena et Amazon S3)}
\label{tab:athena-data}
\end{table}

Nous distinguons trois phases dans cette approche (approche 1). Premièrement,  les données sont récupérées depuis Amazon S3. Plusieurs facteurs affectent cette étapes, par exemple,  les conditions du réseau, les ressources allouées par Amazon pour répondre à chaque requête  Athena à destination des données disponibles sur Amazon S3, l'optimalité de la requête SQL, etc.  En deuxième lieu, les résultats de la requête doivent être désérialisés pour pouvoir les utiliser localement. Enfin, sur base des données récupérées, la détection des anomalies peut être déclenchée.
%
%
%Nous évaluons les temps d'exécution de plusieurs  analyses de délais lancées en variant la taille de données. Rappelons qu'il s'agit de la première possibilité: Amazon S3 pour le stockage de traceroutes et Amazon Athena pour récupérer les traceroutes valides, le reste de traitements sont effectués au sein de la machine locale.  Le Tableau	\ref{tab:awstiming-timing} reprend plus de détails. 
%\begin{table}[H]
%	%\begin{threeparttable}
%	
%	\captionsetup{justification=centering}
%	\begin{tabular}{ccccc}
%		\textbf{Début - fin} &\textbf{Durée (jours)}  & \textbf{Taille}  & \textbf{Nb traceroutes} & \textbf{Temps (secondes)} \\ \hline
%		
%		07/02/2018             &1 &1 GB&& 3870\\ \hline
%		07/02/2018 - 08/02/2018&2 &1 GB&& 2942\\ \hline
%		07/02/2018 - 09/02/2018&3 & 1 GB&& 2991\\ \hline
%		07/02/2018 - 10/02/2018&4 & 3 GB&& 20955\\ \hline
%		07/02/2018 - 11/02/2018&5& && \\ \hline
%		07/02/2018 - 12/02/2018&6& && \\ \hline
%		07/02/2018 - 13/02/2018&7& && \\ \hline
%		07/02/2018 - 14/02/2018&8& && \\ \hline
%		07/02/2018 - 15/02/2018&9& && \\ \hline
%		07/02/2018 - 16/02/2018&10& && \\ \hline
%		07/02/2018 - 17/02/2018&11& && \\ \hline
%		07/02/2018 - 18/02/2018&12& && \\ \hline
%		07/02/2018 - 19/02/2018&13& && \\ \hline
%		07/02/2018 - 20/02/2018&14& && \\ \hline
%	\end{tabular}
%	\caption{La moyenne des temps d'exécution d'analyse de traceroutes en fonction de la taille de données avec Amazon S3 et Amazon Athena }
%	\label{tab:awstiming-timing}
%\end{table}
La Figure \ref{fig:temps-avec-aws} présente un seul essai pour chacune des tailles utilisées auparavant avec MongoDB.  Le temps de chaque essai comprend l'étape de la récupération des traceroutes depuis Amazon S3, le temps de préparation des traceroute et enfin le temps de la détection des anomalies. Autrement dit, le temps nécessaire à la réalisation des phases I, II et celui nécessaire pour sauvegarder les résultats par lien.

\begin{figure}[H]
	\centering
	\captionsetup{justification=centering}
	\includegraphics[width=0.7\linewidth]{illustrations/temps-avec-aws_1}
	\caption{Les temps d'exécution de la détection des anomalies en fonction de la taille de données (Amazon S3 et Amazon Athena)}
	\label{fig:temps-avec-aws}
\end{figure}


\section{Application 4 : Spark Apache avec Scala}
Nous avons implémenté l'outil de détection avec le framework Spark et l'API Scala. Les détails de l'implémentation sont donnés dans le chapitre \ref{application:spark}.  Nous avons évalué le temps d'exécution de l'outil de détection en analysant différents échantillons de traceroutes  en mode local et en mode cluster. Pour le mode local, nous avons lancer l'application Spark sur la machine ayant les caractéristiques  reprises dans la section \ref{machine-openvz-caracteritics}. Pour le mode cluster, nous avons utilisé le cluster EMR.

Les fonctionnalités de Spark sont accessibles avec les  APIs en Scala, Java et Python. Nous avons choisi l'utilisation de l'API en Scala parce que Scala est le langage natif de Spark. De plus, Scala est interopérable avec Java. 

\subsection{Mode Standalone}

A la base Spark est conçu pour être utilisé dans un cluster de machines, sur  lequel l'analyse est distribuée. Toutefois, Spark peut être utilisé en mode local. Dans ce mode, on trouve le \textit{driver} et un seul \textit{executor}. Ce dernier est "lié" au processus initié par le \textit{driver}. 
Dans une application Spark, la taille de la mémoire allouée pour le \textit{driver} et les \textit{executors} est  définie par défaut. D'après la documentation officielle de Spark\footnote{Source : \url{https://spark.apache.org/docs/latest/configuration.html}, consultée le $29/12/2018$.}, Spark réserve $ 1 $ GB pour le $ driver $ et $ 1 $ GB pour chaque \textit{executor}. 




\paragraph{Performances d'Apache Spark dans l'analyse des délais }~


Nous avons évalué le temps d'exécution de l'implémentation de l'outil de détection en utilisant Spark en variant le nombre de traceroutes à analyser. Nous avons aussi varié certains paramètres relatifs à la soumission de l'application au Spark. Nous avons varié la taille de la mémoire allouée au driver afin de choisir celle la plus adaptée. Ensuite nous avons mesuré le temps d'exécution dans le cas de local, local[K] et local[*].
\newpage
\subparagraph{Variant la mémoire allouée au driver}~

En mode local (--master local), le \textit{driver} et le \textit{worker} sont liés au même processus.  Nous avons mesuré le temps d'exécution de l'application Spark en varaint la taille mémoire allouée au \textit{driver} via le paramètre \textit{driver-memory}. Par défaut, la mémoire allouée au driver est de 1 Go. Avec cette valeur, il n'est pas possible  d'analyser un ensemble de traceroutes qui fait $ 1,028,343,572 $ octets. Afin de voir l'effet de la mémoire allouée au \textit{driver}, en mode d'exécution local, nous avons utilisé deux échantillons de traceroutes. La première comprend les traceroutes capturés pendant le $ 07/02/2018 $ dans le cadre de la mesure ayant l'identifiant $ 5004 $, ce qui fait $ 1,028,343,572 $ octets (\textit{data\_1go}). Pour le deuxième échantillon, il reprend les traceroutes effectués  entre le $ 07/02/2018 $ et le $ 08/02/2018 $ effectués aussi dans le cadre de la mesure ayant l'identifiant $ 5004 $, ce qui fait $ 2,055,167,238 $ (\textit{ data\_2go}) octets.

La Figure 	\ref{fig:variantdrivermemory} présente les résultats obtenus. L'axe des abscisses indique la quantité de mémoire allouée au \textit{driver} et l'axe des ordonnées représente le temps d'exécution de détection. Pour les mêmes quantités de mémoire allouée au \textit{driver}, nous mesurons le temps d'exécution une fois pour  \textit{data\_1g}  et une autre fois pour  \textit{data\_2g}.
Pour les valeurs nulles relatives au temps d'exécution, l'exécution de la l'application a échoué. La raison de l'échec revient au manque de mémoire (message d'erreur est OutOfMemoryError: Java heap space).
\begin{figure}[H]
	\centering
	\captionsetup{justification=centering}
	\includegraphics[width=1\linewidth]{illustrations/variant_driver_memory}
	\caption{Mesure des temps d'exécution de l'application Spark selon différentes tailles de mémoire allouées au \textit{driver} et pour deux ensembles de données différentes}
	\label{fig:variantdrivermemory}
\end{figure}

D'après la Figure 	\ref{fig:variantdrivermemory}, nous remarquons qu'à partir d'une taille mémoire allouée au driver, le temps écoulé durant l'exécution de l'application Spark est relativement stable. De plus, nous constatons qu'il faut prévoir une taille mémoire minimal pour assurer l'exécution de l'application Spark. Les tailles mémoire supérieures à cette valeur minimale affectent faiblement le temps d'exécution.  Cette valeur minimale dépend fortement de la quantité de données à analyser. Enfin, malgré que la machine sur laquelle nous lançons l'application Spark ne dispose que de $ 32 $ Go de RAM, le fait d'allouer au driver $ 35 $ Go, $ 40$ Go, $ 45 $ Go n'a pas généré une erreur lors de l'exécution.

\newpage
\subparagraph{Variant le mode de cluster local : local[*] vs local}
\newpage
\subparagraph{Variant la taille de données}

 %L'application Spark lit d'abord les traceroutes présents dans les fichiers donnés en entrée, ensuite, pour chaque période $d_k$, il faut parcourir tous les  traceroutes afin de trouver les traceroutes effectués durant $d_k$, appelée approche 1.  
 
 Le Tableau \ref{tab:spark-timing} contient les résultats de $5$ analyses. Les trois premiers tests correspondent à l'analyse de $1$ GB de traceroutes. Dans le quatrième test, nous avons analyser les données des trois premiers tests. Durant le cinquième test, l'analyse a échoué. 

\begin{table}[H]
%\begin{threeparttable}
\centering
\captionsetup{justification=centering}
	\begin{tabular}{c cccc}
		
		\textbf{Test}&\textbf{Période} &\textbf{Durée}  & \textbf{Taille}   & \textbf{Temps (secondes)} \\ \hline
		
		$ 1 $&$ 07/02/2018 $&$ 1 $ jour&$ 1 $ GB& $ 2498 $\\ \hline
		$ 2 $&$ 08/02/2018 $&$ 1 $ jour&$ 1 $ GB& $ 2942 $\\ \hline
	$ 	3 $&$ 09/02/2018 $&$ 1  $jour& $ 1 $ GB& $ 2991 $\\ \hline
		$ 4 $&$ 07/02/2018 - 09/02/2018 $&$ 3 $ jour& $ 3 $ GB& $ 20955 $\\ \hline
		$ 5 $&$ 07/02/2018 - 10/02/2018 $&$ 4 $ jour& $ 4 $ GB & échec \\ \hline
	\end{tabular}
	\caption{Les temps d'exécution des analyses lancées sur des traceroutes (approche $1$)}
	\label{tab:spark-timing}
\end{table}

Le temps total nécessaire pour analyser les trois échantillons correspondants aux  $ 07 $, $ 08 $ et $ 09 $ février $ 2018 $, chacun à part,  est de $ 8431 $. Tandis que le temps nécessaire à l'analyse des traceroutes correspondants au 3 échantillons en une seule fois est de $20955$, ce que correspond à $ 2.4 $ fois du temps de plus.
En ce qui concerne l'analyse des traceroutes correspondants aux $ 4 $ jours,  l'analyse a échouée. 

En revenant à l'implémentation, pour chaque période $d_k$, il faut consulter tous les traceroutes et ne récupérer que ceux effectués durant $d_k$. Par exemple, 3 GB de données doivent être consultées $ 3 * 24 $ fois. 


Nous avons modifié l'implémentation afin de réduire le temps de l'analyse. Au lieu de chercher les traceroutes à chaque période $d_k$,  on attribue chaque traceroute à une des périodes de l'analyse si le timestamp correspond; le tiemestamp du traceroute $t_{j}$ appartient à la période $d_k$, appelée approche 2.


Soient $d_k$ une période dans l'ensemble de périodes $D$  entre $ start $ et $ end $ et $t_{j}$ est un traceroute dans l'ensemble de traceroutes $T$ donnés en entrée, $j$ $\in$ $[1, n]$ où $n$ est le nombre total de traceroutes à analyser. Nous illustrons la différence entre l'approche $ 1 $ et $ 2 $ avec les deux pseudo-code \ref{algo:grouper-traceroutes_1} et 	\ref{algo:grouper-traceroutes_2}. Avec :

\begin{itemize}
	\item $ findAllPeriods(start, end, timeWindow)  $ : crée la liste des périodes entre la date  \textit{start} et la date \textit{end}, chaque période a la durée \textit{timeWindow};
	
	\item  $ loadTraceroutesFromInputFiles(dataPath)  $: lit les traceroutes qui se trouvent dans le répertoire  \textit{dataPath};
	
	\item   $ checkTracerouteInPeriod (d_k, t_{j}$ : vérifie si le traceroute $ t_{j}  $ a été effectué durant la période $d_k$.
\end{itemize}


\begin{algorithm}[H]
	\hspace*{\algorithmicindent} \textbf{Inputs :} \textit{dataPath} chemin vers les fichiers de données, \textit{start}, \textit{end}, \textit{timeWindow}\\
	\hspace*{\algorithmicindent} \textbf{Outputs :} regroupement des traceroutes par périodes
	\begin{algorithmic}[1]
	\State $D$ $\leftarrow$ $ findAllPeriods(start, end, timeWindow) $
	\ForAll{ $d_k$ $\in$ $D$} \
	\State $T$ $\leftarrow$ \textit{$ loadTraceroutesFromInputFiles(dataPath) $}
	\ForAll{$t_{j}$ $\in$ $T$}
	\State $ checkTracerouteInPeriod $($d_k$, $t_{j}$) 
	\EndFor
	\EndFor
\end{algorithmic}
	\caption{Regroupement des traceroutes par période $d_k$ (étape 1) : approche $ 1 $}
	\label{algo:grouper-traceroutes_1}
\end{algorithm}

\begin{algorithm}[H]
	\hspace*{\algorithmicindent} \textbf{Inputs :} \textit{dataPath} chemin vers les fichiers de données, \textit{start}, \textit{end}, \textit{timeWindow}\\
	\hspace*{\algorithmicindent} \textbf{Outputs :} regroupement des traceroutes par périodes
	\begin{algorithmic}[1]
		\State $D$ $\leftarrow$ $ findAllPeriods(start, end, timeWindow) $
		\State $T$ $\leftarrow$ $ loadTraceroutesFromInputFiles( dataPath) $
		\ForAll{$t_{j}$ $\in$ $T$}
		\ForAll{ $d_k$ $\in$ $D$} \
		\State $ checkTracerouteInPeriod $($d_k$, $t_{j}$) 
		\EndFor
		\EndFor
	\end{algorithmic}
	\caption{  Regroupement des traceroutes par période $d_k$ (étape $ 1 $) : approche $ 2 $}
	\label{algo:grouper-traceroutes_2}
\end{algorithm}

 Le Tableau 	\ref{tab:spark-timing-reajustedcode} reprend des indications sur les résultats obtenus, avec \textit{driver-memory} configuré sur  $ 30 $ GB.  Ce sont les temps d'exécution obtenus lors de deux tests pour les tailles de données : $1$ GB, $2$ GB, $3$ GB, $4$ GB et $5$ GB, ce tableau reprend aussi les périodes correspondantes à ces données et le nombre de traceroutes présents dans chaque échantillon.

%\begin{table}[H]
%	\centering
%	\captionsetup{justification=centering}
%	%\begin{threeparttable}
%	\begin{tabular}{cccccc}
%		\textbf{Test} & \textbf{Début - fin} &\textbf{Durée}  & \textbf{Taille}  & \textbf{Nb traceroutes} & \textbf{Temps (secondes)} \\ \hline
%		
%	1&	07/02/2018&1 jour&1 GB&& 752.03\\ \hline
%	2&	08/02/2018&1 jour&1 GB&&  \\ \hline
%	3&	09/02/2018&1 jour& 1 GB&&  \\ \hline
%	\end{tabular}
%	\caption{}
%	\label{tab:spark-timing-reajustedcode}
%\end{table}

\begin{table}[H]
	\centering
	\captionsetup{justification=centering}
	%\begin{threeparttable}
	\resizebox{1\textwidth}{!}{
	\begin{tabular}{ccccc}
\textbf{Période}&\textbf{Taille (GB) }&\textbf{Nb de traceroutes}&\textbf{Essai 1 (secondes)}&\textbf{Essai 2 (secondes)}\\  \hline
07/02/18&1&494158&838&1137 \\  \hline
07/02/2018 à 08/02/2018&2&987995&936&1034 \\  \hline
07/02/2018 à 09/02/2018&3&1481942&1449&1930 \\  \hline
07/02/2018 à 10/02/2018&4&1976010&2139&1714 \\  \hline
07/02/2018 à 11/02/2018&5&2470027&3116&3030 \\  \hline
	\end{tabular}
}
	\caption{Les temps d'exécution lors de la détection des anomalies en utilisant Spark (approche $2$)}
	\label{tab:spark-timing-reajustedcode}
\end{table}

%    \begin{tablenotes}
%	\small
%	\item This is where authors provide additional information about
%	the data, including whatever notes are needed.
%\end{tablenotes}
%\end{threeparttable}

La Figure 	\ref{fig:moustachespark} permet de présenter les temps d'exécution pendant plusieurs essais et pour plusieurs ensembles de données.
\begin{figure}[H]
	\centering
	\includegraphics[width=0.7\linewidth]{illustrations/moustacheSpark}
	\caption{}
	\label{fig:moustachespark}
\end{figure}

\subsection{EMR Cluster}

Nous avons choisi un cluster EMR pour  lancer la détection des anomalies. En créant un cluster, nous pouvons choisir le nombre de \textit{workers}. De plus, nous pouvons passer des arguments à notre application comme dans le mode local. Nous avons adapté la version destinée au cluster EMR pour qu'elle prennent en compte les fichiers contenant les traceroutes stockés cette fois dans Amazon S3 au lieu de chercher ces fichiers dans l'arborescence de la machine locale. 

\section{Récapitulatif}

Nous avons discuté l'évaluation des technologies Big Data en terme du temps d'exécution. L'objectif de l'analyse des traceroutes est de détecter les changements anormaux dans les délais des liens à travers les changements de leurs RTTs différentiels. Nous avons évalué les temps d'exécution en fonction de la quantité de données, alors qu'on peut aussi évaluer les temps d'exécution en fonction du nombre de traceroutes. Du fait que l'outil de détection se base sur la liste des sauts par traceroute, et sachant que le nombre de saut varie d'un traceroute à un autre, une évaluation de la détection en fonction du nombre de traceroute peut être moins représentable. 
%\paragraph{Le temps d'exécution en fonction du nombre de traceroutes dans l'ensemble de données}~
On ne peut pas traiter un ensemble de $n$ traceroutes de la même manière étant donné que la détection des anomalies se base sur le nombre de sauts, les traceroutes n'ont pas tous un nombre de sauts équitable. C'est que illustre 
la Figure 	\ref{fig:traceroutes-par-sauts}\footnote{Les résultats concernent les traceroutes de la journée $07/02/2018$.}. Ainsi, le temps nécessaire à évaluer un traceroute ayant, par exemple, $10$ saut est différent de celui nécessaire à l'évaluation d'un traceroute ayant $2$ sauts. 
\begin{figure}[H]
	\centering
	\includegraphics[width=0.8\linewidth]{illustrations/traceroutes-par-sauts}
	\caption{Regroupement des traceroutes d'une journée  par leurs nombre de sauts}
	\label{fig:traceroutes-par-sauts}
\end{figure}

\paragraph{Quelques chiffres sur les liens et les anomalies identifiés }~
Notre évaluation des technologies Big Data a concerné, en premier lieu, les temps d'exécution. Nous donnons à travers la Figure 	\ref{fig:tous-vs-anomalies-liens} une idée sur le nombre de liens identifiés ainsi que les liens, parmi ceux identifiés, ceux ayant enregistré un moins une anomalie. La Figure 	\ref{fig:tous-vs-anomalies-liens} présente ces liens par taille de données. Nous précisons que les traceroutes présents dans l'échantillon de $3$ GB sont aussi repris dans l'échantillon de $4$ GB. 
\begin{figure}[H]
	\centering
		\captionsetup{justification=centering}
	\includegraphics[width=0.7\linewidth]{illustrations/tous-vs-anomalies-liens}
	\caption{Les liens et les anomalies identifiés par ensemble de traceroutes}
	\label{fig:tous-vs-anomalies-liens}
\end{figure}



\paragraph{MongoDB vs Amazon Athena et Amazon S3}~

L'évolution du temps d'exécution écoulé en utilisant Amazon Athena est différents de celui avec de MongoDB. Cela peut être dû aux facteurs discutés dans la section \ref{aws-perforsm}. Nous précisons que l'implémentation de l'outil de détection nécessite de chercher les traceroutes à chaque période $d_k$. Nous avons considéré les résultats d'un seul identifiant (5004), le nombre de traceroutes récupérés à chaque $d_k$ est plus de $20.000$ (par requête avec un timeWindow égale à $3600$ secondes). 


\begin{figure}[H]
	\centering
	\captionsetup{justification=centering}
	\includegraphics[width=0.7\linewidth]{illustrations/mondb-vs-athena}
	\caption{Comparaison de l'évolution des temps d'exécution entre MongoDB et Amazon Athena avec Amazon S3}
	\label{fig:mondb-vs-athena}
\end{figure}

\paragraph{Apache Spark vs MongoDB}~

La Figure \ref{fig:mongodbvsspark} illustre la moyenne des temps d'exécution pour MongoDB et Spark. Ces résultats concernent les mêmes ensembles de données.

\begin{figure}[H]
	\centering
	\includegraphics[width=0.7\linewidth]{testPerformances/mongodbvsspark}
	\caption{La moyenne des temps d'exécution pour MongoDB et Spark}
	\label{fig:mongodbvsspark}
\end{figure}

\paragraph{MongoDB vs Amazon Athena et Amazon S3 vs Apache Spark }~
A présent, nous comparons les temps d'exécution pour les trois implémentations : MongoDB, AWS, et Apache Spark comme illustré dans la Figure 	\ref{fig:mondb-vs-spark-vs-athena}. 

\begin{figure}[H]
	\centering
	\includegraphics[width=0.7\linewidth]{illustrations/mondb-vs-spark-vs-athena}
	\caption{Comparaison des temps d'exécution entre MongoDB, AWS et Apache Spark}
	\label{fig:mondb-vs-spark-vs-athena}
\end{figure}


\section{Conclusion}
%A travers ce travail, nous souhaitions évaluer des technologies Big Data pour l'analyse de données en provenance du dépôt RIPE Atlas. 

Nous avons évalué la convenance des technologies Big Data  à l'analyse souhaitée des traceroutes. Dans un premier temps, nous avons utilisé deux technologies conçues pour le stockage de données à grande échelle : MongoDB et DynamoDB, ensuite, nous avons expérimenté trois services d'Amazon, le premier pour le stockage de données, le deuxième  pour la découverte de données et le troisième service  est conçu pour l'interrogation  de données. Enfin, nous avons utilisé un framework qui gère le traitement distribué de données dans un cluster de machines.

%le choix de données
Nous avons à disposition une variété de données,   à titre indicatif, nous pouvons récupérer des années  de mesures effectuées par toutes les sondes Atlas. Notre premier objectif  est d'évaluer la convenance des technologies choisies à l'analyse de données souhaitée. Précisément, pouvoir utiliser l'outil de détection avec ces technologies. C'est pourquoi nous n'avons pas défini des critères pour choisir l'ensemble de données de test. 
%evaluation des technologies
Nous avons évalué  les performances des technologies en terme de temps écoulé tout au long de l'analyse de différents échantillons en fonction de la taille de données. On peut conclure que les performances de MongoDB en terme de temps d'exécution, version locale, dépendent des ressources de la machine dans laquelle MongoDB tourne. Sachant qu'il existe une version disponible sur le cloud, les performances de celle-ci dépendent, à priori,  des options choisies.
Toutefois, les performances des trois services d'Amazon dépendent du nombre de requêtes destinées vers Amazon Athena, la complexité des traitements assurés par la requête Athena et aussi du partitionnement adopté. 
Tandis que Apache Spark, version local,  dépendent de la taille de  mémoire allouée pour le \textit{driver} ainsi que d'autres configurations comme le \textit{timeout} du \textit{driver} et  autres. 






%\chapter*{Conclusion}
%\addcontentsline{toc}{chapter}{Conclusion}
%\markboth{\textsl{Conclusion}}{\textsl{Conclusion}}}

%Quelques lignes pour finir...

%\appendix

%\chapter{Présentation du traceroute}

\paragraph{Les attributs } \label{annexe:traceroute-attributes}

\begin{figure}[H]
	\centering
	\includegraphics[width=0.7\linewidth]{illustrations/traceroute_attributes}
	\caption{Les attributs possibles dans le résultat d'une requête traceroute}
	\label{fig:traceroute_attributes}
\end{figure}


% Si vous utilisez (conseillé) BibTeX pour votre bibliographie :
\bibliographystyle{acm}
\bibliography{references}% si le fichier BibTeX est memoire.bib

\end{document}
%%% Local Variables: 
%%% mode: latex
%%% TeX-master: t
%%% TeX-PDF-mode: t
%%% End: 

