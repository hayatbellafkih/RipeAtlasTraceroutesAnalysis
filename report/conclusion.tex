\chapter*{Conclusion}

%Rappel de la problématique.

L'objectif du présent travail est d'évaluer un ensemble de technologies Big Data sur des données réelles. 
Etant donné que l'évaluation des technologies Big Data peut prendre plusieurs formes, nous avons cadré cette évaluation en précisant les directives de l'évaluation.


%Résultats de recherche et réponses aux questions de recherche.

L'évaluation de la convenance d'une technologie Big Data nécessite de disposer de données assez suffusantes pour élaborer tout les cas d'utilisation. Nous avons choisi des données ayant un sens, en provenance de sources sûres et réelles, traitant le domaine des réseaux informatiques. Il est possible de générer des données avec la quantité et la structure souhaitées, toutefois, nous avons choisi de traiter des données réelles et a de sens. D'où notre intérêt  à maîtriser le projet RIPE Atlas depuis lequel nous avons récupéré les données utilisé dans l'analyse.


Nous premier objectif est de montrer les limites des outils traditionnels en ce qui concerne les données à grande échelle, et la capacité des technologies Big Data à répondre aux besoins. Pour ce faire, nous avons choisi de réutiliser un travail dans lequel un outil de détection d'anomalies a été conçu, qui est basé sur les données en provenance du RIPE Atlas.  




%Ouverture.
En deuxième lieu, nous avions un deuxième objectif qui concerne l'outil de détection et son fonctionnement. Les approches utilisés dans la conception de cet outil se base sur des fondements statistiques. De plus cet outil est personnalisable. Nous aurions aimé évalué les performances de cet outil en pratique, sur des ensembles de données plus récentes. Comme les résultats de la détection dépendent des paramètres de l'outil, serait intéressant d'évaluer ces paramètres,  
























%I) Apports
%II) Limites
%III) Perspectives




%3. Une ouverture

%Les données en provenance du RIPE Atlas analysées dans ce travail peuvent passer à l'échelle 
%dés qu'on considère plusieurs heures de traceroutes en provenance de plusieurs sondes. A travers ce travail, nous avons évalué des technologies Big Data pour analyser des échantillons de données.  


%le choix des technologies

%1. La problématique

%2. Les réponses à la problématique

%A travers ce travail, nous souhaitions évaluer des technologies Big Data pour l'analyse de données en provenance du dépôt RIPE Atlas. Le choix d'une technologie Big Data dépend de plusieurs facteurs. Dans un premier temps, nous avons utilisé deux technologies conçues pour le stockage de données à grande échelle : MongoDB et DynamoDB. Ensuite, nous avons expérimenté trois services d'Amazon, le premier pour le stockage de données, le deuxième  pour la découverte de données et le troisième service  est conçu pour l'interrogation  de données. Enfin, nous avons utilisé un framework qui gère le traitement distribué de données dans un cluster de machines.
%
%%le choix de données
%Nous avons à disposition une variété de données, à titre indicatif, des années  de mesures effectuées par les sondes Atlas. Notre premier objectif de l'évaluation est d'évaluer les  performances des choix technique. C'est pourquoi nous n'avons pas défini des critères pour choisir l'ensemble de données. 
%%evaluation des technologies
%Nous avons évalué  les performances des technologies en terme de temps écoulé tout au long de l'analyse de différents échantillons.
% Les performances de MongoDB dépendent de ... 
% Toutefois, les performances des trois services d'Amazon dépendent de 
% Tandis que Apache Spark dépend de 
 
 

%perspectives

%Si nous aurions plus de temps, nous aurions aimé évalué les performances de l'outil de détection conçu par Fontugne\cite{DBLP:journals/corr/FontugneAPB16} en terme de précision dans la détection d'anomalies dans les délais des liens. 







