\chapter*{Conclusion}

Les données en provenance du RIPE Atlas analysées dans ce travail passent à l'échelle dés qu'on considère plusieurs heures d'échantillons. A travers ce travail, nous avons évalué des technologies Big Data pour analyser des échantillons de données.  


%le choix des technologies

Aujourd'hui, il existe de nombreuses   technologies Big Data. Le choix d'une technologie ou une autre dépend de plusieurs facteurs. Dans un premier temps, nous avons utilisé deux technologies conçues pour le stockage de données à grande échelle : MongoDB et DynamoDB. Ensuite, nous avons expérimenté trois services d'Amazon, le premier pour le stockage de données, le deuxième  pour la découverte de données et le troisième service permet est consacré pour l'interrogation  de données. Enfin, nous avons utilisé un framework qui gère le traitement distribué de données dans un cluster de machines.

%le choix de données
Nous avons à disposition une vérité de données, à titre indicatif, des années  de mesures effectuées par les sondes Atlas. Notre premier objectif de l'évaluation est d'évaluer les  performances des choix technique. C'est pourquoi nous n'avons pas défini des critères pour choisir l'ensemble de données. 
%evaluation des technologies
Nous avons évalué  les performances des technologies en terme de temps écoulé tout au long de l'analyse de différents échantillons.
 Les performances de MongoDB dépendent de ... 
 Toutefois, les performances des trois services d'Amazon dépendent de 
 Tandis que Apache Spark dépend de 
 
 

%perspectives

Si nous aurions plus de temps, nous aurions aimé évalué les performances de l'outil de détection conçu par Fontugne\cite{DBLP:journals/corr/FontugneAPB16} en terme de précision dans la détection d'anomalies dans les délais des liens. 







