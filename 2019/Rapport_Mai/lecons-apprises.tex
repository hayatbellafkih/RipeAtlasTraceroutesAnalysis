\section{Récapitulatif}

% l'utilisation du travail des autres
Le premier défis dans mon travail était de pouvoir lancer le logiciel de détection tel qu'il est. 
% la réutilisation d'u travail acompagné d'une implémentation
% manipuler les données massives
%

La collecte de données est un des défis de l'analyse des données massives. Deux contraintes peuvent être rattachées à ce besoin. D'une part, la variété des sources de données et la récupération et stockage de ces données une fois les sources identifiées. Dans notre cas, les sources étaient identités et la récupération des données peut être automatisées.   

Un outil destiné à la manipulation des données massives révèle différentes contraintes.  Typiquement, ce sont des outils qui se basent sur une quantité importante de données en vue de fournir des premiers résultats. Dans notre cas, il fallait analyser au moins $24$, valeur par défaut, période, et dans chacune de ces périodes, le lien devait être caractérisé par au moins  $4$ RTTs différentiels. 

% Collecte de données



% Nettoyage de données

% stockage de données

% modèl de données

% choix technique

% reprendre un model vs le créer à zero

%travailler avec les clusters