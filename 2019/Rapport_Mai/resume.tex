\chapter*{Résumé}


RIPE Atlas est un projet créé et géré par l'organisme RIPE NCC,  ce projet a donné naissance à de
simples dispositifs, appelés sondes, qu'on  connecte à un routeur. Ces sondes consomment une quantité
légère d'électricité et de bande passante, mais elles sont capables de changer complètement des
implantations physiques dans certaines infrastructures.
La répartition abondante des sondes Atlas engendre quotidiennement une quantité importante
de données qui dépasse la capacité des outils traditionnels à stocker et à traiter  ces dernières avec
efficacité. On parle des données massives. 

Certaines problématiques dans le domaine des réseaux
informatiques nécessitent une exploration plus profonde des données réseaux afin d'aboutir à des
résultats importants, voire d'en tirer les connaissances. Dans ce travail, des technologies adaptées aux données massives ont été évaluées pour étudier un des problèmes liés aux performances des réseaux informatiques. C'est une évaluation d'un outil existant qui utilise    un nombre très important de traceroutes. 


%en utilisant des technologies adaptées aux données massives

 %il s'agit d'étudier le délai des liens topologiques en se basant sur 


%L'étude des délais des liens a été évaluée par un nombre de technologies ad


\textbf{Mots clés} : RIPE Atlas, Traceroute,  Big Data, données massives, MongoDB, Amazon Web Service, Apache Spark.