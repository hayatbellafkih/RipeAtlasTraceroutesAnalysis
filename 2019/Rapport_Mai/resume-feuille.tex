\documentclass[]{report}

\usepackage[utf8]{inputenc}
\usepackage[T1]{fontenc}
\usepackage[francais]{babel}

% Title Page
\title{}
\author{}


\begin{document}

%your name
\begin{center}
	Hayat  \textsc{Bellafkih}
\end{center}
%add vertical space	
\vspace{5mm}

%title
\begin{center}
\textbf{Analyse de données à grande échelle : application au dépôt RIPE Atlas}
\end{center}




%add vertical space	
\vspace{15mm}
RIPE Atlas est un projet créé et géré par l'organisme RIPE NCC.  Ce projet consiste  à déployer de
simples dispositifs, appelés sondes, que l'on  connecte à un routeur. Ces sondes consomment une quantité
faible d'électricité et de bande passante. Elle ont la capacité d'effectuer des mesures liées aux réseaux informatiques.
La répartition abondante des sondes Atlas engendre quotidiennement une quantité importante
de données qui dépasse la capacité des outils traditionnels de stockage et de traitements. On parle de données massives. 

Certaines problématiques dans le domaine des réseaux
informatiques nécessitent une exploration en profondeur des données réseaux afin d'aboutir à des
résultats significatifs, voire d'en tirer des connaissances. Dans ce mémoire, des technologies adaptées aux données massives ont été évaluées pour étudier un des problèmes liés aux performances des réseaux informatiques. Cette  évaluation adepte un outil existant qui utilise    un nombre très important de traceroutes. 


La manipulation des données massives concerne la collecte de ces données, leur stockage et leur traitements. De nombreuses  technologies dédiées aux données à grande échelle  sont disponibles pour répondre à ces besoins. Le choix d'une technologie s'appuie sur différents facteurs. Comme les frais appliqués suite à l'utilisation d'une technologie, sa performance en terme de temps d'analyse, l'évolutivité de la solution adoptée, etc. Ce sont les éléments évalués pour les technologies manipulées en vue d'analyser des traceroutes d'Atlas. Ces technologies sont MongoDB, Amazon S3 avec Amazon Athena et Apache Spark.

%en utilisant des technologies adaptées aux données massives

%il s'agit d'étudier le délai des liens topologiques en se basant sur 


%L'étude des délais des liens a été évaluée par un nombre de technologies ad


\textbf{Mots clés} : RIPE Atlas, Traceroute,  Big Data, données massives, MongoDB, Amazon Web Service, Apache Spark.



\end{document}          
