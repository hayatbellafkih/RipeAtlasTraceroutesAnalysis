\documentclass[]{report}

\usepackage[utf8]{inputenc}
\usepackage[T1]{fontenc}
\usepackage[francais]{babel}
\usepackage{amssymb,amsmath,amsthm}
\usepackage[table]{xcolor}
\usepackage{graphicx,wrapfig,lipsum}
%######################################################################################################
\usepackage[utf8]{inputenc}
\usepackage[T1]{fontenc}
\usepackage{listings}
\usepackage{makecell,tabularx}
\renewcommand\theadfont{\bfseries\scshape}
\usepackage{graphicx}
\setcounter{secnumdepth}{3}
\usepackage{float}
\usepackage{adjustbox}
\usepackage{amsmath}
\setlength{\parindent}{2em}
\usepackage{tcolorbox}
\usepackage{colortbl}
%\usepackage[frensh]{babel}
\usepackage[francais]{babel}
\frenchbsetup{StandardLists=true}
\usepackage{times}
\usepackage[hyphens]{url}
\usepackage{standalone}
\usepackage{comment}
\usepackage[subfigure]{tocloft} 
\usepackage{subfigure} 
%\usepackage{tabularx}
\usepackage{textcomp}
\usepackage{lscape}

\usepackage{color,soul}
\usepackage{csvsimple,booktabs}
\usepackage{tikz}
\usepackage{afterpage}
\usepackage{rotating}
\usepackage{csvsimple}
\usepackage{caption}

%\frenchbsetup{StandardLists=true}
\usepackage[hidelinks]{hyperref} 
\usepackage{array}
\colorlet{punct}{red!60!black}
\definecolor{background}{HTML}{EEEEEE}
\definecolor{delim}{RGB}{20,105,176}
\definecolor{mygray}{gray}{0.6}
\colorlet{numb}{magenta!60!black}
\newcolumntype{C}[1]{>{\centering\let\newline\\\arraybackslash\hspace{0pt}}m{#1}}
\usepackage[left=2.5cm,right=2.5cm,top=2.5cm,bottom=2.5cm]{geometry}
\usepackage[utf8]{inputenc}
\usepackage{amsmath}
\usepackage{amssymb}
\renewcommand{\bibname}{References}
\usepackage{enumerate}
\usepackage{tablefootnote}
\newcommand{\tabitem}{~~\llap{\textbullet}~~}
\usepackage{pstricks}
\usepackage{eqparbox}
\usepackage{hyperref}
\usepackage{tablefootnote}
\usepackage{float}
\usepackage{listings}
\usepackage{color}
\usepackage{graphicx}
\usepackage{verbatim}
\usepackage{pifont}
\usepackage{comment}
\usepackage{enumitem}
\usepackage{blindtext}
\usepackage{appendix}
\usepackage{tikz}
\usepackage{algorithm}
\usepackage{algpseudocode}
\usepackage{amsthm}

\newtheorem{theorem}{Théorème}


%\usepackage{titlesec}
%\titleformat{\chapter}[hang] 
%{\normalfont\huge\bfseries}{\chaptertitlename\ \thechapter:}{1em}{} 


\setlist{itemsep=0pt}
\usepackage[flushleft]{threeparttable}
\newcommand*\circled[1]{\tikz[baseline=(char.base)]{
		\node[shape=circle,draw,inner sep=2pt] (char) {#1};}}
\lstset{
	showstringspaces=false,
	string=[s]{"}{"},
	stringstyle=\color{blue},
	comment=[l]{:},
	commentstyle=\color{black},
	backgroundcolor=\color{lstcolor},
	breaklines=true,
}

\lstdefinelanguage{json}{
	basicstyle=\normalfont\ttfamily,
	numbers=left,
	numberstyle=\scriptsize,
	stepnumber=1,
	numbersep=8pt,
	showstringspaces=false,
	breaklines=true,
	frame=lines,
	backgroundcolor=\color{background},
	literate=
	*{0}{{{\color{numb}0}}}{1}
	{1}{{{\color{numb}1}}}{1}
	{2}{{{\color{numb}2}}}{1}
	{3}{{{\color{numb}3}}}{1}
	{4}{{{\color{numb}4}}}{1}
	{5}{{{\color{numb}5}}}{1}
	{6}{{{\color{numb}6}}}{1}
	{7}{{{\color{numb}7}}}{1}
	{8}{{{\color{numb}8}}}{1}
	{9}{{{\color{numb}9}}}{1}
	{:}{{{\color{punct}{:}}}}{1}
	{,}{{{\color{punct}{,}}}}{1}
	{\{}{{{\color{delim}{\{}}}}{1}
	{\}}{{{\color{delim}{\}}}}}{1}
	{[}{{{\color{delim}{[}}}}{1}
	{]}{{{\color{delim}{]}}}}{1},
}



\definecolor{lstcolor}{HTML}{F5F5F5}
\definecolor{orange}{HTML}{F8F9F9}
\definecolor{ligntrose}{HTML}{FFF8DC}
\tcbset{colback= orange,fonttitle=\bfseries, colframe=white }




\definecolor{dkgreen}{rgb}{0,0.6,0}
\definecolor{gray}{rgb}{0.5,0.5,0.5}
\definecolor{mauve}{rgb}{0.58,0,0.82}

\lstdefinestyle{myScalastyle}{
	frame=tb,
	language=scala,
	aboveskip=3mm,
	belowskip=3mm,
	showstringspaces=false,
	columns=flexible,
	basicstyle={\small\ttfamily},
	numbers=left,
	numberstyle=\tiny\color{gray},
	keywordstyle=\color{blue},
	commentstyle=\color{dkgreen},
	stringstyle=\color{mauve},
	frame=single,
	breaklines=true,
	breakatwhitespace=true,
	tabsize=3,
}
%%%%%%%%%%%%%%%%%%%
\renewcommand\labelitemi{--}
\newcommand{\source}[1]{\caption*{Source: {#1}} }
\definecolor{lightgray}{gray}{0.9}
\renewcommand{\cleardoublepage}{}



%%%%%%%%% like and dislike icons
\usepackage{graphicx}
\newcommand*{\RightThumbsUpAux}[1]{%
	\begingroup
	\sbox0{Ag}%
	\raisebox{-\dp0}{%
		\includegraphics[{%
			height=\dimexpr\dp0+\ht0\relax,
			#1%
		}]{Symbol_thumbs_up-crop.pdf}%
	}%
	\endgroup
}
\newcommand*{\RightThumbsDownAux}[1]{%
	\begingroup
	\sbox0{Ag}%
	\raisebox{-\dp0}{%
		\includegraphics[{%
			height=\dimexpr\dp0+\ht0\relax,
			#1%
		}]{Symbol_thumbs_down-crop.pdf}%
	}%
	\endgroup
}
\newcommand*{\RightThumbsUp}{%
	\RightThumbsUpAux{}%
}
\newcommand*{\RightThumbsDown}{%
	\RightThumbsUpAux{origin=c,angle=180}%
}
\newcommand*{\LeftThumbsUp}{%
	\scalebox{-1}[1]{\RightThumbsUp}%
}
\newcommand*{\LeftThumbsDown}{%
	\scalebox{-1}[1]{\RightThumbsDown}%
}
\begin{document}


\chapter{MongoDB}

\paragraph{Code source } ~

Voir \url{https://github.com/InternetHealthReport/tartiflette}.

\paragraph{Création de l'environement virtualdev}~

Voici la liste des dépendances :
\begin{lstlisting}
(tartiflette-env) 162558@inet-bigd:~$ pip freeze
appdirs==1.4.3
awscli==1.15.50
backports.functools-lru-cache==1.5
backports.os==0.1.1
boto3==1.7.44
botocore==1.10.49
certifi==2018.4.16
chardet==3.0.4
colorama==0.3.9
configparser==3.5.0
cycler==0.10.0
decorator==4.3.0
docutils==0.14
enum34==1.1.6
fs==2.0.23
fs-s3fs==0.1.8
future==0.16.0
futures==3.2.0
get==1.0.3
idna==2.7
jmespath==0.9.3
kiwisolver==1.0.1
matplotlib==2.1.1
networkx==1.11
numpy==1.14.5
pandas==0.23.1
pathlib==1.0.1
patsy==0.5.0
post==1.0.2
psycopg2==2.7.5
psycopg2-binary==2.7.5
public==1.0.3
py-radix==0.10.0
pyasn1==0.4.3
pygeoip==0.3.2
pymongo==3.6.1
pyparsing==2.2.0
python-dateutil==2.7.3
pytz==2018.4
PyYAML==3.12
query-string==1.0.2
request==1.0.2
requests==2.19.1
ripe.atlas.cousteau==1.4.2
rsa==3.4.2
s3transfer==0.1.13
scipy==1.1.0
six==1.11.0
socketIO-client==0.7.2
statsmodels==0.9.0
subprocess32==3.5.2
typing==3.6.4
urllib3==1.23
websocket-client==0.48.0
\end{lstlisting}
\paragraph{Importation des traceroutes dans MOngoDB}
Voir README.md.
\paragraph{Configurer l'analyse}
Il faut mettre à jour le fichier de configuration de l'analyse  disponible sur : https://github.com/InternetHealthReport/tartiflette/blob/master/analysis/conf/getRttData.conf

Exemple : 

\begin{lstlisting}
{
	"comment": "60 min May and June 2015", 
	"af": "", 
	"minASNEntropy": 0.5, 
	"alpha": 0.01, 
	"end": {
		"$date": 1518393600000
	}, 
	"binMult": 3, 
	"timeWindow": 4600, 
	"minSeen": 3, 
	"start": {
		"$date": 1517961600000
	}, 
	"minASN": 3, 
	"nbProcesses": 24, 
	"experimentDate": {
		"$date": 1546093776000
	}, 
	"confInterval": 0.05,
	"prefixes": ""
}
\end{lstlisting}
\paragraph{Préparer les données avec la fonction getRTTData }
Voir  \textit{getRttData} dans  \url{https://github.com/InternetHealthReport/tartiflette/blob/master/analysis/plot.py}
\paragraph{Traiter  les liens : dessiner l'évolution de chaque lien}
Voir  \textit{rttEvolution} dans  \url{https://github.com/InternetHealthReport/tartiflette/blob/master/analysis/plot.py}
\paragraph{Fichier Python : rttEvolution.py}

Le fichier rttEvolution.py doit être créé au même endroit que le script \textit{plot.py}.
\begin{lstlisting}
#imports 
from plot import *
import time
import datetime

start = time.time()
## Prepare data
a,b,c=getRttData()

## Process each link
finalResult =[]
for k, v in a.iteritems():
	result= rttEvolution([a[k],c[k]],k,"")
	finalResult.append(result)

## Save link's details
now = datetime.datetime.now()
filePath= str(now.strftime("%Y-%m-%d_%H-%M"))
fp = open('%s.json'%filePath, 'w') 
for result in finalResult:
json.dump(result, fp, default=str)


end = time.time()
print("\n Total Time is "+ str(end - start))
\end{lstlisting}

\paragraph{Lancer une analyse}

Etapes : 
\begin{itemize}
	\item configurer l'analyse;
	\item importer les données vers MongoDB
	\item activer VirtualEnv;
	\item lancer l'analyse via python $ rttEvolution.py $.  
\end{itemize}

\chapter{Amazon S3 et Amazon Athena}



\paragraph{Créer les partitions dans Amazon S3}
Voir l'arborescence dans le rapport de mémoire.
\paragraph{Importer des traceroutes dans Amazon S3} Via l'interface graphique, l'API ou bien à travers AWS Command Line Interface (CLI).

\paragraph{Créer la table dans Amazon Athena}
Voir la structure dans le rapport de mémoire.
\paragraph{Configurer l'analyse}
Il faut mettre à jour le fichier de configuration appelé \textit{getRttDataAthena.conf}. 

Un exemple de fichier de configuration :
\begin{lstlisting}
{
	"comment": "60 min May and June 2015", 
	"af": "", 
	"minASNEntropy": 0.5, 
	"alpha": 0.01, 
	"end": {
		"$date": 1518307200000
	}, 
	"binMult": 3, 
	"timeWindow": 3600, 
	"minSeen": 3, 
	"start": {
		"$date": 1517961600000
	}, 
	"minASN": 3, 
	"nbProcesses": 24, 
	"experimentDate": {
		"$date": 1456747743895
	}, 
	"confInterval": 0.05,
	"prefixes": "",
	"msmIds" : [5004],
	"msmType" : "builtin"
}
\end{lstlisting}
\paragraph{Préparer les données avec la fonction getRTTData }

\paragraph{Traiter  les liens : dessiner l'évolution de chaque lien}

\paragraph{Fichier Python : rttEvolutionAthena.py}
Voir  \url{https://github.com/hayatbellafkih/RipeAtlasTraceroutesAnalysis/blob/master/2019/AWSAthenaTools/rttEvolutionAthena.py}

Ou :
\begin{lstlisting}
## imports
from athenaTools import *

import time
start = time.time()
import datetime

## Prepare Data
a,b,c=getRttDataAthena()

## Process each link
finalResult =[]
for k, v in a.iteritems():
	result=rttEvolution([a[k],c[k]],k,"")
	finalResult.append(result)

## Save link's details
now = datetime.datetime.now()
filePath= str(now.strftime("%Y-%m-%d_%H-%M"))
fp = open('%s.json'%filePath, 'w')
for result in finalResult:
json.dump(result, fp, default=str)

end = time.time()
print("Total Time is "+ str(end - start))
\end{lstlisting}

\paragraph{Requête SQL pour la récupération des traceroutes} \label{sqlformat}

Nous présentons la requête SQL générique utilisée (Listing \ref{lst:sqlrequest}). Elle est formatée pour chaque période d'analyse afin de :
 
 ligne \ref{lineperiode} : ajouter les limites de la période en cours : le timestamp minimum et celui maximum;
 
 ligne \ref{partitions} : choisir les partitions. Les identifiants de mesures sont ceux dans le fichier de configuration. Les partitions relatives à la date : \textit{year}, \textit{month} et \textit{day} sont déduits de la période en cours.
\begin{lstlisting}[language=scala,firstnumber=1, caption={ Requête SQL dans Athena },label={lst:sqlrequest}, basicstyle = \footnotesize,escapechar=|,numbers=left,
stepnumber=1]
with dataset as( SELECT prb_id ,"from",msm_id,  if (result[1].result[1].err is  null and result[1].result is not null ,transform(result , x -> transform (x.result, entry->  Map(array[ if (entry."from" is not null, concat('"',cast(entry."from" as varchar),'"') , concat('"','none','"')) ],  array[if (entry."from" is not null ,entry."rtt" ,0)]  ))),array[])  as datay from traceroutes_api 
where   "timestamp" >= {} and "timestamp"< {} and  |\label{lineperiode}|
{} |\label{partitions}|
) select "from",prb_id,msm_id,datay as result from dataset;
\end{lstlisting}

\paragraph{Traiter les traceroutes obtenus depuis Amazon Athena}

Nous avons regrouper les fonctions adaptées  dans un seul fichier appelé \textit{athenaTools.py}. Il est disponible sur : 
\url{https://github.com/hayatbellafkih/RipeAtlasTraceroutesAnalysis/blob/master/2019/AWSAthenaTools/athenaTools.py} Nous décrivons brièvement les fonctions dans ce fichier.

\begin{description}
	\item[load\_sql\_request\_by\_file] : la requête SQL présentée précédemment est créée dans un fichier\footnote{\url{https://github.com/hayatbellafkih/RipeAtlasTraceroutesAnalysis/blob/master/2019/AWSAthenaTools/sqlRequestForAthena.sql}} et cette fonction permet de récupérer cette requête comme étant une chaîne de caractère;
	
	\item[generateSQLAthenaRequest] : cette fonction consiste à formater la requête SQL avec les  valeurs adéquates (Voir le paragraphe \ref{sqlformat});
	
	\item[computeRtt\_athena] : équivalente à  la fonction  \textit{computeRtt}  dans l'implémentation MongoDB;
	
	\item[readOneTracerouteAllSigna] : fonction équivalente à  la fonction \textit{readOneTraceroute} dans l'implémentation MongoDB;
	
	\item[mergeRttResults\_athena] : fonction équivalente à la fonction \textit{mergeRttResults}  dans l'implémentation MongoDB;
	
	\item[getRttDataAthena] :  fonction équivalente à la fonction \textit{getRttData}  dans l'implémentation MongoDB;
	
	\item[rttEvolution] : est la même fonction que celle dans l'implémentation MongoDB;
	
	\item[fetchall\_athena] : c'est la fonction qui se charge de faire l'appel à Amazon Athena 
\end{description}


\chapter{Apache Spark}
\paragraph{}
La fonction \textit{main} se trouve dans l'objet .

\paragraph{L'application Spark}
Il est possible de générer le fichier .jar avec Maven.

\paragraph{Soumission de l'application}
param1 : date début (format timestamp).
param2 : date de fin (format timestamp)
param3 : fenêtre   (en secondes)
param4 : endroit des fichiers de données
param5 : chemin destination des résultats de l'analyse

\paragraph{Exemple}



\end{document}          
